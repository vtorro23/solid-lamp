\chapter{Cálculo de integrales}
\begin{eg}
	Consideremos $\displaystyle f\left(x,y\right) = x^{2}y + \cos xy $ y $\displaystyle R = \left[0,1\right] \times \left[2,3\right]  $. Como $\displaystyle f $ es continua en $\displaystyle \R^{2} $, lo es en $\displaystyle R $ por lo que es integrable. Buscamos cómo calcular
	\[\int _{\left[0,1\right] \times\left[2,3\right] }f\left(x,y\right) \; dx \; dy .\]
Veremos que la forma de calcular integrales es a través de integrales iteradas, es decir,
\[
\begin{split}
	\int _{\left[0,1\right] \times\left[2,3\right] }f\left(x,y\right) \; dx \; dy = & \int^{3}_{2} \int^{1}_{0} x^{2}y + \cos xy \; dx \; dy = \int^{3}_{2} \frac{y}{3} + \frac{\sin y}{y} \; dy \\
	= & \left[\frac{y^{2}}{6}\right]^{3}_{2} + \int^{3}_{2} \frac{\sin y}{y} \; dy= \frac{5}{6} + \int^{3}_{2} \frac{\sin y}{y} \; dy .
\end{split}
\]
Podríamos haber planteado también la integral
\[\int^{1}_{0} \int^{3}_{2} x^{2}y + \cos xy \; dy \; dx ,\]
es decir, cambiar el orden de integración. En este caso, los dos resultados coinciden.
\end{eg}
\section{Teorema de Fubini}
\begin{theorem}[Teorema de Fubini]
	Sea $\displaystyle R = \left[a, b \right] \times \left[c,d\right] \subset \R^{2} $ y $\displaystyle f : R \to \R $ integrable. Supongamos que para cada $\displaystyle x \in \left[a,b\right]  $, la función 
	\[f_{x} : \left[c,d\right] \to \R : y \to f\left(x,y\right) ,\]
	es integrable en $\displaystyle \left[c,d\right]  $. Entonces, la función 
	\[x \to \int^{d}_{c} f_{x}\left(y\right) \; dy ,\]
	es integrable en $\displaystyle \left[a,b\right]  $ y 
	\[\int _{\left[a,b\right] \times\left[c,d\right] }f = \int^{b}_{a} \int^{d}_{c} f\left(x,y\right) \; dy \; dx .\]
\end{theorem}
\begin{observation} Podemos hacer un par de observaciones.
	\begin{itemize}
	\item Es importante ver la condición de que $\displaystyle f_{x} $ sea integrable es necesaria puesto que el hecho de que $\displaystyle f $ sea integrable no implica siempre que $\displaystyle f_{x} $ lo sea.
	\item El teorema también se cumple si cambiamos $\displaystyle x $ por $\displaystyle y $.
	\end{itemize}
\end{observation}
\begin{theorem}[Teorema de Fubini en $\displaystyle \R^n $]
Sea $\displaystyle R = R_{1} \times R_{2} \subset \R^{n}$ un rectángulo con $\displaystyle R_{1} \subset \R^{k} $ rectángulo y $\displaystyle R_{2}\subset \R^{n-k} $ rectángulo. Suponemos que $\displaystyle \forall x \in R_{1} $, $\displaystyle f_{x} : R_{2} \to \R : y \to f\left(x,y\right) $ es integrable en $\displaystyle R_{2} $. Entonces la función 
\[x \to \int _{R_{2}}f\left(x,y\right)  ,\]
es integrable en $\displaystyle R_{1} $ y se tiene que 
\[\int _{R}f=\int_{R_{1}} \int_{R_{2}} f\left(x,y\right) \; dy \; dx .\]
\end{theorem}
\begin{eg}
	Consideremos los siguientes ejemplos.
	\begin{enumerate}
	\item Sea $\displaystyle A = \left\{ \left(x,y\right) \in \R^{2} \; : \; x + y\leq1,\; x \geq 0, \; y \geq 0\right\}  $, queremos calcular
	\[\int _{A}f\left(x,y\right) \; dx \; dy .\]
	Si cogemos el rectángulo $\displaystyle R = \left[0,1\right] \times\left[0,1\right]  $, tenemos que
	\[\int _{A}f\left(x,y\right) \; dx \; dy = \int _{R}\tilde{f}\left(x,y\right) \; dx \; dy .\]
	Si cogemos un $\displaystyle x_{0} \in \left[a,b\right]  $, podemos considerar  
	\[ \tilde{f}_{x_{0}} : \left[c,d\right] \to \R: y \to 
	\begin{cases}
	f\left(x_{0}, y\right), \; \left(x_{0}, y \right)\in A \\
	0, \; \left(x_{0}, y \right)\not\in A
	\end{cases}
	.\]
Supongamos que es integrable, entonces tendremos que 
\[\int _{R}\tilde{f} = \int^{1}_{0} \int^{1}_{0} \tilde{f}\left(x,y\right) \; dy \; dx .\]
Dado que nos es incómodo calcular la integral de $\displaystyle \tilde{f} $, podemos decir que 
\[
\begin{split}
	\int^{1}_{0} \left(\int^{y}_{0} \tilde{f}\left(x,y\right) \; dy + \int^{1}_{y} \tilde{f}\left(x,y\right) \; dy\right) \; dx = & \int^{1}_{0} \int^{y}_{0} f\left(x,y\right) \; dy \; dx \\
	= &  \int^{1}_{0} \int^{1-x}_{0} f\left(x,y\right) \; dy \; dx.
\end{split}
\]
Si tomamos por ejemplo $\displaystyle f\left(x,y\right)= x $, tendremos que 
\[\int _{A}f\left(x,y\right) \; dy \; dx = \int^{1}_{0} \int^{1-x}_{0} x \; dy \; dx = \int^{1}_{0} x\left(1-x\right) \; dx = \frac{1}{6} .\]
Podemos cambiar el orden de integración:
\[\int _{A}f\left(x,y\right) \; dx \; dy = \int^{1}_{0} \int^{1-y}_{0} x \; dx \; dy=\int^{1}_{0} \frac{\left(1-y\right)^{2}}{2} \; dy = \frac{1}{6} .\]
\item Consideremos ahora $\displaystyle A = \left\{ \left(x,y\right) \in \R^{2} \; : \; x^{2} + y^{2} \leq 1, \; y \geq 0\right\}  $. Podemos considerar $\displaystyle R = \left[-1,1\right] \times \left[0,1\right]  $. De esta forma, 
	\[\int _{A}f\left(x,y\right) \; dy \; dx = \int^{1}_{-1} \int^{1}_{0} \tilde{f}\left(x,y\right) \; dy \; dx = \int^{1}_{-1} \int^{\sqrt{1-x^{2}}}_{0} f\left(x,y\right) \; dy \; dx .\]
	Podemos cambiar el orden de integración:
	\[\int _{A}f\left(x,y\right) \; dx \; dy = \int^{1}_{0} \int^{1}_{-1} \tilde{f}\left(x,y\right) \; dx \; dy = \int^{1}_{0} \int^{\sqrt{1-y^{2}}}_{-\sqrt{1-y^{2}}}  f\left(x,y\right)\; dx \; dy.\]
\item Sea $\displaystyle A = \left\{ \left(x,y\right) \in \R^{2} \; : \; x \in \left[a,b\right], \; g_{2}\left(x\right) \leq y \leq g_{1}\left(x\right)\right\}  $ para $\displaystyle g_{1}, g_{2} : \R \to \R $ continuas. Tendremos que 
	\[\int _{A}f\left(x,y\right) \; dy \; dx = \int^{b}_{a} \int^{g_{1}\left(x\right)}_{g_{2}\left(x\right)} f\left(x,y\right) \; dy \; dx  .\]
	\end{enumerate}
\end{eg}
\section{Cambio de variable}
\begin{theorem}[Teorema de cambio de variable]
Si $\displaystyle \varphi: B\subset \R^{n} \to A \subset \R^{n} $ es de clase $\displaystyle \mathcal{C}^{1} $ y $\displaystyle \varphi^{-1}\in\mathcal{C}^{1} $ \footnote{Buscamos que $\displaystyle \varphi $ sea difeomorfismo.}, tenemos que 
\[\int _{A}f\left(x\right) \; dx = \int _{B}f\left(\varphi\left(t\right)\right) \left|\det \varphi'\left(t\right)\right| \; dt .\]
Asumimos que $\displaystyle A $ y $\displaystyle B $ son abiertos.
\end{theorem}
\subsection{Coordenadas polares}
El cambio de coordenadas polares consiste en
\[x = \rho \cos\theta \quad \text{y} \quad y = \rho\sin\theta .\]
Para deshacer el cambio es fácil ver que 
\[\rho = \sqrt{x^{2} + y^{2}} \quad \text{y} \quad \theta = \arctan\left(\frac{y}{x}\right) .\]
Es sencillo comprobar que 
\[J\left(\rho, \theta \right) = \begin{pmatrix} \cos\theta & -\rho\sin\theta \\ \sin\theta & \rho\cos\theta \end{pmatrix} \Rightarrow \left|\det J\left(\rho, \theta\right)\right|=\rho	 .\]
En este caso podemos considerar $\displaystyle \varphi : [0, \infty) \times [0, 2\pi) \to \R^{2} $.
\begin{eg}
	Consideremos $\displaystyle A = \left\{ \left(x,y\right) \in \R^{2} \; : \; x^{2} +y^{2}\leq 1\right\}  $. Tenemos que 
	\[v\left(A\right)=\int _{A} \; dx\; dy = \int^{1}_{-1} \int^{\sqrt{1-x^{2}}}_{-\sqrt{1-x^{2}}}  \; dy \; dx = \int^{1}_{-1} 2\sqrt{1-x^{2}} \; dx=\pi .\]
Intentemos resolver el problema pero esta vez con un cambio de variable:
\[v\left(A\right) = \int^{2\pi }_{0} \int^{1}_{0} \rho \; d\rho \; d\theta=\int^{2\pi }_{0} \frac{1}{2} \; d\theta = \pi .\]
Podemos ver que hemos obtenido el mismo valor por los dos métodos.
\end{eg}

