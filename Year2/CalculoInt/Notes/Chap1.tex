\chapter{Integral de Riemann}
\section{Conceptos básicos}
Consideramos el paralelepípedo
\[R = [a_{1}, b_{1}] \times \cdots \times [a_{n}, b_{n}] ,\]
que es producto directo de intervalos compactos en $\displaystyle \R $. Consideremos también una función $\displaystyle f : R \to \R $ acotada. Definimos el \textbf{volumen} del rectángulo $\displaystyle R $ como el producto de las longitudes de sus lados, es decir,
\[v\left(R\right) := \left(b_{1}-a_{1}\right) \cdots \left(b_{n}-a_{n}\right) .\]
\begin{definition}[Partición]
Tomamos particiones
\[P_{1} \in \mathcal{P}\left([a_{1}, b_{1}]\right), \ldots, P_{n} \in \mathcal{P}\left([a_{n}, b_{n}]\right) ,\]
y decimos que $\displaystyle P = \left(P_{1}, \ldots, P_{n}\right) \in \mathcal{P}\left(R\right) $ es una \textbf{partición} de $\displaystyle R $.
\end{definition}
De esta forma, estamos dividiendo el rectángulo $\displaystyle R $ en subrectángulos.
\begin{definition}
Dadas dos particiones $\displaystyle P = \left(P_{1}, \ldots, P_{n}\right) $ y $\displaystyle Q = \left(Q_{1}, \ldots, Q_{n}\right) $, decimos que $\displaystyle P $ es \textbf{más fina} que $\displaystyle Q $, $\displaystyle P \geq Q $, si $\displaystyle P_{i} $ es más fina que $\displaystyle Q_{i} $ para $\displaystyle i = 1, \ldots, n $.
\end{definition}
 Si $\displaystyle T \in \mathcal{P}\left(R\right) $, entonces $\displaystyle T $ es un pequeño paralelepípedo cuyos lados son intervalos de $\displaystyle P_{i} $ (queremos decir que $\displaystyle T $ es uno de los subrectángulos formados por la partición $\displaystyle P $). Podemos ver que para cada partición $\displaystyle P_{i} = \left\{ x^{i}_{0}, \ldots, x^{i}_{m_{i}}\right\}  $, los rectángulos $\displaystyle T $ tendrán la forma  
\[ T = \left[x^{1}_{j_{1}}, x^{1}_{j_{1}+1}\right] \times \cdots \times \left[x^{n}_{j_{n}}, x^{n}_{j_{n}+1}\right], \; 0 \leq j_{i} \leq m_{i}-1.\]
Así, definimos,
\begin{definition}[Suma superior e inferior]
Decimos que la \textbf{suma inferior} de $\displaystyle f $ por $\displaystyle P $ es
\[s\left(f, P\right) := \sum_{T \in P}v\left(T\right)\inf \left\{ f\left(t\right) \; : \; t \in T\right\} .\]
Análogamente, decimos que la \textbf{suma superior} de $\displaystyle f $ por $\displaystyle P $ es
\[S\left(f,P\right):=\sum_{T \in P}v\left(T\right)\sup \left\{ f\left(t\right) \; : \; t \in T\right\}  .\]
\end{definition}
\begin{observation}
A partir de la definición anterior, podemos hacer un par de observaciones. 
\begin{itemize}
\item En primer lugar, como $\displaystyle f $ está acotada, las sumas superiores e inferiores están bien definidas.
\item Para cualquier partición $\displaystyle P \in \mathcal{P}\left(R\right) $ se cumple que $\displaystyle s\left(f,P\right)\leq S\left(f,P\right) $.
\end{itemize}
\end{observation}
Para introducir la noción de \textbf{integral superior} e \textbf{integral inferior}, tenemos que ver que las sumas superiores e inferiores están acotadas, esto se puede ver de dos formas.
\begin{notation}
A partir de ahora, utilizamos la notación siguiente:
\[\alpha_{T} = \inf \left\{ f\left(t\right) \; : \; t \in T\right\} \quad \text{y} \quad \beta_{T} = \sup \left\{ f\left(t\right) \; : \; t \in T\right\}  .\]
\end{notation}

\subsection*{Forma 1}
Sea $\displaystyle P \in \mathcal{P}\left(R\right) $. Como $\displaystyle f $ es acotada en $\displaystyle R $ sabemos que existen $\displaystyle M,m \in \R $ tales que 
\[m \leq f\left(x\right)\leq M, \; \forall x \in R .\]
Así, tenemos que 
\[\sum_{T\in P}mv\left(T\right) \leq s\left(f,P\right) \leq S\left(f,P\right)\leq \sum_{T \in P}Mv\left(T\right) .\]
Demostremos ahora la igualdad
\[\sum_{T \in P} v\left(T\right) = v\left(R\right).\]
En primer lugar, consideremos el caso $\displaystyle n = 1 $. Cogemos $\displaystyle R = [a,b] $ y la partición $\displaystyle P = \left\{ t_{0} = a, t_{1}, \ldots, t_{m} = b\right\}  $. Así, tenemos que 
\[\sum^{m}_{i = 1}v\left(\left[t_{i-1}, t_{i}\right] \right) = \sum^{m}_{i = 1}\left(t_{i}-t_{i-1}\right) = t_{m}-t_{0} = b-a = v\left(R\right).\]
Demostraremos el caso $\displaystyle n = 2 $ pues a partir de este es fácil generalizar la demostración para $\displaystyle n > 2 $. Por tanto, tomamos $\displaystyle R = [a_{1}, b_{1}]\times[a_{2}, b_{2}] $ con la partición $\displaystyle P = \left(P_{1}, P_{2}\right) $ tal que 
\[P_{1} = \left\{ t_{0} = a_{1} , t_{1}, \ldots, t_{m} = b_{1}\right\} , \quad P_{2} = \left\{ q_{0} = a_{2} , q_{1} , \cdots , q_{r} = b_{2}\right\}  .\]
Tendremos que 
\[
\begin{split}
	\sum_{T \in R}v\left(T\right) = & \sum^{m - 1}_{i = 0}\sum^{r - 1}_{j = 0}v\left(\left[t_{i}, t_{i+1}\right] \times \left[q_{j}, q_{j+1}\right] \right) = \sum^{m - 1}_{i = 0}\sum^{r - 1}_{j = 0} \left(t_{i+1}-t_{i}\right)\left(q_{j+1}-q_{j}\right) = \left(b_{1}-a_{1}\right) \left(b_{2}-a_{2}\right).
\end{split}
\]
Así, podemos decir que
\[mv\left(R\right)\leq s\left(f,P\right) \leq S\left(f,P\right) \leq Mv\left(R\right) .\]
\subsection*{Forma 2}

\begin{lema}
Sean $\displaystyle P,T \in \mathcal{P}\left(R\right) $ con $\displaystyle T \geq P $. Entonces,
\[s\left(f,P\right)\leq s\left(f,T\right)\leq S\left(f,T\right)\leq S\left(f,P\right) .\]
\end{lema}
\begin{proof}
	Lo demostramos para $\displaystyle n =2$ puesto que la demostración es fácil de generalizar para $\displaystyle n > 2 $. Sea $\displaystyle P = \left(P_{1},P_{2}\right) \in \mathcal{P}\left(R\right) $. Para demostrar el lema basta con demostrar el caso $\displaystyle P' = \left(P_{1}',P_{2}\right) $, donde $\displaystyle P_{1}' = P_{1} \cup \left\{ u\right\}  $. Claramente tenemos que $\displaystyle P' $ es más fina que $\displaystyle P $. Concretamente, supongamos que 
\[P_{1} = \left\{ t_{0}^{1}, \ldots, t_{n}^{1}\right\} \quad \text{y} \quad P_{1}' = \left\{ t_{0}^{1}, \ldots, t_{i}^{1}, u, t_{i+1}^{1}, \ldots, t_{n}^{1}\right\}  .\]
Sea $\displaystyle P_{2} = \left\{ q_{0}, \ldots, q_{r}\right\}  $, y definimos los conjuntos de rectángulos
\[J_{1} = \left\{ [t_{i}, u] \times [q_{j}, q_{j+1}] \; : \; 0 \leq j \leq r-1 \right\} \quad \text{y} \quad  J_{2} = \left\{ \left[u, t_{i+1}\right] \times \left[q_{j}, q_{j+1}\right] \; : \; 0 \leq j \leq r-1\right\} .\]
Claramente tenemos que $\displaystyle J_{1} \cap J_{2} = \emptyset $. De esta forma, tenemos que
\[
\begin{split}
	s\left(f,P'\right) = &  \sum_{T \in P'}v\left(T\right) \alpha_{T} 
	=  \sum_{T \in P'/\left(J_{1} \cup J_{2}\right)}v\left(T\right) \alpha_{T} + \sum_{T \in J_{1}}v\left(T\right) \alpha_{T} + \sum_{T \in J_{2}}v\left(T\right) \alpha_{T} \\
	= & \sum_{T \in P'/\left(J_{1} \cup J_{2}\right)}v\left(T\right) \alpha_{T} + \sum^{r-1}_{j = 0}\left(u-t_{i}\right)\left(q_{j+1}-q_{j}\right)\alpha_{j}^{1}+\sum^{r-1}_{j = 0}\left(t_{i+1}-u\right)\left(q_{j+1}-q_{j}\right)\alpha_{j}^{2}\\
	\geq & \sum_{T \in P'/\left(J_{1} \cup J_{2}\right)}v\left(T\right) \alpha_{T} + \sum^{r-1}_{j = 0} \left(t_{i+1}-t_{i}\right)\left(q_{j+1}-q_{j}\right)\alpha_{j} = s\left(f,P\right).
\end{split}
\]
La desigualdad para la suma superior se demuestra de forma análoga. 
\end{proof}
\begin{proof}
	Esta es una demostración alternativa. Consideremos que $\displaystyle T $ es más fina que $\displaystyle P $ y sean $\displaystyle R_{1}, \ldots, R_{N} $ los subrectángulos de $\displaystyle P $ y $\displaystyle \tilde{R}_{1}, \ldots, \tilde{R} _{\tilde{N}}$ los de $\displaystyle T $.
	Sea $\displaystyle I_{k} $ el conjunto de índices $\displaystyle j $ tales que $\displaystyle \tilde{R}_{j} \subset R_{k} $. Así, es fácil ver que
	\[R_{k} = \bigsqcup_{j \in I_{k}}\tilde{R}_{j}, \quad v\left(R_{k}\right) = \sum_{j \in I_{k}}v\left(\tilde{R}_{j}\right) .\]
	Denotamos $\displaystyle \alpha_{j} = \inf \left\{ f\left(x\right) \; : \; x \in R_{j}\right\}  $ y $\displaystyle \tilde{\alpha }_{j} = \inf \left\{ f\left(x\right) \; :\; x \in \tilde{R}_{j}\right\}  $. 
	Claramente, si $\displaystyle j \in I_{k} $ se tiene que $\displaystyle \alpha _{k} \leq \tilde{\alpha}_{j} $, así
	\[s\left(f,P\right) = \sum^{N}_{k = 1}\alpha_{k}v\left(R_{k}\right) = \sum^{N}_{k = 1}\sum_{j \in I_{k}}\alpha_{k}v\left(\tilde{R}_{j}\right) \leq \sum^{N}_{k = 1}\sum_{j \in I_{k}}\tilde{\alpha}_{j}v\left(\tilde{R}_{j}\right) = \sum^{\tilde{N}}_{j = 1}\tilde{\alpha}_{j}v\left(\tilde{R}_{j}\right) = s\left(f,T\right) .\]
El caso para la suma superior es análogo.	
\end{proof}

\begin{prop}
Dadas dos particiones, $\displaystyle P, Q \in \mathcal{P}\left(R\right) $
\[s\left(f,P\right) \leq S\left(f,Q\right) .\]
\end{prop}
\begin{proof}
Sea $\displaystyle T = \left(P_{1} \cup Q_{1}, \ldots, P_{n} \cup Q_{n}\right)\in \mathcal{P}\left(R\right) $. Claramente, $\displaystyle T \geq P $ y $\displaystyle T \geq Q $. Por tanto, aplicando el lema anterior
\[ s\left(f,P\right) \leq s\left(f, T\right)\leq S\left(f,T\right)\leq S\left(f,Q\right).\]
\end{proof}
Por ambas formas hemos visto que existen
\[\inf \left\{ S\left(f,P\right) \; : \; P \in \mathcal{P}\left(R\right)\right\} \quad \text{y} \quad \sup \left\{ s\left(f,P\right) \; : \; P \in \mathcal{P}\left(R\right)\right\}  ,\]
por lo que estamos en condiciones de definir la integral superior e inferior. 
\section{Integral de Riemann}
\begin{definition}[Integral superior e inferior]
Se define como \textbf{integral superior} e \textbf{integral inferior} a los valores
\[\overline{\int _{R}}f = \inf \left\{ S\left(f,P\right) \; : \; P \in \mathcal{P}\left(R\right)\right\} \quad \text{y} \quad \underline{\int _{R}}f = \sup \left\{ s\left(f,P\right) \; : \; P \in \mathcal{P}\left(R\right)\right\} ,\]
repsectivamente. Decimos que $\displaystyle f $ es \textbf{integrable} si el valor de la integral superior e inferior coincide.
\end{definition}

\begin{colorary}
Dada $\displaystyle f : R \to \R $ acotada,
\[\underline{\int _{R}}f \leq \overline{\int _{R}}f .\]
\end{colorary}

\begin{eg}
Consideremos $\displaystyle f \equiv c $, con $\displaystyle c $ constante. Tenemos que para una partición $\displaystyle P \in \mathcal{P}\left(R\right)$,
\[s\left(f,P\right) = \sum_{T \in P}v\left(T\right) \inf \left\{ f\left(t\right) \; : \; t\in T\right\}  = \sum_{T \in P}v\left(T\right)c = cv\left(R\right).\]
Por otro lado, 
\[ S\left(f,P\right) = \sum_{T \in P}v\left(T\right)\sup \left\{ f\left(t\right) \; : \; t \in T\right\} = \sum_{T \in P}v\left(T\right)c = cv\left(R\right).\]
Como la integral superior y la inferior coinciden, debe ser que la función es integrable.
\end{eg}
\begin{theorem}[Criterio de integrabilidad de Riemann]
Una función $\displaystyle f: R \to \R $ acotada es integrable si y solo si $\displaystyle \forall \epsilon > 0 $, $\displaystyle \exists P_{\epsilon } \in \mathcal{P}\left(R\right) $ tal que 
\[S\left(f,P_{\epsilon }\right)-s\left(f,P_{\epsilon }\right) < \epsilon  .\]
\end{theorem}
\begin{proof}
\begin{description}
\item[(i)] Supongamos que $\displaystyle f $ es integrable y sea $\displaystyle \epsilon > 0 $. Por definición de supremo e ínfimo, tenemos que existen particiones $\displaystyle P_{1}, P_{2} \in \mathcal{P}\left(R\right) $ tales que
	\[\int _{R}f -\frac{\epsilon }{2} < s\left(f, P_{1}\right), \; \int _{R}f +\frac{\epsilon }{2} > S\left(f,P_{2}\right) .\]
Cogemos $\displaystyle P_{\epsilon} \in \mathcal{P}\left(R\right) $ más fina que $\displaystyle P_{1} $ y $\displaystyle P_{2} $ y tenemos que
\[S\left(f,P_{\epsilon}\right)-s\left(f,P_{\epsilon}\right) \leq S\left(f,P_{2}\right)-s\left(f,P_{1}\right)<\left(\int _{R}f +\frac{\epsilon }{2}\right)-\left(\int _{R}f -\frac{\epsilon }{2}\right) = \epsilon  .\]
\item[(ii)] Sea $\displaystyle \epsilon > 0 $, entonces por hipótesis tenemos que
	\[0 \leq \overline{\int _{R}}f-\underline{\int _{R}}f \leq S\left(f,P_{\epsilon }\right)-s\left(f,P_{\epsilon }\right) < \epsilon .\]
	Como esto es cierto para todo $\displaystyle \epsilon > 0 $, debe ser que la integral superior y la inferior coinciden, por lo que $\displaystyle f $ es integrable. 
\end{description}
\end{proof}
\begin{observation}
La negación del criterio anterior nos permite ver cuándo una función no es integrable, que es si y solo si existe $\displaystyle \epsilon _{0} > 0 $ tal que $\displaystyle S\left(f, P\right)-s\left(f,P\right) \geq \epsilon_{0} $, $\displaystyle \forall P \in \mathcal{P}\left(R\right) $. 
\end{observation}
\begin{eg}
Un ejemplo muy común de función no integrable es la función de Dirichlet,
\[f:\left[0,1\right]  \to \R : x \to 
\begin{cases}
1, \; x \in \Q \\
0, \; x \not\in \Q
\end{cases}
.\]
Este ejemplo lo podemos generalizar en $\displaystyle \R^{n} $. En efecto, podemos tomar $\displaystyle R = \left[0,1\right] ^{n} \subset \R^{n} $, con 
\[f:R \to \R : x = \left(x_{1}, \ldots, x_{n}\right) \to 
\begin{cases}
1, \; x \in \Q^{n} \\
0, \; x\not\in \Q^{n}
\end{cases}
.\]
Tenemos que $\displaystyle \forall P \in \mathcal{P}\left(R\right) $, $\displaystyle S\left(f,P\right) = 1 $ y $\displaystyle s\left(f,P\right) = 0 $, por lo que $\displaystyle S\left(f,P\right) -s\left(f,P\right) = 1 $ y $\displaystyle f $ no es integrable. \\
Esta noción la podemos generalizar. Consideremos $\displaystyle R = \left[0,1\right] ^{n} \subset \R^{n} $ y $\displaystyle A \subset \R $ tal que $\displaystyle A $ y $\displaystyle R/A $ son densos en $\displaystyle R $. Por denso queremos decir que todo rectángulo no trivial $\displaystyle J $, $\displaystyle A \cap J \neq \emptyset $ y $\displaystyle \left(R/A\right)\cap J \neq \emptyset $. Entonces, 
\[f : R \to \R : x \to 
\begin{cases}
1, \; x \in A \\
0, \; x \in R/A
\end{cases}
,\]
no es integrable. 
\end{eg}
\begin{theorem}[Teorema de Darboux]
Una función $\displaystyle f : R \to \R $ acotada es integrable en $\displaystyle R $ con integral $\displaystyle I $ si y solo si para todo $\displaystyle \epsilon > 0 $ existe $\displaystyle \delta > 0 $ tal que si $\displaystyle P \in \mathcal{P}\left(R\right) $ con $\displaystyle \|P\|<\delta  $ \footnote{Para cualquier $\displaystyle J \in P $, el diámetro de $\displaystyle J $ es menor a $\displaystyle \delta  $.}, entonces
\[ \left|\sum_{J \in P}f\left(x_{J}\right)v\left(J\right) - I\right|< \epsilon , \quad \forall x_{J} \in J .\]
\end{theorem}
\begin{proof}
\begin{description}
\item[(i)] Se puede mirar en el libro de Marsden y Hoffmann. 
\item[(ii)] Sea $\displaystyle \epsilon > 0 $, entonces existe $\displaystyle \delta > 0 $ tal que si $\displaystyle \|P\| < \delta  $ entonces
	\[ \left|\sum_{i = 1}^{N}f\left(x_{i}\right)v\left(J_{i}\right) - I\right| < \frac{\epsilon }{2} ,\]
Donde $\displaystyle J_{1}, \ldots, J_{N} $ son los rectángulos que componen la partición $\displaystyle P $. Cogemos $\displaystyle x_{i} \in J_{i} $ tal que 
	\[ \left|f\left(x_i\right)-\beta_{J_{i}}\right| < \frac{\epsilon }{v\left(J_{i}\right)2N} .\]
Así, tenemos que 
\[ \left|S\left(f,P\right)-I\right| \leq \left|S\left(f,P\right)-\sum^{N}_{i = 1}f\left(x_{i}\right)v\left(J_{i}\right)\right| + \left|\sum^{N}_{i = 1}f\left(x_{i}\right)v\left(J_{i}\right)-I\right| .\]
Tenemos que 
\[ \left|S\left(f,P\right)-\sum^{N}_{i = 1}f\left(x_{i}\right)v\left(J_{i}\right)\right| < \sum^{N}_{i = 1}\frac{\epsilon v\left(J_{i}\right)}{v\left(J_{i}\right)2N} = \frac{\epsilon }{2} .\]
Así, obtenemos que $\displaystyle \left|S\left(f,P\right)-I\right| < \epsilon  $. De forma análoga se puede demostrar que $\displaystyle \left|I - s\left(f,P\right)\right| < \epsilon  $. De esta forma, obtenemos que si $\displaystyle \epsilon > 0 $ existe $\displaystyle P \in \mathcal{P}\left(R\right) $ tal que 
\[ \left|S\left(f,P\right)-s\left(f,P\right)\right| \leq \left|S\left(f,P\right)-I\right|+ \left|I -s\left(f,P\right)\right| < \epsilon ,\]
y por el criterio de Riemann tenemos que $\displaystyle f $ es integrable en $\displaystyle R $. Además, por lo visto anteriormente, existe $\displaystyle P \in \mathcal{P}\left(R\right) $ tal que 
\[ \left|\overline{\int_{R}}f - I\right|\leq \left|\overline{\int _{R} }f - S\left(f,P\right)\right| + \left|S\left(f,P\right)-I\right| < \frac{\epsilon }{2} + \frac{\epsilon }{2} = \epsilon , \quad \forall \epsilon > 0  .\]
El caso para la integral inferior es análogo. Así, hemos demostrado que $\displaystyle \int _{R}f = I $. 
\end{description}
\end{proof}
\section{Propiedades de la integral}
\begin{theorem}
Sea $\displaystyle f : R \to \R $ continua, entonces $\displaystyle f $ es integrable. 
\end{theorem}
\begin{proof}
En esta demostración trabajaremos con la norma infinita pero la equivalencia de normas permite generalizar el resultado para cualquier norma en $\displaystyle \R^{n} $. Dado que $\displaystyle f $ es continua en $\displaystyle R $ y este es compacto, tenemos que $\displaystyle f $ es uniformemente continua en $\displaystyle R $. Sea $\displaystyle \epsilon > 0 $ y $\displaystyle \epsilon ' = \frac{\epsilon }{v\left(R\right)} > 0 $. Tenemos que existe $\displaystyle \delta > 0 $ tal que 
\[ \|x -y\|_{\infty} < \delta \Rightarrow \left|f\left(x\right)-f\left(y\right)\right| < \epsilon ' , \; \forall x,y \in R.\]
Así, cogemos una partición $\displaystyle P_{\delta} \in \mathcal{P}\left(R\right) $ que form rectángulos de lados con longitud menor a $\displaystyle \delta  $. Recordamos que dado que $\displaystyle f $ es continua en $\displaystyle R $, lo es también en cada subrectángulo generado por la partición $\displaystyle P_{\delta } $. De esta forma, en cada $\displaystyle T \in P_{\delta } $, $\displaystyle f $ alcanza su máximo y su mínimo, $\displaystyle \beta _{T} $ y $\displaystyle \alpha_{T} $, respectivamente. Por tanto,
\[S\left(f,P_{\delta }\right)-s\left(f,P_{\delta }\right) = \sum_{T \in P_{\delta }}v\left(T\right)\left(\beta_{T}-\alpha_{T}\right) < \epsilon ' \sum_{T \in P_{\delta }}v\left(T\right) = \epsilon .\]
Por el criterio de integrabilidad de Riemann, $\displaystyle f $ es integrable en $\displaystyle R $. 
\end{proof}
\begin{prop}[Linealidad y monotonía] Sea $\displaystyle R \subset \R^{n} $ un rectángulo.
	\begin{description}
	\item[Linealidad.] Si $\displaystyle f_{1}, f_{2} : R \to \R $ son integrables en $\displaystyle R $, entonces
\[\int _{R}f_{1} + f_{2} = \int _{R}f_{1} +\int _{R}f_{2}  .\]
Además, si $\displaystyle \alpha \in \R $ y $\displaystyle f : R \to \R $ es integrable en $\displaystyle R $, entonces
\[\int _{R}\alpha f = \alpha\int_{R} f .\]
\item[Monotonía.] Si $\displaystyle f_{1}, f_{2} : R \to \R$ son integrables en $\displaystyle R $, con $\displaystyle f_{1} \leq f_{2} $, $\displaystyle \forall x \in R $, entonces 
\[\int _{R}f_{1} \leq \int _{R}f_{2}  .\]
	\end{description}
\end{prop}
\begin{proof}
	Aplicamos el teorema de Darboux.
	\begin{enumerate}
	\item Sea $\displaystyle \epsilon > 0 $, como $\displaystyle f $ y $\displaystyle g $ son integrables tenemos que existen $\displaystyle \delta _{1}, \delta _{2} > 0 $ y $\displaystyle P_{1}, P_{2} \in \mathcal{P}\left(R\right) $, con $\displaystyle \|P_{1}\| < \delta_{1} $ y $\displaystyle \|P_{2}\|<\delta_{2} $, tales que
\[ \left|\sum_{J \in P_{1}}f\left(y_{j}\right)v\left(J\right)-I_{1}\right|< \frac{\epsilon }{2}, \quad \left|\sum_{T \in P_{2}}g\left(z_{j}\right)v\left(J\right)-I_{2}\right|< \frac{\epsilon }{2}.\]
Cogemos $\displaystyle \delta = \min \left\{ \delta_{1}, \delta_{2}\right\}  > 0 $, y tomamos $\displaystyle P \in \mathcal{P}\left(R\right) $ con $\displaystyle \|P\|<\delta  $, de esta forma 
\[\left|\sum_{J \in P}\left(f + g\right)\left(x_{j}\right)v\left(J\right)-\left(I_{1} + I_{2}\right)\right| \leq \left|\sum^{}_{J \in P}f\left(x_{J}\right)v\left(J\right) - I_{1}\right| + \left|\sum^{}_{J \in P}g\left(x_{J}\right)v\left(J\right)-I_{2}\right| < \frac{\epsilon }{2} + \frac{\epsilon }{2} = \epsilon.\]
\item Sea $\displaystyle I = \int _{R}f $ y $\displaystyle \epsilon > 0 $. Por el teorema de Darboux, existe $\displaystyle \delta > 0 $ tal que si $\displaystyle P \in \mathcal{P}\left(R\right) $ con $\displaystyle \|P\| < \delta  $, entonces
	\[ \left|\sum_{J \in P}f\left(x_{J}\right)v\left(J\right)-I\right|< \frac{\epsilon }{ \left|\alpha \right|}.\]
	Así, tenemos que 
	\[ \left|\sum_{J \in P}\alpha f\left(x_{J}\right)v\left(J\right)-\alpha I\right| = \left|\alpha \right| \left|\sum_{J \in P}f\left(x_{J}\right)v\left(J\right)-I\right| < \left|\alpha \right|\frac{\epsilon }{ \left|\alpha \right|} = \epsilon  .\]
\item Para demostrar la monotonía primero asumimos que $\displaystyle f : R \to \R $ es integrable y $\displaystyle f\left(x\right) \geq 0 $, $\displaystyle \forall x \in R $. Tenemos que 
	\[s\left(f,P\right) = \sum_{J \in P} \alpha_{J} v\left(J\right) \geq 0, \; \forall P \in \mathcal{P}\left(R\right) .\]
Así, está claro que la integral inferior debe ser superior a 0, por lo que el valor de la integral será superior a 0. Ahora, supongamos que $\displaystyle f_{1}, f_{2} : R \to \R $ son integrables y $\displaystyle f_{1}\left(x\right) \leq f_{2} \left(x\right)$, $\displaystyle \forall x \in R $. Tenemos que la función $\displaystyle f_{2}-f_{1} : R \to \R $ cumple que es integrable (por la linealidad) y además $\displaystyle \left(f_{2}-f_{1}\right)\left(x\right) \geq 0 $, $\displaystyle \forall x \in R $. Por lo que acabamos de demostrar:
\[\int _{R}f_{2}-f_{1} = \int_{R} f_{2} - \int _{R}f_{1} \geq 0 \iff \int _{R}f_{1} \leq \int _{R}f_{2}  .\]
	\end{enumerate}
\end{proof}

\begin{observation}[Cota de la integral]
Supongamos que $\displaystyle f:R \to \R $ es integrable en $\displaystyle R $. Entonces, $\displaystyle f $ es acotada, por lo que existen $\displaystyle m,M \in \R $ tales que 
\[m \leq f\left(x\right)\leq M, \; \forall x \in \R .\]
Así, tenemos que 
\[\int _{R}m \leq \int _{R}f \leq \int _{R}M \Rightarrow mv\left(R\right) \leq \int _{R}f \leq Mv\left(R\right) .\]
Se puede hacer un razonamiento igual sobre un conjunto $\displaystyle A $ que es medible Jordan.
\end{observation}
 
\begin{prop}
Sean $\displaystyle S,R \subset \R^{n} $ rectángulos cerrados tales que $\displaystyle S \subset R $. Si $\displaystyle f : R \to \R $ es integrable, entonces $\displaystyle f $ es integrable en $\displaystyle S $.
\end{prop}
\begin{proof}
	Sea $\displaystyle \epsilon > 0 $ y $\displaystyle P = \left(P_{1}, \ldots, P_{n}\right) \in \mathcal{P}\left(R\right) $ tal que $\displaystyle S\left(f,P\right)-s\left(f,P\right) < \epsilon  $. Podemos asumir que los vértices de $\displaystyle S $ están en $\displaystyle P $, es decir, si $\displaystyle S = \left[a_{1}, b_{1}\right] \times \cdots \times \left[a_{n}, b_{n}\right]  $, entonces $\displaystyle a_{i}, b_{i} \in P_{i} $, $\displaystyle \forall i = 1, \ldots, n $. 
	Ahora, consideramos $\displaystyle \tilde{P} = \left(\tilde{P}_{1}, \ldots, \tilde{P}_{n}\right) \in \mathcal{P}\left(S\right) $ tal que $\displaystyle \tilde{P}_{i} = P_{i} \cap \left[a_{i}, b_{i}\right]  $ con $\displaystyle i = 1, \ldots, n $ . 
	Dado que los subrectángulos de $\displaystyle \tilde{P} $ son subrectángulos de $\displaystyle P $ es fácil ver que $\displaystyle S $ es una unión de subrectángulos de $\displaystyle R $. Sean $\displaystyle R_{1}, \ldots, R_{k} $ los subrectángulos de $\displaystyle P $ que también lo son de $\displaystyle \tilde{P}  $, y sean $\displaystyle R_{k + 1}, \ldots, R_{N} $ el resto. 
	Tenemos que 
\[
\begin{split}
	\epsilon & > S\left(f,P\right)-s\left(f,P\right) = \sum^{k}_{j = 1}\left(\beta_{j}-\alpha_{j}\right)v\left(R_{j}\right) + \sum^{N}_{j = k + 1}\left(\beta_{j}-\alpha_{j}\right)v\left(R_{j}\right) \\
		 & \geq \sum^{k}_{j = 1}\left(\beta_{j}-\alpha_{j}\right)v\left(R_{j}\right) = S\left(f|_{S}, \tilde{P}\right) - s\left(f|_{S}, \tilde{P}\right).
\end{split}
\]
Por el criterio de la integrabilidad de Riemann, $\displaystyle f|_{S} $ es integrable.	
\end{proof}
\begin{lema}
Sean $\displaystyle R, R' \subset \R^{n} $ rectángulos con $\displaystyle R' \subset R $. Supongamos que $\displaystyle f : R \to \R $ es integrable en $\displaystyle R' $ y $\displaystyle f\left(x\right) = 0 $ para $\displaystyle x \in \overline{R/R'} $. Entonces $\displaystyle f $ es integrable en $\displaystyle R $ y 
\[\int _{R'}f =\int _{R}f .\]
\end{lema}
\begin{proof}
	Sea $\displaystyle \tilde{f} := f|_{R'} $ y sea $\displaystyle \epsilon > 0 $, entonces existe $\displaystyle \tilde{P} \in \mathcal{P}\left(R'\right) $ tal que 
	\[S\left(\tilde{f}, \tilde{P}\right)-s\left(\tilde{f}, \tilde{P}\right) < \epsilon  .\]
	Cogemos ahora la partición $\displaystyle P \in \mathcal{P}\left(R\right) $ que resulta de extender $\displaystyle \tilde{P} $ a una partición de $\displaystyle R $, como se muestra en la figura. 
	% Añadir una imagen para explicar. 
	Entonces, tendremos que $\displaystyle R' $ es una unión de subrectángulos de $\displaystyle P $. Sean $\displaystyle R_{1}, \ldots, R_{k} $ los subrectángulos de $\displaystyle P $ que también lo son de $\displaystyle \tilde{P} $ y $\displaystyle R_{k+1}, \ldots, R_{N} $ el resto. Así, tenemos que 
\[
\begin{split}
S\left(f, P\right)-s\left(f,P\right) = \sum^{k}_{i = 1}\left(\beta_{i}-\alpha_{i}\right)v\left(R_{i}\right) + \sum^{N}_{i = k + 1}\left(\beta_{i}-\alpha_{i}\right)v\left(R_{i}\right) < \epsilon + \sum^{N}_{i = k + 1}\left(\beta_{i}-\alpha_{i}\right)v\left(R_{i}\right) .
\end{split}
\]
Los rectángulos $\displaystyle R_{k+1}, \ldots, R_{N} $ pueden cumplir una de dos cosas.
\begin{itemize}
\item Si $\displaystyle R_{j} \cap R' = \emptyset $, entonces $\displaystyle f|_{R_{j}} \equiv 0 $ y se tiene que $\displaystyle \beta_{j}-\alpha_{j} = 0 $.
\item Si $\displaystyle R_{j} \cap R' \neq \emptyset $, entonces, debe ser que su intersección con $\displaystyle R' $ no es más que un segmento del borde de $\displaystyle R' $, es decir, $\displaystyle R_{j} \cap R' \subset \overline{R/R'} $, por lo que de nuevo tenemos que $\displaystyle \beta_{j}-\alpha_{j} = 0 $. 
\end{itemize}
Así, hemos visto que $\displaystyle \sum^{N}_{i=k+1}\left(\beta_{i}-\alpha_{i}\right)v\left(R_{i}\right)= 0 $, por lo que $\displaystyle S\left(f,P\right)-s\left(f,P\right) < \epsilon  $ y tenemos que $\displaystyle f $ es integrable en $\displaystyle R $. Veamos ahora que el valor de las integrales coincide. Sea $\displaystyle I = \int _{R'}\tilde{f} $. Sea $\displaystyle \epsilon > 0 $, podemos coger $\displaystyle \tilde{P} \in \mathcal{P}\left(R'\right) $ tal que $\displaystyle I - s\left(\tilde{f}, \tilde{P}\right) < \epsilon  $. Cogemos $\displaystyle P \in \mathcal{P}\left(R\right) $ extendiendo $\displaystyle \tilde{P} $ a una partición de $\displaystyle R $ como hemos hecho antes, así
\[0 \leq I - s\left(f,P\right) = I - \sum_{J \in P, J \subset R'}\alpha_{J}v\left(J\right) = I - \sum_{J \in \tilde{P}}\tilde{\alpha}_{J}v\left(J\right) =  I - s\left(\tilde{f}, \tilde{P}\right) < \epsilon .\]
Así, hemos obtenido que 
\[\underline{\int _{R}}f = \int _{R}f = I .\]
\end{proof}

\begin{prop}
Sean $\displaystyle R, R' \subset \R^{n} $ rectángulos con $\displaystyle R' \subset R $. Supongamos que $\displaystyle f : R \to \R $ es integrable en $\displaystyle R' $ y $\displaystyle f\left(x\right) = 0 $ para $\displaystyle x \in R/R' $. Entonces $\displaystyle f $ es integrable en $\displaystyle R $ y 
\[\int _{R'}f =\int _{R}f .\]
\end{prop}
\begin{proof}
Supongamos que $\displaystyle f\left(x\right)=g\left(x\right) + h\left(x\right) $ con $\displaystyle g, h : R \to \R $ tales que
\[
	g\left(x\right)=
\begin{cases}
0, \; x \in \overline{R/R'} \\
f\left(x\right), \; \text{resto}
\end{cases}
, \quad h\left(x\right) = 
\begin{cases}
0, \; x \in R/\partial R' \\
f\left(x\right), \; \text{resto}
\end{cases}
.\]
Veamos que $\displaystyle \int _{R}h =\int _{R'}h =0 $. Veamos primero que $\displaystyle h $ es integrable en $\displaystyle R $. Se puede comprobar que $\displaystyle \partial R' $ tiene contenido nulo, por lo que para $\displaystyle \epsilon > 0 $, existe $\displaystyle P \in \mathcal{P}\left(R\right) $ tal que 
\[\sum_{J \in P, J \cap \partial R' \neq \emptyset}v\left(J\right) < \frac{\epsilon }{M}, \; \; \text{donde}\; \left|f\right| \leq M .\]
Así, tendremos que 
\[S\left(h, P\right) = \sum_{J \in P}\beta_{J}v\left(J\right) = \sum_{J \in P, J \cap \partial R' \neq \emptyset} \beta_{J}v\left(J\right) \leq M\sum_{J \in P, J \cap \partial R' \neq \emptyset}v\left(J\right) < \epsilon .\]
Así, $\displaystyle h $ es integrable en $\displaystyle R $ y su integral es nula. Por una proposición anterior, como $\displaystyle R' \subset R $, tenemos que $\displaystyle h $ también es integrable en $\displaystyle R' $. De esta manera, como $\displaystyle s\left(h|_{R'}, P\right) = 0 $, $\displaystyle \forall P \in \mathcal{P}\left(R'\right) $, tenemos que 
\[\underline{\int _{R'} }h = \int _{R'}h = 0 .\]
Veamos ahora que $\displaystyle g $ es integrable en $\displaystyle R' $. Primero, como $\displaystyle f $ está acotada, tendremos que existe $\displaystyle K > 0 $ tal que $\displaystyle \beta_{g,J}-\alpha_{g,J} \leq K $, $\displaystyle \forall J \in P $ con $\displaystyle J \cap \partial R' \neq \emptyset $. Ahora, sea $\displaystyle \epsilon > 0 $, entonces existe $\displaystyle P_{1} \in \mathcal{P}\left(R'\right) $ tal que $\displaystyle S\left(\tilde{f}, P_{1}\right)-s\left(\tilde{f}, P_{1}\right) < \frac{\epsilon }{2}  $. Además, como $\displaystyle \partial R' $ tiene contenido nulo, existe $\displaystyle P_{2}\in \mathcal{P}\left(R'\right) $ tal que $\displaystyle \sum_{J \in P_{2}, J \cap \partial R'}v\left(J\right) < \frac{\epsilon }{2K} $. Así, cogiendo $\displaystyle P \in \mathcal{P}\left(R'\right) $ más fina que $\displaystyle P_{1} $ y $\displaystyle P_{2} $,
\[
\begin{split}
	S\left(g|_{R'}, P\right)-s\left(g|_{R'}, P\right) = & \sum_{J \in P}\left(\beta_{g,J}-\alpha_{g,J}\right)v\left(J\right) \\
	= &  \sum_{J \cap \partial R' = \emptyset} \left(\beta_{g,J}-\alpha_{g,J}\right)v\left(J\right) + \sum_{J \cap \partial R' \neq \emptyset}\left(\beta_{g, J}-\alpha_{g,J}\right)v\left(J\right) \\
	\leq & \sum_{J \in P}\left(\beta_{f, J}-\alpha_{f,J}\right)v\left(J\right) + \sum^{}_{J \cap \partial R' \neq \emptyset}\left(\beta_{g,J}-\alpha_{g,J}\right)v\left(J\right) \\
	< & \frac{\epsilon }{2} + \frac{\epsilon }{2} = \epsilon .
\end{split}
\]
Así, tenemos que $\displaystyle g $ es integrable en $\displaystyle R' $ y por el lema anterior tenemos que también es integrable en $\displaystyle R $ y además
\[\int _{R'}g= \int _{R}g  .\]
Por otro lado, sea $\displaystyle I= \int _{R'}f  $, entonces, suponiendo que $\displaystyle I \geq \int _{R'}g  $ tenemos que 
\[
\begin{split}
	0 \leq & I - \int _{R'}g \leq S\left(f|_{R'}, P\right) - s\left(g|_{R'}, P\right) \\
	= & \sum_{J \cap \partial R' = \emptyset}\beta_{f,J}v\left(J\right) + \sum_{J \cap \partial R'\neq \emptyset}\beta_{f,J}v\left(J\right)  - \sum_{J \cap \partial R' = \emptyset}\alpha_{g,J}v\left(J\right) - \sum^{}_{J \cap \partial R'\neq \emptyset}\alpha_{g,J}v\left(J\right) \\
	= & \sum_{J \cap \partial R' = \emptyset}\left(\beta_{f,J}-\alpha_{f,J}\right)v\left(J\right) + \sum_{J \cap \partial R' \neq \emptyset}\left(\beta_{f,J}-\alpha_{g,J}\right)v\left(J\right) \\
	\leq & \sum_{J \in P}\left(\beta_{f,J}-\alpha_{f,J}\right)v\left(J\right) + \sum_{J \cap \partial R' \neq \emptyset}\tilde{K}v\left(J\right) < \epsilon.
\end{split}
\]
El objetivo de $\displaystyle \tilde{K} $ es acotar la resta $\displaystyle \beta_{f,J}-\alpha_{g,J} \geq 0 $, lo cual podemos hacer puesto que $\displaystyle f $ y $\displaystyle g $ son acotadas y valen lo mismo en casi todos los puntos. El primer término lo podemos hacer arbitrariamente pequeño por la integrabilidad de $\displaystyle f $ en $\displaystyle R' $ y el segundo porque $\displaystyle \partial R' $ tiene contenido nulo. El caso $\displaystyle I \leq \int _{R'}g  $ es análogo.

Finalmente, como $\displaystyle f $ es suma de funciones integrables, tendremos que 
\[\int _{R}f =\int _{R}g + h = \int _{R}g +\int _{R}h  = \int _{R'}g = \int _{R'}f  .\]
\end{proof}

\section{Conjuntos medibles Jordan}
\begin{definition}[Conjunto medible Jordan]
Sea $\displaystyle A \subset \R^{n} $ con $\displaystyle A\neq \emptyset $ y $\displaystyle A $ acotado. Tomamos un rectángulo $\displaystyle R $ tal que $\displaystyle A \subset R $. Definimos la función 
\[\chi_{A} : R \to \R : x \to 
\begin{cases}
1, \; x \in A \\
0, \; x \not\in A
\end{cases}
.\]
Diremos que $\displaystyle A $ tiene \textbf{volumen} (A es \textbf{medible Jordan}) si $\displaystyle \chi_{A} $ es integrable y en este caso diremos que su \textbf{volumen} es 
\[V\left(A\right) = \int _{R}\chi_{A} .\]
\end{definition}

\begin{eg}
\begin{itemize}
\item Ya vimos anteriormente que $\displaystyle \Q^{n} \cap \left[0,1\right] ^{n} $ no tiene volumen.
\item Si $\displaystyle R $ es un rectángulo, $\displaystyle R $ es medible Jordan.
\end{itemize}

	\end{eg}
\begin{observation}
Sea $\displaystyle \emptyset \neq A \subset \R^{n} $ acotado y $\displaystyle R $ un rectángulo tal que $\displaystyle A \subset R $. Cogemos $\displaystyle P \in \mathcal{P}\left(R\right) $. Podemos observar que 
\[\alpha_{J} = 
\begin{cases}
1, \; J \subset A \\
0, \; J \subsetneq A
\end{cases}
.\]
Así, tendremos que 
\[s\left(\chi_{A}, P\right) = \sum_{J \in P}v\left(J\right)\alpha_{J} = \sum_{J \in P, J \subset A}v\left(J\right).\]
De esta forma,
\[\underline{\int _{R}} \chi_{A} = \sup \left\{ \sum_{J \in P, J \subset A}v\left(J\right) \; : \; P \in \mathcal{P}\left(R\right)\right\} .\]
Consideremos ahora las sumas superiores,
\[\beta_{J} = 
\begin{cases}
1, \; J \cap A \neq \emptyset \\
0, \; J \cap A = \emptyset
\end{cases}
.\]
De esta forma,
\[S\left(\chi_{A}, P\right) = \sum_{J \in P, J \cap A \neq \emptyset}v\left(J\right) .\]
Así, 
\[\overline{\int _{R} }\chi_{A} = \inf \left\{\sum_{J \in P, J \cap A \neq \emptyset}v\left(J\right)\; : \; P \in \mathcal{P}\left(R\right) \right\}  .\]
\end{observation}

\begin{definition}[Integral en otros conjuntos]
Sea $\displaystyle A \subset \R^{n} $ acotado y $\displaystyle f : A \to \R $ también acotada. Diremos que $\displaystyle f $ \textbf{es integrable en} $\displaystyle A $ si existe $\displaystyle R $ rectángulo tal que $\displaystyle A \subset R $ y 
\[\tilde{f} : R \to \R : x \to 
\begin{cases}
f\left(x\right), \; x \in A \\
0, \; x \not\in A
\end{cases}
,\]
es integrable en $\displaystyle R $. En este caso
\[\int _{A}f := \int _{R}\tilde{f}  .\]
De forma equivalente, desde $\displaystyle f : R \to \R $ y $\displaystyle A \subset R $, diremos que $\displaystyle f $ es integrable en $\displaystyle A $ si $\displaystyle f \cdot \chi_{A} $ es integrable en $\displaystyle R $ y tomamos
\[\int _{A}f = \int _{R}f \cdot \chi_{A}  .\]
\end{definition}

\begin{observation}
	Tanto para la definición anterior como para la de volumen, tenemos que ver que basta con que exista $\displaystyle R $, puesto que en cuanto existe uno para cualquier otro rectángulo que cumpla estas características el valor de la integral coincide. 

	En efecto, sea $\displaystyle \emptyset\neq  A \subset \R^{n}$ tal que $\displaystyle \chi_{A} $ es integrable en el rectángulo $\displaystyle R \supset A $ y sea $\displaystyle R' \supset A $ otro rectángulo. Tenemos que $\displaystyle R \cap R' $ es un rectángulo que contiene a $\displaystyle A $. Por las últimas proposiciones de la sección anterior tenemos que, por ser $\displaystyle f $ integrable en $\displaystyle R $ lo es en $\displaystyle R \cap R' $ y además
	\[\int _{R}\chi_{A} = \int _{R \cap R'}\chi_{A} = \int _{R'}\chi_{A}  .\]
\end{observation}
Las integrales sobre conjuntos que no sean rectángulos cumplen propiedades parecidas a las integrales sobre rectángulos.
\begin{prop}[Linealidad y monotonía] Sea $\displaystyle A \subset \R^{n} $ con $\displaystyle A \neq \emptyset $. 
	\begin{enumerate}
	\item \textbf{Linealidad.} Si $\displaystyle f_{1}, f_{2} : A \to \R $ son integrables en $\displaystyle A $, entonces
		\[\int _{A}f_{1} + f_{2} = \int _{A}f_{1} +\int _{A}f_{2}  .\]
		Además, si $\displaystyle \alpha\in \R $ y $\displaystyle f : A \to \R $ es integrable en $\displaystyle A $, entonces
		\[\int _{A}\alpha f = \alpha \int _{A}f  .\]
	\item \textbf{Monotonía.} Si $\displaystyle f_{1}, f_{2} : A \to \R $ son integrables en $\displaystyle A $, con $\displaystyle f_{1} \leq f_{2} $, $\displaystyle \forall x \in A $, entonces
		\[\int _{A}f_{1} \leq \int _{A}f_{2}  .\]
	\end{enumerate}
\end{prop}
\begin{proof} Sea $\displaystyle A \subset \R^{n} $ con $\displaystyle A \neq \emptyset $. 
\begin{enumerate}
\item Como $\displaystyle f_{1} $ y $\displaystyle f_{2} $ son integrables en $\displaystyle A $, existe $\displaystyle R $ rectángulo tal que $\displaystyle A \subset R $ y existen $\displaystyle \int _{R}\tilde{f}_{1} $ y $\displaystyle \int _{R}\tilde{f}_{2} $.
Así, tendremos que existe
\[\int _{A}f_{1} + f_{2} = \int _{R}\widetilde{f_{1} + f_{2}} = \int _{R}\tilde{f}_{1} + \tilde{f}_{2} =  \int _{R}\tilde{f}_{1} +\int _{R}\tilde{f}_{2} = \int _{A}f_{1} +\int _{A}f_{2} .\]
Similarmente, si $\displaystyle \alpha \in \R $ y $\displaystyle f : A \to \R $ es integrable en $\displaystyle A $, entonces existe $\displaystyle R \subset \R^{n} $ rectángulo con $\displaystyle A \subset R $ y existe $\displaystyle \int _{R}\tilde{f}  $. Así, existe
\[\int _{A}\alpha f = \int _{R}\widetilde{\alpha f} = \int _{R}\alpha\tilde{f} = \alpha \int _{R}\tilde{f} = \alpha\int _{A}f .\]
\item Como $\displaystyle f_{1} $ y $\displaystyle f_{2} $ son integrables en $\displaystyle A $, existe $\displaystyle R \subset \R^{n} $ rectángulo con $\displaystyle A \subset R $ tal que $\displaystyle \tilde{f}_{1} $ y $\displaystyle \tilde{f}_{2} $ son integrables en $\displaystyle R $. Es fácil ver que $\displaystyle \tilde{f}_{1} \leq \tilde{f}_{2} $, $\displaystyle \forall x \in R $, por lo que 
	\[\int _{A}f_{1} = \int _{R}\tilde{f}_{1} \leq \int _{R}\tilde{f}_{2} =\int _{A}f_{2}  .\]
\end{enumerate}
\end{proof}

\begin{definition}[Media]
Sea $\displaystyle f : A \to \R $ integrable con $\displaystyle A \subset \R^{n} $ medible Jordan y $\displaystyle v\left(A\right) > 0 $, definimos la \textbf{media} de $\displaystyle f $ en $\displaystyle A $ como 
\[m_{A}\left(f\right) = \frac{1}{v\left(A\right)}\int _{A}f .\]
\end{definition}
\begin{observation}
Si $\displaystyle f : A \to \R $ es integrable, entonces está acotada por lo que existen $\displaystyle M,m > 0 $ tales que $\displaystyle m \leq f\left(x\right) \leq M $, así
\[mv\left(A\right) \leq \int _{A}f \leq Mv\left(A\right) \iff m\leq \frac{1}{v\left(A\right)}\int _{A}f \leq M \iff m \leq m_{A}\left(f\right) \leq M .\]
Podemos observar que 
\[\int _{A}m_{A}\left(f\right) = \int _{A}f .\]
\end{observation}
\begin{theorem}[Teorema del valor medio]
Sea $\displaystyle \emptyset \neq A \subset \R^{n} $ un conjunto compacto y conexo, medible Jordan con volumen positivo, y sea $\displaystyle f : A \to \R $ continua. Entonces, existe $\displaystyle x_{m} \in A $ tal que 
\[f\left(x_{m}\right) = m_{A}\left(f\right) = \frac{1}{v\left(A\right)}\int _{A}f .\]
\end{theorem}
\begin{proof}
Como $\displaystyle f  $ es continua y $\displaystyle A $ es compacto y conexo, debe ser que $\displaystyle f\left(A\right) $ también es compacto y conexo (por lo que debe ser un intervalo cerrado y acotado), es decir, existen $\displaystyle x_{1}, x_{2} \in A $ tales que 
\[m = f\left(x_{1}\right) \leq f\left(x\right) \leq M = f\left(x_{2}\right), \; \forall x \in A .\]
Así, tenemos que $\displaystyle f\left(A\right) = \left[m,M\right]  $. Anteriormente hemos visto que $\displaystyle m \leq m_{A}\left(f\right) \leq M $, por lo que $\displaystyle m_{A}\left(f\right) \in f\left(A\right) $ y en consecuencia existe $\displaystyle x_{m} \in A $ tal que $\displaystyle f\left(x_{m}\right) = m_{A}\left(f\right) $. 
\end{proof}
\begin{prop}
Si $\displaystyle f : A \to \R $ es integrable, entonces $\displaystyle \left|f\right| $ es integrable y
\[ \left|\int _{A}f \right| \leq \int _{A} \left|f\right| .\]
\end{prop}
\begin{proof}
Demostramos la segunda parte. Claramente tenemos que 
\[- \left|f\left(x\right)\right| \leq f\left(x\right) \leq \left|f\left(x\right)\right|, \; \forall x \in A .\]
Por la propiedad de la monotonía tenemos que 
\[-\int _{A} \left|f\right| \leq \int _{A}f \leq \int _{A} \left|f\right| \iff \left|\int _{A}f \right|\leq \int _{A} \left|f\right| .\]
\end{proof}

\begin{observation}
Sea $\displaystyle A $ medible Jordan y $\displaystyle \left|f\left(x\right)\right| \leq M $, $\displaystyle \forall x \in A $. Entonces, tenemos que 
\[ \left|\int _{A}f \right|\leq \int _{A} \left|f\right| \leq M v\left(A\right) .\]
De esta manera, podemos utilizar la proposición anterior para obtener una cota de la integral.
\end{observation}

\begin{prop}[Aditividad de la integral respecto de los conjuntos de integración]
Sean $\displaystyle A, B \subset \R^{n} $ acotados con $\displaystyle f : A \cup B\to \R $ acotada. Supongamos que $\displaystyle \int _{A \cap B}f = 0 $ y $\displaystyle f $ es integrable en $\displaystyle A $ y en $\displaystyle B $. Entonces $\displaystyle f $ es integrable en $\displaystyle A \cup B $ y 
\[\int _{A \cup B}f = \int _{A}f + \int _{B}f  .\]
\end{prop}
\begin{proof}
Es fácil ver que 
	\[\chi_{A \cup B} = \chi_{A} + \chi_{B} - \chi_{A \cap B} .\]
Por tanto, tenemos que 
\[f \cdot \chi_{A \cup B} = f \cdot \chi_{A} + f \cdot \chi_{B} - f \cdot \chi_{A \cap B} .\]
Si cogemos $\displaystyle R \subset \R^{n} $ rectángulo tal que $\displaystyle A \cup B \subset R $, aplicando las hipótesis de la proposición y la linealidad de la integral tendremos que
\[\int _{A \cup B}f = \int _{R}f \cdot \chi_{A \cup B} = \int _{R} f \cdot \chi_{A} + \int _{R} f \cdot \chi_{B}  - \int _{R} f \cdot \chi_{A \cap B}  = \int _{A}f + \int _{B}f  .\]
\end{proof}

\section{Conjuntos de medida nula}
\begin{lema}
Sea $\displaystyle \emptyset \neq A \subset \R^{n} $. Entonces, $\displaystyle A $ tiene contenido nulo \footnote{$\displaystyle A $ es medible Jordan y $\displaystyle v\left(A\right) = 0 $.} si y solo si existe $\displaystyle R \supset A $ tal que $\displaystyle \forall \epsilon > 0 $, $\displaystyle \exists P \in \mathcal{P}\left(R\right) $ tal que $\displaystyle \sum^{}_{J \in P, \; J \cap A \neq \emptyset}v\left(J\right) < \epsilon  $.
\end{lema}
\begin{proof}
Tenemos que $\displaystyle A $ tiene contenido nulo si y solo si es medible Jordan y $\displaystyle v\left(A\right) = 0 $, es decir, si existe $\displaystyle R \subset \R^{n} $ rectángulo con $\displaystyle A \subset R $ y 
\[0 \leq \underline{\int _{R} }\chi_{A} \leq \overline{\int _{R} }\chi_{A} \leq 0 .\]
Esto último sucede si y solo si $\displaystyle \inf \left\{ S\left(\chi_{A}, P\right)\; : \; P \in \mathcal{P}\left(R\right)\right\} \leq 0 $, es decir, si $\displaystyle \forall \epsilon > 0 $ existe $\displaystyle P \in \mathcal{P} \left(R\right) $ tal que \footnote{Esto último se deduce de la caracterización de ínfimo: $\displaystyle S\left(\chi_{A}, P\right)< \overline{\int _{R} }\chi_{A} + \epsilon \leq \epsilon  $.} 
\[ S\left(\chi_{A}, P\right) = \sum_{J \in P, \; J \cap A \neq \emptyset}v\left(J\right) < \epsilon  .\]
\end{proof}
\begin{prop}[Caracterización de contenido nulo]
	$\displaystyle A \subset \R^{n} $ tiene contenido nulo si y solo si $\displaystyle \forall \epsilon > 0 $ existe $\displaystyle \left\{ J_{1}, \ldots, J_{N}\right\}  $ familia finita de rectángulos tal que 
	\[A \subset \bigcup_{i =1}^{N}J_{i}\quad \text{y} \quad \sum^{N}_{i =1} v\left(J_{i}\right) < \epsilon  .\]
\end{prop}
\begin{proof}
	La primera implicación es trivial a partir del lema anterior, por lo que demostraremos únicamente la segunda implicación. Sea $\displaystyle \epsilon > 0 $, entonces existe $\displaystyle \left\{ J_{1}, \ldots, J_{N}\right\}  $ familia de rectángulos que recubren $\displaystyle A $ y $\displaystyle \sum^{N}_{i = 1}v\left(J_{i}\right)< \epsilon  $. Cogemos $\displaystyle R \subset \R^{n} $ un rectángulo grande tal que $\displaystyle \bigcup_{i = 1}^{N}J_{i} \subset R $. 
	Podemos obtener una partición $\displaystyle P\in \mathcal{P}\left(R\right) $ tal que cualquiera de los $\displaystyle J_{i} $ es unión de rectángulos de $\displaystyle P $. Entonces,
	\[\sum_{J \in P, \; J \cap A \neq \emptyset}v\left(J\right) = \sum^{N}_{i = 1}\sum_{J \in P, J \subset J_{i}}v\left(J\right) \leq \sum^{N}_{i = 1}v\left(J_{i}\right) < \epsilon  .\]
	Por el lema anterior, tenemos que $\displaystyle A $ tiene contenido nulo.
\end{proof}
\begin{prop}
	Sea $\displaystyle \left\{ A_{1}, \ldots, A_{N}\right\} \subset \R^{n} $ una familia finita de conjuntos de contenido nulo. Entonces, $\displaystyle \bigcup_{i = 1}^{N}A_{i} $ tiene contenido nulo.
\end{prop}
\begin{proof}
Sea $\displaystyle \epsilon > 0 $. Para cada $\displaystyle A_{i} $ existe una familia $\displaystyle \left\{ J_{1}^{i}, \ldots, J_{N_{i}}^{i}\right\}  $ de rectángulos tales que 
	\[ A_{i} \subset \bigcup_{1 \leq j \leq N_{i}}J_{j}^{i} \quad \text{y} \quad \sum^{N_{i}}_{j = 1}v\left(J_{j}^{i}\right) < \frac{\epsilon }{N}.\]
	Así, si cogemos la familia $\displaystyle \mathcal{F}=\bigcup_{i = 1}^{N} \left\{ J^{i}_{1}, \ldots, J_{N_{i}}^{i}\right\}  $, tendremos que claramente $\displaystyle \bigcup_{i = 1}^{N}A_{i} $ está recubierto por $\displaystyle \mathcal{F} $ y además
	\[ \sum_{J \in \mathcal{F}}v\left(J\right) = \sum^{N}_{i = 1}\sum^{N_{i}}_{j = 1}v\left(J^{i}_{j}\right) < \sum^{N}_{i = 1}\frac{\epsilon }{N} = \epsilon  .\]
Una demostración alternativa es hacerlo por inducción. Supongamos que $\displaystyle A $ y $\displaystyle B $ tienen contenido nulo y veamos que $\displaystyle A \cup B $ también tiene contenido nulo. Como $\displaystyle A \cap B \subset A $, es sencillo ver que también tendrá contenido nulo, y por la propiedad de aditividad tendremos que para un rectángulo $\displaystyle R \supset A \cup B $ se tiene que
\[v\left(A \cup B\right) = \int _{R} \chi_{A \cup B} = \int _{R} \chi_{A} + \int _{R}\chi_{B} -\int _{R}\chi_{A \cap B} = v\left(A\right) + v\left(B\right)-v\left(A \cap B\right) = 0 .\]
A partir de aquí es sencillo ver inducitvamente que la proposición se cumple para una familia finita de conjuntos de contenido nulo.
\end{proof}

\begin{observation}
	En la proposición anterior es importante que la familia de conjuntos sea finita y no numerable. En efecto, ya vimos que $\displaystyle \Q \cap [0,1]$  no tiene contenido nulo.
\end{observation}

\begin{definition}[Medida nula]
	Sea $\displaystyle A \subset \R^{n} $. Decimos que $\displaystyle A $ tiene \textbf{medida nula} si $\displaystyle \forall \epsilon > 0 $, $\displaystyle \exists \left\{ J_{k}\right\} _{k \in \N} $ familia de rectángulos tales que 
	\[A \subset \bigcup_{k \in \N}J_{k} \quad \text{y} \quad \sum^{\infty}_{k = 1}v\left(J_{k}\right) < \epsilon  .\]
\end{definition}
\begin{lema}
	Un conjunto no vacío $\displaystyle A \subset \R^{n} $ tiene medida nula si y solo si $\displaystyle \forall \epsilon > 0 $ existe una sucesión de rectángulos $\displaystyle \left\{ R_{j}\right\} _{j \in \N} $ tales que 
	\[ A \subset \bigcup_{j \in \N}\Int\left(R_{j}\right) \quad \text{y}\quad \sum^{\infty}_{j = 1}v\left(R_{j}\right) < \epsilon  .\]
\end{lema}
\begin{proof}
	La segunda implicación es trivial, por lo que sólo demostraremos la primera. Dado $\displaystyle \epsilon > 0 $, existe una sucesión de rectángulos $\displaystyle \left\{ Q_{j}\right\} _{j \in \N} $ tales que 
	\[A \subset \bigcup_{j \in \N}Q_{j} \quad \text{y}\quad \sum^{\infty}_{j= 1}v\left(Q_{j}\right)< \frac{\epsilon }{2} ,\]
	donde $\displaystyle Q_{j} = \left[a_{1}^{j}, b_{1}^{j}\right] \times \left[a_{n}^{j}, b_{n}^{j}\right]  $ y $\displaystyle v\left(Q_{j}\right)=\prod^{n}_{k=1}\left(b^{j}_{k}-a^{j}_{k}\right) $, $\displaystyle \forall j \in \N $. Para $\displaystyle \delta_{j} > 0 $ definimos
	\[R_{j}=\left[a_{1}^{j}-\delta_{j}, b_{1}^{j}+\delta_{j}\right] \times \cdots \times \left[a_{n}^{j}-\delta_{j}, b_{n}^{j}+\delta_{j}\right]  .\]
Claramente tenemos que $\displaystyle Q_{j}\subset \Int\left(R_{j}\right) $ y además
\[v\left(R_{j}\right) = \prod^{n}_{k = 1}\left(b^{j}_{k}-a^{j}_{k}+2\delta_{j}\right)\xrightarrow{\delta_{j}\to0}v\left(Q_{j}\right), \; \forall j \in \N .\]
Cogemos $\displaystyle \delta_{j} $ de forma que $\displaystyle v\left(R_{j}\right) \leq v\left(Q_{j}\right)+\frac{\epsilon }{2 \cdot 2^{j}} $. De esta forma tenemos que $\displaystyle A \subset \bigcup_{j \in \N}R_{j} $ y además
\[\sum^{\infty}_{j = 1}v\left(R_{j}\right) \leq \sum^{\infty}_{j = 1}\left(v\left(Q_{j}\right)+\frac{\epsilon }{2 \cdot 2^{j}}\right)=\sum^{\infty}_{j = 1}v\left(Q_{j}\right) + \sum^{\infty}_{j = 1}\frac{\epsilon }{2 \cdot 2^{j}} < \frac{\epsilon }{2} + \frac{\epsilon }{2} = \epsilon.\]
\end{proof}

\begin{prop}
Si $\displaystyle A \subset \R^{n} $ tiene contenido nulo entonces también tiene medida nula.
\end{prop}
\begin{proof}
	Supongamos que $\displaystyle A $ tiene contenido nulo y sea $\displaystyle \epsilon > 0 $. Tenemos que existe $\displaystyle \left\{ J_{1}, \ldots, J_{N}\right\}  $ familia de rectángulos tales que 
	\[A \subset \bigcup_{i = 1}^{N}J_{i} \quad \text{y} \quad \sum^{N}_{i = 1}v\left(J_{i}\right) < \frac{\epsilon }{2}  .\]
	Ahora, tomamos una sucesión cualquiera de rectángulos $\displaystyle \left\{ J_{N+k}\right\} _{k \in \N} $ tal que $\displaystyle v\left(J_{N + k}\right) < \frac{\epsilon }{2^{k+1}} $. De esta forma, es trivial que $\displaystyle \left\{ J_{i}\right\} _{i \in \N} $ recubre a $\displaystyle A $ y
	\[\sum^{\infty}_{i = 1}v\left(J_{i}\right) = \sum^{N}_{i = 1}v\left(J_{i}\right) + \sum^{\infty}_{k = 1}v\left(J_{N + k}\right) < \frac{\epsilon }{2} + \sum^{\infty}_{k = 1}\frac{\epsilon }{2^{k+1}} = \epsilon  .\]
\end{proof}
\begin{prop}
	Si $\displaystyle \left\{ A_{i}\right\} _{i \in I} $ es una sucesión o familia finita de conjuntos de medida nula, entonces $\displaystyle \bigcup_{i \in I} A_{i}$ también tiene medida nula.
\end{prop}
\begin{proof}
Sea $\displaystyle A = \bigcup_{i \in I}A_{i} $. Tomamos $\displaystyle \epsilon > 0 $ y para cada $\displaystyle i \in I $ tomamos un rectángulo $\displaystyle J_{i} $ tal que $\displaystyle A_{i} \subset J_{i} $ y $\displaystyle v\left(J_{i}\right)< \frac{\epsilon }{2^{i}} $. Claramente se tiene que
\[A \subset \bigcup_{i \in I}J_{i} \quad \text{y} \quad \sum_{i \in I}v\left(J_{i}\right) < \sum_{i \in I}\frac{\epsilon }{2^{i}} < \epsilon  .\]
Alternativamente, podemos ver que para un $\displaystyle i \in I $ se tiene que existe $\displaystyle \left\{ J_{ij}\right\} _{j \in \N} $ tal que 
\[A_{i} \subset \bigcup_{j \in \N}J_{ij} \quad \text{y} \quad \sum^{\infty}_{j = 1}v\left(J_{ij}\right) < \frac{\epsilon }{2^{i}} .\]
De esta forma, podemos coger la familia numerable $\displaystyle \left\{ J_{ij}\right\} _{i,j \in \N} $ que recubre a $\displaystyle A $ y tendremos que \footnote{El hecho de que podamos sumar primero en función de $\displaystyle J $ y luego de $\displaystyle i $ se debe a que los términos se pueden reordenar en una serie doble que es absolutamente convergente.} 
\[ \sum_{i \in I, j \in \N} = \sum_{i \in I}\sum^{\infty}_{j = 1}v\left(J_{ij}\right) < \sum_{i \in I}\frac{\epsilon }{2^{i}} < \epsilon .\]
\end{proof}
\begin{eg}
	Por la proposición anterior tenemos que $\displaystyle \Q \cap \left[0,1\right]  $ tiene medida nula pero como vimos no tiene contenido nulo puesto que no es medible Jordan. Es decir, \textbf{que un conjunto tenga medida nula no significa que tenga contenido nulo}.
\end{eg}
\begin{prop}
Si $\displaystyle \emptyset \neq B \subset A \subset \R^{n} $ tal que $\displaystyle A $ tiene contenido (resp. medida) nulo, entonces $\displaystyle B $ tiene contenido (resp. medida) nulo.
\end{prop}
\begin{proof}
	Supongamos que $\displaystyle B \subset A $ y $\displaystyle A $ tiene contenido nulo. Si $\displaystyle \epsilon > 0 $ tendremos que existe $\displaystyle \left\{ J_{1}, \ldots, J_{N}\right\}  $ familia de rectángulos que recubre a $\displaystyle A $ y $\displaystyle \sum^{N}_{i = 1}v\left(J_{i}\right) < \epsilon  $. Como $\displaystyle B \subset A $ se tiene que también recubren a $\displaystyle B $, por lo que $\displaystyle B $ también tiene contenido nulo. La demostración para el caso de medida nula es análoga.
\end{proof}
\begin{prop}
Sea $\displaystyle A \subset \R^{n} $ medible Jordan. Entonces, $\displaystyle A $ tiene contenido nulo si y solo si tiene medida nula.
\end{prop}
\begin{proof}
La primera implicación es cierta para cualquier subconjunto de $\displaystyle \R^{n} $, por lo que sólo demostraremos la segunda implicación. Supongamos que $\displaystyle A $ tiene medida nula
\end{proof}

\begin{eg}
Sea $\displaystyle R \subset \R^{n} $ un rectángulo (no trivial, es decir, no tiene volumen nulo). Entonces, $\displaystyle v\left(R\right) > 0 $ por lo que $\displaystyle R $ no tiene contenido nulo, que se puede deducir a partir de la definición. Además, como $\displaystyle R $ es medible Jordan y no tiene contenido nulo tenemos que no tiene medida nula.
\end{eg}

\begin{colorary}
Sea $\displaystyle A \subset \R^{n} $ tal que $\displaystyle A \supset R $ con $\displaystyle R  $ rectángulo no trivial. Entonces $\displaystyle A $ no tiene contenido ni medida nula.
\end{colorary}
\begin{proof}
	Como contenido nulo implica medida nula, basta con demostrar que no puede tener medida nula. Supongamos que $\displaystyle A $ tiene medida nula, entonces para $\displaystyle \epsilon > 0 $ existe $\displaystyle \left\{ J_{k}\right\} _{k \in \N} $ tal que esta familia recubre a $\displaystyle A $ y $\displaystyle \sum^{\infty}_{k = 1}v\left(J_{k}\right)< \epsilon  $. Como $\displaystyle R \subset A $, tendremos que esta familia también recubre a $\displaystyle R $ y por tanto $\displaystyle R $ también tiene medida nula, lo cual es una contradicción como se ha visto en el ejemplo anterior. \footnote{La demostración no es necesaria realmente puesto que este enunciado es una consecuencia directa de la proposición anterior.}  
\end{proof}

\begin{theorem}
Sea $\displaystyle K \subset \R^{n} $ compacto. Entonces $\displaystyle K $ tiene contenido nulo si y solo si $\displaystyle K $ tiene medida nula.
\end{theorem}
\begin{proof}
	La primera implicación es cierta para cualquier subconjunto de $\displaystyle \R^{n} $, por lo que demostraremos únicamente la segunda implicación. Supongamos que $\displaystyle K \subset \R^{n} $ tiene medida nula, entonces existen $\displaystyle \left\{ R_{j}\right\} _{j \in \N} $ rectángulos tales que 
	\[K \subset \bigcup_{j \in \N}\Int\left(R_{j}\right) \quad \text{y} \quad \sum^{\infty}_{j= 1}v\left(R_{j}\right)< \epsilon  .\]
	Como se trata de un recubrimiento abierto de $\displaystyle K $, por ser $\displaystyle K $ compacto debe ser que existe un subrecubrimiento finito, es decir, exsiten $\displaystyle \left\{ R_{j_{1}}, \ldots, R_{j_{N}}\right\}  $ tal que $\displaystyle K \subset \bigcup_{i = 1}^{N}R_{j_{i}} $ y además
	\[\sum^{N}_{i=1}v\left(R_{j_{i}}\right)\leq \sum^{\infty}_{j = 1}v\left(R_{j}\right) < \epsilon  .\]
	Por tanto, $\displaystyle K $ tiene contenido nulo.
\end{proof}

\section{Teorema de Lebesgue}
\begin{theorem}[Teorema de Lebesgue]
Sea $\displaystyle f : R\subset \R^{n} \to \R $ acotada y $\displaystyle R $ un rectángulo. Entonces $\displaystyle f $ es integrable si y solo si el conjunto de discontinuidades de $\displaystyle f $ en $\displaystyle R $, $\displaystyle D\left(f\right) $, tiene medida nula.
\end{theorem}
Ahora estamos preparados para demostrar la proposición siguiente (que ya habíamos mencionado antes):

\begin{colorary}
Si $\displaystyle f : R \to \R $ es integrable, entonces $\displaystyle \left|f\right| $ también es integrable.
\end{colorary}

\begin{proof}
Si $\displaystyle f $ es continua en $\displaystyle x \in R $, entonces $\displaystyle \left|f\right| $ también es continua en $\displaystyle x $. De esta manera, tenemos que $\displaystyle D\left( \left|f\right|\right)\subset D\left(f\right) $. Así, como $\displaystyle D\left(f\right) $ tiene medida nula, por una proposición anterior tenemos que $\displaystyle D\left( \left|f\right|\right) $ también tiene medida nula, por lo que $\displaystyle \left|f\right| $ también es integrable en $\displaystyle R $.
\end{proof}

\begin{colorary}
Sean $\displaystyle f,g : R \to \R $ integrables. Entonces, $\displaystyle f + g $ es integrable en $\displaystyle R $.
\end{colorary}

\begin{proof}
Es sencillo ver que $\displaystyle D\left(f + g\right)\subset D\left(f\right) \cup D\left(g\right) $. Como $\displaystyle D\left(f\right) $ y $\displaystyle D\left(g\right) $ tienen medida nula su unión también, por lo que $\displaystyle D\left(f + g\right) $ tiene medida nula y $\displaystyle f + g $ es integrable.
\end{proof}
\begin{colorary}
Sean $\displaystyle f, g : R \to \R $ integrables en $\displaystyle R $. Entonces, $\displaystyle f \cdot g $ es integrable en $\displaystyle R $.
\end{colorary}
\begin{prop}
Un conjunto no vacío y acotado $\displaystyle A \subset \R^{n} $ es medible Jordan si y solo si $\displaystyle \partial A $ tiene medida nula \footnote{Recordamos que $\displaystyle \partial A = \left\{ x \in R \; : \; B\left(x,\epsilon \right)\cap A \neq \emptyset \; \text{y} \; B\left(x,\epsilon \right) \cap \left(R/A\right) \neq \emptyset, \; \forall \epsilon > 0\right\}  $. }.
\end{prop}
\begin{proof}
$\displaystyle A $ es medible Jordan si y solo si existe $\displaystyle R \subset \R^{n} $ rectángulo con $\displaystyle A \subset R $ tal que $\displaystyle \chi_{A} $ es integrable en $\displaystyle R $. Esto es equivalente a que $\displaystyle D\left(\chi_{A}\right) $ tenga medida nula. Veamos que $\displaystyle D\left(\chi_{A}\right)=\partial A $.
\begin{description}
\item[($\displaystyle \Rightarrow$)] Si $\displaystyle x \in \partial A $, entonces se cumple que $\displaystyle \forall \epsilon > 0 $, $\displaystyle B\left(x,\epsilon \right)\cap A \neq \emptyset $ y $\displaystyle B\left(x,\epsilon \right)\cap \left(R/A\right) \neq \emptyset $. 
Así, existen sucesiones $\displaystyle \left\{ y_{j}\right\} _{j \in \N} \subset A $ y $\left\{ z_{j}\right\} _{j \in \N}\subset R/A $ tales que $\displaystyle y_{j},z_{j}\to x $. Así, tenemos que $\displaystyle \chi_{A}\left(y_{j}\right) \to 1 $ y $\displaystyle \chi_{A}\left(z_{j}\right)\to 0 $, por lo que $\displaystyle \chi_{A} $ es discontinua en $\displaystyle x $.
\item[($\displaystyle \Leftarrow $)] Si $\displaystyle x \not\in \partial A $, existe $\displaystyle \epsilon_{0} > 0 $ para el cual podemos distinguir dos casos:
	\begin{itemize}
	\item $\displaystyle B\left(x, \epsilon_{0}\right) \cap A = \emptyset $, por lo que debe ser que $\displaystyle B\left(x,\epsilon_{0}\right) \subset R/A $. Tenemos que $\displaystyle \chi_{A}|_{B\left(x, \epsilon_{0}\right)}\equiv 0 $, por lo que es continua en $\displaystyle x $ y $\displaystyle x \not\in D\left(\chi_{A}\right) $.
	\item $\displaystyle B\left(x, \epsilon _{0}\right)\cap \left(R/A\right)=\emptyset $, por lo que debe ser que $\displaystyle B\left(x, \epsilon_{0}\right)\subset A $. Tenemos que $\displaystyle \chi_{A}|_{B\left(x, \epsilon _{0}\right)} \equiv 1 $, por lo que es continua en $\displaystyle x $ y $\displaystyle x \not\in D\left(\chi_{A}\right) $.
	\end{itemize}
\end{description}
\end{proof}

\begin{observation}
	Recordamos que $\displaystyle A \subset \R^{n} $ acotado es medible Jordan si y solo si existe $\displaystyle R \subset \R^{n} $ rectángulo tal que $\displaystyle \chi_{A} $ es integrable en $\displaystyle R $. Entonces, por el criterio de Lebesgue, $\displaystyle A $ es medible Jordan si y solo si $\displaystyle D\left(\chi_{A}\right) $ tiene medida nula. Tenemos que $\displaystyle D\left(\chi_{A}\right) = \partial A $, por lo que buscamos que $\displaystyle \partial A $ tenga medida nula .
\end{observation}
\begin{eg}
	Consideremos $\displaystyle A = \Q \cap \left[0,1\right]  $. Tenemos que $\displaystyle \partial A = \left[0,1\right]  $ y como $\displaystyle v\left(\partial A\right) = 1 \neq 0$, $\displaystyle \partial A $ no tiene medida nula por lo que $\displaystyle A $ no es medible Jordan.
\end{eg}
\begin{observation}
Supongamos que $\displaystyle A $ y $\displaystyle B $ son medibles Jordan. Tenemos que 
\[\chi_{A \cap B} = \chi_{A} \cdot \chi_{B} .\]
Como $\displaystyle \chi_{A} $ y $\displaystyle \chi_{B} $ son integrables Riemann, tenemos que $\displaystyle \chi_{A \cap B} $ es integrable Riemann.
\end{observation}
\section{11/2/2026}
\begin{prop}
Sea $\displaystyle A $ de contenido nulo y $\displaystyle f : A \to \R $ acotada. Entonces, $\displaystyle f $ es integrable en $\displaystyle A $ y $\displaystyle \int _{A}f = 0 $. 
\end{prop}
\begin{proof}
	Sea $\displaystyle A \subset \R^{n} $ de contenido nulo y $\displaystyle f : A \to \R $ acotada. Cogemos $\displaystyle R \subset \R^{n} $ rectángulo tal que $\displaystyle A \subset R $ y consideramos la función
	\[\tilde{f} : R \to \R : x \to 
	\begin{cases}
	f\left(x\right) , \; x \in A \\
	0, \; x \not\in A
	\end{cases}
	.\]
	Tendremos que $\displaystyle D\left(\tilde{f}\right) \subset \overline{A} $. En efecto, si $\displaystyle x \not\in \overline{A} $ tendremos que existe $\displaystyle \epsilon > 0 $ tal que $\displaystyle B\left(x,\epsilon \right)\subset R/A $, por lo que $\displaystyle \tilde{f}|_{B\left(x,\epsilon \right)} \equiv 0 $ y por tanto es continua en $\displaystyle x $. Como $\displaystyle A $ tiene contenido nulo, $\displaystyle \overline{A} $ también tiene contenido nulo (Hoja 3, Ejercicio 2). Por tanto, $\displaystyle \overline{A} $ tiene medida nula. Así, $\displaystyle \tilde{f} $ es integrable en $\displaystyle R $ y en consecuencia $\displaystyle f $ es integrable en $\displaystyle A $. 
	Veamos ahora que el valor de la integral es nulo. Como $\displaystyle \tilde{f} $ es integrable, basta con ver que la integral inferior de $\displaystyle \left|\tilde{f}\right| $ vale 0, es decir,
	\[\underline{\int _{R}} \left|\tilde{f}\right| = \sup \left\{ s\left( \left|\tilde{f}\right|, P\right) \; : \; P \in \mathcal{P}\left(R\right)\right\}  = 0.\]
Sea $\displaystyle P \in \mathcal{P}\left(R\right) $, entonces
\[s\left( \left|\tilde{f}\right|, P\right) = \sum_{J \in P} \alpha_{J}v\left(J\right) .\]
Como $\displaystyle A $ tiene volumen nulo y $\displaystyle J $ no, no puede ser que $\displaystyle J \subset A $. Así, tenemos que existe $\displaystyle x \in J $ con $\displaystyle x \not\in A $, por lo que $\displaystyle \alpha_{J} = 0 $, $\displaystyle \forall J \in P $. Así, tenemos que $\displaystyle s\left( \left|\tilde{f}\right|, P\right) = 0 $, por lo que la integral inferior vale 0. Así, 
\[ \left|\int _{R} \tilde{f} \right| \leq \int _{R} \left|\tilde{f}\right| = 0 \Rightarrow \int _{R}\tilde{f} = \int _{A}f = 0.\]
Otra forma de verlo es ver que 
\[ \left|\int _{A}f \right|\leq \int _{A} \left|f\right| \leq \int _{A} M  = Mv\left(A\right)= 0 ,\]
donde $\displaystyle \left|f\right| \leq M $, $\displaystyle \forall x \in A $. 
\end{proof}
\begin{observation}
	Nos podemos preguntar, es válido el resultado si en vez de tener $\displaystyle A $ contenido nulo tenemos que $\displaystyle A $ tiene medida nula? Claramente, la respuesta es que no. En efecto, tenemos que $\displaystyle \chi_{\left[0,1\right] \cap \Q} $ tiene medida nula pero no es integrable. Sin embargo, sí es cierto que si $\displaystyle A $ tiene medida nula y $\displaystyle f : A \to \R $ es integrable, entonces $\displaystyle \int _{A}f =0 $. 
\end{observation}
\begin{colorary}
	Sean $\displaystyle f,g : R \to \R $ acotadas con $\displaystyle f $ integrable en $\displaystyle A \subset R $ y $\displaystyle \left\{ x \in R \; : \; f\left(x\right) \neq g\left(x\right)\right\}  $ tiene contenido nulo. Entonces, $\displaystyle g $ es integrable en $\displaystyle A $ y 
	\[\int _{A}g = \int _{A}f  .\]
\end{colorary}
\begin{proof}
Tenemos que $\displaystyle g\left(x\right) = g\left(x\right)-f\left(x\right) + f\left(x\right) $. Por la proposición anterior tenemos que $\displaystyle g\left(x\right)-f\left(x\right) $ es integrable y tiene integral nula, por ser suma de funciones integrables tenemos que $\displaystyle g $ es integrable y
\[\int _{A}g = \int _{A}g - f + \int _{A}f = \int _{A}f  .\]
\end{proof}
\begin{observation}
	Al igual que antes, el enunciado no es cierto sis sustituimos 'contenido nulo' con 'medida nula'. En efecto, basta con considerar $\displaystyle f \equiv 0 $ y $\displaystyle g = \chi_{\Q \cap \left[0,1\right] } $. 
\end{observation}
\begin{observation}
Podemos repasar nuestro resultado sobre la aditividad de la integral en conjuntos integrables. Si $\displaystyle A = A_{1} \cup A_{2} $ con $\displaystyle A, A_{1} $ y $\displaystyle A_{2} $ medibles Jordan, si además $\displaystyle A_{1} \cap A_{2} $ tiene contenido nulo entonces 
\[\int _{A}f = \int _{A_{1}}f + \int _{A_{2}}f  .\]
\end{observation}

\begin{prop}
	Sea $\displaystyle f : R \to \R $ con $\displaystyle f \geq 0 $, y $\displaystyle \int _{R}f = 0 $. Entonces el conjunto $\displaystyle \left\{ x \in R \; : \; f\left(x\right) \neq 0\right\}  $ tiene medida nula. Es decir, existe $\displaystyle A \subset R $ de medida nula tal que $\displaystyle f\left(x\right) = 0 $, $\displaystyle \forall x \in R/A $. 
\end{prop}
\begin{proof}
	Queremos ver que el conjunto $\displaystyle X =\left\{ x \in \R \; : \; f\left(x\right) \neq 0\right\}  $ tiene medida nula. Tenemos que 
	\[ X = \bigcup_{k = 1}^{\infty} A_{k}  ,\]
con $\displaystyle A_{k} = \left\{ x \in R \; : \; f\left(x\right) > \frac{1}{k}\right\} $. Basta probar que $\displaystyle A_{k} $ tienen medida nula y basta probar que tienen contenido nulo. Supongamos que $\displaystyle A_{k} $ es medible Jordan, entonces
	\[0 = \int _{R}f \geq \int _{A_{k}}f \geq \int _{A_{k}}\frac{1}{k} = \frac{1}{k}v\left(A_{k}\right) \geq 0 .\]
	Así, debe ser que $\displaystyle \frac{1}{k}v\left(A_{k}\right) = 0 $, por lo que $\displaystyle v\left(A_{k}\right) = 0 $. Sea $\displaystyle k_{0} \in \N $ y sea $\displaystyle \epsilon > 0 $, existe $\displaystyle P \in \mathcal{P}\left(R\right) $ tal que $\displaystyle S\left(f,P\right)< \frac{\epsilon }{k_{0}} $. Así, tenemos que 
	\[\frac{\epsilon }{k_{0}} > \sum_{J \in P}\beta_{J}v\left(J\right) \geq \sum_{J \cap A_{k_{0}}\neq \emptyset}\beta_{J}v\left(J\right) \geq \sum_{J \cap A_{k_{0}} \neq \emptyset} v\left(J\right)\frac{1}{k_{0}}.\]
Así, tenemos que $\displaystyle \sum^{}_{J \cap A_{k_{0}}\neq \emptyset}v\left(J\right) < \epsilon  $, por lo que $\displaystyle A_{k_{0}} $ tiene contenido nulo. 	
\end{proof}

\begin{notation}
A partir de ahora, si decimos que algo se cumple en casi todo punto es porque se cumple en todos los puntos salvo en un conjunto de medida nula.
\end{notation}
\begin{observation}
	No podemos concluir que $\displaystyle \left\{ x \in R \; : \; f\left(x\right) \neq 0\right\}  $ tiene contenido nulo. En efecto, basta tomar la función de Thomae,
	\[x \to 
	\begin{cases}
	0, \; x \in \Q \\
	\frac{1}{q}, \; x=\frac{p}{q} \in \Q \; \left(\text{la fracción $\displaystyle \frac{p}{q} $ es irreducible}\right)
	\end{cases}
	.\]
	que es continua en los irracionales y discontinua en los racionales. Así, $\displaystyle D\left(f\right) = \Q \cap \left[0,1\right]  $ que tiene medida nula, por lo que $\displaystyle f $ es integrable.  
\end{observation}

