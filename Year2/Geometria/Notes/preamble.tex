% Basic commands
\usepackage[margin=1.5in]{geometry}
\usepackage[spanish]{babel}
\usepackage{amsmath, amsfonts, mathtools, amsthm, amssymb}
\usepackage[usenames,dvipsnames]{xcolor}
\usepackage{float}
\usepackage{array}

% Tikz
\usepackage{tikz}
\usepackage{tikz-cd}
\usetikzlibrary{intersections, angles, quotes, calc, positioning}
\usetikzlibrary{arrows.meta}
\usepackage{pgfplots}
\pgfplotsset{compat=1.13}

% For math environments
\usepackage{thmtools}
\usepackage[framemethod=TikZ]{mdframed}
\mdfsetup{skipabove=1em,skipbelow=0em}

\theoremstyle{definition}

\declaretheoremstyle[
headfont=\bfseries\sffamily\color{NavyBlue!70!black}, bodyfont=\normalfont,
    mdframed={nobreak,
        linewidth=2pt,
        rightline=false, topline=false, bottomline=false,
        linecolor=NavyBlue, backgroundcolor=Gray!5,
    }
]{thmgreenbox}

%\declaretheoremstyle[headfont=\bfseries\sffamily\color{OliveGreen!70!black}, bodyfont=\normalfont, mdframed={linewidth=2pt, rightline=false, topline=false, bottomline=false, linecolor=OliveGreen}]{thmblueline}

\declaretheoremstyle[
headfont=\bfseries\sffamily\color{YellowOrange!70!black}, bodyfont=\normalfont,
    mdframed={nobreak,
        linewidth=2pt,
        rightline=false, topline=false, bottomline=false,
        linecolor=YellowOrange, backgroundcolor=Gray!5,
    }
]{thmredbox}

%\declaretheoremstyle[headfont=\bfseries\sffamily\color{BrickRed!70!black}, bodyfont=\normalfont, mdframed={linewidth=2pt, rightline=false, topline=false, bottomline=false, linecolor=BrickRed}]{thmbluebox}

\declaretheoremstyle[headfont=\bfseries\sffamily\color{ForestGreen!70!black}, bodyfont=\normalfont, mdframed={linewidth=2pt, rightline=false, topline=false, bottomline=false, linecolor=ForestGreen, backgroundcolor=Gray!5,}]{thmblueline}

%\declaretheoremstyle[headfont=\bfseries\sffamily, bodyfont=\normalfont]{thmblueline}

\declaretheoremstyle[headfont=\bfseries\sffamily\color{BrickRed}, bodyfont=\normalfont]{thmbluebox}

%\declaretheoremstyle[
%headfont=\bfseries\sffamily\color{Purple!70!black}, bodyfont=\normalfont,
   % mdframed={nobreak,
        %linewidth=2pt,
        %rightline=false, topline=false, bottomline=false,
        %linecolor=Purple, 
        %backgroundcolor=Gray!5,
    %}
%]{thmpurplebox}

\declaretheorem[style=thmgreenbox, name=Definición, numberwithin=chapter]{definition}
\declaretheorem[style=thmbluebox, numbered=no, name=Observación]{observation}
\declaretheorem[style=thmredbox, name=Proposición, numberwithin=chapter]{prop}
\declaretheorem[style=thmredbox, name=Teorema, numberwithin=chapter]{theorem}
\declaretheorem[style=thmredbox, name=Lema, numberwithin=chapter]{lema}
\declaretheorem[style=thmredbox, numbered=no, name=Corolario, numberwithin=chapter]{corollary}
%\declaretheorem[style=thmpurplebox, name=Notación, numbered=no]{notation}

%\declaretheorem[style=thmproofbox, name=Demostración]{replacementproof}
%\renewenvironment{proof}[1][\proofname]{\vspace{-10pt}\begin{replacementproof}}{\end{replacementproof}}

\declaretheorem[style=thmblueline, numbered=no, name=Ejemplo]{eg}

% Headers and footers
\usepackage{fancyhdr}
\pagestyle{fancy}

\fancyhead[L]{Victoria Eugenia Torroja}
\fancyhead[R]{\rightmark}
\fancyfoot[C]{\leftmark}
\fancyfoot[R]{\thepage}

% Set chapter counter to 0
\setcounter{chapter}{-1}

% New commands
\newcommand{\R}{\mathbb{R}}
\newcommand{\C}{\mathbb{C}}
\newcommand{\F}{\mathbb{F}}
\newcommand{\N}{\mathbb{N}}
\newcommand{\Q}{\mathbb{Q}}
\newcommand{\Z}{\mathbb{Z}}
\newcommand{\K}{\mathbb{K}}
\newcommand{\mcd}{\text{mcd}}
\newcommand{\mcm}{\text{mcm}}
\DeclareMathOperator{\Ker}{Ker}
\DeclareMathOperator{\Imagen}{Im}
\DeclareMathOperator{\Hom}{Hom}
\DeclareMathOperator{\Aut}{Aut}
\DeclareMathOperator{\ran}{ran}
\DeclareMathOperator{\GL}{GL}
\DeclareMathOperator{\SL}{SL}
\DeclareMathOperator{\SO}{SO}
\DeclareMathOperator{\ord}{ord}
\DeclareMathOperator{\ind}{ind}
\DeclareMathOperator{\sig}{sig}
\DeclareMathOperator{\dom}{dom}
\DeclareMathOperator{\End}{End}
\DeclareMathOperator{\Adj}{Adj}
\DeclareMathOperator{\grad}{grad}
\DeclareMathOperator{\traz}{traz}
\DeclareMathOperator{\arctanh}{arctanh}
\DeclareMathOperator{\arcsinh}{arcsinh}
\DeclareMathOperator{\arccosh}{arccosh}
\DeclareMathOperator{\rad}{rad}
\DeclareMathOperator{\ad}{ad}
\DeclareMathOperator{\Bil}{Bil}
\DeclareMathOperator{\sech}{sech}
\let\epsilon\varepsilon
