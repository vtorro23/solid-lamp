\chapter{Grupos de permutaciones}
Sea $\displaystyle f : G \to G' $ una biyección. Consideramos la aplicación $\displaystyle \Biy\left(G\right) \to \Biy\left(G'\right) : \sigma \to f^{-1}\sigma f $, que es un isomorfismo de grupos.  
Vamos a considerar un conjunto finito de elementos al que llamaremos $\displaystyle X_{n} = \left\{ 1, 2, \ldots, n\right\}  $ y $\displaystyle \Biy\left(X_{n}\right) $, para $\displaystyle n \geq 1 $.
\begin{definition}[Grupo de permutaciones]
	El \textbf{grupo de permutaciones} de $\displaystyle n $ elementos, o el \textbf{$\displaystyle n $-ésimo grupo de permutaciones}, es el grupo $\displaystyle \mathcal{S}_{n} = \Biy\left(X_{n}\right) $ con la composición de funciones, donde $\displaystyle \tau \cdot \sigma = \sigma \circ \tau $. 
\end{definition}
\begin{observation}
El orden de $\displaystyle \mathcal{S}_{n} $ es $\displaystyle n! $. 
\end{observation}
\begin{notation}
Dado $\displaystyle \mathcal{S}_{n} $ grupo de permutaciones, si $\displaystyle \sigma \in \mathcal{S}_{n} $ entonces podemos expresar $\displaystyle \sigma  $ de la forma
\[\sigma = \begin{pmatrix} 1 & 2 & \cdots & n \\ \sigma\left(1\right) & \sigma\left(2\right) & \cdots & \sigma\left(n\right) \end{pmatrix} .\]
\end{notation}
\begin{eg}
Dado $\displaystyle \sigma \in \mathcal{S}_{4} $, 
\[\sigma = \begin{pmatrix} 1 & 2 & 3 & 4 \\ 1 & 3 & 2 & 4 \end{pmatrix}= \left(3,2\right) .\]
Similarmente, dado $\displaystyle \sigma \in \mathcal{S}_{6} $ 
\[\sigma = \begin{pmatrix} 1 & 2 & 3 & 4 & 5 & 6 \\ 2 & 3 & 4 & 1 & 5& 6 \end{pmatrix} = \left(1,2,3,4\right) .\]
Esta última notación es la que utilizaremos con más frecuencia.
\end{eg}
\begin{eg}
Consideremos $\displaystyle \sigma, \tau \in \mathcal{S}_{4} $ tales que $\displaystyle \sigma = \left(1,2,3\right) $ y $\displaystyle \tau = \left(3,4\right)\left(1,2\right) $. Tenemos que 
\[\sigma \cdot \tau = \tau \circ \sigma = \left(3,4\right)\left(1,2\right)\left(1,2,3\right) = \left(2,4,3\right) .\]
\[\tau \cdot \sigma = \sigma \circ \tau = \left(1,2,3\right)\left(3,4\right)\left(1,2\right) = \left(1,3,4\right) .\]
\end{eg}
\begin{eg} Calculemos algunos grupos de permutación.
	\begin{itemize}
	\item Tenemos que $\displaystyle \mathcal{S}_{2} = \left\{ id, \left(1,2\right)\right\}  $.
	\item Tenemos que $\displaystyle \mathcal{S}_{3}= \left\{ id, \left(1,2\right), \left(1,3\right), \left(2,3\right), \left(1,2,3\right), \left(1,3,2\right)\right\}  $. Podemos ver que $\displaystyle \mathcal{S}_{3} \cong D_{3} $.
	\end{itemize}	 
\end{eg}
\begin{theorem}[Teorema de Cayley]
Todo grupo finito es isomorfo a un subgrupo de un grupo de permutaciones.
\end{theorem}
\begin{proof}
	Sea $\displaystyle G $ un grupo finito y $\displaystyle g \in G $. Consideremos la aplicación $\displaystyle \tilde{g} : G \to G : x \to x \cdot g $. Es fácil ver que $\displaystyle \tilde{g} \in \Biy\left(G\right) $. Ahora, consideremos $\displaystyle \phi : G \to \Biy\left(G\right) : g \to \tilde{g} $. Veamos que $\displaystyle \phi $ es un homomorfismo de grupos:
	\[\phi\left(gh\right)\left(x\right) = \widetilde{gh}\left(x\right) = x \cdot \left(gh\right) = \tilde{g}\left(x\right)h = \tilde{h}\left(\tilde{g}\left(x\right)\right) = \tilde{g} \cdot \tilde{h}\left(x\right) .\]
	Ahora, veamos que es inyectiva. Si $\displaystyle g \in \Ker\left(\phi\right) $, tenemos que $\displaystyle \tilde{g} = id $, es decir, $\displaystyle \forall x \in G $, 
	\[g\left(x\right) = x \cdot g = e .\]
	Así, tenemos que $\displaystyle \Ker\left(\phi\right) = \left\{ e\right\}  $, por lo que $\displaystyle \phi $ es inyectiva. Así, tenemos que $\displaystyle G \cong \Imagen\left(\phi\right) \leq \Biy\left(G\right) = \mathcal{S}_{ \left|G\right|}$. 	
\end{proof}
\begin{definition}[Soporte]
	Sea $\displaystyle \sigma \in \mathcal{S}_{n} $. Llamamos \textbf{soporte} de $\displaystyle \sigma  $ al conjunto $\displaystyle \sop\left(\sigma \right) = \left\{ a \in X_{n} \; : \; \sigma\left(a\right) \neq a\right\}  $. Diremos que $\displaystyle \sigma, \tau \in \mathcal{S}_{n} $ son \textbf{disjuntos} si $\displaystyle sop\left(\sigma \right) \cap \sop\left(\tau\right) = \emptyset $. 
\end{definition}
\begin{eg}
Consideremos $\displaystyle \sigma, \tau \in \mathcal{S}_{6} $ tales que
\[\sigma = \begin{pmatrix} 1 & 2 & 3 & 4 & 5 & 6 \\ 1 & 3 & 4 & 5 & 2 & 6 \end{pmatrix} = \left(2,3,4,5\right), \quad \tau = \left(1,6\right) .\]
Tenemos que $\displaystyle \sop\left(\sigma \right)= \left\{ 2,3,4,5\right\}  $ y $\displaystyle \sop\left(\tau\right) = \left\{ 1,6\right\}  $, por lo que $\displaystyle \tau $ y $\displaystyle \sigma  $ son disjuntos. Podemos ver que la notación de los ciclos nos facilita mucho el cálculo del soporte.
\end{eg}
\begin{observation}
\begin{enumerate}
\item $\displaystyle \sop\left(\sigma \right)= \emptyset $ si y solo si $\displaystyle \sigma = id $. 
\item $\displaystyle \sop\left(\sigma \right)= \sop\left(\sigma ^{-1}\right) $. En efecto, si $\displaystyle a \in \sop\left(\sigma \right) $, tenemos que $\displaystyle a \neq \sigma \left(a\right) $, por lo que $\displaystyle \sigma^{-1}\left(a\right) \neq a $ y $\displaystyle a \in \sop\left(\sigma^{-1}\right) $. El recíproco es análogo.
\item $\displaystyle m \geq 2 $, $\displaystyle \sop\left(\sigma^{m}\right) \subset \sop\left(\sigma \right) $. En efecto, si $\displaystyle a \not\in \sop\left(\sigma \right) $ tenemos que $\displaystyle a = \sigma\left(a\right) $, por lo que $\displaystyle a = \sigma^{m}\left(a\right) $ y $\displaystyle a \not\in \sop\left(\sigma^{m}\right) $. 
\end{enumerate}
\end{observation}
\begin{lema}
Sean $\displaystyle \sigma, \tau \in \mathcal{S}_{n} $ dos permutaciones disjuntas.
\begin{enumerate}
\item $\displaystyle \sigma \cdot \tau = \tau \cdot \sigma  $.
\item $\displaystyle \forall m \in \N $, se tiene que $\displaystyle \left(\sigma \cdot \tau\right)^{m} = id $ si y solo si $\displaystyle \sigma^{m} = \tau^{m} = id $.
\end{enumerate}
\end{lema}
\begin{proof} Supongamos que $\displaystyle \sop\left(\sigma \right) \cap \sop\left(\tau \right) = \emptyset $.
\begin{enumerate}
\item Si $\displaystyle x \not\in \sop\left(\sigma \right)\cup \sop\left(\tau \right) $ tenemos que $\displaystyle \sigma\left(x\right) = x $ y $\displaystyle \tau\left(x\right) = x $, por lo que
	\[\sigma\left(\tau x\right) = \sigma \left(x\right) = \tau\left(x\right) = \tau\left(\sigma\left(x\right)\right) .\]
	Ahora, supongamos sin pérdida de generalidad que $\displaystyle x \in \sop\left(\sigma \right) $. Como $\displaystyle \sigma  $ y $\displaystyle \tau $ son disjuntos, debe ser que $\displaystyle x \not\in \sop\left(\tau\right) $, es decir, $\displaystyle \tau\left(x\right) = x $. Por otro lado, tenemos que $\displaystyle \sigma\left(x\right) \in \sop\left(\sigma \right) $ y en consecuencia $\displaystyle \sigma\left(x\right) \not\in \sop\left(\tau\right) $. Así, podemos concluir que
	\[ \sigma\left(\tau\left(x\right)\right) = \sigma \left(x\right) = \tau\left(\sigma \left(x\right)\right) .\]
\item La segunda implicación es trivial. Supongamos que $\displaystyle \left(\sigma \cdot \tau\right)^{m} = id $, es decir, $\displaystyle \sigma^{m} = \left(\tau^{m}\right)^{-1} $. Así, nos queda que
	\[\sop\left(\sigma \right) \supset \sop\left(\sigma ^{m}\right) = \sop\left(\tau^{m}\right)\subset \sop\left(\tau\right) .\]
Así, por ser $\displaystyle \sigma  $ y $\displaystyle \tau $ disjuntos tenemos que $\displaystyle \sop\left(\sigma ^{m}\right) = \sop\left(\tau^{m}\right)= \emptyset $, por lo que $\displaystyle \sigma^{m} = \tau^{m} = id $.	
\end{enumerate}
\end{proof}
\begin{observation}
	Tenemos que $\displaystyle \mathcal{S}_{2} \cong C_{2} $. Para $\displaystyle n \geq 3 $, tenemos que $\displaystyle Z\left(\mathcal{S}_{n}\right) = \left\{ id\right\}  $. 
\end{observation}
\section{Ciclos}
\begin{definition}[Ciclo]
Sea $\displaystyle \sigma \in \mathcal{S}_{n} $. Diremos que $\displaystyle \sigma  $ es un \textbf{ $\displaystyle k $-ciclo} o \textbf{ciclo de orden $\displaystyle k $} si dados $\displaystyle i_{1}, \ldots, i_{k} \in X_{n} $, tenemos que $\displaystyle \sigma\left(i_{j}\right)=i_{j+1} $ (con $\displaystyle \sigma\left(i_{k}\right)=i_{1} $) y para el resto $\displaystyle i_{k+1}, \ldots, i_{n} \in X_{n} $ se tiene que $\displaystyle \sigma\left(i_{t}\right) = i_{t} $. Lo escribimos $\displaystyle \left(i_{1}, \ldots, i_{k}\right) $.
\end{definition}
\begin{eg}
\begin{enumerate}
\item En $\displaystyle \mathcal{S}_{4} $ podemos considerar el 3-ciclo $\displaystyle \left(1,2,3\right) $ y el 4-ciclo $\displaystyle \left(1,4,2,3\right) $. 
\item En $\displaystyle \mathcal{S}_{3} $ podemos considerar $\displaystyle \sigma = \left(1,3,2\right) $. Tenemos que $\displaystyle \sigma^{-1} = \left(2,3,1\right) $. En efecto, tenemos que
	\[\sigma \circ \sigma^{-1} = \left(1,3,2\right)\left(2,3,1\right) = \left(1\right)\left(2\right)\left(3\right) .\]
\item Considerando nuevamente en $\displaystyle \mathcal{S}_{4} $ el ciclo $\displaystyle \left(1,2,3\right) $, tenemos que 
	\[\left(1,2,3\right) = \left(2,3,1\right) = \left(3,1,2\right) .\]
\end{enumerate}
\end{eg}
\begin{prop}
Sea $\displaystyle 2 \leq k \leq n $. 
\begin{enumerate}
\item Si $\displaystyle l \leq k $, tenemos que $\displaystyle \left(i_{1}, \ldots, i_{k}\right) = \left(i_{l}, i_{l+1}, \ldots, i_{k}, i_{1}, \ldots, i_{l-1}\right) $.
\item El inverso de $\displaystyle \left(i_{1}, i_{2}, \ldots, i_{k}\right) $ es $\displaystyle \left(i_{k}, i_{k-1}, \ldots, i_{2}, i_{1}\right) $.
\item Todo $\displaystyle k $-ciclo tiene orden $\displaystyle k $.
\item Si $\displaystyle \sigma \in \mathcal{S}_{n} $ es un $\displaystyle k $-ciclo, entonces $\displaystyle \sigma = \left(i, \sigma\left(i\right), \ldots, \sigma^{k-1}\left(i\right)\right) $, $\displaystyle \forall i \in \sop\left(\sigma \right) $. Además $\displaystyle k = \left|\sop\left(\sigma \right)\right| $. 
\end{enumerate}
\end{prop}
\begin{proof}
Consideremos $\displaystyle 2 \leq k \leq n $.
\begin{enumerate}
\item Es trivial a partir de la definición. 
\item Basta con comprobar que su composición es la identidad:
	\[ \left(i_{1}, i_{2}, \ldots, i_{k}\right)\left(i_{k}, i_{k-1}, \ldots, i_{1}\right) = \left(i_{1}\right) \cdots \left(i_{k}\right) = id .\]
Comprobar la otra composición es análogo. 
\item Si tomamos $\displaystyle \sigma = \left(i_{1}, \ldots, i_{k}\right) $ y $\displaystyle l \leq k $, tenemos que $\displaystyle \sigma^{l}\left(i_{1}\right) = i_{l+1} $. Como buscamos la identidad, necesitamos que $\displaystyle i_{l+1} = i_{1} $, que solo ocurre cuando $\displaystyle l = k $. No hay un menor elemento que lo cumpla.
\item Se deduce de \textbf{(1)} y \textbf{(3)} por como están construidos. 
\end{enumerate}
\end{proof}
\begin{prop}[Descomposición en ciclos disjuntos]
Todo $\displaystyle \sigma \in \mathcal{S}_{n} $ se puede descomponer como producto de ciclos disjuntos dos a dos tal que $\displaystyle \sigma = \sigma_{1} \cdots \sigma_{k} $.
\end{prop}
\begin{proof}
Sea $\displaystyle \sigma \in \mathcal{S}_{n} $ y vamos a considerar la siguiente relación de equivalencia:
\[x \sim y \iff \exists s \in \N, \; \sigma^{s}\left(x\right) = y \iff \exists \tau \in \left\langle \sigma  \right\rangle , \tau\left(x\right) = y .\]
Esta relación de equivalencia genera una partición de $\displaystyle X_{n} $. Consideremos $\displaystyle \left\{ j_{1}, \ldots, j_{t}\right\}  $ representantes de las clases de equivalencia con más de un elemento y llamamos $\displaystyle O_{i} $ a la clase de equivalencia de $\displaystyle j_{i} $. Para cada $\displaystyle 1 \leq i \leq t $, definimos $\displaystyle \sigma_{i} : X_{i} \to X_{i} $ tal que 
\[\sigma_{i}\left(x\right) = 
\begin{cases}
\sigma\left(x\right), \; x \in O_{i} \\ 
x, \; x \not\in O_{i}
\end{cases}
.\]
Así, tenemos que $\displaystyle \sigma_{i} = \left(j_{i}, \sigma\left(j_{i}\right), \ldots, \sigma^{s-1}\left(j_{i}\right)\right) $ es un $\displaystyle s $-ciclo donde $\displaystyle \sop\left(\sigma _{i}\right) = O_{i} $. Como $\displaystyle O_{i} \cap O_{j} = \emptyset $ si $\displaystyle i \neq j $, tenemos que $\displaystyle \sop\left(\sigma _{i}\right) \cap \sop\left(\sigma _{2}\right) = \emptyset $, $\displaystyle \forall i,j \in \left\{ 1, \ldots, t\right\}  $ con $\displaystyle i \neq j $. Así, tenemos que 
\[\sop\left(\sigma \right) = \bigsqcup^{t}_{i = 1}\sop\left(\sigma_{i}\right) \Rightarrow \sigma = \sigma _{1} \cdots \sigma _{t} .\]
\end{proof}
\begin{eg}
Tenemos que
\[\left(1,4,2,3\right) = \left(1,2,3,4\right)\left(1,3,4\right) .\]
\end{eg}
\begin{prop}
Sea $\displaystyle \sigma \in \mathcal{S}_{n} $ con $\displaystyle \sigma = \sigma_{1} \cdots \sigma _{k} $ ciclos disjuntos. Entonces, $\displaystyle o\left(\sigma \right) = \mcm\left(o\left(\sigma_{1}\right), \ldots, o\left(\sigma_{k}\right)\right) $. 
\end{prop}
\begin{proof}
Por ser $\displaystyle \sigma_{1}, \ldots, \sigma_{k} $ disjuntos tenemos que para cualquier $\displaystyle m \in \Z $,
\[\sigma^{m} = \sigma^{m}_{1} \cdots \sigma^{m}_{k} .\]
Además, hemos visto que $\displaystyle \sigma^{m} = id $ si y solo si $\displaystyle \sigma^{m}_{i} = id $, $\displaystyle \forall i = 1, \ldots, k $. Por tanto, necesitamos que $\displaystyle o\left(\sigma_{1}\right) |m $, $\displaystyle \forall i = 1, \ldots, k $, por lo que claramente debe ser que $\displaystyle \mcm\left(o\left(\sigma_{1}\right), \ldots, o\left(\sigma_{k}\right)\right) | m $, por lo que 
\[o\left(\sigma \right) = \mcm\left(o\left(\sigma_{1}\right), \ldots, o\left(\sigma_{k}\right)\right) .\]
\end{proof}
\begin{definition}[Trasposiciones]
A los 2-ciclos los llamamos \textbf{trasposiciones}.
\end{definition}

\begin{colorary}
Sea $\displaystyle \sigma \in \mathcal{S}_{n} $. Entonces, podemos escribir $\displaystyle \sigma  $ como producto de 2-ciclos.	
\end{colorary}
 \begin{proof}
Si $\displaystyle \sigma \in \mathcal{S}_{n} $, sabemos que $\displaystyle \sigma = \sigma_{1} \cdots \sigma_{k} $ ciclos disjuntos. Está claro que cualquier $\displaystyle n $-ciclo lo podemos expresar como 
\[\left(i_{1}, \ldots, i_{n}\right) = \left(i_{1}, i_{n}\right)\left(i_{n}, i_{2}\right) \cdots \left(i_{n-1}, i_{n}\right) .\]
% esto no está bien hay que arreglarlo
Así, cada $\displaystyle n $-ciclo es producto de trasposiciones y $\displaystyle \sigma  $ lo es.
 \end{proof}
 
