\chapter{Grupos finitos abelianos}
\begin{definition}[Exponente de un grupo]
Se define \textbf{exponente} de un grupo finito $\displaystyle G $, $\displaystyle \exp\left(G\right) $, como el mínimo común múltiplo de los órdenes de los elementos de $\displaystyle G $.
\end{definition}
\begin{observation}
El exponente de un grupo divide al orden del grupo. 
\end{observation}
\begin{lema}
En un grupo finito abeliano el exponente coincide con el orden del elemento de mayor orden. 
\end{lema}
\begin{proof}
Sea $\displaystyle a \in G $ de tal forma que $\displaystyle a $ tiene orden máximo, por lo que $\displaystyle o\left(a\right) \leq \exp\left(G\right) $. Supongamos que $\displaystyle o\left(a\right) < \exp\left(G\right) $, entonces existe $\displaystyle b \in G $ tal que $\displaystyle o\left(b\right) \not | o\left(a\right) $, es decir, $\displaystyle b^{o\left(a\right)} \neq e $. Así existe un primo $\displaystyle p $ y un $\displaystyle k \geq 1 $ tal que $\displaystyle p^{k} | o\left(b\right) $ pero $\displaystyle p^{k} \not | o\left(a\right) $. Escribimos 
\[o\left(a\right) = p^{i}m, \; i < k, \; \mcd\left(m,p\right) = 1 .\]
Tenemos que $\displaystyle m | o\left(a\right) $ y $\displaystyle p^{k} | o\left(b\right) $, por tanto existen $\displaystyle x \in \left\langle a \right\rangle  $ e $\displaystyle y \in \left\langle b \right\rangle  $ tales que $\displaystyle o\left(x\right) = m $ y $\displaystyle o\left(y\right) = p^{k} $. Como el grupo es abeliano $\displaystyle x,y $ conmutan y $\displaystyle \mcd\left(o\left(x\right), o\left(y\right)\right) = 1 $, podemos escribir
\[o\left(xy\right) = o\left(x\right)o\left(y\right) = m \cdot p^{k} > o\left(a\right) .\]
Esto es una contradicción puesto que $\displaystyle o\left(a\right) $ era el máximo, por lo que debe ser que $\displaystyle \exp\left(G\right) = o\left(a\right) $. 
\end{proof}
\begin{observation}
\begin{enumerate}
\item Dos grupos finitos isomorfos tienen el mismo exponente.
\item Si $\displaystyle G $ no es abeliano no se cumple en general el lema anterior. Por ejemplo, si consideramos $\displaystyle D_{3} $, tenemos que $\displaystyle  \exp\left(D_{3}\right)=6$ y todos sus elementos tienen órdenes $\displaystyle 2 $ o $\displaystyle 3 $, por lo que no se cumple el lema. 
\end{enumerate}	
\end{observation}
\begin{lema}
Sea $\displaystyle G $ un grupo finito abeliano. Sea $\displaystyle a \in G $ tal que $\displaystyle o\left(a\right) = \exp\left(G\right) $. Entonces, existe un subgrupo $\displaystyle K \leq G $ tal que $\displaystyle G \cong \left\langle a \right\rangle \times K $. 
\end{lema}
\begin{proof}
	Basta probar la existencia de un subgrupo $\displaystyle K \leq G $ tal que $\displaystyle G = \left\langle a \right\rangle \cdot \K $ y $\displaystyle \left\langle a \right\rangle \cap K = \left\{ e\right\}  $. Procedemos por inducción en $\displaystyle \left|G\right| $, siendo el caso $\displaystyle \left|G\right| = 1 $ trivial. \\
Sea $\displaystyle H = \left\langle a \right\rangle  $ y observemos que si $\displaystyle G = H $, el enunciado es trivial. Por tanto, supongamos que $\displaystyle G-H $ es no vacío, y de entre todos sus elementos escogemos un elemento $\displaystyle x \in G - H $ de orden minimal. Es obvio que $\displaystyle x \neq e $. \\
Veamos que $\displaystyle o\left(x\right) $ es primo. Para todo número primo $\displaystyle p $ que sea divisor de $\displaystyle o\left(x\right) $ tenemos que $\displaystyle o\left(x^{p}\right) = \frac{o\left(x\right)}{p} < o\left(x\right) $, por el lema anterior. En particular, por minimalidad de $\displaystyle o\left(x\right) $ deducimos que $\displaystyle x^{p} \in H $ y por tanto, como $\displaystyle o\left(x^{p}\right) | \exp\left(G\right) = o\left(a\right) = \left|H\right| $, deducimos que $\displaystyle \left\langle x^{p} \right\rangle  $ es el único subgrupo de $\displaystyle H $ de orden $\displaystyle o\left(x^{p}\right) $.
Por otro lado, como $\displaystyle o\left(x\right) | \exp\left(G\right) = o\left(a\right) $, el lema anterior también implica que $\displaystyle o\left(a\right) = o\left(a^{p}\right) \cdot p $, con lo que $\displaystyle o\left(x^{p}\right) | o\left(a^{p}\right) $, por lo que $\displaystyle \left\langle a^{p} \right\rangle \leq H $ posee un subgrupo de orden $\displaystyle o\left(x^{p}\right) $. 
\end{proof}
\begin{theorem}[Teorema de caracterización de grupos finitos abelianos]
Sea $\displaystyle G $ un grupo finito abeliano. Entonces existe $\displaystyle m_{1}, \ldots, m_{k} $ tales que $\displaystyle m_{i} $ divide a $\displaystyle m_{i-1} $ enteros con $\displaystyle k \geq 1 $ natural tal que 
\[ G \cong \Z_{m_{1}} \times Z_{m_{2}} \times \cdots \times \Z_{m_{k}} .\]
Además, $\displaystyle m_{1}, \ldots, m_{k} $ son únicos con esta propiedad. 
\end{theorem}
\begin{definition}[Coeficientes de torsión]
Los números $\displaystyle m_{1}, \ldots, m_{k} $ son los \textbf{coeficientes de torsión} de $\displaystyle G $. 
\end{definition}
\begin{observation}
\begin{enumerate}
\item Sabemos que $\displaystyle \left|G\right| = m_{1} \cdots m_{k} $.
\item Como $\displaystyle m_{i}|m_{i-1} $, tenemos que $\displaystyle \exp\left(G\right) = m_{1} $.
\end{enumerate}
\end{observation}
\begin{eg}
Sea $\displaystyle G = \Z_{2} \times \Z_{2} \times \Z_{5^{2}} \times \Z_{5^{2}} \times \Z_{5} \times \Z_{2} $. Tenemos que $\displaystyle \left|G\right| = 2^{3} \cdot 5^{5} $. Queremos expresar $\displaystyle G $ de la forma del teorema anterior. Sabemos que si tienen órdenes coprimos entre ellos, son isomorfos al grupo cíclico que es producto de esos órdenes. Así,
\[G \cong \Z_{5^{2} \cdot 2} \times \Z_{5^{2} \cdot 2} \times \Z_{5 \cdot 2} .\]
Así, tenemos que los coeficientes de torsión serán $\displaystyle \left(5^{2} \cdot 2, 5^{2} \cdot 2, 5 \cdot 2\right) $. 
\end{eg}
\begin{prop}
Sea $\displaystyle G $ un grupo abeliano finito de orden $\displaystyle n $. Sea $\displaystyle m $ un divisor de $\displaystyle n $. Entonces existe un $\displaystyle H \leq G $ con $\displaystyle \left|H\right| = m $. En particular, si $\displaystyle m $ es primo, entonces existe en $\displaystyle G $ un elemento de orden $\displaystyle m $.
\end{prop}
\begin{proof}
Como $\displaystyle G $ es un grupo abeliano finito, existen $\displaystyle m_{1}, \ldots, m_{k} \in \N $ tales que $\displaystyle G \cong \Z_{m_{1}} \times \cdots \times \Z_{m_{k}} $. Sabemos que $\displaystyle n = m_{1} \cdots m_{k} $. Como $\displaystyle m | n $, entonces existen $\displaystyle n_{1}, \ldots, n_{k} \in \N $ con $\displaystyle n_{i}|m_{i} $, $\displaystyle \forall i = 1, \ldots, k $ tal que $\displaystyle m = n_{1} \cdots n_{k} $. 
Por ser $\displaystyle \left(\Z, +\right) $ cíclico, tenemos que para cada $\displaystyle i $ existe $\displaystyle H_{i}\leq \Z_{m_{i}} $ de orden $\displaystyle n_{i} $.
Así, tenemos que existe $\displaystyle H \leq G $ con $\displaystyle H \cong H_{n_{1}} \times \cdots \times H_{n_{k}} $ donde $\displaystyle H_{n_{i}} \leq \Z_{n_{i}} $. Además obtenemos que $\displaystyle \left|H\right| = n_{1} \cdots n_{k} = m $. 
\end{proof}
\begin{eg}
Vamos a construir, dado un orden $\displaystyle n $, los distintos grupos finitos abelianos de ese orden. 
\begin{enumerate}
\item Consideremos $\displaystyle n = 24 $. Podemos considerar varios casos:
	\begin{description}
	\item[Caso 1.] Consideremos que $\displaystyle m_{1} = 24 $, tenemos que $\displaystyle G \cong \Z_{24} $. 
	\item[Caso 2.] Consideremos que $\displaystyle m_{1} = 12 $, por lo que $\displaystyle m_{2} = 2 $. Así, tenemos que $\displaystyle G \cong \Z_{12} \times \Z_{2} $. 
	\item[Caso 3.] Consideremos que $\displaystyle m_{1} = 8 $, por lo que $\displaystyle m_{2} = 3 $. Así, tendríamos que $\displaystyle m_{2} = 3 $, pero esto no puede ser porque $\displaystyle \mcd\left(8,3\right) = 1 $ y 3 no divide a 8. Por tanto, $\displaystyle G \cong \Z_{24} $.
	\item[Caso 4.] Consideremos que $\displaystyle m_{1} = 6 $. No podemos tomar $\displaystyle m_{2} =4 $ porque 4 no divide a 6. Así, nos queda que la única posibilidad es que $\displaystyle m_{2} = m_{3} = 2 $. Así, $\displaystyle G \cong \Z_{6} \times \Z_{2} \times \Z_{2} $.
	\item[Caso 5.] Si consideramos $\displaystyle m_{1}=3 $, o $\displaystyle m_{1}=2 $, volvemos a los casos anteriores. 
	\end{description}
\item Consideremos $\displaystyle n=196 = 2^{2} \cdot 7^{2} $. 
	\begin{description}
	\item[Caso 1.] Consideremos $\displaystyle m_{1} = 196 $, por lo que $\displaystyle G \cong \Z_{196} $.
	\item[Caso 2.] Consideremos $\displaystyle m_{1} = 98 $, por lo que necesariamente $\displaystyle m_{2} = 2 $ y tenemos que $\displaystyle G \cong \Z_{98} \times \Z_{2} $.
	\item[Caso 3.] Consideremos $\displaystyle m_{1} = 28 $, por lo que necesariamente $\displaystyle m_{2} = 7 $ y tenemos que $\displaystyle G \cong \Z_{28} \times \Z_{7} $.
	\item[Caso 4.] Consideremos $\displaystyle m_{1} = 14 $, por lo que necesariamente debe ser que $\displaystyle m_{2} = 14 $ y tenemos que $\displaystyle G \cong \Z_{14} \times \Z_{14} $. 
	\end{description}
\end{enumerate}
\end{eg}
\begin{observation}
Para agilizar los cálculos podemos darnos cuenta de que en el $\displaystyle m_{1} $ deben estar contenidos todos los factores primos de $\displaystyle n $.
\end{observation}
\begin{observation}
Sea $\displaystyle G $ un grupo finito y $\displaystyle \left|G\right| = p_{1}^{\alpha_{1}} \cdots p_{k}^{\alpha_{k}} $, donde $\displaystyle p_{i} $ es primo y $\displaystyle \alpha_{i} \in \N $. Si considero las distintas descomposiciones de $\displaystyle \alpha_{i} $, en el sentido de cuántas maneras tengo de expresar $\displaystyle \alpha_{i} $ como suma de naturales más el cero, es decir,
\[\alpha_{i} = j_{i_{1}} + \cdots + j_{i_{s}}, \; j_{i_{t}} \in \N \cup \left\{ 0\right\}, \; i_{t} \in \N  ,\]
entonces el número de grupos abelianos finitos de orden $\displaystyle \left|G\right| $ es el producto de las cantidad de descomposiciones de cada $\displaystyle \alpha_{i} $. 
\end{observation}
\begin{theorem}
Sea $\displaystyle G $ un grupo finito abeliano no trivial de orden $\displaystyle n = p_{1}^{\alpha_{1}} \cdots p_{s}^{\alpha_{s}} $, con $\displaystyle p_{i} $ primos y $\displaystyle \alpha_{i} \geq 1 $, $\displaystyle \forall i = 1, \ldots, s $. Para cada primo $\displaystyle p_{i} $ existe un subgrupo $\displaystyle G_{i} $ de $\displaystyle G $ tal que 
\[G \cong G_{1} \times \cdots \times G_{s} ,\]
y cada $\displaystyle G_{i} $ es isomorfo a $\displaystyle \Z_{j_{i,1}} \times \cdots \times \Z_{j_{i,r_{i}}} $, donde $\displaystyle j_{i,1} \geq \cdots \geq j_{i,r_{i}} $ y $\displaystyle j_{i,1} + \cdots + j_{i,r_{i}} = \alpha_{i} $. 
\end{theorem}
\begin{proof}
Por la proposición anterior, existe subgrupos $\displaystyle G_{i} $ de orden $\displaystyle p^{\alpha_{i}} $. Como consecuencia de la fórmula de Lagrange tenemos que 
\[G_{i} \cap \prod_{j \neq i}G_{j} = \left\{ e\right\}  ,\]
para cada $\displaystyle i $, por lo que se verifica que $\displaystyle G \cong G_{1} \times \cdots \times G_{s} $. Finalmente, por el Teorema de Caracterización tenemos que cada $\displaystyle G_{i} $ cumple la propiedad deseada. 
\end{proof}

\begin{eg}
\begin{enumerate}
\item Tomemos $\displaystyle n = 24  = 3 \cdot 2^{2}$. Entonces, $\displaystyle \alpha_{1} = 1 + 0 $, solo lo podemos expresar de esta forma; y $\displaystyle \alpha_{2} = 3 = 3 + 0 = 2 + 1=1+1+1 $, que se puede expresar de estas tres formas. Por tanto, hay $\displaystyle 1 \cdot 3 $ grupos finitos abelianos de orden 24. Nos salen los siguientes grupos:
	\[
	\begin{split}
		\Z_{3} \times \Z_{2^{3}} & \cong \Z_{24} \\
		\Z_{3} \times \Z_{2^{2}} \times \Z_{2} & \cong \Z_{12} \times \Z_{2} \\
		\Z_{3} \times \Z_{2} \times \Z_{2} \times \Z_{2} & \cong \Z_{6} \times \Z_{2} \times \Z_{2}.
	\end{split}
	\]
\item Tomemos $\displaystyle n = 196 = 2^{2} \cdot 7^{2} $. Tenemos que 
	\[
	\begin{split}
		\alpha_{1} = \alpha_{2} = & 2 = 2 + 0 = 1 + 1 .
	\end{split}
	\]
	Así, tenemos $\displaystyle 2 \cdot 2 = 4$ posibles grupos. 
\item Tomemos $\displaystyle n = 3969 = 7^{2} \cdot 3^{4} $. Calculemos el número de grupos que nos tienen que salir:
	\[
	\begin{split}
		\alpha_{1} = & 2 = 2 + 0 = 1 + 1 \\
		\alpha_{2} = & 4 = 4 + 0 = 3 + 1 = 2 + 2 = 2 + 1 + 1 = 1+ 1 + 1 + 1.
	\end{split}
	\]
	Así, hay $\displaystyle 2 \cdot 5 = 10 $ grupos abelianos finitos. Tenemos que $\displaystyle m_{1}  $ es múltiplo de $\displaystyle 7 \cdot 3 = 21 $. 
	\begin{description}
	\item[Caso 1.] Supongamos que $\displaystyle m_{1} = 21 $. Tenemos que $\displaystyle m_{2} | m_{1} $, por lo que debe ser que $\displaystyle m_{2} = 7 \cdot 3 $. Similarmente, como $\displaystyle m_{3} | m_{2} $, debe ser que $\displaystyle m_{3} = m_{4} = 3 $. Así, $\displaystyle G \cong \Z_{21} \times \Z_{21} \times \Z_{3}\times\Z_{3} $.
	\item[Caso 2.] Consideremos que $\displaystyle m_{1}=21 \cdot 3 $. Tenemos que hay dos opciones para $\displaystyle m_{2} $. La primera es considerar $\displaystyle G \cong \Z_{63} \times \Z_{63} $. La otra es coger $\displaystyle G \cong \Z_{63} \times \Z_{21} \times \Z_{3} $.
	\item[Caso 3.] Consideremos $\displaystyle m_{1} = 147 = 7^{2} \cdot 3 $. En este caso, solo tenemos la opción $\displaystyle G \cong \Z_{147} \times \Z_{3} \times \Z_{3} \times \Z_{3} $.
	\item[Caso 4.] Consideremos $\displaystyle m_{1} = 189 = 7 \cdot 3^{3} $. Entonces, necesariamente $\displaystyle G \cong \Z_{189} \times \Z_{21} $. 
	\item[Caso 5.] Consideremos $\displaystyle m_{1} = 441 = 7^{2} \cdot 3^{2} $. En este caso tenemos las opciones $\displaystyle G \cong \Z_{441} \times \Z_{3} \times \Z_{3} $ y $\displaystyle G \cong \Z_{441} \times \Z_{9} $. 
	\item[Caso 6.] Consideremos $\displaystyle m_{1} = 567 = 7 \cdot 3^{4} $, entonces tenemos que $\displaystyle G \cong \Z_{567} \times \Z_{7} $. 
	\item[Caso 7.] Consideremos $\displaystyle m_{1} = 1323 = 7^{2} \times 3^{3} $, entonces tenemos que $\displaystyle G \cong \Z_{1323} \times \Z_{3} $.
	\item[Caso 8.] Consideremos $\displaystyle m_{1} = 3969 $, por lo que $\displaystyle G \cong \Z_{3969} $.
	\end{description}
\end{enumerate}
\end{eg}

