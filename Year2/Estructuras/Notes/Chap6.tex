\part{Anillos}
\chapter{Generalidades de Anillos}
\begin{definition}[Anillo]
Sea $\displaystyle \left(A, +, \cdot \right) $, donde $\displaystyle A $ es un conjunto no vacío y $\displaystyle + : A \times A \to A $ y $\displaystyle \cdot : A \times A \to A $ son dos operaciones internas. Diremos que $\displaystyle A $ es un \textbf{anillo} si:
\begin{enumerate}
\item $\displaystyle \left(A, +\right) $ es un grupo abeliano.
\item El producto es asociativo, es decir, $\displaystyle \forall x,y,z \in A $, $\displaystyle \left(x \cdot y\right) \cdot z = x \cdot \left(y \cdot z\right) $.
\item El producto es distributivo por la derecha y por la izquierda, es decir, $\displaystyle \forall x,y,z \in A $, 
	\[ \left(x + y\right) \cdot z = x \cdot z + y \cdot z \quad \text{y} \quad  z \cdot \left(x + y\right) = z \cdot x + z \cdot y.\]
\end{enumerate}
\end{definition}
\begin{definition}
Sea $\displaystyle \left(A, + , \cdot \right) $ un anillo. 
\begin{enumerate}
\item Diremos que es \textbf{unitario} si existe $\displaystyle 1_{A} \in A $ tal que $\displaystyle a \cdot 1_{A} = 1_{A} \cdot a $, $\displaystyle \forall a \in A $. 
\item Diremos que es un \textbf{anillo conmutativo} si $\displaystyle \forall a,b \in A $, $\displaystyle a \cdot b = b \cdot a $. 
\end{enumerate}
\end{definition}
\begin{observation}
Si el anillo es unitario el elemento neutro para el producto es único. En efecto, si $\displaystyle 1_{A} $ y $\displaystyle 1_{A}' $ son elementos neutros para el producto tenemos que
\[1_{A} = 1_{A} \cdot 1_{A}' = 1_{A}' .\]
\end{observation}
\begin{eg}
\begin{enumerate}
\item Tenemos que $\displaystyle \left(\Z, +, \cdot \right) $ es un anillo unitario conmutativo. De manera similar, $\displaystyle \left(\Q, +, \cdot\right) $, $\displaystyle \left(\R, +, \cdot \right) $ y $\displaystyle \left(\C, +, \cdot \right) $ también son anillos unitarios conmutativos.
\item El conjunto $\displaystyle \left(\mathcal{M}_{n}\left(\R\right), +, \cdot \right) $ es un anillo unitario (es no conmutativo si $\displaystyle n \geq 2 $). 
\item $\displaystyle \left(\Z_{n}, +, \cdot \right) $ con $\displaystyle n \geq 2 $ es un anillo unitario conmutativo. 
\item $\displaystyle \left(2\Z, + , \cdot \right) $ es un anillo conmutativo no unitario (esto es cierto en general para $\displaystyle d\Z $ con $\displaystyle d \in \Z/ \left\{ -1,0,1\right\}  $).
\end{enumerate}
\end{eg}
\begin{notation}
Dado un anillo $\displaystyle \left(A, + , \cdot \right) $, $\displaystyle a \in A $ y $\displaystyle n \in \N $, denotamos 
\[na := \underbrace{a + \cdots + a}_{n \; \text{veces}}, \quad -na = \underbrace{\left(-a\right) + \cdots + \left(-a\right)}_{n \; \text{veces}} \quad a^{n} = \underbrace{a \cdots a}_{n \; \text{veces}} .\]
Además, si $\displaystyle A $ es unitario definimos que $\displaystyle a^{0}=1 $. Decimos que $\displaystyle 0 $ es el elemento neutro para la suma y $\displaystyle 1 $ es el elemento neutro para el producto, si es unitario. 
\end{notation}
\begin{prop}
Sea $\displaystyle \left(A, +, \cdot \right) $ un anillo. 
\begin{enumerate}
\item $\displaystyle \forall a \in A $, $\displaystyle a \cdot 0 = 0 \cdot a = 0 $.
\item $\displaystyle \forall a,b \in A $, $\displaystyle -\left(ab\right) = \left(-a\right)b = a\left(-b\right) $.
\item Si $\displaystyle \left(B, \oplus, *\right) $ es un anillo, entonces $\displaystyle A \times B $ también es un anillo con las operaciones coordenada a coordenada. Además, si $\displaystyle A $ y $\displaystyle B $ son unitarios, $\displaystyle A \times B $ es unitario.
\end{enumerate}
\end{prop}
\begin{proof}
\begin{enumerate}
\item Sea $\displaystyle a \in A $,
	\[a \cdot 0 = a \cdot \left(0 + 0\right) = a \cdot 0 + a \cdot 0 \iff a \cdot 0 = 0 .\]
Por la izquierda se hace igual.
\item Si $\displaystyle a,b \in A $ tenemos que 
	\[ab + \left(-a\right)b = \left(a + \left(-a\right)\right)b = 0 \cdot b = 0 .\]
	Los otros casos son análogos.
\item Tenemos que $\displaystyle A \times B $ es un grupo abeliano con la suma por serlo $\displaystyle A $ y $\displaystyle B $. Es fácil ver que el producto es una operación interna, veamos que es asociativo. Si $\displaystyle \left(a,b\right), \left(c,d\right), \left(e,f\right) \in A \times B $,
	\[
	\begin{split}
		\left(\left(a,b\right) \cdot \left(c,d\right)\right) \cdot \left(e,f\right) = & \left(a \cdot c, b * d\right) \cdot \left(e,f\right) = \left(\left(a \cdot c\right) \cdot e, \left(b * d\right)* f\right) \\
		= &  \left(a \cdot \left(c \cdot e\right), b * \left(d * f\right)\right) = \left(a, b\right) \cdot \left(c \cdot e, d * f\right) \\
		= &  \left(a,b\right) \cdot \left(\left(c, d\right) \cdot \left(e,f\right)\right).
	\end{split}
	\]
Veamos que el se cumple la propiedad distributiva:
\[
\begin{split}
	\left[\left(a,b\right) + \left(c,d\right)\right] \cdot \left(e,f\right) = & \left(a + c, b \oplus d\right) \cdot \left(e,f\right) = \left(\left(a+c\right) \cdot e, \left(b \oplus d\right) * f\right) \\
	= & \left(a \cdot e + c \cdot e, b * f \oplus d * f\right) = \left(a \cdot e, b * f\right) + \left(c \cdot e, d * f\right) \\
	= & \left(a,b\right) \cdot \left(e,f\right) + \left(c,d\right) \cdot \left(e,f\right).
\end{split}
\]
Por otro lado, si son los dos unitarios está claro que $\displaystyle A \times B $ también lo será puesto que $\displaystyle \left(1_{A}, 1_{B}\right) \in A \times B $ y $\displaystyle \forall \left(a,b \right)\in A\times B $ se tiene que
\[\left(1_{A}, 1_{B}\right) \cdot \left(a,b\right) = \left(a,b\right) \cdot \left(1_{A}, 1_{B}\right) = \left(a,b\right) .\]
\end{enumerate}
\end{proof}
\begin{definition}[Subanillo]
Sea $\displaystyle \left(A, +, \cdot \right) $ un anillo y $\displaystyle \emptyset \neq B \subset A $. Diremos que $\displaystyle \left(B, +, \cdot \right) $ es \textbf{subanillo} de $\displaystyle A $ si $\displaystyle \left(B, +\right) $ es subgrupo de $\displaystyle \left(A, +\right) $ y $\displaystyle B $ es cerrado para el producto, es decir, $\displaystyle \forall b_{1}, b_{2} \in B $, $\displaystyle b_{1} \cdot b_{2} \in B $.
\end{definition}
\begin{observation}
Sea $\displaystyle \left(A, +, \cdot \right) $ un anillo unitario y $\displaystyle \left(B, +, \cdot \right) $ un subanillo.
\begin{enumerate}
\item $\displaystyle B $ puede no ser unitario. En efecto, en el ejemplo anterior vimos que $\displaystyle \left(2\Z, + , \cdot \right) $ es subanillo no unitario de $\displaystyle \left(\Z, + , \cdot \right) $, que es unitario. 
\item Puede ser que $\displaystyle B $ sea unitario pero $\displaystyle 1_{A} \not\in B $. En efecto, tenemos que $\displaystyle A \times B $ es unitario y $\displaystyle \left\{ 0\right\} \times B $ es un subanillo unitario, pero las unidades son distintas puesto que en el primer caso la unidad es $\displaystyle \left(1_{A}, 1_{B}\right) $ y en el segundo es $\displaystyle \left(0, 1_{B}\right) $.
\end{enumerate}
\end{observation}
\begin{definition}[Subanillo unitario]
Dado un anillo $\displaystyle \left(A, +, \cdot\right) $, llamamos \textbf{subanillo unitario} de $\displaystyle A $ a un subanillo tal que $\displaystyle 1_{A} \in B $ (por tanto $\displaystyle 1_{B} = 1_{A} $). 
\end{definition}
\begin{definition}[Unidad]
Sea $\displaystyle A $ un anillo conmutativo unitario y $\displaystyle a \in A $. Diremos que $\displaystyle a $ es \textbf{unidad} si existe $\displaystyle b \in A $ tal que $\displaystyle a \cdot b = 1_{A} $. 
\end{definition}
\begin{observation}
	\begin{enumerate}
	\item Está claro que $\displaystyle 1_{A} $ es siempre unidad y $\displaystyle 0_{A} $ nunca es unidad. 
	\item Si $\displaystyle a \in A $ es unidad, podemos hablar de su inverso multiplicativo como $\displaystyle a^{-1} $, puesto que de haberlo es único. En efecto, supongamos que existe $\displaystyle b, b' \in A $ tales que $\displaystyle a \cdot b = a \cdot b' = 1 $, entonces,
		\[b = b \cdot 1 = b \cdot \left(a b'\right) = b' .\]
	\end{enumerate}
\end{observation}
\begin{definition}[Conjunto de unidades]
Sea $\displaystyle A $ un anillo conmutativo unitario. Definimos el \textbf{conjunto de todas las unidades} de $\displaystyle A $ como
\[\mathcal{U}\left(A\right) = \left\{ a \in A \; : \; a \; \text{unidad}\right\}  .\]
\end{definition}
\begin{observation}
Es fácil ver que el conjunto de las unidades es un grupo abeliano con el producto. En efecto:
\begin{itemize}
\item Si $\displaystyle a,b \in \mathcal{U}\left(A\right) $, tenemos que 
	\[ab\left(b^{-1}a^{-1}\right) = 1_{A} .\]
	Por tanto, $\displaystyle ab \in \mathcal{U}\left(A\right) $, por lo que la operación es interna. 
\item La asociatividad de la operación se deduce por ser $\displaystyle A $ un anillo, al igual que la existencia de los inversos se deduce de la definición de unidad. 
\item El elemenento neutro claramente es $\displaystyle 1_{A} $. 
\end{itemize}
Por tanto, los inversos con el producto son únicos y si $\displaystyle a \cdot b = 1_{A} $, podemos escribir $\displaystyle b = a^{-1} $.
\end{observation}
\begin{eg}
\begin{enumerate}
\item $\displaystyle \mathcal{U}\left(\Q\right) = \Q^{*} $. Lo mismo sucede en $\displaystyle \R $ y $\displaystyle \C $. 
\item $\displaystyle \mathcal{U}\left(\Z\right) = \left\{ 1, -1\right\}  $. 
\item $\displaystyle \mathcal{U}\left(\Z_{n}\right)= \left\{ a \in \Z_{n} \; : \; \mcd\left(a,n\right)=1\right\}  $. 
\end{enumerate}
\end{eg}
\begin{prop}
Si $\displaystyle A \times B $ es un anillo unitario conmutativo, entonces $\displaystyle \mathcal{U}\left(A \times B\right) = \mathcal{U}\left(A\right) \times \mathcal{U}\left(B\right) $. 
\end{prop}
\begin{proof}
Sea $\displaystyle \left(a,b\right) \in \mathcal{U}\left(A \times B\right) $, por lo que existe $\displaystyle \left(c,d\right) \in A \times B $ tal que 
\[ \left(a,b\right) \cdot \left(c,d\right) = \left(a \cdot c, b \cdot d\right) = \left(1_{A}, 1_{B}\right) .\]
Por tanto, $\displaystyle a \in \mathcal{U}\left(A\right) $ y $\displaystyle b \in \mathcal{U}\left(B\right) $ y $\displaystyle \left(a,b\right) \in \mathcal{U}\left(A\right) \times \mathcal{U}\left(B\right) $. El recíproco es análogo. 
\end{proof}
\begin{definition}[Divisor de cero y dominio de integridad]
	Sea $\displaystyle A $ un anillo y $\displaystyle a \in A / \left\{ 0\right\}  $. Diremos que $\displaystyle a $ es \textbf{divisor de cero} si existe $\displaystyle b \in A/ \left\{ 0\right\}  $ tal que $\displaystyle a \cdot b = 0 $. Decimos que $\displaystyle A $ es \textbf{dominio de integridad} si no tiene divisores de cero.
\end{definition}
\begin{eg}
\begin{enumerate}
	\item Podemos ver que $\displaystyle \Z_{6} $ no es dominio de integridad, puesto que $\displaystyle [2] \cdot [3] = [0] $ y $\displaystyle [2], [3] \neq 0 $, por lo que $\displaystyle [2] $ y $\displaystyle [3] $ son divisores de cero. 
	\item En general, $\displaystyle \Z_{p} $ con $\displaystyle p $ primo es dominio de integridad. 
\end{enumerate}
\end{eg}
\begin{prop}
Una unidad no es divisor de cero. 
\end{prop}
\begin{proof}
	Supongamos que $\displaystyle a \in \mathcal{U}\left(A\right) $ es divisor de cero. Por tanto, $\displaystyle \exists b \in A / \left\{ 0\right\}  $ tal que $\displaystyle ab = 0_{A} $. Como $\displaystyle a $ es unidad, existe $\displaystyle a^{-1} $, por tanto
	\[ab = 0_{A} \Rightarrow a^{-1}\left(ab\right) = a^{-1}0_{A} \Rightarrow b = 0_{A} .\]
	Esto contradice nuestra hipótesis, por lo que debe ser que $\displaystyle a $ no es divisor de cero.
\end{proof}
\begin{definition}[Cuerpo]
	Sea $\displaystyle \left(A, +, \cdot\right) $ un anillo conmutativo unitario. Diremos que $\displaystyle A $ es un \textbf{cuerpo} si $\displaystyle \mathcal{U}\left(A\right) = A / \left\{ 0\right\}  $, es decir, todo elemento salvo el 0 tiene inverso multiplicativo.
\end{definition}
\begin{eg}
\begin{enumerate}
\item Antes hemos visto que $\displaystyle \Q $, $\displaystyle \R $ y $\displaystyle \C $ son cuerpos.
\item Si $\displaystyle p \geq 2 $ es primo, $\displaystyle \left(\Z_{p}, +, \cdot \right) $ también es cuerpo y lo llamaremos $\displaystyle \F_{p} $. 
\item Como $\displaystyle \mathcal{U}\left(\Z\right) = \left\{ -1,1\right\}  $ está claro que $\displaystyle \Z $ no es un cuerpo. De manera similar, $\displaystyle \Z_{m} $ con $\displaystyle m $ no primo tampoco es un cuerpo.
\item $\displaystyle \left(\Q[\mathtt{t}], +, \cdot \right) $ no es un cuerpo. 
\end{enumerate}
\end{eg}
\begin{observation}
\begin{enumerate}
\item Ser dominio de integridad no se conserva bajo productos directos. En efecto, sabemos que $\displaystyle \R $ es un dominio de integridad, pero $\displaystyle \left(\R^{2}, +, \cdot \right) $ no es un dominio de integridad, puesto que $\displaystyle \left(0,1\right) \cdot \left(1,0\right) = \left(0,0\right) $.
\item Un cuerpo es un dominio de integridad, puesto que las unidades del cuerpo son todas menos el cero y una unidad no puede ser divisor de cero. 
\end{enumerate}
\end{observation}
\begin{definition}[Característica de un anillo]
Sea $\displaystyle A $ un anillo unitario. Definimos la \textbf{característica} de $\displaystyle A $, $\displaystyle \Char\left(A\right) $, al mínimo $\displaystyle k \in \N $ tal que $\displaystyle k \cdot 1_{A}= 1_{A} + \cdots + 1_{A} = 0_{A} $. Si no existe $\displaystyle k \in \N $ con $\displaystyle k \cdot 1_{A} = 0_{A} $ decimos que $\displaystyle A $ tiene \textbf{característica cero}.
\end{definition}
\begin{eg}
Es fácil ver que $\displaystyle \Q $, $\displaystyle \R $ y $\displaystyle \C $ tienen característica cero. Por otro lado, $\displaystyle \F_{p} $ tiene caracterísctica $\displaystyle p $.
\end{eg}
\begin{observation}
Si $\displaystyle A $ es un anillo con característica finita y es dominio de integridad, entonces la característica es un número primo. Si suponemos que $\displaystyle \Char(A) = p \cdot m $ con $\displaystyle p $ divisor primo, entonces tenemos que
\[\left(p1_{A}\right)\left(m1_{A}\right) = pm1_{A} = 0 .\]
Como se trata de un dominio de integridad no puede haber divisores de cero, debe ser que $\displaystyle m = 1 $. 
\end{observation}
\subsection{Enteros de Gauss}
Consideremos el anillo $\displaystyle \left(\C, +, \cdot \right) $ (sabemos que es un cuerpo) e $\displaystyle i \in \C $. Definimos el conjunto 
\[\Z[i] = \left\{ a + bi \; : \; a,b \in \Z\right\}  .\]
Como $\displaystyle \left\{ 1, i\right\}  $ es una base de $\displaystyle \C $, sabemos que todos los elementos de $\displaystyle \Z[i] $ se pueden expresar de forma única como combinación lineal de esos dos elementos. Es fácil ver que $\displaystyle \Z[i] $ es un subanillo unitario de $\displaystyle \C $. En particular, tenemos que $\displaystyle \Z[i] $ tiene estructura de anillo unitario conmutativo. Como $\displaystyle \C $ es dominio de integridad y $\displaystyle \Z[i] $ es subanillo, $\displaystyle \Z[i] $ es también dominio de integridad. Veamos que 
\[\mathcal{U}\left(\Z[i]\right) = \left\{ \pm1, \pm i\right\}  .\]
Es trivial ver que $\displaystyle \left\{ \pm 1, \pm i \right\} \subset \mathcal{U}\left(\Z[i]\right) $. Recíprocamente, si $\displaystyle a + bi \in \mathcal{U}\left(\Z[i]\right) $ existe $\displaystyle c + di \in \mathcal{U}\left(\Z[i]\right) $ tal que 
\[\left(a + bi\right)\left(c + di\right) = 1 \iff \left(a-bi\right)\left(c-di\right)=1 .\]
Así, nos queda que 
\[1 = \left(a +bi\right)\left(c + di\right)\left(a-bi\right)\left(c - di\right) = \left(a^{2}+b^{2}\right)\left(c^{2}+d^{2}\right) .\]
Necesariamente debe ser que $\displaystyle a^{2}+b^{2} = 1 $ y $\displaystyle c^{2}+d^{2} = 1 $. Así, podemos considerar cuatro casos:
\begin{itemize}
\item Si $\displaystyle a = 0 $ y $\displaystyle b = 1 $, tenemos que $\displaystyle a + bi = i $.
\item Si $\displaystyle a = 1 $ y $\displaystyle b = 0 $, tenemos que $\displaystyle a + bi = 1$. 
\item Si $\displaystyle a = -1 $ y $\displaystyle b = 0 $, tenemos que $\displaystyle a +bi = -1$.
\item Si $\displaystyle a = 0 $ y $\displaystyle b = -1 $, tenemos que $\displaystyle a +bi = -i $.
\end{itemize}
\begin{observation}
	Un entero de Gauss $\displaystyle z \in \Z[i] $ es una unidad si y solo si $\displaystyle z \cdot \overline{z} = 1 $. 
\end{observation}
\subsection{Anillo de polinomios}
Sea $\displaystyle A $ un anillo conmutativo unitario. Llaremos \textbf{anillo de polinomios} en la variable $\displaystyle \mathtt{x} $ a $\displaystyle A[\mathtt{x} ] $ donde sus elementos son
\[a_{n}\mathtt{x} ^{n} + \cdots + a_{1}\mathtt{x} +a_{0} , \; a_{i} \in A, \; n \in \N.\]
Además, decimos que $\displaystyle a_{n}\mathtt{x}^{n} + \cdots +a_{1}\mathtt{x} + a_{0} $ es el polinomio 0 si y solo si $\displaystyle a_{0} = \cdots = a_{n} = 0 $. \\
Dotamos a $\displaystyle A[\mathtt{x} ] $ de una operación de suma y producto que conocemos. Sean $\displaystyle p,q \in A[\mathtt{x} ] $ con 
\[p = a_{n}\mathtt{x} ^{n} + \cdots + a_{1}\mathtt{x} +a_{0}, \; a_{i} \in A, \; n \in \N .\]
\[q = b_{m}\mathtt{x} ^{m} + \cdots + b_{1}\mathtt{x} +b_{0}, \; b_{j} \in A, \; m \in \N .\]
La suma la definimos de la forma:
\[p + q : = \sum^{ \max \left\{ n,m\right\} }_{k = 0}\left(a_{k}+b_{k}\right)\mathtt{x} ^{k} .\]
El producto lo deinimos de la forma:
\[p \cdot q:= \sum^{n+m}_{k = 0}c_{k}\mathtt{x} ^{k}, \quad c_{k} = \sum_{i + j = k}a_{i}b_{j} .\]
Es fácil ver que $\displaystyle A[\mathtt{x} ] $ es un anillo conmutativo unitario y tiene como subanillo a $\displaystyle A \subset A[\mathtt{x} ] $. 
\begin{definition}
	Sea $\displaystyle A[\mathtt{x} ] $ un anillo de polinomios y sea $\displaystyle p \in A[\mathtt{x} ] $, con $\displaystyle p = a_{n}\mathtt{x} ^{n} + \cdots + a_{1}\mathtt{x} +a_{0} $. 
	\begin{enumerate}
	\item Llamamos \textbf{grado} de $\displaystyle p $, $\displaystyle \grad\left(p\right)= n $. \footnote{Por conveción decimos que $\displaystyle \grad\left(0\right) = -\infty $, así tenemos que los elementos de $\displaystyle A $ son los polinomios de grado menor o igual a 0.} 
	\item Diremos que $\displaystyle a_{n}, \ldots, a_{1}, a_{0} $ son los \textbf{coeficientes} de $\displaystyle p $.
	\item Al coeficiente $\displaystyle a_{0} $ lo llamamos \textbf{coeficiente independiente} y al coeficiente $\displaystyle a_{n} $ lo llamamos \textbf{coeficiente director}, y escribimos $\displaystyle l\left(p\right)=a_{n} $.
	\end{enumerate}
\end{definition}
\begin{lema}
	Sean $\displaystyle p,q \in A[\mathtt{x} ] $ no nulos. 
	\begin{enumerate}
		\item $\displaystyle \grad\left(p + q\right) \leq \max \left\{ \grad\left(p\right), \grad\left(q\right)\right\}  $ y se da la igualdad si $\displaystyle l\left(p\right)+l\left(q\right)\neq 0 $.
		\item $\displaystyle \grad\left(p \cdot q\right) \leq \grad\left(p\right)+\grad\left(q\right) $ y se da la igualdad si $\displaystyle l\left(p\right)l\left(q\right) \neq 0 $ y tendremos que $\displaystyle l\left(p \cdot q\right) = l\left(p\right) \cdot l\left(q\right) $.
	\end{enumerate}
\end{lema}
\begin{eg}
	Consideremos $\displaystyle \Z_{6}[\mathtt{x} ] $ y $\displaystyle 3\mathtt{x} +1, 4\mathtt{x} +2 \in \Z_{6}[\mathtt{x} ] $. Tenemos que 
	\[\left(3\mathtt{x} +1\right)+\left(4\mathtt{x} +2\right) = 7\mathtt{x} +3 = \mathtt{x} +3 .\]
\[\left(3\mathtt{x} +1\right) \cdot \left(4\mathtt{x} +2\right) = 12\mathtt{x} ^{2}+10\mathtt{x} +2 = 4\mathtt{x} + 2 .\]	
\end{eg}
\begin{prop} 
	Sea $\displaystyle A $ un anillo conmutativo unitario.
\begin{enumerate}
	\item Si $\displaystyle A $ es dominio de integridad, entonces $\displaystyle \mathcal{U}\left(A\right) = \mathcal{U}\left(A[\mathtt{x} ]\right)$. 
	\item $\displaystyle A $ es dominio de integridad si y solo si $\displaystyle A[\mathtt{x} ] $ es dominio de integridad. 
\end{enumerate}
\end{prop}
\begin{proof}
\begin{enumerate}
	\item Está claro que $\displaystyle \mathcal{U}\left(A\right) \subset \mathcal{U}\left(A[\mathtt{x} ]\right) $ por ser $\displaystyle A \subset A[\mathtt{x} ] $. Sea $\displaystyle p \in \mathcal{U}\left(A[\mathtt{x} ]\right) $, entonces existe $\displaystyle q \in \mathcal{U}\left(A[\mathtt{x} ]\right) $ tal que $\displaystyle pq = 1 $. Como $\displaystyle p,q \neq 0 $, tenemos que $\displaystyle l\left(p\right), l\left(q\right) \neq 0 $, por lo que $\displaystyle l\left(p\right)l\left(q\right) \neq 0 $. 
	Así, tenemos que 
	\[0 =\grad\left(1\right) = \grad\left(pq\right) = \grad\left(p\right)+\grad\left(q\right) .\]
	Por tanto, necesariamente debe ser que $\displaystyle \grad\left(p\right) = \grad\left(q\right) = 0 $. Por tanto, $\displaystyle p = a_{0}, q = b_{0} \in A $, por lo que $\displaystyle a_{0}b_{0} = 1 $. Así, $\displaystyle p \in \mathcal{U}\left(A\right) $. 
\item Supongamos que $\displaystyle A $ es dominio de integridad y sean $\displaystyle p,q \in A[\mathtt{x}] $. Como $\displaystyle p,q \neq 0 $ tenemos que $\displaystyle l\left(p\right), l\left(q\right) \neq 0 $, por lo que $\displaystyle l\left(p\right)l\left(q\right) \neq 0 $. Así,
	\[l\left(pq\right) = l\left(p\right)l\left(q\right) \neq 0 \Rightarrow pq \neq 0 .\]
	Recíprocamente, como $\displaystyle A \subset A[\mathtt{x}] $ es subanillo tenemos que si $\displaystyle A[\mathtt{x}] $ es dominio de integridad, $\displaystyle A $ también lo es. 
\end{enumerate}
\end{proof}
\begin{observation}
\begin{enumerate}
	\item Es fácil ver que $\displaystyle \Char\left(A\right) = \Char\left(A[\mathtt{x}]\right) $. 
	\item El anillo $\displaystyle A[\mathtt{x}] $ no es un cuerpo, puesto que $\displaystyle \mathtt{x} $ no es unidad. 
\end{enumerate}
\end{observation}
\section{Ideales}
\begin{definition}[Ideal]
	Sea $\displaystyle A $ un anillo unitario conmutativo y $\displaystyle \mathfrak{a} \subset A $ no vacío. Diremos que $\displaystyle \mathfrak{a} $ es un \textbf{ideal} si:
\begin{description}
\item[(a)] $\displaystyle \left(\mathfrak{a}, +\right) $ es subgrupo de $\displaystyle \left(A, +, 0_{A}\right) $. 
\item[(b)] $\displaystyle\forall b \in \mathfrak{a} $ y $\displaystyle\forall x \in A $ se tiene que $\displaystyle bx \in \mathfrak{a} $, es decir, $\displaystyle \mathfrak{a} A \subset \mathfrak{a} $. 
\end{description}
\end{definition}
\begin{eg}
\begin{enumerate}
	\item Algunos ideales triviales son $\displaystyle \left\{ 0\right\}  $ y $\displaystyle A $. 
	\item Si $\displaystyle \mathfrak{a} \subsetneq A $ es un ideal, decimos que es un \textbf{ideal propio}. 
	\item Si $\displaystyle A = \Z $ y $\displaystyle \mathfrak{a} = 3\Z $, tenemos que $\displaystyle \mathfrak{a} $ es un ideal. En general, $\displaystyle n\Z $ para $\displaystyle n \in \N $ es un ideal de $\displaystyle \Z $. 
\end{enumerate}
\end{eg}
\begin{observation}
\begin{enumerate}
\item Si $\displaystyle \mathfrak{a} \subset A $ es un ideal y $\displaystyle u \in \mathfrak{a} $ es unidad, tenemos que $\displaystyle \mathfrak{a} = A $. En efecto, si $\displaystyle x \in A $, tenemos que $\displaystyle  \left(xu^{-1}\right)u \in A\mathfrak{a} \subset \mathfrak{a} $. Así, los ideales propios no tienen unidades.
\item Sea $\displaystyle b \in A $. Decimos que $\displaystyle \left(b\right) := bA = \left\{ bx \; : \; x \in A\right\} $ es el \textbf{ideal principal generado por} $\displaystyle b $. 
\end{enumerate}
\end{observation}
\subsection{Construcción del anillo cociente}
Sea $\displaystyle \left(A, +, \cdot \right) $ un anillo conmutativo unitario y $\displaystyle \mathfrak{a} \subsetneq A $ un ideal. Entonces, sabemos que $\displaystyle \mathfrak{a} $ es subgrupo normal de $\displaystyle A $, por ser $\displaystyle A $ abeliano. Por tanto, podemos considerar el grupo $\displaystyle A/\mathfrak{a} $. Recordamos que 
\[[a] = [b] \iff a - b \in \mathfrak{a}  .\]
Definíamos la suma de la forma: 
\[ + : A/\mathfrak{a} \times A/\mathfrak{a} \to A/\mathfrak{a} : \left([a_{1}], [a_{2}]\right) \to [a_{1}] + [a_{2}] = [a_{1}+a_{2}].\]
De forma análoga podemos definir el producto, 
\[ \cdot : A/\mathfrak{a} \times A/\mathfrak{a} \to A/\mathfrak{a} : \left([a_{1}], [a_{2}]\right) \to [a_{1}] \cdot [a_{2}] := [a_{1} \cdot a_{2}] .\]
Veamos que la aplicación está bien definida. Supongamos que $\displaystyle [a_{1}] = [a_{2}] $ y $\displaystyle [b_{1}] = [b_{2}] $. Tenemos que 
\[a_{1}b_{1}-a_{2}b_{2} = a_{1}b_{1}-a_{2}b_{1}-a_{2}b_{2}+a_{2}b_{1} = b_{1}\underbrace{\left(a_{1}-a_{2}\right) }_{\in \mathfrak{a} }+a_{2}\underbrace{\left(b_{1}-b_{2}\right)}_{\in \mathfrak{a} } \in \mathfrak{a}  .\]
Así tenemos que $\displaystyle [a_{1}b_{1}] = [a_{2}b_{2}] $ y la operación está bien definida. Así, diremos que $\displaystyle \left(A/\mathfrak{a} , +, \cdot \right) $ es el \textbf{anillo cociente} que es conmutativo y unitario. 
\begin{eg}
\begin{enumerate}
\item Consideremos $\displaystyle A = \Z $ y $\displaystyle \mathfrak{a} = n\Z $, que es un ideal por un ejemplo anterior. Sabemos que $\displaystyle \Z/n\Z \cong \Z_{n} $. Tenemos que $\displaystyle \Z $ es dominio de integridad y $\displaystyle n\Z $ también lo es por ser subanillo, sin embargo $\displaystyle \Z_{n} $ sólo es dominio de integridad si $\displaystyle n $ es primo. Por tanto, ser dominio de integridad no se conserva por cocientes. 
\item Sea $\displaystyle A = \R[\mathtt{x}] $ y $\displaystyle \mathfrak{a} = \left(\mathtt{x}^{2}+1\right) $. Estudiemos el conjunto cociente $\displaystyle \R[\mathtt{x}]/\mathfrak{a} $. Tenemos que $\displaystyle [\mathtt{x}^{2}+1] = [0]$, por lo que $\displaystyle [\mathtt{x}^{2}] = [-1]$. 
	Así, si $\displaystyle k \in \N $, tenemos que 
	\[ \left[\mathtt{x}^{2k}\right] = \left[\mathtt{x}^{2}\right] ^{k} = \left[-1\right] ^{k} .\]
	\[\left[\mathtt{x}^{2k+1}\right] = \left[\mathtt{x}^{2}\right] ^{k} \cdot \left[\mathtt{x}\right] = \left[\left(-1\right)^{k}\right] \left[\mathtt{x}\right]  .\]
Así, si tomamos el polinomio $\displaystyle a_{n}\mathtt{x}^{n} + \cdots + a_{1}\mathtt{x}+a_{0} $, tenemos que 
\[ \left[a_{n}\mathtt{x}^{n}+\cdots + a_{1}\mathtt{x}+a_{0}\right] = [b] [\mathtt{x}] + [c].\]
Así, si $\displaystyle p \in \R[\mathtt{x}]/\mathfrak{a} $, tenemos que $\displaystyle p = b\mathtt{x}+ c $. Análogamente, si $\displaystyle q \in \R[\mathtt{x}]/\mathfrak{a} $ con $\displaystyle q = d\mathtt{x}+e $, tenemos que 
\[ p + q = \left(b + d\right)\mathtt{x} + \left(c + e\right) .\]
\[p \cdot q = \left(b\mathtt{x}+c\right)\left(d\mathtt{x}+e\right)=bd\mathtt{x}^{2}+\left(be+cd\right)\mathtt{x}+ce = \left(be + cd\right)\mathtt{x}+\left(ce-bd\right) .\]
Así, para cada elemento $\displaystyle p \in \R[\mathtt{x}]/\mathfrak{a} $ podemos encontrar un elemento inverso respecto a la multiplicación. Así, tenemos que $\displaystyle \R[\mathtt{x}]/\mathfrak{a} $ es un cuerpo. En efecto, se trata de $\displaystyle \C $. 	
\end{enumerate}
\end{eg}
\begin{definition}
	Sea $\displaystyle A $ un anillo conmutativo unitario y $\displaystyle \mathfrak{a}, \mathfrak{b} \subset A $ ideales. 
	\begin{enumerate}
		\item Definimos el \textbf{ideal suma} como $\displaystyle \mathfrak{a} + \mathfrak{b} := \left\{ a + b \; : \; a \in \mathfrak{a}, b \in \mathfrak{b}\right\}  $. 
		\item Definimos el \textbf{ideal intersección} como $\displaystyle \mathfrak{a} \cap \mathfrak{b} := \left\{ x \in A \; : \; x \in \mathfrak{a}, x \in \mathfrak{b}\right\}  $.
		\item Definimos el \textbf{ideal producto} como $\displaystyle \mathfrak{a} \cdot \mathfrak{b} := \left\{ \sum^{n}_{k = 1}x_{k}y_{k} \; : \; x_{k} \in \mathfrak{a}, y_{k}\in \mathfrak{b}, n \in \N\right\}  $. 
	\end{enumerate}	
\end{definition}
\begin{definition}[Ideal generado por un conjunto]
Sea $\displaystyle A $ un anillo conmutativo unitario y $\displaystyle \emptyset \neq S \subset A $. Llamamos \textbf{ideal generado por} $\displaystyle S $ a
\[ \left(S\right) := \left\{ a_{1}s_{1} + \cdots + a_{k}s_{k} \; : \; s_{i} \in S, a_{i} \in A, k \in \N\right\}  .\]
Además, $\displaystyle \left(S\right) $ es el menor ideal que contiene a $\displaystyle S $ y si $\displaystyle S = \left\{ s_{1}, \ldots, s_{n}\right\}  $, tenemos que $\displaystyle \left(S\right) = \left(s_{1}, \ldots, s_{n}\right) = \left(s_{1}\right)+\cdots + \left(s_{n}\right) $. 
\end{definition}
\begin{eg}
En $\displaystyle \Z $ tenemos que $\displaystyle \left(4,6\right) = \left(2\right) $. Sin embargo, en $\displaystyle \R $, $\displaystyle \left(4,6\right) = \R $. 
\end{eg}
\begin{definition}[Ideal primo]
	Sea $\displaystyle A $ un anillo conmutativo unitario. Sea $\displaystyle \mathfrak{p} \subsetneq A $ un ideal. Diremos que $\displaystyle \mathfrak{p} $ es un \textbf{ideal primo} si $\displaystyle xy \in \mathfrak{p} $ implica que $\displaystyle x \in \mathfrak{p} $ o $\displaystyle y \in \mathfrak{p} $, $\displaystyle \forall x ,y \in A $. 
\end{definition}
\begin{eg}
	En $\displaystyle \Z $ sabemos que $\displaystyle n\Z $ son los ideales, con $\displaystyle n \in \N $. En efecto, si $\displaystyle \mathfrak{a} \subset \Z $ es un ideal, tenemos que es subgrupo aditivo, por lo que $\displaystyle \mathfrak{a} = n\Z $ para algún $\displaystyle n \in \N $. Entonces, tenemos que $\displaystyle n\Z $ es ideal primo si y solo si $\displaystyle n $ es primo. 
\end{eg}
\begin{prop}
Sea $\displaystyle A $ un anillo conmutativo unitario y $\displaystyle \mathfrak{p} \subset A $ un ideal. Tenemos que $\displaystyle \mathfrak{p} $ es primo si y solo si $\displaystyle A/\mathfrak{p} $ es dominio de integridad.
\end{prop}
\begin{proof}
Sabemos que $\displaystyle \mathfrak{p}\neq A $ y que $\displaystyle A/\mathfrak{p} $ es un anillo también conmutativo unitario. Además, si $\displaystyle a,b \in A $, tenemos que 
\[\left[ab\right]  = \left[a\right] \left[b\right] = \left[0\right] \iff ab \in \mathfrak{p} .\]
\begin{description}
	\item[(i)] Supongamos que $\displaystyle \mathfrak{p} $ es primo y sean $\displaystyle [a], [b] \in A/\mathfrak{p} $ con $\displaystyle [a] \cdot [b] = [0]$. Por lo visto anteriormente, tenemos que $\displaystyle ab \in \mathfrak{p} $, por lo que debe ser que $\displaystyle a \in \mathfrak{p} $ o $\displaystyle b \in \mathfrak{p} $, y en consecuencia debe ser que $\displaystyle [a] = [0] $ o $\displaystyle [b] = [0] $. Por tanto, $\displaystyle A/\mathfrak{p} $ es dominio de integridad.
	\item[(ii)] Supongamos que $\displaystyle A/\mathfrak{p} $ es dominio de integridad y $\displaystyle ab \in \mathfrak{p} $. Así, tenemos que $\displaystyle [ab] = [a] [b]= [0] $. Como $\displaystyle A/\mathfrak{p} $ es dominio de integridad debe ser que $\displaystyle [a] = 0 $ o $\displaystyle [b] = 0 $, es decir, $\displaystyle a \in \mathfrak{p} $ o $\displaystyle b \in \mathfrak{p} $. Por tanto, $\displaystyle \mathfrak{p} $ es primo.
\end{description}
\end{proof}
\begin{definition}[Ideal maximal]
Sea $\displaystyle A $ un anillo y $\displaystyle \mathfrak{m} \subsetneq A $ un ideal. Diremos que es un \textbf{ideal maximal} si no existe un ideal $\displaystyle \mathfrak{a} \subset A $ tal que $\displaystyle \mathfrak{m} \subsetneq \mathfrak{a} $. 
\end{definition}
\begin{prop}
Sea $\displaystyle A $ un anillo conmutativo unitario y $\displaystyle \mathfrak{m} \subset A $ un ideal. Tenemos que $\displaystyle \mathfrak{m} $ es maximal si y solo si $\displaystyle A/\mathfrak{m} $ es cuerpo.
\end{prop}
\begin{proof}
\begin{description}
	\item[(i)] Supongamos que $\displaystyle \mathfrak{m} \subset A $ es un ideal maximal. Tenemos que ver que $\displaystyle A/\mathfrak{m} - \left\{ [0]\right\} \subset \mathcal{U}\left(A/\mathfrak{m}\right) $. Sea $\displaystyle [a] \in A/\mathfrak{m} $ con $\displaystyle [a] \neq [0] $, es decir, $\displaystyle a \in A-\mathfrak{m} $. Como $\displaystyle a \not\in \mathfrak{m} $, podemos considerar el ideal $\displaystyle \left(\mathfrak{m}\cup \left\{ a\right\} \right) \supsetneq \mathfrak{m} $. Como $\displaystyle \mathfrak{m} $ es maximal, necesariamente debe ser que $\displaystyle A = \left( \mathfrak{m} \cup \left\{ a\right\} \right)  $. 
		Así, $\displaystyle 1 \in A = \left(\mathfrak{m} \cup \left\{ a\right\} \right)  $, por lo que existe $\displaystyle b \in \mathfrak{m} $ y $\displaystyle c \in A $ tal que $\displaystyle 1 = b + ac $. Por tanto, 
		\[[1] = [ac] = [a] [c] \Rightarrow [a] \in \mathcal{U}\left(A/\mathfrak{m}\right) .\]
	\item[(ii)] Supongamos que $\displaystyle A/\mathfrak{m} $ es un cuerpo y existe un ideal $\displaystyle \mathfrak{m}' $ tal que $\displaystyle \mathfrak{m} \subsetneq \mathfrak{m}' $. Queremos ver que $\displaystyle \mathfrak{m}' = A $, para ello basta con ver que $\displaystyle 1\in \mathfrak{m}' $. Sea $\displaystyle a \in \mathfrak{m}' - \mathfrak{m} $. Como $\displaystyle A/\mathfrak{m} $ es un cuerpo, existe $\displaystyle b \in A/\mathfrak{m} $ tal que $\displaystyle [1] = [a][b]=[ab] $. Así, tenemos que 
		\[[1-ab] = [0] \Rightarrow 1-ab \in \mathfrak{m} \Rightarrow 1 \in \mathfrak{m} + \left(a\right) .\]
	Por tanto, debe ser que $\displaystyle 1 \in \mathfrak{m}' $. 
\end{description}
\end{proof}
\begin{eg}
	En $\displaystyle \R[\mathtt{x}] $, el ideal $\displaystyle \left(\mathtt{x}^{2}+1\right) $ es maximal puesto que $\displaystyle \R[\mathtt{x}]/\left(\mathtt{x}^{2}+1\right) $ es un cuerpo.
\end{eg}
\begin{colorary}
Todo ideal maximal es primo.
\end{colorary}
\begin{proof}
Si $\displaystyle \mathfrak{m} \subset A $ es un ideal maximal tenemos que $\displaystyle A/\mathfrak{m} $ es un cuerpo, por lo que $\displaystyle A/\mathfrak{m} $ es dominio de integridad, por lo que $\displaystyle \mathfrak{m} $ es primo.
\end{proof}
\begin{eg}
En $\displaystyle \Z $, los ideales maximales son los $\displaystyle p\Z $ donde $\displaystyle p $ es primo. Es decir, en $\displaystyle \Z $ los ideales primos son los maximales. 
\end{eg}
\section{Homomorfismos de anillos}
\begin{definition}[Homomorfismo de anillos]
Sean $\displaystyle A $ y $\displaystyle B $ anillos y $\displaystyle f : A \to B $. Diremos que $\displaystyle f $ es un \textbf{homomorfismo de anillos} si:
\begin{enumerate}
\item $\displaystyle f\left(a + b\right) = f\left(a\right)+f\left(b\right) $, $\displaystyle \forall a,b \in A $. 
\item $\displaystyle f\left(a \cdot b\right) = f\left(a\right) \cdot f\left(b\right) $, $\displaystyle \forall a,b \in A $.
\end{enumerate}
Diremos que es un \textbf{homomorfismo de anillos unitarios} si $\displaystyle A $ y $\displaystyle B $ son unitarios y $\displaystyle f\left(1\right) = 1 $. 
\end{definition}
\begin{definition}
Dado un homomorfismo de anillos $\displaystyle f : A \to B $:
\begin{enumerate}
\item Si $\displaystyle f $ es biyectiva la llamamos \textbf{isomorfismo}.
\item Si $\displaystyle f $ es inyectiva la llamamos \textbf{monomorfismo}.
\item Si $\displaystyle f $ es sobreyectiva la llamamos \textbf{epimorfismo}. 
\item Si $\displaystyle A = B $, a $\displaystyle f $ la llamamos \textbf{endomorfismo}.
\end{enumerate}
\end{definition}
\begin{prop}
Sea $\displaystyle f:A \to B $ un homomorfismo de anillos unitarios.
\begin{enumerate}
\item $\displaystyle f\left(0\right) = 0 $. 
\item $\displaystyle \forall n \in \Z, \forall a \in A $, $\displaystyle f\left(na\right) = nf\left(a\right) $. 
\item Si $\displaystyle n \in \N, \forall a \in A $, $\displaystyle f\left(a^{n}\right) = f\left(a\right)^{n} $. 
\item Si $\displaystyle a \in \mathcal{U}\left(A\right) $, entonces $\displaystyle f\left(a\right) \in \mathcal{U}\left(B\right) $. En particular, $\displaystyle f|_{\mathcal{U}\left(A\right)}:\mathcal{U}\left(A\right) \to \mathcal{U}\left(B\right) $ es un homomorfismo de grupos y $\displaystyle f\left(a^{n}\right) = f\left(a\right)^{n} $, $\displaystyle \forall n \in \Z $.
\end{enumerate}
\end{prop}
\begin{proof}
\begin{enumerate}
\item Como $\displaystyle f $ es homomorfismo de grupos es trivial. 
\item Sea $\displaystyle n \in \N $ y $\displaystyle a \in A $,
	\[f\left(na\right) = f\left(a+\cdots + a\right)= f\left(a\right)+\cdots +f\left(a\right) = nf\left(a\right) .\]
\item Sea $\displaystyle n \in \N $ y $\displaystyle a \in A $, 
	\[f\left(a^{n}\right) = f\left(a \cdots a\right) = f\left(a\right) \cdots f\left(a\right) = f\left(a\right)^{n} .\]
\item Sea $\displaystyle a \in \mathcal{U}\left(A\right) $, por lo que existe $\displaystyle x \in A $ tal que $\displaystyle ax = 1 $. Así, tenemos que 
	\[f\left(a\right)f\left(x\right) = f\left(ax\right) = f\left(1\right) = 1 \Rightarrow f\left(a\right) \in \mathcal{U}\left(B\right) .\]
\end{enumerate}
\end{proof}
\begin{eg}
\begin{enumerate}
\item Sea $\displaystyle A $ un anillo, entonces la identidad es un homomorfismo de anillos unitario. 
\item Si $\displaystyle B \subset A $ es un subanillo unitario, la función inclusión $\displaystyle i : B \to A $ es homomorfismo de anillos unitarios. 
\item La aplicación $\displaystyle f : \Z \to \R : n \to 2n $ no es un homomorfismo de anillos unitario puesto que $\displaystyle f\left(1\right) = 2 \neq 1 $. 
\item La aplicación $\displaystyle f : A \to A \times A : a \to \left(a,0\right) $ es un homomorfismo de anillos no unitario, puesto que $\displaystyle f\left(1\right) = \left(1,0\right) \neq \left(1,1\right) $. 
\end{enumerate}
\end{eg}

