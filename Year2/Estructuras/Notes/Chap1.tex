\chapter{Grupos}
\begin{definition}[Grupo]
Sea la terna $\displaystyle \left(G, \cdot, e\right) $ donde $\displaystyle G $ es un conjunto no vacío, $\displaystyle \cdot : G \times G \to G $ una operación interna y $\displaystyle e \in G $. Diremos que la terna $\displaystyle \left(G, \cdot , e\right) $ es un \textbf{grupo} si se cumple:
\begin{description}
\item[Asociativa.] $\displaystyle \forall a, b, c \in G $, $\displaystyle \left(a \cdot b\right) \cdot c = a \cdot \left(b \cdot c\right) $.
\item[Elemento neutro.] $\displaystyle \forall a \in G $, $\displaystyle a \cdot e = e \cdot a = a $.
\item[Inversa.] $\displaystyle \forall a \in G, \exists b \in G $, $\displaystyle a \cdot b = b \cdot a = e $.
\end{description}
Además, diremos que $\displaystyle \left(G, \cdot, e\right) $ es \textbf{abeliano} si se cumple la propiedad conmutativa, es decir, $\displaystyle \forall a,b \in G $, $\displaystyle a \cdot b = b \cdot a $.
\end{definition}
\begin{definition}[Orden de un grupo]
Dado un grupo $\displaystyle \left(G, \cdot, e\right) $, llamamos \textbf{orden} del grupo a la cardinalidad de $\displaystyle G $, $\displaystyle \left|G\right| $.
\end{definition}
\begin{eg}
Algunos ejemplos de grupos son:
\begin{enumerate}
\item $\displaystyle \left(\R, +, 0\right) $ es un grupo abeliano.
\item $\displaystyle \left(\R / \left\{ 0\right\} , \cdot, 1\right) $ es un grupo abeliano.
\item $\displaystyle \left(\Z, +, 0\right) $ es un grupo abeliano. 
\item $\displaystyle \left(\N \cup \left\{ 0\right\} , + , 0\right) $ no es un grupo por no haber inversos.
\end{enumerate}
\end{eg}
\begin{prop}
Sea $\displaystyle \left(G, \cdot, e\right) $ un grupo. Entonces se tiene que:
\begin{enumerate}
\item El elemento neutro es único.
\item Dado $\displaystyle a \in G $, existe un único elemento inverso.
\end{enumerate}
\end{prop}
\begin{proof}
Demostremos \textbf{1}. Supongamos que $\displaystyle e $ y $\displaystyle e' $ son ambos elementos neutros. Tenemos que 
\[e = e \cdot e' = e' \cdot e = e' .\]
Así, hemos visto que $\displaystyle e = e' $. Ahora, demostremos \textbf{2}. Si $\displaystyle a \in G $, supongamos que $\displaystyle b, c \in G $ son sus inversos. Entonces tenemos que
\[ b = b \cdot e = b \cdot (a \cdot c) = \left(b \cdot a \right) \cdot c = e \cdot c = c .\]
Así, tenemos que $\displaystyle b = c $.
\end{proof}
\begin{observation}
\begin{enumerate}
\item De ahora en adelante, en vez de escribir $\displaystyle \left(G, \cdot , e\right) $ para nombrar el grupo, escribiremos sólamente $\displaystyle G $. De manera similar, no escribiremos $\displaystyle a \cdot b $ sino $\displaystyle ab $.
\item Dado $\displaystyle a \in G $ finito, a su inverso lo denotaremos por $\displaystyle a^{-1} $.
\item Dado un grupo $\displaystyle G $, va a estar totalmente definido por su tabla de multiplicación (tabla de Cayley). Esta será de la forma
	\begin{center}
	\begin{tabular}{c | c | c | c | c}
	& $\displaystyle e $  & $\displaystyle a_{1} $ & $\displaystyle \cdots $ & $\displaystyle a_{n} $ \\
	\hline 
		$\displaystyle e $ & $\displaystyle e $ & $\displaystyle a_{1} $ & $\displaystyle \cdots  $ & $\displaystyle a_{n} $ \\
		\hline
		$\displaystyle a_{1} $ & $\displaystyle a_{1} $ & $\displaystyle a_{1}^{2} $ & $\displaystyle \cdots  $ & $\displaystyle a_{1} a_{n} $ \\
		\hline
		$\displaystyle \vdots  $ & $\displaystyle \vdots $  & $\displaystyle \vdots $  & $\displaystyle \vdots $ & $\displaystyle \vdots $ \\
		\hline 
		$\displaystyle a_{n} $ & $\displaystyle a_{n} $ & $\displaystyle a_{n}a_{1} $ & $\displaystyle \cdots  $ & $\displaystyle a_{n}^{2} $ 
	\end{tabular}
	\end{center}
\end{enumerate}
\end{observation}
\begin{eg}
	Consideremos el grupo $\displaystyle \left(\Z_{5} / \left\{ 0\right\} , \cdot\right) $. Su tabla de Cayley será:
\begin{center}
\begin{tabular}{c|c|c|c|c}
$\displaystyle \cdot $ & 1 & 2 & 3 & 4 \\
\hline
	1 & 1 & 2 & 3 & 4 \\
\hline 
	2 & 2 & 4 & 1 & 3 \\
\hline 
	3 & 3 & 1 & 4 & 2 \\
\hline 
	4 & 4 & 3 & 2 & 1
\end{tabular}
\end{center}
\end{eg}
\begin{prop}
Sea $\displaystyle G $ un grupo. Entonces, 
\begin{enumerate}
\item $\displaystyle \forall a \in G $, $\displaystyle \left(a^{-1}\right)^{-1} = a $.
\item $\displaystyle \forall a,b,c \in G $, $\displaystyle \left(ab\right)^{-1} = b^{-1}a^{-1} $.
\item $\displaystyle \forall a, b, c \in G $, si $\displaystyle ba = ca $ o $\displaystyle ab = ac $, entonces $\displaystyle b = c $.
\end{enumerate}
\end{prop}
\begin{proof}
Demostramos \textbf{1}. Si $\displaystyle a \in G $, tenemos que 
\[a^{-1}a = a \cdot a^{-1} = e .\]
Dado que el inverso es único, tenemos que $\displaystyle \left(a^{-1}\right)^{-1} = a $. Ahora demostramos \textbf{2}. Si $\displaystyle a,b \in G $, 
\[\left(ab\right)\left(b^{-1}a^{-1}\right) = a  e  a^{-1} = a a^{-1} = e .\]
Por la inversa del inverso, tenemos que $\displaystyle \left(ab\right)^{-1} = b^{-1}a^{-1} $. Finalmente, demostramos \textbf{3}. Si $\displaystyle a,b,c \in G $ y, sin pérdida de generalidad, $\displaystyle ba = ca $, dado que existe $\displaystyle a^{-1}\in G $, tenemos que
\[ ba = ca \iff ba a^{-1} = c a a^{-1} \iff b e = ce \iff b = c .\]

\end{proof}
\begin{eg}
\begin{enumerate}
	\item Consideremos un conjunto $\displaystyle X \neq \emptyset $ y el conjunto de sus biyecciones $\displaystyle \Biy\left(X\right) = \left\{ f : X \to X \; : \; f \; \text{biyección}\right\}  $. Como operación tomamos la composición de funciones. Entonces, $\displaystyle \left(\Biy\left(X\right), \circ\right) $ es un grupo. En efecto:
		\begin{description}
		\item[Asociativa.] La composición de funciones es asociativa.
		\item[Elemento neutro.] Tomamos como elemento neutro la función identidad. En efecto, $\displaystyle id \in \Biy\left(X\right) $ y $\displaystyle \forall f \in \Biy\left(X\right) $, 
			\[\left(f\circ id \right)\left(x\right) = f\left(id \left(x\right)\right) = f\left(x\right) .\]
		\[\left(id \circ f\right)\left(x\right) = id \left(f\left(x\right)\right) = f\left(x\right) .\]
	\item[Inverso.]	Si $\displaystyle f \in \Biy\left(X\right) $, sabemos que por ser $\displaystyle f $ biyectiva existe $\displaystyle f^{-1} \in \Biy\left(X\right) $ tal que $\displaystyle f \circ f^{-1} = id $ y $\displaystyle f^{-1} \circ f = id $. 
		\end{description}
	Así, hemos visto que $\displaystyle \left(\Biy\left(X\right), \circ\right) $ es un grupo, pero no tiene por qué ser abeliano.
\item Sea $\displaystyle \mathcal{M}_{n}\left(\R\right) $, $\displaystyle n \geq 1 $, el conjunto de matrices reales cuadradas con coeficientes en $\displaystyle \R $, y consideremos el producto de matrices usual. El par $\displaystyle \left(\mathcal{M}_{n}, \cdot\right) $ no es un grupo, puesto que las matrices con determinante nulo no tienen inverso. \\ 
	Tomemos así solo las matrices cuyo determinante es distinto de cero, y por tanto sabemos que tienen inverso. A este conjunto lo llamamos \textbf{grupo lineal general}, $\displaystyle \GL_{n}\left(\R\right) = \left\{ A \in \mathcal{M}_{n}\left(\R\right) \; : \; \left|A\right| \neq 0\right\}  $. Así, $\displaystyle \left(\GL_{n}\left(\R\right), \cdot\right) $ forma un grupo. \\
	De manera similar, el conjunto $\displaystyle \SL_{n}\left(\R\right) = \left\{ A \in \mathcal{M}_{n}\left(\R\right) \; : \; \left|A\right| = 1\right\}  $, al que llamamos \textbf{grupo lineal especial}, también forma un grupo con la multiplicación. 
\end{enumerate}
\end{eg}
\begin{observation}
Se puede ver que $\displaystyle \SL_{n}\left(\R\right) \subset \GL_{n}\left(\R\right) $. 
\end{observation}
\begin{definition}[Subgrupo]
Sea $\displaystyle G $ un grupo y $\displaystyle H \subset G $. Diremos que $\displaystyle H $ es \textbf{subgrupo} de $\displaystyle G $, $\displaystyle H \leq G $, si $\displaystyle H $ es cerrado para la operación de $\displaystyle G $, esto es
\begin{itemize}
\item $\displaystyle H \neq \emptyset $.
\item $\displaystyle \forall a,b \in H $, $\displaystyle ab \in H $.
\item $\displaystyle \forall a \in H $, $\displaystyle a^{-1} \in H $.
\end{itemize}
\end{definition}
\begin{eg}
\begin{description}
	\item[(i)] Sea $\displaystyle G $ un grupo. Tenemos que $\displaystyle \left\{ e\right\} \leq G $ es el \textbf{subgrupo trivial} .
	\item[(ii)] $\displaystyle \SL_{n}\left(\R\right) \leq \GL_{n}\left(\R\right) $.
	\item[(iii)] $\displaystyle \Z \leq \Q \leq \R \leq \C $.
	\item[(iv)] $\displaystyle \Q/ \left\{ 0\right\}  \leq \R/ \left\{ 0\right\}  \leq \C / \left\{ 0\right\}  $.
\end{description}
\end{eg}

\begin{prop}
Sea $\displaystyle G $ un grupo y $\displaystyle H \subset G $. Así, $\displaystyle H \leq G $ si y solo si $\displaystyle e \in H $ y $\displaystyle \forall a,b \in H $ se cumple que $\displaystyle ab^{-1} \in H $.
\end{prop}
\begin{proof}
Demostremos la primera implicación. Si $\displaystyle H \leq G $, tenemos que $\displaystyle H \neq \emptyset $ por lo que existe $\displaystyle a \in H $, por lo que $\displaystyle a ^{-1} \in H $ y $\displaystyle e = a a^{-1} \in H $. Ahora, si $\displaystyle a,b \in H $, tenemos que $\displaystyle b^{-1} \in H $, por lo que $\displaystyle a b^{-1} \in H $. \\
Recíprocamente, $\displaystyle H \neq \emptyset $ puesto que $\displaystyle e \in H $. Sea $\displaystyle a \in H $. Tenemos que $\displaystyle a^{-1} = e \cdot a^{-1} \in H $. Falta que si $\displaystyle a,b \in H$, entonces $\displaystyle ab \in H $. Sean $\displaystyle a,b \in H $, entonces $\displaystyle a^{-1}, b^{-1} \in H $. Entonces $\displaystyle ab = a\left(b^{-1}\right)^{-1} \in H $. Así, demostramos las tres propiedades. 
\end{proof}
\begin{eg}[Producto cartesiano de dos grupos]
Sean $\displaystyle \left(G_{1}, \cdot _{G_{1}}, e_{G_{1}}\right) $ y $\displaystyle \left(G_{2}, \cdot _{G_{2}}, e_{G_{2}}\right) $ dos grupos. Vamos a ver que su producto cartesiano también es un grupo. Definimos la siguiente operación para el producto cartesiano:
\[
\begin{split}
	\cdot : (G_{1} \times G_{2}) \times \left(G_{1} \times G_{2}\right) & \to G_{1} \times G_{2} \\
	\left(g_{1}, g_{2}\right) \times \left(g_{1}', g_{2}'\right) & \to \left(g_{1} \cdot _{G_{1}}g_{1}', g_{2} \cdot _{G_{2}}g'_{2}\right).
\end{split}
\]
Está claro que $\displaystyle G = G_{1}\times G_{2} \neq \emptyset $ y que se trata de una operación interna. 
\begin{description}
\item[Asociatividad.] Si $\displaystyle \left(a_{1}, a_{2}\right), \left(b_{1}, b_{2}\right) , \left(c_{1}, c_{2}\right) \in G_{1} \times G_{2}$, tenemos que
	\[
	\begin{split}
		\left(\left(a_{1}, a_{2}\right) \cdot \left(b_{1}, b_{2}\right)\right) \cdot \left(c_{1}, c_{2}\right) & = \left(a_{1} \cdot b_{1}, a_{2} \cdot b_{2}\right) \cdot \left(c_{1},c_{2}\right) = \left(a_{1} \cdot b_{1} \cdot c_{1}, a_{2} \cdot b_{2} \cdot c_{2}\right) \\
		= & \left(a_{1}, a_{2}\right) \left(b_{1} \cdot c_{1}, b_{2} \cdot c_{2}\right) = \left(a_{1}, a_{2}\right) \cdot \left(\left(b_{1}, b_{2}\right) \cdot \left(c_{1}, c_{2}\right)\right).
	\end{split}
	\]
\item[Elemento neutro.] Tenemos que $\displaystyle e = \left(e_{G_{1}}, e_{G_{2}}\right) $. En efecto, si $\displaystyle \left(g_{1}, g_{2}\right) \in G_{1} \times G_{2} $, tenemos que
	\[
	\begin{split}
		\left(e_{G_{1}}, e_{G_{2}}\right) \cdot \left(g_{1}, g_{2}\right) & = \left(g_{1}, g_{2}\right) \\
		\left(g_{1}, g_{2}\right) \cdot \left(e_{G_{1}}, e_{G_{2}}\right) & = \left(g_{1}, g_{2}\right).
	\end{split}
	\]
\item[Inverso.] Si $\displaystyle \left(g_{1}, g_{2}\right) \in G_{1} \times G_{2} $, tenemos que su inverso será $\displaystyle \left(g_{1}^{-1}, g_{2}^{-1}\right) \in G_{1} \times G_{2} $. En efecto,
	\[
	\begin{split}
		\left(g_{1}, g_{2}\right) \cdot \left(g_{1}^{-1}, g_{2}^{-1}\right) & = \left(e_{G_{1}}, e_{G_{2}}\right) \\
		\left(g_{1}^{-1}, g_{2}^{-1}\right) \cdot \left(g_{1}, g_{2}\right) & = \left(e_{G_{1}}, e_{G_{2}}\right).
	\end{split}
	\]
\end{description}
Así, está claro que $\displaystyle G_{1} \times G_{2} $ es un grupo.	
\end{eg}

\begin{definition}
Sea $\displaystyle G $ un grupo. Entonces, 
\begin{description}
\item[(a)] Llamamos \textbf{centro} de $\displaystyle G $ al conjunto 
	\[Z\left(G\right) = \left\{ a \in G \; : \; ax = xa, \forall x \in G\right\}  .\]
\item[(b)] Llamamos \textbf{centralizador} de $\displaystyle x \in G $ al conjunto 
	\[C_{G}\left(x\right) = \left\{ a \in G \; : \; ax = xa\right\}  .\]
\end{description}
\end{definition}
\begin{observation}
Los conjuntos $\displaystyle Z\left(G\right) $ y $\displaystyle C_{G}\left(x\right) $ son subgrupos. En efecto:
\begin{description}
\item[(i)] Tenemos que $\displaystyle e \in Z\left(G\right) $ y si $\displaystyle a \in Z\left(G\right) $, también tenemos que $\displaystyle a^{-1} \in Z\left(G\right) $. En efecto, 
	\[a^{-1}x = xa^{-1} \iff a a^{-1}x = a x a^{-1} \iff x = x a a^{-1} = x e = x .\]
	Así, si $\displaystyle a,b \in Z\left(G\right) $, tenemos que $\displaystyle b^{-1} \in Z\left(G\right) $ y $\displaystyle \forall x \in G $,
	\[ab^{-1}x = axb^{-1} = xab^{-1} .\]
	Por lo que $\displaystyle ab^{-1} \in Z\left(G\right) $ y se trata de un subgrupo.
\item[(ii)] El argumento para demostrar que $\displaystyle C_{G}\left(x\right) $ es un subgrupo de $\displaystyle G $ es análogo al anterior.
\end{description}
\end{observation}

\begin{observation}
Se puede comprobar que $\displaystyle Z\left(G\right) = \bigcap_{x \in G}C_{G}\left(x\right) $. En efecto:
\begin{description}
\item[(i)] Si $\displaystyle x \in Z\left(G\right) $ tenemos que $\displaystyle \forall g \in G $, $\displaystyle x g = g x $, por lo que $\displaystyle \forall g \in G $, $\displaystyle x \in C_{G}\left(g\right) \iff x \in \bigcap_{g \in G}C_{G}\left(g\right) $.
\item[(ii)] Si $\displaystyle x \in \bigcap_{g \in G}C_{G}\left(g\right) $, $\displaystyle x \in C_{G}\left(g\right) $, $\displaystyle \forall g \in G $. Por lo que $\displaystyle xg = gx $, $\displaystyle \forall g \in G $ y $\displaystyle x \in Z\left(G\right) $.
\end{description}
\end{observation}
\begin{definition}[Homomorfismo]
Sean $\displaystyle G_{1} $ y $\displaystyle G_{2} $ grupos tales que $\displaystyle \cdot _{G_{1}} $ y $\displaystyle \cdot _{G_{2}} $ son sus operaciones y $\displaystyle e_{G_{1}} $ y $\displaystyle e_{G_{2}} $ sus elementos neutros. Entonces, $\displaystyle f : G_{1} \to G_{2} $ es un \textbf{homomorfismo} de grupos si $\displaystyle \forall a,b \in G_{1} $, 
\[f\left(a \cdot _{G_{1}} b\right) = f\left(a\right) \cdot _{G_{2}}f\left(b\right) .\]
\end{definition}
\begin{observation}
Si $\displaystyle f_{1}: G_{1} \to G_{2} $ y $\displaystyle f_{2} : G_{2} \to G_{3} $ son homomorfismos de grupos, entonces $\displaystyle f_{2} \circ f_{1} $ es un homomorfismo de grupos. Es decir, la composición de homomorfismos de grupos sigue siendo homomorfismo de grupos. En efecto, si $\displaystyle a,b \in G_{1} $, 
\[f_{2}\circ f_{1}\left(ab\right) = f_{2}\left(f_{1}\left(ab\right)\right) = f_{2}\left(f_{1}\left(a\right)f_{1}\left(b\right)\right) = f_{2}\left(f_{1}\left(a\right)\right)f_{2}\left(f_{1}\left(b\right)\right) = f_{2} \circ f_{1}\left(a\right) f_{2}\circ f_{1}\left(b\right).\]
\end{observation}
\begin{eg}
Consideremos la aplicación 
\[
\begin{split}
	f : \R/ \left\{ 0\right\}  & \to \GL_{n}\left(\R\right) \\
	t & \to \begin{pmatrix} t & 0 & \cdots & 0 \\
	0 & t & \cdots & 0 \\
\vdots & \vdots & \vdots & \vdots \\
0 & 0 & \cdots & t\end{pmatrix} = t \cdot I_{n}.
\end{split}
\]
Está aplicación es un homomorfismo de grupos.
\end{eg}
\begin{definition}
Sea $\displaystyle f : G_{1} \to G_{2} $ homomorfismo de grupos. Entonces:
\begin{description}
\item[(a)] Llamamos \textbf{núcleo} de $\displaystyle f $ al conjunto
	\[\Ker\left(f\right) = \left\{ a \in G_{1} \; : \; f\left(a\right) = e _{G_{2}}\right\}  .\]
\item[(b)] Llamamos \textbf{imagen} de $\displaystyle f $ al conjunto
	\[\Imagen\left(f\right) = \left\{ b \in G_{2} \; : \; \exists a \in G_{1}, f\left(a\right) = b\right\}  .\]
\end{description}
\end{definition}
\begin{prop}
Sea $\displaystyle f : G_{1} \to G_{2} $ un homomorfismo de grupos. Entonces:
\begin{enumerate}
\item $\displaystyle f\left(e_{G_{1}}\right) = e_{G_{2}} $.
\item $\displaystyle \forall a \in G_{1} $, $\displaystyle f\left(a^{-1}\right) = f\left(a\right)^{-1} $.
\item Si $\displaystyle H \leq G_{1} $, entonces $\displaystyle f\left(H\right) \leq G_{2} $. En particular, tenemos que $\displaystyle \Imagen\left(f\right) \leq G_{2} $.
\item $\displaystyle f $ es inyectiva si y solo si $\displaystyle \Ker\left(f\right) = \left\{ e_{G_{1}}\right\}  $.
\item Si $\displaystyle N \leq G_{2} $, entonces $\displaystyle f^{-1}\left(N\right) \leq G_{1} $ que contiene a $\displaystyle \Ker\left(f\right) $.
\end{enumerate}
\end{prop}
\begin{proof}
\begin{enumerate}
\item Sabemos que $\displaystyle e_{G_{1}} = e_{G_{1}} \cdot e_{G_{1}} $, por lo que:
	\[f\left(e_{G_{1}}\right) = f\left(e_{G_{1}} \cdot e_{G_{1}}\right)=f\left(e_{G_{1}}\right)f\left(e_{G_{1}}\right) .\]
Así, tenemos que 
\[ .\]
\[
\begin{split}
	e_{G_{2}}  &= f\left(e_{G_{1}}\right)^{-1}f\left(e_{G_{1}}\right) = f\left(e_{G_{1}}\right)^{-1}\left(f\left(e_{G_{1}}\right)f\left(e_{G_{1}}\right)\right) \\
	= &  \left(f\left(e_{G_{1}}\right)^{-1}f\left(e_{G_{1}}\right)\right)f\left(e_{G_{1}}\right) = e_{G_{2}}f\left(e_{G_{1}}\right) = f\left(e_{G_{1}}\right).
\end{split}
\]
\item Sea $\displaystyle a \in G_{1} $, entonces por la unicidad del inverso y por \textbf{1}:
	\[f\left(a\right) f\left(a^{-1}\right) = f\left(a a^{-1}\right) = f\left(e_{G_{1}}\right) = e_{G_{2}} .\]
\item Si $\displaystyle H \leq G_{1} $, tenemos que $\displaystyle e_{G_{1}} \in H $, por lo que $\displaystyle e_{G_{2}} \in f\left(H\right) $. Además, tenemos que $\displaystyle \forall a,b \in H $ se cumple que $\displaystyle a b^{-1} \in H $. Por tanto, si $\displaystyle x,y \in f\left(H\right) $, $\displaystyle \exists a,b \in H $ tales que $\displaystyle x = f\left(a\right) $ y $\displaystyle y = f\left(b\right) $, 
	de esta manera, tenemos que $\displaystyle ab^{-1} \in H $, por lo que $\displaystyle f\left(ab^{-1}\right) \in f\left(H\right) $. Así, 
	\[xy^{-1} = f\left(a\right)f\left(b\right)^{-1} = f\left(a\right)f\left(b^{-1}\right) = f\left(ab^{-1}\right) \in f\left(H\right) .\]
	Así, queda demostrado que $\displaystyle f\left(H\right) \leq G_{2} $.
\item Si $\displaystyle \Ker\left(f\right) = \left\{ e_{G_{1}}\right\}  $ y $\displaystyle f\left(a\right) = f\left(b\right) $, tenemos que 
	\[f\left(a\right)f\left(b\right)^{-1} = e_{G_{2}} \iff f\left(ab^{-1}\right) = e_{G_{2}} .\]
	Por tanto, $\displaystyle ab^{-1} = e_{G_{1}} $, por lo que $\displaystyle a = b $. Así, hemos visto que $\displaystyle f $ es inyectiva. Supongamos que $\displaystyle f $ es inyectiva y que $\displaystyle a \in \Ker\left(f\right) $. Entonces, tenemos que $\displaystyle f\left(a\right)=f\left(e_{G_{1}}\right) = e_{G_{2}} $, por lo que $\displaystyle a = e_{G_{1}} $ y $\displaystyle \Ker\left(f\right) = \left\{ e_{G_{1}}\right\}  $.
\item Supongamos que $\displaystyle N \leq G_{2} $. Tenemos que $\displaystyle e_{G_{2}} \in N $, por lo que $\displaystyle e_{G_{1}} \in f^{-1}\left(N\right) $. Si $\displaystyle x,y \in f^{-1}\left(N\right) $ tenemos que $\displaystyle f\left(x\right), f\left(y\right) \in N $, así, 
	\[f\left(xy^{-1}\right)= f\left(x\right)f\left(y^{-1}\right) = f\left(x\right) f\left(y\right)^{-1} \in N .\]
	Por tanto, $\displaystyle \forall x,y \in f^{-1}\left(N\right) $, tenemos que $\displaystyle xy^{-1} \in f^{-1}\left(N\right) $, por lo que $\displaystyle f^{-1}\left(N\right) \leq G_{1} $. Ahora, si $\displaystyle x \in \Ker\left(f\right) $, tenemos que $\displaystyle f\left(x\right) = e_{G_{2}} \in N $, por lo que $\displaystyle x \in f^{-1}\left(N\right) $ y consecuentemente $\displaystyle \Ker\left(f\right) \leq f^{-1}\left(N\right) $.
\end{enumerate}
\end{proof}
\begin{eg}
\begin{enumerate}
	\item Consideremos $\displaystyle f_{m} : \Z \to \Z $ con $\displaystyle m \in \Z $, con la suma, tal que $\displaystyle f\left(z\right) = mz $. Tenemos que $\displaystyle f_{m} $ es un homomorfismo de grupos. Por proposición anterior, tenemos que 
		\[ m\Z := f\left(\Z\right)= \left\{ z \in \Z \; : \; z = km, k \in \Z\right\}  \leq \Z .\]
	Similarmente, tenemos que $\displaystyle \Ker\left(f_{m}\right) $ es el subgrupo trivial si $\displaystyle m \neq 0 $ y es $\displaystyle \Z $ si $\displaystyle m = 0 $.	
\item Es homomorfismo la aplicación $\displaystyle \det : \GL_{n}\left(\R\right) \to \R / \left\{ 0\right\}  : M \to \det\left(M\right) $. En concreto, se trata de un homomorfismo sobreyectivo. Además, podemos ver que $\displaystyle \Ker\left(\det \right) = \SL_{n}\left(\R\right) $.
\end{enumerate}
\end{eg}
\begin{definition}[Isomorfismo y automorfismo]
Sea $\displaystyle f : G_{1} \to G_{2} $ un homomorfismo de grupos. Si $\displaystyle f $ es biyectiva, entonces $\displaystyle f $ es un \textbf{isomorfismo} y lo escribimos $\displaystyle G_{1} \cong G_{2} $. Si $\displaystyle f : G_{1} \to G_{1} $ es un isomorfismo, se llama \textbf{automorfismo}.
\end{definition}
\begin{observation}
\begin{enumerate}
\item Si $\displaystyle G_{1} \cong G_{2} $ tenemos que $\displaystyle \left|G_{1}\right| = \left|G_{2}\right| $ y tienen la misma tabla de Cayley. 
\item Si $\displaystyle f: G_{1} \to G_{2} $ es un isomorfismo, tenemos que $\displaystyle f^{-1} : G_{2} \to G_{1} $ también lo es. En efecto, Si $\displaystyle x,y \in G_{2} $ existen $\displaystyle a,b \in G_{1} $ tales que $\displaystyle x = f\left(a\right) $ e $\displaystyle y = f\left(b\right) $. Así, 
	\[f^{-1}\left(xy\right) = f^{-1}\left(f\left(a\right)f\left(b\right)\right) = f^{-1}\left(f\left(ab\right)\right) = ab = f^{-1}\left(x\right)f^{-1}\left(y\right).\]
	
\item Si $\displaystyle f: G_{1} \to G_{2} $ es un homomorfismo sobreyectivo, tenemos que $\displaystyle f\left(G_{1}\right) \cong G_{2} $, es decir, $\displaystyle \Imagen\left(f\right)\cong G_{2} $.
\item Si $\displaystyle f: G_{1} \to G_{2} $ es un homomorfismo inyectivo, entonces $\displaystyle G_{1} \cong \Imagen\left(f\right) $.
\item La relación de ser isomorfo es una relación de equivalencia.
\item El conjunto de automorfismos de $\displaystyle G $, $\displaystyle \Aut\left(G\right) $, es un subgrupo de $\displaystyle \Biy\left(G\right) $.
\end{enumerate}
\end{observation}
\section{Grupos cíclicos}
\begin{notation}
Sea $\displaystyle \left(G, \cdot\right) $ un grupo, $\displaystyle a \in G $ y $\displaystyle k \in \Z $. Entonces utilizaremos la siguiente notación:
\[a^{0} = e, \quad a^{n} = \underbrace{a \cdot a \cdots a}_{n\; \text{veces}}, \quad a^{-n} = \underbrace{a^{-1} \cdot a^{-1} \cdots a^{-1}}_{n\; \text{veces}}  .\]
\end{notation}
\begin{lema}
Sea $\displaystyle \left(G, \cdot \right) $ un grupo, $\displaystyle a \in G $ y $\displaystyle k, l \in \Z $. Entonces $\displaystyle a^{l + k} = a^{l}a^{k} $ y $\displaystyle \left(a^{-1}\right)^{k}=a^{-k}=\left(a^{k}\right)^{-1} $.
\end{lema}
\begin{proof}
Está claro que, por la propiedad asociativa, si $\displaystyle l,k \in \N $ (o $\displaystyle l, k \leq 0 $, se procede igual):
\[ a^{l + k} = \underbrace{a \cdot a \cdots a}_{l + k \; \text{veces}} = \underbrace{a \cdot a \cdots a }_{l \; \text{veces}} \cdot \underbrace{a \cdot a \cdots a}_{k \; \text{veces}}=a^{l}a^{k} .\]
Sin pérdida de generalidad, supongamos que $\displaystyle l \leq 0 $ y $\displaystyle k > 0 $. Entonces, es evidente que
\[a^{l}a^{k} = a \cdots a \cdot a^{-1} \cdots a^{-1} = a^{l - k} .\]
Por otro lado, tenemos que
\[\left(a^{-1}\right)^{k}a^{k} = \left(a^{-1} \cdots a^{-1}\right) \cdot \left(a \cdots a\right) = a^{-1} \cdots a^{-1} \cdot \left(a^{-1} \cdot a\right) \cdot a \cdots a = e .\]
Al haber el mismo número de $\displaystyle a^{-1} $ que de $\displaystyle a $, está claro que el resultado será el elemento neutro. Por la unicidad del inverso, tenemos que $\displaystyle \left(a^{k}\right)^{-1} = \left(a^{-1}\right)^{k} $.
\end{proof}
\begin{notation}
Dado un grupo $\displaystyle \left(G, \cdot \right) $ y $\displaystyle a \in G $, utilizaremos la siguiente notación:
\[\left\langle a \right\rangle = \left\{ a^{k} \; : \; k \in \Z\right\}  .\]
\end{notation}
\begin{prop}
Si $\displaystyle G $ es un grupo y $\displaystyle a \in G $, se tiene que $\displaystyle \left\langle a \right\rangle \leq G $ y $\displaystyle \left\langle a \right\rangle  $ es abeliano.
\end{prop}
\begin{proof}
Dado que $\displaystyle G $ es un grupo, su operación es cerrada, por lo que $\displaystyle \left\langle a \right\rangle \subset G $. Tenemos que $\displaystyle e \in \left\langle a \right\rangle  $. Por otro lado, si $\displaystyle x,y \in \left\langle a \right\rangle  $, existen $\displaystyle n,m \in \Z $ tales que $\displaystyle x = a^{n} $ e $\displaystyle y = a^{m} $.
Así, tenemos que $\displaystyle y^{-1} = a^{-m} $, así, $\displaystyle xy^{-1} = a^{n}a^{-m} = a^{n - m} \in \left\langle a \right\rangle  $, puesto que $\displaystyle n - m \in \Z $. Además, es abeliano, puesto que 
\[ x y = a^{n}a^{m} = a^{n + m} = a^{m + n} = a^{m}a^{n} = yx.\]
\end{proof}
\begin{notation}
Si la operación del grupo fuera aditiva, en lugar de $\displaystyle a^{k} $ escribiríamos $\displaystyle ka $.
\end{notation}
\begin{observation}
Está claro que $\displaystyle \left\langle a \right\rangle  = \left\langle a^{-1} \right\rangle  $. En efecto, 
\[x \in \left\langle a \right\rangle \iff x = a^{n}, n \in \Z \iff x = \left(a^{-1}\right)^{-n}, n \in \Z \iff x \in \left\langle a^{-1} \right\rangle  .\]
\end{observation}
\begin{definition}[Grupo cíclico]
Un grupo $\displaystyle G $ es \textbf{cíclico} si existe $\displaystyle a \in G $ tal que $\displaystyle G = \left\langle a \right\rangle  $. Decimos que $\displaystyle a $ es \textbf{generador} de $\displaystyle G $ o que $\displaystyle G $ \textbf{está generado} por $\displaystyle a $.
\end{definition}
\begin{eg}
Consideremos el grupo $\displaystyle \left(\Z, +\right) $. Tenemos que este grupo es cíclico y tiene dos generadores, $\displaystyle 1 $ y $\displaystyle -1 $. En efecto, se cumple que $\displaystyle \Z = \left\langle 1 \right\rangle  = \left\langle -1 \right\rangle  $.
\end{eg}
\begin{prop}
Si $\displaystyle G $ es un grupo cíclico, cualquier subgrupo $\displaystyle H \leq G $ también es cíclico.
\end{prop}
\begin{proof}
	Supongamos que $\displaystyle H \neq \left\{ e\right\}  $ y $\displaystyle H \neq G $, puesto que estos casos son triviales. Sea $\displaystyle k \in \N $ el más pequeño tal que $\displaystyle a^{k} \in H $. Podemos observar que dado que $\displaystyle H \leq G $, tenemos que $\displaystyle a^{-k} \in H $. Vamos a ver que $\displaystyle H = \left\langle a^{k} \right\rangle  $.
	\begin{description}
	\item[(i)] Si $\displaystyle x \in H $, tenemos que existe $\displaystyle l \in \Z $ tal que $\displaystyle x = a^{l} $. Por el algoritmo de la división, tenemos que existen $\displaystyle q, r \in \Z $ tales que 
		\[l = qk + r, \quad 0 \leq r < k .\]
	Entonces, tenemos que 
	\[a^{l} = a^{qk + r} = \left(a^{k}\right)^{q}a^{r} .\]
	Dado que $\displaystyle a^{l}, \left(a^{k}\right)^{q} \in H $, debe ser que $\displaystyle a^{r} \in H $. Como $\displaystyle k \in \N $ era el menor tal que $\displaystyle a^{k} \in H $ y $\displaystyle r < k $, debe ser que $\displaystyle r = 0 $, por lo que $\displaystyle x = a^{l} = \left(a^{k}\right)^{q} \in H $. Así, hemos visto que $\displaystyle H \leq \left\langle a^{k} \right\rangle  $.
\item[(ii)] Por otro lado, si $\displaystyle x \in \left\langle a^{k} \right\rangle  $, tenemos que existe $\displaystyle n \in \Z $ tal que $\displaystyle x = \left(a^{k}\right)^{l} \in H $. Así, tenemos que $\displaystyle \left\langle a^{k} \right\rangle \subset H $.  
	\end{description}
	Así, hemos visto que $\displaystyle H = \left\langle a^{k} \right\rangle  $, por lo que es cíclico.
\end{proof}
\begin{colorary}
Todo $\displaystyle H \leq \Z $ es un subgrupo cíclico, es decir, existe $\displaystyle m \in \Z $ tal que $\displaystyle H = \left\langle m \right\rangle  $.
\end{colorary}
\begin{proof}
Se deduce fácilmente a partir de la proposición y de la observación anterior.
\end{proof}
\begin{eg}
\begin{enumerate}
	\item El conjunto $\displaystyle U_{n} = \left\{ z \in \C \; : \; z^{2} = 1\right\}  $, de las raíces $\displaystyle n $-ésimas de la unidad, es un grupo cíclico con la multiplicación. Recordamos que $\displaystyle w_{k} = e^{i\frac{2\pi k}{n}} $, para $\displaystyle k = 0, \ldots, n-1 $. Es sencillo ver que $\displaystyle \left(U_{n}, \cdot, 1\right) \leq \left(\C / \left\{ 0\right\} , 1\right) $. En efecto, 
		\[ e^{i\frac{2\pi \cdot 0}{n} } = e^{0} = 1 .\]
		Ahora, si $\displaystyle w_{1}, w_{2} \in U_{n} $, tenemos que si $\displaystyle k_{1} > k_{2} $:
		\[w_{1}w_{2}^{-1} = e^{i\frac{2\pi k_{1}}{n}}e^{i\frac{2\pi \left(-k_{2}\right)}{n}} = e^{i\frac{e\pi\left(k_{1}-k_{2}\right)}{n }}\in U_{n}.\]
		Así, está claro que $\displaystyle \left(U_{n}, \cdot , 1\right) \leq \left(\C / \left\{ 0\right\} \; \cdot , 1\right) $. Para ver que es cíclico basta con ver que $\displaystyle U_{n} = \left\langle e^{i\frac{2\pi }{n}} \right\rangle  $. 
	\item En $\displaystyle \Z $, tenemos que $\displaystyle \forall m \in \Z $, $\displaystyle m\Z \leq \Z $. Sabemos que $\displaystyle \Z / m\Z \cong \Z_{m} = \left\{ \left[0\right] , [1], \ldots, [m -1]\right\}  $. Podemos definir la operación:
	\[
	\begin{split}
		+ :	\Z_{m} \times \Z_{m} & \to \Z_{m} \\
		\left([a]_{m}, [b]_{m}\right) & \to [a + b]_{m}.
	\end{split}
	\]
	Vamos a ver que está operación está bien definida. Si $\displaystyle x \in [a]_{m} $ e $\displaystyle y \in [b]_{m} $, tenemos que 
	\[m | x - a \quad \text{y} \quad m | y - b .\]
	Así, existen $\displaystyle \lambda, \mu \in \Z $ tales que $\displaystyle x = a + \lambda m $ e $\displaystyle y = b + \mu m $. Por tanto, obtenemos que
	\[ x + y = a + \lambda m + b + \mu m = \left(a+b\right) + \left(\lambda + \mu\right)m \iff x + y \equiv a + b \mod m \iff [x +y]_{m} = [a + b]_{m} .\]
	Queremos ver ahora que $\displaystyle \left(Z_{m}, +, [0]_{m}\right) $ es un grupo. Está claro que $\displaystyle \Z_{m} \neq \emptyset $ y que el elemento neutro es $\displaystyle [0]_{m} $. Ahora comprobamos que hay inversos. Si $\displaystyle [a]_{m} \in \Z_{m} $, tenemos que $\displaystyle [-a]_{m} \in \Z_{m} $ y, por definición, $\displaystyle [a]_{m} + [-a]_{m} = [0]_{m} $. También se puede ver que $\displaystyle \Z_{m} $ es cíclico, es decir, que $\displaystyle \Z_{m} = \left\langle [1]_{m} \right\rangle  $.
\end{enumerate}
\end{eg}
\begin{lema}
	Sea $\displaystyle G $ un grupo cíclico, por lo que $\displaystyle G = \left\langle a \right\rangle  $. Entonces si $\displaystyle a^{k} \neq e $, $\displaystyle \forall k \in \N $, tenemos que $\displaystyle G $ tiene orden infinito. En caso contrario, si $\displaystyle m = \min \left\{ k \in \N \; : \; a^{k} = e\right\}  $ tenemos que $\displaystyle G = \left\langle a \right\rangle = \left\{ e, a, \ldots, a^{m-1}\right\}  $. Además, $\displaystyle a^{k} = e $ si y solo si $\displaystyle m | k $.
\end{lema}
\begin{proof}
\begin{description}
	\item[(i)] Sea $\displaystyle a^{k} \neq e $, $\displaystyle \forall k \in \N $. Entonces, $\displaystyle a^{k} \neq e $, $\displaystyle \forall \Z / \left\{0 \right\}  $, por lo que el orden de $\displaystyle G $ es infinito. En efecto, si existieran $\displaystyle i, j \in \Z $ distintos tales que $\displaystyle a^{i} = a^{j} $, tendríamos que $\displaystyle a^{i-j} = e $, lo que es una contradicción. 
	\item[(ii)] Por otro lado, sea $\displaystyle m = \min \left\{ k \in \N \; : \; a^{k} = e\right\}  $. Vamos a ver que $\displaystyle G = \left\langle a \right\rangle = \left\{ e, a, \ldots, a^{m - 1}\right\}  $. Es trivial que $\displaystyle \left\{ e, a, \ldots, a^{m - 1}\right\} \subset G $. Recíprocamente, si $\displaystyle g \in G $, tenemos que existe $\displaystyle l \in \Z / \left\{ 0\right\}  $ tal que $\displaystyle g = a^{l} $. Por el algoritmo de la división, tenemos que existen $\displaystyle q, r \in \Z $ tales que
		\[l = mq + r, \; 0 \leq r < m .\]
		Así, tenemos que 
		\[a^{l} = a^{mq + r}= \left(a^{m}\right)^{q}a^{r} = a^{r}.\]
		Así, como $\displaystyle 0 \leq r < m $, debe ser que $\displaystyle g \in \left\{ e, a, \ldots, a^{m - 1}\right\}  $, por lo que $\displaystyle G \subset \left\{ e, a, \ldots, m - 1\right\}  $. Consecuentemente, $\displaystyle G = \left\{ e, a, \ldots, a^{m - 1}\right\}  $. \\
		Finalmente, como $\displaystyle l = qm+r $, es trivial que $\displaystyle a^{l} = e \iff r = 0 $.
\end{description}
\end{proof}
\begin{observation}
	En el lema podemos ver que $\displaystyle m = \min \left\{ k \in \N\; : \; a^{k} = e\right\}  $ es también el orden de $\displaystyle G $.
\end{observation}

\begin{prop}
Dos grupos $\displaystyle G $ y $\displaystyle H $ cícliclos del mismo orden son isomorfos.
\end{prop}
\begin{proof}
Sea $\displaystyle G = \left\langle a \right\rangle  $ y $\displaystyle H = \left\langle b \right\rangle  $. Consideremos la aplicación
\[
\begin{split}
	f : G & \to H \\
	a^{k} & \to b^{k}.
\end{split}
\]
Vamos a ver que se trata de un homomorfismo de grupos:
\[f\left(a^{k}\right)f\left(a^{t}\right) = b^{k}b^{t} = b^{k+t} = f\left(a^{k+t}\right) = f\left(a^{k}a^{t}\right) .\]
Ahora vamos a ver que es biyectiva. 
\begin{description}
\item[Inyectiva.] Si $\displaystyle \left|G\right| > k \geq t $ y $\displaystyle f\left(a^{k}\right) = f\left(a^{t}\right) $, tenemos que $\displaystyle f\left(a^{k-t}\right) = b^{k -t} = e $. Como $\displaystyle \left|G\right|>k - t \geq 0 $, debe ser que $\displaystyle k - t = 0 $, por lo que $\displaystyle a^{k} = a^{t} $.
\item[Sobreyectiva.]  Si $\displaystyle c \in H $ con $\displaystyle c = b^{k} $ para algún $\displaystyle k = 0, \ldots, \left|H\right|-1 $, tenemos que $\displaystyle f\left(a^{k}\right)= b^{k} = c $.
\end{description}
Así, está claro que $\displaystyle f $ es un isomorfismo.
\end{proof}

