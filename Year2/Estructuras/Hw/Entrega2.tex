\documentclass{article}

% packages

\usepackage{graphicx} % Required for images
\usepackage[spanish]{babel}
\usepackage{mdframed}
\usepackage{amsthm}
\usepackage{amssymb}
\usepackage{fancyhdr}
\usepackage{amsmath}
\usepackage{geometry}[margin=1in]
\usepackage{pgfplots}
\usepackage{url}
\usepackage{float}

% for math environments

\theoremstyle{definition}
\newtheorem*{theorem}{Teorema}
\newtheorem*{definition}{Definición}
\newtheorem*{prop}{Proposición}
\newtheorem*{observation}{Observación}
\newtheorem{ej}{Ejercicio}
\newtheorem{sol}{Solución}

% for headers and footers

\pagestyle{fancy}

%\fancyhead[R]{Victoria Eugenia Torroja}
% Store the title in a custom command
\newcommand{\mytitle}{}

% Redefine \title to store the title in \mytitle
\let\oldtitle\title
\renewcommand{\title}[1]{\oldtitle{#1}\renewcommand{\mytitle}{#1}}

% Set the center header to the title
\lhead{\mytitle}

% Custom commands

\newcommand{\R}{\mathbb{R}}
\newcommand{\C}{\mathbb{C}}
\newcommand{\F}{\mathbb{F}}
\newcommand{\N}{\mathbb{N}}
\newcommand{\Q}{\mathbb{Q}}
\newcommand{\Z}{\mathbb{Z}}
\newcommand{\K}{\mathbb{K}}
\newcommand{\mcd}{\text{mcd}}
\newcommand{\mcm}{\text{mcm}}
\DeclareMathOperator{\Ker}{Ker}
\DeclareMathOperator{\Imagen}{Im}
\DeclareMathOperator{\ord}{ord}
\DeclareMathOperator{\GL}{GL}
\DeclareMathOperator{\Biy}{Biy}


\begin{document}

\title{Estructuras Algebraicas - Entrega 2}
\author{Victoria Eugenia Torroja Rubio}
\date{27/10/2025}

\maketitle

\begin{ej}
Demostrar que $\displaystyle U\left(\Z_{p}\right) = \Z_{p}^{*} $ si y solo si $\displaystyle p \geq 2 $ es primo. Deducir que $\displaystyle \Z^{*}_{p} $ es un grupo si y solo si $\displaystyle p \geq 2 $ es primo.
\end{ej}

\begin{ej}
Sea $\displaystyle G = D_{6} $. Encuentra una serie normal
\[ \left\{ e\right\} \lhd H_{1} \lhd H_{2} \lhd G \]
tal que cada cociente $\displaystyle H_{i+1}/H_{i} $ sea abeliano.
\end{ej}

\begin{ej}
Sea 
\[H = \left\{ \begin{pmatrix} 1 & a & c \\ 0 & 1 & b \\ 0 & 0 & 1 \end{pmatrix} \; : \; a,b,c \in \Z\right\}  \]
con la multiplicación matricial usual.
\begin{description}
\item[(a)] Muestra que $\displaystyle H $ es finitamente generado.
\item[(b)] Da un conjunto de generadores mínimos.
\item[(c)] ¿Es abeliano?
\end{description}
\end{ej}

\begin{sol}
\begin{description}
\item[(c)] El grupo $\displaystyle H $ no es abeliano. En efecto, sean 
	\[h_{1} = \begin{pmatrix} 1 & -4 & -4 \\ 0 & 1 & 3 \\ 0 & 0 & 1 \end{pmatrix}, \quad h_{2} = \begin{pmatrix} 1 & -11 & -13 \\ 0 & 1 & 5 \\ 0 & 0 & 1 \end{pmatrix} .\]
	Tenemos que 
	\[h_{1} \cdot h_{2} = \begin{pmatrix} 1 & -15 & -37 \\ 0 & 1 & 8 \\ 0 & 0 & 1 \end{pmatrix} \neq \begin{pmatrix} 1 & -15 & -50 \\ 0 & 1 & 8 \\ 0 & 0 & 1 \end{pmatrix} = h_{2} \cdot h_{1} .\]
\end{description}
\end{sol}

\begin{ej}
Demuestre o refute cada una de las siguientes proposiciones.
\begin{description}
\item[(a)] Todos los generadores de $\displaystyle \Z_{60} $ son primos.
\item[(b)] $\displaystyle U_{8} $ es cíclico.
\item[(c)] $\displaystyle \Q $ es cíclico.
\item[(d)] Si todo subgrupo propio de un grupo $\displaystyle G $ es cíclico, entonces $\displaystyle G $ es un grupo cíclico.
\item[(e)] Un grupo con un número finito de subgrupos es finito.
\end{description}
\end{ej}

\begin{sol}
\begin{description}
\item[(a)] No es cierto, puesto que $\displaystyle \Z_{60} = \left\langle 49 \right\rangle  $. En efecto, tenemos que 
	\[60 | 49k \iff 60 | k .\]
	Por tanto, tenemos que $\displaystyle o\left(14\right) = 60 $ y en consecuencia $\displaystyle \Z_{60} = \left\langle 49 \right\rangle  $.
\item[(b)] Es cierto. 
\item[(c)] No es cierto que $\displaystyle \Q $ es cíclico. En efecto, está claro que $\displaystyle \Q \neq \left\langle 0 \right\rangle  $. Ahora, supongamos que $\displaystyle \Q = \left\langle x \right\rangle  $ con $\displaystyle x = \frac{a}{b} $, $\displaystyle a ,b \in \Z^{*} $. Podemos encontrar $\displaystyle m \in \Z $ tal que $\displaystyle \mcd\left(m,b\right) = 1 $, de esta forma tenemos que $\displaystyle \frac{1}{m} \in \Q $ pero $\displaystyle \frac{1}{m} \in \left\langle x \right\rangle  $. En efecto, si $\displaystyle \frac{1}{m} \in \left\langle x \right\rangle  $ tendríamos que
	\[\frac{1}{m} = \frac{a}{b} \iff b = ma .\]
	Esto último es una contradicción puesto que $\displaystyle \mcd\left(b,m\right) = 1 $. Así, tenemos que no puede ser que $\displaystyle \Q = \left\langle x \right\rangle  $, $\displaystyle \forall x \in \Q $, por lo que $\displaystyle \Q $ no es cíclico. 
\item[(d)] Supongamos que $\displaystyle \forall H < G $, $\displaystyle H $ es cíclico y $\displaystyle G $ no es cíclico. Así, como $\displaystyle G $ no es cíclico tenemos que $\displaystyle \forall g \in G $, $\displaystyle G \neq \left\langle g \right\rangle  $. Así, tenemos que 
\end{description}

\end{sol}

\begin{ej}
Sea $\displaystyle G = \left\langle R, S/R^{4}=S^{4}=\left(RS\right)^{2} = \left(R^{-1}S\right)^{2} = 1 \right\rangle  $ un grupo finito.
\begin{description}
\item[(a)] ¿Qué orden tiene el grupo $\displaystyle G $?
\item[(b)] ¿Cuál es el exponente de $\displaystyle G $?
\item[(c)] ¿Qué orden tiene el elemento $\displaystyle R^{2}SRS^{3}RS^{2} $?
\end{description}
\end{ej}

\end{document}
