\chapter{Cálculo diferencial}
Consideremos, en primer lugar, funciones de la forma $\displaystyle f : \R \to \R^{m} $, que podemos considerar curvas paramétricas.
\section{Caso $\displaystyle f: \R \to \R^{m} $}
\begin{definition}[Curva paramétrica]
Una \textbf{curva} en $\displaystyle \R^{m} $ es una función continua $\displaystyle \sigma : I \to \R^{m} $, donde $\displaystyle I $ es un intervalo de $\displaystyle \R $. 
\end{definition}
\begin{definition}[Derivabilidad]
Se dice que $\displaystyle \sigma  $ es \textbf{derivable} en $\displaystyle t_{0} \in I $ cuando existe $\displaystyle  $ 
\[ \sigma'\left(t_{0}\right) =\lim_{t \to t_{0}} \frac{\sigma\left(t\right)-\sigma\left(t_{0}\right)}{t -t_{0}} = \lim_{h \to 0}\frac{\sigma\left(t_{0}+h\right)-\sigma\left(t_{0}\right)}{h}.\]
Si $\displaystyle \sigma'\left(t_{0}\right) \neq 0 $, se define la \textbf{recta tangente} a $\displaystyle \sigma  $ en $\displaystyle t_{0} $ como la recta que pasa por $\displaystyle \sigma\left(t_{0}\right) $ con vector director $\displaystyle \sigma'\left(t_{0}\right) $.
\end{definition}
\begin{observation}
Si denotamos $\displaystyle \sigma\left(t\right) = \left(\sigma_{1}\left(t\right), \ldots, \sigma_{m}\left(t\right)\right) $, entonces $\displaystyle \sigma  $ es derivable en $\displaystyle t_{0} $ si y solo si $\displaystyle \forall j = 1, \ldots, m $, $\displaystyle \sigma_{j} $ es derivable en $\displaystyle t_{0} $. Entonces tendremos que $\displaystyle \sigma'\left(t_{0}\right) = \left(\sigma_{1}'\left(t_{0}\right), \ldots, \sigma'_{m}\left(t_{0}\right)\right) $. Esto es consecuencia de que en $\displaystyle \R^{m} $ los límites se hacen coordenada a coordenada.
\end{observation}
\begin{eg}
\begin{enumerate}
	\item Consideremos $\displaystyle \sigma : \R \to \R^{2} $ con $\displaystyle \sigma\left(t\right) = \left(\cos t, \sin t\right) $. Tenemos que $\displaystyle \Imagen\left(\sigma \right) = \left\{ \left(x,y\right) \; : \; x^{2} +y^{2} = 1\right\}  $. Por la observación anterior tenemos que
		\[\sigma'\left(t\right) = \left(-\sin t, \cos t\right) .\]
	\item Si consideramos $\displaystyle \gamma\left(t\right) = \left(\cos t, - \sin t\right) $, tenemos que $\displaystyle \Imagen\left(\gamma \right) = \Imagen\left(\sigma \right) $, pero como 
		\[\gamma ' \left(t\right) = \left(-\sin t, - \cos t\right) .\]
		Por lo que los vectores tangentes recorren la curva en sentido contrario.
	\item Consideremos $\displaystyle \beta\left(t\right) = \left(\cos\left(2t\right), \sin\left(2t\right)\right) $. Nuevamente, $\displaystyle \Imagen\left(\beta \right) = \Imagen\left(\sigma \right) $ pero 
		\[\beta'\left(t\right) = \left(-2\sin 2t, 2\cos2t\right) = 2\left(-\sin2t, \cos2t\right) .\]
		Por tanto, podemos interpretar que los vectores tangentes de $\displaystyle \beta  $ van el doble de rápido que los de $\displaystyle \sigma  $.
\end{enumerate}
\end{eg}
\begin{theorem}
Sean $\displaystyle I \subset \R $ un intervalo abierto y $\displaystyle \sigma : I \to \R^{m} $ una curva. Son equivalentes para $\displaystyle t_{0} \in I $
\begin{enumerate}
\item $\displaystyle \sigma  $ es derivable en $\displaystyle t_{0} $.
\item Existe $\displaystyle L : \R \to \R^{m} $ lineal tal que 
	\[\lim_{h \to 0}\frac{\sigma\left(t_{0}+h\right)-\sigma\left(t_{0}\right)-L\left(h\right)}{ \left|h\right|} = 0 .\]
\end{enumerate}
\end{theorem}
\begin{proof}
\begin{description}
\item[(i)] Supongamos que existe $\displaystyle \sigma'\left(t_{0}\right) \in \R^{m} $. Definimos $\displaystyle L : \R \to \R^{m} $ tal que $\displaystyle L\left(h\right) = h\sigma'\left(t_{0}\right) $, $\displaystyle h \in \R $. Sabemos que
	\[
	\begin{split}
		0 = & \lim_{h \to 0}\left(\frac{\sigma \left(t_{0}+h\right)-\sigma\left(t_{0}\right)}{h}-\sigma'\left(t_{0}\right)\right) = \lim_{h \to 0}\frac{\sigma\left(t_{0}+h\right)-\sigma\left(t_{0}\right)-L\left(h\right)}{h} \\
	\iff & \lim_{h \to 0}\left\|\frac{\sigma\left(t_{0}+h\right)-\sigma\left(t_{0}\right)-L\left(h\right)}{h}\right\| = 0 \iff \lim_{h \to 0} \frac{\sigma\left(t_{0}+h\right)-\sigma(t_{0}) - L\left(h\right)}{ \left|h\right|} = 0 \in \R^{m}.
	\end{split}
	\]
\item[(ii)] Si tenemos $\displaystyle L : \R \to \R^{m} $ lineal, definimos $\displaystyle w = L\left(1\right) \in \R^{m} $. Entonces, tenemos que $\displaystyle L\left(h\right) = L\left(h \cdot 1\right) = hL\left(1\right) $. Veamos que $\displaystyle w = \sigma'\left(t_{0}\right) $. Sabemos que
	\[
	\begin{split}
		0 = & \lim_{h \to 0}\frac{\sigma\left(t_{0}+h\right)-\sigma\left(t_{0}\right)-hw}{ \left|h\right|} \\
	\iff & 0 = \lim_{h \to 0}\frac{\sigma\left(t_{0}+h\right)-\sigma\left(t_{0}\right)-hw}{h} =\lim_{h \to 0}\left(\frac{\sigma\left(t_{0}+h\right)-\sigma\left(t_{0}\right)}{h}-w\right) \\
	\Rightarrow & w = \lim_{h \to 0}\frac{\sigma\left(t_{0}+h\right)-\sigma\left(t_{0}\right)}{h} .
	\end{split}
	\]
\end{description}
\end{proof}
\section{Derivadas parciales y direccionales}
\begin{eg}
Consideremos $\displaystyle f\left(x,y\right) = \sin\left(x^{2}-y^{2}+3xy\right) $. Tenemos que
\[\frac{\partial f}{\partial x}\left(x,y\right) = \left(2x+3y\right)\cos\left(x^{2}-y^{2}+3xy\right) .\]
\[\frac{\partial f}{\partial y}\left(x,y\right) = \left(-2y + 3x\right)\cos\left(x^{2}-y^{2}+3xy\right) .\]
\end{eg}
Dado $\displaystyle f : \R^{2} \to \R^{2} $, podemos definir las derivadas parciales de la siguiente forma
\[
\begin{split}
\frac{\partial f}{\partial x}\left(x,y\right) = \lim_{t \to 0}\frac{f\left(x_{0}+ t, y_{0}\right)-f\left(x_{0}, y_{0}\right)}{t} = \lim_{h \to 0}\frac{f\left(\left(x_{0},y_{0}\right)+t\left(1,0\right)\right)-f\left(x_{0}, y_{0}\right)}{t} .
\end{split}
\]

\begin{definition}[Derivadas parciales]
	Sea $\displaystyle U \subset \R^{n} $ abierto, $\displaystyle f : U \to \R^{m} $ y $\displaystyle a \in U $. Se define $\displaystyle \forall i = 1, \ldots, n $, la \textbf{derivada parcial $\displaystyle i $-ésima} de $\displaystyle f $ en $\displaystyle a $ como el límite, cuando existe,
	\[\frac{\partial f}{\partial x_{i}}\left(a\right) = \lim_{t \to 0}\frac{f\left(a+t e_{i}\right)-f\left(a\right)}{t} \in \R^{m} ,\]
	donde $\displaystyle \left\{ e_{1}, \ldots, e_{n}\right\}  $ es la base canónica de $\displaystyle \R^{n} $.
\end{definition}
\begin{observation}
Otra forma de escribir la definición anterior es:
\[\frac{\partial f}{\partial x_{i}}\left(a\right) = \lim_{t \to 0}\frac{f\left(a_{1}, \ldots, a_{i}+t, \ldots,a_{n}\right)-f\left(a_{1}, \ldots, a_{n}\right)}{t} .\]
\end{observation}
\begin{definition}[Derivada direccional]
Sea $\displaystyle U \subset \R^{n} $ abierto, $\displaystyle f : U \to \R^{m} $ y $\displaystyle a \in U $. Dado $\displaystyle w \in \R^{n} $, se define la \textbf{derivada direccional} de $\displaystyle f $ en $\displaystyle a $ según el vector $\displaystyle w $ al límite, si existe
\[D_{w}f\left(a\right) = \lim_{t \to 0}\frac{f\left(a+tw\right)-f\left(a\right)}{t} \in \R^{m} .\]
\end{definition}
\begin{observation}
Es fácil ver que $\displaystyle D_{e_{i}}f\left(a\right) = \frac{\partial f}{\partial x_{i}}\left(a\right) $. 
\end{observation}
\begin{observation}
Podemos deducir que $\displaystyle D_{w}f\left(a\right) = \frac{d}{dt}|_{t = 0}f\left(a+tw\right) $. En efecto, si tomamos $\displaystyle \varphi : \R \to \R^{m} : t \to f\left(a + tw\right) $, tenemos que
\[\varphi'\left(0\right) = \lim_{h \to 0}\frac{\varphi\left(t\right)-\varphi\left(0\right)}{t} = \lim_{t \to 0}\frac{f\left(a+tw\right)-f\left(a\right)}{t} = D_{w}f\left(a\right).\]
\end{observation}
\begin{eg}
Consideremos $\displaystyle f : \R^{2} \to \R $ tal que 
\[f\left(x,y\right) = 
\begin{cases}
x + \frac{xy}{\sqrt{x^{2}+y^{2}}}, \; \left(x,y\right) \neq \left(0,0\right) \\
0, \; \left(x,y\right) = \left(0,0\right)
\end{cases}
.\]
Calculemos las derivadas parciales en $\displaystyle \left(0,0\right) $:
\[\frac{\partial f}{\partial x}\left(0,0\right) = \lim_{t \to 0}\frac{f\left(t,0\right)-f\left(0,0\right)}{t} = \lim_{t \to 0}\frac{1}{t}\left(t + \frac{t \cdot 0}{\sqrt{t^{2}}}-0\right) = \lim_{t \to 0}\frac{t}{t} = 1 .\]
\[\frac{\partial f}{\partial y}\left(0,0\right) = \lim_{t \to 0}\frac{f\left(0,t\right)-f\left(0,0\right)}{t} = \lim_{t \to 0}\frac{1}{t} \cdot 0 = 0 .\]
Por otro lado, si $\displaystyle w = \left(u,v\right) \in \R^{2} $, tenemos que 
\[
\begin{split}
	D_{w}f\left(0,0\right) = & \lim_{t \to 0}\frac{f\left(tu,tv\right)-f\left(0,0\right)}{t} = \lim_{t \to 0}\frac{1}{t}\left(tu + \frac{t^{2}uv}{ \left|t\right|\sqrt{u^{2}+v^{2}}}\right) \\
	= &  \lim_{t \to 0}\left(u + \frac{t}{ \left|t\right|} \frac{uv}{\sqrt{u^{2}+v^{2}}}\right).
\end{split}
\]
Si $\displaystyle vu \neq 0 $, tenemos que $\displaystyle \frac{uv}{\sqrt{u^{2}+v^{2}}}\neq 0 $, por lo que el límite no existe \footnote{El problema es que al tener $\displaystyle \left|t\right| $ en el denominador, al hacer el límite por la izquierda y por la derecha obtenemos $\displaystyle -1 $ y 1, respectivamente.}.
\end{eg}
\begin{observation}
Denotamos $\displaystyle f = \left(f_{1}, \ldots, f_{m}\right) $. Entonces, puesto que los límites en $\displaystyle \R^{m} $ se obtienen  coordenada a coordenada, tenemos que
\[D_{w}f\left(a\right) = \left(D_{w}f_{1}\left(a\right), \ldots, D_{w}f_{m}\left(a\right)\right) \in \R^{m} .\]
\end{observation}
\section{Diferenciabilidad}
\begin{definition}[Diferenciabilidad]
Sea $\displaystyle U \subset \R^{n} $ abierto, $\displaystyle a \in U $ y $\displaystyle f : U \to \R^{m} $. Se dice que $\displaystyle f $ es \textbf{diferenciable} en $\displaystyle a $ si existe $\displaystyle L : \R^{n} \to \R^{m} $ lineal tal que 
\[\lim_{h \to 0}\frac{f\left(a+h\right)-f\left(a\right)-L\left(h\right)}{ \|h\|} = 0.\]
\end{definition}
\begin{observation}
Podemos observar que $\displaystyle f\left(a+h\right) = f\left(a\right) + L\left(h\right) + r\left(h\right) $, donde $\displaystyle r\left(h\right) = f\left(a+h\right)-f\left(a\right)-L\left(h\right) $. Entonces, $\displaystyle f $ es diferenciable en $\displaystyle a $ si y solo si $\displaystyle \lim_{h \to 0}\frac{r\left(h\right)}{ \|h\|} = 0 $. En este caso, se denota $\displaystyle r\left(h\right) = o\left(h\right) $ y así tenemos que $\displaystyle f $ es diferenciable en $\displaystyle a $ si y solo si $\displaystyle f\left(a+h\right) = f\left(a\right)+L\left(h\right)+o\left(h\right) $, donde $\displaystyle L : \R^{n} \to \R^{m} $ es lineal.
\end{observation}
\begin{prop}
Sea $\displaystyle U \subset \R^{n} $ abierto, $\displaystyle f : U \to \R^{m} $ y $\displaystyle a \in U $. Si $\displaystyle f $ es diferenciable en $\displaystyle a $ con aplicación lineal $\displaystyle L $, entonces $\displaystyle \forall v \in \R^{n} $ existe $\displaystyle D_{v}f\left(a\right) = L\left(v\right) $. Por tanto, la aplicación $\displaystyle L $ es única y la denotaremos $\displaystyle L = Df\left(a\right) $ y la llamaremos \textbf{diferencial} de $\displaystyle f $ en $\displaystyle a $.
\end{prop}
\begin{proof}
Tomamos $\displaystyle v \in \R^{n} $ con $\displaystyle v \neq 0 $, y consieramos $\displaystyle h = tv $ con $\displaystyle t \in \R $. Entonces, 
\[0 = \lim_{t \to 0}\frac{f\left(a+tv\right)-f\left(a\right)-L\left(tv\right)}{ \|tv\|} = \lim_{t \to 0}\frac{f\left(a+tv\right)-f\left(a\right)-tL\left(v\right)}{ \left|t\right|} .\]
Por tanto, tenemos que 
\[0 = \lim_{t \to 0} \left|\frac{f\left(a+tv\right)-f\left(a\right)-tL\left(v\right)}{t}\right| = \lim_{t \to 0} \left|\frac{f\left(a+tv\right)-f\left(a\right)}{t}-L\left(v\right)\right| .\]
Por tanto, existe $\displaystyle \lim_{t \to 0}\frac{1}{t}\left(f\left(a+tv\right)-f\left(a\right)\right) = L\left(v\right) $. Por otro lado, si $\displaystyle v = 0 $, tenemos que $\displaystyle D_{v}f\left(a\right) = \lim_{t \to 0}\left(f\left(a\right)-f\left(a\right)\right) = 0 = L\left(0\right) $.
\end{proof}
\begin{eg}
Consideremos $\displaystyle f : \R^{2} \to \R $ con 
\[f\left(x,y\right) = 
\begin{cases}
x + \frac{x \left|y\right|}{\sqrt{x^{2}+y^{2}}}, \; \left(x,y\right) \neq \left(0,0\right) \\
0, \; \left(x,y\right) = \left(0,0\right)
\end{cases}
.\]
Estudiemos si $\displaystyle f $ es diferenciable en $\displaystyle a = \left(0,0\right) $. Tenemos que
\[\frac{\partial f}{\partial x}\left(0,0\right) = \lim_{t \to 0}\frac{1}{t}\left(f\left(t,0\right)-f\left(0,0\right)\right) = \lim_{t \to 0}\frac{1}{t}\left(t+0\right) = 1 .\]
\[\frac{\partial f}{\partial y}\left(0,0\right) = \lim_{t \to 0}\frac{1}{t}\left(f\left(0,t\right)-f\left(0,0\right)\right) = \lim_{t \to 0}\frac{1}{t}\left(0 - 0\right) = 0 .\]
Veamos si $\displaystyle L $ que buscamos es lineal. Si $\displaystyle \left(u,v\right) \in \R^{2} $, por ser $\displaystyle L $ lineal tendríamos que
\[L\left(u,v\right) = uL\left(e_{1}\right) + vL\left(e_{2}\right) = u D_{e_{1}}f\left(a\right)+vD_{e_{2}}f\left(a\right)  .\]
Así, podemos ver que si $\displaystyle f $ es diferenciable en $\displaystyle a $, entonces $\displaystyle L\left(u,v\right) = u \cdot 1 + v \cdot 0 = u$. Vamos a ver si $\displaystyle f $ es diferenciable en $\displaystyle a = \left(0,0\right) $:
\[
\begin{split}
	\lim_{\left(x,y\right) \to \left(0,0\right)}\frac{f\left(x,y\right)-f\left(0,0\right)-L\left(x,y\right)}{\sqrt{x^{2}+y^{2}}} = & \lim_{\left(x,y\right) \to \left(0,0\right)}\frac{1}{\sqrt{x^{2}+y^{2}}} \left(x + \frac{x \left|y\right|}{\sqrt{x^{2}+y^{2}}}-0-x\right) \\
	= & \lim_{\left(x,y\right) \to \left(0,0\right)}\frac{x \left|y\right|}{x^{2}+y^{2}} \sim \frac{r^{2}\cos\theta \left|\sin\theta \right|}{r^{2}}.
\end{split}
\]
Como el valor del límite depende de $\displaystyle \theta $, tenemos que el límite no existe, por lo que $\displaystyle f $ no es diferenciable en $\displaystyle \left(0,0\right) $. Otra forma de verlo es, dado $\displaystyle w = \left(u,v\right) \in \R^{2} $, tenemos que 
\[
\begin{split}
	D_{w}f\left(0,0\right) = & \lim_{t \to 0}\frac{1}{t}\left(f\left(tu,tv\right)-f\left(0,0\right)\right) = \lim_{t \to 0}\frac{1}{t}\left(tu + \frac{t \left|t\right|u \left|v\right|}{ \left|t\right|\sqrt{u^{2}+v^{2}}}\right) \\
	= & \lim_{t \to 0}\left(u +\frac{u \left|v\right|}{ \sqrt{u^{2}+v^{2}}}\right) = u + \frac{u \left|v\right|}{\sqrt{u^{2}+v^{2}}} .
\end{split}
\]
Esta última expresión no es lineal, por lo que $\displaystyle f $ no es diferenciable en $\displaystyle \left(0,0\right)$.
\end{eg}
\begin{prop}
Sean $\displaystyle U \subset \R^{n} $ abierto, $\displaystyle f : U \to \R^{m} $ y $\displaystyle a \in U $. Entonces $\displaystyle f $ es diferenciable en $\displaystyle a $ si y solo si $\displaystyle f_{1}, \ldots, f_{m} $ son diferenciables en $\displaystyle a $ y en este caso 
\[Df\left(a\right)\left(v\right) = \left(Df_{1}\left(a\right)\left(v\right), \ldots, Df_{m}\left(a\right)\left(v\right)\right) .\]
\end{prop}
\begin{proof}
Sea $\displaystyle L : \R^{n} \to \R^{m} $ y denotamos $\displaystyle L = \left(L_{1}, \ldots, L_{m}\right) $. Entonces, el límite de la definición es cero si y solo si cada componente tiene límite cero, es decir, si y solo si cada $\displaystyle f_{j} $ es diferenciable en $\displaystyle a $ con $\displaystyle L_{j} $, $\displaystyle \forall j = 1, \ldots, m $. 
\[  D\left(f\left(a\right)\right)\left(v\right) = D_{v}f\left(a\right) = \left(D_{v}f_{1}\left(a\right), \ldots, D_{v}f_{m}\left(a\right)\right) = \left(Df_{1}\left(a\right)\left(v\right), \ldots, Df_{m}\left(a\right)\left(v\right)\right).\]
\end{proof}
\begin{definition}[Matriz jacobiana]
Sean $\displaystyle U \subset \R^{n} $ abierto, $\displaystyle a \in U $ y $\displaystyle f : U \to \R^{m} $. Si $\displaystyle f $ admite todas las derivadas parciales en $\displaystyle a $, se define la \textbf{matriz jacobiana} de $\displaystyle f $ en $\displaystyle a $ como 
\[Jf\left(a\right) = \begin{pmatrix} \frac{\partial f_{1}}{\partial x_{1}}\left(a\right) & \cdots & \frac{\partial f_{1}}{\partial x_{n}} \left(a\right)\\ \vdots & & \vdots \\ \frac{\partial f_{m}}{\partial x_{1}}\left(a\right) & \cdots & \frac{\partial f_{m}}{\partial x_{n}}\left(a\right) \end{pmatrix} \in \mathcal{M}_{m \times n} .\]
\end{definition}
\begin{prop}
Sean $\displaystyle U \subset \R^{n} $ abierto, $\displaystyle f : U \to \R^{m} $ y $\displaystyle a \in U $. Si $\displaystyle f $ es diferenciable en $\displaystyle a $, la matriz jacobiana, $\displaystyle Jf\left(a\right) $, es la matriz de $\displaystyle Df\left(a\right) $ con respecto a las bases canónicas.
\end{prop}
\begin{proof}
Sea $\displaystyle v = \sum^{m}_{i = 1}v_{i}e_{i}\in \R^{m} $. Tenemos que 
\[
\begin{split}
	Df\left(a\right)\left(v\right) = & \left(Df_{1}\left(a\right)\left(v\right), \ldots, Df_{m}\left(a\right)\left(v\right)\right) 
	=  \left(Df_{1}\left(a\right)\left(\sum^{m}_{i = 1}v_{i}e_{i}\right), \ldots, Df_{m}\left(a\right)\left(\sum^{n}_{i = 1}v_{i}e_{i}\right)\right) \\
	= & \left(\sum^{n}_{i = 1}v_{i}Df_{1}\left(a\right)\left(e_{i}\right), \ldots, \sum^{n}_{i = 1}v_{i}Df_{m}\left(a\right)\left(e_{i}\right)\right) 
	=  \left(\sum^{n}_{i = 1}v_{i}\frac{\partial f_{1}}{\partial x_{i}}\left(a\right), \ldots, \sum^{n}_{ i= 1}v_{i}\frac{\partial f_{m}}{\partial x_{i}}\left(a\right)\right) \\
	= & Jf\left(a\right) \begin{pmatrix} v_{1} \\ \vdots \\ v_{n} \end{pmatrix}.
\end{split}
\]
\end{proof}
\begin{observation}
Podemos establecer dos límites equivalentes que definen la diferenciabilidad:
\[\lim_{h \to 0}\frac{f\left(a + h\right)-f\left(a\right)-L\left(h\right)}{\|h\|} = \lim_{x \to a}\frac{f\left(x\right)-f\left(a\right)-L\left(x-a\right)}{\|x-a\|} .\]
\end{observation}
\begin{prop}
Si $\displaystyle f $ es constante, entonces $\displaystyle f $ es diferenciable en todos los puntos y $\displaystyle Df\left(a\right) = 0 $, $\displaystyle \forall a \in U $.
\end{prop}
\begin{proof}
En efecto, tenemos que si $\displaystyle x \to a $ 
\[\frac{f\left(x\right)-f\left(a\right)-0}{\|x-a\|} = \frac{0}{ \| x- a\|} \to 0 .\]
\end{proof}
\begin{prop}
Si $\displaystyle f $ es lineal, tenemos que $\displaystyle f $ es diferenciable en todos los puntos y $\displaystyle Df\left(a\right) = f $, $\displaystyle \forall a \in U $. 
\end{prop}
\begin{proof}
En efecto, tenemos que
\[\frac{f\left(x\right)-f\left(a\right)-f\left(x-a\right)}{ \|x-a\|} = \frac{f\left(x\right)-f\left(a\right)-f\left(x\right)+f\left(a\right)}{\|x-a\|} = 0 .\]	
\end{proof}
\begin{prop}
Si $\displaystyle f,g : U \to \R^{m} $ son diferenciables en $\displaystyle a \in U $ y $\displaystyle \lambda, \mu \in \R $, entonces $\displaystyle \lambda f + \mu g $ es diferenciable en $\displaystyle a $ y  
\[ D\left(\lambda f + \mu g\right)\left(a\right) = \lambda Df\left(a\right) + \mu Df\left(a\right).\]
\end{prop}
\begin{proof}
En efecto, tenemos que
\[
\begin{split}
 & \frac{\left(\lambda f + \mu g\right)\left(x\right) - \left(\lambda f + \mu g\right)\left(a\right) - \left(\lambda Df\left(a\right)+\mu Dg\left(a\right)\right)\left(x-a\right)}{\|x-a\|} \\
	= & \frac{\lambda \left[f\left(x\right)-f\left(a\right) - Df\left(a\right)\left(x-a\right)\right] +\mu \left[g\left(x\right)-g\left(a\right)-Dg\left(a\right)\left(x-a\right)\right] }{\|x-a\|} \\
	= & \lambda \left(\frac{f\left(x\right)-f\left(a\right)-Df\left(a\right)\left(x-a\right)}{\|x-a\|}\right) + \mu \left(\frac{g\left(x\right)-g\left(a\right)-Dg\left(a\right)\left(x-a\right)}{\|x-a\|}\right) \to 0.
\end{split}
\]
\end{proof}
\begin{notation}
	Denotamos $\displaystyle \mathcal{L}\left(\R^{n}, \R^{m}\right) = \left\{ L : \R^{n} \to \R^{m}\; : \; L \; \text{lineal}\right\}  $, que es un espacio vectorial de dimensión $\displaystyle n \times m $. Para $\displaystyle L \in \mathcal{L}\left(\R^{n}, \R^{m}\right) $ definimos la norma matricial \footnote{Por ser $\displaystyle L $ una aplicación lineal (y por ello continua) y por ser $\displaystyle \overline{B}\left(0,1\right) $ un conjunto compacto, existe el supremo del conjunto.} 
	\[ \|L\| = \sup \left\{ \|L\left(u\right)\| : \|u\|\leq 1\right\} = \sup \left\{ \|L\left(u\right)\| \; : \; u \in \overline{B}\left(0,1\right)\right\} < \infty .\]
\end{notation}
\begin{prop}
La aplicación definida anteriormente es efectivamente una norma.
\end{prop}
\begin{proof}
	\begin{itemize}
	\item Veamos que si $\displaystyle \|L\| = 0 $, entonces $\displaystyle L = 0 $. Para ello, supongamos que $\displaystyle L \neq 0 $. Entonces, existe $\displaystyle x \in \R^{n} $ tal que $\displaystyle L\left(x\right) \neq 0 $. Así, tenemos que $\displaystyle x \neq 0 $ y podemos considerar $\displaystyle u = \frac{x}{\|x\|} \in \overline{B}\left(0,1\right) $. Así, tenemos que
\[ \|L\| \geq \|L\left(u\right)\| = \left\|L\left(\frac{x}{\|x\|}\right)\right\| = \left\|\frac{1}{\|x\|}L\left(x\right)\right\| = \frac{\|L\left(x\right)\|_{\R^{m}}}{\|x\|_{\R^{n}}} \neq 0.\]
Así, tenemos que si $\displaystyle \|L\| = 0 $ entonces $\displaystyle L = 0 $. 
\item Por otro lado, tenemos que si $\displaystyle \lambda \in \R $,
\[\|\lambda L\| = \sup \left\{ \|\lambda L\left(u\right)\| \; : \; u \in \overline{B}\left(0,1\right)\right\} = \left|\lambda \right| \sup \left\{ \|L\left(u\right)\| \; : \; u \in \overline{B}\left(0,1\right)\right\} = \left|\lambda \right| \|L\| .\]
\item Similarmente, tenemos que $\displaystyle \forall u \in \overline{B}\left(0,1\right) $,
\[\|\left(L_{1}+L_{2}\right)\left(u\right)\| \leq \|L_{1}\left(u\right) + L_{2}\left(u\right)\| \leq \|L_{1}\left(u\right)\| + \|L_{2}\left(u\right)\| \leq \|L_{1}\| + \|L_{2}\| .\]
Así, tenemos que $\displaystyle \|L_{1} + L_{2} \| \leq \|L_{1}\| + \|L_{2}\| $.
	\end{itemize}
\end{proof}
\begin{observation}
Por tanto, $\displaystyle \| \cdot \| $ define una norma en $\displaystyle \mathcal{L}\left(\R^{n}, \R^{m}\right) $ que se llamará \textbf{matricial} porque 
\[\|L\left(x\right)\|_{\R^{m}} \leq \|L\| \|x\|_{\R^{n}}, \; \forall x \in \R^{n} .\]
En efecto, para $\displaystyle x = 0 $ es trivial. Si $\displaystyle x \neq 0 $, tomamos $\displaystyle u = \frac{x}{\|x\|} \in \overline{B}\left(0,1\right) $ por lo que $\displaystyle \|L\left(u\right)\| \leq \|L\| $. Así, obtenemos que
\[\|L\left(x\right)\|_{\R^{m}} \leq \|L\| \cdot \|x\|_{\R^{n}} .\]
\end{observation}
\begin{prop}
Sean $\displaystyle U \subset \R^{n} $ abierto, $\displaystyle a \in U $ y $\displaystyle f : U \to \R^{m} $. Si $\displaystyle f $ es diferenciable en $\displaystyle a $, entonces existen $\displaystyle r > 0 $ y $\displaystyle M > 0 $ tales que
\[ \|f\left(x\right)-f\left(a\right)\| \leq M \|x-a\|, \;  \|x - a\| \leq r .\]
\end{prop}
\begin{proof}
Dado $\displaystyle \epsilon > 0 $, existe $\displaystyle \delta > 0 $ tal que si $\displaystyle \|x-a\| \leq \delta  $, entonces 
\[ \|f\left(x\right)-f\left(a\right)-Df\left(a\right)\left(x-a\right)\|\leq \epsilon \|x-a\| .\]
Dado $\displaystyle \epsilon = 1 $, existe $\displaystyle r > 0 $ tal que si $\displaystyle \|x-a\| \leq r $, entonces 
\[ \|f\left(x\right)-f\left(a\right)-Df\left(a\right)\left(x-a\right)\| \leq  \|x-a\| .\]
Así, nos queda que 
\[ .\]
\[
\begin{split}
	\|f\left(x\right)-f\left(a\right)\| \leq & \|x-a\| + \|Df\left(a\right)\left(x-a\right)\| \\
	\leq & \|x-a\| + \|Df\left(a\right)\| \|x-a\| = \left(\|Df\left(a\right)\| + 1\right) \|x-a\|.
\end{split}
\]

Así, basta con tomar $\displaystyle M = \|Df\left(a\right)\|+1 $.
\end{proof}
\begin{colorary}
Sean $\displaystyle U \subset \R^{n} $ abierto, $\displaystyle a \in U $ y $\displaystyle f : U \to \R^{m} $. Si $\displaystyle f $ es diferenciable en $\displaystyle a $ entonces $\displaystyle f $ es continua en $\displaystyle a $.
\end{colorary}
\begin{theorem}[Regla de la cadena]
Sean $\displaystyle U \subset \R^{n}, V \subset \R^{m} $ abiertos, $\displaystyle f : U \to \R^{m} $ y $\displaystyle g : V \to \R^{p} $ con $\displaystyle f\left(U\right) \subset V $, y sea $\displaystyle a \in U $. Entonces, si $\displaystyle f $ es diferenciable en $\displaystyle a $ y $\displaystyle g $ es diferenciable en $\displaystyle f\left(a\right) $, entonces $\displaystyle g \circ f $ es diferenciable en $\displaystyle a $ y 
\[D\left(g\circ f\right)\left(a\right) = Dg\left(f\left(a\right)\right)\circ Df\left(a\right) .\]
Por tanto, $\displaystyle J\left(g\circ f\right)\left(a\right) = Jg\left(f\left(a\right)\right) \cdot J\left(f\left(a\right)\right) $.
\end{theorem}
\begin{proof}
Por la proposición anterior sabemos que existen $\displaystyle r > 0 $ y $\displaystyle M > 0 $ tales que 
\[\|f\left(x\right)-f\left(a\right)\| \leq M \|x-a\|, \; \|x-a\| \leq r .\]
Sea ahora $\displaystyle M' = \|Dg\left(f\left(a\right)\right)\| $ y denotamos $\displaystyle b = f\left(a\right) $. Dado $\displaystyle \epsilon > 0 $ consideramos $\displaystyle \epsilon ' = \frac{\epsilon }{M + M'} > 0 $. Existe $\displaystyle \delta _{1} > 0 $ tal que si $\displaystyle \|y - b\| \leq \delta _{1} $, entonces 
\[\|g\left(y\right)-g\left(b\right)-Dg\left(b\right)\left(y-b\right)\| \leq \epsilon '\|y - b\| .\]
También tenemos que existe $\displaystyle \delta _{2} > 0 $ tal que si $\displaystyle \|x-a\| \leq \delta_{2} $, entonces
\[\|f\left(x\right)-f\left(a\right)-Df\left(a\right)\left(x-a\right)\| \leq \epsilon ' \|x-a\| .\]
Sea $\displaystyle \delta = \min \left\{ r,\frac{\delta _{1}}{M}, \delta _{2}\right\} > 0 $. Si $\displaystyle \|x-a\| \leq \delta  $, en particular tenemos que
\[\|f\left(x\right)-f\left(a\right)\| \leq M \|x-a\| \leq M \cdot \frac{\delta _{1}}{M} = \delta _{1} .\]
Así, tomando $\displaystyle y = f\left(x\right) $ tenemos que
\[
\begin{split}
 \|g\left(f\left(x\right)\right)-g\left(f\left(a\right)\right) - Dg\left(f\left(a\right)\right)\left(f\left(x\right)-f\left(a\right)\right) \| \leq \epsilon ' \|f\left(x\right)-f\left(a\right)\| \leq \epsilon ' M \|x-a\|.
\end{split}
\]
Así, nos queda 
\[
\begin{split}
& \|g\left(f\left(x\right)\right)-g\left(f\left(a\right)\right) -Dg\left(f\left(a\right)\right)\left(Df\left(a\right)\left(x-a\right)\right)\| \\
\leq & \|g\left(f\left(x\right)\right)-g\left(f\left(a\right)\right) - Dg\left(f\left(a\right)\right)\left(f\left(x\right)-f\left(a\right)\right) \| + \|Dg\left(f\left(a\right)\right) \left(f\left(x\right)-f\left(a\right) -Df\left(a\right)\left(x-a\right) \right)\| \\
\leq & \epsilon ' M \|x-a\| + \|Dg\left(f\left(a\right)\right)\| \|f\left(x\right)-f\left(a\right) - Df\left(a\right)\left(x-a\right)\| \leq \epsilon ' M' \|x-a\| + M\epsilon ' \|x-a\|\\
= & \epsilon'\left(M + M'\right)\|x-a\| = \epsilon \|x-a\|.
\end{split}
\]
\end{proof}
\begin{theorem}[Condición suficiente de diferenciabilidad]
Sean $\displaystyle U \subset \R^{n} $ abierto, $\displaystyle a \in U $ y $\displaystyle f : U \to \R^{m} $. Supongamos que $\displaystyle \forall j = 1, \ldots, m $, $\displaystyle \forall i = 1, \ldots, n $, $\displaystyle \forall x \in U $, existe $\displaystyle \frac{\partial f_{j}}{\partial x_{i}}\left(x\right) $ y además las funciones $\displaystyle \frac{\partial f_{j}}{\partial x_{i}} : U \to \R^{m} $ son continuas en $\displaystyle a $. Entonces, $\displaystyle f $ es diferenciable en $\displaystyle a $.
\end{theorem}
\begin{proof}
Es suficiente considerar el caso $\displaystyle m = 1 $, puesto que $\displaystyle f = \left(f_{1}, \ldots, f_{m}\right) $ es diferenciable en $\displaystyle a $ si y solo si $\displaystyle f_{j}  $ es diferenciable en $\displaystyle a $, $\displaystyle \forall j = 1, \ldots, m $. Como $\displaystyle U $ es abierto y $\displaystyle a \in U $, existe $\displaystyle r > 0 $ tal que $\displaystyle B_{\infty}\left(a,r\right) \subset U $.
Sea $\displaystyle x \in B_{\infty}\left(a,r\right) $, tenemos que
\[
\begin{split}
	f\left(x\right)-f\left(a\right) = & f\left(x_{1}, \ldots, x_{n}\right) - f\left(a_{1}, \ldots, a_{n}\right) \\
	= & f\left(x_{1}, \ldots, x_{n}\right) -f\left(a_{1}, x_{2}, \ldots, x_{n}\right) + f\left(a_{1}, x_{2}, \ldots, x_{n}\right) \\
	- &  f\left(a_{1}, a_{2}, x_{3}, \ldots, x_{n}\right) + \cdots + f\left(a_{1}, a_{2}, \ldots, a_{n-1}, x_{n}\right) - f\left(a_{1}, \ldots, a_{n}\right).
\end{split}
\]
Denotamos $\displaystyle \varphi\left(t\right) = f\left(t,x_{2}, \ldots, x_{n}\right) $ con $\displaystyle t \in \left(a_{1}-r, a_{1}+r\right) $, tenemos que si aplicamos el teorema del valor medio encontramos $\displaystyle c_{1}  $ un punto intermedio entre $\displaystyle a_{1} $ y $\displaystyle x_{1} $ tal que
\[f\left(x_{1}, \ldots, x_{n}\right) -f\left(a_{1}, x_{2}, \ldots, x_{n}\right) = \varphi\left(x_{1}\right)-\varphi\left(a_{1}\right) = \varphi'\left(c_{1}\right)\left(x_{1}-a_{1}\right)   .\]
En efecto, podemos aplicar el teorema del valor medio puesto que $\displaystyle \varphi $ es derivable en $\displaystyle \left(a_{1}-r, a_{1}+r\right) $ y 
\[\varphi'\left(t\right) = \lim_{h \to 0}\frac{f\left(t+h, x_{2}, \ldots, x_{n}\right)-f\left(t, x_{2}, \ldots, x_{n}\right)}{t} = \frac{\partial f}{\partial x_{1}}\left(t, x_{2}, \ldots, x_{n}\right) .\]
Así, aplicando el mismo razonamiento nos queda que 
\[
\begin{split}
	f\left(x\right)-f\left(a\right) = & \left(x_{1}-a_{1}\right)\frac{\partial f}{\partial x_{1}}\left(c_{1}, x_{2}, \ldots, x_{n}\right) + \left(x_{2}-a_{2}\right)\frac{\partial f}{\partial x_{2}}\left(a_{1}, c_{2}, \ldots, x_{n}\right) \\
	+ &  \cdots + \left(x_{n}-a_{n}\right)\frac{\partial f}{\partial x_{n}}\left(a_{1}, \ldots, a_{n-1}, c_{n}\right) .
\end{split}
\]
donde cada $\displaystyle c_{i} $ es un punto intermedio entre $\displaystyle a_{i} $ y $\displaystyle x_{i} $, $\displaystyle \forall i = 1, \ldots, n $. Ahora, definimos $\displaystyle L\left(h\right) = \sum^{n}_{i = 1}h_{i}\frac{\partial f}{\partial x_{i}}\left(a\right) $, por tanto tendremos que
\[L\left(x-a\right) = \sum^{n}_{i = 1}\left(x_{i}-a_{i}\right) \cdot \frac{\partial f}{\partial x_{i}}\left(a\right) .\]
Así, nos queda que
\[
\begin{split}
	 f\left(x\right)-f\left(a\right) - L\left(x-a\right) 
	 = & \left(x_{1}-a_{1}\right) \left(\frac{\partial f}{\partial x_{1}}\left(c_{1}, x_{2}, \ldots, x_{n}\right)-\frac{\partial f}{\partial x_{1}}\left(a_{1}, \ldots, a_{n}\right)\right) \\
	+ &  \cdots + \left(x_{n}-a_{n}\right)\left(\frac{\partial f}{\partial x_{n}}\left(a_{1}, \ldots, a_{n-1}, c_{n}\right)-\frac{\partial f}{\partial x_{n}}\left(a_{1}, \ldots, a_{n}\right)\right) .
\end{split}
\]
De esta forma la fracción nos queda
\[
\begin{split}
	\frac{ \left|f\left(x\right)-f\left(a\right)-L\left(x-a\right)\right|}{ \|x-a\|} \leq & \frac{ \left|x_{1}-a_{1}\right|}{\|x-a\|} \left|\frac{\partial f}{\partial x_{1}}\left(c_{1}, x_{2}, \ldots, x_{n}\right)-\frac{\partial f}{\partial x_{1}}\left(a_{1}, \ldots, a_{n}\right)\right| \\
	+ & \cdots + \frac{ \left|x_{n}-a_{n}\right|}{\|x-a\|} \left|\frac{\partial f}{\partial x_{n}}\left(a_{1}, \ldots, a_{n-1}, c_{n}\right)-\frac{\partial f}{\partial x_{n}}\left(a_{1}, \ldots, a_{n}\right)\right|.
\end{split}
\]
Por tanto, cuando $\displaystyle x \to a $ tenemos que $\displaystyle \forall i = 1, \ldots, n $, $\displaystyle c_{i} \to a_{i} $ por lo que 
\[
\begin{split}
	\frac{\partial f}{\partial x_{1}}\left(c_{1}, x_{2}, \ldots, x_{n}\right) & \to \frac{\partial f}{\partial x_{1}}\left(a\right) \\
& \vdots \\
	\frac{\partial f}{\partial x_{n}}\left(a_{1}, \ldots, a_{n-1}, c_{n}\right) & \to \frac{\partial f}{\partial x_{n}}\left(a\right).
\end{split}
\]
\end{proof}
\begin{eg}
Consideremos $\displaystyle f\left(x,y,z\right) = x^{2}z + \sin\left(x+y+z\right) $. Tenemos que $\displaystyle f : \R^{3} \to \R $. Además, 
\[\frac{\partial f}{\partial x} = 2xz + \cos\left(x+y+z\right), \quad \frac{\partial f}{\partial y}= \cos\left(x+y+z\right), \quad \frac{\partial f}{\partial z} = x^{2} + \cos\left(x+y+z\right) .\]
Todas las derivadas son continuas en $\displaystyle \R^{3} $, por lo que $\displaystyle f $ es diferenciable en todo punto de $\displaystyle \R^{3} $. 
\end{eg}
\begin{observation}
El mismo resultado se obtiene $\displaystyle \forall j = 1, \ldots, m $, $\displaystyle \forall i = 1, \ldots, n-1 $, $\displaystyle \frac{\partial f}{\partial x_{i}} $ es continua en $\displaystyle a $ y además existe $\displaystyle \frac{\partial f}{\partial x_{n}}\left(a\right) $.
\end{observation}
\begin{colorary}
Sea $\displaystyle U \subset \R^{n} $ abierto y $\displaystyle f : U \to \R^{m} $. Si $\displaystyle f $ admite todas sus derivadas parciales en $\displaystyle U $ y son continuas en $\displaystyle U $, entonces $\displaystyle f $ es diferenciable en $\displaystyle U $.
\end{colorary}
\begin{lema}
La función $\displaystyle p : \R^{2} \to \R $ con $\displaystyle p\left(x,y\right) = xy $ es diferenciable en $\displaystyle \R^{2} $ y $\displaystyle \forall \left(x_{0}, y_{0}\right) \in \R^{2} $,
\[Dp\left(x_{0},y_{0}\right)\left(u,v\right) = x_{0}v + y_{0}u .\]
\end{lema}
\begin{proof}
Es trivial ver que
\[\frac{\partial p}{\partial x}\left(x,y\right) = y, \quad \frac{\partial f}{\partial y}\left(x,y\right) = x .\]
Como ambas funciones son continuas en $\displaystyle \R^{2} $, por el teorema anterior tenemos que $\displaystyle p $ es diferenciable en $\displaystyle \R^{2} $. Por otro lado,
\[Dp\left(x_{0},y_{0}\right)\left(u,v\right) = \left(\frac{\partial p}{\partial x}\left(x_{0}, y_{0}\right), \frac{\partial p}{\partial y}\left(x_{0}, y_{0}\right)\right)\begin{pmatrix} u \\ v \end{pmatrix} = x_{0}v + y_{0}u .\]
\end{proof}
\begin{lema}
	Sean $\displaystyle A = \left\{ \left(x,y\right) \; : \; y \neq 0\right\}  $ y $\displaystyle q : A \to \R $ con $\displaystyle q\left(x,y\right) = \frac{x}{y} $. Entonces $\displaystyle q $ es diferenciable en $\displaystyle A $ y $\displaystyle \forall \left(x_{0}, y_{0}\right) \in A $
	\[Dq\left(x_{0},y_{0}\right)\left(u,v\right) = \frac{y_{0}u - x_{0}v}{y_{0}^{2}}.\]
\end{lema}
\begin{proof}
Es trivial ver que
\[\frac{\partial q}{\partial x}\left(x,y\right) = \frac{1}{y}, \quad \frac{\partial q}{\partial x}\left(x,y\right) = -\frac{x}{y^{2}} .\]
Como son continuas en $\displaystyle A $ tenemos que $\displaystyle q $ es diferenciable en $\displaystyle A $. Por otro lado,
\[Dq\left(x_{0}, y_{0}\right)\left(u,v\right) = \left(\frac{1}{y_{0}}, -\frac{x_{0}}{y_{0}^{2}}\right)\begin{pmatrix} u \\ v \end{pmatrix} = \frac{y_{0}u - x_{0}v}{y_{0}^{2}} .\]
\end{proof}
\begin{colorary}[de la Regla de la Cadena]
Sean $\displaystyle U \subset \R^{n} $ abierto y $\displaystyle f, g : U \to \R $ diferenciables en $\displaystyle a $. 
\begin{enumerate}
\item Entonces, $\displaystyle f \cdot g $ es diferenciable en $\displaystyle a $ y 
\[D\left(f \cdot g\right)\left(a\right) = g\left(a\right)Df\left(a\right) + f\left(a\right)Dg\left(a\right) .\]
\item Si además, $\displaystyle g\left(a\right) \neq 0 $, entonces $\displaystyle \frac{f}{g} $ es diferenciable en $\displaystyle a $ y
	\[D\left(\frac{f}{g}\right)\left(a\right) = \frac{g\left(a\right)Df\left(a\right)-f\left(a\right)Dg\left(a\right)}{g\left(a\right)^{2}} .\]
\end{enumerate}
\end{colorary}
\begin{proof}
\begin{enumerate}
\item Podemos considerar la composición de funciones
\[
\begin{split}
	U & \to \R^{2} \to \R \\
	x & \to \left(f\left(x\right), g\left(x\right)\right) \to f\left(x\right)g\left(x\right).
\end{split}
\]
Así, nos queda que $\displaystyle f \cdot g = p\circ \left(f,g\right)$. De esta forma, tenemos que $\displaystyle \forall v \in \R^{n} $ 
\[
\begin{split}
	D\left(f \cdot g\right)\left(a\right)\left(v\right) = & \left[D\left(p\left(f\left(a\right), g\left(a\right)\right)\right) \circ D\left(f,g\right)\left(a\right)\right]\left(v\right) \\
= & Dp\left(f\left(a\right), g\left(a\right)\right) \left(Df\left(a\right)\left(v\right), Dg\left(a\right)\left(v\right)\right) = g\left(a\right) \cdot Df\left(a\right)\left(v\right) + f\left(a\right) \cdot Dg\left(a\right)\left(v\right).
\end{split}
\]
\item Según el lema previo, la función $\displaystyle q : A =\left\{ \left(x,y\right) \; : \; y \neq 0\right\}  \to \R : \left(x,y\right) \to \frac{x}{y} $ es diferenciable en $\displaystyle A $. Como $\displaystyle g\left(a\right) \neq 0 $ y $\displaystyle g $ es continua, el conjunto $\displaystyle V = \left\{ x \in U \; : \; g\left(x\right) \neq 0\right\}  $ es abierto de $\displaystyle \R^{n} $. Así, la función $\displaystyle \frac{f}{g} $ está bien definida en $\displaystyle V $ y tenemos que
	\[
	\begin{split}
		\frac{f}{g}: V & \to^{\left(f,g\right)} A \to^{q} \R \\
		x & \to \left(f\left(x\right), g\left(x\right)\right)  \to \frac{f\left(x\right)}{g\left(x\right)}.
	\end{split}
	\]
Puesto que $\displaystyle h = \left(f,g\right) $ es diferenciable en $\displaystyle A $ y $\displaystyle Dh\left(a\right)\left(v\right) = \left(Df\left(a\right)\left(v\right), Dg\left(a\right)\left(v\right)\right) $, tenemos que
\[
\begin{split}
D\left(\frac{f}{g}\right)\left(a\right) = D\left(q\circ h\right)\left(a\right) = Dq\left(h\left(a\right)\right) \circ Dh\left(a\right) .
\end{split}
\]
Así, $\displaystyle \forall w \in \R^{n} $ se cumple que
\[
\begin{split}
	D\left(\frac{f}{g}\right)\left(a\right)\left(w\right) = & Dq\left(f\left(a\right), g\left(a\right)\right)\left[Dh\left(a\right)\left(w\right)\right] = Dq\left(f\left(a\right), g\left(a\right)\right)\left(Df\left(a\right)\left(w\right), Dg\left(a\right)\left(w\right)\right) \\
	= & \frac{g\left(a\right)Df\left(a\right)\left(w\right)-f\left(a\right)Dg\left(a\right)\left(w\right)}{g\left(a\right)^{2}} .
\end{split}
\]
\end{enumerate}
\end{proof}
\begin{observation}
Sean $\displaystyle f,g $ diferenciables y $\displaystyle h = g\circ f $. Por la regla de la cadena tenemos que $\displaystyle Dh\left(a\right)=Dg\left(f\left(a\right)\right)\circ Df\left(a\right) $ y que $\displaystyle Jh\left(a\right)=Jg\left(f\left(a\right)\right) \cdot Jf\left(a\right) $. Así, nos queda que
\[\begin{pmatrix} \frac{\partial h_{1}}{\partial x_{1}} & \cdots & \frac{\partial h_{1}}{\partial x_{n}} \\ \vdots & & \vdots \\ \frac{\partial h_{p}}{\partial x_{1}} & \cdots & \frac{\partial h_{p}}{\partial x_{n}} \end{pmatrix}_{a}= \begin{pmatrix} \frac{\partial g_{1}}{\partial y_{1}} & \cdots & \frac{\partial g_{1}}{\partial y_{m}} \\ \vdots & & \vdots \\ \frac{\partial g_{p}}{\partial y_{1}} & \cdots & \frac{\partial g_{p}}{\partial y_{m}} \end{pmatrix}_{f\left(a\right)}
\cdot \begin{pmatrix} \frac{\partial f_{1}}{\partial x_{1}} & \cdots & \frac{\partial f_{1}}{\partial x_{n}} \\ 
\vdots & & \vdots \\ 
\frac{\partial f_{m}}{\partial x_{1}} & \cdots & \frac{\partial f_{m}}{\partial x_{n}}\end{pmatrix}_{a}.\]
Así, si cogemos el elemento que está en la columna $\displaystyle i $ y la fila $\displaystyle k $ obtenemos que 
\[\frac{\partial h_{k}}{\partial x_{i}}\left(a\right) = \sum^{m}_{j = 1}\frac{\partial g_{k}}{\partial y_{j}}f\left(a\right) \cdot \frac{\partial f_{j}}{\partial x_{i}}\left(a\right) .\]
Convencionalmente, la regla de la cadena se escribe de la forma
\[\frac{\partial z_{k}}{\partial x_{i}} = \sum^{m}_{ j= 1}\frac{\partial z_{k}}{\partial y_{j}} \cdot \frac{\partial y_{j}}{\partial x_{i}} .\]
\end{observation}
\begin{eg}
Consideremos $\displaystyle u = x^{2}+3y $, $\displaystyle v = \sin\left(x-y\right) $ y $\displaystyle w = y $. Consideremos la aplicación $\displaystyle z = 2u-v^{2}+w $. Tenemos que 
\[
\begin{split}
	\frac{\partial z}{\partial x} = & \frac{\partial z}{\partial u} \cdot \frac{\partial u}{ \partial x} + \frac{\partial z}{\partial v} \cdot \frac{\partial v}{\partial x} + \frac{\partial z}{\partial w} \cdot \frac{\partial w}{\partial x} \\
	= &  4x -2v\cos\left(x-y\right) = 4x -2\sin\left(x-y\right)\cos\left(x-y\right).
\end{split}
\]
Análogamente, 
\[
\begin{split}
	\frac{\partial z}{\partial y} = & \frac{\partial z}{\partial u} \cdot \frac{\partial u}{ \partial y} + \frac{\partial z}{\partial v} \cdot \frac{\partial v}{\partial y} + \frac{\partial z}{\partial w} \cdot \frac{\partial w}{\partial y} \\
	= &  6 +2v\cos\left(x-y\right) + 1 = 7 + 2\sin\left(x-y\right)\cos\left(x-y\right).
\end{split}
\]
También las podríamos haber calculado con el producto de las matrices jacobianas. 
\end{eg}
\begin{definition}[Vector gradiente]
Sean $\displaystyle U \subset \R^{n} $ abierto, $\displaystyle f : U \to \R $ diferenciable en $\displaystyle a \in U $. Se define el \textbf{vector gradiente} de $\displaystyle f $ en $\displaystyle a $ como 
\[\nabla f\left(a\right)= \left(\frac{\partial f}{\partial x_{1}}\left(a\right), \ldots, \frac{\partial f}{\partial x_{n}}\left(a\right)\right) \in \R^{n} .\]
\end{definition}
\begin{observation}
Tenemos que $\displaystyle \forall w \in \R^{n} $,
\[Df\left(a\right)\left(w\right) =D_{w}f\left(a\right) = \sum^{n}_{i = 1}\frac{\partial f\left(a\right)}{\partial x_{i}}w_{i} = \left\langle \nabla f\left(a\right), w \right\rangle  .\]
\end{observation}
\begin{observation}
Sea $\displaystyle I \subset \R $ un intervalo abierto y $\displaystyle f : I \to \R^{m} $. Sabemos que $\displaystyle f $ es diferenciable en $\displaystyle t_{0} \in I $ si y solo si $\displaystyle f $ es derivable en $\displaystyle t_{0} $ y además $\displaystyle f'\left(t_{0}\right) = Df\left(t_{0}\right)\left(1\right) $. 
\end{observation}
\begin{colorary}
Sean $\displaystyle I \subset \R $ un intervalo abierto, $\displaystyle t_{0} \in I $; $\displaystyle f : I \to \R^{m} $ derivable en $\displaystyle t_{0} $ con $\displaystyle f\left(I\right) \subset U $ abierto; $\displaystyle g : U \to \R $ diferenciable en $\displaystyle a = f\left(t_{0}\right) $. Entonces, $\displaystyle g\circ f : I \to U \to \R $ es derivable en $\displaystyle t_{0} $ y además
\[\left(g\circ f\right)'\left(t_{0}\right) = \left\langle \nabla g\left(f\left(t_{0}\right)\right), f'\left(t_{0}\right) \right\rangle  .\]
\end{colorary}
\begin{proof}
Es fácil ver que
\[\left(g\circ f\right)'\left(t_{0}\right) = D\left(g\circ f\right)\left(t_{0}\right)\left(1\right) = Dg\left(f\left(t_{0}\right)\right)\left(f'\left(t_{0}\right)\right) = \left\langle \nabla g\left(f\left(t_{0}\right)\right), f'\left(t_{0}\right) \right\rangle  .\]
\end{proof}
\section{Conjuntos de nivel}
\begin{definition}[Conjuntos de nivel]
Sean $\displaystyle U \subset \R^{n} $ abierto y $\displaystyle F : U \to \R $ diferenciable en $\displaystyle U $. Se define el \textbf{conjunto de nivel} de $\displaystyle F $ correspondiente al valor $\displaystyle r \in \R $ como 
\[S_{r} = \left\{ x \in U \; : \; F\left(x\right) = r\right\}  .\]
\end{definition}
\begin{eg}
\begin{enumerate}
\item Si consideramos $\displaystyle F : \R^{2} \to \R $ con $\displaystyle F\left(x,y\right) = x^{2}+y^{2} $. Para un $\displaystyle r \in \R $ tenemos que 
	\[S_{r} = 
	\begin{cases}
	\emptyset, \; r < 0 \\ 
	\left\{ \left(0,0\right)\right\} , \; r = 0 \\
	\left\{ \left(x,y\right) \; : \; x^{2} +y^{2} = r\right\} , \; r > 0
	\end{cases}
	.\]
\item Consideremos $\displaystyle F\left(x,y\right) = xy $. Tenemos que sus curvas de nivel son hipérbolas si $\displaystyle r \neq 0 $ y los ejes de coordenadas en el caso de que $\displaystyle r = 0 $. 	
\item Si tomamos $\displaystyle F\left(x,y,z\right) = x^{2}+y^{2}+z^{2} $, obtenemos superficies esféricas para $\displaystyle r > 0 $; el origen si $\displaystyle r = 0 $; y el conjunto vacío si $\displaystyle r < 0 $.
\item Consideremos $\displaystyle F : \R^{3} \to \R $ con $\displaystyle F\left(x,y,z\right)= x^{2}+y^{2} -z^{2} $. En el caso $\displaystyle r = 0 $ tenemos el conjunto de nivel dado por la ecuación $\displaystyle z^{2} = x^{2} +y^{2} $, que es un cono. Si $\displaystyle r = 1 $, tenemos el conjunto de nivel dado por la ecuación $\displaystyle x^{2}+y^{2} = r + z^{2} $ por lo que obtenemos un hiperboloide. Si $\displaystyle r = -1 $, obtenemos el conjunto de nivel dado por la ecuación $\displaystyle x^{2} +y^{2} - r = + z^{2} $, es decir, $\displaystyle z = \pm\sqrt{x^{2}+y^{2}-r} $, que es un paraboloide.
\end{enumerate}
\end{eg}
\begin{prop}
Sea $\displaystyle U \subset \R^{n} $ abierto, $\displaystyle F : U \to \R $ diferenciable, $\displaystyle r \in \R $ y sea $\displaystyle p \in S_{r} $. Sea $\displaystyle \sigma : I \to \R^{n} $ una curva diferenciable con $\displaystyle \sigma\left(t_{0}\right) = p $ y tal que $\displaystyle \Imagen\left(\sigma \right)\subset S_{r} $. Entonces, $\displaystyle \nabla F\left(p\right) \perp \sigma'\left(t_{0}\right) $.
\end{prop}
\begin{proof}
Consideremos la composición
\[I \to ^{\sigma } S_{r} \subset U \to ^{F} \R .\]
Tenemos que $\displaystyle F\circ \sigma\left(t\right) = r $, $\displaystyle \forall t \in I $. Consecuentemente, 
\[0 = \left(F\circ \sigma \right)'\left(t\right) = \left\langle \nabla F\left(\sigma \left(t\right)\right), \sigma'\left(t\right) \right\rangle  .\]
Por lo que $\displaystyle \nabla F\left(p\right) \perp\sigma'\left(t_{0}\right) $.
\end{proof}
\begin{definition}[Hiperplano tangente]
Sean $\displaystyle U\subset\R^{n} $ abierto, $\displaystyle F:U \to \R $ y $\displaystyle r \in \R $. Sea $\displaystyle p \in S_{r} $. Si $\displaystyle \nabla F\left(p\right)\neq 0 $, se dice que $\displaystyle \nabla F\left(p\right) $ es un \textbf{vector normal} a $\displaystyle S_{r} $ en $\displaystyle p $ y se define el \textbf{hiperplano tangente} a $\displaystyle S_{r} $ en $\displaystyle p $ como el hiperplano que pasa por $\displaystyle p $ y es perpendicular a $\displaystyle \nabla F\left(p\right) $.
\end{definition}
\begin{eg}
	Consideremos $\displaystyle M = \left\{ \left(x,y,z\right) \; : \; 2x^{2} +y^{2}+3z^{2} = 6\right\}  $, que es un elipsoide. Tomamos $\displaystyle p= \left(1,2,0\right) \in M $. Calculamos el hiperplano tangente a $\displaystyle M $ en $\displaystyle p $. Tenemos que 
	\[\nabla F\left(p\right)= \left(4x,2y,6z\right)|_{p}= \left(4,4,0\right) .\]
Así, el hiperplano que buscamos es
\[4\left(x-1\right) + 4\left(y-2\right) = 0 .\]
\end{eg}
\subsection*{Superficies explícitas}
Una superficie es explícita si tiene la forma $\displaystyle z = f\left(x,y\right) $ con $\displaystyle f $ diferenciable. Una forma de calcular el hiperplano tangente es definir $\displaystyle F\left(x,y,z\right)= f\left(x,y\right) -z $. Así, tenemos que
\[z = f\left(x,y\right) \iff F\left(x,y,z\right) = 0 .\]
Por tanto, toda superficie explícita puede considerarse una superficie implícita. Tendremos que
\[\nabla F\left(x_{0}, y_{0},z_{0}\right) = \left(\frac{\partial f}{\partial x}\left(x_{0}, y_{0}\right), \frac{\partial f}{\partial y}\left(x_{0}, y_{0}\right), -1\right) .\]
Tomando $\displaystyle z_{0} = f\left(x_{0}, y_{0}\right) $ tendremos que el plano tangente en $\displaystyle p=\left(x_{0}, y_{0}, z_{0}\right) $ será
\[\left(x-x_{0}\right)\frac{\partial f}{\partial x}\left(x_{0}, y_{0}\right) + \left(y - y_{0}\right)\frac{\partial f}{\partial y}\left(x_{0}, y_{0}\right) - \left(z - z_{0}\right) = 0.\]
Es decir,
\[z = f\left(x_{0}, y_{0}\right) + \left(x-x_{0}\right)\frac{\partial f}{\partial x}\left(x_{0}, y_{0}\right) + \left(y - y_{0}\right)\frac{\partial f}{\partial y}\left(x_{0}, y_{0}\right) .\]
Otra forma de calcular el hiperplano tangente es, dada $\displaystyle f : U\subset \R^{2} \to \R $ diferenciable, considerar su gráfica $\displaystyle z = f\left(x,y\right) $. Dado $\displaystyle p=\left(x_{0},y_{0}\right) \in \Imagen\left(z\right) $, podemos considerar la curva
\[\sigma\left(t\right) = \left(x_{0}+t, y_{0}, f\left(x_{0}+t,y_{0}\right)\right) , \; \sigma\left(0\right) = p\]
Así, tenemos que 
\[\sigma'\left(0\right) = \left(1,0, \frac{\partial f}{\partial x}\left(x_{0}, y_{0}\right)\right).\]
De manera análoga, podemos fijar $\displaystyle x_{0} $ y obtener la curva
\[\gamma \left(t\right)= \left(x_{0}, y_{0}+t, f\left(x_{0}, y_{0}+t\right)\right), \; \gamma\left(0\right) = t .\]
Así, tendremos que el vector tangente a la superficie será 
\[\gamma'\left(0\right) = \left(0,1,\frac{\partial f}{\partial y}\left(x_{0}, y_{0}\right)\right) .\]
Así, el plano tangente será el que viene dado por las ecuaciones
\[
\begin{cases}
x = x_{0} + \lambda \\
y = y_{0} + \mu \\
z = f\left(x_{0}, y_{0}\right) + \lambda \frac{\partial f}{\partial x}\left(x_{0}, y_{0}\right) + \mu \frac{\partial f}{\partial y}\left(x_{0}, y_{0}\right)
\end{cases}
.\]
Claramente, obtenemos el mismo plano por ambos métodos.
\section{Teorema del valor medio}
\begin{definition}[Segmento]
Dados $\displaystyle x,y \in \R^{n} $, el \textbf{segmento} que los une es
\[\left[x,y\right]  = \left\{ \left(1-t\right)x + ty \; : \; 0 \leq t \leq 1\right\}  .\]
\end{definition}
\begin{theorem}[Teorema del valor medio]
	Sea $\displaystyle U \subset \R^{n} $ abierto, $\displaystyle f : U \to \R $ diferenciable en $\displaystyle U $, y supongamos que $\displaystyle \left[x,y\right]  \subset U $. Entonces, existe $\displaystyle z \in \left[x,y\right]  $ tal que 
	\[f\left(y\right)-f\left(x\right) = \left\langle \nabla f\left(z\right), y - x \right\rangle = Df\left(z\right) \left(y-x\right) .\]
\end{theorem}
\begin{proof}
	Consideramos $\displaystyle \sigma : \left[0,1\right] \to U $, $\displaystyle \sigma\left(t\right) = \left(1-t\right)x +ty = x + t\left(y - x\right)$ es derivable en $\displaystyle \left[0,1\right]  $ y además $\displaystyle \sigma'\left(t\right) = y - x $, $\displaystyle \forall t \in [0,1] $. Consideremos $\displaystyle g = f \circ \sigma : \left[0,1\right] \to \R $ donde $\displaystyle g\left(t\right)=f\left(\sigma\left(t\right)\right) = f\left(\left(1-t\right)x + ty\right) $. Por la regla de la cadena tenemos que $\displaystyle g $ es derivable en $\displaystyle [0,1] $ y
	\[g'\left(t\right)=\left\langle \nabla f\left(\sigma\left(t\right)\right), \sigma'\left(t\right) \right\rangle = \left\langle \nabla f\left(\left(1-t\right)x + ty\right), y-x \right\rangle  .\]
Aplicando el teorema del valor medio en $\displaystyle \R $, tenemos que existe $\displaystyle c \in \left(0,1\right) $ tal que 
\[g\left(1\right) -g\left(0\right) = g'\left(c\right) \iff f\left(y\right)-f\left(x\right) = \left\langle \nabla f\left(\left(1-c\right)x + cy\right), y-x \right\rangle  .\]
Llamando $\displaystyle z = \left(1-c\right)x + cy \in \left[x,y\right]  $, tenemos que 
\[f\left(x\right)-f\left(y\right) = \left\langle \nabla f\left(z\right), y - x \right\rangle  .\]
\end{proof}
\begin{colorary}
Sean $\displaystyle U \subset \R^{n} $ abierto convexo, $\displaystyle f : U \to \R $ diferenciable en $\displaystyle U $ y supongamos que $\displaystyle \|\nabla f\left(z\right) \| \leq K $, $\displaystyle \forall z \in U $ \footnote{Consideramos la norma euclídea.}. Entonces
\[|f\left(x\right)-f\left(y\right)|\leq K \|x-y\|, \; \forall x,y \in U .\]
\end{colorary}
\begin{proof}
	Dados $\displaystyle x,y \in U $, tenemos que $\displaystyle \left[x,y\right] \subset U $ y por el teorema anterior existe $\displaystyle z \in \left[x,y\right]  $ tal que, aplicando la desigualdad Cauchy-Swartz,
	\[|f\left(y\right)-f\left(x\right)| = |\left\langle \nabla f\left(z\right), y-x \right\rangle| \leq \|\nabla f\left(z\right)\| \|y -x\| \leq K \|y-x\|  .\]
\end{proof}
\begin{colorary}
Sean $\displaystyle U \subset \R^{n} $ abierto convexo y $\displaystyle f : U \to \R $ diferenciable en $\displaystyle U $ tal que $\displaystyle \nabla f\left(z\right) = 0 $, $\displaystyle \forall z \in U $. Entonces $\displaystyle f $ es constante en $\displaystyle U $.
\end{colorary}
\begin{proof}
Tenemos que $\displaystyle \forall x,y \in U $, 
\[ \left|f\left(x\right)-f\left(y\right)\right| \leq \|\nabla f\left(z\right) \| \|x-y\| = 0 .\]
Así, tenemos que $\displaystyle f\left(x\right) = f\left(y\right) $.
\end{proof}
\begin{observation} %esta observación hay que conectarla con una observación anterior
Sea $\displaystyle U \subset \R^{n} $ abierto conexo y $\displaystyle f : U \to \R $ diferenciable en $\displaystyle U $ tal que $\displaystyle \nabla f\left(z\right) = 0 $, $\displaystyle \forall z \in U $. Entonces $\displaystyle f $ es constante en $\displaystyle U $. 
\end{observation}
\begin{proof}
	Sea $\displaystyle x_{0} \in U $, tenemos que existe $\displaystyle r > 0 $ tal que $\displaystyle B\left(x_{0}, r\right)\subset U $, que es convexo, por lo que $\displaystyle f $ es constante en $\displaystyle B\left(x,r\right) $. Sea $\displaystyle A = \left\{ x \in U \; : \; f\left(x\right) = f\left(x_{0}\right)\right\}  $. 
	\begin{itemize}
	\item Tenemos que $\displaystyle A \neq \emptyset $ puesto que $\displaystyle x_{0} \in A $. 
	\item Tenemos que $\displaystyle A $ es abierto en $\displaystyle U $. En efecto, dado $\displaystyle x \in A $, existe $\displaystyle r > 0 $ tal que $\displaystyle B\left(x,r\right) \subset U $ y por el corolario anterior tenemos que $\displaystyle f $ es constante en $\displaystyle B\left(x,r\right) $ por ser esta convexa. Así, $\displaystyle f\left(y\right) = f\left(x_{0}\right) $, $\displaystyle \forall y \in B\left(x,r\right) $ por lo que $\displaystyle B\left(x,r\right)\subset A $.
	\item Análogamente, tenemos que $\displaystyle A $ es cerrado en $\displaystyle U $. Veamos que $\displaystyle U/A $ es abierto. Dado $\displaystyle x \in U/A $ tenemos que $\displaystyle f\left(x\right) \neq f\left(x_{0}\right) $. Existe $\displaystyle r > 0 $ tal que $\displaystyle B\left(x,r\right)\subset U $ y sabemos que $\displaystyle f|_{B\left(x,r\right)} $ es constante. Por tanto, debe ser que $\displaystyle f\left(y\right) = f\left(x\right) \neq f\left(x_{0}\right) $, $\displaystyle \forall y \in B\left(x,r\right) $. Por lo que $\displaystyle B\left(x,r\right) \subset U/A $.
	\end{itemize}
	Por conexión debe ser que $\displaystyle A = U $. 
\end{proof}

\begin{theorem}[Desigualdad del valor medio]
	Sea $\displaystyle U \subset \R^{n} $ abierto, $\displaystyle f : U \to \R^{m} $ diferenciable en $\displaystyle U $. Supongamos que $\displaystyle \left|\frac{\partial f_{j}}{\partial x_{i}}\left(z\right)\right|\leq K $, $\displaystyle \forall i = 1, \ldots, n $, $\displaystyle \forall j = 1, \ldots, m $, $\displaystyle \forall z \in [x,y] \subset U $. Entonces \footnote{Consideramos la norma euclídea.},
	\[\|f\left(x\right)-f\left(y\right)\| \leq K\sqrt{n \cdot m}\|y-x\| .\]
\end{theorem}
\begin{proof}
	Tenemos que $\displaystyle \forall j= 1, \ldots, m $, $\displaystyle f_{j} : U \to \R $ es diferenciable, luego por el teorema del valor medio existe $\displaystyle z_{j} \in [x,y] $ tal que 
	\[|f_{j}\left(y\right)-f_{j}\left(x\right)| = |\left\langle \nabla f_{j}\left(z_{j}\right), y-x \right\rangle| \leq \|\nabla f_{j}\left(z_{j}\right)\| \|y - x\|  .\]
Por otro lado, tenemos que
\[\|\nabla f_{j}\left(z_{j}\right)\|^{2} = \sum^{n}_{i = 1} \left|\frac{\partial f_{j}}{\partial x_{i}}\left(z_{j}\right)\right|^{2} \leq nK^{2} .\]
Así, la desigualdad anterior nos queda
\[ \left|f_{j}\left(y\right)-f_{j}\left(x\right)\right| \leq \sqrt{n}K \|x-y\| .\]
Así, tenemos que 
\[\|f\left(y\right)-f\left(x\right)\|^{2} = \sum^{m}_{j = 1} \left|f_{j}\left(y\right)-f_{j}\left(x\right)\right|^{2} \leq \sum^{m}_{j = 1}nK^{2}\|y -x\|^{2} = nmK^{2}\|y-x\|^{2} .\]
Quitando los cuadrados obtenemos el resultado deseado. 
\end{proof}
\begin{colorary}
Sean $\displaystyle U \subset \R^{n} $ abierto convexo, $\displaystyle f : U \to \R^{m} $ diferenciable en $\displaystyle U $, y supongamos que $\displaystyle \left|\frac{\partial f_{j}}{\partial x_{i}}\left(z\right)\right| \leq K $, $\displaystyle \forall j = 1, \ldots, m $, $\displaystyle \forall i = 1, \ldots, n $, $\displaystyle \forall z \in U $. Entonces, 
\[\|f\left(x\right)-f\left(y\right)\| \leq K\sqrt{nm}\|x-y\|, \; \forall x,y \in U .\]
\end{colorary}
\section{Derivadas de orden superior}
% quizás es nuevo capítulo?
\begin{eg}
Consideremos $\displaystyle f\left(x,y,z\right)= x^{3}y^{2} + xz + \sin\left(x-y^{2}\right) $. Tenemos que 
\[\frac{\partial f}{\partial x} = 3x^{2}y^{2}+z+ \cos\left(x-y^{2}\right) .\]
Así, podemos calcular las derivadas:
\[\frac{\partial}{\partial y}\left(\frac{\partial f}{\partial x}\right)= \frac{\partial ^{2}f}{\partial y\partial x} = 6x^{2}y + 2y\sin\left(x-y^{2}\right) .\]
\[\frac{\partial^{2} f}{\partial x \partial y} = \frac{\partial }{\partial x}\left(\frac{\partial f}{\partial y}\right)=6x^{2}y +2y\sin\left(x-y^{2}\right) .\]
\end{eg}
\begin{definition}[Derivadas parciales de orden 2]
Sean $\displaystyle U \subset \R^{n} $ abierto, consideremos $\displaystyle f : U \to \R $ y $\displaystyle a \in U $. Se define, cuando exixte, la \textbf{derivada parcial de orden 2} de la forma
\[\frac{\partial^{2}f}{\partial x_{i}x_{j}}\left(a\right) = \frac{\partial }{\partial x_{i}}\left(\frac{\partial f}{\partial x_{j}}\left(a\right)\right), \; \forall i,j = 1, \ldots, n .\]
\end{definition}
\begin{definition}
Sea $\displaystyle U \subset \R^{n} $ abierto y $\displaystyle f : U \to \R $. Se dice que $\displaystyle f $ es de \textbf{clase} $\displaystyle \mathcal{C}^{2} $ en $\displaystyle U $, y se denota, $\displaystyle f \in \mathcal{C}^{2}\left(U\right) $, cuando $\displaystyle f $ admite todas las derivadas parciales segundas en $\displaystyle U $ y todas son continuas en $\displaystyle U $.
\end{definition}
\begin{theorem}[Teorema de Schwarz]
Sean $\displaystyle U\subset \R^{n} $ abierto y $\displaystyle f \in \mathcal{C}^{2}\left(U\right) $. Entonces, $\displaystyle \forall i,j = 1, \ldots, n $, $\displaystyle \forall a \in U $, se tiene que 
\[\frac{\partial^{2}f}{\partial x_{i}\partial x_{j}}\left(a\right)= \frac{\partial^{2}f}{\partial x_{j}\partial x_{i}} \left(a\right).\]
\end{theorem}
\begin{proof}
Podemos limitarnos al caso $\displaystyle n = 2 $. Así, supongamos que $\displaystyle U \subset \R^{2} $ y $\displaystyle a = \left(x_{0}, y_{0}\right) \in U $. Sea $\displaystyle r > 0 $ tal que $\displaystyle B_{\infty}\left(\left(x_{0}, y_{0}\right), r\right) \subset U $ \footnote{Recordamos que $\displaystyle B_{\infty}\left(\left(x_{0}, y_{0}\right), r\right) = \left(x_{0}-r, x_{0}+r\right) \times \left(y_{0}-r, y_{0}+r\right) $.}. 
Dados $\displaystyle \left(h,k\right) \in B_{\infty}\left(\left(0,0\right), r\right) $, es decir, $\displaystyle \left|h\right|, \left|k\right| < r $, obtenemos que el rectángulo $\displaystyle R_{\left(h,k\right)} $ de vértices $\displaystyle \left(x_{0}, y_{0}\right) $, $\displaystyle \left(x_{0}+h, y_{0}\right) $, $\displaystyle \left(x_{0}, y_{0}+k\right) $ y $\displaystyle \left(x_{0}+h, y_{0}+k\right) $ está contenido en $\displaystyle U $.
Consideramos 
\[S\left(h,k\right) = f\left(x_{0}+h, y_{0}+k\right)-f\left(x_{0}+h, y_{0}\right) - f\left(x_{0}, y_{0}+k\right)+f\left(x_{0},y_{0}\right) .\]
Consideremos $\displaystyle \varphi_{k}\left(t\right)= f\left(t,y_{0}+k\right) -f\left(t,y_{0}\right) $, así, tenemos que
\[S\left(h,k\right)=\varphi_{k}\left(x_{0} + h\right)-\varphi\left(x_{0}\right) = \varphi_{k}'\left(\alpha\left(h,k\right)\right)h .\]
En efecto, tenemos que 
\[\varphi'_{k}\left(t\right)= \frac{\partial f}{\partial x}\left(t, y_{0}+k\right) - \frac{\partial f}{\partial x}\left(t, y_{0}\right) ,\]
que existe por hipótesis. Así, nos queda que 
\[S\left(h,k\right)= h \left[\frac{\partial f}{\partial x}\left(\alpha\left(h,k\right), y_{0}+k\right) - \frac{\partial f}{\partial x}\left(\alpha\left(h,k\right), y_{0}\right)\right]  ,\]
donde $\displaystyle \alpha\left(h,k\right) $ es un punto intermedio entre $\displaystyle x_{0} $ y $\displaystyle x_{0}+h $. Denotamos $\displaystyle g\left(t\right)= \frac{\partial f}{\partial x}\left(\alpha\left(h,k\right), t\right) $, y obtenemos que 
\[S\left(h,k\right)= h \left[g\left(y_{0}+k\right) -g\left(y_{0}\right)\right]  .\]
Por otro lado, 
\[g'\left(t\right)= \frac{\partial }{\partial y}\left(\frac{\partial f}{\partial x}\right)\left(\alpha\left(h,k\right), t\right) .\]
Aplicando nuevamente el teorema del valor medio obtenemos que existe $\displaystyle \beta\left(h,k\right) $ punto intermedio entre $\displaystyle y_{0} $ e $\displaystyle y_{0}+k $,
\[ S\left(h,k\right)= hkg'\left(\beta\left(h,k\right)\right) .\]
Por tanto, para cada par $\displaystyle \left(h,k\right) \in B_{\infty}\left(\left(0,0\right), r\right) $ existen puntos intermedios $\displaystyle \left(\alpha\left(h,k\right), \beta\left(h,k\right)\right) \in R\left(h,k\right) $ tales que 
\[\frac{S\left(h,k\right)}{hk} = \frac{\partial^{2}f}{\partial y \partial x}\left(\alpha\left(h,k\right), \beta\left(h,k\right)\right) \to \frac{\partial^{2}f}{\partial y\partial x}\left(x_{0}, y_{0}\right) ,\]
cuando $\displaystyle \left(h,k\right)\to \left(0,0\right) $, por lo que $\displaystyle \left(\alpha\left(h,k\right), \beta\left(h,k\right)\right)\to \left(x_{0}, y_{0}\right) $, puesto que $\displaystyle f \in \mathcal{C}^{2}\left(U\right) $.
De forma análoga, podemos plantear $\displaystyle \psi_{h}\left(t\right) = f\left(x_{0}+h, t\right)-f\left(x_{0}, t\right) $, obteniendo
\[\psi_{h}'\left(t\right)= \frac{\partial f}{\partial y}\left(x_{0}+h, t\right)-\frac{\partial f}{\partial y}\left(x_{0}, t\right) .\]
Aplicando el teorema del valor medio, existe $\displaystyle \tilde{\beta}\left(h,k\right) $ un punto intermedio entre $\displaystyle y_{0} $ e $\displaystyle y_{0} + k $ tal que
\[
\begin{split}
	S\left(h,k\right) = \psi_{h}\left(y_{0}+k\right) - \psi\left(y_{0}\right) = k \psi'_{h}\left(\tilde{\beta}\left(h,k\right)\right) = k \left[\frac{\partial f}{\partial y}\left(x_{0}+h, \tilde{\beta}\left(h,k\right)\right)- \frac{\partial f}{\partial y}\left(x_{0}, \tilde{\beta}\left(h,k\right)\right)\right]  .
\end{split}
\]
Considerando $\displaystyle \tilde{g} = \frac{\partial f}{\partial y}\left(t, \tilde{\beta}\left(h,k\right)\right) $, tenemos que 
\[S\left(h,k\right) = k \left[\tilde{g}\left(x_{0}+h\right)-\tilde{g}\left(x_{0}\right)\right] = kh \frac{\partial^{2}f}{\partial x\partial y}\left(\tilde{\alpha}\left(h,k\right), \tilde{\beta}\left(h,k\right)\right),\]
donde $\displaystyle \tilde{\alpha}\left(h,k\right) $ es un punto intermedio entre $\displaystyle x_{0} $ y $\displaystyle x_{0}+h $. Así, obtenemos que 
\[\lim_{\left(h,k\right) \to \left(0,0\right)}\frac{S\left(h,k\right)}{hk} = \lim_{\left(h,k\right) \to \left(0,0\right)} \frac{\partial^{2}f}{\partial x\partial y}\left(\tilde{\alpha}\left(h,k\right), \tilde{\beta}\left(h,k\right)\right) = \frac{\partial f^{2}}{\partial x \partial y}\left(x_{0}, y_{0}\right) .\]
Nuevamente hemos usado la continuidad de $\displaystyle \frac{\partial^{2}f}{\partial x \partial y} $. 
\end{proof}
\begin{definition}
	Sean $\displaystyle U \subset \R^{n} $ abierto, $\displaystyle f : U \to \R $ y $\displaystyle a \in U $. Dados $\displaystyle i_{1}, \ldots, i_{k} \in \left\{ 1, \ldots, n\right\}  $ se define recursivamente, cuando existe
	\[\frac{\partial ^{k}f}{\partial x_{i_{1}} \cdots \partial x_{i_{k}}}\left(a\right) = \frac{\partial }{\partial x_{i_{1}}}\left(\frac{\partial^{k-1}f}{\partial x_{i_{2}} \cdots \partial x_{i_{k}}} \right)\left(a\right) .\]	
\end{definition}
\begin{definition}
Se dice que $\displaystyle f $ es de clase $\displaystyle \mathcal{C}^{k} $ en $\displaystyle U $, y se denota $\displaystyle f \in \mathcal{C}^{k}\left(U\right) $, si existen todas las derivadas parciales de orden $\displaystyle k $ en $\displaystyle U $ y son continuas. Análogamente, decimos que $\displaystyle f \in \mathcal{C}^{\infty}\left(U\right) $ si $\displaystyle f \in \mathcal{C}^{k}\left(U\right) $, $\displaystyle \forall k \in \N $.
\end{definition}
\begin{colorary}
	Sean $\displaystyle U\subset \R^{n} $ abierto y $\displaystyle f : U \to \R $ de clase $\displaystyle \mathcal{C}^{k} $ en $\displaystyle U $. Sea $\displaystyle i_{1}, \ldots, i_{k} \in \left\{ 1, \ldots, n\right\}  $ y sea $\displaystyle \left(i_{1}', \ldots, i_{k}'\right) $ una permutación de $\displaystyle \left(i_{1}, \ldots, i_{k}\right) $. Entonces,
	\[\frac{\partial ^{k}f}{\partial x_{i_{1}} \cdots \partial x_{i_{k}}} = \frac{\partial ^{k}f}{\partial x_{i'_{1}} \cdots \partial x_{i'_{k}}}, \; \text{en }U .\]
\end{colorary}
\begin{proof}
Sabemos que toda permutación es composición de trasposiciones. Por tanto, trasponiendo $\displaystyle \left(i_{1}, \ldots, i_{k}\right) $ podemos pasar de $\displaystyle \left(i_{1}, \ldots, i_{k}\right) $ a $\displaystyle \left(i_{1}', \ldots, i_{k'}\right) $ en una candidad finita de pasos en los cuales intercambiamos dos índices entre sí. 
\end{proof}
\begin{definition}
	Sea $\displaystyle U\subset \R^{n} $ abierto y sea $\displaystyle f : U \to \R^{m} $. Se dice que $\displaystyle f $ es de \textbf{clase $\displaystyle \mathcal{C}^{k} $} en $\displaystyle U $, y se denota $\displaystyle f \in \mathcal{C}^{k}\left(U, \R^{m}\right) $, si $\displaystyle \forall j = 1, \ldots, m $ se tiene que $\displaystyle f_{j} \in \mathcal{C}^{k}\left(U\right) $.
\end{definition}
\begin{observation}
Es fácil comprobar que 
\[\mathcal{C}^{k}\left(U, \R^{m}\right) \subset \mathcal{C}^{k-1}\left(U, \R^{m}\right) \subset \cdots \subset \mathcal{C}\left(U, \R^{m}\right) ,\]
donde $\displaystyle \mathcal{C}\left(U,\R^{m}\right) $ representa el conjunto de las funciones continuas de $\displaystyle U $ en $\displaystyle \R^{m} $.
\end{observation}
\begin{lema}
Sean $\displaystyle U\subset \R^{n} $ abierto y $\displaystyle f : U \to \R $. Son equivalentes, para $\displaystyle k \in \N $ con $\displaystyle k \geq 2 $:
\begin{enumerate}
\item $\displaystyle f\in \mathcal{C}^{k}\left(U\right) $.
\item $\displaystyle \forall i= 1, \ldots, n $, $\displaystyle \exists \frac{\partial f}{\partial x_{i}} \in \mathcal{C}^{k-1}\left(U\right) $.
\end{enumerate}
\end{lema}
\begin{proof}
\begin{description}
	\item[(i)] Tenemos que $\displaystyle \forall i_{1}, \ldots, i_{k-1} \in \left\{ 1, \ldots, n\right\}  $,
		\[\frac{\partial^{k}f}{\partial x_{i_{1}} \cdots \partial x_{i_{k-1}}\partial x_{i}} = \frac{\partial ^{k-1}}{\partial x_{i_{1}} \cdots \partial x_{i_{k-1}}}\left(\frac{\partial f}{\partial x_{i}}\right) ,\]
estas últimas existen y son continuas $\displaystyle \forall i = 1, \ldots, n $.
\item[(ii)] Por otro lado, $\displaystyle \forall i_{1}, \ldots, i_{k} \in \left\{ 1, \ldots, n\right\}  $, tenemos que 
	\[\frac{\partial^{k}f}{\partial x_{i_{1}} \cdots \partial x_{i_{k}}} = \frac{\partial ^{k-1}}{\partial x_{i_{1}} \cdots \partial x_{i_{k-1}}}\left(\frac{\partial f}{\partial x_{i_{k}}}\right) ,\]
estas últimas existen y son continuas.	
\end{description}
\end{proof}
\begin{theorem}[Regla de la cadena - $\displaystyle \mathcal{C}^r $]
	Esto es cierto si $\displaystyle r \in \N \cup \left\{ \infty\right\}  $. Sean $\displaystyle U \subset \R^{n} $ abierto, $\displaystyle V \subset \R^{m} $ abierto, $\displaystyle f : U \to \R^{m} $ con $\displaystyle f\left(U\right) \subset V $; $\displaystyle g : V \to \R^{p} $. 
	Si $\displaystyle f \in \mathcal{C}^{r}\left(U\right)$ y $\displaystyle g \in \mathcal{C}^{r}\left(V\right) $, tenemos que $\displaystyle g\circ f \in \mathcal{C}^{r}\left(U\right) $.
\end{theorem}
\begin{proof}
Consideremos varios casos:
\begin{itemize}
\item Para $\displaystyle r \in \N $, aplicamos inducción sobre $\displaystyle r $. 
	\begin{itemize}
		\item Si $\displaystyle r = 1 $, tenemos que $\displaystyle f,g \in \mathcal{C}^{1} $. Por tanto, tenemos que $\displaystyle f $ y $\displaystyle g $ son diferenciables en $\displaystyle U $ y $\displaystyle V $, respectivamente, luego por la regla de la cadena tenemos que $\displaystyle h = g\circ f $ es diferenciable y $\displaystyle \forall k \in\left\{ 1, \ldots, p\right\}  $, $\displaystyle \forall i \in \left\{ 1, \ldots, n\right\}  $, como $\displaystyle J\left(g\circ f\right)\left(x\right) = Jg\left(f\left(x\right)\right) \cdot Jf\left(x\right) $,
			\[\frac{\partial h_{k}}{\partial x_{i}}\left(x\right) = \sum^{m}_{j = 1}\frac{\partial g_{k}}{\partial y_{j}}\left(f\left(x\right)\right) \cdot \frac{\partial f_{j}}{\partial x_{i}}\left(x\right) = \sum^{m}_{j = 1}\left(\frac{\partial g_{k}}{\partial y_{j}}\circ f\right)\left(x\right) \cdot \frac{\partial f_{j}}{\partial x_{i}}\left(x\right) ,\]
			que es continua.
		\item Supongamos que la composición de funciones en $\displaystyle \mathcal{C}^{r} $ está también en $\displaystyle \mathcal{C}^{r} $. Veamos que pasa con $\displaystyle r + 1 $. Sean $\displaystyle f,g \in \mathcal{C}^{r+1} $, tenemos que las derivadas parciales de $\displaystyle f $ y de $\displaystyle g $ son $\displaystyle \mathcal{C}^{r} $, por el lema anterior. Es fácil aplicar la fórmula anterior para obtener el resultado deseado.			
	\end{itemize}
\item El caso $\displaystyle r = \infty $, tenemos que $\displaystyle f, g \in \mathcal{C}^{\infty} $, que es cierto si y solo si $\displaystyle \forall r \in \N $ se tiene que $\displaystyle f,g \in \mathcal{C}^{r} $. Por tanto, $\displaystyle g\circ f \in \mathcal{C}^{r} $, $\displaystyle \forall r \in \N $, que es lo mismo que decir que $\displaystyle g\circ f \in \mathcal{C}^{\infty} $. 
\end{itemize}
\end{proof}
\begin{eg}[Ejemplos de funciones de $\displaystyle \mathcal{C}^{\infty}$] Algunos ejemplos ilustrativos:
	\begin{itemize}
	\item Polinomios en varias variables:
		\[f\left(x,y,z\right)= x + 5y^{2} + 3xyz -2x^{3}+7yz^{2} .\]
	\item $\displaystyle f\left(x,y\right) = \sin\left(x^{2}-y^{2}+1\right) $.
	\item $\displaystyle f\left(x,y,z\right)=e^{x+y+z} $.
	\end{itemize}
\end{eg}
\section{Diferenciales de orden superior}
\begin{definition}[Diferencial segunda]
Sean $\displaystyle U \subset \R^{n} $ abierto y $\displaystyle f \in \mathcal{C}^{2}\left(U, \R\right) $. Para cada $\displaystyle a \in U $, se define la \textbf{diferencial segunda} de $\displaystyle f $ en $\displaystyle a $ como la forma cuadrática $\displaystyle D^{2}f\left(a\right) : \R^{n} \to \R $ definida por 
\[D^{2}f\left(a\right)\left(h\right) = \sum^{n}_{i,j = 1}\frac{\partial^{2}f\left(a\right)}{\partial x_{i}\partial x_{j}}h_{i}h_{j} .\]
Es decir, 
\[Df^{2}\left(a\right) \left(h_{1}, \ldots h_{n}\right) = \left(h_{1}, \ldots, h_{n}\right)\begin{pmatrix} \frac{\partial^{2}f\left(a\right)}{\partial x_{1}\partial x_{n}} & \cdots & \frac{\partial^{2}f\left(a\right)}{\partial x_{1}\partial x_{n}} \\ 
\vdots & & \vdots \\
\frac{\partial^{2}f\left(a\right)}{\partial x_{n}\partial x_{1}} & \cdots & \frac{\partial^{2}f\left(a\right)}{\partial x_{n}\partial x_{n}}\end{pmatrix}\begin{pmatrix} h_{1} \\ \vdots \\ h_{n}\end{pmatrix} .\]
\end{definition}
\begin{definition}[Polinomio homogéneo de grado $\displaystyle k $]
Un \textbf{polinomio homogéneo de grado $\displaystyle k $} en $\displaystyle \R^{n} $ es una función $\displaystyle P : \R^{n} \to \R $ de la forma:
\[P\left(h\right) = \sum^{n}_{i_{1}, \ldots, i_{k} = 1}a_{i_{1}, \ldots, i_{k}}h_{i_{1}}\cdots h_{i_{k}} ,\]
donde $\displaystyle a_{i_{1}}, \ldots, a_{i_{k}} \in \R $, $\displaystyle \forall i_{1}, \ldots, i_{k} \in \left\{ 1, \ldots, n\right\}  $.
\end{definition}
\begin{definition}[Diferencial $\displaystyle k $-ésima]
Sea $\displaystyle U \subset \R^{n} $ aiberto y $\displaystyle f \in \mathcal{C}^{k}\left(U,\R\right) $ donde $\displaystyle k \in \N $. Para cada $\displaystyle a \in U $ se define la \textbf{diferencial $\displaystyle k $-ésima} de $\displaystyle f $ en $\displaystyle a $ como el polinomio homogéneo de grado $\displaystyle k $ dado por
\[D^{k}f\left(a\right)\left(h\right) = \sum^{n}_{i_{1}, \ldots, i_{k}= 1}\frac{\partial^{k}f\left(a\right)}{\partial x_{i_{1}} \cdots \partial x_{i_{k}}}h_{i_{1}} \cdots h_{i_{k}} .\]
\end{definition}
\begin{eg}
Consideremos $\displaystyle f\left(x,y\right) = e^{2x+3y} $ y $\displaystyle a = \left(0,0\right) $. Es fácil ver que $\displaystyle f\in \mathcal{C}^{\infty}\left(\R^{2}\right) $. Tenemos que 
\[\frac{\partial f}{\partial x}\left(a\right) = 2e^{2x+3y}|_{a} = 2, \; \frac{\partial f}{\partial y}\left(a\right) = 3e^{2x+3y}|_{a} = 3 .\]
Así, tendremos que la diferencial es $\displaystyle Df\left(a\right)\left(h_{1}, h_{2}\right) = 2h_{1} + 3h_{2} $. Ahora calculamos la diferencial de segundo orden:
\[\frac{\partial^{2}f}{\partial x^{2}}\left(a\right) = 4, \; \frac{\partial^{2}f}{\partial x\partial y}\left(a\right) = \frac{\partial^{2}f}{\partial y \partial x}\left(a\right) = 6, \; \frac{\partial^{2} f}{\partial y^{2}}\left(a\right) = 9 .\]
Así, nos queda 
\[D^{2}f\left(a\right) \left(h_{1}, h_{2}\right) = 4h_{1}^{2}+12h_{1}h_{2}+9h^{2}_{2} .\]
Ahora calculamos la diferencial de tercer orden:
\[\frac{\partial^{3}f}{\partial x^{3}}\left(a\right) = 8, \; \frac{\partial^{3}f}{\partial x^{2}\partial y}\left(a\right) = 12, \; \frac{\partial^{3}f}{\partial x\partial y^{2}}\left(a\right) = 18, \; \frac{\partial^{3}f}{\partial y^{3}}\left(a\right) = 27 .\]
Así, tenemos que 
\[Df^{3}\left(a\right)\left(h_{1}, h_{2}\right) = 8h_{1}^{3} + 3 \cdot 12 h_{1}^{2}h_{2} + 3 \cdot 18 h_{1}h_{2}^{2} + 27h_{2}^{3} .\]
\end{eg}
\begin{theorem}[Teorema de Taylor]
	Sea $\displaystyle U \subset \R^{n} $ abierto, $\displaystyle a \in U $, $\displaystyle f \in \mathcal{C}^{k}\left(U\right) $ donde $\displaystyle k \in \N $. Si $\displaystyle \left[a, a+h\right] \subset U $, entonces existe $\displaystyle \theta \in \left[0,1\right]  $ tal que 
	\[
	\begin{split}
		f\left(a+h\right) = & f\left(a\right) + Df\left(a\right)\left(h\right) + \frac{1}{2}D^{2}f\left(a\right)\left(h\right) \\
		+ &  \cdots + \frac{1}{\left(k-1\right)!}D^{k-1}\left(a\right)\left(h\right)+\frac{1}{k!}D^{k}f\left(a+\theta h\right)\left(h\right).
	\end{split}
	\]
\end{theorem}
\begin{proof}
Aplicamos el teorema de Taylor con resto de Lagrange en una variable. Denotamos $\displaystyle \varphi\left(t\right)= a + th $ para $\displaystyle t \in \R $, donde $\displaystyle \varphi\left(0\right) = a $ y $\displaystyle \varphi\left(1\right) = a+h $. Consideramos la función $\displaystyle g\left(t\right)=\left(f\circ \varphi\right)\left(t\right) = f\left(\varphi\left(t\right)\right) $. 
Como $\displaystyle f \in \mathcal{C}^{k} $ y $\displaystyle \varphi \in \mathcal{C}^{\infty} $, tenemos que $\displaystyle g \in \mathcal{C}^{k} $ en un abierto que contiene a $\displaystyle \left[0,1\right]  $ \footnote{Estamos hablando del conjunto $\displaystyle \varphi^{-1}\left(U\right) $.}. Así, podemos aplicar a $\displaystyle g $ el teorema de Taylor con resto de Lagrange:
\[
\begin{split}
	g\left(1\right) = g\left(0\right) + g'\left(0\right) \cdot 1 + \cdots + \frac{1}{\left(k-1\right)!}g^{ \left(k-1\right)}\left(0\right) \cdot 1 + \frac{1}{k!}g^{\left(k\right)}\left(\theta \right) \cdot 1.
\end{split}
\]
Por otro lado, tenemos que
\[g'\left(0\right) = \left(f\circ\varphi\right)'\left(0\right) = \left\langle \nabla f\left(\varphi\left(0\right)\right), \varphi'\left(0\right) \right\rangle = \sum^{n}_{i = 1}\frac{\partial f\left(a\right)}{\partial x_{i}}h_{i} .\]
\[
\begin{split}
	g''\left(t\right)= & \left(f\circ\varphi\right)''\left(t\right) = \sum^{n}_{i = 1}h_{i}\left(\sum^{n}_{j = 1}\frac{\partial }{\partial x_{j}}\left(\frac{\partial f}{\partial x_{i}}\left(\varphi\left(t\right)\right)\right)h_{j}\right) = \sum^{n}_{i,j = 1}\frac{\partial ^{2}f\left(\varphi\left(t\right)\right)}{\partial x_{i}\partial x_{j}}h_{i}h_{j}.
\end{split}
\]
Por tanto, $\displaystyle g''\left(0\right) = D^{2}f\left(a\right)\left(h\right) $. Por inducción se demuestra que $\displaystyle \forall 2 \leq j \leq k $, se cumple que 
\[g^{\left(j\right)}\left(t\right) = \sum^{n}_{i_{1}, \ldots, i_{j} = 1}\frac{\partial^{j}\left(\varphi\left(t\right)\right)}{\partial x_{i_{1}} \cdots \partial x_{i_{j}}}h_{i_{1}} \cdots h_{i_{j}} .\]
Por lo que, 
\[g^{\left(j\right)}\left(0\right) = \sum^{n}_{i_{1}, \ldots, i_{j} = 1}\frac{\partial ^{j}f\left(a\right)}{\partial x_{i_{1}} \cdots \partial x_{i_{j}}}h_{i_{1}} \cdots h_{i_{j}} .\]
Así, obtenemos que  
\[
\begin{split}
	f\left(a+h\right) = & f\left(a\right) + Df\left(a\right)\left(h\right) + \cdots + \frac{1}{\left(k-1\right)!}D^{k-1}f\left(a\right)\left(h\right)+\frac{1}{k!}D^{k}f\left(a+\theta h\right)\left(h\right) .
\end{split}
\]
\end{proof}
\begin{observation}
Haciendo $\displaystyle h = x - a $ obtenemos 
\[f\left(x\right) = f\left(a\right) + Df\left(a\right)\left(x-a\right) + \cdots + \frac{1}{\left(k-1\right)!}D^{k-1}f\left(a\right)\left(x-a\right) + \frac{1}{k!}D^{k}f\left(\left(1-\theta \right)a + \theta x\right)\left(x-a\right) .\]
\end{observation}
\begin{definition}[Polinomio de Taylor]
Sean $\displaystyle U \subset \R^{n} $ abierto, $\displaystyle a \in U $, $\displaystyle f \in \mathcal{C}^{k}\left(U\right) $. Se define el \textbf{polinomio de Taylor} de orden $\displaystyle k $ de $\displaystyle f $ en $\displaystyle a $ como:
\[P^{k}_{a}\left(h\right)=f\left(a\right)+Df\left(a\right)\left(h\right)+\cdots + \frac{1}{k!}D^{k}f\left(a\right)\left(h\right) = f\left(a\right) + \sum^{k}_{j = 1} \frac{1}{j!}D^{j}f\left(a\right)\left(h\right).\]
\end{definition}
\begin{observation}
También podemos escribir el polinomio de Taylor de la forma
\[P_{a}^{k}\left(x-a\right) = f\left(a\right) + \sum^{k}_{j = 1}\frac{1}{j!}D^{j}f\left(a\right)\left(x-a\right) .\]
\end{observation}
\begin{colorary}[al Teorema de Taylor]
Sea $\displaystyle U\subset \R^{n} $ abierto, $\displaystyle a \in U $ y $\displaystyle f \in \mathcal{C}^{k}\left(U\right) $, con $\displaystyle k \in \N $. Entonces, 
\[\lim_{h \to 0}\frac{f\left(a+h\right)-P^{k}_{a}\left(h\right)}{\|h\|^{k}} = 0 .\]
\end{colorary}
\begin{proof}
	Por el Teorema de Taylor, para $\displaystyle h $ suficientemente pequeño tenemos que $\displaystyle a + h \in U $ y por tanto existe $\displaystyle \theta \in [0,1] $ tal que 
	\[f\left(a + h\right)-P^{k}_{a}\left(h\right) = \frac{1}{k!} \left( D^{k}f\left(a+\theta h\right)\left(h\right) - D^{k}f\left(a\right)\left(h\right)\right) .\]
Así, nos queda que 
\[
\begin{split}
\left|f\left(a+h\right)-P^{k}_{a}\left(h\right)\right| \leq \frac{1}{k!}\sum^{n}_{i_{1}, \ldots, i_{k} = 1} \left|\frac{\partial ^{k}f\left(a+\theta h\right)}{\partial x_{i_{1}} \cdots \partial x_{i_{k}}}-\frac{\partial ^{k}f\left(a\right)}{\partial x_{i_{1}} \cdots \partial x_{i_{k}}} \right| \left|h_{i_{1}}\right| \cdots \left|h_{i_{k}}\right| .
\end{split}
\]
Teniendo en cuenta que $\displaystyle \forall i = 1, \ldots, n $ se tiene que
\[ \left|h_{i}\right| \leq \|h\| = \sqrt{ \left|h_{1}\right|^{2} + \cdots + \left|h_{n}\right|^{2}} \Rightarrow \frac{ \left|h_{i}\right|}{\|h\|} \leq 1 .\]
Por tanto, tenemos que 
\[ \frac{ \left|f\left(a+h\right)-P^{k}_{a}\left(h\right)\right|}{\|h\|^{k}} \leq \frac{1}{k!} \sum^{n}_{i_{1}, \ldots, i_{k} = 1} \left|\frac{\partial ^{k}f\left(a+\theta h\right)}{\partial x_{i_{1}} \cdots \partial x_{i_{k}}}-\frac{\partial ^{k}f\left(a\right)}{\partial x_{i_{1}} \cdots \partial x_{i_{k}}}\right| \cdot 1 \to 0.\]
Este último paso se deduce de que $\displaystyle \frac{\partial^{k}f}{\partial x_{i_{1}} \cdots \partial x_{i_{k}}} $ es continua y $\displaystyle a+\theta h \to a $ (porque $\displaystyle h \to 0 $). 
\end{proof}
\begin{observation}
El equivalente en la otra notación sería
\[\lim_{x \to a}\frac{f\left(x\right)-P_{a}^{k}\left(x-a\right)}{\|x-a\|^{k}} = 0 .\]
\end{observation}
\begin{observation}
Si $\displaystyle f \in \mathcal{C}^{k}\left(U\right) $, el polinomio de Taylor de orden $\displaystyle k $ de $\displaystyle f $ en $\displaystyle a $ satisface que
\[\lim_{h \to 0}\frac{f\left(a+h\right)-P^{k}_{a}\left(h\right)}{\|h\|^{k}} = 0 .\]
Se puede demostrar que el único polinomio $\displaystyle P $ de orden menor o igual que $\displaystyle k $ que cumpla está condición es el polinomio de Taylor, es decir, $\displaystyle P = P_{a}^{k} $. 
\end{observation}
\begin{notation}[Notación de multi-índices]
Un \textbf{multi-índice} en $\displaystyle \R^{n} $ es $\displaystyle \alpha = \left(\alpha_{1}, \ldots, \alpha_{n}\right) $, donde cada $\displaystyle \alpha_{i} \in \Z^{+} $, $\displaystyle \forall i = 1, \ldots, n $. La \textbf{longitud} de $\displaystyle \alpha  $ es $\displaystyle \left|\alpha \right|= \alpha_{1} + \cdots + \alpha_{n} $. Sea $\displaystyle k = \left|\alpha \right| $ y $\displaystyle f \in \mathcal{C}^{k}\left(U\right) $. Si $\displaystyle \left(i_{1}, \ldots, i_{k}\right) $ es una permutación de 
\[(\underbrace{1, \ldots, 1}_{\alpha_{1} \; \text{veces}}; \ldots; \underbrace{n, \ldots, n}_{\alpha_{n}\; \text{veces}}) .\]
Así, podemos escribir
\[\frac{\partial^{k}f\left(a\right)}{\partial x_{i_{1}} \cdots \partial x_{i_{k}}} = \frac{\partial^{k}f\left(a\right)}{\partial x_{1}^{\alpha_{1}} \cdots \partial x_{n}^{\alpha_{n}}} .\]
Por tanto, tenemos que
\[D^{k}f\left(a\right)\left(h\right) = \sum_{ \left|\alpha \right|=k}\frac{k!}{\alpha_{1}! \cdots \alpha_{n}!}\frac{\partial^{k}f\left(a\right)}{\partial x_{1}^{\alpha_{1}} \cdots \partial x_{n}^{\alpha_{n}}}h_{1}^{\alpha_{1}} \cdots h_{n}^{\alpha_{n}} .\]
Así, podemos escribir
\[P_{a}^{k}\left(h\right) = f\left(a\right) + \sum^{k}_{j = 1}\sum_{ \left|\alpha \right|=j}\frac{j!}{\alpha_{1}! \cdots \alpha_{n}!}\frac{\partial^{j}f\left(a\right)}{\partial x_{1}^{\alpha_{1}} \cdots \partial x_{n}^{\alpha_{n}}}h_{1}^{\alpha_{1}} \cdots h_{n}^{\alpha_{n}} .\]
\end{notation}
\section{Máximos y mínimos locales}
\begin{definition}[Extremos locales]
Sea $\displaystyle M \subset \R^{n} $, $\displaystyle a \in M $ y $\displaystyle f : M \to \R $. Se dice que 
\begin{itemize}
\item $\displaystyle a $ es un \textbf{máximo local} de $\displaystyle f $ si existe $\displaystyle r > 0 $ tal que $\displaystyle f\left(a\right) \geq f\left(x\right) $, $\displaystyle \forall x \in M \cap B\left(a,r\right) $. Se dice que es \textbf{estricto} si $\displaystyle f\left(a\right) > f\left(x\right) $, $\displaystyle \forall x \in M \cap B\left(a,r\right) $. 
\item $\displaystyle a $ es un \textbf{mínimo local} de $\displaystyle f $ si existe $\displaystyle r > 0 $ tal que $\displaystyle f\left(a\right) \leq f\left(x\right) $, $\displaystyle \forall x \in M \cap B\left(a,r\right) $. Se dice que es \textbf{estricto} si $\displaystyle f\left(a\right) < f\left(x\right) $, $\displaystyle \forall x\in M \cap B\left(a,r\right) $.
\end{itemize}
Se dice que $\displaystyle a $ es un \textbf{extremo local} de $\displaystyle f $ si es un máximo o un mínimo local.
\end{definition}
\begin{prop}[Condición necesaria de extremo]
Sean $\displaystyle U \subset \R^{n} $ abierto, $\displaystyle a \in U $ y $\displaystyle f : U \to \R $ diferenciable en $\displaystyle a $. Si $\displaystyle a $ es un extremo local de $\displaystyle f $, entonces $\displaystyle Df\left(a\right) = 0 $, es decir, $\displaystyle \nabla f\left(a\right)= 0 $.
\end{prop}
\begin{proof}
Supongamos sin pérdida de generalidad que $\displaystyle a $ es un máximo local y sea $\displaystyle v \in \R^{n} $ con $\displaystyle v \neq 0 $. Tenemos que existe $\displaystyle r > 0 $ tal que $\displaystyle f\left(a\right) \geq f\left(x\right) $, $\displaystyle \forall x \in B\left(a,r\right) $. Además, existe un $\displaystyle \delta > 0 $ tal que si $\displaystyle \left|t\right| < \delta  $ entonces $\displaystyle a+tv\in U $ y $\displaystyle f\left(a\right) \geq f\left(a+tv\right) $. Consideremos la función
	\[g : \left(-\delta, \delta \right) \to \R : t \to f\left(a + tv\right) .\]
Así, tenemos que 
\[g\left(0\right) = f\left(a\right) \geq f\left(a+tv\right) = g\left(t\right), \; \forall t \in \left(-\delta, \delta \right) .\]
Así, $\displaystyle 0 $ es un máximo local de $\displaystyle g $, por lo que $\displaystyle g'\left(0\right) = 0 $. Por tanto, tenemos que 
\[\lim_{t \to 0}\frac{g\left(t\right)-g\left(0\right)}{t} = \lim_{t \to 0}\frac{f\left(a+tv\right)-f\left(a\right)}{t} = D_{v}f\left(a\right) = Df\left(a\right)\left(v\right) = 0 .\]
Así, $\displaystyle Df\left(a\right)\left(v\right) = 0 $, $\displaystyle \forall v \in \R^{n} $, por lo que $\displaystyle Df\left(a\right) = 0 $.
\end{proof}

\begin{eg}
La función $\displaystyle f\left(x,y\right) = x^{2}-y^{2} $ en $\displaystyle a = \left(0,0\right) $ cumple que $\displaystyle \nabla f\left(0,0\right) = 0 $, pero $\displaystyle \left(0,0\right) $ no es ni máximo ni mínimo local. En efecto, tenemos que $\displaystyle f\left(0,0\right) = 0 $, pero $\displaystyle f\left(x,0\right) = x^{2}>0 $ y $\displaystyle f\left(0,y\right) = -y^{2} < 0 $. 
\end{eg}
\begin{definition}[Punto crítico y punto de silla]
Sean $\displaystyle U\subset\R^{n} $ abierto, $\displaystyle a \in U $ y $\displaystyle f : U \to \R $ diferenciable en $\displaystyle a $. Si $\displaystyle \nabla f\left(a\right) = 0 $, se dice que $\displaystyle a $ es un \textbf{punto crítico} de $\displaystyle f $. Si $\displaystyle a $ es un punto crítico pero no es un extremo relativo se dice que es un \textbf{punto de silla} de $\displaystyle f $.
\end{definition}
\begin{definition}
Sea $\displaystyle Q : \R^{n} \to \R $ una forma cuadrática.
\begin{description}
\item[(a.1)] $\displaystyle Q $ es \textbf{definida positiva} ($\displaystyle Q > 0 $) si $\displaystyle Q\left(h\right) > 0 $, $\displaystyle \forall h \neq 0 $.
\item[(a.2)] $\displaystyle Q $ es \textbf{semidefinida positiva} ($\displaystyle Q \geq 0 $) si $\displaystyle Q\left(h\right) \geq 0 $, $\displaystyle \forall h \in \R^{n} $. 
\item[(b.1)] $\displaystyle Q $ es \textbf{definida negativa} ($\displaystyle Q < 0 $) si $\displaystyle Q\left(h\right) < 0 $, $\displaystyle \forall h \neq 0 $.
\item[(b.2)] $\displaystyle Q $ es \textbf{semidefinida negativa} ($\displaystyle Q \leq 0 $) si $\displaystyle Q\left(h\right) \leq 0 $, $\displaystyle \forall h \in \R^{n} $. 
\item[(c)] $\displaystyle Q $ es \textbf{indefinida} si existen $\displaystyle h_{1}, h_{2} \in \R^{n} $ tales que $\displaystyle Q\left(h_{1}\right) > 0 $ y $\displaystyle Q_{2} < 0 $.
\end{description}
\end{definition}
\begin{theorem}[Clasificación de puntos críticos]
Sean $\displaystyle U \subset \R^{n} $ abierto, $\displaystyle a \in U $, $\displaystyle f \in \mathcal{C}^{2}\left(U\right) $ tal que $\displaystyle Df\left(a\right) = 0 $.
\begin{description}
\item[(a)] $\displaystyle D^{2}f\left(a\right) > 0 $ $\displaystyle \Rightarrow $ $\displaystyle a $ es un mínimo local de $\displaystyle f $ $\displaystyle \Rightarrow $  $\displaystyle D^{2}f\left(a\right) \geq 0 $.
\item[(b)] $\displaystyle D^{2}f\left(a\right) < 0 $ $\displaystyle \Rightarrow $  $\displaystyle a $ es un máximo local de $\displaystyle f $ $\displaystyle \Rightarrow $  $\displaystyle D^{2}f\left(a\right) \leq 0 $.
\item[(c)] Si $\displaystyle D^{2}f\left(a\right) $ es indefinida, entonces $\displaystyle a $ es un punto de silla. 
\end{description}
\end{theorem}
\begin{proof}
\begin{description}
	\item[(a)] Demostramos las dos implicaciones. En primer lugar, sea $\displaystyle Q = D^{2}f\left(a\right) : \R^{n} \to \R $, que es una forma cuadrática definida positiva. Como $\displaystyle Q $ es continua en $\displaystyle \R^{n} $ y $\displaystyle S = \left\{ u  \in \R^{n} \; : \; \|u\|=1\right\}  $ es compacto, entonces $\displaystyle Q $ alcanza un mínimo en $\displaystyle S $, es decir, $\displaystyle \exists u_{0} \in S $ tal que 
		\[\alpha = \inf \left\{ Q\left(u\right) \; : \; \|u\| = 1\right\}  = Q\left(u_{0}\right) > 0 .\]
	Ahora, $\displaystyle \forall h \neq 0 $ (también funciona para $\displaystyle h = 0 $),
	\[Q\left(h\right) = Q\left( \|h\| \cdot \frac{h}{\|h\|}\right) = \|h\|^{2} Q\left(\frac{h}{\|h\|}\right) \geq \alpha \|h\|^{2} .\]
	Por el corolario al teorema de Taylor tenemos que
	\[f\left(a+h\right) = f\left(a\right) + Df\left(a\right)\left(h\right) + \frac{1}{2}D^{2}f\left(a\right)\left(h\right) + R\left(h\right), \quad \lim_{h \to 0}\frac{R\left(h\right)}{\|h\|^{2}}=0 .\]
	Así, tenemos que 
	\[f\left(a+h\right)-f\left(a\right) = \frac{1}{2}Q\left(h\right)+R\left(h\right) .\]
	Dado $\displaystyle \epsilon = \frac{\alpha }{4} > 0 $, existe $\displaystyle \delta > 0 $ tal que $\displaystyle 0 < \|h\| < \delta  $ implica que $\displaystyle a+h \in U $ y $\displaystyle \frac{ \left|R\left(h\right)\right|}{\|h\|^{2}} < \epsilon  $. Por otro lado, 
	\[\frac{ \left|R\left(h\right)\right|}{\|h\|^{2}} < \epsilon \Rightarrow \left|R\left(h\right)\right| \leq \epsilon \|h\|^{2} \iff -\epsilon \|h\|^{2} \leq R\left(h\right) \leq
	 \epsilon \|h\|^{2} .\]
Por tanto, si $\displaystyle 0 < \|h\| < \delta  $, tenemos que 
\[f\left(a+h\right)-f\left(a\right) \geq \frac{1}{2}Q\left(h\right) - \epsilon \|h\|^{2} \geq \frac{\alpha }{2}\|h\|^{2}-\epsilon \|h\|^{2} = \frac{\alpha }{4}\|h\|^{2} > 0 .\]
Por tanto, $\displaystyle f\left(a+h\right)-f\left(a\right) > 0 $ si $\displaystyle 0 < \|h\| < \delta  $, por lo que $\displaystyle a  $ es un mínimo local estricto de $\displaystyle f $. \\ 
Supongamos ahora que $\displaystyle a $ es un mínimo local de $\displaystyle f $. Sea $\displaystyle h \in \R^{n} $ con $\displaystyle h \neq 0 $. Por el corolario al teorema de Taylor, 
\[f\left(a+th\right) = f\left(a\right) + Df\left(a\right)\left(th\right) + \frac{1}{2}D^{2}f\left(a\right)\left(th\right) + R\left(th\right) , \quad \lim_{t \to 0}\frac{R\left(th\right)}{t ^{2}} = \lim_{t \to 0}\left(\frac{R\left(th\right)}{\|th\|^{2}} \|h\|^{2}\right) = 0\]
Así, tenemos que si $\displaystyle \epsilon > 0 $, entonces existe $\displaystyle \delta > 0 $ tal que si $\displaystyle 0 < \left|t\right| < \delta  $, entonces $\displaystyle \frac{ \left|R\left(th\right)\right|}{t ^{2}} < \epsilon  $ y $\displaystyle f\left(a+th\right) \geq f\left(a\right) $. En este caso, 
\[
\begin{split}
	0 \leq & f\left(a+th\right)-f\left(a\right) = \frac{1}{2}Q\left(th\right) + R\left(th\right) \\
	\Rightarrow 0 \leq & \frac{1}{2}Q\left(th\right) + R\left(th\right) .
\end{split}
\]
Así, nos queda que 
\[\frac{1}{2}t ^{2}Q\left(h\right) =\frac{1}{2} Q\left(th\right) \geq - R\left(th\right) > -\epsilon t ^{2} \Rightarrow Q\left(h\right) \geq -\epsilon, \; \forall \epsilon > 0 .\]
Así, tenemos que $\displaystyle Q\left(h\right) \geq 0 $. 
\item[(b)] Es análogo al apartado anterior. Alternativamente podemos aplicar el apartado anterior a la función $\displaystyle -f $. 
\item[(c)] Supongamos que $\displaystyle D^{2}f\left(a\right) $ es indefinida. Por lo visto anteriormente, $\displaystyle a $ no puede ser ni mínimo local ni máximo local de $\displaystyle f $, por lo que $\displaystyle a $ es un punto de silla.
\end{description}
\end{proof}
\begin{definition}[Matriz Hessiana]
Llamamos \textbf{matriz Hessiana} de $\displaystyle f $ en $\displaystyle a $ a la matriz de $\displaystyle D^{2}f\left(a\right) $, es decir
\[\text{Hess}f\left(a\right) = \begin{pmatrix} \frac{\partial^{2}f\left(a\right)}{\partial x_{1} \partial x_{1}} & \cdots & \frac{\partial^{2}f\left(a\right)}{\partial x_{1}\partial x_{n}} \\ \vdots & & \vdots \\ \frac{\partial^{2}f\left(a\right)}{\partial x_{n}\partial x_{1}} & \cdots & \frac{\partial ^{2}f\left(a\right)}{\partial x_{n} \partial x_{n}} \end{pmatrix} ,\]
que es simétrica al ser $\displaystyle f \in \mathcal{C}^{2} $.
\end{definition}
\begin{eg}
Consideremos $\displaystyle f : \R^{2} \to \R $ con $\displaystyle f\left(x,y\right) = x^{2}-y^{2} $ y $\displaystyle a = \left(0,0\right) $. Como vimos en un ejemplo anterior, tenemos que $\displaystyle \nabla f\left(a\right)= 0 $. Así, 
\[\text{Hess} f = \begin{pmatrix} \frac{\partial^{2}f}{\partial x^{2}} & \frac{\partial^{2}f}{\partial x \partial y}  \\ \frac{\partial ^{2}f}{\partial y \partial x} & \frac{\partial ^{2}f}{\partial y^{2}} \end{pmatrix}= \begin{pmatrix} 2 & 0 \\ 0 & - 2 \end{pmatrix} .\]
Claramente, $\displaystyle D^{2}f\left(a\right) $ es indefinida, por lo que $\displaystyle \left(0,0\right) $ es un punto de silla. 
\end{eg}
\begin{eg}
Consideremos $\displaystyle f\left(x,y\right) = x^{2} \geq 0 $. El punto $\displaystyle a = \left(0,0\right) $ es un mínimo absoluto de $\displaystyle f $. Por tanto, tenemos que $\displaystyle \nabla f\left(a\right) = 0 $. Además, 
\[\text{Hess} f\left(0,0\right) = \begin{pmatrix} 2 & 0 \\ 0 & 0  \end{pmatrix} ,\]
que no es definida positiva, pero sí es semidefinida positiva. 
\end{eg}
\begin{theorem}[Criterio de los autovalores]
Sea $\displaystyle Q $ una forma cuadrática en $\displaystyle \R^{n} $ dada por 
\[Q\left(h_{1}, \ldots, h_{n}\right) = \left(h_{1}, \ldots, h_{n}\right)\begin{pmatrix} a_{11} & \cdots & a_{1n} \\ \vdots & & \vdots \\ a_{n1} & \cdots & a_{nn} \end{pmatrix}\begin{pmatrix} h_{1} \\ \vdots \\ h_{n} \end{pmatrix} ,\]
con $\displaystyle A $ simétrica. Entonces, 
\begin{itemize}
\item $\displaystyle Q $ es definida positiva si y solo si todos los autovalores de $\displaystyle A $ son positivos.
\item $\displaystyle Q $ es definida negativa si y solo si todos los autovalores de $\displaystyle A $ son negativos.
\item $\displaystyle Q $ es indefinida si y solo si existen autovalores $\displaystyle \lambda_{1} $ y $\displaystyle \lambda_{2} $ de $\displaystyle A $ con $\displaystyle \lambda_{1} < 0 < \lambda_{2}$.
\end{itemize}	
\end{theorem}
\begin{eg}
\begin{enumerate}
\item $\displaystyle f\left(x,y\right) = x^{3}-3x^{2}+y^{2} $. Tenemos que 
	\[\nabla f = \left(3x^{2}-6x, 2y\right) = \left(0,0\right) \iff 
	\begin{cases}
	3x\left(x-2\right) = 0 \\
	2y = 0
	\end{cases}
	.\]
Así, los puntos críticos son $\displaystyle P_{1} = \left(0,0\right) $ y $\displaystyle P_{2} = \left(2,0\right) $. 
Así, tenemos que 
\[\text{Hess}f = \begin{pmatrix} 6x-6 & 0  \\ 0 & 2 \end{pmatrix} .\]
Por tanto, 
\[\text{Hess}f\left(P_{1}\right) = \begin{pmatrix} -6 & 0 \\ 0 & 2 \end{pmatrix} ,\]
como los autovalores son $\displaystyle -6 $ y $\displaystyle 2 $ tenemos que se trata de un punto de silla. Por otro lado, 
\[\text{Hess}f\left(P_{2}\right) = \begin{pmatrix} 6 & 0 \\ 0 & 2 \end{pmatrix} ,\]
como los autovalores son $\displaystyle 6 $ y $\displaystyle 2 $ tenemos que $\displaystyle P_{2} $ es un mínimo local (estricto).
\item Sea $\displaystyle f\left(x,y,z\right) = xy + y + z^{2} $. Tenemos que 
	\[\nabla f = \left(y, x+1, 2z\right) = \left(0,0,0\right)\iff 
	\begin{cases}
	y = 0\\
	x + 1 = 0 \\
	z = 0
	\end{cases}
	.\]
	Así, el único punto crítico es $\displaystyle P = \left(-1,0,0\right) $. Así, tenemos que 
	\[\text{Hess}f = \begin{pmatrix} 0 & 1 & 0 \\ 1 & 0 & 0 \\ 0 & 0 & 2 \end{pmatrix} .\]
Calculemos los autovalores de la matriz:
\[0 = \det\left(A - \lambda I\right) = \left(2-\lambda \right)\left(\lambda + 1\right)\left(\lambda - 1\right) .\]
Como los autovalores son $\displaystyle \lambda \in \left\{ -1,1,2\right\}  $, se trata de un punto de silla. 
\end{enumerate}
\end{eg}
\begin{theorem}[Criterio de Sylvester]
Para $\displaystyle j = 1, \ldots, n $ denotamos 
\[\Delta_{j} = \begin{vmatrix} a_{11} & \cdots & a_{1j} \\ \vdots & & \vdots \\ a_{j1} & \cdots & a_{jj} \end{vmatrix}  ,\]
a los menores angulares. 
\begin{itemize}
\item $\displaystyle Q $ es definida positiva si y solo si $\displaystyle \Delta_{j} > 0 $, $\displaystyle \forall j = 1, \ldots, n $. 
\item $\displaystyle Q $ es definida negativa si y solo si $\displaystyle \left(-1\right)^{j}\Delta _{j} > 0 $, $\displaystyle \forall j = 1, \ldots, n $. 
\item Si $\displaystyle \det\left(A\right) \neq 0 $ y no se cumple ninguna de las codiciones entonces es indefinida. 
\end{itemize}
\end{theorem}
\begin{notation}
En algunos casos utilizamos la notación
\[f_{x x} : = \frac{\partial^{2}f}{\partial x \partial x}, \quad f_{xy} := \frac{\partial^{2}f}{\partial x\partial y} .\]
\end{notation}
\begin{eg}
Sea $\displaystyle U \subset \R^{2} $ abierto, $\displaystyle a \in U $, $\displaystyle f \in \mathcal{C}^{2}\left(U\right) $ con $\displaystyle Df\left(a\right) = 0 $. Tenemos que 
\[\text{Hess}f\left(a\right) =\begin{pmatrix} f_{x x }\left(a\right) & f_{xy}\left(a\right) \\ f_{yx}\left(a\right) & f_{yy}\left(a\right) \end{pmatrix} .\]
Tenemos que
\[\Delta_{1} = f_{x x}\left(a\right), \quad \Delta_{2} = f_{ x x}\left(a\right) \cdot f_{yu}\left(a\right)- \left(f_{xy}\left(a\right)\right)^{2}.\]
Así, tenemos que 
\begin{itemize}
\item Si $\displaystyle \Delta_{1} > 0 $ y $\displaystyle \Delta_{2}> 0 $, entonces $\displaystyle a $ es un mínimo local. 
\item Si $\displaystyle \Delta_{1} < 0 $ y $\displaystyle \Delta_{2} > 0 $, entonces $\displaystyle a $ es un máximo local. 
\item Si $\displaystyle \Delta_{2} < 0$, entonces $\displaystyle a $ es un punto de silla.
\end{itemize}
\end{eg}
\begin{eg}
Consideremos $\displaystyle f\left(x,y,z\right) = x^{3}+y^{3}+z^{3}-3\left(xy + xz + yz\right) $. Calculemos los puntos críticos:
\[
\begin{cases}
f_{x} = 3x^{2} - 3\left(x+y\right) = 0 \\
f_{y} = 3y^{2} - 3\left(x+z\right) = 0 \\
f_{z} = 3z^{2}-3\left(x + y \right) = 0
\end{cases} \Rightarrow
\begin{cases}
x^{2} = y + z \\
y^{2} = x + z \\
z^{2} = x + y
\end{cases}
.\]
Resolviendo el sistema obtenemos que los puntos críticos son $\displaystyle P_{1} = \left(0,0,0\right) $ y $\displaystyle P_{2} = \left(2,2,2\right) $. Tenemos que 
\[\text{Hess}f = \begin{pmatrix} 6x & - 3 & - 3 \\ - 3 & 6y & -3 \\ -3 & - 3 & 6z \end{pmatrix} .\]
Así, tenemos que 
\[\text{Hess}f\left(P_{1}\right)= \begin{pmatrix} 0 & -3 & -3 \\ -3 & 0 & - 3 \\ -3 & - 3 & 0 \end{pmatrix} \Rightarrow 
\begin{cases}
\Delta_{1} = 0 \\
\Delta_{2} = -9 < 0
\end{cases}
.\]
Por tanto, tenemos que $\displaystyle P_{1} $ es un punto de silla. Por otro lado, 
\[\text{Hess}f\left(P_{2}\right) = 3 \cdot \begin{pmatrix} 4 & - 1 & - 1 \\ - 1 & 4 & 1 \\ - 1 & - 1 & 4 \end{pmatrix} \Rightarrow 
\begin{cases}
\Delta_{1} > 0 \\
\Delta_{2} > 0 \\
\Delta_{3} > 0
\end{cases}
.\]
Por tanto, $\displaystyle P_{2} $ es un mínimo local. 
\end{eg}

