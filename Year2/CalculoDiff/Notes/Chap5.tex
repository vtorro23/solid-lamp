\section{Teorema de la función inversa}
\begin{theorem}[Teorema de la función inversa]
Sean $\displaystyle U \subset \R^{n} $ abierto, $\displaystyle f : U \to \R^{n} $ de clase $\displaystyle \mathcal{C}^{1} $ en $\displaystyle U $; $\displaystyle a\in U $ tal que $\displaystyle \det Jf\left(a\right) \neq 0 $. Entonces, existe $\displaystyle V \subset \R^{n} $ abierto tal que $\displaystyle a \in V \subset U $ y existe $\displaystyle W \subset \R^{n} $ abierto tal que $\displaystyle f\left(a\right) \in W $, tales que $\displaystyle f\left(V\right) = W $ y $\displaystyle f|_{V} : V \to W $ admite una función inversa diferenciable $\displaystyle g = \left(f|_{V}\right)^{-1} : W \to V $, que satisface 
\[Dg\left(f\left(x\right)\right) = Df\left(x\right)^{-1}, \; \forall x \in V .\]
Además, si $\displaystyle f \in \mathcal{C}^{m}\left(U\right) $, entonces $\displaystyle g \in \mathcal{C}^{m}\left(W\right) $ para $\displaystyle m \in \N $.
\end{theorem}
\begin{eg}
Consideremos $\displaystyle f : \R^{2} \to \R^{2} $ con $\displaystyle f\left(x,y\right) = \left(e^{x}\cos y, e^{x}\sin y\right) $. Consideremos 
\[
\begin{cases}
u = e^{x}\cos y \\
v = e^{x} \sin y
\end{cases}
.\]
Buscamos saber si existen $\displaystyle x = x\left(u,v\right) $ e $\displaystyle y = y\left(u,v\right) $. Tenemos que si $\displaystyle a = \left(0,0\right) $, $\displaystyle f\left(a\right) = \left(1,0\right) $. Además, 
\[ Jf\left(a\right) = \begin{pmatrix} 1 & 0 \\ 0 & 1 \end{pmatrix} .\]
Como $\displaystyle \det Jf\left(a\right) \neq 0 $, tenemos que se cumple el teorema de la función inversa y por tanto podemos encontrar las expresiones de $\displaystyle x $ e $\displaystyle y $ en función de $\displaystyle u $ y $\displaystyle v $ en un entorno cercano a $\displaystyle a $. 
\end{eg}
\begin{theorem}[Teorema de la función inyectiva]
Sea $\displaystyle U \subset \R^{n} $ abierto, $\displaystyle a \in U $, $\displaystyle f : U \to \R^{m} $ de clase $\displaystyle \mathcal{C}^{1} $ en $\displaystyle U $, tal que $\displaystyle Df\left(a\right) : \R^{n} \to \R^{m} $ es inyectiva. Entonces existen $\displaystyle C,r > 0 $ tales que 
\[ \|x-y\| \leq C \|f\left(x\right)-f\left(y\right) \|, \; \forall x,y \in \overline{B}\left(a,r\right) \subset U .\]
En particular, $\displaystyle f|_{\overline{B}\left(a,r\right)} $ es inyectiva. 
\end{theorem}
\begin{proof}
	Sea $\displaystyle L = Df\left(a\right) : \R^{n} \to \R^{m} $ que es inyectiva, por lo que $\displaystyle \Ker\left(L\right) = \left\{ 0\right\}  $. Por tanto, consideremos 
	\[\alpha = \inf \left\{ \|L\left(v\right)\| \; : \; \|v\| = 1\right\}  .\]
	Como la aplicación $\displaystyle v \to \|L\left(v\right)\| $ es continua en $\displaystyle \R^{n} $ y $\displaystyle S = \left\{ v \in \R^{n} \; : \; \|v\|=1\right\}  $ es compacto, tenemos que existe $\displaystyle v_{0} \in S $ tal que $\displaystyle \alpha = \|L\left(v_{0}\right)\| > 0 $. En consecuencia, $\displaystyle \forall x \neq 0 $, tenemos que
	\[ \|L\left(x\right)\| = \left\|L\left(\|x\|\frac{x}{\|x\|}\right)\right\| = \|x\|\left\|L\left(\frac{x}{\|x\|}\right)\right\|\geq \|x\|\alpha  .\]
Ahora, como $\displaystyle f $ es de clase $\displaystyle \mathcal{C}^{1} $ en $\displaystyle U $, tenemos que $\displaystyle \forall i = 1, \ldots, n $ y $\displaystyle \forall j = 1, \ldots, m $, tenemos que $\displaystyle \frac{\partial f_{j}}{\partial x_{i}} $ es continua en $\displaystyle U $. Tomando $\displaystyle \epsilon = \frac{\alpha }{2\sqrt{nm}} $, existe $\displaystyle r > 0 $ tal que $\displaystyle \overline{B}\left(a,r\right) \subset U $ y además
\[ \left|\frac{\partial f_{j}}{\partial x_{i}}\left(x\right)-\frac{\partial f_{j}}{\partial x_{i}}\left(a\right)\right| < \frac{\alpha }{2\sqrt{nm}}, \; \forall i = 1, \ldots, n; \forall j = 1, \ldots, m; \forall x \in \overline{B}\left(a,r\right) .\]
Consideremos ahora $\displaystyle g : U \to \R^{m} $ con $\displaystyle g\left(x\right)= f\left(x\right)-L\left(x\right) $. Tenemos que $\displaystyle \forall j= 1, \ldots, m $, 
\[ g_{j}\left(x\right) = f_{j}\left(x\right)-L_{j}\left(x\right) = f_{j}\left(x\right)-\sum^{n}_{i = 1}\frac{\partial f_{j}\left(a\right)}{\partial x_{i}} x_{i}.\]
Así, tenemos que 
\[ \left|\frac{\partial g_{j}}{\partial x_{i}}\left(x\right)\right| = \left|\frac{\partial f_{j}}{\partial x_{i}}\left(x\right)-\frac{\partial f_{j}}{\partial x_{i}}\left(a\right)\right| < \frac{\alpha }{2\sqrt{nm}}, \; \forall x \in \overline{B}\left(a,r\right) .\]
Por la desigualdad del valor medio tenemos que, como $\displaystyle V = B\left(a,r\right) $ es abierto convexo y se cumple la desigualdad anterior, tenemos que
\[ \|g\left(x\right)-g\left(y\right)\| \leq \sqrt{nm}\frac{\alpha }{2\sqrt{nm}} \|x-y\| = \frac{\alpha }{2}\|x-y\|, \; \forall x,y \in B\left(a,r\right)  .\]
Por continuidad, podemos extender la desigualdad anterior a la bola $\displaystyle \overline{B}\left(a,r\right) $. Así, nos queda que 
\[
\begin{split}
	\|f\left(x\right)-f\left(y\right)-L\left(x\right)+L\left(y\right)\| = & \|f\left(x\right)-f\left(y\right)-L\left(x-y\right)\| \geq \|L\left(x-y\right)\| -\|f\left(x\right)-f\left(y\right)\| \\
	\geq & \alpha \|x-y\|-\|f\left(x\right)-f\left(y\right)\| .
\end{split}
\]
Así, nos queda que 
\[\frac{\alpha }{2}\|x-y\| \geq \alpha \|x-y\| - \|f\left(x\right)-f\left(y\right)\| .\]
Así, obtenemos que 
\[\|f\left(x\right)-f\left(y\right)\| \geq \alpha\|x-y\| - \frac{\alpha }{2}\|x-y\| = \frac{\alpha }{2}\|x-y \|, \; \forall x,y \in \overline{B}\left(a,r\right).\]
\end{proof}
\begin{lema}
	El conjunto $\displaystyle \GL\left(\R^{n}\right) = \left\{ L : \R^{n} \to \R^{n} \; : \; L\; \text{lineal biyectiva}\right\}  $ es abierto en $\displaystyle \mathcal{L}\left(\R^{n}, \R^{n}\right) \cong \mathcal{M}_{n \times n} \left(\R\right)\cong \R^{n^{2}}$ y además la aplicación
	\[J : \GL\left(\R^{n}\right) \to \GL\left(\R^{n}\right) : L \to L^{-1}	\]
	es de clase $\displaystyle \mathcal{C}^{\infty} $. 
\end{lema}
\begin{proof}
Podemos decir que 
\[\GL\left(\R^{n}\right) = \left\{ L \in \mathcal{L}\left(\R^{n}, \R^{n}\right) \; : \; \det\left(L\right) \neq 0\right\}  .\]
Además, la aplicación $\displaystyle \det : \mathcal{L}\left(\R^{n}, \R^{n}\right) \cong \mathcal{M}_{n \times n} \left(\R\right) \to \R : L \to \det\left(L\right)$  es continua y es de clase $\displaystyle \mathcal{C}^{\infty} $ puesto que $\displaystyle \det\left(L\right) $ se obtiene a partir de operaciones algebraicas de sus componentes en la base canónica. 
Otra forma de verlo es que 
\[L^{-1} = \frac{1}{\det\left(L\right)}\Adj\left(L^{T}\right) ,\]
se obtiene por medio de operaciones algebraicas, por lo que es de clase $\displaystyle \mathcal{C}^{\infty} $. 
\end{proof}
Ahora estamos en condiciones para demostrar el \textbf{Teorema de la Función Inversa}. 
\begin{theorem}[Teorema de la función inversa]
Sean $\displaystyle U \subset \R^{n} $ abierto, $\displaystyle f : U \to \R^{n} $ de clase $\displaystyle \mathcal{C}^{1} $ en $\displaystyle U $; $\displaystyle a\in U $ tal que $\displaystyle \det Jf\left(a\right) \neq 0 $. Entonces, existe $\displaystyle V \subset \R^{n} $ abierto tal que $\displaystyle a \in V \subset U $ y existe $\displaystyle W \subset \R^{n} $ abierto tal que $\displaystyle f\left(a\right) \in W $, tales que $\displaystyle f\left(V\right) = W $ y $\displaystyle f|_{V} : V \to W $ admite una función inversa diferenciable $\displaystyle g = \left(f|_{V}\right)^{-1} : W \to V $, que satisface 
\[Dg\left(f\left(x\right)\right) = Df\left(x\right)^{-1}, \; \forall x \in V .\]
Además, si $\displaystyle f \in \mathcal{C}^{m}\left(U\right) $, entonces $\displaystyle g \in \mathcal{C}^{m}\left(W\right) $ para $\displaystyle m \in \N $.
\end{theorem}
\begin{proof}
\begin{description}
\item[Paso 1.] Consideramos $\displaystyle \varphi : U \to \R $, con $\displaystyle \varphi\left(x\right) = \det\left(Df\left(x\right)\right) $, que es continua en $\displaystyle U $ puesto que $\displaystyle \varphi $ se obtiene por medio de operaciones algebraicas con las derivadas parciales primeras $\displaystyle \left(\frac{\partial f_{j}}{\partial x_{i}}\right)_{i,j = 1, \ldots, n} $, las cuales son continuas. Como $\displaystyle \varphi\left(a\right) = \det\left(Df\left(a\right)\right) \neq 0 $, existe $\displaystyle U_{0} \subset U $ abierto con $\displaystyle a\in U_{0} $ tal que $\displaystyle \varphi\left(x\right) = \det\left(f\left(x\right)\right) \neq 0 $, $\displaystyle \forall x \in U_{0} $. 
\item[Paso 2.] Por el teorema de la función inyectiva, existen $\displaystyle C,r > 0 $ tales que $\displaystyle \overline{B}\left(a,r\right) \subset U_{0} $ y 
	\[\|f\left(x_{1}\right)-f\left(x_{2}\right)\| \geq C \|x_{1}-x_{2}\|, \; \forall x_{1}, x_{2} \in \overline{B}\left(a,r\right) .\]
	En particular, $\displaystyle f|_{\overline{B}\left(a,r\right)} $ es inyectiva. Sea $\displaystyle S = \partial\left(\overline{B}\left(a,r\right)\right) = \left\{ x \in \R^{n} \; : \; \|x-a\| = r\right\}  $, que es un conjunto compacto por lo que $\displaystyle f\left(S\right) $ es también compacto. Además, $\displaystyle f\left(a\right) \not\in f\left(S\right) $ por la inyectividad de $\displaystyle f $ y $\displaystyle a \not\in S $.
	Así, podemos definir
	\[d = \min \left\{ \|f\left(a\right)-f\left(x\right)\|\; : \; x \in S\right\} > 0 .\]
Por tanto, tenemos que
\[ \|f\left(x\right)-f\left(a\right)\| \geq d, \; \forall x \in S .\]
\item[Paso 3.] Consideremos $\displaystyle W = B\left(f\left(a\right), \frac{d}{2}\right) $ y $\displaystyle B = B\left(a,r\right) $. Tenemos que ver que $\displaystyle W \subset f\left(B\right) $. Sea $\displaystyle \hat{y} \in W $ y consideremos la función
	\[ h : \overline{B}\left(a,r\right) \to \R : x \to \|f\left(x\right)-\hat{y}\|^{2} .\]
Como $\displaystyle \overline{B}\left(a,r\right) $ es complacto y $\displaystyle h $ es continua, existe $\displaystyle \hat{x} \in \overline{B}\left(a,r\right) $ donde $\displaystyle h $ alcanza un mínimo. Veamos que $\displaystyle \hat{x} \not\in S $. En efecto, si $\displaystyle x \in S $ tenemos que 
\[\|f\left(x\right)-\hat{y}\| \geq \|f\left(x\right)-f\left(a\right)\| - \|f\left(a\right)-\hat{y}\| \geq d -\frac{d}{2} = \frac{d}{2} > \|f\left(a\right)-\hat{y}\| .\]
Así, tenemos que si $\displaystyle x \in S $, $\displaystyle h\left(x\right) > h\left(a\right) $, por lo que $\displaystyle \hat{x} \not\in S $ y debe ser que $\displaystyle \hat{x} \in B $, por lo que $\displaystyle \hat{x} $ es un mínimo local de $\displaystyle h $. Así, debe ser que $\displaystyle \nabla h\left(\hat{x}\right) = 0 $. Si denotamos $\displaystyle \hat{y} = \left(y_{1}, \ldots, y_{n}\right) $, tenemos que 
\[h\left(x\right) = \sum^{n}_{j = 1}\left(f_{j}\left(x\right)-y_{j}\right)^{2} .\]
Por tanto, $\displaystyle \forall j = 1, \ldots, n $,
\[\frac{\partial h}{\partial x_{i}}\left(\hat{x}\right) = \sum^{n}_{j = 1}2\underbrace{\left(f_{j}\left(x\right)-y_{j}\right)}_{u_{j}} \cdot \underbrace{\frac{\partial f_{j}}{\partial x_{i}}\left(\hat{ x}\right)}_{a_{ij}} = 0 .\]
Así, obtenemos el sistema
\[
\begin{cases}
a_{11}u_{1} + \cdots + a_{1n}u_{n} = 0 \\
\vdots \\
a_{n1}u_{1} + \cdots + a_{nn}u_{n} = 0
\end{cases}
.\]
Tenemos que 
\[\det\begin{pmatrix} a_{11} & \cdots & a_{1n} \\ \vdots & & \vdots \\ a_{n1} & \cdots & a_{nn} \end{pmatrix} = \det\left(\frac{\partial f_{j}}{\partial x_{i}}\left(\hat{x}\right)\right) = \det\left(Jf\left(\hat{x}\right)\right) \neq 0 .\]
Así, debe ser que es sistema tiene sólo una solución, la solución trivial, por lo que $\displaystyle u_{1} = u_{2} = \cdots = u_{n} = 0 $. Por tanto, $\displaystyle f_{j}\left(\hat{x}\right) = y_{j} $, $\displaystyle \forall j = 1, \ldots,n $. Así, tenemos que $\displaystyle f\left(\hat{x}\right) = \hat{y} $. 
\item[Paso 4.] Definimos $\displaystyle V = B\left(a,r\right) \cap f^{-1}\left(W\right) = \left\{ x \in B\left(a,r\right) \; : \; f\left(x\right) \in W\right\}  $ que es abierto y $\displaystyle a \in V $. Por construcción tenemos que $\displaystyle f|_{V} : V \to W $, que es inyectiva y sobreyectiva, por lo que es biyectiva. 
\item[Paso 5.] Veamos que $\displaystyle g $ es diferenciable con $\displaystyle Dg\left(f\left(x\right)\right) = \left(Df\left(x\right)\right)^{-1} $, $\displaystyle \forall x \in V $. Sea $\displaystyle x_{0} \in V $, $\displaystyle  x \in V $ y denotamos $\displaystyle y_{0} = f\left(x_{0}\right) $ e $\displaystyle y = f\left(x\right) $, así como $\displaystyle L = Df\left(x_{0}\right) $. Tenemos que 
	\[
	\begin{split}
		\frac{\|g\left(y\right)-g\left(y_{0}\right)-L^{-1}\left(y-y_{0}\right)\|}{\|y-y_{0}\|} = & \frac{\|x-x_{0}-L^{-1}\left(f\left(x\right)-f\left(x_{0}\right)\right)\|}{\|f\left(x\right)-f\left(x_{0}\right)\|} \\
		= & \frac{\|x-x_{0}\|}{\|f\left(x\right)-f\left(x_{0}\right)\|} \cdot \frac{\|L^{-1}\left(L\left(x-x_{0}\right)\right)-L^{-1}\left(f\left(x\right)-f\left(x_{0}\right)\right)}{\|x-x_{0}\|} \\
		\leq &  C \|L^{-1}\| \frac{\|f\left(x\right)-f\left(x_{0}\right)-L\left(x-x_{0}\right)\|}{\|x-x_{0}\|}  \to 0.
	\end{split}
	\]
	Hemos aplicado el teorema de la función inyectiva. Si $\displaystyle y \to y_{0} $ tenemos que $\displaystyle f\left(x\right) \to f\left(x_{0}\right) $, por lo que $\displaystyle x \to x_{0} $ (puesto que $\displaystyle \|x-x_{0}\| \leq C\|f\left(x\right)-f\left(x_{0}\right)\| $). 
Así, tenemos que si $\displaystyle x \in V $, $\displaystyle Dg\left(f\left(x\right)\right) = \left(Df\left(x\right)\right)^{-1} $. Equivalentemente, podemos decir que si $\displaystyle y \in W $, $\displaystyle Dg\left(y\right) = \left(Df\left(g\left(y\right)\right)\right)^{-1} $.
\item[Paso 6.] Sabemos que 
	\[Jg\left(y\right) = \left(Jf\left(g\left(y\right)\right)\right)^{-1} ,\]
	es continua puesto que 
	\[y \to g\left(y\right) \to Jf\left(g\left(y\right)\right) \to \left(Jf\left(g\left(y\right)\right)\right)^{-1} ,\]
	y son todas continuas. Por tanto, tenemos que $\displaystyle g $ es de clase $\displaystyle \mathcal{C}^{1} $ siempre. Inductivamente podemos ver que si $\displaystyle f $ es $ \mathcal{C}^{m} $, entonces $\displaystyle g $ es $  \mathcal{C}^{m} $. 
\end{description}
\end{proof}
\section{Teorema de la función implícita}
\begin{eg}
Una par $\displaystyle \left(x,y\right) \in \R^{2} $ satisface la ecuación implícita $\displaystyle x^{2}+y^{2} = 1 $ si y solo si satisface
\[y = \sqrt{1-x^{2}} \quad \text{o} \quad y = -\sqrt{1-x^{2}} .\]
Podemos considerar $\displaystyle F\left(x\right) = x^{2}+y^{2}-1 $ y $\displaystyle f\left(x\right) = \sqrt{1-x^{2}} $ y $\displaystyle g\left(x\right)= -\sqrt{1-x^{2}} $. Tendremos que $\displaystyle F\left(x,y\right) = 0 \iff f\left(x\right) = 0  $ o $\displaystyle g\left(x\right) = 0 $, dependiendo del conjunto en el que estemos.
\end{eg}
\begin{notation}
Sea $\displaystyle U \subset \R^{n} \times \R^{m}\cong\R^{n+m} $ abierto, $\displaystyle F: U \to \R^{m} $ de clase $\displaystyle \mathcal{C}^{1} $. Denotamos los elementos de $\displaystyle U $ como $\displaystyle \left(x,y\right) = \left(x_{1}, \ldots, x_{n} ; y_{1}, \ldots, y_{m}\right) $. Para $\displaystyle \left(x_{0}, y_{0}\right) \in U $,
\[JF\left(x_{0}, y_{0}\right) = \begin{pmatrix} \frac{\partial F_{1}}{\partial x_{1}} & \cdots & \frac{\partial F_{1}}{\partial x_{n}} & \frac{\partial F_{1}}{\partial y_{1}} & \cdots & \frac{\partial F_{1}}{\partial y_{m}} \\ \vdots & \vdots & \vdots & \vdots & \vdots & \vdots \\ \frac{\partial F_{m}}{\partial x_{1}} & \cdots & \frac{\partial F_{m}}{\partial x_{n}} & \frac{\partial F_{m}}{\partial y_{1}} & \cdots & \frac{\partial F_{m}}{\partial y_{m}} \end{pmatrix} .\]
A la submatriz izquierda la denotamos $\displaystyle D_{1}F\left(x_{0}, y_{0}\right) $ y a la de la derecha la llamamos $\displaystyle D_{2}F\left(x_{0}, y_{0}\right) $. 
\end{notation}
\begin{theorem}[Teorema de la función implícita]
Sean $\displaystyle U \subset \R^{n} \times \R^{m} \cong \R^{n+m} $ abierto, $\displaystyle F : U \to \R^{m} $ de clase $\displaystyle \mathcal{C}^{1} $, $\displaystyle \left(x_{0}, y_{0}\right) \in U $ tal que $\displaystyle F\left(x_{0}, y_{0}\right) = 0 $ y $\displaystyle \det\left(D_{2}F\left(x_{0}, y_{0}\right)\right) \neq 0 $. 
Existe $\displaystyle V \subset \R^{n} $ abierto con $\displaystyle x_{0} \in V $ y existe $\displaystyle W \subset \R^{m} $ abierto con $\displaystyle y_{0} \in W $ con $\displaystyle V \times W \subset U $, tales que existe $\displaystyle f : V \to W $ de clase $\displaystyle \mathcal{C}^{1} $ tal que $\displaystyle \forall \left(x,y\right) \in V \times W $,
\[ F\left(x,y\right) = 0 \iff y = f\left(x\right).\]
Además, $\displaystyle \forall x \in V $,
\[Jf\left(x\right) = -D_{2}F\left(x,f\left(x\right)\right)^{-1} \cdot D_{1}F\left(x,f\left(x\right)\right) .\]
Si $\displaystyle F \in \mathcal{C}^{k} $, entonces $\displaystyle f \in \mathcal{C}^{k} $ para $\displaystyle k \in \N $.
\end{theorem}
\begin{notation}
Tenemos que en $\displaystyle V \times W $,
\[
\begin{cases}
F_{1}\left(x_{1}, \ldots, x_{n}; y_{1}, \ldots, y_{m}\right) = 0 \\
\vdots \\
F_{m}\left(x_{1}, \ldots, x_{n}; y_{1}, \ldots, y_{m}\right) = 0
\end{cases} 
\iff 
\begin{cases}
y_{1} = f_{1}\left(x_{1}, \ldots, x_{n}\right) \\
\vdots \\
y_{m} = f_{m}\left(x_{1}, \ldots, x_{n}\right)
\end{cases}
.\]
Muchas veces omitiremos la función $\displaystyle f $ y denotaremos 
\[
\begin{cases}
y_{1} = y_{1}\left(x_{1}, \ldots, x_{n}\right) \\
\vdots \\ 
y_{m} = y_{m}\left(x_{1}, \ldots, x_{n}\right)
\end{cases}
.\]
\end{notation}
\begin{proof}
No veremos la demostración pero daremos una idea de cómo se demuestra. Definimos 
\[H:U \subset \R^{n} \times \R^{m} \to \R^{n} \times \R^{m} : \left(x,y\right) \to \left(x, F\left(x,y\right)\right) .\]
Tendremos pues
\[JH\left(x_{0},y_{0}\right) = \begin{pmatrix} Id & 0 \\ D_{1}F\left(x_{0}, y_{0}\right) & D_{2}F\left(x_{0}, y_{0}\right) \end{pmatrix} .\]
Por tanto, obtenemos que $\displaystyle \det\left(DH\left(x_{0}, y_{0}\right)\right) \neq 0 $, por lo que $\displaystyle H $ es localmente invertible alrededor de $\displaystyle \left(x_{0}, y_{0}\right) $ y $\displaystyle F\left(x,y\right) = 0 \iff H\left(x,y\right) = \left(x,0\right) $. Si consideramos $\displaystyle A = \left\{ \left(x,y\right) \; : \; F\left(x,y\right) = 0\right\}  \subset U$, tenemos que $\displaystyle H|_{A} : A \to B $ es invertivle y podemos considerar $\displaystyle G = \left(H|_{A}\right)^{-1} $.
\end{proof}
\begin{eg}
\begin{enumerate}
\item Consideremos $\displaystyle z^{3}+x\left(z-y\right) = 1 $. Nos preguntamos si podemos despejar $\displaystyle z = z\left(x,y\right) $ alrededor de $\displaystyle p = \left(0,0,1\right) $. Tenemos que 
	\[F\left(x,y,z\right) = z^{3}+x\left(z-y\right)-1, \; U = \R^{3} .\]
Podemos observar varias cosas:
\begin{itemize}
\item $\displaystyle F \in \mathcal{C}^{\infty} $ en $\displaystyle \R^{3} $.
\item $\displaystyle F\left(p\right) = F\left(0,0,1\right) = 0 $.
\item En este caso tenemos que $\displaystyle JF = \nabla F = \left(\frac{\partial F}{\partial x}, \frac{\partial F}{\partial y}, \frac{\partial F}{\partial z}\right) = \left(z-y, -x, 3z^{2}+x\right)$. En nuestro caso tenemos que $\displaystyle n = 2 $ y $\displaystyle m = 1 $. 
	\[JF\left(p\right) = \left(1,0,3\right) \Rightarrow \det\left(D_{2}F\left(p\right)\right) = 3 \neq 0 .\]
\end{itemize}
Como se cumplen las hipótesis del teorema, existen $\displaystyle V^{\left(0,0\right)} \subset \R^{2} $, $\displaystyle W^{1} \subset \R $ y $\displaystyle f : V \to W $ de clase $\displaystyle \mathcal{C}^{\infty} $ tal que $\displaystyle \forall \left(x,y,z\right) \in V \times W $,
\[F\left(x,y,z\right) = 0 \iff z = f\left(x,y\right) .\]
Además, $\displaystyle \forall \left(x,y\right) \in V $, tenemos que 
\[h\left(x,y\right) := F\left(x,y,f\left(x,y\right)\right) = 0 .\]
\[\Rightarrow 
\begin{cases}
0 = \frac{\partial h}{\partial x} = \frac{\partial F}{\partial x} \cdot 1 + \frac{\partial F}{\partial y} \cdot 0 + \frac{\partial F}{\partial z}\frac{\partial f}{\partial x} \\
0 = \frac{\partial h}{\partial y} = \frac{\partial F}{\partial x} \cdot 0 + \frac{\partial F}{\partial y} \cdot 1 + \frac{\partial F}{\partial z}\frac{\partial f}{\partial z}
\end{cases}
.\]
Así, tenemos que 
\[\frac{\partial f}{\partial x} = - \frac{\frac{\partial F}{\partial x}}{\frac{\partial F}{\partial z}} = -\frac{1}{3}, \quad \frac{\partial f}{\partial y}= - \frac{\frac{\partial F}{\partial y}}{\frac{\partial F}{\partial z}} = 0 .\]
Podemos calcular también las derivadas de segundo orden. 
\[\frac{\partial^{2}f}{\partial x^{2}} = \frac{\partial }{\partial x}\left(\frac{\partial f}{\partial x}\right) = \frac{\partial }{\partial x}\left(\frac{y-z}{3z^{2}+x}\right) = \frac{-\frac{\partial f}{\partial x}\left(3z^{2}+x\right)-\left(y-z\right)\left(6z\frac{\partial f}{\partial x}+1\right)}{\left(3z^{2}+x\right)^{2}}  .\]
En $\displaystyle p $, es fácil ver que la derivada segunda será 0. El resto de derivadas segundas se calculan de forma análoga. \\ 
Otra manera de calcular las derivadas es de forma implícita:
\[\frac{\partial }{\partial x}\left(z^{3}+x\left(z-y\right)\right) = 0 \Rightarrow 3z^{2}\frac{\partial f}{\partial x} + \left(z-1\right) + x \frac{\partial f}{\partial x} = 0 .\]
\[\frac{\partial }{\partial y}\left(z^{3}+x\left(z-y\right)\right)= 0 \Rightarrow 3z^{2}\frac{\partial f}{\partial y}+x\left(\frac{\partial f}{\partial y }-1\right)= 0 .\]
\item Consideremos el sistema 
	\[
	\begin{cases}
	xu^{3} +y^{2}v^{3} + z^{2} - 1 = 0 \\
	2xy^{3} + uv^{2} + vz = 0
	\end{cases}
	,\]
y el punto $\displaystyle p = \left(0,1,0,0,1\right) $. Buscamos despejar $\displaystyle u $ y $\displaystyle v $. Veamos que se cumplen las condiciones del teorema:
\begin{itemize}
\item $\displaystyle F \in \mathcal{C}^{\infty} $, $\displaystyle F : \R^{3}\times\R^{2} \to \R^{2} $. 
\item $\displaystyle F\left(0,1,0,0,1\right) = \left(0,0\right) $.
\item Calculamos $\displaystyle \det\left(D_{2}F\left(p\right)\right) $. 
	\[JF = \begin{pmatrix} \nabla F_{1} \\ \nabla F_{2} \end{pmatrix} = \begin{pmatrix} u^{3} & 2yv^{3} & 2z & 3xu^{2} & 3v^{2}y^{2} \\ 2y^{3} & 6xy^{2} & v & v^{2} & 2uv+z \end{pmatrix}.\]
	Así, tendremos que $\displaystyle D_{2}F = \begin{pmatrix} 3xu^{2} & 3v^{2}y^{2} \\ v^{2} & 2uv +z \end{pmatrix} $. Claramente $\displaystyle \det\left(D_{2}F\left(p\right)\right) = -3 \neq 0 $. 
\end{itemize}
Así, tenemos que existen $\displaystyle V^{\left(0,1,0\right)} \subset \R^{3} $, $\displaystyle W ^{\left(0,1\right)}\subset \R^{2} $ y $\displaystyle f : V \to W $ tales que $\displaystyle \forall \left(x,y,z,u,v\right) \in V \times W $,
\[F\left(x,y,z,u,v\right) = 0 \iff \left(u,v\right) = f\left(x,y,z\right) .\]
Tenemos que el jacobiano de $\displaystyle f $ evaluado en $\displaystyle p $ será
\[Jf\left(p\right) = \begin{pmatrix} \frac{\partial u}{\partial x} & \frac{\partial u}{\partial y} & \frac{\partial u}{\partial z} \\ \frac{\partial v}{\partial x} & \frac{\partial v}{\partial y} & \frac{\partial v}{\partial z} \end{pmatrix} = -\begin{pmatrix} 0 & 3 \\ 1 & 0 \end{pmatrix}^{-1}\begin{pmatrix} 0 & 2 & 0 \\ 2 & 0 & 1 \end{pmatrix} = \begin{pmatrix} 2 & 0 & 1 \\ 0 & \frac{2}{3} & 0 \end{pmatrix} .\]
Si derivamos de forma implícita:
\[
\begin{cases}
u^{3} + 3xu^{2}\frac{\partial u}{\partial x} + 3y^{2}v^{2}\frac{\partial v}{\partial x} = 0 \\
2y^{3} + \frac{\partial u}{\partial x}v^{2} + 2uv\frac{\partial v}{\partial x} + \frac{\partial v}{\partial z}z = 0
\end{cases}
.\]
Para calcular las derivadas tenemos que resolver el sistema.
\end{enumerate}
\end{eg}
\section{Extremos condicionados}
\begin{eg}
	Calcular el máximo y el mínimo valor de $\displaystyle f\left(x,y\right) = 2x+3y $ sobre la elipse $\displaystyle E = \left\{ \left(x,y\right) \; : \; x^{2}+4y^{2} = 1\right\}  $. \\
	Tenemos que $\displaystyle f $ es continua en $\displaystyle \R^{2} $ y $\displaystyle E $ es compacto, por lo que existen $\displaystyle p_{m}, p_{M} \in E $ tales que 
	\[f\left(p_{m} \right)\leq f\left(p\right) \leq f\left(p_{M}\right), \; \forall p \in E .\]
Tenemos que $\displaystyle \nabla f\left(x,y\right) = \left(2,3\right) \neq \left(0,0\right) $ y $\displaystyle f $ es lineal. El problema se puede ver de forma más sencilla geométricamente. Buscamos el punto $\displaystyle \left(x,y\right) \in E $ tal que $\displaystyle \nabla F\left(x,y\right) = \nabla f\left(x,y\right) $, es decir, existe $\displaystyle \lambda \in \R $ con 
\[\nabla f\left(x,y\right) = \lambda \nabla F\left(x,y\right) ,\]
donde $\displaystyle F\left(x,y\right) = x^{2}+4y^{2}-1 $. Como $\displaystyle \nabla F\left(x,y\right) = \left(2x,8y\right) $, tenemos que resolver el sistema
\[
\begin{cases}
2 = \lambda 2x \\ 
3 = \lambda 8y \\ 
x^{2}+4y^{2} = 1
\end{cases}
.\]
Obtenemos dos puntos: $\displaystyle p_{1} = \left(\frac{4}{5}, \frac{3}{10}\right) $ y $\displaystyle p_{2} = \left(-\frac{4}{5}, \frac{3}{10}\right) $. Así, obtenemos que el máximo y el mínimo de $\displaystyle f $ serán, 
\[f\left(p_{1}\right)=\frac{25}{10} \quad \text{y} \quad f\left(p_{2}\right) = -\frac{25}{10},\]
respectivamente. 
\end{eg}
\begin{definition}[Variedades diferenciables implícitas]
	Sean $\displaystyle 1\leq k \leq n $; $\displaystyle U \subset \R^{n} $ abierto, $\displaystyle F : U \to \R^{k} $ de clase $\displaystyle \mathcal{C}^{p} $. Consideramos el conjunto 
	\[M = \left\{ x \in U \; : \; F\left(x\right) = 0\right\} = \left\{ x \in U \; : \; F_{1}\left(x\right) = 0, \ldots, F_{k}\left(x\right) = 0\right\}  .\]
Supongamos que $\displaystyle \forall x \in U $ la matriz $\displaystyle JF\left(x\right) $ tiene rango $\displaystyle k $. Decimos entonces que $\displaystyle M $ es una \textbf{variable diferenciable} de clase $\displaystyle \mathcal{C}^{p} $ en $\displaystyle \R^{n} $, con dimensión $\displaystyle d = n - k  $. Diremos también que 
\[
\begin{cases}
F_{1}\left(x_{1}, \ldots, x_{n}\right) = 0 \\
\vdots \\ 
F_{k}\left(x_{1}, \ldots, x_{n}\right) = 0
\end{cases}
\]
son las \textbf{ecuaciones implícitas} de $\displaystyle M $.
\end{definition}
\begin{definition}[Espacio normal y tangente]
	Dada una variedad diferenciable $\displaystyle M $,
	decimos que 
	\[N_{p}\left(M\right) = L\left( \left\{ \nabla F_{1}\left(p\right), \ldots, \nabla F_{k}\left(p\right)\right\} \right) \]
	es el \textbf{espacio vectorial normal} a $\displaystyle M $ en $\displaystyle p $. Por otro lado, decimos que 
	\[T_{p}\left(M\right) = L\left( \left\{ \nabla F_{1}\left(p\right), \ldots, \nabla F_{k}\left(p\right)\right\} \right)^{\perp} ,\]
	es el \textbf{espacio vectorial tangente} a $\displaystyle M $ en $\displaystyle p $. 
\end{definition}
\begin{eg}
\begin{enumerate}
	\item Consideremos $\displaystyle M = \left\{ \left(x,y,z\right) \; : \; x^{2}+y^{2}+z^{2} = 1\right\} \subset \R^{3} $ y 
		\[F\left(x,y,z\right) = x^{3}+y^{3}+z^{3}-1 .\]
Para que sea variedad debe ser que $\displaystyle \nabla F = \left(2x, 2y, 2z\right) $ tenga rango máximo en $\displaystyle M $, que lo tiene puesto que es 0 si y solo si $\displaystyle \left(x,y,z\right)=\left(0,0,0\right) $. 
\item Consideremos $\displaystyle M = \left\{ \left(x,y,z\right) \;  :\; x^{2}+xy+y^{2}+z^{2} = 1\right\} \subset \R^{3} $ y 
	\[F\left(x,y,z\right)=x^{2}+xy+y^{2}+z^{2}-1 ,\]
	que es de clase infinito. Tenemos que 
	\[\nabla F = \left(2x+y, 2y+x, 2z\right) .\]
	Si $\displaystyle \nabla F = 0 $, resolviendo el sistema obtenemos que no tiene solución más que la trivial, que no pertenece a nuestro conjunto, por lo que sí es una variedad diferencial. 
\item Sea $\displaystyle M = \left\{ \left(x,y,z\right) \; : \; x^{2}+y^{2}+z^{2} = 1 , \; x + y +z = 0\right\}  $. Las ecuaciones implícitas son,
\[
\begin{cases}
F_{1}\left(x,y,z\right)=x^{2}+y^{2}+z^{2}-1\\
F_{2}\left(x,y,z\right)=x+y+z
\end{cases}
.\]
Queremos ver si el rango es máximo,
\[\ran\begin{pmatrix} \nabla F_{1} \\ \nabla F_{2} \end{pmatrix} = \begin{pmatrix} 2x & 2y & 2z \\ 1 & 1 & 1 \end{pmatrix} .\]
Como $\displaystyle \left(x,y,z\right) = \left(0,0,0\right) \in M $, no se trata de una variedad diferenciable.	
\end{enumerate}
\end{eg}
\begin{theorem}[Teorema de los multiplicadores de Lagrange]
	Sean $\displaystyle 1\leq k < n $; $\displaystyle U \subset \R^{n} $ abierto; $\displaystyle F : U \to \R^{k} $ de clase $\displaystyle \mathcal{C}^{1} $, donde $\displaystyle F = \left(F_{1}, \ldots, F_{k}\right) $. Supongamos que $\displaystyle M= \left\{ x \in U \; : \; F\left(x\right) = 0\right\}  $ es una variedad diferenciable. Sea además, $\displaystyle f : U \to \R $ diferenciable. Si $\displaystyle p \in M $ es un punto máximo o mínimo local de $\displaystyle f|_{M} $, entonces existen $\displaystyle \lambda_{1}, \ldots, \lambda_{k} $ (a los que llamaremos \textbf{multiplicadores de Lagrange}) tales que 
	\[\nabla f\left(p\right) = \sum^{k }_{i = 1}\lambda_{i}\nabla F_{i}\left(p\right) .\]
\end{theorem}
\begin{lema}[Lema del espacio tangente]
Sea $\displaystyle M \subset \R^{n}$ una variedad diferenciable de clase $\displaystyle \mathcal{C}^{1} $ y sea $\displaystyle p \in M $. Si $\displaystyle w \in \R^{n} $, son equivalentes:
\begin{enumerate}
\item $\displaystyle w \in T_{p}\left(M\right) $.
\item Existe $\displaystyle \gamma : \left(a,b\right) \to \R^{n} $ curva de clase $\displaystyle \mathcal{C}^{1} $ con $\displaystyle \Imagen\left(\gamma \right)\subset M $ y existe $\displaystyle t_{0} \in \left(a,b\right) $ tal que $\displaystyle \gamma\left(t_{0}\right)=p $ y $\displaystyle \gamma'\left(t_{0}\right)=w $.
\end{enumerate}
\end{lema}
\begin{eg}
\begin{enumerate}
	\item Calcular el máximo y el mínimo de $\displaystyle f\left(x,y,z\right) = 2x-z^{2} $ en la esfera $\displaystyle S = \left\{ \left(x,y,z\right) \; : \; x^{2} +y^{2} + z^{2} = 5\right\}  $. Como $\displaystyle f $ es continua y $\displaystyle S $ es compacto, existen $\displaystyle p_{m}, p_{M} \in S $ tales que 
		\[f\left(p_{m}\right) \leq f\left(x\right) \leq f\left(p_{M}\right), \; \forall x \in S .\]
		Sea $\displaystyle F\left(x,y,z\right) = x^{2} +y^{2} + z^{2} -5 $ y cogemos $\displaystyle S = \left\{ \left(x,y,z\right) \; : \; F\left(x,y,z\right) = 0\right\}  $. Como $\displaystyle k = 1 $, tenemos que $\displaystyle S $ es una variedad diferenciable de clase $\displaystyle \mathcal{C}^{\infty} $ y $\displaystyle \dim S = n - k = 3 - 1 = 2$, siempre que 
		\[\ran\nabla F = \left(2x, 2y, 2z\right) = 1, \; \forall \left(x,y,z\right) \in S .\]
	Como esto se da, lo que hemos dicho anteriormente también se cumple. Tenemos que $\displaystyle p_{m} $ y $\displaystyle p_{M} $ son extremos locales de $\displaystyle f|_{S} $, por lo que existe $\displaystyle \lambda \in \R $ tal que $\displaystyle \nabla f = \lambda \nabla F $. Como $\displaystyle \nabla f = \left(2, 0, -2z\right) $, tenemos que
	\[
	\begin{cases}
	2 = \lambda 2x \\
	0 = \lambda 2y \\
	-2z = \lambda 2z \\
	x^{2} +y^{2} +z^{2} = 5
	\end{cases}
	.\]
Debe ser que $\displaystyle y = 0 $ y $\displaystyle x = \frac{1}{\lambda } $, y tenemos dos opciones:
\begin{itemize}
\item Si $\displaystyle z = 0 $, tenemos que $\displaystyle x^{2} = 5 $, por lo que $\displaystyle x = \pm \sqrt{5} $ y obtenemos dos posibles puntos
	\[p_{1} = \left(\sqrt{5}, 0, 0\right) \quad \text{y} \quad p_{2} =\left(-\sqrt{5}, 0, 0\right) .\]
\item Si $\displaystyle \lambda = -1 $, tenemos que $\displaystyle x = - 1 $ por lo que $\displaystyle z^{2} = 4 $ y $\displaystyle z = \pm 2 $. Así, obtenemos dos puntos
	\[p_{3} = \left(-1, 0, 2\right) \quad \text{y} \quad p_{4} = \left(-1,0,-2\right) .\]
\end{itemize}
Necesariamente debe ser que $\displaystyle p_{m}, p_{M} \in \left\{ p_{1}, p_{2}, p_{3}, p_{4}\right\}  $. Para calcular el valor exacto calculamos el valor de las imágenes
\[f\left(p_{1}\right)=2\sqrt{5}, \quad f\left(p_{2}\right) = -2\sqrt{5}, \quad f\left(p_{3}\right) = -6, \quad f\left(p_{4}\right)=-6 .\]
Por tanto, el mínimo se alcanza en $\displaystyle p_{3} $ y $\displaystyle p_{4} $ y el máximo se alcanza en $\displaystyle p_{1} $. 
\item Calcular el máximo y el mínimo de $\displaystyle f\left(x,y,z\right) = x^{2} +y^{2} +z^{2} +x + y + z $ en $\displaystyle C = \left\{ \left(x,y,z\right) \; : \; x^{2} +y^{2} + z^{2} \leq 4, \; z \leq 1\right\}  $. Como $\displaystyle f $ es continua y $\displaystyle C $ es compacto, existe $\displaystyle p_{m}, p_{M} \in C $ tales que 
	\[f\left(p_{m}\right) \leq f\left(x\right) \leq f\left(p_{M}\right), \quad \forall x \in C .\]
Tenemos que 
\[
\begin{split}
	C = & \Int\left(C\right) \cup \partial C \\
	= &  \left\{x^{2}+y^{2}+z^{2} = 4 , z < 1\right\} \cup \left\{ x^{2}+y^{2}+z^{2} < 4,  z = 1\right\}  \cup \left\{ x^{2} +y^{2}+z^{2} = 4, z = 1\right\} .
\end{split}
\]
Podemos considerar $\displaystyle F\left(x,y,z\right)=x^{2}+y^{2}+z^{2}-4 $ y $\displaystyle G\left(x,y,z\right) = z-1 $. Tenemos que 
\[\nabla f = \left(2x+1, 2y+1, 2z+1\right) = \left(0,0,0\right) \Rightarrow \left(x,y,z\right) = \left(-\frac{1}{2}, -\frac{1}{2}, -\frac{1}{2}\right) .\]
Cogemos $\displaystyle p_{0} = \left(-\frac{1}{2}, -\frac{1}{2}, -\frac{1}{2}\right) \in \Int\left(C\right) $. Así, existe $\displaystyle \lambda \in \R $ tal que $\displaystyle \nabla f = \lambda \nabla F $, por lo que 
\[
\begin{cases}
2x+1 = \lambda 2x \\
2y+1 = \lambda 2y \\
2z+1 = \lambda 2z \\
x^{2}+y^{2}+z^{2} = 4 \\
z < 1
\end{cases}
.\]
De aquí sacamos $\displaystyle x = y = z $ y sacamos dos puntos
\[p_{1} = \left(\frac{2}{\sqrt{3}}, \frac{2}{\sqrt{3}}, \frac{2}{\sqrt{3}}\right) \not\in M_{1}, \quad p_{2} = \left(-\frac{2}{\sqrt{3}}, -\frac{2}{\sqrt{3}}, -\frac{2}{\sqrt{3}}\right) \in M_{1}.\]
Así, el único punto que nos vale es $\displaystyle p_{2} $. Por otro lado, existe $\displaystyle \mu \in \R $ tal que $\displaystyle \nabla f = \mu \nabla G $:
\[
\begin{cases}
2x+1 = \mu \cdot 0 \\
2y+1 = \mu \cdot 0 \\
2z + 1 = \mu \cdot 1 \\
z = 1 \\
x^{2} +y^{2}+z^{2} < 4
\end{cases}
.\]
De aquí sacamos el punto $\displaystyle p_{3} = \left(-\frac{1}{2}, -\frac{1}{2}, 1\right) \in M_{2} $. Ahora estudiamos $\displaystyle M_{3} $ (hay que comprobar previamente que $\displaystyle M_{1}, M_{2} $ y $\displaystyle M_{3} $ son variedades diferenciables). 
Si $\displaystyle p_{m}, p_{M} \in M_{3} $, existen $\displaystyle \lambda, \mu \in \R $ tales que $\displaystyle \nabla = \lambda \nabla F + \mu \nabla G $. Así, tenemos que
\[
\begin{cases}
2x+1 = \lambda 2x + 0 \cdot \mu \\
2y+1 = \lambda 2y + 0 \cdot \mu \\
2z + 1 = \lambda 2z + \mu \cdot 1 \\
x^{2} +y^{2} +z^{2} = 4 \\
z = 1
\end{cases}
.\]
De aquí sacamos 
\[p_{4} = \left(\sqrt{\frac{3}{2}}, \sqrt{\frac{3}{2}}, 1\right) \quad \text{y} \quad p_{5} = \left(-\sqrt{\frac{3}{2}}, -\sqrt{\frac{3}{2}}, 1\right) .\]
Tenemos que $\displaystyle p_{m}, p_{M} \in \left\{ p_{0}, p_{2}, p_{3}, p_{4}, p_{5}\right\}  $. Comparando los valores de las imágenes obtenemos que el máximo se obtiene en $\displaystyle p_{4} $ con $\displaystyle f\left(p_{4}\right) = 5 +\sqrt{6} $; y el mínimo en $\displaystyle p_{0} $ con $\displaystyle f\left(p_{0}\right) = -\frac{3}{4} $.
\end{enumerate}

\end{eg}

