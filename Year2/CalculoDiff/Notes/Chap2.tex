\chapter{Continuidad}
\begin{definition}[Continuidad]
Sea $\displaystyle f : \left(X,d _{X}\right) \to \left(Y, d _{Y}\right) $ una función entre dos espacios métricos y $\displaystyle x_{0} \in X $. Se dice que $\displaystyle f $ es \textbf{continua} en $\displaystyle x_{0} $ si 
\[\forall \epsilon > 0, \exists \delta > 0, \; d _{X}\left(x,x_{0}\right) < \delta \Rightarrow d _{Y}\left(f\left(x\right), f\left(x_{0}\right)\right) < \epsilon .\]
. Decimos que $\displaystyle f $ es continua en un subconjunto $\displaystyle M \subset X $ si es continua en $\displaystyle x_{0} $, $\displaystyle \forall x_{0} \in M $.
\end{definition}
\begin{observation}
Una definición equivalente es
\[\forall \epsilon > 0, \exists \delta > 0, \; x \in \left(B_{X}\left(x_{0}, \delta \right)\right) \Rightarrow f\left(x\right) \in B _{Y}\left(f\left(x_{0}\right), \epsilon \right) .\]
Es decir, 
\[\forall \epsilon > 0, \exists \delta > 0, \; f\left(B_{X}\left(x_{0}, \delta \right)\right) \subset B_{Y}\left(f\left(x_{0}\right), \epsilon \right) .\]
\end{observation}
\begin{prop}
Sea $\displaystyle f : \left(X, d _{X}\right) \to \left(Y, d _{Y}\right) $ y $\displaystyle x_{0} \in X $. Son equivalentes:
\begin{enumerate}
\item $\displaystyle f $ es continua en $\displaystyle x_{0} $.
\item $\displaystyle \forall \left\{ x_{n}\right\} _{n\in\N}\subset X $ con $\displaystyle x_{n} \to x_{0} $ en $\displaystyle X $, entonces la sucesión $\displaystyle \left\{ f\left(x_{n}\right)\right\} _{n\in\N} $ converge a $\displaystyle f\left(x_{0}\right) $ en $\displaystyle Y $.
\end{enumerate}
\end{prop}
\begin{proof}
\begin{description}
	\item[(i)] Sea $\displaystyle \left\{ x_{n}\right\} _{n\in\N}\subset X $ una sucesión cualquiera con $\displaystyle x_{n} \to x_{0} $. Dado $\displaystyle \epsilon > 0 $, sabemos que existe $\displaystyle \delta > 0 $ tal que si $\displaystyle d _{X}\left(x,x_{0}\right) < \delta  $, entonces $\displaystyle d _{Y}\left(f\left(x\right), f\left(x_{0}\right)\right) < \epsilon  $. Tenemos que existe $\displaystyle n_{0} \in \N $ tal que si $\displaystyle n \geq n_{0} $ se tiene que $\displaystyle d _{X}\left(x_{n}, x_{0}\right) < \delta  $. Por tanto, $\displaystyle \forall n \geq n_{0} $ se tiene que $\displaystyle d _{Y}\left(f\left(x_{n}\right), f\left(x_{0}\right)\right) < \epsilon  $.
	\item[(ii)] Supongamos que $\displaystyle f $ no es continua en $\displaystyle x_{0} $. Así, existe un $\displaystyle \epsilon > 0 $ tal que $\displaystyle \forall \delta > 0 $ existe $\displaystyle x_{\delta } \in X $ tal que $\displaystyle d _{X}\left(x, x_{0}\right) < \delta  $ y $\displaystyle d _{Y}\left(f\left(x\right), f\left(x_{0}\right)\right) \geq \epsilon  $. 
		Si $\displaystyle n \in \N $ podemos tomar $\displaystyle x_{n} \in X $ tal que $\displaystyle d _{X}\left(x_{n}, x_{0}\right) < \delta = \frac{1}{n} $ y $\displaystyle d _{Y}\left(f\left(x\right), f\left(x_{0}\right)\right) \geq \epsilon  $. Por tanto, tenemos que $\displaystyle x_{n} \to x_{0} $ en $\displaystyle X $, pero $\displaystyle f\left(x_{n}\right) \not \to f\left(x_{0}\right) $ en $\displaystyle Y $.  
\end{description}
\end{proof}
\begin{observation}
	Si $\displaystyle f $ no es continua en $\displaystyle x_{0} $, tenemos que existe una sucesión $\displaystyle \left\{ x_{n}\right\} _{n\in\N} \subset X $ tal que 
	\[ d _{Y}\left(f\left(x_{n}\right), f\left(x_{0}\right)\right) \geq \epsilon, \; \forall n \in \N .\]
	Por tanto, $\displaystyle f\left(x_{n}\right) \not \to f\left(x_{0}\right) $ y ninguna subsucesión suya converge a $\displaystyle f\left(x_{0}\right) $.
\end{observation}
\begin{prop}
Las funciones $\displaystyle s : \R^{2} \to \R : \left(x,y\right) \to x + y $ y $\displaystyle p : \R^{2} \to \R : \left(x,y\right) \to x \cdot y $, son continuas en $\displaystyle \R^{2} $ (con las normas $\displaystyle \| \cdot \|_{1}, \| \cdot \|_{2}, \| \cdot \| _{\infty} $).
\end{prop}
\begin{proof}
	Sea $\displaystyle \left\{ z_{n} = \left(x_{n},y_{n}\right)\right\} _{n\in\N} \subset \R^{2} $ tal que $\displaystyle z_{n} \to \left(x_{0}, y_{0}\right) \in \R^{2} $. Tenemos que $\displaystyle x_{n} \to x_{0} $ e $\displaystyle y_{n} \to y_{0} $ en $\displaystyle \R $. Ahora, 
	\[s\left(x_{n}, y_{n}\right) = x_{n} +y_{n} \to x_{0} +y_{0} = s\left(x_{0}, y_{0}\right) .\]
	De forma similar, 
	\[p\left(x_{n}, y_{n}\right) = x_{n} \cdot y_{n} \to x_{0} \cdot y_{0} = p\left(x_{0}, y_{0}\right) .\]
Así, tenemos que $\displaystyle s $ y $\displaystyle p $ son continuas.	
\end{proof}
\begin{prop}
Sean $\displaystyle f : \left(X, d _{X}\right) \to \left(Y, d _{Y}\right) $ y $\displaystyle g : \left(Y, d _{Y}\right) \to \left(Z, d _{Z}\right) $. Supongamos que $\displaystyle f $ es continua en $\displaystyle x_{0} $ y $\displaystyle g $ es continua en $\displaystyle f\left(x_{0}\right) $. Entonces $\displaystyle g\circ f : \left(X, d _{X}\right) \to \left(Z, d _{Z}\right) $ es continua en $\displaystyle x_{0} $.
\end{prop}
\begin{proof}
Dado $\displaystyle \epsilon > 0 $, tenemos que existe $\displaystyle \delta _{1} > 0 $ tal que si $\displaystyle d _{Y}\left(y, f\left(x_{0}\right)\right) < \delta _{1} $, entonces $\displaystyle d _{Z}\left(g\left(y\right), g\left(f\left(x_{0}\right)\right)\right) < \epsilon  $. Así, existe $\displaystyle \delta > 0 $ tal que si $\displaystyle d _{X}\left(x,x_{0}\right) < \delta  $ entonces $\displaystyle d _{Y}\left(f\left(x\right), f\left(x_{0}\right)\right) < \delta_{1} $. Por tanto, si $\displaystyle d _{X}\left(x,x_{0}\right) < \delta  $, tenemos que $\displaystyle d _{Z}\left(g\left(f\left(x\right)\right), g\left(f\left(x_{0}\right)\right)\right) < \epsilon  $.
\end{proof}
\begin{eg}
Las funciones siguientes son continuas: $\displaystyle f,g : \R^{2} \to \R $.
\begin{itemize}
\item $\displaystyle f\left(x,y\right) = \cos\left(xy\right) $.
\item $\displaystyle g\left(x,y\right) = \sin\left(x + y\right)^{3} $. 
\end{itemize}
\end{eg}
\begin{prop}
Sea $\displaystyle f : \left(X, d _{X}\right) \to \R^{n} $ (con $\displaystyle \| \cdot \|_{1}, \| \cdot \|_{2} $ o $\displaystyle \| \cdot \|_{\infty} $). Entonces, $\displaystyle f\left(x\right) = \left(f_{1}\left(x\right), f_{2}\left(x\right), \ldots, f_{n}\left(x\right)\right) $. Por tanto, $\displaystyle f $ es continua en $\displaystyle x_{0} \in X $ si y solo si $\displaystyle f_{1}, \ldots, f_{n} $ son continuas en $\displaystyle x_{0} $.
\end{prop}
\begin{proof}
\begin{description}
	\item[(i)] Supongamos que $\displaystyle f $ es continua en $\displaystyle x_{0} $. Sea $\displaystyle \left\{ x_{j}\right\} _{j\in \N} \subset X $ tal que $\displaystyle x_{j} \to x_{0} $ en $\displaystyle \left(X, d _{X}\right) $. Entonces, tenemos que 
	\[f\left(x_{j}\right) = \left(f_{1}\left(x_{j}\right), \ldots, f_{n}\left(x_{j}\right)\right) \to f\left(x_{0}\right) = \left(f_{1}\left(x_{0}\right), \ldots, f_{n}\left(x_{0}\right)\right) .\]
	Así, tenemos que $\displaystyle \forall i = 1, \ldots, n $, $\displaystyle f_{i}\left(x_{j}\right) \to f_{i}\left(x_{0}\right) $, por lo que $\displaystyle \forall i = 1, \ldots, n $ se tiene que $\displaystyle f_{i} $ es continua. 
\item[(ii)] Sea $\displaystyle \left\{ x_{j}\right\} _{j \in \N} \subset X $ tal que $\displaystyle x_{j} \to x_{0} $ en $\displaystyle \left(X, d _{X}\right) $. Tenemos que $\displaystyle \forall i = 1, \ldots, n $, $\displaystyle f_{i}\left(x_{j}\right) \to f_{i}\left(x_{0}\right) $, por lo que $\displaystyle f\left(x_{j}\right) \to f\left(x_{0}\right) $ en $\displaystyle \R^{n} $ y $\displaystyle f $ es continua en $\displaystyle x_{0} $.
\end{description}
\end{proof}
\begin{colorary}
Sean $\displaystyle f,g : \left(X, d _{X}\right) \to \R $. Si $\displaystyle f $ y $\displaystyle g $ son continuas en $\displaystyle x_{0} $, entonces $\displaystyle f + g $ y $\displaystyle f \cdot g $ también son continuas en $\displaystyle x_{0} $. 
\end{colorary}
\begin{proof}
Consideremos $\displaystyle h = \left(f,g\right) : \left(X, d _{X}\right) \to \R^{2} $ tal que $\displaystyle h\left(x\right) = \left(f\left(x\right), g\left(x\right)\right) $. Tenemos que $\displaystyle h $ es continua en $\displaystyle x_{0} $ por la proposición anterior. Además,
\[
\begin{split}
	f + g : \left(X, d _{X}\right) & \to^{h} \R^{2} \to^{s} \R \\
	x & \to \left(f\left(x\right), g\left(x\right)\right) \to f\left(x\right) + g\left(x\right),
\end{split}
\]
que es continua en $\displaystyle x_{0} $. Similarmente, 
\[
\begin{split}
	f \cdot g : \left(X, d _{X}\right) & \to^{h} \R^{2} \to^{p} \R \\
	x & \to \left(f\left(x\right), g\left(x\right)\right) \to f\left(x\right) \cdot g\left(x\right),
\end{split}
\]
que también es continua en $\displaystyle x_{0} $.
\end{proof}
\begin{eg}
	\begin{itemize}
	\item La función $\displaystyle u\left(x,y\right) = \sin\left(x+y\right)^{3} + \cos\left(xy\right) $ es continua en $\displaystyle \R^{2} $.
	\item La función $\displaystyle w\left(x,y,z\right) = \sin\left(x + y +z \right)^{3} + \cos\left(xyz + 2xy\right) $ es continua en $\displaystyle \R^{3} $.
	\item La función $\displaystyle v\left(x,y,z\right)= \log\left(1 + x^{2} +y^{2}\right) $ es continua en $\displaystyle \R^{3} $. 
	\end{itemize}
\end{eg}
\begin{observation}
Las proyecciones $\displaystyle \pi_{i} : \R^{n} \to \R $, $\displaystyle \pi_{i}\left(x_{1}, \ldots, x_{n}\right) = x_{i} $, son continuas en $\displaystyle \R^{n} $ para $\displaystyle \forall i = 1, \ldots, n $. En efecto, tenemos que la identidad es continua y sus componentes son $\displaystyle \left(\pi_{1}, \ldots, \pi_{n}\right) $,
\[id \left(x\right) = id\left(x_{1}, \ldots, x_{n}\right) = \left(\pi_{1}\left(x\right), \ldots, \pi_{n}\left(x\right)\right) = \left(x_{1}, \ldots, x_{n}\right) .\]
\end{observation}
\begin{observation}
Toda aplicación lineal $\displaystyle T : \R^{n} \to \R^{m} $ es continua (con $\displaystyle \| \cdot \|_{1}, \| \cdot \|_{2} $ y $\displaystyle \| \cdot \|_{\infty} $). En efecto, tenemos que $\displaystyle T\left(x\right) = \left(T_{1}\left(x\right), \ldots, T_{m}\left(x\right)\right) \in \R^{m} $ y existe $\displaystyle A \in \mathcal{M}_{n \times m}\left(\R\right) $ tal que 
\[A \begin{pmatrix} x_{1} \\ \vdots \\ x_{m} \end{pmatrix} = \begin{pmatrix} T_{1}\left(x\right) \\ \vdots \\ T_{m}\left(x\right) \end{pmatrix} .\]
Así, tenemos que $\displaystyle T_{i}\left(x\right) = \sum^{n}_{k = 1}a_{ik}x_{k} $, que es continua $\displaystyle \forall i = 1, \ldots, m $. Como las coordenadas son continuas tenemos que $\displaystyle T $ es continua.
\end{observation}
\begin{observation}
Veremos que $\displaystyle \mathcal{M}_{n \times m} \cong \R^{n \cdot m} $.
\end{observation}
\section{Continuidad global}
\begin{notation}
Sea $\displaystyle f : X \to Y $. 
\begin{itemize}
	\item Para $\displaystyle y \in Y $ definimos $\displaystyle f^{-1}\left( \left\{ y\right\} \right) = \left\{ x \in X \; : \; f\left(x\right) = y\right\}  $.
	\item Para $\displaystyle M \subset Y $ definimos $\displaystyle f^{-1}\left(M\right) = \left\{ x \in X \; : \; f\left(x\right) \in M\right\}  $.
\end{itemize}
\end{notation}
\begin{theorem}
Sea $\displaystyle f : \left(X, d _{X}\right) \to \left(Y, d _{Y}\right) $. Son equivalentes:
\begin{enumerate}
\item $\displaystyle f $ es continua en $\displaystyle X $.
\item $\displaystyle \forall V \subset Y $ abierto, $\displaystyle f^{-1}\left(V\right) $ es abierto en $\displaystyle X $.
\item $\displaystyle \forall H \subset Y $ cerrado, $\displaystyle f^{-1}\left(H\right) $ es cerrado en $\displaystyle X $.
\end{enumerate}
\end{theorem}
\begin{proof}
\begin{description}
\item[(1) $\displaystyle \Rightarrow $ (2)] Sea $\displaystyle V \subset Y$ abierto y $\displaystyle x_{0} \in f^{-1}\left(V\right) $. Tenemos que $\displaystyle f\left(x_{0}\right) \in V $, que es abierto en $\displaystyle Y $, por lo que existe $\displaystyle \epsilon > 0 $ tal que $\displaystyle B _{Y}\left(f\left(x_{0}\right), \epsilon \right) \subset V $. 
Por ser $\displaystyle f $ continua en $\displaystyle X $ existe $\displaystyle \delta > 0 $ tal que $\displaystyle f\left(B _{X}\left(x_{0}, \delta \right)\right) \subset B_{Y}\left(f\left(x_{0}\right), \epsilon \right) \subset V $, por lo que $\displaystyle B_{X}\left(x_{0}, \delta \right)\subset f^{-1}\left(V\right) $.
\item[(2) $\displaystyle \Rightarrow $ (1)] Sea $\displaystyle x_{0} \in X $. Dado $\displaystyle \epsilon > 0 $ podemos considerar $\displaystyle V =  B_{Y}\left(f\left(x_{0}\right), \epsilon \right)$, que es abierto en $\displaystyle Y $. Por hipótesis tenemos que $\displaystyle f^{-1}\left(V\right) $ es abierto en $\displaystyle X $, por lo que existe $\displaystyle \delta > 0 $ tal que $\displaystyle B_{X}\left(x_{0}, \delta \right) \subset f^{-1}\left(V\right) $. Así, nos queda que 
	\[f\left(B_{X}\left(x_{0}, \delta \right)\right) \subset V = B_{Y}\left(f\left(x_{0}\right), \epsilon \right) .\]
	Es decir, $\displaystyle f $ es continua en $\displaystyle x_{0} $. 
\item[(2) $\displaystyle  \Rightarrow $ (3)] Sea $\displaystyle H \subset Y $ cerrado, por lo que $\displaystyle V = Y / H $ es abierto. Así, tenemos que $\displaystyle f^{-1}\left(V\right) $ es abierto en $\displaystyle X $, por lo que
	\[f^{-1}\left(V\right) = \left\{ x \in X \; : \; f\left(x\right) \not\in H\right\} = X / f^{-1}\left(H\right) .\]
	Por tanto, $\displaystyle f^{-1}\left(H\right) $ es cerrado en $\displaystyle Y $.
\item[(3) $\displaystyle  \Rightarrow $ (2)] Sea $\displaystyle V \subset Y $ abierto. Consideramos $\displaystyle H = Y / V $ cerrado. Tenemos que $\displaystyle f^{-1}\left(H\right) $ es cerrado en $\displaystyle X $. Como  
	\[f^{-1}\left(H\right) = \left\{ x \in X \; : \; f\left(x\right) \not\in V\right\}  = X / f^{-1}\left(V\right) ,\]
	se sigue que $\displaystyle f^{-1}\left(V\right) $ es abierto en $\displaystyle X $.
\end{description}
\end{proof}
\begin{eg}
\begin{enumerate}
	\item Sea $\displaystyle A = \left\{ \left(x,y\right) \in \R^{2} \; : \; x^{2} < 1 + y^{2}\right\} = \left\{ \left(x,y\right) \in \R^{2} \; : \; x^{2} - y^{2} - 1 < 0\right\}  $. Sea $\displaystyle f\left(x,y\right) = x^{2} - y^{2} -1 $, que es continua en $\displaystyle \R^{2} $. Podemos decir que $\displaystyle A = \left\{ \left(x,y\right)\in \R^{2} \; : \; f\left(x,y\right) \in \left(-\infty, 0\right)\right\} = f^{-1}\left(\left(-\infty,0\right)\right) $. Como $\displaystyle \left(-\infty,0\right) $ es abierto en $\displaystyle \R $, tenemos que $\displaystyle A $ es abierto en $\displaystyle \R^{2} $.
	\item Sea $\displaystyle B = \left\{ \left(x,y,z\right) \in \R^{3} \; : \; x^{2} +y ^{2} \leq 1 + z^{2}\right\}  $. Podemos considerar $\displaystyle g : \R^{3} \to \R : \left(x,y,z\right) \to x^{2} + y^{2}-z^{2} -1 $, que es continua. Tenemos, pues que $\displaystyle B = g^{-1}\left(\left(-\infty,0\right]\right) $ que es cerrado, por lo que $\displaystyle B $ es cerrado en $\displaystyle \R^{3} $.
	\item Sea $\displaystyle C = \left\{ \left(x,y,z\right) \in \R^{3} \; : \; x^{2} + y^{2} = 1 + z^{2}\right\}  $. Podemos tomar $\displaystyle h\left(x,y,z\right) = x^{2} +y^{2} -z^{2} -1 $, que es continua. Así, tenemos que $\displaystyle C = h^{-1}\left( \left\{ 0\right\} \right) $ y como $\displaystyle \left\{ 0\right\}  $ es cerrado en $\displaystyle \R $, tenemos que $\displaystyle C $ es cerrado. 
\end{enumerate}
\end{eg}
\section{Continuidad y restricciones}
\begin{prop}
Sea $\displaystyle f : \left(X,d _{X}\right) \to \left(Y, d _{Y}\right) $ es continua en $\displaystyle X $ y sea $\displaystyle M \subset X $. Entonces la restricción de $\displaystyle f $ a $\displaystyle M $ \footnote{Se define la función de $\displaystyle f $ restringida a $\displaystyle M $ como $\displaystyle f|_{M}:\left(M, d _{X}|_{M}\right) \to \left(Y, d _{Y}\right) $} es continua en $\displaystyle M $. 
\end{prop}
\begin{proof}
	Sea $\displaystyle z_{0} \in M \subset X $ y sea $\displaystyle \left\{ x_{k}\right\} _{k \in \N} \subset M $ con $\displaystyle x_{k} \to z_{0} $ en $\displaystyle d _{X} |_{M} $, por lo que $\displaystyle x_{k} \to z_{0} $ en $\displaystyle d _{X} $. Como $\displaystyle f $ es cotinua tenemos que $\displaystyle f\left(x_{k}\right) \to f\left(z_{0}\right) $, por lo que $\displaystyle f|_{M} $ es continua en $\displaystyle z_{0} $.
\end{proof}
\begin{prop}
Supongamos que $\displaystyle f : \left(X, d _{X}\right) \to \left(Y, d _{Y}\right) $ y sea $\displaystyle A \subset X $ abierto. Si $\displaystyle f|_{A} : \left(A, d _{X}|_{A}\right) \to \left(Y, d _{Y}\right) $ es continua en $\displaystyle A $, entonces $\displaystyle f $ es continua en $\displaystyle A $.
\end{prop}
\begin{proof}
	Sea $\displaystyle x_{0} \in A $ y $\displaystyle \left\{ x_{k}\right\} _{k \in \N} \subset X $ con $\displaystyle x_{k} \to x_{0} $. Como $\displaystyle A $ es abierto, existe $\displaystyle r > 0 $ tal que $\displaystyle B_{X}\left(x_{0},r\right) \subset A $. Por otro lado, existe $\displaystyle k_{0} \in \N $ tal que $\displaystyle \forall k \geq k_{0} $, $\displaystyle x_{k} \in B_{X}\left(x_{0}, r\right) $.
	Así, como $\displaystyle \left\{ x_{k}\right\}_{k \geq k_{0}} \subset A $, tenemos que $\displaystyle x_{k} \to x_{0} $ por lo que $\displaystyle f\left(x_{k}\right) \to f\left(x_{0}\right) $ \footnote{Hemos usado que el límite de dos sucesiones que difieren en un número finito de términos es el mismo.}. Por tanto, $\displaystyle f $ es continua en $\displaystyle x_{0} $.
\end{proof}
\begin{eg}
Sea $\displaystyle f : \R^{2} \to \R $ con 
\[f\left(x,y\right) = 
\begin{cases}
\frac{xy}{x^{2}+y^{2}}, \; x^{2} +y^{2} \neq 0 \\
0, \; \left(x,y\right) = \left(0,0\right)
\end{cases}
.\]
Consideremos $\displaystyle A =\R^{2} / \left\{ \left(0,0\right)\right\} = \left\{ \left(x,y\right) \in \R^{2} \; : \; \left(x,y\right) \neq \left(0,0\right)\right\}  $, que es abierto en $\displaystyle \R^{2} $. Tenemos que $\displaystyle f|_{A}\left(x,y\right) = \frac{xy}{x^{2} +y^{2}} $ es continua en $\displaystyle A $. Por la proposición anterior tenemos que $\displaystyle f $ es continua en todos los puntos de $\displaystyle A $. \\
Veamos si es continua en $\displaystyle \left(0,0\right) $. Dado $\displaystyle \lambda \in \R $ consideramos $\displaystyle M_{\lambda } = \left\{ \left(x,y\right) \; : \; y = \lambda x\right\}  $. Tenemos que
\[f | _{M_{\lambda }}\left(x,y\right) = \frac{xy}{x^{2} + y^{2}} = \frac{\lambda x^{2}}{x^{2} + \lambda^{2}x^{2}} = \frac{\lambda }{1 + \lambda ^{2}} .\]
Como $\displaystyle f|_{M_{\lambda }} $ es constante en $\displaystyle M_{\lambda } $, tenemos que $\displaystyle f|_{M_{\lambda }} $ es continua. Sin embargo, tenemos que $\displaystyle f $ no es continua en $\displaystyle \left(0,0\right) $, puesto que adopta valores distintios para cada $\displaystyle M_{\lambda} $.
\end{eg}
\begin{lema}[Lema de Pegado]
Sea $\displaystyle f : \left(X, d _{X}\right) \to \left(Y, d _{Y}\right) $. Supongamos que $\displaystyle X = F_{1} \cup \cdots \cup F_{m} $ es una unión finita de cerrados tales que $\displaystyle f|_{F_{i}} : \left(F_{i}, d _{X}|_{F_{i}}\right) \to \left(Y, d _{Y}\right) $ es continua $\displaystyle \forall i = 1, \ldots, m $. Entonces, $\displaystyle f $ es continua en $\displaystyle X $.
\end{lema}
\begin{proof}
Sea $\displaystyle H \subset Y $ cerrado. Tenemos que 
\[f^{-1}\left(H\right) = \left\{ x \in X \; : \; f\left(x\right) \in H\right\} = \bigcup_{1 \leq i \leq m} \left\{ x \in F_{i} \; : \; f\left(x\right) \in H\right\} = \bigcup_{1\leq i \leq m}\left(f|_{F_{i}}\right)^{-1}\left(H\right) .\]
Tenemos que $\displaystyle \left(f|_{F_{i}}\right)^{-1}\left(H\right) $ es cerrado relativo en $\displaystyle \left(F_{i}, d _{X}|_{F_{i}}\right) $ y que $\displaystyle F_{i} $ es cerrado en $\displaystyle X $. Por tanto, $\displaystyle \left(f|_{F_{i}}\right)^{-1}\left(H\right) $ es cerrado en $\displaystyle X $ \footnote{Hemos usado que $\displaystyle C $ es cerrado relativo en $\displaystyle \left(Z, d _{Z}\right) $ si y solo si existe $\displaystyle F $ cerrado en $\displaystyle X $ tal que $\displaystyle C = Z \cap F $.} y $\displaystyle f $ es continua en $\displaystyle X $.
\end{proof}

