\chapter{Continuidad}
\begin{definition}[Continuidad]
Sea $\displaystyle f : \left(X,d _{X}\right) \to \left(Y, d _{Y}\right) $ una función entre dos espacios métricos y $\displaystyle x_{0} \in X $. Se dice que $\displaystyle f $ es \textbf{continua} en $\displaystyle x_{0} $ si 
\[\forall \epsilon > 0, \exists \delta > 0, \; d _{X}\left(x,x_{0}\right) < \delta \Rightarrow d _{Y}\left(f\left(x\right), f\left(x_{0}\right)\right) < \epsilon .\]
. Decimos que $\displaystyle f $ es continua en un subconjunto $\displaystyle M \subset X $ si es continua en $\displaystyle x_{0} $, $\displaystyle \forall x_{0} \in M $.
\end{definition}
\begin{observation}
Una definición equivalente es
\[\forall \epsilon > 0, \exists \delta > 0, \; x \in \left(B_{X}\left(x_{0}, \delta \right)\right) \Rightarrow f\left(x\right) \in B _{Y}\left(f\left(x_{0}\right), \epsilon \right) .\]
Es decir, 
\[\forall \epsilon > 0, \exists \delta > 0, \; f\left(B_{X}\left(x_{0}, \delta \right)\right) \subset B_{Y}\left(f\left(x_{0}\right), \epsilon \right) .\]
\end{observation}
\begin{prop}
Sea $\displaystyle f : \left(X, d _{X}\right) \to \left(Y, d _{Y}\right) $ y $\displaystyle x_{0} \in X $. Son equivalentes:
\begin{enumerate}
\item $\displaystyle f $ es continua en $\displaystyle x_{0} $.
\item $\displaystyle \forall \left\{ x_{n}\right\} _{n\in\N}\subset X $ con $\displaystyle x_{n} \to x_{0} $ en $\displaystyle X $, entonces la sucesión $\displaystyle \left\{ f\left(x_{n}\right)\right\} _{n\in\N} $ converge a $\displaystyle f\left(x_{0}\right) $ en $\displaystyle Y $.
\end{enumerate}
\end{prop}
\begin{proof}
\begin{description}
	\item[(i)] Sea $\displaystyle \left\{ x_{n}\right\} _{n\in\N}\subset X $ una sucesión cualquiera con $\displaystyle x_{n} \to x_{0} $. Dado $\displaystyle \epsilon > 0 $, sabemos que existe $\displaystyle \delta > 0 $ tal que si $\displaystyle d _{X}\left(x,x_{0}\right) < \delta  $, entonces $\displaystyle d _{Y}\left(f\left(x\right), f\left(x_{0}\right)\right) < \epsilon  $. Tenemos que existe $\displaystyle n_{0} \in \N $ tal que si $\displaystyle n \geq n_{0} $ se tiene que $\displaystyle d _{X}\left(x_{n}, x_{0}\right) < \delta  $. Por tanto, $\displaystyle \forall n \geq n_{0} $ se tiene que $\displaystyle d _{Y}\left(f\left(x_{n}\right), f\left(x_{0}\right)\right) < \epsilon  $.
	\item[(ii)] Supongamos que $\displaystyle f $ no es continua en $\displaystyle x_{0} $. Así, existe un $\displaystyle \epsilon > 0 $ tal que $\displaystyle \forall \delta > 0 $ existe $\displaystyle x_{\delta } \in X $ tal que $\displaystyle d _{X}\left(x, x_{0}\right) < \delta  $ y $\displaystyle d _{Y}\left(f\left(x\right), f\left(x_{0}\right)\right) \geq \epsilon  $. 
		Si $\displaystyle n \in \N $ podemos tomar $\displaystyle x_{n} \in X $ tal que $\displaystyle d _{X}\left(x_{n}, x_{0}\right) < \delta = \frac{1}{n} $ y $\displaystyle d _{Y}\left(f\left(x\right), f\left(x_{0}\right)\right) \geq \epsilon  $. Por tanto, tenemos que $\displaystyle x_{n} \to x_{0} $ en $\displaystyle X $, pero $\displaystyle f\left(x_{n}\right) \not \to f\left(x_{0}\right) $ en $\displaystyle Y $.  
\end{description}
\end{proof}
\begin{observation}
	Si $\displaystyle f $ no es continua en $\displaystyle x_{0} $, tenemos que existe una sucesión $\displaystyle \left\{ x_{n}\right\} _{n\in\N} \subset X $ tal que 
	\[ d _{Y}\left(f\left(x_{n}\right), f\left(x_{0}\right)\right) \geq \epsilon, \; \forall n \in \N .\]
	Por tanto, $\displaystyle f\left(x_{n}\right) \not \to f\left(x_{0}\right) $ y ninguna subsucesión suya converge a $\displaystyle f\left(x_{0}\right) $.
\end{observation}
\begin{prop}
Las funciones $\displaystyle s : \R^{2} \to \R : \left(x,y\right) \to x + y $ y $\displaystyle p : \R^{2} \to \R : \left(x,y\right) \to x \cdot y $, son continuas en $\displaystyle \R^{2} $ (con las normas $\displaystyle \| \cdot \|_{1}, \| \cdot \|_{2}, \| \cdot \| _{\infty} $).
\end{prop}
\begin{proof}
	Sea $\displaystyle \left\{ z_{n} = \left(x_{n},y_{n}\right)\right\} _{n\in\N} \subset \R^{2} $ tal que $\displaystyle z_{n} \to \left(x_{0}, y_{0}\right) \in \R^{2} $. Tenemos que $\displaystyle x_{n} \to x_{0} $ e $\displaystyle y_{n} \to y_{0} $ en $\displaystyle \R $. Ahora, 
	\[s\left(x_{n}, y_{n}\right) = x_{n} +y_{n} \to x_{0} +y_{0} = s\left(x_{0}, y_{0}\right) .\]
	De forma similar, 
	\[p\left(x_{n}, y_{n}\right) = x_{n} \cdot y_{n} \to x_{0} \cdot y_{0} = p\left(x_{0}, y_{0}\right) .\]
Así, tenemos que $\displaystyle s $ y $\displaystyle p $ son continuas.	
\end{proof}
\begin{prop}
Sean $\displaystyle f : \left(X, d _{X}\right) \to \left(Y, d _{Y}\right) $ y $\displaystyle g : \left(Y, d _{Y}\right) \to \left(Z, d _{Z}\right) $. Supongamos que $\displaystyle f $ es continua en $\displaystyle x_{0} $ y $\displaystyle g $ es continua en $\displaystyle f\left(x_{0}\right) $. Entonces $\displaystyle g\circ f : \left(X, d _{X}\right) \to \left(Z, d _{Z}\right) $ es continua en $\displaystyle x_{0} $.
\end{prop}
\begin{proof}
Dado $\displaystyle \epsilon > 0 $, tenemos que existe $\displaystyle \delta _{1} > 0 $ tal que si $\displaystyle d _{Y}\left(y, f\left(x_{0}\right)\right) < \delta _{1} $, entonces $\displaystyle d _{Z}\left(g\left(y\right), g\left(f\left(x_{0}\right)\right)\right) < \epsilon  $. Así, existe $\displaystyle \delta > 0 $ tal que si $\displaystyle d _{X}\left(x,x_{0}\right) < \delta  $ entonces $\displaystyle d _{Y}\left(f\left(x\right), f\left(x_{0}\right)\right) < \delta_{1} $. Por tanto, si $\displaystyle d _{X}\left(x,x_{0}\right) < \delta  $, tenemos que $\displaystyle d _{Z}\left(g\left(f\left(x\right)\right), g\left(f\left(x_{0}\right)\right)\right) < \epsilon  $.
\end{proof}
\begin{eg}
Las funciones siguientes son continuas: $\displaystyle f,g : \R^{2} \to \R $.
\begin{itemize}
\item $\displaystyle f\left(x,y\right) = \cos\left(xy\right) $.
\item $\displaystyle g\left(x,y\right) = \sin\left(x + y\right)^{3} $. 
\end{itemize}
\end{eg}
\begin{prop}
Sea $\displaystyle f : \left(X, d _{X}\right) \to \R^{n} $ (con $\displaystyle \| \cdot \|_{1}, \| \cdot \|_{2} $ o $\displaystyle \| \cdot \|_{\infty} $). Entonces, $\displaystyle f\left(x\right) = \left(f_{1}\left(x\right), f_{2}\left(x\right), \ldots, f_{n}\left(x\right)\right) $. Por tanto, $\displaystyle f $ es continua en $\displaystyle x_{0} \in X $ si y solo si $\displaystyle f_{1}, \ldots, f_{n} $ son continuas en $\displaystyle x_{0} $.
\end{prop}
\begin{proof}
\begin{description}
	\item[(i)] Supongamos que $\displaystyle f $ es continua en $\displaystyle x_{0} $. Sea $\displaystyle \left\{ x_{j}\right\} _{j\in \N} \subset X $ tal que $\displaystyle x_{j} \to x_{0} $ en $\displaystyle \left(X, d _{X}\right) $. Entonces, tenemos que 
	\[f\left(x_{j}\right) = \left(f_{1}\left(x_{j}\right), \ldots, f_{n}\left(x_{j}\right)\right) \to f\left(x_{0}\right) = \left(f_{1}\left(x_{0}\right), \ldots, f_{n}\left(x_{0}\right)\right) .\]
	Así, tenemos que $\displaystyle \forall i = 1, \ldots, n $, $\displaystyle f_{i}\left(x_{j}\right) \to f_{i}\left(x_{0}\right) $, por lo que $\displaystyle \forall i = 1, \ldots, n $ se tiene que $\displaystyle f_{i} $ es continua. 
\item[(ii)] Sea $\displaystyle \left\{ x_{j}\right\} _{j \in \N} \subset X $ tal que $\displaystyle x_{j} \to x_{0} $ en $\displaystyle \left(X, d _{X}\right) $. Tenemos que $\displaystyle \forall i = 1, \ldots, n $, $\displaystyle f_{i}\left(x_{j}\right) \to f_{i}\left(x_{0}\right) $, por lo que $\displaystyle f\left(x_{j}\right) \to f\left(x_{0}\right) $ en $\displaystyle \R^{n} $ y $\displaystyle f $ es continua en $\displaystyle x_{0} $.
\end{description}
\end{proof}
\begin{colorary}
Sean $\displaystyle f,g : \left(X, d _{X}\right) \to \R $. Si $\displaystyle f $ y $\displaystyle g $ son continuas en $\displaystyle x_{0} $, entonces $\displaystyle f + g $ y $\displaystyle f \cdot g $ también son continuas en $\displaystyle x_{0} $. 
\end{colorary}
\begin{proof}
Consideremos $\displaystyle h = \left(f,g\right) : \left(X, d _{X}\right) \to \R^{2} $ tal que $\displaystyle h\left(x\right) = \left(f\left(x\right), g\left(x\right)\right) $. Tenemos que $\displaystyle h $ es continua en $\displaystyle x_{0} $ por la proposición anterior. Además,
\[
\begin{split}
	f + g : \left(X, d _{X}\right) & \to^{h} \R^{2} \to^{s} \R \\
	x & \to \left(f\left(x\right), g\left(x\right)\right) \to f\left(x\right) + g\left(x\right),
\end{split}
\]
que es continua en $\displaystyle x_{0} $. Similarmente, 
\[
\begin{split}
	f \cdot g : \left(X, d _{X}\right) & \to^{h} \R^{2} \to^{p} \R \\
	x & \to \left(f\left(x\right), g\left(x\right)\right) \to f\left(x\right) \cdot g\left(x\right),
\end{split}
\]
que también es continua en $\displaystyle x_{0} $.
\end{proof}
\begin{eg}
	\begin{itemize}
	\item La función $\displaystyle u\left(x,y\right) = \sin\left(x+y\right)^{3} + \cos\left(xy\right) $ es continua en $\displaystyle \R^{2} $.
	\item La función $\displaystyle w\left(x,y,z\right) = \sin\left(x + y +z \right)^{3} + \cos\left(xyz + 2xy\right) $ es continua en $\displaystyle \R^{3} $.
	\item La función $\displaystyle v\left(x,y,z\right)= \log\left(1 + x^{2} +y^{2}\right) $ es continua en $\displaystyle \R^{3} $. 
	\end{itemize}
\end{eg}
\begin{observation}
Las proyecciones $\displaystyle \pi_{i} : \R^{n} \to \R $, $\displaystyle \pi_{i}\left(x_{1}, \ldots, x_{n}\right) = x_{i} $, son continuas en $\displaystyle \R^{n} $ para $\displaystyle \forall i = 1, \ldots, n $. En efecto, tenemos que la identidad es continua y sus componentes son $\displaystyle \left(\pi_{1}, \ldots, \pi_{n}\right) $,
\[id \left(x\right) = id\left(x_{1}, \ldots, x_{n}\right) = \left(\pi_{1}\left(x\right), \ldots, \pi_{n}\left(x\right)\right) = \left(x_{1}, \ldots, x_{n}\right) .\]
\end{observation}
\begin{observation}
Toda aplicación lineal $\displaystyle T : \R^{n} \to \R^{m} $ es continua (con $\displaystyle \| \cdot \|_{1}, \| \cdot \|_{2} $ y $\displaystyle \| \cdot \|_{\infty} $). En efecto, tenemos que $\displaystyle T\left(x\right) = \left(T_{1}\left(x\right), \ldots, T_{m}\left(x\right)\right) \in \R^{m} $ y existe $\displaystyle A \in \mathcal{M}_{n \times m}\left(\R\right) $ tal que 
\[A \begin{pmatrix} x_{1} \\ \vdots \\ x_{m} \end{pmatrix} = \begin{pmatrix} T_{1}\left(x\right) \\ \vdots \\ T_{m}\left(x\right) \end{pmatrix} .\]
Así, tenemos que $\displaystyle T_{i}\left(x\right) = \sum^{n}_{k = 1}a_{ik}x_{k} $, que es continua $\displaystyle \forall i = 1, \ldots, m $. Como las coordenadas son continuas tenemos que $\displaystyle T $ es continua.
\end{observation}
\begin{observation}
Veremos que $\displaystyle \mathcal{M}_{n \times m} \cong \R^{n \cdot m} $.
\end{observation}
\section{Continuidad global}
\begin{notation}
Sea $\displaystyle f : X \to Y $. 
\begin{itemize}
	\item Para $\displaystyle y \in Y $ definimos $\displaystyle f^{-1}\left( \left\{ y\right\} \right) = \left\{ x \in X \; : \; f\left(x\right) = y\right\}  $.
	\item Para $\displaystyle M \subset Y $ definimos $\displaystyle f^{-1}\left(M\right) = \left\{ x \in X \; : \; f\left(x\right) \in M\right\}  $.
\end{itemize}
\end{notation}
\begin{theorem}
Sea $\displaystyle f : \left(X, d _{X}\right) \to \left(Y, d _{Y}\right) $. Son equivalentes:
\begin{enumerate}
\item $\displaystyle f $ es continua en $\displaystyle X $.
\item $\displaystyle \forall V \subset Y $ abierto, $\displaystyle f^{-1}\left(V\right) $ es abierto en $\displaystyle X $.
\item $\displaystyle \forall H \subset Y $ cerrado, $\displaystyle f^{-1}\left(H\right) $ es cerrado en $\displaystyle X $.
\end{enumerate}
\end{theorem}
\begin{proof}
\begin{description}
\item[(1) $\displaystyle \Rightarrow $ (2)] Sea $\displaystyle V \subset Y$ abierto y $\displaystyle x_{0} \in f^{-1}\left(V\right) $. Tenemos que $\displaystyle f\left(x_{0}\right) \in V $, que es abierto en $\displaystyle Y $, por lo que existe $\displaystyle \epsilon > 0 $ tal que $\displaystyle B _{Y}\left(f\left(x_{0}\right), \epsilon \right) \subset V $. 
Por ser $\displaystyle f $ continua en $\displaystyle X $ existe $\displaystyle \delta > 0 $ tal que $\displaystyle f\left(B _{X}\left(x_{0}, \delta \right)\right) \subset B_{Y}\left(f\left(x_{0}\right), \epsilon \right) \subset V $, por lo que $\displaystyle B_{X}\left(x_{0}, \delta \right)\subset f^{-1}\left(V\right) $.
\item[(2) $\displaystyle \Rightarrow $ (1)] Sea $\displaystyle x_{0} \in X $. Dado $\displaystyle \epsilon > 0 $ podemos considerar $\displaystyle V =  B_{Y}\left(f\left(x_{0}\right), \epsilon \right)$, que es abierto en $\displaystyle Y $. Por hipótesis tenemos que $\displaystyle f^{-1}\left(V\right) $ es abierto en $\displaystyle X $, por lo que existe $\displaystyle \delta > 0 $ tal que $\displaystyle B_{X}\left(x_{0}, \delta \right) \subset f^{-1}\left(V\right) $. Así, nos queda que 
	\[f\left(B_{X}\left(x_{0}, \delta \right)\right) \subset V = B_{Y}\left(f\left(x_{0}\right), \epsilon \right) .\]
	Es decir, $\displaystyle f $ es continua en $\displaystyle x_{0} $. 
\item[(2) $\displaystyle  \Rightarrow $ (3)] Sea $\displaystyle H \subset Y $ cerrado, por lo que $\displaystyle V = Y / H $ es abierto. Así, tenemos que $\displaystyle f^{-1}\left(V\right) $ es abierto en $\displaystyle X $, por lo que
	\[f^{-1}\left(V\right) = \left\{ x \in X \; : \; f\left(x\right) \not\in H\right\} = X / f^{-1}\left(H\right) .\]
	Por tanto, $\displaystyle f^{-1}\left(H\right) $ es cerrado en $\displaystyle Y $.
\item[(3) $\displaystyle  \Rightarrow $ (2)] Sea $\displaystyle V \subset Y $ abierto. Consideramos $\displaystyle H = Y / V $ cerrado. Tenemos que $\displaystyle f^{-1}\left(H\right) $ es cerrado en $\displaystyle X $. Como  
	\[f^{-1}\left(H\right) = \left\{ x \in X \; : \; f\left(x\right) \not\in V\right\}  = X / f^{-1}\left(V\right) ,\]
	se sigue que $\displaystyle f^{-1}\left(V\right) $ es abierto en $\displaystyle X $.
\end{description}
\end{proof}
\begin{eg}
\begin{enumerate}
	\item Sea $\displaystyle A = \left\{ \left(x,y\right) \in \R^{2} \; : \; x^{2} < 1 + y^{2}\right\} = \left\{ \left(x,y\right) \in \R^{2} \; : \; x^{2} - y^{2} - 1 < 0\right\}  $. Sea $\displaystyle f\left(x,y\right) = x^{2} - y^{2} -1 $, que es continua en $\displaystyle \R^{2} $. Podemos decir que $\displaystyle A = \left\{ \left(x,y\right)\in \R^{2} \; : \; f\left(x,y\right) \in \left(-\infty, 0\right)\right\} = f^{-1}\left(\left(-\infty,0\right)\right) $. Como $\displaystyle \left(-\infty,0\right) $ es abierto en $\displaystyle \R $, tenemos que $\displaystyle A $ es abierto en $\displaystyle \R^{2} $.
	\item Sea $\displaystyle B = \left\{ \left(x,y,z\right) \in \R^{3} \; : \; x^{2} +y ^{2} \leq 1 + z^{2}\right\}  $. Podemos considerar $\displaystyle g : \R^{3} \to \R : \left(x,y,z\right) \to x^{2} + y^{2}-z^{2} -1 $, que es continua. Tenemos, pues que $\displaystyle B = g^{-1}\left(\left(-\infty,0\right]\right) $ que es cerrado, por lo que $\displaystyle B $ es cerrado en $\displaystyle \R^{3} $.
	\item Sea $\displaystyle C = \left\{ \left(x,y,z\right) \in \R^{3} \; : \; x^{2} + y^{2} = 1 + z^{2}\right\}  $. Podemos tomar $\displaystyle h\left(x,y,z\right) = x^{2} +y^{2} -z^{2} -1 $, que es continua. Así, tenemos que $\displaystyle C = h^{-1}\left( \left\{ 0\right\} \right) $ y como $\displaystyle \left\{ 0\right\}  $ es cerrado en $\displaystyle \R $, tenemos que $\displaystyle C $ es cerrado. 
\end{enumerate}
\end{eg}
\section{Continuidad y restricciones}
\begin{prop}
Sea $\displaystyle f : \left(X,d _{X}\right) \to \left(Y, d _{Y}\right) $ es continua en $\displaystyle X $ y sea $\displaystyle M \subset X $. Entonces la restricción de $\displaystyle f $ a $\displaystyle M $ \footnote{Se define la función de $\displaystyle f $ restringida a $\displaystyle M $ como $\displaystyle f|_{M}:\left(M, d _{X}|_{M}\right) \to \left(Y, d _{Y}\right) $} es continua en $\displaystyle M $. 
\end{prop}
\begin{proof}
	Sea $\displaystyle z_{0} \in M \subset X $ y sea $\displaystyle \left\{ x_{k}\right\} _{k \in \N} \subset M $ con $\displaystyle x_{k} \to z_{0} $ en $\displaystyle d _{X} |_{M} $, por lo que $\displaystyle x_{k} \to z_{0} $ en $\displaystyle d _{X} $. Como $\displaystyle f $ es cotinua tenemos que $\displaystyle f\left(x_{k}\right) \to f\left(z_{0}\right) $, por lo que $\displaystyle f|_{M} $ es continua en $\displaystyle z_{0} $.
\end{proof}
\begin{prop}
Supongamos que $\displaystyle f : \left(X, d _{X}\right) \to \left(Y, d _{Y}\right) $ y sea $\displaystyle A \subset X $ abierto. Si $\displaystyle f|_{A} : \left(A, d _{X}|_{A}\right) \to \left(Y, d _{Y}\right) $ es continua en $\displaystyle A $, entonces $\displaystyle f $ es continua en $\displaystyle A $.
\end{prop}
\begin{proof}
	Sea $\displaystyle x_{0} \in A $ y $\displaystyle \left\{ x_{k}\right\} _{k \in \N} \subset X $ con $\displaystyle x_{k} \to x_{0} $. Como $\displaystyle A $ es abierto, existe $\displaystyle r > 0 $ tal que $\displaystyle B_{X}\left(x_{0},r\right) \subset A $. Por otro lado, existe $\displaystyle k_{0} \in \N $ tal que $\displaystyle \forall k \geq k_{0} $, $\displaystyle x_{k} \in B_{X}\left(x_{0}, r\right) $.
	Así, como $\displaystyle \left\{ x_{k}\right\}_{k \geq k_{0}} \subset A $, tenemos que $\displaystyle x_{k} \to x_{0} $ por lo que $\displaystyle f\left(x_{k}\right) \to f\left(x_{0}\right) $ \footnote{Hemos usado que el límite de dos sucesiones que difieren en un número finito de términos es el mismo.}. Por tanto, $\displaystyle f $ es continua en $\displaystyle x_{0} $.
\end{proof}
\begin{eg}
Sea $\displaystyle f : \R^{2} \to \R $ con 
\[f\left(x,y\right) = 
\begin{cases}
\frac{xy}{x^{2}+y^{2}}, \; x^{2} +y^{2} \neq 0 \\
0, \; \left(x,y\right) = \left(0,0\right)
\end{cases}
.\]
Consideremos $\displaystyle A =\R^{2} / \left\{ \left(0,0\right)\right\} = \left\{ \left(x,y\right) \in \R^{2} \; : \; \left(x,y\right) \neq \left(0,0\right)\right\}  $, que es abierto en $\displaystyle \R^{2} $. Tenemos que $\displaystyle f|_{A}\left(x,y\right) = \frac{xy}{x^{2} +y^{2}} $ es continua en $\displaystyle A $. Por la proposición anterior tenemos que $\displaystyle f $ es continua en todos los puntos de $\displaystyle A $. \\
Veamos si es continua en $\displaystyle \left(0,0\right) $. Dado $\displaystyle \lambda \in \R $ consideramos $\displaystyle M_{\lambda } = \left\{ \left(x,y\right) \; : \; y = \lambda x\right\}  $. Tenemos que
\[f | _{M_{\lambda }}\left(x,y\right) = \frac{xy}{x^{2} + y^{2}} = \frac{\lambda x^{2}}{x^{2} + \lambda^{2}x^{2}} = \frac{\lambda }{1 + \lambda ^{2}} .\]
Como $\displaystyle f|_{M_{\lambda }} $ es constante en $\displaystyle M_{\lambda } $, tenemos que $\displaystyle f|_{M_{\lambda }} $ es continua. Sin embargo, tenemos que $\displaystyle f $ no es continua en $\displaystyle \left(0,0\right) $, puesto que adopta valores distintios para cada $\displaystyle M_{\lambda} $.
\end{eg}
\begin{lema}[Lema de pegado]
Sea $\displaystyle f : \left(X, d _{X}\right) \to \left(Y, d _{Y}\right) $. Supongamos que $\displaystyle X = F_{1} \cup \cdots \cup F_{m} $ es una unión finita de cerrados tales que $\displaystyle f|_{F_{i}} : \left(F_{i}, d _{X}|_{F_{i}}\right) \to \left(Y, d _{Y}\right) $ es continua $\displaystyle \forall i = 1, \ldots, m $. Entonces, $\displaystyle f $ es continua en $\displaystyle X $.
\end{lema}
\begin{proof}
Sea $\displaystyle H \subset Y $ cerrado. Tenemos que 
\[f^{-1}\left(H\right) = \left\{ x \in X \; : \; f\left(x\right) \in H\right\} = \bigcup_{1 \leq i \leq m} \underbrace{\left\{ x \in F_{i} \; : \; f\left(x\right) \in H\right\}}_{f^{-1}\left(H\right) \cap F_{i}} = \bigcup_{1\leq i \leq m}\left(f|_{F_{i}}\right)^{-1}\left(H\right) .\]
Tenemos que $\displaystyle \left(f|_{F_{i}}\right)^{-1}\left(H\right) $ es cerrado relativo en $\displaystyle \left(F_{i}, d _{X}|_{F_{i}}\right) $ y que $\displaystyle F_{i} $ es cerrado en $\displaystyle X $. Por tanto, $\displaystyle \left(f|_{F_{i}}\right)^{-1}\left(H\right) $ es cerrado en $\displaystyle X $ \footnote{Hemos usado que $\displaystyle C $ es cerrado relativo en $\displaystyle \left(Z, d _{Z}\right) $ si y solo si existe $\displaystyle F $ cerrado en $\displaystyle X $ tal que $\displaystyle C = Z \cap F $.} y $\displaystyle f $ es continua en $\displaystyle X $.
\end{proof}
\begin{eg}
\begin{enumerate}
\item Sea $\displaystyle f : \R^{3} \to \R $ con 
	\[ f\left(x,y,z\right) = 
	\begin{cases}
	\sqrt{x^{2}+y^{2}+z^{2}}, \; \sqrt{x^{2}+y^{2}+z^{2}}\leq 1 \\
	\frac{1}{\sqrt{x^{2}+y^{2}+z^{2}}} \geq 1
	\end{cases}
	.\]
	Podemos considerar 
	\[ \displaystyle F_{1} = \left\{ \left(x,y,z\right) \; : \; \sqrt{x^{2}+y^{2}+z^{2}}\leq 1\right\} \; \text{y} \; F_{2} = \left\{ \left(x,y,z\right) \; : \; \sqrt{x^{2}+y^{2}+z^{2}} \geq 1\right\},\]
	que son cerrados. Tenemos que $\displaystyle \R^{3} = F_{1} \cup F_{2} $ pero $\displaystyle F_{1} \cap F_{2} \neq \emptyset $. Tenemos que $\displaystyle f|_{F_{1}} $ y $\displaystyle f|_{F_{2}} $ son continuas, por lo que $\displaystyle f $ es continua en $\displaystyle \R^{3} $.
\item Veamos que $\displaystyle f : \R^{2} \to \R : \left(x,y\right) \to \max \left\{ x,y\right\}  $ es continua. Podemos considerar los conjuntos cerrados
	\[F_{1} = \left\{ \left(x,y\right) \; : \; y \geq x\right\} \; \text{y} \; F_{2} = \left\{ \left(x,y\right) \; : \; y \leq x\right\}  .\]
	Tenemos que $\displaystyle f|_{F_{1}}\left(x,y\right) = y $ y $\displaystyle f|_{F_{2}} \left(x,y\right) = x $, que son continuas por lo que $\displaystyle f $ es continua en $\displaystyle \R^{2} $.
\end{enumerate}
\end{eg}
\begin{theorem}
Sea $\displaystyle f : \left(X, d _{X}\right) \to \left(Y, d _{Y}\right) $ continua en $\displaystyle X $ \footnote{Bastaría con que fuese continua en $\displaystyle K $.}. Si $\displaystyle \K \subset X $ es compacto, entonces $\displaystyle f\left(K\right) $ es compacto en $\displaystyle Y $.
\end{theorem}
\begin{proof}
	Sea $\displaystyle \left\{ y_{n}\right\} _{n\in\N}\subset f\left(K\right) $ una sucesión cualquiera. Para cada $\displaystyle n \in \N $, como $\displaystyle y_{n} \in f\left(K\right) $, existe $\displaystyle x_{n} \in K $ tal que $\displaystyle f\left(x_{n}\right) = y_{n} $. Así, tenemos la sucesión $\displaystyle \left\{ x_{n}\right\} _{n\in\N} \subset K $. Como $\displaystyle K $ es compacto, existe $\displaystyle \left\{ x_{n_{j}}\right\} _{j \in \N} \subset \left\{ x_{n}\right\} _{n\in\N} $ tal que $\displaystyle x_{n_{j}} \to x_{0} \in \K $. Por ser $\displaystyle f $ continua, tenemos que $\displaystyle f\left(x_{n_{j}}\right) = y_{n_{j}} \to f\left(x_{0}\right) \in f\left(K\right) $.
	Así, hemos visto que la sucesión $\displaystyle \left\{ y_{n}\right\} _{n\in\N} $ tiene una subsucesión convergente en $\displaystyle f\left(K\right) $.
\end{proof}
\begin{colorary}
Sean $\displaystyle \left(X,d _{X}\right) $ un espacio métrico y $\displaystyle f : \left(X, d _{X}\right) \to \R $ continua en $\displaystyle X $. Si $\displaystyle K \subset X $ es compacto, $\displaystyle f $ alcanza un valor máximo y un valor mínimo en $\displaystyle K $, es decir, $\displaystyle \exists x_{m}, x_{M} \in K $ tales que $\displaystyle f\left(x_{m}\right) \leq f\left(x\right) \leq f\left(x_{M}\right), \; \forall x \in K $.
\end{colorary}
\begin{proof}
Como $\displaystyle K \subset X $ es compacto, tenemos que $\displaystyle f\left(K\right) $ es compacto en $\displaystyle \R $, por lo que $\displaystyle f\left(K\right) $ es cerrado y acotado. Por ser $\displaystyle f\left(K\right) $ acotado en $\displaystyle \R $ existen $\displaystyle \alpha = \inf \left(f\left(K\right)\right) $ y $\displaystyle \beta = \sup\left(f\left(K\right)\right) $ con $\displaystyle \alpha, \beta \in \R $. Tenemos que $\displaystyle \alpha, \beta \in \overline{f\left(K\right)} = f\left(K\right) $ puesto que $\displaystyle f\left(K\right) $ es cerrado. Así, existen $\displaystyle x_{m}, x_{M} \in K $ tales que $\displaystyle \alpha = f\left(x_{m}\right) $ y $\displaystyle \beta = f\left(x_{M}\right) $.
\end{proof}
\begin{observation}
Si $\displaystyle M \subset \R $ es acotado, entonces $\displaystyle \alpha = \inf\left(M\right) $ y $\displaystyle \beta = \sup\left(M\right) $ son puntos adherentes a $\displaystyle M $, puesto que $\displaystyle \forall \epsilon > 0 $, $\displaystyle \exists x\in M $ tal que 
\[x \in M \cap \left(\beta - \epsilon, \beta + \epsilon \right) = M \cap B\left(\beta, \epsilon \right) \neq \emptyset .\]
\end{observation}
\begin{theorem}[Normas equivalentes]
En $\displaystyle \R^{n} $ todas las normas son equivalentes.
\end{theorem}
\begin{proof}
Sea $\displaystyle \| \cdot \| $ una norma en $\displaystyle \R^{n} $. Vamos a demostrar que $\displaystyle \| \cdot \| $ es equivalente a $\displaystyle \| \cdot \|_{\infty} $. 
\begin{enumerate}
	\item Consideramos $\displaystyle \left\{ e_{i}\right\} _{i = 1}^{n} $ la base canónica de $\displaystyle \R^{n} $ y sea $\displaystyle M = \sum^{n}_{i = 1}\|e_{i}\| $. Tenemos que si $\displaystyle x \in \R^{n} $,
		\[\|x\| = \left\|\sum^{n}_{i = 1}x_{i}e_{i}\right\| \leq \sum^{n}_{i = 1} \|x_{i}e_{i}\| = \sum^{n}_{i = 1} \left|x_{i}\right| \|e_{i}\| \leq \sum^{n}_{i = 1} \| x \|_{\infty}\|e_{i}\| = M\|x\|_{\infty} .\]
	\item Consideramos $\displaystyle f : \left(\R^{n}, \| \cdot \|_{\infty}\right) \to \R : x \to \| x\|$. Veamos que $\displaystyle f $ es continua en $\displaystyle \R^{n} $. Sea $\displaystyle \left\{ x_{k}\right\} _{k \in \N} \subset \R^{n}$ con $\displaystyle x_{k} \to x_{0} $ en $\displaystyle \left(\R^{n}, \| \cdot \| _{\infty}\right) $. 
Vamos a ver que $\displaystyle f\left(x_{k}\right) \to f\left(x_{0}\right) $:
\[ \left|f\left(x_{k}\right) -f\left(x_{0}\right)\right| = \left| \| x_{k}\| - \|x_{0}\|\right| \leq \| x_{k}-x_{0}\| \leq M\|x_{k}-x_{0}\|_{\infty} \to 0 .\]
\item El conjunto $\displaystyle S_{\| \cdot \|_{\infty}} = \left\{ x \in \R^{n} \; : \; \|x\|_{\infty} = 1\right\}  $ es compacto en $\displaystyle \left(\R^{n}, \| \cdot \|_{\infty}\right) $ puesto que es cerrado y acotado. Por tanto, $\displaystyle f $ alcanza un valor mínimo en $\displaystyle S_{\| \cdot \|_{\infty}} $:
	\[\exists u_{m} \in S_{\| \cdot \|_{\infty}}, \; 0 < \alpha = f\left(u_{m}\right) = \|u_{m}\| \leq \|u\|, \; \forall u \in S_{\| \cdot \|_{\infty}} .\]
	Tenemos que $\displaystyle \|u\| \geq \alpha > 0 $, $\displaystyle \forall u \in \R^{n} $ con $\displaystyle \|u\|_{\infty} = 1 $. Por tanto, $\displaystyle \forall x \neq 0 $,
	\[\|x\| = \left\|\frac{x}{\|x\|_{\infty}} \cdot \|x\|_{\infty}\right\| = \|x\|_{\infty} \cdot \left\|\frac{x}{\|x\|_{\infty}}\right\| \geq \|x\|_{\infty} \cdot \alpha .\]
	Así, nos queda que $\displaystyle \alpha \cdot \|x\|_{\infty} \leq \|x\| $, $\displaystyle \forall x \in \R^{n} $. 
\end{enumerate}
\end{proof}
\begin{colorary}
En un espacio vectorial real de dimensión finita todas las normas son equivalentes. 
\end{colorary}
\begin{proof}
Sea $\displaystyle E $ con $\displaystyle \dim\left(E\right) = n $, entonces $\displaystyle E $ es linealmente isomorfo a $\displaystyle \R^{n} $. 
\end{proof}
\begin{eg}
En el espacio $\displaystyle \mathcal{M}_{n \times m} $ de matrices $\displaystyle n \times m $, todas las normas son equivalentes. 
\end{eg}
\section{Conexión}
\begin{definition}[Intervalo]
Un conjunto $\displaystyle I \subset \R $ es un \textbf{intervalo} si $\displaystyle \forall x,y \in I $ con $\displaystyle x < y $ y $\displaystyle \forall z \in \R $ con $\displaystyle x < z < y $, entonces $\displaystyle z \in I $.
\end{definition}
\begin{theorem}[Teorema de Bolzano]
Sean $\displaystyle I \subset \R $ un intervalo y sea $\displaystyle f : I \to \R $ continua. Si $\displaystyle f $ toma dos valores, entonces toma los valores intermedios. Es decir, si $\displaystyle \exists x,y \in I $ con $\displaystyle f\left(x\right) = u, f\left(y\right) = v $ y $\displaystyle u < w < v $, entonces existe $\displaystyle z \in I $ tal que $\displaystyle f\left(z\right) = w $.
\end{theorem}
\begin{colorary}
Sea $\displaystyle M \subset \R $. Son equivalentes:
\begin{enumerate}
\item $\displaystyle M $ es un intervalo.
\item No existe ninguna función continua $\displaystyle f : M \to \R $ tal que $\displaystyle f\left(M\right) = \left\{ 0,1\right\}  $. 
\item Para toda función continua $\displaystyle f : M \to \R $ con $\displaystyle f\left(M\right) \subset \left\{ 0,1\right\}  $, se tiene que $\displaystyle f $ es constante. 
\end{enumerate}
\end{colorary}
\begin{proof}
\begin{description}
\item[(1) $\displaystyle \Rightarrow $ (2)] Es trivial a partir del teorema de Bolzano. 
\item[(2) $\displaystyle \Rightarrow $ (1)] Supongamos que $\displaystyle M $ no es un intervalo. Por tanto, existen $\displaystyle x,y \in M $ con $\displaystyle x < y $ y existe $\displaystyle z \in \R $ con $\displaystyle x < z < y $ con $\displaystyle z \not\in M $. Definimos $\displaystyle f : M \to \R $ por 
	\[f\left(t\right) =
	\begin{cases}
	0, \; t < z \\ 
	1, \; t > z
	\end{cases}
	.\]
	Está claro que $\displaystyle f\left(M\right) = \left\{ 0,1\right\}  $ y que $\displaystyle f $ es continua en $\displaystyle M $. 
\item[(2) $\displaystyle \iff $ (3)] Trivial.
\end{description}
\end{proof}
\begin{definition}[Conjunto conexo]
Sean $\displaystyle \left(X,d\right) $ un espacio métrico y $\displaystyle M \subset X $. 
\begin{description}
	\item[(a)] Diremos que $\displaystyle M $ es \textbf{conexo} cuando $\displaystyle \forall f : M \to \R $ continua con $\displaystyle f\left(M\right) \subset \left\{ 0,1\right\}  $, entonces $\displaystyle f $ es constante. 
	\item[(b)] Diremos que $\displaystyle M $ es \textbf{desconexo} cuando no es conexo, es decir, existe alguna $\displaystyle f : M \to \R $ continua tal que $\displaystyle f\left(M\right) = \left\{ 0,1\right\}  $.
\end{description}
\end{definition}
\begin{theorem}
Sea $\displaystyle f : \left(X, d _{X}\right) \to \left(Y, d _{Y}\right) $ una función continua entre espacios métricos. Si $\displaystyle M \subset X $ es conexo, entonces $\displaystyle f\left(M\right) $ es conexo. 
\end{theorem}
\begin{proof}
	Sea $\displaystyle h : f\left(M\right) \to \R $ continua con $\displaystyle h\left(f\left(M\right)\right) \subset \left\{ 0,1\right\}  $. Tenemos que $\displaystyle h \circ f|_{M} $ es continua y $\displaystyle h \circ f|_{M}\left(M\right) \subset \left\{ 0,1\right\}  $. Por ser $\displaystyle M $ conexo, tenemos que $\displaystyle h \circ f|_{M} $ es constante, por lo que $\displaystyle h $ es constante. 
\end{proof}
\begin{lema}
Sean $\displaystyle \left(X,d \right) $ un espacio métrico y $\displaystyle A \subset X $. Consideramos $\displaystyle \chi_{A} : X \to \R $ definida por
\[\chi_{A}\left(x\right) = 
\begin{cases}
0, \; x \in X / A \\
1, \; x \in A
\end{cases}
.\]
Entonces $\displaystyle \chi_{A} $ es continua en $\displaystyle X $ si y solo si $\displaystyle A $ es abierto y cerrado a la vez. 
\end{lema}
\begin{proof}
\begin{description}
\item[(i)] Como $\displaystyle \chi_{A} $ es continua, hemos visto anteriormente que el siguiente conjunto es cerrado:
	\[A = \left\{ x \in X \; : \; \chi_{A}\left(x\right) = 1\right\}  = \chi_{A}^{-1}\left( \left\{ 1\right\} \right) .\]
	Tenemos que 
	\[X/A = \left\{ x \in X \; : \; \chi_{A}\left(x\right) = 0\right\}  = \chi_{A}^{-1}\left( \left\{ 0\right\} \right) .\]
	De esta última igualdad obtenemos que $\displaystyle A $ es abierto, por lo que podemos concluir que $\displaystyle A $ es abierto y cerrado a la vez. 
\item[(ii)] Sea $\displaystyle U \subset \R $ abierto. 
	\begin{itemize}
	\item Si $\displaystyle 0,1 \in U $, entonces $\displaystyle \chi_{A}^{-1}\left(U\right) = X $, que es abierto. 
	\item Si $\displaystyle 0,1 \not\in U $, entonces $\displaystyle \chi_{A}^{-1}\left(U\right) = \emptyset $, que es abierto.
	\item Si $\displaystyle 1 \in U $ y $\displaystyle 0 \not\in U $, tenemos que $\displaystyle \chi_{A}^{-1}\left(U\right) = A $, que es abierto.
	\item Si $\displaystyle 1 \not\in U $ y $\displaystyle 0 \in U $, tenemos que $\displaystyle \chi_{A}^{-1}\left(U\right) = X / A $ que es abierto. 
	\end{itemize}
\end{description}
\end{proof}
\begin{eg}
	Consideremos $\displaystyle X = \R $ y $\displaystyle M = \R/ \left\{ z\right\}  $. Tenemos que 
	\[ A =\left(-\infty, z\right) = \left(-\infty,z\right] \cap M ,\]
	por lo que $\displaystyle A $ es abierto y cerrado relativo en $\displaystyle M $. 
\end{eg}
\begin{theorem}
Sean $\displaystyle \left(X,d\right) $ un espacio métrico y $\displaystyle M \subset X $. Son equivalentes:
\begin{enumerate}
\item $\displaystyle M $ es desconexo.
\item $\displaystyle M = A \sqcup B $ donde $\displaystyle A $ y $\displaystyle B $ son abiertos relativos y no vacíos de $\displaystyle \left(M, d|_{M}\right) $. 
\item Existe $\displaystyle A \subset M $ con $\displaystyle \emptyset \neq A \neq M $ tal que $\displaystyle A $ es abierto y cerrado relativo en $\displaystyle \left(M, d|_{M}\right) $. 
\end{enumerate}
\end{theorem}
\begin{proof}
\begin{description}
	\item[(1) $\displaystyle \Rightarrow $ (2)] Si $\displaystyle M $ es desconexo, existe $\displaystyle f : M \to \R $ continua tal que $\displaystyle f\left(M\right) = \left\{ 0,1\right\}  $. Consideramos $\displaystyle A = f^{-1}\left( \left\{ 0\right\} \right) $ y $\displaystyle B = f^{-1}\left( \left\{ 1\right\} \right) $. Es fácil ver que $\displaystyle A $ y $\displaystyle B $ son cerrados relativos en $\displaystyle \left(M, d|_{M}\right) $ y que $\displaystyle A \cap B= \emptyset $. Además, $\displaystyle A = M/B $ y $\displaystyle B = M/A $, por lo que $\displaystyle A $ y $\displaystyle B $ son a la vez abiertos y cerrados relativos en $\displaystyle \left(M, d|_{M}\right) $. 
	\item[(2) $\Rightarrow$ (3)] Si $\displaystyle \emptyset \neq A \neq M $, entonces $\displaystyle A $ es abierto y $\displaystyle B = M / A $ es abierto por lo que $\displaystyle A $ también es cerrado. 
	\item[(3) $\displaystyle \Rightarrow $ (1)] Basta considerar $\displaystyle f = \chi_{A} : M \to \R $ tal que 
		\[f\left(x\right) = 
		\begin{cases}
		0, \; x \in M / A \\
		1, \; x \in M / B
		\end{cases}
		.\]
		Por el lema tenemos que $\displaystyle f $ es continua en $\displaystyle M $ y $\displaystyle f\left(M\right) = \left\{ 0,1\right\}  $. 		
\end{description}
\end{proof}
\begin{definition}[Separación]
Sean $\displaystyle \left(X,d\right) $ un espacio métrico y $\displaystyle M \subset X $. Una \textbf{separación} de $\displaystyle M $ en $\displaystyle X $ es un par $\displaystyle \left(U,V\right) $ donde:
\begin{enumerate}
\item $\displaystyle U $ y $\displaystyle V $ son conjuntos abiertos en $\displaystyle X $.
\item $\displaystyle M \subset U \cup V $. 
\item $\displaystyle M \cap U, M \cap V \neq \emptyset $. 
\item $\displaystyle M \cap U \cap V = \emptyset$.
\end{enumerate}
\end{definition}
\begin{theorem}
Sean $\displaystyle \left(X,d\right) $ un espacio métrico y $\displaystyle M \subset X $. Entonces $\displaystyle M $ es desconexo si y solo si $\displaystyle M $ admite una separación en $\displaystyle X $.
\end{theorem}
\begin{proof}
\begin{description}
\item[(i)] Si $\displaystyle M $ es desconexo, existen $\displaystyle A,B \subset X $ abiertos relativos en $\displaystyle M $ tales que $\displaystyle M = A \sqcup B$. Por ser abiertos relativos, existen $\displaystyle U,V \subset X $ abiertos tales que $\displaystyle A = U \cap M $ y $\displaystyle B = V \cap M $. Es fácil comprobar que $\displaystyle \left(U,V\right) $ es una separación.
\item[(ii)] Si $\displaystyle M $ admite una separación tenemos que $\displaystyle M \subset \left(U \cap M\right) \sqcup \left(V \cap M\right)$, que son abiertos relativos en $\displaystyle M $, por lo que se cumple \textbf{(2)} del teorema anterior y $\displaystyle M $ es desconexo.
\end{description}
\end{proof}
\begin{eg}
Sea $\displaystyle M = \Q $ subespacio de $\displaystyle X = \left(\R, \left| \cdot\right|\right) $. 
\begin{description}
\item[(a)] Un ejemplo de abierto y cerrado relativo en $\displaystyle M $ es
	\[A = \left\{ x \in \Q \; : \; \pi < x < \pi+1\right\}  = \Q \cap \left(\pi , \pi + 1\right) .\]
Está claro que $\displaystyle A $ es abierto relativo en $\displaystyle \Q $ por ser la intersección de un abierto en $\displaystyle \R $ y $\displaystyle \Q $. Además, tenemos que
\[A = \Q \cap [\pi , \pi + 1] .\]
Por la misma razón, $\displaystyle A $ es también cerrado relativo en $\displaystyle \Q $, por lo que $\displaystyle \Q \subset \R $ es desconexo. 
\item[(b)] Sea $\displaystyle C \subset\Q $ conexo. Nos preguntamos, cuántos elementos puede contener $\displaystyle C $? 
\end{description}
\end{eg}
\begin{lema}[Lema del pivote]
	Sea $\displaystyle \left(X,d\right) $ un espacio métrico y sea $\displaystyle \left\{ M_{i}\right\} _{i\in I} $ una familia de subconjuntos conexos de $\displaystyle X $. Supongamos que $\displaystyle \bigcap_{i \in I}M_{i} \neq \emptyset $. Entonces, $\displaystyle \bigcup_{i \in I}M_{i} $ es conexo.
\end{lema}
\begin{proof}
	Sea $\displaystyle x_{0} \in \bigcap_{i \in I}M_{i} \neq \emptyset $ y sea $\displaystyle M = \bigcup_{i \in I}M_{i}$. Sea $\displaystyle f : M \to \R $ continua con $\displaystyle f\left(M\right) \subset \left\{ 0,1\right\}  $. Supongamos que $\displaystyle f\left(x_{0}\right) = 0 $. Como $\displaystyle M_{i} $ es conexo $\displaystyle \forall i  \in I $, tenemos que $\displaystyle f|_{M_{i}} $ es constante y por tener $\displaystyle x_{0} \in \bigcap_{i \in I}M_{i} $, $\displaystyle f|_{M_{i}} \equiv 0 $.
	Luego, $\displaystyle f \equiv 0 $ en $\displaystyle M = \bigcup_{i \in I}M_{i} $.
\end{proof}
\begin{lema}[Lema de la percha]
	Sea $\displaystyle \left(X,d\right) $ un espacio métrico y sea $\displaystyle \left\{ M_{i}\right\} _{i\in I} $ una familia de subconjuntos conexos de $\displaystyle X $. Supongamos que existe $\displaystyle M_{0} \subset X $ conexo tal que $\displaystyle M_{i} \cap M_{0} \neq \emptyset $, $\displaystyle \forall i \in I $. Entonces, $\displaystyle M_{0} \cup \left(\bigcup_{i \in I}M_{i}\right) $ es conexo.
\end{lema}
\begin{proof}
	Sean $\displaystyle M = M_{0} \cup \left(\bigcup_{ i \in I}M_{i}\right) $ y $\displaystyle f : M \to \R $ continua con $\displaystyle f\left(M\right) \subset \left\{ 0,1\right\}  $. 
	\begin{itemize}
	\item Como $\displaystyle M_{0} $ es conexo, tenemos que $\displaystyle f|_{M_{0}} $ es constante. Por ejemplo, supongamos que $\displaystyle f|_{M_{0}} \equiv 0 $. 
	\item Tenemos que $\displaystyle \forall i \in I $, $\displaystyle f|_{M_{i}} $ es constante y $\displaystyle M_{i} \cap M_{0} \neq \emptyset $, por lo que $\displaystyle f|_{M_{i}} \equiv 0 $.
	\end{itemize}
\end{proof}
\begin{observation}
La unión de dos conjuntos no tiene por qué se conexa. En efecto, podemos considerar en $\displaystyle \R $ el conjunto $\displaystyle M = \left(0,1\right) \cup \left(3,4\right) $, que claramente no es conexo.
\end{observation}
\begin{theorem}
Sean $\displaystyle \left(X,d\right) $ un espacio métrico y $\displaystyle M \subset X $. Si $\displaystyle M $ es conexo y $\displaystyle M \subset A \subset \overline{M} $, entonces $\displaystyle A $ es conexo. En particular, si $\displaystyle M $ es conexo, $\displaystyle \overline{M} $ también lo es.
\end{theorem}
\begin{proof}
	Sea $\displaystyle f:A \to \R $ continua con $\displaystyle f\left(A\right) \subset \left\{ 0,1\right\}  $. Tenemos que $\displaystyle f|_{M} $ es constante y ponemos que $\displaystyle f|_{M}\equiv 0 $. Sea $\displaystyle x \in A \subset \overline{M} $. Tenemos que existe $\displaystyle \left\{ x_{n}\right\} _{n\in\N}\subset M $ tal que $\displaystyle x_{n} \to x_{0} $.
Por ser $\displaystyle f $ continua, tenemos que $\displaystyle f\left(x_{n}\right)\to f\left(x\right) = 0 $, puesto que $\displaystyle f\left(x_{n}\right) = 0 $, $\displaystyle \forall n \in \N $. Así, hemos demostrado que $\displaystyle f|_{A} $ es constante.
\end{proof}
\begin{definition}[Camino]
	Un \textbf{camino} en un espacio métrico $\displaystyle \left(X,d\right) $ es una función continua $\displaystyle \sigma : [a,b] \to X $. 
\end{definition}
\begin{eg}
	En $\displaystyle \R^{n} $, dados $\displaystyle x,y \in \R^{n} $, tenemos que el segmento que los une $\displaystyle \sigma : [0,1] \to \R^{n} : t \to \left(1-t\right)x +ty $ es un camino. 
\end{eg}
\begin{definition}[Conexión por caminos]
	Sean $\displaystyle \left(X,d\right) $ un espacio métrico y $\displaystyle M \subset X $. Se dice que $\displaystyle M $ es \textbf{conexo por caminos} si $\displaystyle \forall x,y \in M $ existe $\displaystyle \sigma : [a,b] \to X $ camino tal que $\displaystyle \Imagen\left(\sigma \right) \subset M $, $\displaystyle \sigma\left(a\right) = x $ y $\displaystyle \sigma\left(b\right) = y $ \footnote{Es decir, cada par de puntos en $\displaystyle M $ se puede unir mediante un camino en $\displaystyle M $.}.
\end{definition}
\begin{definition}[Convexidad]
	Un conjunto $\displaystyle C \subset \R^{n} $ es \textbf{convexo} si $\displaystyle \forall x,y \in C $ el segmento $\displaystyle \sigma\left(t\right) = \left(1-t\right)x + ty $, $\displaystyle t \in [0,1] $, está contenido en $\displaystyle C $.
\end{definition}
\begin{observation}
Está claro que si un conjunto es convexo, es conexo por caminos. Puede darse que sea conexo por caminos pero que no sea convexo.
\end{observation}
\begin{theorem}
Sea $\displaystyle \left(X,d\right) $ un espacio métrico. Si $\displaystyle M \subset X $ es conexo por caminos, entonces $\displaystyle M $ es conexo.
\end{theorem}
\begin{proof}
	Sea $\displaystyle M $ conexo por caminos y sea $\displaystyle f : M \to \R $ continua con $\displaystyle f\left(M\right) \subset \left\{ 0,1\right\}  $. Dados $\displaystyle x,y \in M $, existe $\displaystyle \sigma : [a,b] \to M $ camino. Tenemos que $\displaystyle f\circ \sigma  $ es continua y $\displaystyle f\circ \sigma\left( \left[a,b\right] \right) \subset \left\{ 0,1\right\}  $. Por ser $\displaystyle [a,b] $ conexo, tenemos que $\displaystyle f\circ \sigma  $ es constante en $\displaystyle \left[a,b\right]  $, es decir, $\displaystyle f\left(\sigma\left(a\right)\right) = f\left(\sigma\left(b\right)\right) $ y $\displaystyle f\left(x\right) = f\left(y\right) $, por lo que $\displaystyle f|_{M} $ es constante.
\end{proof}
\begin{theorem}
	Sea $\displaystyle U \subset \R^{n} $ abierto y conexo. Entonces, $\displaystyle U $ es conexo por caminos.
\end{theorem}
\begin{proof}
	Fijamos $\displaystyle x_{0} \in U $. Dado $\displaystyle x \in U $ veamos que existe un camino de $\displaystyle x $ a $\displaystyle x_{0} $. Consideremos  
	\[ A = \left\{ x \in U \; : \; \exists \; \text{camino en $\displaystyle U $ de $\displaystyle x_{0} $ a $\displaystyle x $ }\right\} .\]
	\begin{itemize}
	\item Está claro que $\displaystyle A \neq \emptyset $, puesto que $\displaystyle x_{0} \in A $. 
	\item Veamos que $\displaystyle A $ es abierto relativo en $\displaystyle U $. Si $\displaystyle x \in A \subset U $, tenemos que existe $\displaystyle r > 0 $ tal que $\displaystyle B\left(x,r\right) \subset U $. Tenemos que $\displaystyle \forall y \in B\left(x,r\right) $ podemos conectar $\displaystyle x $ con $\displaystyle y $ mediante un segmento, por lo que existe un camino en $\displaystyle U $ de $\displaystyle x_{0} $ a $\displaystyle x $ y de $\displaystyle x $ a $\displaystyle y $, luego existe uno de $\displaystyle x_{0} $ a $\displaystyle y $. Así, $\displaystyle B\left(x,r\right) \subset A $. 
	\item Veamos que $\displaystyle A $ es cerrado relativo en $\displaystyle U $. Para ello, veamos que $\displaystyle U/A $ es abierto. Dado $\displaystyle x \in U/A $ existe $\displaystyle r>0 $ tal que $\displaystyle B\left(x,r\right) \subset U $. Tenemos que $\displaystyle \forall y \in B\left(x,r\right) $, no existe camino en $\displaystyle U $ de $\displaystyle x_{0} $ a $\displaystyle y $. 
		Si existiera, existiría un camino en $\displaystyle U $ de $\displaystyle x_{0} $ a $\displaystyle x $, que es una contradicción. Por tanto, $\displaystyle B\left(x,r\right) \subset U / A $, por lo que $\displaystyle A $ es cerrado relativo en $\displaystyle U $.
	\item Por ser $\displaystyle U $ conexo y $\displaystyle A \neq \emptyset $, debe ser que $\displaystyle A = U $.
	\end{itemize}
\end{proof}
\begin{eg}
	En general, conexo no implica conexo por caminos. En efecto, podemos considerar $\displaystyle M = \left\{ \left(0,0\right)\right\} \cup \left\{ \left(x,\sin \frac{1}{x}\right) \; : \; 0 < x \leq 1\right\} \subset \R^{2} $. Si $\displaystyle f : (0,1] \to \R $ con $\displaystyle f\left(x\right) = \left(x,\sin \frac{1}{x}\right) $, tenemos que
	\begin{itemize}
		\item $\displaystyle f $ es continua en $\displaystyle (0,1] $.
		\item $\displaystyle (0,1] $ es convexo.
		\item Si $\displaystyle G_{f}\left(x\right) $ es la segunda parte de la unión, entonces $\displaystyle G_{f} = f\left((0,1]\right) $ es convexo.
		\item $\displaystyle \left(0,0\right) \in \overline{G}_{f} $.
		\item $\displaystyle \left(0,0\right)\cup G_{f} = M $ es conexo.
	\end{itemize}
	Sin embargo, gráficamente se puede ver que no podemos conectar los puntos $\displaystyle \left(1,\sin 1\right) $ y $\displaystyle \left(0,0\right) $ por un camino contenido en $\displaystyle M $.
\end{eg}
\section{Continuidad uniforme}
\begin{definition}[Continuidad uniforme]
Sean $\displaystyle f : \left(X, d _{X}\right) \to \left(Y, d _{Y}\right) $ y $\displaystyle M \subset X $. Se dice que $\displaystyle f $ es \textbf{uniformemente continua} en $\displaystyle M $ si $\displaystyle \forall \epsilon > 0 $, $\displaystyle \exists \delta > 0 $ tal que si $\displaystyle x, x' \in M $ con $\displaystyle d _{X}\left(x,x'\right) < \delta  $, entonces $\displaystyle d _{Y}\left(f\left(x\right), f\left(x'\right)\right) < \epsilon  $. 
\end{definition}
\begin{observation}
\begin{itemize}
\item $\displaystyle f : X \to Y $ es continua en $\displaystyle X $ si y solo si $\displaystyle \forall x \in X $, $\displaystyle \forall \epsilon > 0 $, $\displaystyle \exists \delta > 0 $ tal que $\displaystyle d _{X}\left(x,x'\right) < \delta  $ implica que $\displaystyle d _{Y}\left(f\left(x\right), f\left(x'\right)\right) < \epsilon  $. 
\item $\displaystyle f : X \to Y $ es uniformemente continua en $\displaystyle X $ si y solo si $\displaystyle \forall \epsilon > 0 $, $\displaystyle \exists \delta > 0 $ tal que si $\displaystyle x,x' \in X $ con $\displaystyle d _{X}\left(x,x'\right) < \delta  $ entonces $\displaystyle d _{Y}\left(f\left(x\right), f\left(x'\right)\right) < \epsilon  $. 
\end{itemize}
\end{observation}
\begin{observation}
Si $\displaystyle f $ es uniformemente continua en $\displaystyle M $, entonces $\displaystyle f|_{M} $ es continua en $\displaystyle M $. 
\end{observation}
\begin{theorem}
Sean $\displaystyle f : \left(X, d _{X}\right) \to \left(Y, d _{Y}\right) $ y $\displaystyle M \subset X $. Son equivalentes:
\begin{enumerate}
\item $\displaystyle f $ es uniformemente continua en $\displaystyle M $.
\item $\displaystyle \forall \left\{ x_{n}\right\} _{n\in\N}, \left\{ y_{n}\right\} _{n\in\N} \subset M $ con $\displaystyle d _{Y}\left(x_{n}, y_{n}\right) \to 0 $, entonces $\displaystyle d _{Y} \left(f\left(x_{n}\right), f\left(y_{n}\right)\right) \to 0 $.
\end{enumerate}
\end{theorem}
\begin{proof}
\begin{description}
	\item[(1) $\displaystyle \Rightarrow $ (2)] Sean $\displaystyle \left\{ x_{n}\right\} _{n\in\N}, \left\{ y_{n}\right\} _{n\in\N} \subset M $ con $\displaystyle d _{X}\left(x_{n}, y_{n}\right) \to 0 $. Dado $\displaystyle \epsilon > 0 $, por la continuidad uniforme existe $\displaystyle \delta > 0 $ tal que si $\displaystyle d _{X}\left(x,x'\right) < \delta  $ entonces $\displaystyle d _{Y}\left(f\left(x\right), f\left(x'\right)\right) < \epsilon  $. Existe $\displaystyle n_{0} \in \N $ tal que $\displaystyle \forall n \geq n_{0} $ se tiene que $\displaystyle d _{X}\left(x_{n}, y_{n}\right) < \delta  $, por lo que $\displaystyle d _{Y}\left(f\left(x_{n}\right), f\left(y_{n}\right)\right) < \epsilon  $. 
	\item[(2) $\displaystyle \Rightarrow $ (1)] Supongamos que $\displaystyle f $ no es uniformemente continua en $\displaystyle M $. Así, existe $\displaystyle \epsilon > 0 $ tal que $\displaystyle \forall \delta > 0 $ existen $\displaystyle x_{\delta }, x'_{\delta } \in M $ con $\displaystyle d _{X}\left(x_{\delta }, x'_{\delta }\right) < \delta  $ pero $\displaystyle d _{Y}\left(f\left(x_{\delta }\right), f\left(x'_{\delta }\right)\right) \geq \epsilon $. En concreto, si tomamos $\displaystyle \delta = \frac{1}{n} $ tenemos que existen $\displaystyle x_{n}, y_{n} \in M $ tal que 
		\[ d _{X}\left(x_{n}, y_{n}\right) < \frac{1}{n}, \quad d _{Y}\left(f\left(x_{n}\right), f\left(y_{n}\right)\right) \geq \epsilon  .\]
		Así, tenemos dos sucesiones $\displaystyle \left\{ x_{n}\right\} _{n\in\N}, \left\{ y_{n}\right\} _{n\in\N} \subset M $ tal que $\displaystyle d _{X}\left(x_{n}, y_{n}\right) \to 0 $ pero $\displaystyle d _{Y}\left(f\left(x_{n}\right), f\left(y_{n}\right)\right) \not \to 0 $.
\end{description}
\end{proof}
\begin{theorem}
Sea $\displaystyle f : \left(X, d _{X}\right) \to \left(Y, d _{Y}\right) $ y $\displaystyle K \subset X $ compacto. Si $\displaystyle f $ es continua en $\displaystyle K $, entonces $\displaystyle f $ es uniformemente continua en $\displaystyle K $.
\end{theorem}
\begin{proof}
	Sean $\displaystyle \left\{ x_{n}\right\} _{n\in\N}, \left\{ y_{n}\right\} _{n\in\N} \subset K $ tales que $\displaystyle d _{X}\left(x_{n}, y_{n}\right) \to 0 $. Como $\displaystyle K $ es compacto, existe $\displaystyle \left\{ x_{n_{k}}\right\} _{k \in \N}  $ subsucesión con $\displaystyle x_{n_{k}} \to x_{0} \in K$. Consideramos ahora $\displaystyle \left\{ y_{n_{k}}\right\}_{k \in \N}  $, que tiene una subsucesión $\displaystyle \left\{ y_{n_{k_{j}}}\right\} _{j \in \N} $ tal que $\displaystyle x_{n_{k_{j}}} \to x_{0}' \in K $. Tenemos que
	\[d _{X}\left(x_{0}, x_{0}'\right) \leq d _{X}\left(x_{0}, x_{n_{k_{j}}}\right) + d _{X}\left(x_{n_{k_{j}}}, y_{n_{k_{j}}}\right)+ d _{X}\left(y_{n_{k_{j}}}, x_{0}'\right) \to 0 .\]
	Así, tenemos que $\displaystyle x_{0} = x_{0}' $. Como $\displaystyle f $ es continua en $\displaystyle x_{0} \in X $, tenemos que 
	\[ d _{Y}\left(f\left(x_{n_{k_{j}}}\right), f\left(y_{n_{k_{j}}}\right)\right) \leq d _{Y}\left(f\left(x_{n_{k_{j}}}\right), f\left(x_{0}\right)\right) + d _{Y}\left(f\left(x_{0}\right), f\left(y_{n_{k_{j}}}\right)\right) \to 0 .\]
	Si suponemos que $\displaystyle f $ no es uniformemente continua en $\displaystyle K $, existe $\displaystyle \epsilon > 0 $ y existen $\displaystyle \left\{ x_{n}\right\} _{n\in\N}, \left\{ y_{n}\right\} _{n\in\N} \subset K $ tales que $\displaystyle d _{X}\left(x_{n}, y_{n}\right)\to 0 $ pero $\displaystyle d _{Y}\left(f\left(x_{n}\right), f\left(y_{n}\right)\right) \not \to 0 $. Esto supone una contradicción con lo que hemos visto anteriormente. 
\end{proof}
\begin{eg}
La función $\displaystyle f: \R \to \R  $ con $\displaystyle f\left(x\right) = x^{2} $ es continua pero no es uniformemente continua en $\displaystyle \R $. En efecto, si $\displaystyle \delta > 0 $ tenemos que si $\displaystyle x \to \infty $,
\[ \left|f\left(x\right)-f\left(x + \delta \right)\right| = \left|x^{2} - \left(x+ \delta \right)^{2}\right| = 2x\delta + \delta ^{2} \to \infty .\]
Otra forma de verlo es tomar $\displaystyle x_{n} = n $ e $\displaystyle y_{n} = n + \frac{1}{n} $. Tenemos que
\[ d\left(x_{n}, y_{n}\right) = \left|n + \frac{1}{n} - n\right| = \frac{1}{n} \to 0 .\]
Sin embargo, nos queda que
\[ d\left(f\left(x_{n}\right), f\left(y_{n}\right)\right) = \left|n^{2}-\left(n+\frac{1}{n}\right)^{2}\right| = 2 + \frac{1}{n^{2}} \geq 2 \not \to 0 .\]
\end{eg}
\begin{definition}[Función Lipschitz]
Se dice que $\displaystyle f : \left(X, d _{X}\right) \to \left(Y, d _{Y}\right) $ es \textbf{Lipschitz} si existe $\displaystyle K \geq 0 $ tal que 
\[d _{Y}\left(f\left(x\right), f\left(x'\right)\right) \leq K \cdot d _{X}\left(x,x'\right), \; \forall x,x' \in X .\]
\end{definition}
\begin{observation}
Es fácil ver que si $\displaystyle f $ es de Lipschitz entonces $\displaystyle f $ es uniformemente continua. 
\end{observation}
\begin{prop}
Sea $\displaystyle I \subset \R $ un intervalo y supongamos que $\displaystyle f : I \to \R $ es derivable con derivada acotada en $\displaystyle I $. Entonces, $\displaystyle f $ es Lipschitz en $\displaystyle I $ y por tanto uniformemente continua.
\end{prop}
\begin{proof}
Existe $\displaystyle K \geq 0 $ tal que $\displaystyle \left|f'\left(t\right)\right|\leq K $, $\displaystyle \forall t \in I $. Sean $\displaystyle x,y \in I $, por el Teorema del Valor Medio tenemos que existe $\displaystyle \xi \in \left(x,y\right) $ tal que 
\[f\left(x\right)-f\left(y\right) = f'\left(\xi\right)\left(x-y\right) .\]
Por tanto, tenemos que 
\[ \left|f\left(x\right)-f\left(y\right)\right| = \left|f'\left(\xi \right)\right| \left|x - y\right|\leq K \left|x-y\right|, \; \forall x,y \in I .\]
\end{proof}
\begin{eg}
Sea $\displaystyle f : [0,\infty) \to \R $ con $\displaystyle \lim_{x \to \infty}f\left(x\right) = 0 $. Tenemos que $\displaystyle f $ es uniformemente continua en $\displaystyle [0,\infty) $. En efecto, sabemos que para $\displaystyle \epsilon > 0 $, existe $\displaystyle R > 0 $ tal que si $\displaystyle x > R $ entonces $\displaystyle \left|f\left(x\right)\right| < \frac{\epsilon }{3} $.
\begin{itemize}
\item Como $\displaystyle [0,R] $ es compacto, existe $\displaystyle \delta > 0 $ tal que si $\displaystyle x,y \in [0,R] $ con $\displaystyle \left|x-y\right| < \delta  $, entonces $\displaystyle \left|f\left(x\right)-f\left(y\right)\right| < \frac{\epsilon }{3} $.
\item Si $\displaystyle x,y \geq R $ tenemos que $\displaystyle \left|f\left(x\right)-f\left(y\right)\right| < \frac{2\epsilon }{3} $. 
\item Si $\displaystyle x \leq R < y$ con $\displaystyle \left|x-y\right| < \delta  $, tenemos que $\displaystyle \left|x - R\right| < \delta  $, por lo que 
\[ \left|f\left(x\right)-f\left(y\right)\right| \leq \left|f\left(x\right) -f\left(R\right)\right| + \left|f\left(R\right)\right| + \left|f\left(y\right)\right| < \frac{\epsilon }{3} + \frac{\epsilon }{3} + \frac{\epsilon }{3} = \epsilon .\]
\end{itemize}
\end{eg}
\begin{lema}
	Sea $\displaystyle f : \left(X, d _{X}\right) \to \left(Y, d _{Y}\right) $ uniformemente continua en $\displaystyle X $. Si $\displaystyle \left\{ x_{n}\right\}_{n\in\N}\subset X $ es de Cauchy, entonces $\displaystyle \left\{ f\left(x_{n}\right)\right\} _{n\in\N} $ es de Cauchy.
\end{lema}
\begin{proof}
	Sea $\displaystyle \left\{ x_{n}\right\} _{n\in\N} \subset X $ una sucesión de Cauchy. Por ser $\displaystyle f $ uniformemente continua, si $\displaystyle \epsilon > 0 $ existe $\displaystyle \delta > 0 $ tal que si $\displaystyle d _{X}\left(x,y\right) < \delta  $, entonces $\displaystyle d _{Y}\left(f\left(x\right), f\left(y\right)\right) < \epsilon  $. Como $\displaystyle \left\{ x_{n}\right\} _{n\in\N} $ es de Cauchy, existe $\displaystyle n_{0} \in \N $ tal que $\displaystyle \forall n,m \geq n_{0} $,
	\[ d _{X}\left(x_{n}, x_{m}\right) < \delta \Rightarrow d _{Y}\left(f\left(x_{n}\right), f\left(x_{m}\right)\right) < \epsilon  .\]
\end{proof}

\begin{theorem}[Teorema de extensión]
Sean $\displaystyle \left(X,d _{X}\right) $ e $\displaystyle \left(Y, d _{Y}\right) $ espacios métricos, siendo $\displaystyle \left(Y, d _{Y}\right) $ completo. Sean $\displaystyle M \subset X $ y $\displaystyle f : M \to Y $ uniformemente continua en $\displaystyle M $. Entonces $\displaystyle f $ admite una extensión $\displaystyle \overline{f} : \overline{M} \to Y $ uniformemente continua en $\displaystyle \overline{M} $ (es decir, $\displaystyle \overline{f} $ es uniformemente continua en $\displaystyle \overline{M} $ y $\displaystyle \overline{f}|_{M} = f $).
\end{theorem}
\begin{proof}
	Sea $\displaystyle x_{0} \in \overline{M} $, tenemos que existe $\displaystyle \left\{ x_{n}\right\} _{n\in\N} \subset M $ con $\displaystyle x_{n} \to x_{0} $. Por tanto, $\displaystyle \left\{ x_{n}\right\} _{n\in\N} $ es de Cauchy. Por ser $\displaystyle f $ uniformemente continua, es tenemos que $\displaystyle f $ transforma sucesiones de Cauchy en sucesiones de Cauchy. Así, $\displaystyle \left\{ f\left(x_{n}\right)\right\} _{n\in\N} $ es de Cauchy. Por ser $\displaystyle Y $ completo, tenemos que existe $\displaystyle y_{0} = \lim_{n \to \infty}f\left(x_{n}\right) \in Y $. Definimos $\displaystyle \overline{f}\left(x_{0}\right) = y_{0} $.
	Veamos que hemos definido bien $\displaystyle \overline{f}\left(x_{0}\right) $, es decir, que si $\displaystyle \left\{ x'_{n}\right\} _{n\in\N} \subset M $ con $\displaystyle x'_{n} \to x_{0} $, entonces $\displaystyle f\left(x_{n}'\right) \to y_{0} $. En efecto
	\[d _{Y}\left(f\left(x'_{n}\right), y_{0}\right) \leq d _{Y}\left(f\left(x'_{n}\right), f\left(x_{n}\right)\right) + d _{Y}\left(f\left(x_{n}\right), y_{0}\right) \to 0 .\]
Hemos utilizado que $\displaystyle d _{Y}\left(f\left(x'_{n}\right), f\left(x_{n}\right)\right) \to 0 $ puesto que $\displaystyle d _{X}\left(x'_{n}, x_{n}\right) \to 0 $ por tender ambas sucesiones al mismo límite y ser $\displaystyle f $ uniformemente continua en $\displaystyle M $. 
Está claro que $\displaystyle \overline{f}|_{M} = f $ por definición. Veamos que $\displaystyle \overline{f} $ es uniformemente continua en $\displaystyle \overline{M} $. Hay dos posibles casos:
\begin{itemize}
\item Si $\displaystyle x,y \in M $, como $\displaystyle \overline{f}|_{M} = f $ es trivial que se cumple la condición de continuidad uniforme. 
\item Si $\displaystyle x,y \in \overline{M} $ tomamos sucesiones $\displaystyle \left\{ x_{n}\right\} _{n\in\N}, \left\{ y_{n}\right\} _{n\in\N} \subset M $ con $\displaystyle x_{n}\to x $ e $\displaystyle y_{n} \to y $. 
Así, $\displaystyle \forall n \in \N $ se tiene que
\[ d _{Y}\left(\overline{f}\left(x\right), \overline{f}\left(y\right)\right) \leq d _{Y}\left(\overline{f}\left(x\right), f\left(x_{n}\right)\right) + d _{Y}\left(f\left(x_{n}\right), f\left(y_{n}\right)\right) + d _{Y}\left(f\left(y_{n}\right), \overline{f}\left(y\right)\right)  .\]
Esta expresión tiende a 0 si $\displaystyle d _{X}\left(x,y\right) \to 0 $ por la convergencia de las sucesiones $\displaystyle \left\{ f\left(x_{n}\right)\right\} _{n\in\N}, \left\{ f\left(y_{n}\right)\right\} _{n\in\N} $ a $\displaystyle \overline{f}\left(x\right) $ y $\displaystyle \overline{f}\left(y\right) $, respectivamente, y a la continuidad uniforme de $\displaystyle \overline{f} $ en $\displaystyle M $ por ser $\displaystyle \overline{f}|_{M} = f $.
\end{itemize}
\end{proof}
\begin{eg}
	Consideremos $\displaystyle f:(0,1] \to \R $ con $\displaystyle f\left(x\right) = \sin \frac{1}{x} $. Está claro que $\displaystyle f $ es continua en $\displaystyle (0,1] $ pero no es uniformemente continua en este intervalo puesto que no admite ninguna extensión continua a $\displaystyle [0,1] $ porque no existe $\displaystyle \lim_{x \to 0^{+}}\sin \frac{1}{x} $. 
\end{eg}
\section{Teorema del punto fijo de Banach}
\begin{definition}[Función contractiva]
Se dice que $\displaystyle f : \left(X,d\right)\to \left(X,d\right) $ es \textbf{contractiva} si existe $\displaystyle 0 \leq \alpha < 1 $ tal que $\displaystyle d\left(f\left(x\right),f\left(y\right)\right) \leq \alpha \cdot d\left(x,y\right) $, $\displaystyle \forall x,y \in X $.
\end{definition}
\begin{theorem}[Teorema del punto fijo de Banach]
	Sea $\displaystyle \left(X,d\right) $ un espacio métrico completo. Toda aplicación contractiva $\displaystyle f:\left(X,d\right) \to \left(X,d\right) $ tiene un único punto fijo, es decir, un único $\displaystyle x \in X $ tal que $\displaystyle f\left(x\right) = x $. Además, $\displaystyle \forall x_{0} \in X $ la sucesión $\displaystyle \left\{ x_{n}\right\} _{n\in\N} \subset X $ definida por $\displaystyle x_{n+1} = f\left(x_{n}\right) $ verifica que $\displaystyle x_{n} \to x_{0} $.
\end{theorem}
\begin{proof}
	Sea $\displaystyle x_{0} \in X $ y $\displaystyle \forall n \geq 1 $ definimos $\displaystyle x_{n+1} = f\left(x_{n}\right) $. Consideramos pues la sucesión $\displaystyle \left\{ f\left(x_{0}\right), f^{2}\left(x_{0}\right), \ldots, f^{n}\left(x_{0}\right), \ldots\right\}  $. \\
 Veamos que la sucesión $\displaystyle \left\{ x_{n}\right\} _{n\in\N} $ es de Cauchy en $\displaystyle \left(X,d\right) $. Tenemos que 
			\[d\left(x_{n}, x_{n+1}\right) = d\left(f\left(x_{n-1}\right), f\left(x_{n}\right)\right) \leq \alpha \cdot d\left(x_{n-1}, x_{n}\right) \leq \cdots \leq \alpha^{n} \cdot d\left(x_{0}, x_{1}\right) .\]
		Sea $\displaystyle m > n $, haciendo un cálculo similar tenemos que
		\[
		\begin{split}
			d\left(x_{m}, x_{n}\right) \leq & d\left(x_{m}, x_{m - 1}\right) + \cdots + d\left(x_{n+1}, x_{n}\right) \\
			\leq & \alpha ^{m - 1} d\left(x_{1}, x_{0}\right) + \alpha^{m - 2}d\left(x_{1}, x_{0}\right) + \cdots + \alpha^{n}d\left(x_{1}, x_{0}\right) \\
			= & \alpha^{n}\left(1 + \alpha + \cdots + \alpha^{m - 1 - n}\right)d\left(x_{1}, x_{0}\right) \leq \alpha^{n}\left(\sum^{\infty}_{j = 0}\alpha^{j}\right)d\left(x_{1}, x_{0}\right)\\
			= & \frac{\alpha^{n}}{1-\alpha }d\left(x_{1}, x_{0}\right) \to 0.
		\end{split}
		\]
		Así, está claro que la sucesión $\displaystyle \left\{ x_{n}\right\} _{n\in\N} $ es de Cauchy. 
Como $\displaystyle \left(X,d\right) $ es completo, existe $\displaystyle x = \lim_{n \to \infty}x_{n} \in X $. Como $\displaystyle f $ es continua, tenemos que  
		\[ f\left(x\right) = \lim_{n \to \infty}f\left(x_{n}\right) = \lim_{n \to \infty}x_{n+1} = \lim_{n \to \infty}x_{n} = x.\]
Veamos que el punto fijo que hemos calculado es único. Sea $\displaystyle y \in X $ con $\displaystyle y \neq x $, por lo que $\displaystyle d\left(y,x\right) > 0 $, y $\displaystyle f\left(y\right) = y $. Tenemos que
		\[d\left(x,y\right) = d\left(f\left(x\right), f\left(y\right)\right) \leq \alpha d\left(x,y\right) < d\left(x,y\right) .\]
		Esto es una contradicción, por lo que debe ser que $\displaystyle x =y $.
\end{proof}

