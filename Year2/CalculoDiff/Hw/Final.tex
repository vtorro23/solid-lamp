\documentclass{article}

% packages

\usepackage{graphicx} % Required for images
\usepackage[spanish]{babel}
\usepackage{mdframed}
\usepackage{amsthm}
\usepackage{amssymb}
\usepackage{fancyhdr}
\usepackage{amsmath}
\usepackage{geometry}[margin=1in]
\usepackage{pgfplots}
\usepackage{url}
\usepackage{float}

% for math environments

\theoremstyle{definition}
\newtheorem*{theorem}{Teorema}
\newtheorem*{definition}{Definición}
\newtheorem*{prop}{Proposición}
\newtheorem*{observation}{Observación}
\newtheorem{ej}{Ejercicio}
\newtheorem{sol}{Solución}

% for headers and footers

\pagestyle{fancy}

%\fancyhead[R]{Victoria Eugenia Torroja}
% Store the title in a custom command
\newcommand{\mytitle}{}

% Redefine \title to store the title in \mytitle
\let\oldtitle\title
\renewcommand{\title}[1]{\oldtitle{#1}\renewcommand{\mytitle}{#1}}

% Set the center header to the title
\lhead{\mytitle}

% Custom commands

\newcommand{\R}{\mathbb{R}}
\newcommand{\C}{\mathbb{C}}
\newcommand{\F}{\mathbb{F}}
\newcommand{\N}{\mathbb{N}}
\newcommand{\Q}{\mathbb{Q}}
\newcommand{\Z}{\mathbb{Z}}
\newcommand{\K}{\mathbb{K}}
\newcommand{\mcd}{\text{mcd}}
\newcommand{\mcm}{\text{mcm}}
\DeclareMathOperator{\Ker}{Ker}
\DeclareMathOperator{\Imagen}{Im}
\DeclareMathOperator{\ord}{ord}
\DeclareMathOperator{\GL}{GL}
\DeclareMathOperator{\Biy}{Biy}


\begin{document}

\title{Cálculo Diferencial - Enero 2026}
%\author{Victoria Eugenia Torroja Rubio}
\date{1/9/2026}

\maketitle

\subsection*{Ejercicio 1}
\subsection*{Ejercicio 2}
\subsection*{Ejercicio 3}
Sean $\displaystyle C \subset \R^{n} $ compacto y $\displaystyle D\subset\R^{n} $ cerrado, tales que $\displaystyle C \cap D = \emptyset $. 
\begin{description}
\item[(a)] Demostrar que existen puntos $\displaystyle p \in C,\; q \in D $ tales que 
	\[\inf \left\{ \|x-y\| \; : \; x \in C; y \in D\right\}  = \|p - q\| > 0 .\]
\item[(b)] Sean ahora $\displaystyle \gamma : \left(a,b\right) \to \R^{n} $ y $\displaystyle \sigma : \left(c,d\right) \to \R^{n} $ curvas diferenciables, tales que $\displaystyle \Imagen\left(\gamma \right) \subset C $ con $\displaystyle \gamma\left(t_{0}\right) = p $ y $\displaystyle \Imagen\left(\sigma \right) \subset D $ con $\displaystyle \sigma\left(s_{0}\right) = q $. Demostrar que los vectores tangentes $\displaystyle \gamma'\left(t_{0}\right) $ y $\displaystyle \sigma'\left(s_{0}\right) $ son ambos ortogonales al vector $\displaystyle \overrightarrow{pq} $.  
\end{description}
\subsection*{Ejercicio 4}
\subsection*{Ejercicio 5}

\end{document}
