\chapter{Estructuras sobre las partes de un conjunto}
Sea $\displaystyle \Omega  $ un cojunto fijo que en lo sucesivo se denominará \textbf{espacio total}. Consideramos el conjunto $\displaystyle \mathcal{P}\left(\Omega \right) $ de las partes de $\displaystyle \Omega  $. \\ 
Recibirá el nombre de sucesión de conjuntos toda aplicación de $\displaystyle \N $ en $\displaystyle \mathcal{P}\left(\Omega \right) $ y se representará por $\displaystyle \left\{ A_{n}\right\} _{n\in\N}\subset \mathcal{P}\left(\Omega \right) $.
\begin{definition}[Límite inferior]
	Se denomina \textbf{límite inferior} de la sucesión $\displaystyle \left\{ A_{n}\right\} _{n\in\N} $ y se representa por $\displaystyle \lim \inf A_{n} $ al conjunto de los puntos de $\displaystyle \Omega  $ que pertenecen a todos los $\displaystyle A_{n} $, excepto a lo sumo a un número finito de ellos.
\end{definition}
Podemos apreciar que también se puede definir el límite inferior como el conjunto de puntos de $\displaystyle \Omega  $ cuyos elementos pertenecen a todos los $\displaystyle A_{n} $ desde un $\displaystyle n $ en adelante. 
\begin{definition}[Límite superior]
	Se denomina \textbf{límite superior} de la sucesión $\displaystyle \left\{ A_{n}\right\} _{n\in\N} $ y se representa por $\displaystyle \lim \sup A_{n} $, al conjunto de los puntos de $\displaystyle \Omega  $ que pertenecen a infinitos $\displaystyle A_{n} $.
\end{definition}
\begin{observation}
	Podemos observar que las definiciones anteriores caracterizan a dos conjuntos que pueden ser distintos. En efecto, si $\displaystyle x \in \left\{ A_{2n}\right\} _{n\in\N} $, tenemos que $\displaystyle x \in \lim \sup A_{n} $ pero $\displaystyle x \not\in \lim \inf A_{n} $.
\end{observation}
\begin{prop}
	Sea $\displaystyle \left\{ A_{n}\right\} _{n\in\N} $ una sucesión de conjuntos. Se cumple:
	\[\lim \inf A_{n} = \bigcup_{k = 1}^{\infty}\bigcap_{n = k}^{\infty}A_{n}, \quad \lim \sup A_{n} = \bigcap_{k = 1}^{\infty}\bigcup_{n = k}^{\infty}A_{n} .\]
\end{prop}
\begin{proof}
\begin{description}
\item[(i)] Demostramos la primera igualdad. En primer lugar, si $\displaystyle x \in \lim \inf A_{n} $ tenemos que existe $\displaystyle N \in \N $ tal que $\displaystyle \forall n \geq N $, $\displaystyle x \in A_{n} $, así, $\displaystyle x \in \bigcap_{n = N}^{\infty}A_{n} $. De esta forma, $\displaystyle x \in \bigcup_{k = 1}^{\infty}A_{n} $. Por tanto, $\displaystyle \lim \inf A_{n} \subset \bigcup_{k = 1}^{\infty}\bigcap_{n = k}^{\infty}A_{n} $. Recíprocamente, si $\displaystyle x \in \bigcup_{k = 1}^{\infty}\bigcap_{n = k}^{\infty}A_{n} $ tenemos que existe un $\displaystyle N \in \N $ tal que $\displaystyle \forall n \geq N $, $\displaystyle x \in \bigcap_{n = N}^{\infty}A_{n} $.
	Es decir, tenemos que a partir de un número $\displaystyle N $, $\displaystyle x $ pertenece a todos los $\displaystyle A_{n} $, por lo que $\displaystyle x \in \lim \inf A_{n} $. De esta manera $\displaystyle \bigcup_{k = 1}^{\infty}\bigcap_{n = k}^{\infty}A_{n} \subset \lim \inf A_{n} $ y queda demostrada la igualdad.
\item[(ii)] Si $\displaystyle x \in \lim \sup A_{n} $, tenemos que $\displaystyle x $ pertenece a infinitos $\displaystyle A_{n} $, por lo que para cualquier $\displaystyle n_{0} \in \N $, existe $\displaystyle n \geq n_{0} $ tal que $\displaystyle x \in A_{n} $.
	Así, tenemos que $\displaystyle \forall k \in \N $, $\displaystyle x \in \bigcup_{n = k}A_{n} $, es decir, $\displaystyle x \in \bigcap_{k = 1}^{\infty}\bigcup_{n = k}^{\infty}A_{n} $. Así, queda demostrado que $\displaystyle \lim \sup A_{n} \subset \bigcap_{k = 1}^{\infty}\bigcup_{n = k}^{\infty}A_{n} $. 
	Recíprocamente, si $\displaystyle x \in \bigcap_{k = 1}^{\infty}\bigcup_{n = k}^{\infty}A_{n} $, tenemos que $\displaystyle \forall k \in \N $, $\displaystyle x \in \bigcup_{n = k}^{\infty}A_{n} $, por lo que debe ser que $\displaystyle x $ está en infinitos $\displaystyle A_{n} $, es decir, $\displaystyle x \in \lim \sup A_{n} $. Así, $\displaystyle \bigcap_{k = 1}^{\infty}\bigcup_{n = k}^{\infty}A_{n} \subset \lim \sup A_{n} $ y queda demostrada la igualdad. 
\end{description}
\end{proof}
\begin{prop}
	Para toda sucesión $\displaystyle \left\{ A_{n}\right\} _{n\in\N} \subset \mathcal{P}\left(\Omega \right) $ se verifica que $\displaystyle \lim \inf A_{n} \subset \lim \sup A_{n} $.
\end{prop}
\begin{proof}
Si $\displaystyle x \in \lim \inf A_{n} $, tenemos que $\displaystyle x $ pertenece a todos los $\displaystyle A_{n} $ sin, como mucho, un número finito de ellos. Por tanto, es trivial que $\displaystyle x \in \lim \sup A_{n} $, puesto que pertenece a infinitos $\displaystyle A_{n} $.
\end{proof}
\begin{definition}[Sucesión convergente]
	Se dice que una sucesión $\displaystyle \left\{ A_{n}\right\} _{n\in\N} \subset \mathcal{P}\left(\Omega \right) $ es \textbf{convergente} si $\displaystyle \lim \inf A_{n} = \lim \sup A_{n} $, y en este caso el límite de la sucesión es 
	\[\lim_{n \to \infty}A_{n} = \lim \inf A_{n} = \lim \sup A_{n} .\]
\end{definition}
\begin{definition}[Sucesión monótona]
	La sucesión $\displaystyle \left\{ A_{n}\right\} _{n\in\N} $ es \textbf{monótona creciente} o \textbf{expansiva} si $\displaystyle \forall n \in\N $ se tiene que $\displaystyle A_{n} \subset A_{n+1} $. \\
	Similarmente, la sucesión $\displaystyle \left\{ A_{n}\right\} _{n\in\N} $ es \textbf{monótona decreciente} o \textbf{contractiva} si $\displaystyle \forall n \in \N $ se tiene que $\displaystyle A_{n} \supset A_{n+1} $.
\end{definition}
\begin{prop}
Toda sucesión monótona creciente o decreciente tiene límite.
\end{prop}
\begin{proof} 
	Como hemos visto antes, dado que $\displaystyle \lim \inf A_{n} \subset \lim \sup A_{n} $ para cualquier $\displaystyle \left\{ A_{n}\right\} _{n\in\N} $, basta con demostrar que el límite superior es subconjunto del límite inferior.
\begin{description}
	\item[(i)] Supongamos primero que la sucesión $\displaystyle \left\{ A_{n}\right\} _{n\in\N} \subset \mathcal{P}\left(\Omega \right) $ es decreciente. Así, tenemos que $\displaystyle \forall n \in \N $, $\displaystyle A_{n} \subset A_{n-1} $. 
		De esta manera, si $\displaystyle x \in \lim \sup A_{n} $, tenemos que existe una subsucesión $\displaystyle \left\{ A_{n_{j}}\right\} _{j\in\N} $ tal que $\displaystyle x \in A_{n_{j}} $, $\displaystyle \forall j \in \N $. Así, para cualquier $\displaystyle k \in \N $, podemos coger $\displaystyle n_{j} \in \N $ suficientemente grande tal que $\displaystyle n_{j} \geq k $. Así, tenemos que, por ser la sucesión decreciente, 
		\[A_{n_{j}} \subset A_{n_{j}-1} \subset \cdots \subset A_{k} \subset \cdots \subset A_{1} .\]
	De esta manera, está claro que $\displaystyle x \in A_{k} $. Así, hemos demostrado que $\displaystyle x $ pertenece a todos los $\displaystyle A_{k} $, por lo que $\displaystyle x \in \lim \inf A_{n} $ y la sucesión converge.
\item[(ii)] Supongamos que la sucesión $\displaystyle \left\{ A_{n}\right\} _{n\in\N} $ es creciente, es decir, $\displaystyle \forall n \in \N $ se tiene que $\displaystyle A_{n} \subset A_{n+1} $. Entonces, si $\displaystyle x \in \lim \sup A_{n} $ tenemos que existe una subsucesión $\displaystyle \left\{ A_{n_{j}}\right\} _{j\in\N} $ tal que $\displaystyle x $ pertenece a todos los $\displaystyle A_{n_{j}} $.
	Así, para $\displaystyle n_{1} $ tenemos que 
	\[x \in A_{n_{1}} \subset A_{n_{1}+1} \subset \cdots .\]
	De esta manera, $\displaystyle x $ pertenece a todos los $\displaystyle A_{n} $ salvo excepto a un número finito de ellos. Por tanto, $\displaystyle x \in \lim \inf A_{n} $ y $\displaystyle \lim \sup A_{n} \subset \lim \inf A_{n} $, por lo que la sucesión converge.
\end{description}
\end{proof}
\begin{definition}[Semianillo]
Dado el espacio total $\displaystyle \Omega  $, una clase $\displaystyle \mathcal{C} \subset \mathcal{P}\left(\Omega \right) $ tiene una estructura de \textbf{semianillo} si 
\begin{description}
\item[(a)] $\displaystyle \emptyset \in \mathcal{C} $.
\item[(b)] $\displaystyle \forall A,B \in \mathcal{C} $, $\displaystyle A \cap B \in \mathcal{C} $.
\item[(c)] $\displaystyle \forall A, B \in \mathcal{C} $ existe una sucesión finita $\displaystyle C_{1}, \ldots, C_{n} \in \mathcal{C} $ con $\displaystyle C_{i} \cap C_{j} = \emptyset $, $\displaystyle \forall i \neq j $ tal que $\displaystyle A - B = \bigcup_{1 \leq j \leq n}C_{j} $.
\end{description}
\end{definition}
\begin{prop}
\begin{enumerate}
\item $\displaystyle \forall B \in \mathcal{C}, \forall C_{1}, \ldots, C_{n} \in \mathcal{C} $, $\displaystyle \exists A_{1}, \ldots, A_{m} \in \mathcal{C} $ con $\displaystyle A_{i} \cap A_{j} = \emptyset $, $\displaystyle \forall i \neq j $ tales que 
	\[B - \bigcup_{i = 1}^{n}C_{i} = \bigcup_{j = 1}^{m}A_{j} .\]
\item Cualesquiera que sean $\displaystyle C_{1}, \ldots, C_{n} \in \mathcal{C} $, $\displaystyle \exists A_{1}, \ldots, A_{m} \in \mathcal{C} $ con $\displaystyle A_{i} \cap A_{j} = \emptyset $, $\displaystyle \forall i \neq j $ tales que
	\[\bigcup_{i = 1}^{n}C_{i} = \bigcup_{j = 1}^{m}A_{j} .\]
\item Cualesquiera que sean $\displaystyle C_{1}, \ldots, C_{n} \in \mathcal{C} $ se tiene que $\displaystyle C_{1} \cap C_{2} \cap \cdots \cap C_{n} \in \mathcal{C} $.
\end{enumerate}
\end{prop}
\begin{proof}
Sea $\displaystyle \mathcal{C} \subset \mathcal{P}\left(\Omega \right) $ un semianillo.
\begin{enumerate}
\item Si $\displaystyle B, C_{1}, \ldots, C_{n} \in \mathcal{C} $, tenemos que 
\[
\begin{split}
	B - \bigcup_{1 \leq i \leq n}C_{i} = & B - \left(C_{1} \cup C_{2} \cup \cdots \cup C_{n}\right) = \left(B - C_{1}\right) \cap \left(B - C_{2}\right) \cap \cdots \cap \left(B - C_{n}\right) 
\end{split}
\]
\item Aplicando que $\displaystyle \emptyset \in \mathcal{C} $ y \textbf{(c)}, tenemos que $\displaystyle \exists A_{1}, \ldots, A_{m} \in \mathcal{C} $, con $\displaystyle A_{i} \cap A_{j}  = \emptyset $, $\displaystyle i \neq j $, tales que 
	\[\bigcup_{1 \leq i \leq n} .\]
\end{enumerate}

\end{proof}

\begin{definition}[Anillo]
Dado el espacio total $\displaystyle \Omega  $, una clase $\displaystyle \mathcal{R} \subset \mathcal{P}\left(\Omega \right) $ tiene estructura de \textbf{anillo} si 
\begin{description}
\item[(a)] $\displaystyle \forall A,B \in \mathcal{R} $, $\displaystyle A \cap B \in \mathcal{R} $.
\item[(b)] $\displaystyle \forall A, B \in \mathcal{R} $, $\displaystyle A \Delta B = \left(A - B\right) \cup \left(B - A\right)$.
\end{description}
\end{definition}
\begin{prop}
\begin{enumerate}
\item $\displaystyle \forall A_{1}, \ldots, A_{n} \in \mathcal{R} $ se tiene que $\displaystyle \bigcup_{1 \leq i \leq n}A_{i} \in \mathcal{R} $.
\item $\displaystyle \forall A,B \in \mathcal{R} $ se tiene que $\displaystyle A - B \in \mathcal{R} $.
\item Todo anillo es semianillo. 
\item La intersección de anillos es un anillo.
\end{enumerate}
\end{prop}
\begin{definition}[Álgebra]
Dado el espacio total $\displaystyle \Omega  $, una clase $\displaystyle \mathcal{Q}\subset \mathcal{P}\left(\Omega \right) $ tiene una estructura de \textbf{álgebra} si
\begin{description}
\item[(a)] $\displaystyle \Omega \in \mathcal{Q} $.
\item[(b)] $\displaystyle \forall A \in \mathcal{Q} $, $\displaystyle A^{c} \in \mathcal{Q} $.
\item[(c)] $\displaystyle A,B \in \mathcal{Q} $, $\displaystyle A \cup B \in \mathcal{Q} $.
\end{description}
\end{definition}
\begin{prop}
\begin{enumerate}
\item $\displaystyle \forall A,B \in \mathcal{Q} $ se tiene que $\displaystyle A \cap B \in \mathcal{Q} $.
\item $\displaystyle \forall A, B \in \mathcal{Q} $ se tiene que $\displaystyle A - B \in \mathcal{Q} $.
\item $\displaystyle \forall A, B \in \mathcal{Q} $ se tiene que $\displaystyle A \Delta B \in \mathcal{Q} $.
\item Para cualquier sucesión finita $\displaystyle A_{1}, \ldots, A_{n} $ con $\displaystyle A_{i} \in \mathcal{Q} $, $\displaystyle i = 1, \ldots, n $, se tiene que $\displaystyle \bigcup_{1 \leq i\leq n}A_{i} \in \mathcal{Q} $.
\item Todo álgebra es un anillo.
\end{enumerate}
\end{prop}
\begin{definition}[$\displaystyle \sigma $-álgebra]
Dado el espacio total $\displaystyle \Omega  $, una clase $\displaystyle \mathcal{A} \subset \mathcal{P}\left(\Omega \right) $ tiene estructura de $\displaystyle \sigma  $\textbf{-álgebra} si 
\begin{description}
\item[(a)] $\displaystyle \Omega \subset \mathcal{A} $.
\item[(b)] $\displaystyle \forall A \in \mathcal{A} \Rightarrow A^{c} \in \mathcal{A} $.
\item[(c)] Dada cualquier sucesión $\displaystyle \left\{ A_{n}\right\} _{n\in\N} \in \mathcal{A} $ verifica $\displaystyle \bigcup_{n \in \N}A_{n}\in \mathcal{A} $.
\end{description}
\end{definition}
\begin{prop}
\begin{enumerate}
\item $\displaystyle \emptyset \in \mathcal{A} $.
\item Para cualquier sucesión $\displaystyle \left\{ A_{n}\right\} _{n\in\N} $ con $\displaystyle A_{n} \in \mathcal{A} $ para todo $\displaystyle n \in \N $, se tiene que $\displaystyle \bigcap_{n \in \N}A_{n} \in \mathcal{A} $.
\item Para cualquier sucesión finita $\displaystyle A_{1}, \ldots, A_{n} $ con $\displaystyle A_{i} \in \mathcal{A} $, $\displaystyle i = 1, \ldots, n $, se tiene que $\displaystyle \bigcup_{1\leq i \leq n}A_{i} \in \mathcal{A} $.
\item Para cualquier sucesión finita $\displaystyle A_{1}, \ldots, A_{n} $ con $\displaystyle A_{i} \in \mathcal{A} $, $\displaystyle i = 1, \ldots, n $, se tiene que $\displaystyle \bigcap_{1\leq i \leq n}A_{i} \in \mathcal{A} $.
\item Toda $\displaystyle \sigma  $-álgebra es un álgebra.
\item Todo álgebra con un número finito de elementos es $\displaystyle \sigma  $-algebra.
\item Toda $\displaystyle \sigma  $-álgebra es cerrada respecto de la operación paso al l´ˆmite para cualquier sucesión.
\item La intersección de $\displaystyle \sigma  $-algebras definidas sobre el mismo espacio total es una $\displaystyle \sigma  $-álgebra.
\item Dada una clase $\displaystyle \mathcal{B} \subset \mathcal{P}\left(\Omega\right)$ existe una mínima $\displaystyle \sigma  $-álgebra que la contiene. Esta será la intersección de todas las $\displaystyle \sigma  $-álgebras que contengan a $\displaystyle \mathcal{B} $. Se indicará por $\displaystyle \sigma\left(\mathcal{B}\right) $.
\end{enumerate}
\end{prop}
\begin{observation}
La unión de dos $\displaystyle \sigma  $-álgebras puede no ser $\displaystyle \sigma  $-álgebra.
\end{observation}
\begin{definition}[Clase monótona]
Dado el espacio total $\displaystyle \Omega  $, una clase $\displaystyle \mathcal{M} \subset \mathcal{P}\left(\Omega \right) $ tiene estructura de clase monótona si y solo si es cerrada bajo la operación paso al límite para sucesiones monótonas de subconjuntos de $\displaystyle \mathcal{M} $.
\end{definition}
\begin{prop}
\begin{enumerate}
\item La intersección de dos clases monótonas, del mismo espacio total es otra clase monótona.
\item La intersección de una familia arbitraria de clases monótonas es clase monótona. 
\item Dada una clase $\displaystyle \mathcal{B} \subset \mathcal{P}\left(\Omega \right) $ siempre existirá una mínima clase monótona que contenga a $\displaystyle \mathcal{B} $. Se denotará por $\displaystyle \mathcal{M}\left(\mathcal{B}\right) $. Será la intersección de todas las clases monótonas que contengan a $\displaystyle \mathcal{B} $.
\item Toda $\displaystyle \sigma  $-álgebra es clase monótona.
\item Toda clase monótona que sea álgebra, es $\displaystyle \sigma  $-álgebra.
\item $\displaystyle \mathcal{A} \subset \mathcal{P}\left(\Omega \right) $ es $\displaystyle \sigma  $-álgebra si y solo si $\displaystyle \mathcal{A} \subset \mathcal{P}\left(\Omega \right) $ es álgebra y clase monótona.
\end{enumerate}
\end{prop}
\begin{definition}
La $\displaystyle \sigma  $-álgebra engendrada por una clase $\displaystyle \mathcal{B}\subset \mathcal{P}\left(\Omega \right) $ es la $\displaystyle \sigma  $-álgebra más pequeña que contiene a $\displaystyle \mathcal{B} $ que se representa por $\displaystyle \sigma\left(\mathcal{B}\right) $.
\end{definition}
\begin{definition}[Espacio medible]
Al par $\displaystyle \left(\Omega, \mathcal{A}\right) $, donde $\displaystyle \mathcal{A} \subset \mathcal{P}\left(\Omega \right) $ es una $\displaystyle \sigma  $-álgebra se le denomina \textbf{espacio medible} o \textbf{espacio probabizable}. A los elementos de $\displaystyle \mathcal{A} $ se les llama conjuntos medibles.
\end{definition}

