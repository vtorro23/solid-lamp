\chapter{Introducción al cálculo de probabilidades}
Por experimento entendemos cualquier acción que pueda dar lugar a resultados identificables. Un experimento que da lugar siempre al mismo resultado, recibe el nombre de \textbf{experimento determinista}.
\begin{definition}[Experimento aleatorio]
Un experimento que pueda dar lugar a varios resultados, sin que sea posible anunciar con certeza cuál de éstos va a ser observado, recibe el nombre de \textbf{experimento aleatorio}. Estos tienen tres propiedades:
\begin{enumerate}
\item Los posibles resultados del experimento son conocidos previamente.
\item No se puede predecir de antemano el resultado.
\item Realizado en condiciones análogas puede dar lugar a resultados distintos en cada experiencia particular.
\end{enumerate}	
\end{definition}
\begin{eg}
Un ejemplo de experimento aleatorio es el del lanzamiento de una moneda, mientras que un ejemplo de un experimento determinista es el del fenómeno de los días y las noches.
\end{eg}
\begin{definition}[Espacio muestral]
Llamaremos \textbf{espacio muestral}, $\displaystyle \Omega  $, al conjunto de todos los posibles resultados de un experimento aleatorio.
\end{definition}
\begin{observation}
Hay que tener en cuenta que un mismo experimento puede dar lugar a distintos espacios muestrales.
\end{observation}
\begin{eg}
Consideremos como experimento aleatorio el lanzamiento de un dado.
\begin{itemize}
	\item Si el interés es el resultado numérico, tenemos que $\displaystyle \Omega = \left\{ 1, 2, 3, 4, 5, 6\right\}  $.
	\item Si el interés es que el resultado sea múltiplo de 2, tenemos que $\displaystyle \Omega = \left\{ par, impar\right\}  $.
\end{itemize}
\end{eg}
\begin{observation}
Los espacios muestrales pueden ser finitos o infinitos.
\end{observation}
\begin{definition}[Suceso]
Un \textbf{suceso} es un subconjunto del espacio muestral. Se trata de un suceso \textbf{elemental} si están formados por un único elemento, mientras que son sucesos \textbf{compuestos} aquellos que son unión de sucesos elementales. 
\end{definition}
\begin{eg}
En el lanzamiento de un dado, el suceso de que salga un 2 es un suceso elemental, mientras que el suceso de que salga par es un suceso compuesto.
\end{eg}
\section{Axiomas de la probabilidad}
\begin{definition}[Axiomática de Kolmogorov]
Sea $\displaystyle \left(\Omega, \mathcal{A}\right) $ un espacio probabilizable. Definimos una función de conjunto $\displaystyle P $ dediante una aplicación de $\displaystyle \mathcal{A} $ sobre $\displaystyle \R $ que cumple que:
\begin{description}
\item[Axioma 1.] $\displaystyle \forall A \in \mathcal{A} $, $\displaystyle P\left(A\right) \geq 0 $.
\item[Axioma 2.] $\displaystyle P\left(\Omega \right)= 1 $.
\item[Axioma 3.] $\displaystyle \forall A,B \in \mathcal{A} $ con $\displaystyle A \cap B \neq \emptyset $, $\displaystyle P\left(A \cup B\right) = P\left(A\right) + P\left(B\right) $.
\item[Axioma 3 generalizado.] Si $\displaystyle \left\{ A_{n}\right\} _{n\in \N} \subset \mathcal{A} $, con $\displaystyle A_{i} \cap A_{j} = \emptyset $ si $\displaystyle i \neq j $, entonces
	\[P\left(\bigcup_{i \in \N}A_{i}\right) = \sum^{\infty}_{i = 1}P\left(A_{i}\right) .\]
\end{description}
\end{definition}
\begin{observation}
\begin{enumerate}
\item Está claro que $\displaystyle P\left(\emptyset\right) = 0 $, puesto que $\displaystyle \emptyset \cap \emptyset = \emptyset $, podemos afirmar que
	\[P\left(\emptyset\right) = P\left(\emptyset \cup \emptyset\right) = P\left(\emptyset\right) + P\left(\emptyset\right) \iff P\left(\emptyset\right) = 0 .\]
\item Si $\displaystyle A \in \mathcal{A} $, tenemos que $\displaystyle P\left(A^{c}\right) = 1 - P\left(A\right) $. En efecto, tenemos que $\displaystyle A^{c} \cap A = \emptyset $ y $\displaystyle A^{c} \cup A = \Omega  $, por lo que
	\[P\left(A \cup A^{c} \right) = P\left(A\right) + P\left(A^{c}\right) = P\left(\Omega \right) = 1 \iff P\left(A^{c}\right) = 1 - P\left(A\right) .\]
\item Si $\displaystyle A,B \in \mathcal{A} $ con $\displaystyle A \subset B $, entonces $\displaystyle P\left(A\right) \leq P\left(B\right) $. En efecto, tenemos que $\displaystyle \left(B/A\right) \cap A = \emptyset $, por lo que 
	\[P\left(B\right) = P\left(\left(B/A\right) \cup A\right) = P\left(B/A\right) + P\left(A\right) \iff P\left(B\right)-P\left(A\right) = P\left(B/A\right) \geq 0 .\]
	Así, tenemos que $\displaystyle P\left(B\right) \geq P\left(A\right) $.
\item $\displaystyle P\left(A\right) \leq 1 $, $\displaystyle \forall A \in \mathcal{A} $. En efecto, como $\displaystyle \forall A \in \mathcal{A} $ se tiene que $\displaystyle A \subset \Omega  $, por el apartado anterior tenemos que $\displaystyle P\left(A\right) \leq P\left(\Omega \right) = 1 $.
\end{enumerate}
\end{observation}
\begin{prop}
Sea un experimento aleatorio cualquiera con espacio muestral $\displaystyle \Omega  $ y dos sucesos cualesquiera de este experimento $\displaystyle A,B \in \mathcal{A} $. Entonces, 
\[P\left(A\cup B\right) = P\left(A\right) + P\left(B\right) - P\left(A \cap B\right) .\]
\end{prop}
\begin{proof}
Tenemos que 
\[A \cup B = \left(A-B\right) \cup \left(A \cap B\right) \cup \left(B-A\right) .\]
Está claro que estos conjuntos son disjuntos dos a dos, por lo que podemos aplicar el \textbf{Axioma 3} para obtener que
\[P\left(A \cup B\right) = P\left(A-B\right) + P\left(A \cap B\right) + P\left(B - A\right) .\]
Por otra parte tenemos que 
\[A = \left(A - B \right) \cup \left(A \cap A\right) \quad \text{y} \quad B = \left(B-A\right) \cup \left(A \cap B\right) .\]
Es decir, hemos expresado los sucesos $\displaystyle A $ y $\displaystyle B $ como uniones de sucesos incompatibles, por lo que podemos usar nuevamente el \textbf{Axioma 3}:
\[P\left(A\right) = A\left(A-B\right) + P\left(A \cap B\right) \quad \text{y} \quad P\left(B\right) = P\left(B - A\right) + P\left(A \cap B\right) .\]
Por tanto, tenemos que $\displaystyle P\left(A - B\right) = P\left(A\right) - P\left(A \cap B\right) $ y $\displaystyle P\left(B -A\right) = P\left(B\right)-P\left(A\cap B\right) $. Así, es fácil obtener la igualdad:
\[
\begin{split}
	P\left(A \cup B\right) = & P\left(A\right) - P\left(A \cap B\right) + P\left(A \cap B\right) + P\left(B\right) -P\left(A \cap B\right)\\
	= & P\left(A\right) + P\left(B\right) -P\left(A \cap B\right) .
\end{split}
\]
\end{proof}
\begin{prop}
Sea un experimento aleatorio cualquiera con espacio muestral $\displaystyle \Omega  $. Sean tres sucesos cualesquiera de este experimento $\displaystyle A,B, C \in \mathcal{A} $, entonces
\[ .\]

\end{prop}

