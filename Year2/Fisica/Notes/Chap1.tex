\chapter{Cinemática (Geometría diferencial de curvas)}
La \textbf{mecánica comprende tres partes:}
\begin{itemize}
\item \textbf{Cinemática:} estudia el movimiento sin atender a las causas.
\item \textbf{Dinámica:} estudia el movimiento atendiendo a las fuezas que lo causan.
\item \textbf{Estática:} estudia las condiciones para que no se produzca movimiento.
\end{itemize}
En cinemática estudiamos el movimiento de un punto material sin dimensiones al que llamamos $\displaystyle P $. \\ \\
Utilizaremos habitualmente el \textbf{sistema de referencia cartesiano} con origen $\displaystyle O $, de tres ejes a los que llamamos $\displaystyle X $, $\displaystyle Y $ y $\displaystyle Z $, con vectores directores unitarios $\displaystyle \hat{i}, \hat{j} $ y $\displaystyle \hat{k} $, respectivamente. \\
% poner aquí figura del sistema de coordenadas
Así, podemos describir a nuestro punto $\displaystyle P $ de la forma $\displaystyle P\left(x,y,z\right) $. 
\begin{definition}[Vector de posición]
Llamamos \textbf{vector de posición} al vector $\displaystyle \vec{r} = \overrightarrow{OP} $. 
\end{definition}
% poner imagen del vector de posición
A los ángulos que forman el vector de posición con los ejes $\displaystyle X $, $\displaystyle Y $ y $\displaystyle Z $ los llamamos $\displaystyle \alpha  $, $\displaystyle \beta  $ y $\displaystyle \gamma  $, respectivamente. 
\begin{observation}
Claramente, 
\[\cos\alpha = \frac{x}{\|\overrightarrow{OP}\|} = \frac{x}{\sqrt{x^{2}+y^{2}+z^{2}}} = \frac{x}{\|\vec{r}\|}:=\frac{x}{r} .\]
Así, escribimos,
\[\cos\alpha = \frac{x}{r}, \quad \cos\beta=\frac{y}{r}, \quad \cos\gamma=\frac{z}{r} .\]
\end{observation}
Dado que la posición de nuestro objeto va a cambiar a lo largo del tiempo, podemos escribir $\displaystyle \vec{r}\left(t\right) $. 
\begin{definition}[Desplazamiento]
Decimos que el \textbf{desplazamiento} de nuestra partícula en un periodo de tiempo $\displaystyle \Delta t $ es
\[\Delta \vec{r} = \vec{r}\left(t + \Delta t\right)-\vec{r}\left(t\right) .\]
\end{definition}
\begin{definition}[Velocidad media]
	La \textbf{velocidad media} se describe como el vector
	\[\vec{v}_{\text{media}} = \frac{\Delta \vec{r}}{\Delta t} .\]
\end{definition}
\begin{definition}[Ley del movimiento]
Se define \textbf{ley del movimiento} a la aplicación $\displaystyle \vec{r} : I \subset \R \to \R^{3} $.
\end{definition}
\begin{definition}[Trayectoria]
	La \textbf{trayectoria} es $\displaystyle \Imagen\left(\vec{r}\right) = \left\{ \vec{r}\left(t\right) \; : \; t \in I\right\}  $. 
\end{definition}
\begin{definition}[Velocidad instantánea]
La \textbf{velocidad instantánea} es la velocidad que lleva la partícula en cada instante, es decir,
\[\vec{v} = \dot{\vec{r}} = \frac{d\vec{r}\left(t\right)}{dt} = \lim_{\Delta t \to 0}\frac{\Delta \vec{r}}{\Delta t} .\]
\end{definition}
En general, la ley del movimiento la podemos escribir de la forma
\[\vec{r}\left(t\right)= \left(x\left(t\right), y\left(t\right), z\left(t\right)\right) = x\left(t\right)\hat{i} + y\left(t\right)\hat{j} + z\hat{k} .\]
\begin{notation}
En muchos casos resulta desagradable escribir la $\displaystyle t $ del tiempo, por lo que podemos escribir, 
\[\vec{r}\left(t\right) = x\hat{i} +y\hat{j} + z\hat{k} .\]
\[\dot{\vec{r}}\left(t\right)= \dot{x}\hat{i} + \dot{y}\hat{j} + \dot{z}\hat{k} .\]
\end{notation}
\begin{definition}[Aceleración]
Se define la \textbf{aceleración} de la forma
\[\vec{a} = \frac{d^{2}\vec{r}\left(t\right)}{dt ^{2}} = \frac{d\vec{v}}{dt} = \ddot{\vec{r}}\left(t\right) .\]
\end{definition}
Es sencillo ver que el vector tangente a la curva será 
\[\hat{t} =\frac{\vec{v}}{\|\vec{v}\|} .\]
Sea $\displaystyle \hat{n} $ el vector normal a la curva el perpendicular al vector tangente. Tiene el mismo sentido que el que sigue la curvatura de la trayectoria. \\
Así, podemos escribir $\displaystyle \vec{v} = v\hat{t} $. Así, tenemos que 
\[\vec{a} = \underbrace{\frac{dv}{dt}\hat{t}}_{\text{aceleración tangencial}}+\underbrace{v\frac{d\hat{t}}{dt}}_{\text{aceleración normal}} .\]
Tenemos que 
\[\lim_{\Delta t \to 0}\frac{\Delta \hat{t}}{\Delta t} = \lim_{\Delta t \to 0}\frac{\hat{t}' - \hat{t}}{\Delta t}  .\]
Podemos ver que 
\[\Delta \hat{t} = \hat{t}' - \hat{t} = \Delta \phi \cdot \hat{n} .\]
Por tanto, 
\[\lim_{\Delta t \to 0}\frac{\Delta \hat{t}}{\Delta t} = \lim_{\Delta t \to 0}\frac{\hat{t}' - \hat{t}}{\Delta t} = \lim_{\Delta t \to 0}\frac{\Delta \phi}{\Delta t} \cdot \hat{n}  .\]
Tenemos que 
\[\Delta \phi = \frac{\text{arco}}{R} = \frac{v\Delta t}{R} .\]
Así, 
\[\lim_{\Delta t \to 0}\frac{\Delta \hat{t}}{\Delta t} = \lim_{\Delta t \to 0}\frac{\hat{t}' - \hat{t}}{\Delta t} = \lim_{\Delta t \to 0}\frac{\Delta \phi}{\Delta t} \cdot \hat{n} 
= \frac{v}{R}\hat{n}.\]
De esta forma, nos queda que la expresión de la aceleración normal es
\[\hat{ a} = \frac{dv}{dt}\hat{t} + \frac{v^{2}}{R}  \hat{n} .\]
Decimos que $\displaystyle R $ es el \textbf{radio de curvatura}. Para intervalos de tiempo muy pequeños la trayectoria se puede aproximar a una circunferencia, pero en general tenemos que el radio de curvatura depende del tiempo: $\displaystyle R\left(t\right) $. \\ \\
El vector normal está definido cuando hay cambios de dirección, es decir, si la trayectoria es una línea recta no hablamos de vector normal. \\ \\
Tipos de movimiento en el plano:
\begin{itemize}
\item \textbf{Movimiento rectilíneo uniforme (MRU):} son movimientos en los que $\displaystyle \vec{a} = 0 $, por lo que $\displaystyle \vec{v}\left(t\right) = \vec{v}_{0} $. Así, tenemos que 
	\[\vec{r}\left(t\right) = \int^{t}_{t_{0}} \vec{v}_{0} \; dt \Rightarrow \vec{r}\left(t\right)-\vec{r}\left(t_{0}\right) = \vec{v}_{0}\left(t -t_{0}\right) .\]
	Así, nos queda que 
	\[\boxed{\vec{r}\left(t\right) = \vec{r}\left(t_{0}\right) + \vec{v}_{0}\left(t - t_{0}\right).} \]
\item \textbf{Movimiento rectilíneo uniformemente acelerado (MRUA):} son movimientos en los que $\displaystyle \vec{a}\left(t\right) = \vec{a}_{0} $. De esta forma, tenemos que
	\[\vec{v}\left(t\right) = \int^{t}_{t_{0}} \vec{a}_{0} \; dt \Rightarrow \vec{v}\left(t\right) = \vec{v}\left(t_{0}\right) + \vec{a}_{0}\left(t - t_{0}\right) .\]
	Así, tenemos que
	\[\vec{r}\left(t\right)=\int^{t}_{t_{0}} \vec{v}\left(t\right) \; dt=\int^{t}_{t_{0}} \vec{v}\left(t_{0}\right)+\vec{a}_{0}\left(t - t_{0}\right) \; dt = \vec{v}\left(t_{0}\right)\left(t-t_{0}\right)+\frac{1}{2}\vec{a}_{0}\left(t - t_{0}\right)^{2} .\]
Así, nos queda que
\[\boxed{\vec{r}\left(t\right) = \vec{r}\left(t_{0}\right) + \vec{v}\left(t_{0}\right)\left(t - t_{0}\right)+\frac{1}{2}\vec{a}_{0}\left(t-t_{0}\right)^{2}.} \]
\item \textbf{Movimiento circular uniforme (MCU):} en este caso está claro que 
	\[\vec{r}\left(t\right) = \vec{r}_{0}+ R\cos\left(\omega t\right)\hat{i} + R\sin\left(\omega t\right)\hat{j} .\]
De esta manera, podemos calcular la velocidad y consecuentemente la aceleración,
\[\vec{v}\left(t\right) = \dot{\vec{r}}\left(t\right) = R\omega \left[-\sin\left(\omega t\right)\hat{i} + \cos\left(\omega t\right)\hat{j}\right] \Rightarrow v\left(t\right) = \|\vec{v}\left(t\right)\| = R\omega , \; \hat{t} = -\sin\left(\omega t\right)\hat{i} + \cos\left(\omega t\right)\hat{j} .\]
Análogamente,
\[\vec{a}\left(t\right) = \ddot{\vec{r}}\left(t\right) = \omega^{2}R\left[-\cos\left(\omega t\right)\hat{i} - \sin\left(\omega t\right)\hat{j}\right] \Rightarrow a\left(t\right) = \| \vec{a}\left(t\right)\| = \omega^{2}R, \; \hat{n} = -\cos\left(\omega t\right)\hat{i}-\sin\left(\omega t\right)\hat{j} .\]
Como $\displaystyle v = \omega R $, tenemos que $\displaystyle \omega = \frac{v}{R} $, por lo que 
\[\vec{a}\left(t\right) = \frac{v^{2}}{R}\hat{n} .\]
\end{itemize}
\section{Sistemas de referencia}
\subsection*{Sistemas de referencia móvil: triedro de Frenet-Serret}


