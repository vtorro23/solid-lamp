\chapter{Introducción a las ecuaciones diferenciales ordinarias}

\section{Motivación: problema directo e inverso}

\begin{itemize}
\item \textbf{Cuestión directa:} nos dan una función $\displaystyle x\left(t\right) : I \subset \R \to \R $ y nos piden que calculemos su derivada.
\begin{eg}
Consideremos $\displaystyle x\left(t\right) = e^{t ^{2}} $. Tenemos que $\displaystyle x'\left(t\right) = 2te^{t ^{2}} = 2tx\left(t\right) $. 
\end{eg}
\item \textbf{Cuestión inversa:} busco $\displaystyle x\left(t\right) : I \subset \R \to \R$ que satisfazca una ecuación diferencial. 
\begin{eg}
	Encontrar $\displaystyle x\left(t\right) $ tal que $\displaystyle x'\left(t\right) = 2tx\left(t\right) $. Una solución es $\displaystyle x\left(t\right) = e^{t ^{2}} $, como hemos visto en el ejemplo anterior. Otra solución es $\displaystyle x_{C}\left(t\right) = Ce^{t ^{2}} $ con $\displaystyle C \in \R $. En el \textbf{Tema 2} veremos que no hay más soluciones. Diremos que $\displaystyle \left\{ x_{C}\right\}  $ es la familia de soluciones de la ecuación diferencial
	\[x' = 2tx \]	
	y se denomina \textbf{familia monoparamétrica} porque depende de un único parámetro. Para conseguir una única solución podemos imponer alguna condición más, como que $\displaystyle x\left(0\right) = 1 $. Entonces, tenemos que 
	\[Ce^{0 ^{2}} = 1 \Rightarrow C = 1 .\]
	Así, la única función que verifica esto es $\displaystyle x\left(t\right) = e^{t ^{2}} $. A este par, ecuación más condición, se le llama \textbf{problema de valor inicial} o \textbf{problema de Cauchy}. 
\end{eg}
\begin{observation}
Una pregunta natural es cómo debe ser $\displaystyle x\left(t\right) $ para que el problema de Cauchy tenga solución y sea única. Esta pregunta dio lugar a los \textbf{teoremas de existencia y unicidad.} 
\end{observation}
\end{itemize}

\section{Notación y conceptos básicos}
\begin{notation}
De forma habitual usaremos la notación de Newton para las derivadas de $\displaystyle x\left(t\right) $:  
\[x'\left(t\right), x''\left(t\right), \ldots, x^{\left(n\right)}\left(t\right) .\]
Puede que eventualmente aparezca la notación de Leibniz:
\[\frac{dx}{dt}, \frac{d^{2}x}{dt ^{2}}, \ldots, \frac{d^{n}x}{dt ^{n}} .\]
A menudo escribirmos simplemente $\displaystyle x, x', \ldots, x^{\left(n\right)} $ en vez de $\displaystyle x\left(t\right), x'\left(t\right), \ldots, x^{\left(n\right)}\left(t\right) $, pues en las dos sólo hay una variable independiente.
\end{notation}
\begin{definition}[EDO]
Entendemos por \textbf{ecuación diferencial ordinaria} una relación que implica una o varias derivadas respecto de una única variable $\displaystyle t $ (\textbf{variable independiente}) de una función especificada $\displaystyle x\left(t\right) $ (\textbf{variable dependiente} o \textbf{función incógnita}), pudiendo implicar también a funciones de dichas variables $\displaystyle x $ y $\displaystyle t $. Llamamos \textbf{orden} de una EDO al valor de la derivada de mayor orden que aparece en la ecuación.
\end{definition}
\begin{observation}
Utilizamos el adjetivo \textbf{ordinarias} para diferenciarlas de las \textbf{ecuaciones en derivadas parciales}, es decir, ecuaciones diferenciales donde una o más funciones dependen de dos o más variables independientes.
\end{observation}

\begin{eg} Calculemos los órdenes de estas ecuaciones diferenciales. 
\begin{itemize}
\item La ecuación
\[x^{'''}+x'=0 ,\]
es una EDO de orden 3.
\item La ecuación
	\[\frac{\partial^{2}u}{\partial x^{2}} + \frac{\partial^{2}u}{\partial t ^{2}} = 0 ,\]
	es una EDP de orden 2.
\end{itemize}
\end{eg}
\begin{eg}[Teorema Fundamental del Cálculo]
Supongamos que $\displaystyle f\left(t\right) $ es una función continua y acotada en un cierto intervalo $\displaystyle \left(a,b\right) $ y queremos resolver la EDO $\displaystyle x'\left(t\right) = f\left(t\right) $. Por el TFC, sabemos que 
\[x\left(t\right) = \int^{t}_{a} f\left(t\right) \; dt + x\left(a\right), \; \forall t \in \left(a,b\right) .\]
Si conocemos el valor de $\displaystyle x\left(a\right) $ (problema de Cauchy), entonces podemos resolver la EDO.
\end{eg}
Aunque ya hemos definido lo que es una EDO, ahora lo hacemos de manera más formal. 
\begin{definition}[EDO]
	Una \textbf{ecuación diferencial ordinaria} es una ecuación que contiene a una función y sus derivadas con respecto a una variable. Formalmente, consideramos $\displaystyle x : I \subset \R \to \R $ (incógnita), con $\displaystyle I $ abierto, y $\displaystyle F : \tilde{\Omega} \subset \R^{n+2}\to \R $ con $\displaystyle \tilde{\Omega} $ abierto, y la EDO asociada será
	\[F\left(t, x, x', \ldots, x^{\left(n\right)}\right) = 0 .\]
\end{definition}
Asumimos que $\displaystyle F \in \mathcal{C}^{1}\left(\tilde{\Omega }\right) $ y que existen $\displaystyle \left(t_{0}, x_{0}, \ldots, x_{0}^{\left(n\right)}\right) \in \tilde{\Omega } $ de manera que 
\[F\left(t_{0}, x_{0}, \ldots, x^{\left(n\right)}_{0}\right) = 0 .\]
\[\frac{\partial F}{\partial x^{\left(n\right)}}\left(t_{0}, x_{0}, \ldots, x_{0}^{\left(n\right)}\right) \neq 0 .\]
Esto es importante porque tiene que ver con el teorema de la función implícita. La forma que aparece en la definición se llama la \textbf{forma implícita} del problema. Por otro lado, si en la EDO podemos despejar la variable $\displaystyle x^{\left(n\right)} $, escribiendo entonces
\[x^{\left(n\right)} = f\left(t, x, \ldots, x^{\left(n-1\right)}\right) ,\]
para $\displaystyle f : \Omega \subset \R^{n+1} \to \R $, diremos que la EDO está en \textbf{forma explícita}. Finalmente, a $\displaystyle n $ se le llama el \textbf{orden} de la EDO. \\
Recordamos el \textbf{Teorema de la Función Implícita:}
\begin{theorem}[Teorema de la función implícita]
Sea $\displaystyle G : U \subset \R^{n} \times \R^{m} \to \R^{m} $ con $\displaystyle U $ abierto, donde $\displaystyle G $ es de clase $\displaystyle \mathcal{C}^{k}\left(U\right) $. Si existe un punto $\displaystyle \left(x_{0}, y_{0}\right) \in U $ tal que 
\begin{itemize}
\item $\displaystyle G\left(x_{0}, y_{0}\right) = 0 $ y,
\item $\displaystyle D_{2}G\left(x_{0}, y_{0}\right) $ es inversible,
\end{itemize}
entonces existen $\displaystyle \epsilon, \delta  >0 $ y una función $\displaystyle g : B\left(x_{0}, \epsilon \right) \subset \R^{n} \to \R^{m} $, con $\displaystyle g \in \mathcal{C}^{k}\left(B\left(x_{0}, \epsilon \right)\right) $ tal que $\displaystyle g\left(B\left(x_{0}, \epsilon \right)\right)\subset B\left(y_{0}, \delta \right) $ y
\[G\left(x, g\left(x\right)\right) = 0, \; \forall x \in B\left(x_{0}, \epsilon \right). \]
Además, las únicas soluciones de $\displaystyle G\left(x,y\right) = 0 $ en $\displaystyle B\left(x_{0}, \epsilon \right)\times B\left(y_{0}, \delta \right) $ son las que cumplen la ecuación anterior. 
\end{theorem}

\subsection{Clasificación de EDOS}

Podemos clasificar una EDO atendiendo a diferentes conceptos.
\begin{itemize}
	\item Atenciendo al \underline{orden}, es decir, el orden de la derivada de mayor orden involucrada en la EDO. 
	\begin{eg}
	La EDO $\displaystyle x''' +tx' = 0 $ es una EDO de orden 3.
	\end{eg}
\item Atendiendo a la \underline{expresión dada}, pueden ser
	\begin{enumerate}
	\item \textbf{Expresión explícita} o \textbf{forma normal} 
	\item \textbf{Expresión implícita},
	\item Referidas a las EDOS de orden 1, a veces trabajamos con la \textbf{expresión diferencial}, que viene en general dada por $\displaystyle M\left(t,x\right)dx + N\left(t,x\right)dt = 0 $. 
	\end{enumerate}
	\begin{eg}
	La EDO
	\[x' = \frac{3x^{2}+t ^{2}}{x + t} ,\]
	esta en forma explícita, mientras que la forma diferencial será
	\[\left(x+t\right)dx-\left(3x^{2}+t ^{2}\right) dt = 0 .\]
	\end{eg}
	En general, dada la EDO de orden 1, $\displaystyle F\left(t,x,x'\right) = 0 $ (expresión implícita), $\displaystyle x' = f\left(t,x\right) $ (expresión implícita), tenemos que
		\[\frac{dx}{dt}=f\left(t,x\right) \iff dx - f\left(t,x\right)dt = 0 ,\]
		es la \textbf{expresión diferencial}.
	\item Otra forma de clasificarlas es por \underline{linealidad}. 
	\begin{definition}[EDO Lineal]
	Una EDO es \textbf{lineal} cuando se puede escribir de la forma
	\[L\left(x\left(t\right)\right) = b\left(t\right) ,\]
	siendo $\displaystyle L\left(x\left(t\right)\right) $ el operador lineal 
	\[L\left(x\left(t\right)\right) = \sum^{n}_{j = 0}a_{j}\left(t\right)x^{\left(j\right)}\left(t\right),\]
	donde $\displaystyle x^{\left(0\right)}\left(t\right) = x\left(t\right) $. Los \textbf{coeficientes} $\displaystyle a_{j}\left(t\right) $ son funciones que dependen sólo de $\displaystyle t $. Cuando todos los coeficientes $\displaystyle a_{j}\left(t\right) $ hablamos de una EDO lineal \textbf{con coeficientes constantes}.
	Decimos que $\displaystyle b\left(t\right) $ es el \textbf{término independiente} de la EDO lineal. Cuando $\displaystyle b\left(t\right)\equiv 0 $ decimos que la EDO lineal se denomina \textbf{lineal homogénea}. 
	\end{definition}
\begin{observation}
La clave de las EDOS lineales homogéneas (y de ahí su nombre de lineal) es que el conjunto de soluciones tiene estructura de $\displaystyle \R $-espacio vectorial. Formalmente, escribimos que si $\displaystyle L\left(x\right) = 0 $ es una EDO lineal de orden $\displaystyle n $, entonces
\begin{itemize}
\item Si $\displaystyle x_{1}\left(t\right) $ y $\displaystyle x_{2}\left(t\right) $ son soluciones, $\displaystyle x_{1}\left(t\right) + x_{2}\left(t\right) $ también lo es.
\item $\displaystyle \forall \lambda \in \R $ y $\displaystyle x\left(t\right) $ solución de la EDO, $\displaystyle \lambda x\left(t\right) $ también es solución.
\end{itemize}
\end{observation}
\begin{proof}
Tenemos que $\displaystyle L\left(x_{1}\left(t\right)\right) = L\left(x_{2}\left(t\right)\right) = 0 $. Por tanto, tenemos que 
\[L\left(x_{1}\left(t\right) + x_{2}\left(t\right)\right) = L\left(x_{1}\left(t\right)\right)+L\left(x_{2}\left(t\right)\right) = 0 .\]
Hemos aplicado que la derivada de la suma es la suma de las derivadas. Del mismo modo, si $\displaystyle x\left(t\right) $ es solución, $\displaystyle L\left(x\left(t\right)\right)\equiv 0 $ y por tanto
\[L\left(\lambda x\left(t\right)\right) = \sum^{n}_{j = 0}a_{j}\left(t\right)\left(\lambda x\left(t\right)\right)^{\left(j\right)} = \lambda\sum^{n}_{j = 0}a_{j}\left(t\right)x^{\left(j\right)}\left(t\right) = \lambda L\left(x\left(t\right)\right) = 0 .\]
\end{proof}
\begin{observation}
Una EDO es lineal si su expresión implícita es lineal, es decir, todas las derivadas están elevadas a exponente 0 o 1 y no se multiplican entre sí.
\end{observation}
\begin{eg}
Consideremos la EDO anterior:
\[x' = \frac{3x^{2}+t ^{2}}{x + t} .\]
Podemos despejar de forma que 
\[\left(x + t\right)x' - 3x^{2} = t ^{2} .\]
Está claro que no es lineal, puesto que el término $\displaystyle x $ está al cuadrado.  
Sin embargo, la EDO $\displaystyle x'' + 3tx' + x = 0 $ sí es lineal. Finalmente, 
\[x x'' + tx' = t ^{2} ,\]
no es lineal, puesto que se están multiplicando $\displaystyle x $ y $\displaystyle x'' $. 
\end{eg}
\item Atendiendo a la \underline{dependencia o no de la variable independiente}, cuando una EDO no depende explícitamente de la variable independiente se denomina \textbf{autónoma}.
\begin{definition}[EDO autónoma]
En una EDO, cuando no aparece explícitamente la variable independiente decimos que es \textbf{autónoma}. Formalmente, su expresión explícita será
\[x^{\left(n\right)} = f\left(x, x', \ldots, x^{\left(n-1\right)}\right) .\]
\end{definition}

\begin{eg}
	\begin{itemize}
	\item La EDO $\displaystyle x'' + x' x = 0 $ es \textbf{autónoma} (la $\displaystyle t $ sólo puede aparecer implícitamente con la variable dependiente).
	\item La EDO $\displaystyle x''' + x' = 0 $ es autónoma pero $\displaystyle x' = 2tx $ no es autónoma.
	\end{itemize}
\end{eg}
\begin{observation}
Veremos que de las EDOs autónomas de orden 1 es fácil sacar mucha información cualitativa con la representación.
\end{observation}
\begin{observation}
 Las únicas EDOs lineales que son autónomas son las lineales de coeficientes constantes con término independiente también constante, es decir
\[a_{n}x^{\left(n\right)} + a_{n-1}x^{\left(n-1\right)} + \cdots + a_{0}x = b .\]
\end{observation}
\end{itemize}
\subsection{Solución de una EDO}
El problema principal asociado a una EDO es encontrar sus soluciones. 
\begin{definition}[Solución de una EDO]
Dada una EDO de orden $\displaystyle n $ donde $\displaystyle t \in I \subset \R $, decimos que la función $\displaystyle x = x\left(t\right) $ definida en $\displaystyle J \subset I $ es \textbf{solución} de la EDO si
\begin{itemize}
\item Existen sus $\displaystyle n $ primeras derivadas: $\displaystyle x' $, \ldots, $\displaystyle x^{\left(n\right)} $ en $\displaystyle J $. 
\item Satisfacen la ecuación dada para todo $\displaystyle t \in J $.
\end{itemize}
\end{definition}
El proceso de obtención de las soluciones de una EDO se denomina también \textbf{integración de la ecuación} y a sus soluciones \textbf{curvas integrales}. Además, cuando nos dan una EDO de forma explícita, es decir
\[x^{\left(n\right)} = f\left(t, x, x', \ldots, x^{\left(n-1\right)}\right) ,\]
decimos que $\displaystyle f $ es el \textbf{campo} asociado a la ecuación anterior. Lo ideal es encontrar una solución explícitamente pero en muchos casos la solución vendrá dada de manera implícita.
\begin{notation}
Normalmente $\displaystyle I $ denota el intervalo de definición de la EDO y $\displaystyle J $ el intervalo donde está definida la solución. En general, buscamos el intervalo $\displaystyle J $ más grande posible. Si no se indica lo contrario, supondremos que $\displaystyle I $ y $\displaystyle J $ son abiertos, donde las derivadas de los extremos del intervalo se consideran siempre laterales.
\end{notation}
\begin{definition}
	Dadas dos soluciones $\displaystyle \left(x,J\right) $ y $\displaystyle \left(\tilde{x}, \tilde{J}\right) $ de $\displaystyle x' = f\left(t,x\right) $, se dice que $\displaystyle \left(x,J\right) $ es una \textbf{prolongación} de $\displaystyle \left(\tilde{x}, \tilde{J}\right) $ o que $\displaystyle x\left(t\right) $ \textbf{se extiende} a $\displaystyle \tilde{x}\left(t\right) $ si $\displaystyle \tilde{J} \subset J $ y $\displaystyle \tilde{x}\left(t\right) = x\left(t\right) $, $\displaystyle \forall t \in \tilde{J} $.
\end{definition}
\begin{eg}
Consideremos los siguientes ejemplos.
\begin{enumerate}
\item Dada la EDO $\displaystyle x' = x\cos t $, tenemos que la solución $\displaystyle x\left(t\right) = e^{\sin t} $ es solución para todo $\displaystyle \R $. Lo mismo sucede con la solución $\displaystyle x_{C} = Ce^{\sin t} $ para $\displaystyle C \in \R $. Sin embargo, la ecuación $\displaystyle x' + \sqrt{t} x = t ^{2} $ sólo tiene sentido para $\displaystyle t \geq 0 $. Respecto a la solución, se puede ver que $\displaystyle x\left(t\right) = t ^{\frac{3}{2}} - \frac{3}{2} $ es solución.
\item Es importante ver que la solución debe ser solución en un intervalo. Por ejemplo, $\displaystyle x' = x $ tiene como solución $\displaystyle x\left(t\right) = e^{t } $ en $\displaystyle t \in \R $. Podríamos pensar que $\displaystyle x = \sin t $ también, pero no es solución salvo para $\displaystyle t = \frac{\pi }{4} + k \pi  $, que no es un intervalo. Por tanto, esta última no es solución.
\end{enumerate}
\end{eg}
Normalmente encontrar la solución a una EDO es muy complicado. Sin embargo, ver si una función es o no es solución es muy sencillo, basta con derivar, sustituir en la ecuación y ver si obtenemos o no una identidad para algún intervalo de la variable independiente.
\begin{eg}
\begin{enumerate}
\item Consideremos $\displaystyle x'' - 2x' + x = 0 $, es sencillo comprobar que $\displaystyle x\left(t\right)=\left(1+2t\right)e^{t} $ es solución. 
\item Las soluciones a una EDO no siempre se pueden dar explícitamente. En efecto, para la EDO
	\[x' = \frac{x}{x^{2}+1} ,\]
	una solución es
	\[\ln \left|x\left(t\right)\right| + \frac{x\left(t\right)^{2}}{2} = t + C, \; C \in \R .\]
Decimos que esta es una solución \textbf{implícita} de la EDO. 	
\end{enumerate}

\end{eg}

\begin{eg}
La solución de una EDO puede ser una función definida a trozos. En efecto, consideremos la EDO, $\displaystyle x'^{2} - 9tx = 0 $. Una solución es
\[x\left(t\right) = 
\begin{cases}
0, \; t \leq 0 \\
t ^{3}, \; t > 0
\end{cases}
.\]
Se puede comprobar que $\displaystyle x\in\mathcal{C}^{1} $ y es solución de la ecuación. 
\end{eg}
Obtener la \textbf{solución general} de una EDO es hallar todas las soluciones que verifican la ecuación. En una EDO lineal de orden $\displaystyle n $ veremos que la solución general es una familia que depende de $\displaystyle n $ parámetros y se denomina \textbf{familia $\displaystyle n $-paramétrica}. 
\begin{eg}
	Para la EDO $\displaystyle x' = x $ vimos que la solución general era la familia $\displaystyle \left\{ x_{c}\right\}  $ con $\displaystyle x_{C}\left(t\right) = Ce^{t} $, $\displaystyle C \in \R $. Decimos que es una \textbf{familia monoparamétrica}, puesto que depende de un único parámetro $\displaystyle C $. 
\end{eg}
\begin{eg}
La EDO $\displaystyle x'' - x = 0 $ tiene por solución general 
\[x\left(t\right) = C_{1}e^{t} + C_{2}e^{-t}, \quad C_{1}, C_{2} \in \R ,\]
que es una \textbf{familia biparamétrica}. La base del espacio vectorial de soluciones $\displaystyle \left\{ e^{t}, e^{-t}\right\}  $, puesto que son linealmente independientes. 
\end{eg}
En el \underline{caso no lineal}, obtener una solución genera es mas complicado. En este escenario podemos, por ejemplo, tener \textbf{soluciones singulares} que son aquellas que no pertenecen a una familia de funciones dependiente de un parámetro que también es solución.
\begin{eg}
Consideremos $\displaystyle x' = t\sqrt{x} $. Una solución de la EDO es
\[x\left(t\right) = \left(\frac{t ^{2}}{4} + C\right)^{2}, C \in \R ,\]
es una familia solución de la EDO. Sin embargo, esta familia no incluye la solución trivial, $\displaystyle x\left(t\right)\equiv 0 $, por tanto decimos que esta es una solución singular.
\end{eg}
En el caso anterior, para encontrar la solución general tenemos que encontrar la familia y la solución singular. 
\begin{observation}
No confundir solución singular con solución particular. Una \textbf{solución particular} es una solución concreta de la EDO; es decir un miembro de la familia si la solución es una familia. 
\end{observation}
\begin{eg}
En el ejemplo anterior, la solución singular también es solución particular. Otro ejemplo es que $\displaystyle x\left(t\right)= e^{t} $ es una solución particular de $\displaystyle x' = x $.
\end{eg}
\begin{notation}
La \textbf{solución trivial} es la solución idénticamente nula. 
\end{notation}
\subsection{Problemas de valor inicial y contornos}
\begin{definition}[Problema de valor inicial]
Un \textbf{problema de valor inicial} o \textbf{de Cauchy} es un sistema
\[
\begin{cases}
x^{\left(n\right)} = f\left(t, x, x', \ldots, x^{\left(n-1\right)}\right), \; t\in I \\
x\left(t_{0}\right) = x_{0}, \; t_{0} \in I, \; x_{0} \in \R\\
x'\left(t_{0}\right) = x_{1} , \; x_{1} \in \R\\
\vdots \\
x^{\left(n-1\right)}\left(t_{0}\right)= x_{n-1}, \; x_{n-1} \in \R
\end{cases}
.\]
\end{definition}
\begin{eg}
\begin{enumerate}
\item La EDO $\displaystyle x' = x $ sujeta a $\displaystyle x\left(0\right) = 3 $ es un problema de valor inicial que tiene por solución única la función $\displaystyle x\left(t\right) = 3e^{t} $.
\item La EDO $\displaystyle x'' + 16x = 0 $ tiene por soluciones la familia biparamétrica $\displaystyle x\left(t\right) = C_{1}\cos4t + C_{2}\sin4t $. El problema de valor inicial
	\[
	\begin{cases}
	x'' + 16x = 0 \\
	x\left(\frac{\pi }{2}\right) = - 2 \\
	x'\left(\frac{\pi }{2}\right) = 1
	\end{cases}
	\]
tiene una única solución, que es $\displaystyle x\left(t\right) = -2\cos4t + \frac{1}{4}\sin4t $. 	
\end{enumerate}
\end{eg}
\begin{definition}[Problema de contorno]
Se define \textbf{problema de contorno} como un par formado por una EDO y unas condiciones inciales de frontera, que pueden o no implicar a las derivadas \footnote{La principal diferencia respecto de las anteriores es que las condiciones iniciales no están todas asociadas al mismo instante $\displaystyle t = t_{0} $.} . Existen dos casos especiales de problemas de contorno:
\begin{itemize}
\item \textbf{Problema de Dirichlet:} se proporciona el valor de la función en puntos diferentes. 
\item \textbf{Problema de Neumann:} se proporciona el valor de la derivada de una función en puntos diferentes. 
\end{itemize}
\end{definition}
\begin{eg}
\begin{enumerate}
\item El siguiente es un ejemplo de condición de frontera de tipo Dirichlet:
	\[
	\begin{cases}
	x'' + x = 0 \\
	x\left(0\right) = 1 \\
	x\left(\pi \right) = 0
	\end{cases}
	.\]
\item El siguiente es un ejemplo de condición de frontera de tipo Neumann:
	\[
	\begin{cases}
	x'' + 3x = 1 \\
	x'\left(0\right) = x'\left(\pi \right) = 0
	\end{cases}
	.\]
\end{enumerate}
\end{eg}
\subsection{Sistemas de ecuaciones diferenciales}
\begin{definition}[Sistema de EDOS de orden $\displaystyle 1 $ y $\displaystyle n $ ecuaciones]
El orden indica la derivada de mayor orden involucrada. Se trata de hallar $\displaystyle n $ funciones incógnita $\displaystyle x_{1}\left(t\right), \ldots, x_{n}\left(t\right) $ referidas a la misma variable independiente $\displaystyle t $ tales que
\[ \left(S\right)
\begin{cases}
x_{1}' = f_{1}\left(t, x_{1}, \ldots, x_{n}\right) \\
\vdots \\
x_{n}' = f_{n}\left(t, x_{1}, \ldots, x_{n}\right)
\end{cases}
.\]
Para resolver el sistema tendremos que encontrar las $\displaystyle x_{i}\left(t\right) $ soluciones definidas en un intervalo de definición $\displaystyle J $ común a todas ellas.
\end{definition}
Podemos reescribir este sistema de $\displaystyle n $ incógnitas como una EDO vectorial de orden 1. En efecto, podemos considerar
\[\vec{x}\left(t\right) = \begin{pmatrix} x_{1}\left(t\right) \\ \vdots \\ x_{n}\left(t\right) \end{pmatrix} \quad \text{y} \quad \vec{f}\left(t, \vec{x}\left(t\right)\right) = \begin{pmatrix} f_{1}\left(t, x_{1}, \ldots, x_{n}\right) \\ \vdots \\ f_{n}\left(t, x_{1}, \ldots, x_{n}\right) \end{pmatrix},\]
Además, una EDO de orden $\displaystyle n $ se puede reescribir como un sistema de EDOS de orden 1.En efecto, consideremos $\displaystyle x^{\left(n\right)} = f\left(t, x, x', \ldots, x^{\left(n-1\right)}\right) $. Podemos reescribir $\displaystyle x \equiv x_{1} $, $\displaystyle x' \equiv x_{2} $, \ldots, $\displaystyle x^{\left(n-1\right)} \equiv x_{n} $. Así, nos queda el sistema
\[
\begin{cases}
	x_{1}' = x_{2} \\
	x_{2}' = x_{3} \\
	\vdots \\
	x_{n-1}' = x_{n} \\
	x_{n}' = f\left(t, x_{1}, x_{2}, \ldots, x_{n}\right)
\end{cases}
.\]
\begin{observation}
Un sistema lineal de orden 1 son siempre de la forma 
\[
\begin{cases}
x_{1}' = a_{11}\left(t\right)x_{1} +a_{12}\left(t\right)x_{2} + \cdots + a_{1n}\left(t\right)x_{n} + b_{1}\left(t\right) \\
\vdots \\
x_{i}' = a_{i1}\left(t\right)x_{1} + a_{i2}\left(t\right)x_{2} + \cdots + a_{in}\left(t\right)x_{n}+ b_{i}\left(t\right) \\ 
\vdots \\
x_{n}' = a_{n1}\left(t\right)x_{1} + a_{n2}\left(t\right)x_{2} + \cdots + a_{nn}\left(t\right)x_{n} + b_{n}\left(t\right)
\end{cases}
.\]
Matricialmente, podemos escribir $\displaystyle \vec{x}'\left(t\right) = A\left(t\right)\vec{x}\left(t\right) + \vec{b}\left(t\right) $, con $\displaystyle A\left(t\right) \in \mathcal{M}_{n \times n} $. Posteriormente estudiaremos las EDOS lineales de coeficientes constantes puesto que entonces tendremos que $\displaystyle A \in \mathcal{M}_{n \times n}\left(\R\right) $. 
\end{observation}

\begin{eg}
\textbf{Hoja 1, Ejercicio 10.} Convierte $\displaystyle x'' + x\left(\frac{3}{2}x - 1\right) = 0 $ en un sistema de dos ecuaciones de primer orden. Denotamos $\displaystyle x_{1} \equiv x $ y $\displaystyle x_{2} \equiv x' $, así, tenemos que 
\[
\begin{cases}
x_{1}' = x_{2} \\
x_{2}' = x_{1}\left(1 - \frac{3}{2}x_{1}\right)
\end{cases}
.\]

\end{eg}

