\documentclass{article}

% packages

\usepackage{graphicx} % Required for images
\usepackage[spanish]{babel}
\usepackage{mdframed}
\usepackage{amsthm}
\usepackage{amssymb}
\usepackage{fancyhdr}
\usepackage{amsmath}
\usepackage{geometry}[margin=1in]
\usepackage{pgfplots}
\usepackage{url}
\usepackage{float}

% for math environments

\theoremstyle{definition}
\newtheorem*{theorem}{Teorema}
\newtheorem*{definition}{Definición}
\newtheorem*{prop}{Proposición}
\newtheorem*{observation}{Observación}
\newtheorem{ej}{Ejercicio}
\newtheorem{sol}{Solución}

% for headers and footers

\pagestyle{fancy}

%\fancyhead[R]{Victoria Eugenia Torroja}
% Store the title in a custom command
\newcommand{\mytitle}{}

% Redefine \title to store the title in \mytitle
\let\oldtitle\title
\renewcommand{\title}[1]{\oldtitle{#1}\renewcommand{\mytitle}{#1}}

% Set the center header to the title
\lhead{\mytitle}

% Custom commands

\newcommand{\R}{\mathbb{R}}
\newcommand{\C}{\mathbb{C}}
\newcommand{\F}{\mathbb{F}}
\newcommand{\N}{\mathbb{N}}
\newcommand{\Q}{\mathbb{Q}}
\newcommand{\Z}{\mathbb{Z}}
\newcommand{\K}{\mathbb{K}}
\newcommand{\mcd}{\text{mcd}}
\newcommand{\mcm}{\text{mcm}}
\DeclareMathOperator{\Ker}{Ker}
\DeclareMathOperator{\Imagen}{Im}
\DeclareMathOperator{\ord}{ord}
\DeclareMathOperator{\GL}{GL}
\DeclareMathOperator{\Biy}{Biy}


\begin{document}

\title{Métodos Numéricos - Demostraciones de las observaciones}
\author{Victoria Eugenia Torroja Rubio}
\date{8/9/2025}

\maketitle

\begin{prop}[Proposición 2.11, Página 71]
	Sea $\displaystyle \| \cdot \|  $ una norma en $\displaystyle V $. La aplicación $\displaystyle | | | \cdot | | | : \mathcal{M}_{n} \to \R^{+}\cup \left\{ 0\right\}  $ dada por 
	\[ | | | A | | | = \sup_{v \neq 0}\frac{\| Av \|}{\|v\|} = \sup_{\|v\|=1}\|Av\| ,\]
	es una norma matricial.
\end{prop}
\begin{proof}
Veamos que se cumplen las propiedades de las normas matriciales.
\begin{description}
\item[(i)] Está claro que como $\displaystyle \| A v\| \geq 0 $, $\displaystyle \forall v \in V $, si $\displaystyle | | | A | | | = 0 $, debe ser que $\displaystyle A = 0$.
\item[(ii)] Si $\displaystyle \lambda \in \K $, 
	\[| | | \lambda A | | | = \sup_{\|v\| = 1}\|\lambda Av\| = \sup_{\|v\| = 1} \left|\lambda \right|\|Av\| = \left|\lambda \right|\sup_{\|v\| = 1} = \left|\lambda \right| | | | A | | | .\]
\item[(iii)] Si $\displaystyle A,B \in \mathcal{M}_{n} $,
	\[ | | | A + B | | | = \sup_{\|v\| = 1}\| \left(A + B\right) v \| = \sup_{\|v\| = 1} \| A v + Bv \| \leq \sup_{\|v\| = 1}\left(\|Av\|+\|Bv\|\right) = \sup_{\|v\| = 1}\|Av\| + \sup_{\|v\| = 1}\|Bv\| .\]
\item[(iv)] Si $\displaystyle A,B \in \mathcal{M}_{n} $,
	\[| | | AB | | | = \sup_{\|v\| = 1}\|ABv\| \leq | | | A | | | \cdot \sup_{\|v\| = 1}\|Bv\| = | | | A | | | \cdot | | | B | | | .\]
	En efecto, por definición tenemos que 
	\[ | | | A | | | = \sup_{v \neq 0}\frac{\|Av\|}{\|v\|} \geq \frac{\|Av\|}{\|v\|} \iff | | | A | | | \cdot \|v\| \geq \|Av\| .\]
\end{description}
\end{proof}
\end{document}
