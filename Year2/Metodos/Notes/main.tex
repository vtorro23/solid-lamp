\documentclass{article}

% packages

\usepackage{graphicx} % Required for images
\usepackage[spanish]{babel}
\usepackage{mdframed}
\usepackage{amsthm}
\usepackage{amssymb}
\usepackage{fancyhdr}
\usepackage{amsmath}
\usepackage{geometry}[margin=1in]
\usepackage{pgfplots}
\usepackage{url}
\usepackage{float}

% for math environments

\theoremstyle{definition}
\newtheorem*{theorem}{Teorema}
\newtheorem*{definition}{Definición}
\newtheorem*{prop}{Proposición}
\newtheorem*{observation}{Observación}
\newtheorem{ej}{Ejercicio}
\newtheorem{sol}{Solución}

% for headers and footers

\pagestyle{fancy}

%\fancyhead[R]{Victoria Eugenia Torroja}
% Store the title in a custom command
\newcommand{\mytitle}{}

% Redefine \title to store the title in \mytitle
\let\oldtitle\title
\renewcommand{\title}[1]{\oldtitle{#1}\renewcommand{\mytitle}{#1}}

% Set the center header to the title
\lhead{\mytitle}

% Custom commands

\newcommand{\R}{\mathbb{R}}
\newcommand{\C}{\mathbb{C}}
\newcommand{\F}{\mathbb{F}}
\newcommand{\N}{\mathbb{N}}
\newcommand{\Q}{\mathbb{Q}}
\newcommand{\Z}{\mathbb{Z}}
\newcommand{\K}{\mathbb{K}}
\newcommand{\mcd}{\text{mcd}}
\newcommand{\mcm}{\text{mcm}}
\DeclareMathOperator{\Ker}{Ker}
\DeclareMathOperator{\Imagen}{Im}
\DeclareMathOperator{\ord}{ord}
\DeclareMathOperator{\GL}{GL}
\DeclareMathOperator{\Biy}{Biy}


\begin{document}

\title{Métodos Numéricos - Demostraciones de las observaciones}
\author{Victoria Eugenia Torroja Rubio}
\date{8/9/2025}

\maketitle
\begin{observation}[Observación 2.5, Página 57]
	\[\left(AB\right)^{-1} = B^{-1}A^{-1}, \quad \left(A^{*}\right)^{-1} = \left(A^{-1}\right)^{*}, \quad \left(A^{T}\right)^{-1} = \left(A^{-1}\right)^{T} .\]	
\end{observation}
\begin{proof}
\begin{description}
\item[(i)] Aprovechamos que $\displaystyle \GL_{n}\left(\K\right) $ es un grupo para usar la unicidad del inverso:
	\[AB \cdot B^{-1}A^{-1} = A A ^{-1} = I, \quad B^{-1}A^{-1}AB = B^{-1}B = I .\]
\item[(ii)] Sabemos que $\displaystyle \left(AB\right)^{*} = B^{*}A^{*} $:
	\[ A^{*} \cdot \left(A^{-1}\right)^{*} = \left(A^{-1} \cdot A\right)^{*} = I^{*} = I, \quad \left(A^{-1}\right)^{*} \cdot A^{*} = \left(A \cdot A^{-1}\right)^{*} = I^{*} = I .\]
\item[(iii)] Sabemos que $\displaystyle \left(AB\right)^{T} = B^{T}A^{T} $:
	\[A^{T}\left(A^{-1}\right)^{T} = \left(A^{-1} A\right)^{T}= I^{T} = I, \quad \left(A^{-1}\right)^{T}A^{T} = \left(A A^{-1}\right)^{T}= I^{T} = I .\]
\end{description}
\end{proof}
\begin{observation}[Observación 2.6, Página 58]
\begin{enumerate}
\item Toda matriz hermítica o unitaria es normal. 
\item Si $\displaystyle A $ es hermítica e inversible, $\displaystyle A^{-1} $ también es hermítica.
\item Si $\displaystyle A $ es normal e inversible, $\displaystyle A^{-1} $ también es normal. 
\end{enumerate}
\end{observation}
\begin{proof}
\begin{enumerate}
\item Trivial a partir de la definición.
\item Sea $\displaystyle A $ hermítica e inversible, veamos que $\displaystyle A^{-1} $ también es hermítica:
	\[A = A^{*} \iff A^{-1} = \left(A^{*}\right)^{-1} \iff A^{-1} = \left(A^{-1}\right)^{*} .\]
\item Sea $\displaystyle A $ normal e inversible, veamos que $\displaystyle A^{-1} $ también es normal:
	\[\left(A^{-1}\right)^{*}A^{-1} = A^{-1}\left(A^{-1}\right)^{*} \iff \left(A^{*}\right)^{-1}A^{-1} = A^{-1}\left(A^{*}\right)^{-1} \iff A A^{*} = A^{*}A .\]
\end{enumerate}
\end{proof}

\begin{prop}[Proposición 2.11, Página 71]
	Sea $\displaystyle \| \cdot \|  $ una norma en $\displaystyle V $. La aplicación $\displaystyle | | | \cdot | | | : \mathcal{M}_{n} \to \R^{+}\cup \left\{ 0\right\}  $ dada por 
	\[ | | | A | | | = \sup_{v \neq 0}\frac{\| Av \|}{\|v\|} = \sup_{\|v\|=1}\|Av\| ,\]
	es una norma matricial.
\end{prop}
\begin{proof}
Veamos que se cumplen las propiedades de las normas matriciales.
\begin{description}
\item[(i)] Está claro que como $\displaystyle \| A v\| \geq 0 $, $\displaystyle \forall v \in V $, si $\displaystyle | | | A | | | = 0 $, debe ser que $\displaystyle A = 0$.
\item[(ii)] Si $\displaystyle \lambda \in \K $, 
	\[| | | \lambda A | | | = \sup_{\|v\| = 1}\|\lambda Av\| = \sup_{\|v\| = 1} \left|\lambda \right|\|Av\| = \left|\lambda \right|\sup_{\|v\| = 1} \|Av\| = \left|\lambda \right| | | | A | | | .\]
\item[(iii)] Si $\displaystyle A,B \in \mathcal{M}_{n} $,
	\[ | | | A + B | | | = \sup_{\|v\| = 1}\| \left(A + B\right) v \| = \sup_{\|v\| = 1} \| A v + Bv \| \leq \sup_{\|v\| = 1}\left(\|Av\|+\|Bv\|\right) = \sup_{\|v\| = 1}\|Av\| + \sup_{\|v\| = 1}\|Bv\| .\]
\item[(iv)] Si $\displaystyle A,B \in \mathcal{M}_{n} $,
	\[| | | AB | | | = \sup_{\|v\| = 1}\|ABv\| \leq | | | A | | | \cdot \sup_{\|v\| = 1}\|Bv\| = | | | A | | | \cdot | | | B | | | .\]
	En efecto, por definición tenemos que 
	\[ | | | A | | | = \sup_{v \neq 0}\frac{\|Av\|}{\|v\|} \geq \frac{\|Av\|}{\|v\|} \iff | | | A | | | \cdot \|v\| \geq \|Av\| .\]
\end{description}
\end{proof}
\begin{prop}[Proposición 2.12, Página 72]
Sea $\displaystyle | | | \cdot | | | $ una norma matricial subordinada.
\begin{description}
\item[(i)] $\displaystyle \|Av\| \leq | | | A | | | \cdot  \|v\| $, $\displaystyle A \in \mathcal{M}_{n} $ y $\displaystyle v \in V $.
\item[(ii)] $\displaystyle | | | A | | | = \inf \left\{ \lambda \geq 0 \; : \; \|Av\| \leq \lambda \|v\|, \; v \in V\right\}  $.
\item[(iv)] Existe $\displaystyle u \in V^{*} $ tal que $\displaystyle \|Au\| = | | | A | | | \cdot \|u\| $.
\item[(v)] $\displaystyle | | | I | | | = 1 $.
\end{description}
\end{prop}
\begin{proof}
\begin{description}
\item[(i)] A partir de la definición
	\[ | | | A | | | = \sup_{v \neq 0}\frac{\|Av\|}{\|v\|} \geq \frac{\|Av\|}{\|v\|} \iff | | | A | | | \cdot \|v\| \geq \|Av\| .\]
\item[(ii)] Como $\displaystyle | | | A | | | = \sup_{ v \neq 0}\frac{\|Av\|}{\|v\|} $, si $\displaystyle M = | | | A | | | $ tenemos que 
	\[M \geq \frac{\|Av\|}{\|v\|}, \; \forall v \in V^{*} \Rightarrow \|Av\| \leq M \|v\|, \; \forall v \in V .\]
	Por ser $\displaystyle M $ el supremo, tenemos que ningún elemento de valor inferior va a cumplir esta propiedad, por lo que debe ser que $\displaystyle M = \inf \left\{ \lambda \geq 0 \; : \; \|Av\| \leq \lambda \|v\|\right\}  $.
\item[(iii)] Se deduce de la continuidad de $\displaystyle v \to \|Av\| $ sobre la esfera unidad, que es compacta, por lo que se alcanza el supremo y $\displaystyle \|Au\| = | | | A | | |$. Como $\displaystyle \|u\| = 1 $, se tiene que 
	\[\|Au\| = | | | A | | | \cdot \|u\| .\]
\item[(iv)] En efecto, como $\displaystyle \forall v \in V $, $\displaystyle I \cdot v = v $ tenemos que
	\[ | | | I | | | = \sup_{ \|v\|= 1} \|Iv\| = \sup_{ \|v\| = 1}\|v\| = 1 .\]
\end{description}
\end{proof}

\end{document}
