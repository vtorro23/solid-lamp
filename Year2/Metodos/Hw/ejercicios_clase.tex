\documentclass{article}

% packages

\usepackage{graphicx} % Required for images
\usepackage[spanish]{babel}
\usepackage{mdframed}
\usepackage{amsthm}
\usepackage{amssymb}
\usepackage{fancyhdr}

% for math environments

\theoremstyle{definition}
\newtheorem{theorem}{Teorema}
\newtheorem{definition}{Definición}
\newtheorem{ej}{Ejercicio}
\newtheorem{sol}{Solución}

% for headers and footers

\pagestyle{fancy}

\fancyhead[R]{Victoria Eugenia Torroja}
% Store the title in a custom command
\newcommand{\mytitle}{}

% Redefine \title to store the title in \mytitle
\let\oldtitle\title
\renewcommand{\title}[1]{\oldtitle{#1}\renewcommand{\mytitle}{#1}}

% Set the center header to the title
\lhead{\mytitle}

% Custom commands

\newcommand{\R}{\mathbb{R}}
\newcommand{\C}{\mathbb{C}}
\newcommand{\F}{\mathbb{F}}




\begin{document}

\title{Métodos Numéricos - Ejercicios de clase}
\author{Victoria Eugenia Torroja Rubio}
\date{8/9/2025}

\maketitle

\section*{Tema 1}

\begin{ej}
Tenemos que la $\displaystyle Bx \in \mathcal{M}_{m \times 1} $. Así, tenemos que
\[\left(Bx\right)_{i} = \sum^{n}_{j = 0} b_{ij}x_{j} = b^{T}_{i}x .\]
De esta manera, nos queda que 
\[Bx = \left(b^{T}_{i}x\right)_{i = 1}^{n} = \begin{pmatrix} b_{1}^{T}x \\ \vdots \\ b^{T}_{m}x \end{pmatrix} .\]
\end{ej}
\begin{ej}

\end{ej}

\end{document}
