%packages

\usepackage{graphicx} % Required for inserting images
\usepackage[spanish]{babel} % Required for automatic text to be in spanish
\usepackage{mdframed} % Required for boxed theorems and definitions
\usepackage{amsthm} % For theorem environment
\usepackage{amssymb} % For math symbols
\usepackage{fancyhdr} % For headers and footers
\usepackage{amsmath}
\usepackage{geometry}[margin=1in]
\usepackage{tikz}
\usepackage{url}
%\usepackage{xifthen}

% Headers and footers configuration
\pagestyle{fancy}

\fancyhead[L]{Victoria Eugenia Torroja}
\fancyhead[R]{\rightmark}
\fancyfoot[C]{\leftmark}
\fancyfoot[R]{\thepage}

% Math environments
\newtheorem{definition}{Definición}[chapter]
\newenvironment{fdefinition}
{\begin{mdframed}\begin{definition}}{\end{definition}\end{mdframed}}

\newtheorem{theorem}{Teorema}[chapter]
\newenvironment{ftheorem}
{\begin{mdframed}\begin{theorem}}{\end{theorem}\end{mdframed}}

\newtheorem{eg}{Ejemplo}[chapter]
\newenvironment{feg}
{\begin{mdframed}\begin{eg}}{\end{eg}\end{mdframed}}

\newtheorem{lema}{Lema}[chapter]
\newenvironment{flema}
{\begin{mdframed}\begin{lema}}{\end{lema}\end{mdframed}}

\newtheorem{prop}{Proposición}[chapter]
\newenvironment{fprop}
{\begin{mdframed}\begin{prop}}{\end{prop}\end{mdframed}}

\newtheorem{colorary}{Corolario}[chapter]
\newenvironment{fcolorary}
{\begin{mdframed}\begin{colorary}}{\end{colorary}\end{mdframed}}

\newtheorem{axiom}{Axioma}[]
\newenvironment{faxiom}
{\begin{mdframed}\begin{axiom}}{\end{axiom}\end{mdframed}}

\newtheorem*{notation}{Notación}

\newtheorem*{observation}{Observación}

% Custom commands

\newcommand{\R}{\mathbb{R}}
\newcommand{\C}{\mathbb{C}}
\newcommand{\F}{\mathbb{F}}
\newcommand{\N}{\mathbb{N}}
\newcommand{\Q}{\mathbb{Q}}
\newcommand{\Z}{\mathbb{Z}}
\newcommand{\K}{\mathbb{K}}
\newcommand{\mcd}{\text{mcd}}
\newcommand{\mcm}{\text{mcm}}
\DeclareMathOperator{\Ker}{Ker}
\DeclareMathOperator{\Imagen}{Im}
\DeclareMathOperator{\Hom}{Hom}
\DeclareMathOperator{\Aut}{Aut}
\DeclareMathOperator{\ran}{ran}
\DeclareMathOperator{\GL}{GL}
\DeclareMathOperator{\SL}{SL}
\DeclareMathOperator{\SO}{SO}
\DeclareMathOperator{\ord}{ord}
\DeclareMathOperator{\ind}{ind}
\DeclareMathOperator{\sig}{sig}
\DeclareMathOperator{\dom}{dom}
\DeclareMathOperator{\End}{End}
\DeclareMathOperator{\Adj}{Adj}
\DeclareMathOperator{\grad}{grad}
\DeclareMathOperator{\traz}{traz}
\DeclareMathOperator{\arctanh}{arctanh}
\DeclareMathOperator{\arcsinh}{arcsinh}
\DeclareMathOperator{\arccosh}{arccosh}
\DeclareMathOperator{\rad}{rad}
\DeclareMathOperator{\ad}{ad}
\DeclareMathOperator{\Bil}{Bil}
\DeclareMathOperator{\sech}{sech}
\setlength{\headheight}{13.07225pt}
\addtolength{\topmargin}{-1.07225pt}
\let\epsilon\varepsilon
