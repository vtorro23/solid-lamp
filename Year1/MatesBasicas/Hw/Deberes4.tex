\documentclass{article}

% packages

\usepackage{graphicx} % Required for images
\usepackage[spanish]{babel}
\usepackage{mdframed}
\usepackage{amsthm}
\usepackage{amssymb}
\usepackage{fancyhdr}
\usepackage{amsmath}
\usepackage{geometry}[margin=1in]
\usepackage{pgfplots}
\usepackage{url}
\usepackage{float}

% for math environments

\theoremstyle{definition}
\newtheorem*{theorem}{Teorema}
\newtheorem*{definition}{Definición}
\newtheorem*{prop}{Proposición}
\newtheorem*{observation}{Observación}
\newtheorem{ej}{Ejercicio}
\newtheorem{sol}{Solución}

% for headers and footers

\pagestyle{fancy}

%\fancyhead[R]{Victoria Eugenia Torroja}
% Store the title in a custom command
\newcommand{\mytitle}{}

% Redefine \title to store the title in \mytitle
\let\oldtitle\title
\renewcommand{\title}[1]{\oldtitle{#1}\renewcommand{\mytitle}{#1}}

% Set the center header to the title
\lhead{\mytitle}

% Custom commands

\newcommand{\R}{\mathbb{R}}
\newcommand{\C}{\mathbb{C}}
\newcommand{\F}{\mathbb{F}}
\newcommand{\N}{\mathbb{N}}
\newcommand{\Q}{\mathbb{Q}}
\newcommand{\Z}{\mathbb{Z}}
\newcommand{\K}{\mathbb{K}}
\newcommand{\mcd}{\text{mcd}}
\newcommand{\mcm}{\text{mcm}}
\DeclareMathOperator{\Ker}{Ker}
\DeclareMathOperator{\Imagen}{Im}
\DeclareMathOperator{\ord}{ord}
\DeclareMathOperator{\GL}{GL}
\DeclareMathOperator{\Biy}{Biy}


\begin{document}

\title{Matemáticas Básicas - Deberes 4}
\author{Victoria Eugenia Torroja Rubio}
\date{2/10/24}

\maketitle

\begin{ej}
Encontrar dos números complejos cuyo cuadrado sea
\[z_{0} := -8+6i .\]
\end{ej}

\begin{sol}
Sea $\displaystyle z\in\C $ tal que $\displaystyle z^{2}=z_{0} $, entonces:
\[z = \sqrt{z_{0}} = \sqrt{-8+6i} = \sqrt{10_{\arctan\frac{6}{-8}}}
.\]
Existen dos posibles soluciones:
\[
\begin{split}
& \sqrt{10}_{\arctan\frac{6}{-8}\cdot\frac{1}{2}}=1+3i\\
& \sqrt{10}_{\arctan\frac{6}{-8}\cdot\frac{1}{2}+180}=-1-3i
\end{split}
\]
\end{sol}

\begin{ej}
Se define la transformación $\displaystyle T:\C \to \C $ como $\displaystyle T\left(z\right)=z^{2} $. Calcula la imagen por $\displaystyle T $ de las rectas $\displaystyle y = x $, $\displaystyle y = -x $ y $\displaystyle x = 1 $. Representa gráficamente las rectas anteriores y sus imágenes por $\displaystyle T $. Encuentra un subconjunto de $\displaystyle \C $ cuya imagen sea una circunferencia.
\end{ej}

\begin{sol}
Sea $\displaystyle f \subset \C $ la recta $\displaystyle y = x $. Consideramos la recta $\displaystyle y = x $ y $\displaystyle z\in\C $ perteneciente a esta recta. Tenemos que $\displaystyle z = x + xi $, por tanto:
\[T\left(z\right)=\left(x+xi\right)^{2} = 2x^{2}i .\]
Dado que $\displaystyle x^{2}\geq0 $ , la imagen de esta recta es el conjunto $\displaystyle \left\{ z \; : \; z = xi, \; x\geq0\right\}  $. \\
\begin{center}
\begin{tikzpicture}
  \begin{axis}[
    axis lines = middle,
    title = {Transformación de $\displaystyle f $ },
    ymin=-3, ymax=3,
    xmin=-3, xmax=3,
    domain=-3:3,
    samples=100,
    legend pos=north west,
    every axis plot/.append style={ultra thick},
    xtick=\empty,  % Remove x-axis numbering
    ytick=\empty   % Remove y-axis numbering
  ]
    % Plot y = x, label it "f"
    \addplot[blue] {x};
    \addlegendentry{$f$};
    
    % Plot x = 0 (vertical line), label it "T(f)"
    \addplot[red, thick] coordinates {(0,0) (0,3)};
    \addlegendentry{$T(f)$};
  \end{axis}
\end{tikzpicture}
\end{center}

Sea $\displaystyle g \subset \C $ la recta $\displaystyle y = -x $. Consideramos la recta $\displaystyle y = -x $, entonces tenemos $\displaystyle z \in \C $ tal que $\displaystyle z = x - xi $, por tanto
\[T\left(z\right) = z^{2} = \left(x-xi\right)^{2} = -2x^{2}i .\]
Como $\displaystyle x^{2}\geq0 $, tenemos que la imagen de la recta $\displaystyle y = -x $ será el conjunto $\displaystyle \left\{ z \in \C \; : \; z = xi, \; x\leq 0\right\}  $. \\ 

\begin{center}
\begin{tikzpicture}
  \begin{axis}[
    axis lines = middle,
    title = {Transformación de $\displaystyle g$ },
    ymin=-3, ymax=3,
    xmin=-3, xmax=3,
    domain=-3:3,
    samples=100,
    legend pos=north east,
    every axis plot/.append style={ultra thick},
    xtick=\empty,  % Remove x-axis numbering
    ytick=\empty   % Remove y-axis numbering
  ]
    % Plot y = x, label it "f"
    \addplot[blue] {-x};
    \addlegendentry{$g$};
    
    % Plot x = 0 (vertical line), label it "T(f)"
    \addplot[red, thick] coordinates {(0,0) (0,-3)};
    \addlegendentry{$T(g)$};
  \end{axis}
\end{tikzpicture}
\end{center}

A continuación, sea $\displaystyle h \subset \C $ la recta $\displaystyle x=1 $. Consideramos la recta $\displaystyle x = 1 $ y $\displaystyle z \in \C $ tal que $\displaystyle z $ pertenece a la recta mencionada. Por tanto, $\displaystyle z = 1 + yi $ y 
\[T\left(z\right) = \left(1+yi\right)^{2} = \left(1-y^{2}\right) + 2yi .\]
Si $\displaystyle w $ pertenece a la transformación de $\displaystyle x = 1 $, tenemos que $\displaystyle w = a + bi = \left(1-y^{2}\right) + 2yi $. Por todo ello,
\[
\begin{split}
& a = 1 - y^{2} \\
& b = 2y.
\end{split}
\]
De esto concluimos que:
\[
\begin{split}
& a = 1 - \left(\frac{b}{2}\right)^{2} \\
\therefore & b = 2\sqrt{1-a}.
\end{split}
\]
Como $\displaystyle a,b \in \R $, tenemos que $\displaystyle a \leq 1 $. \\ 
\begin{center}
\begin{tikzpicture}
  \begin{axis}[
    axis lines = middle,
    title = {Transformación de $\displaystyle h $ },
    ymin=-3, ymax=3,
    xmin=-3, xmax=3,
    domain=-3:3,
    samples=100,
    every axis plot/.append style={ultra thick},
    xtick=\empty,  % Remove x-axis numbering
    ytick=\empty   % Remove y-axis numbering
  ]
    % Plot T(h) (red curves)
    \addplot[red] {2*sqrt(1-x)};
    \addplot[red] {-2*sqrt(1-x)};
    
    % Plot h (blue vertical line)
    \addplot[blue, thick] coordinates {(1,-3) (1,3)};
    
    % Label T(h)
    \node at (axis cs:0.5,2.5)  {$T(h)$};
    
    % Label h
    \node at (axis cs:1.2,2)  {$h$};
    
  \end{axis}
\end{tikzpicture}
\end{center}
En cuanto al último apartado, consideremos el conjunto $\displaystyle S =\left\{ z \in \C \; : \; \left|z\right| = a \; (a \in \R^{+})\right\}\subset \C $. Si $\displaystyle w \in S $, tenemos que su transformación será:
\[T\left(w\right) = w ^{2} = \left(|w|_{\theta}\right)^{2}= \left|w\right|^{2}_{2\theta}=a^{2}_{2\theta} .\]
Como $\displaystyle \theta \in \left[0,2\pi\right]  $, tenemos que $\displaystyle 2\theta \in \left[0,2\pi\right]  $ (los ángulos superiores a $\displaystyle 2\pi $ los reducimos a otros que estén en el intervalo $\displaystyle \left[0,2\pi\right]  $). Por tanto, la imagen de $\displaystyle S $ será la circunferencia con centro $\displaystyle \left(0,0\right) $ y radio $\displaystyle a^{2} \in \R $.  
\end{sol}

\end{document}
