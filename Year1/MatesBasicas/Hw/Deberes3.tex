\documentclass{article}

% packages

\usepackage{graphicx} % Required for images
\usepackage[spanish]{babel}
\usepackage{mdframed}
\usepackage{amsthm}
\usepackage{amssymb}
\usepackage{fancyhdr}

% for math environments

\theoremstyle{definition}
\newtheorem{theorem}{Teorema}
\newtheorem{definition}{Definición}
\newtheorem{ej}{Ejercicio}
\newtheorem{sol}{Solución}

% for headers and footers

\pagestyle{fancy}

\fancyhead[R]{Victoria Eugenia Torroja}
% Store the title in a custom command
\newcommand{\mytitle}{}

% Redefine \title to store the title in \mytitle
\let\oldtitle\title
\renewcommand{\title}[1]{\oldtitle{#1}\renewcommand{\mytitle}{#1}}

% Set the center header to the title
\lhead{\mytitle}

% Custom commands

\newcommand{\R}{\mathbb{R}}
\newcommand{\C}{\mathbb{C}}
\newcommand{\F}{\mathbb{F}}




\begin{document}

\title{Matemáticas Básicas - Deberes 3}
\author{Victoria Eugenia Torroja Rubio}
\date{25/9/24}

\maketitle

\begin{ej}
Sean $\displaystyle f $ y $\displaystyle g $ $\displaystyle : \R \to \R $ las funciones definidas por $\displaystyle f\left(x\right) = x^{2} $ y $\displaystyle g\left(x\right) = x^{2}-1 $. Hallar las funciones $\displaystyle f\circ f $, $\displaystyle f \circ g $, $\displaystyle g\circ f $ y $\displaystyle g \circ g $ y determinar el conjunto
\[ \left \{ x \in \R \; : \left(f \circ g\right)\left(x\right)=\left(g\circ f\right)\left(x\right)\right \}  .\]
\end{ej}

\begin{sol}
Calculamos las funciones compuestas que nos pide el enunciado:
\[
\begin{split}
& f\left(f\left(x\right)\right) = \left(x^{2}\right)^{2} = x^{4},\quad \quad f\left(g\left(x\right)\right) = \left(x^{2}-1\right)^{2} \\
& g\left(f\left(x\right)\right) = \left(x^{2}\right)^{2}-1 = x^{4}- 1,\quad \quad g\left(g\left(x\right)\right)= \left(x^{2}-1\right)^{2}-1 = x^{4} - 2x^{2}=x^{2}\left(x^{2}-2\right).
\end{split}
\]
A continuación, resolvemos el siguiente apartado. Si $\displaystyle \left(f\circ g\right)\left(x\right) = \left(g\circ f\right)\left(x\right) $, entonces tenemos que 
\[
\begin{split}
& \left(x^{2}-1\right)^{2} = x^{4}-1 \\
\Rightarrow & x^{4} -2x^{2} + 1 = x^{4}-1 \\
\Rightarrow & 2x^{2} = 2 \\
\Rightarrow & x = \pm 1.
\end{split}
\]
Por tanto, 
\[ \left \{ x \in \R \; : \left(f \circ g\right)\left(x\right)=\left(g\circ f\right)\left(x\right)\right \} = \left \{ 1, \; -1\right \}  .\]

\end{sol}


\begin{ej}
	Se define en $\displaystyle \R^{2} $ la relación $\displaystyle \left(x,y\right)\mathcal{R}\left(a,b\right) $ si y solo si $\displaystyle y-b = x^{2}-a^{2} $. Demuestra que $\displaystyle \mathcal{R} $ es una relación de equivalencia. Describe las clases de equivalencia $\displaystyle \left[\left(0,0\right)\right]  $, $\displaystyle \left[\left(0,2\right)\right]  $ y $\displaystyle \left[\left(1,1\right)\right]  $. Describe la clase de un punto cualquiera $\displaystyle \left(a,b\right)\in\R^{2} $. Describe el conjunto cociente $\displaystyle \R^{2}/\mathcal{R} $. 
\end{ej}

\begin{sol}
	En primer lugar, demostramos que la relación $\displaystyle \mathcal{R} \subset \R^{2} $ es una relación de equivalencia:
	\begin{description}
		\item[(i)] Tenemos que es reflexiva, pues $\displaystyle \forall\left(x,y\right)\in\R^{2}\; , \left(x,y\right)\mathcal{R}\left(x,y\right) $:
			\[\forall x,y \in \R, \quad y - y = x^{2} - x^{2} \quad \Rightarrow \quad 0=0  .\]
		\item[(ii)] Si $\displaystyle \left(x,y\right)\mathcal{R}\left(w,z\right) $, 
			\[y - z = x^{2}-w^{2}, \; \text{multiplicamos ambos lados por $\displaystyle -1 $ }, \; z-y = w^{2} - x^{2} .\]
			Por tanto, $\displaystyle \left(x,y\right)\mathcal{R}\left(w,z\right) \Rightarrow \left(w,z\right)\mathcal{R}\left(x,y\right) $ y $\displaystyle \mathcal{R} $ es simétrica. 
		\item[(iii)] Si $\displaystyle \left(a,b\right)\mathcal{R}\left(c,d\right) $ y $\displaystyle \left(c,d\right)\mathcal{R}\left(e,f\right) $, entonces
			\[
			\begin{split}
			& b-d = a^{2}-c^{2} \\
			& d-f = c^{2}-e^{2} \\
			\therefore & b - f = a^{2}-e^{2} .
			\end{split}
			\]
			Por tanto, $\displaystyle \mathcal{R} $ es transitiva. 
	\end{description}
	En conclusión, $\displaystyle \mathcal{R} $ es una relación de equivalencia. \\ \\
	A continuación, encontramos las clases de equivalencia que nos pide el enunciado. Para ello, sustituimos los puntos que nos dan ((0,0), (0,2) y (1,1)) y encontramos los puntos $\displaystyle \left(x,y\right) $ que satisfacen dicha condición. Mostramos como ejemplo el caso de $\displaystyle \left[\left(1,1\right)\right]  $:
	\[1-y = 1- x^{2} \quad \Rightarrow y = x^{2} .\]
	Por lo que los pares $\displaystyle \left(x,y\right) $ que componen la clase de equivalencia de $\displaystyle \left[\left(1,1\right)\right]  $ son $\displaystyle \left(x,x^{2}\right) $ con $\displaystyle x\in\R $. \\ \\
	Las otras clases de equivalencia se obtienen de la misma manera.
	\[
	\begin{split}
	& \left[\left(0,0\right)\right] = \left \{ \left(x,y\right)\in \R^{2}\; : \; \left(0,0\right)\mathcal{R}\left(x,y\right)\right \} = \left \{ \left(x, x^{2}\right)\; : \; x\in \R\right \} \\
	 & \left[\left(1,1\right)\right] = \left \{ \left(x,y\right)\in \R^{2}\; : \; \left(1,1\right)\mathcal{R}\left(x,y\right)\right \} = \left \{ \left(x, x^{2}\right)\; : \; x\in \R\right \}\\
	 & \left[\left(0,2\right)\right] = \left \{ \left(x,y\right)\in \R^{2}\; : \; \left(0,2\right)\mathcal{R}\left(x,y\right)\right \} = \left \{ \left(x, x^{2}+2\right)\; : \; x\in \R\right \} .
	\end{split}
	\]
	En general, la clase de un punto $\displaystyle \left(a,b\right)\in\R^{2} $ será el conjunto de puntos $\displaystyle \left(x,y\right)\in\R^{2} $ tales que $\displaystyle \left(a,b\right)\mathcal{R} \left(x,y\right) $:
	\[b-y = a^{2}-x^{2} \; \Rightarrow y = x^{2} - \left(a^{2}-b\right) .\]
Es decir,
\[\left[\left(a,b\right)\right] = \left \{ \left(x,x^{2}-\left(a^{2}-b\right)\right) \; : \; x \in \R\right \} = \left[\left(0,b-a^{2}\right)\right]   .\]
Por tanto, todas las clases de equivalencia de los puntos $\displaystyle \left(a,b\right) $ con $\displaystyle a\neq0 $ se pueden expresar como la clase de equivalencia de $\displaystyle \left(0,k\right) $ con $\displaystyle k \in \R $. Consecuentemente, el conjunto cociente $\displaystyle \R^{2}/\mathcal{R} $ será el conjunto de todas las parábolas de la forma $\displaystyle y = x^{2}+c $ con $\displaystyle c\in\R $. 
\[\R^{2}/\mathcal{R} = \left \{ \left[\left(0,a\right)\right] \; : \; a \in \R\right \}  .\]
\end{sol}


\end{document}
