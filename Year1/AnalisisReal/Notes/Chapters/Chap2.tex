\chapter{Sucesiones y límites}

\begin{fdefinition}[]
	\normalfont Se dice que $\displaystyle \left\{ x_{n}\right\}_{n\in\N}  $ es una \textbf{sucesión} de números reales si existe una función
\[
\begin{split}
	\varphi: &\N \to \R\\
& n \to \varphi\left(n\right) = x_{n}.
\end{split}
\]
\end{fdefinition}

\begin{eg}
\normalfont 
\begin{description}
\item[(i)] $\displaystyle \varphi\left(n\right) = \frac{1}{n} $.
\item[(ii)] Sucesión de Fibonacci.
	\[\varphi\left(1\right) = 1, \; \varphi\left(2\right) = 1, \; \varphi\left(n\right) = \varphi\left(n-1\right) + \varphi\left(n-2\right) .\]
\end{description}
\end{eg}

\begin{fdefinition}[]
	\normalfont Se dice que una sucesión $\displaystyle \left\{ x_{n}\right\}_{n\in \N} $ \textbf{converge} a $\displaystyle x \in \R $, y lo escribiremos de esta manera
	\[\lim_{n \to \infty}x_{n}=x ,\]
si
\[\forall \epsilon > 0, \; \exists n_{0}\in \N,\;  \left|x - x_{n}\right|<\epsilon, \; \forall n \geq n_{0}  .\]
\end{fdefinition}

\begin{fprop}[]
	\normalfont Si $\displaystyle x_{n}= \frac{1}{n} $, entonces $\displaystyle \lim_{n \to \infty}\frac{1}{n} = 0 $.
\end{fprop}

\begin{proof}
Sea $\displaystyle \epsilon > 0 $, tomamos $\displaystyle x = 0 $, tenemos que 
\[ \left|x - x_{n}\right| = \left|0 - \frac{1}{n}\right| = \frac{1}{n} .\]
Entonces, queremos probar que 
\[\frac{1}{n}<\epsilon, \; n \geq n_{0} .\]
Como sabemos que $\displaystyle \inf \left\{ \frac{1}{n} \; : \; n \in \N\right\} = 0 $, tenemos que $\displaystyle \forall \epsilon > 0 $ 
\[0 \leq \frac{1}{n_{0}} < 0 + \epsilon ,\]
para algún $\displaystyle n_{0} \in \N $. Si $\displaystyle n \geq n_{0} $, tenemos que
\[n \geq n_{0} \iff \frac{1}{n} \leq \frac{1}{n_{0}}<\epsilon .\]
Por tanto, hemos encontrado $\displaystyle n_{0}\in \N $ tal que $\displaystyle \forall n \geq n_{0} $, 
\[ \left|x - x_{n}\right| \leq \epsilon  .\]
\end{proof}

\begin{eg}
\normalfont 
\begin{description}
\item[(i)] Cogemos $\displaystyle x_{n} = \frac{\left(-1\right)^{n}}{n} $. Vamos a ver que $\displaystyle \lim_{n \to \infty}x_{n} = 0 $, mientras que $\displaystyle \sup x_{n} \neq \inf x_{n} \neq 0 $. 
	\[ \left|x_{n}-0\right| = \frac{1}{n} .\]
$\displaystyle \forall \epsilon > 0, \exists n_{0} \in \N $ tales que $\displaystyle  \forall n\geq n_{0} $, 
\[ \left|x_{n} -0\right| = \frac{1}{n} < \epsilon .\]
Por lo que
\[\lim_{n \to \infty} x_{n} = \lim_{n \to \infty} \frac{\left(-1\right)^{n}}{n} = 0 .\]
\item[(ii)] $\displaystyle x_{n} = \left(-1\right)^{n} $. Esta sucesión no converge, pues oscila. Tenemos que ver que $\displaystyle \forall x, \exists \epsilon > 0, \forall n_{0}\in \N, \exists n\geq n_{0} $ tal que $\displaystyle \left|x-x_{n}\right| \geq \epsilon  $. Supongamos que $\displaystyle x > 1 $ y tomamos $\displaystyle \epsilon = 2 $ y $\displaystyle n_{0} \in \N $. Sea $\displaystyle n \geq n_{0} $ impar. Entonces 
	\[ \left|x_{n}-x\right| = \left|-1-x\right| = 2 + x -1 = 1 + x > 2 = \epsilon  .\]
Si $\displaystyle x < -1 $, tenemos que $\displaystyle -x>1 $. Tomamos $\displaystyle \epsilon = 2 $. Podemos encontrar $\displaystyle n_{0} \in \N $ tal que $\displaystyle n \geq n_{0} $ y $\displaystyle n $ es par. 
\[ \left|x - 1\right| = 2 + \left(-1-x\right) = 1 - x \geq 2 = \epsilon.\]
Finalmente, si $\displaystyle -1\leq x \leq 1 $, tomamos $\displaystyle \epsilon = 1 $. Si $\displaystyle x = 0 $, tenemos que 
\[ \left|1 - 0\right| = \left|0 - 1\right| = 1 \geq 1 = \epsilon  .\]
Si $\displaystyle x > 0 $, tenemos que si $\displaystyle n_{0} \in \N $ podemos encontrar $\displaystyle n \geq n_{0} $ impar, tal que
\[ \left|x - \left(-1\right)\right| = x + 1 \geq 1 = \epsilon .\]
Similarmente, si $\displaystyle x < 0 $ ($\displaystyle -x > 0 $) y $\displaystyle n_{0} \in \N $ podemos encontrar $\displaystyle  $ tal que $\displaystyle n $ sea par:
\[ \left|1 - x\right| = 1 - x \geq 1 = \epsilon  .\]
\end{description}
\end{eg}

\begin{fprop}[]
	\normalfont Si una sucesión $\displaystyle \left\{ x_{n}\right\}_{n \in \N} $ converge, entonces el límite es único. 
\end{fprop}

\begin{proof}
	Supongamos que existen $\displaystyle x, x' \in \R $ con $\displaystyle x \neq x' $ tales que $\displaystyle \forall \epsilon > 0, \exists n_{0}, n'_{0} \in \N $ tales que si $\displaystyle n \geq n_{0}$ y $ n \geq n_{0} $, $\displaystyle \left|x_{n}-x\right| < \epsilon  $ y $\displaystyle \left|x_{n}-x\right|<\epsilon  $. Sea $\displaystyle \epsilon = \frac{ \left|x - x'\right|}{3} $. Entonces, podemos encontrar $\displaystyle n_{0} \in \N $ tal que $\displaystyle  \left|x_{n}-x\right| < \epsilon  $ si $\displaystyle n \geq n_{0} $. Lo mismo sucede con $\displaystyle n'_{0} \in \N $. Tenemos que $\displaystyle \left|x-x'\right| = 3 \epsilon $. Sea $\displaystyle n \geq \max \left\{ n_{0}, n'_{0}\right\}  $, 
	\[ 3 \epsilon = \left|x - x_{n} + x_{n}-x'\right| \leq \left|x - x_{n}\right| + \left|x_{n}-x'\right| < 2\epsilon .\]
	Esto es una contradicción, por lo que $\displaystyle x = x' $. \\ \\
Otra demostración consiste en asumir que existen dos límites de la sucesión, $\displaystyle x $ y $\displaystyle x' $. Tenemos que si $\displaystyle \epsilon > 0 $, existe $\displaystyle n_{1} \in \N $ tal que si $\displaystyle n \geq n_{1} $ 
\[ \left|x_{n}-x\right| < \frac{\epsilon }{2} .\]
Similarmente, existe $\displaystyle n_{2}\in\N $ tal que si $\displaystyle n \geq n_{2} $, entonces 
\[ \left|x_{n}-x'\right| < \frac{\epsilon }{2} .\]
Entonces, si cogemos $\displaystyle n_{0} = \max \left\{ n_{1}, n_{2}\right\}  $ y $\displaystyle n \geq n_{0} $, tenemos que 
\[ \left|x - x'\right| = \left|x - x_{n} + x_{n}- x'\right| \leq \left|x - x_{n}\right| + \left|x_{n}-x'\right| < \frac{\epsilon }{2} + \frac{\epsilon }{2} = \epsilon  .\]
Por tanto, $\displaystyle x' = x $.
\end{proof}

\begin{fdefinition}[]
\normalfont Sea $\displaystyle \left\{ x_{n}\right\}_{n\in\N} \subset \R$. 
\begin{description}
\item[(i)] Diremos que la sucesión diverge a $\displaystyle \infty $, es decir, $\displaystyle \lim_{n \to \infty}x_{n} = \infty $, si 
	\[\forall c>0, \exists n_{0} \in \N, \forall n \geq n_{0}, \; x_{n} \geq c .\]
\item[(ii)] Diremos que la suciesión diverge a $\displaystyle -\infty $, es decir, $\displaystyle \lim_{n \to \infty}x_{n} = - \infty $ si 
	\[\forall c < 0, \exists n_{0} \in \N, \forall n \geq n_{0}, \; x\leq c .\]
\end{description}
\end{fdefinition}

\begin{observation}
\normalfont Existen sucesiones que convergen y las que no convergen. Dentro de las que no convergen están las que divergen (a $\displaystyle \infty $ y $\displaystyle -\infty $) y las que no divergen ($\displaystyle \left(-1\right)^{n} $).
\end{observation}

\begin{eg}
\normalfont Tenemos que $\displaystyle x_{n} = n $ satisface que $\displaystyle \lim_{n \to \infty}x_{n} = \infty $.
\end{eg}

\begin{eg}
\normalfont Consideremos $\displaystyle x_{n} = \frac{2n}{n+1} $, $\displaystyle n \in \N $. Demostramos que $\displaystyle \lim_{n \to \infty}x_{n} = 2 $. \\ \\
Queremos decir que $\displaystyle \forall \epsilon > 0, \exists n_{0} \in \N, \forall n\geq n_{0}, \; \left|x_{n}-2\right| < \epsilon  $. Tenemos que
\[ \left|\frac{2n}{n+1} -2\right| = \left|\frac{2n-2n-2}{n+1}\right| = \frac{2}{n+1} .\]
\[\frac{2}{n_{0}+1} < \epsilon \iff \frac{2}{\epsilon }-1 < n_{0} .\]
Por la propiedad arquimediana, existe $\displaystyle n_{0} \in \N $. Si $\displaystyle n \geq n_{0} $, tenemos que 
\[\frac{2}{n+1} < \epsilon  .\]
\end{eg}

\begin{fprop}[]
	\normalfont Sea $\displaystyle \left\{ x_{n}\right\} _{n\in\N}\subset\R $ y sea $\displaystyle m \in \N $. Sea $\displaystyle y_{n} = x_{n+m} $. Entonces, 
	\[\lim_{n \to \infty}x_{n} = x \iff \lim_{n \to \infty}y_{n} = x .\]
\end{fprop}

\begin{proof}
\begin{description}
\item[(i)] Asumimos que $\displaystyle \lim_{n \to \infty}x_{n} = x $. Entonces tenemos que 
	\[\forall \epsilon > 0, \exists n_{0} \in \N, \forall n \geq n_{0}, \; \left|x - x_{n}\right| < \epsilon  .\]
Como $\displaystyle m \in \N $, tenemos que $\displaystyle m + n_{0} > n_{0} $, por lo que, si $\displaystyle n \geq n_{0} $, $\displaystyle \left|x - x_{m+n}\right|< \epsilon  $. Es decir, 
\[\forall \epsilon > 0, \exists n_{0} \in \N, \forall n \geq n_{0}, \left|x - y_{n}\right| < \epsilon  .\]
\item[(ii)] Recíprocamente, si $\displaystyle \lim_{n \to \infty}y_{n} = x $, tenemos que 
	\[\forall \epsilon > 0, \exists n_{0} \in \N, \forall n \geq n_{0}, \; \left|x - y_{n}\right| < \epsilon  .\]
Es decir, 
	\[\forall \epsilon > 0, \exists n_{0}+m \in \N, \forall n \geq n_{0}, \; \left|x - x_{m+n}\right| < \epsilon  .\]
Si cogemos $\displaystyle n'_{0} = m + n_{0}$, tenemos que 
\[\forall \epsilon > 0, \exists n'_{0}, \forall n\geq n'_{0}, \left|x - x_{n}\right| < \epsilon  .\]
\end{description}
\end{proof}

\begin{ftheorem}[Regla del bocadillo]
	\normalfont Sean $\displaystyle \left\{ x_{n}\right\}_{n\in\N} $, $\displaystyle \left\{ y_{n}\right\} _{n\in\N}, \left\{ z_{n}\right\} _{n\in\N}\subset\R $ con 
	\[\forall n \in \N, \; y_{n} \leq z_{n} \leq x_{n} .\]
Supongamos que $\displaystyle \lim_{n \to \infty}y_{n} = \lim_{n \to \infty}x_{n} = x $. Entonces, $\displaystyle \lim_{n \to \infty}z_{n} =x $.
\end{ftheorem}
\begin{proof}
	Sea $\displaystyle \epsilon > 0 $. Cogemos $\displaystyle n_{1} \in \N $ tal que $\displaystyle \forall n \geq n_{1} $ 
	\[ \left|y_{n}-x\right| < \frac{\epsilon }{4} .\]
Similarmente, sea $\displaystyle n_{2} \in \N $ tal que $\displaystyle \forall n\geq n_{2} $, 
\[ \left|x_{n} - x\right|< \frac{\epsilon }{6} .\]
Sea $\displaystyle n_{0} = \max \left\{ n_{1}, n_{2}\right\}  $. Sea $\displaystyle n \geq n_{0} $,
\[
\begin{split}
	\left|z_{n} - x\right| & = \left|z_{n}-y_{n}+y_{n}-x_{n}+x_{n}-x\right| \\
			       &\leq \left|z_{n}-y_{n}\right| + \left|y_{n}-x_{n}\right| + \left|x_{n}-x\right|\\
			       &= z_{n}-y_{n}+x_{n}-y_{n}+ \left|x_{n}-x\right| \\
			       & \leq x_{n}-y_{n} + x_{n}-y_{n} + \left|x_{n}-x\right| \\
			       & = 2\left(x_{n}-y_{n}\right) + \left|x_{n}-x\right| .
\end{split}
\]
\begin{observation}
\normalfont Tenemos que
\[ \left|x_{n}-y_{n}\right| = \left|x_{n}-x + x - y_{n}\right| \leq \left|x_{n}-x\right|+ \left|y_{n}-x\right| .\]
\end{observation}
Por tanto, 
\[2\left(x_{n}-y_{n}\right) + \left|x_{n}-x\right| \leq 3 \left|x_{n}-x\right| + 2 \left|x - y_{n}\right| < 3 \cdot \frac{\epsilon }{6} + 2 \cdot \frac{\epsilon }{4} = \epsilon .\]
Una demostración alternativa es decir que existen $\displaystyle n_{1}, n_{2} \in \N $ tales que si $\displaystyle \epsilon > 0 $, tenemos que 
\[
\begin{split}
& \forall n \geq n_{1}, \; \left|x_{n} - x\right| < \epsilon \iff -\epsilon < x_{n}-x < \epsilon  \iff -\epsilon + x < x_{n}< \epsilon + x\\
& \forall n \geq n_{2}, \; \left|y_{n}-x\right| < \epsilon \iff - \epsilon < y_{n}-x < \epsilon\iff -\epsilon + x < y_{n} < \epsilon + x.
\end{split}
\]
Sea $\displaystyle n > \max \left\{ n_{1}, n_{2}\right\}  $, por hipótesis tenemos que 
\[-\epsilon + x < x_{n} < z_{n} < y_{n} < \epsilon + x .\]
\[\therefore \left|z_{n}-x\right|<\epsilon  .\]
\end{proof}

\begin{eg}
\normalfont 
\begin{description}
\item[(i)] $\displaystyle \forall k \in \N $, 
	\[\lim_{n \to \infty}\frac{1}{n^{k}} =0 .\]
Como $\displaystyle n^{k} \geq n $, tenemos que $\displaystyle n^{k-1} \geq 1 $. Además, podemos deducir que 
\[0 \leq \frac{1}{n^{k}} \leq \frac{1}{n} .\]
La primera sucesión converge a 0 y la segunda también converge a 0 (propiedad arquimediana), por lo que $\displaystyle \frac{1}{n^{k}} \to 0 $.
\item[(ii)] Si $\displaystyle k \in \N $, 
	\[\lim_{n \to \infty}\frac{\sin n}{n^{k}} .\]
Tenemos que
\[ 0 \leq \left|\frac{\sin n}{n^{k}}\right| \leq \frac{1}{n^{k}} .\]
Como $\displaystyle 0 \to 0 $ y $\displaystyle \frac{1}{n^{k}} \to 0 $, tenemos que $\displaystyle \frac{\sin n}{n^{k}} \to 0 $.
\end{description}
\end{eg}

\begin{observation}
\normalfont En la regla del bocadillo, basta que las estimaciones sean ciertas a partir de un cierto valor. Es decir, si $\displaystyle y_{n} \leq z_{n} \leq x_{n}, \; n\geq n_{0} $, si $\displaystyle y_{n}, x_{n} \to x $, tenemos que $\displaystyle z_{n} \to x $.
\end{observation}

\begin{eg}
\normalfont La sucesión $\displaystyle \frac{n}{2^{n}} \to 0$. Esto lo demostramos diciendo que $\displaystyle \frac{n}{2^{n}} \leq \frac{1}{n} $. En el caso $\displaystyle n = 3 $ esto no se cumple, porque se cumple en $\displaystyle n\geq 4 $. 
\end{eg}

\begin{fdefinition}[]
	\normalfont Se dice que $\displaystyle \left\{ x_{n}\right\}_{n\in\N}  \subset \R$ está acotada si existe $\displaystyle c > 0 $, tal que 
	\[ \left|x_{n}\right| \leq c, \; \forall n \in \N .\]
\footnote{Es decir, $\displaystyle -c \leq x_{n} \leq c $, o sea, está acotado superior e inferiormente.} 
\end{fdefinition}

\begin{ftheorem}[]
	\normalfont Si $\displaystyle \left\{ x_{n}\right\} _{n\in\N} \subset \R $ converge, entonces está acotada. 
\end{ftheorem}

\begin{proof}
Sea $\displaystyle \epsilon = 1 $, entonces existe $\displaystyle n_{0}\in \N $ tal que para $\displaystyle n \geq n_{0} $
\[ \left|x_{n}-x\right| < 1 .\]
Además, $\displaystyle x = \lim_{n \to \infty}x_{n} $. Tenemos que si $\displaystyle n \geq n_{0} $:
\[
\begin{split}
	\left|x_{n}\right| & = \left|x_{n}-x+x\right| \\
			   & \leq \left|x_{n}-x\right| + \left|x\right| \\
			   & \leq 1 + \left|x\right|.
\end{split}
\]
Sea $\displaystyle c = \max \left\{ 1+ \left|x\right|, \left|x_{n}\right|\; : \; n \leq n_{0}\right\}  $. Entonces tenemos que 
\[ \left|x_{n}\right| \leq c, \; \forall n \in \N .\]
\end{proof}

\begin{observation}
\normalfont El recíproco del teorema anterior no es cierto en general. Considera $\displaystyle \left(-1\right)^{n} $ que está acotada por 1 pero no converge.
\end{observation}

\begin{eg}
	\normalfont Si $\displaystyle  x \in \R $, existe $\displaystyle a_{0 } \in \Z $ y existen $\displaystyle a_{n}\in \left\{ 0, \ldots, 9\right\}  $ tales que 
	\[\underbrace{a _{0} + \frac{a_{1}}{10} + \cdots + \frac{a_{n}}{10^{n}}}_{x_{n}} \leq x \leq a_{0} + \frac{a_{1}}{10} + \cdots + \frac{a_{n}+1}{10^{n}} .\]
Vamos a demostrar que $\displaystyle \lim_{n \to \infty}x_{n} = x $. En efecto, 
\[x_{n} \leq x < x_{n} + \frac{1}{10^{n}} .\]
Entonces, tenemos que
\[0 \leq \left|x - x_{n}\right| \leq \frac{1}{10^{n}} .\]
Tenemos que $\displaystyle 0 \to 0 $ y $\displaystyle \frac{1}{10^{n}} \to 0 $, por lo que $\displaystyle \left|x-x_{n}\right| \to 0 $, por lo que $\displaystyle x_{n} \to x $. 
\end{eg}

\begin{ftheorem}[]
	\normalfont Sean $\displaystyle \left\{ x_{n}\right\} _{n\in\N}, \left\{ x_{n}\right\} _{n\in\N}\subset\R $, tales que $\displaystyle x_{n} \to x $ y $\displaystyle y_{n}\to y $. 
	\begin{description}
	\item[(i)] $\displaystyle x_{n} + y _{n} \to x+y \iff \lim_{n \to \infty}\left(x_{n}+y_{n}\right) = \lim_{n \to \infty}x_{n} + \lim_{n \to \infty}y_{n}$.
	\item[(ii)] $\displaystyle x_{n}y_{n} \to xy \iff \lim_{n \to \infty}x_{n}y_{n} = \left(\lim_{n \to \infty}x_{n}\right)\left(\lim_{n \to \infty}y_{n}\right)$.
	\item[(iii)] Si $\displaystyle y_{n}\neq 0 $, $\displaystyle y\neq 0 $, 
		\[\frac{x_{n}}{y_{n}} \to \frac{x}{y} \iff \lim_{n \to \infty}\frac{x_{n}}{y_{n}} = \frac{\lim_{n \to \infty}x_{n}}{\lim_{n \to \infty}y_{n}} .\]
	\item[(iv)] $\displaystyle \left|x - x_{n}\right| \to 0 $.
	\end{description}
\end{ftheorem}

\begin{proof}
\begin{description}
	\item[(i)] Si $\displaystyle \epsilon > 0 $, $\displaystyle \exists n_{1} \in \N, \forall n \geq n_{1}, \; \left|x_{n}-x\right|<\frac{\epsilon }{2} $. Similarmente, existe $\displaystyle n_{2}\in\N, \forall n \geq n_{2}, \; \left|y_{n}-y\right|< \frac{\epsilon }{2} $. Tomamos $\displaystyle n_{0} = \max \left\{ n_{1}, n_{2}\right\}  $. 
\[
\begin{split}
 \left|x_{n}+y_{n} - \left(x+ y\right)\right| = \left|\left(x_{n}-x\right) + \left(y_{n}-y\right)\right| \leq \left|x_{n}-x\right| + \left|y_{n}-y\right| < \frac{\epsilon }{2} + \frac{\epsilon }{2} = \epsilon .
\end{split}
\]
\item[(ii)] 
	\[ \left|x_{n}y_{n} - xy\right| = \left|x_{n}y_{n} - x_{n}y + x_{n}y - xy\right| = \left|x_{n}\left(y_{n}-y\right) + y \left(x_{n}-x\right)\right| \leq \left|x_{n}\right| \left|y_{n}-y\right| + \left|y\right| \left|x_{n}-x\right| .\]
Cogemos $\displaystyle n_{1} \in \N $ tal que $\displaystyle \forall n \geq n_{1}, \; \left|x_{n}-x\right| < \left|x\right| $,
\[ \left|x_{n}\right| = \left|x_{n} - x + x\right| \leq \left|x_{n}-x\right| + \left|x\right| < 2 \left|x\right| .\]
Así, 
\[ \left|x_{n}y_{n} - xy\right| \leq 2 \left|x\right| \left|y_{n}-y\right| + \left|y\right| \left|x_{n}-x\right|.\]
Cogemos $\displaystyle n_{2}, n_{3} \in \N $ tales que $\displaystyle \forall n \geq n_{2} $ y $\displaystyle \forall n \geq n_{3} $, 
\[ \left|x_{n}-x\right| < \frac{ \epsilon}{2 \left|y\right| } \quad \text{y} \quad \left|y_{n}-y\right|<\frac{\epsilon }{4 \left|x\right|} .\]
Entonces, si cogemos $\displaystyle n_{0} = \max \left\{ n_{1}, n_{2}, n_{3}\right\}  $, tenemos que $\displaystyle \forall n \geq n_{0} $, 
\[ \left|x_{n}y_{n} - xy\right| \leq 2 \left|x\right| \left|y_{n}-y\right| + \left|y\right| \left|x_{n}-x\right| < \frac{\epsilon }{2} + \frac{\epsilon }{2 }=\epsilon  .\]

\item[(iii)] 
\[
\begin{split}
	\left|\frac{x_{n}}{y_{n}} - \frac{x}{y}\right|  = & \left| \frac{x_{n}y-xy_{n}}{y_{n}y}\right| = \left|\frac{y \left(x_{n}-x\right) - x \left(y_{n}-y\right)}{y_{n}y}\right|\\
	\leq & \frac{1}{ \left|y_{n}\right|} \frac{ \left|x_{n}-x\right|}{|y_{n}|} + \frac{1}{ \left|y_{n}\right|} \left|\frac{x}{y}\right| \left|y_{n}-y\right|.
\end{split}
\]
Si cogemos $\displaystyle n_{1} \in \N $ tal que $\displaystyle \forall n\geq n_{1} $, $\displaystyle \left|y_{n}-y\right| < \frac{ \left|y\right|}{2} $. Entonces tenemos que
\[ \left|y_{n}\right| = \left|y_{n}-y+y\right| \geq \left|y\right| - \left|y_{n}-y\right|> \frac{ \left|y\right|}{2} .\]
Entonces, 
\[ \left|\frac{x_{n}}{y_{n}} - \frac{x}{y}\right| \leq \frac{1}{ \left|y_{n}\right|} \frac{ \left|x_{n}-x\right|}{|y_{n}|} + \frac{1}{ \left|y_{n}\right|} \left|\frac{x}{y}\right| \left|y_{n}-y\right| < \frac{2}{ \left|y\right|} \left|x_{n}-x\right| + \left|\frac{2x}{y^{2}}\right| \left|y_{n}-y\right|.\]
Cogemos $\displaystyle n_{2}, n_{3}\in\N $ tal que $\displaystyle \forall n\geq n_{2} $ y $\displaystyle \forall n\geq n_{3} $ tenemos que 
\[ \left|x_{n}-x\right| < \frac{\epsilon \left|y\right|}{4} \quad \text{y} \quad \left|y_{n}-y\right| < \epsilon \left|\frac{y^{2}}{4x}\right| .\]
Si cogemos $\displaystyle n_{0} = \max \left\{ n_{1}, n_{2}, n_{3}\right\}  $, tenemos que $\displaystyle \forall n\geq n_{0} $, 
\[\left|\frac{x_{n}}{y_{n}} - \frac{x}{y}\right| \leq \frac{1}{ \left|y_{n}\right|} \frac{ \left|x_{n}-x\right|}{|y_{n}|} + \frac{1}{ \left|y_{n}\right|} \left|\frac{x}{y}\right| \left|y_{n}-y\right| < \frac{2}{ \left|y\right|} \left|x_{n}-x\right| + \left|\frac{2x}{y^{2}}\right| \left|y_{n}-y\right|<\frac{\epsilon }{2} + \frac{\epsilon }{2} = \epsilon .\]

\item[(iv)] Decir que $\displaystyle \left|x -x_{n}\right|\to 0 $ es lo mismo que decir que $\displaystyle \lim_{n \to \infty}x_{n} = 0 $. En efecto, si $\displaystyle \left|x -x_{n}\right|\to 0 $, tenemos que
\[\forall \epsilon > 0, \exists n_{0}\in\N, \forall n \geq n_{0}, \; \left|x_{n}-x\right|<\epsilon  .\]
Esta es la definición de $\displaystyle \lim_{n \to \infty}x_{n}=n $.

\end{description}
\end{proof}

\begin{fcolorary}[]
\normalfont 
\begin{description}
\item[(i)] Si cada una de estas sucesiones converge \footnote{Si la suma converge, las sucesiones individuales no tienen por qué converger. Considera $\displaystyle x_{n} = n $ y $\displaystyle y _{n} = -n $.} :
	\[\lim_{n \to \infty}\left(a_{n} + b_{n} + \cdots + z_{n}\right) = \lim_{n \to \infty}a_{n} + \cdots + \lim_{n \to \infty}z_{n} .\]
\item[(ii)] Si $\displaystyle k\in \N $,
	\[\lim_{n \to \infty} a_{n}^{k} = \left(\lim_{n \to \infty}a_{n}\right)^{k} .\]
\end{description}
\end{fcolorary}

\begin{ftheorem}[]
	\normalfont Si $\displaystyle \left\{ x_{n}\right\} _{n\in\N}\subset\R $, con $\displaystyle x_{n}\geq 0 $ y $\displaystyle \lim_{n \to \infty}x_{n} = x $, entonces, $\displaystyle x \geq 0 $. 
\end{ftheorem}

\begin{proof}
Supongamos que $\displaystyle x < 0 $. Sea $\displaystyle \epsilon = \frac{ \left|x\right|}{2} > 0 $. Sea $\displaystyle n_{0}\in\N $ tal que $\displaystyle \forall n \geq n_{0} $, $\displaystyle \left|x_{n}-x\right| < \epsilon  $.
Tenemos que
\[ \left|x_{n}-x\right| = x_{n}-x<\epsilon = \frac{ \left|x\right|}{2} = \frac{-x}{2} \Rightarrow x_{n} < \frac{x}{2} \iff x_{n}<\frac{x}{2} < 0 \iff x_{n} < 0, \forall n \geq n_{0} .\]
 Esto es una contradicción.
\end{proof}

\begin{fcolorary}[]
\normalfont Si $\displaystyle x_{n} \leq y_{n} $, $\displaystyle x_{n} \to x $ y $\displaystyle y_{n} \to y $. Entonces, $\displaystyle x \leq y $.
\end{fcolorary}

\begin{proof}
Si $\displaystyle y_{n}-x_{n}\geq 0 $, entonces $\displaystyle y_{n}-x_{n} \to y-x $. Por el teorema anterior tenemos que 
\[y - x \geq 0 \iff y \geq x .\]
\end{proof}

\begin{fcolorary}[]
\normalfont Si $\displaystyle x_{n}\to x $. Entonces, $\displaystyle \left|x_{n}\right|\to \left|x\right| $.\footnote{El recíproco no se cumple, comprueba $\displaystyle \left(-1\right)^{n} $.} 

\end{fcolorary}

\begin{proof}
Tenemos que 
\[0 \leq \left| \left|x_{n}\right| - \left|x\right|\right| \leq \underbrace{\left|x_{n}-x\right|}_{\to0} .\]
Entonces, $\displaystyle \left| \left|x_{n}\right| - \left|x\right|\right| \to 0 $.
\end{proof}

\begin{ftheorem}[]
	\normalfont Sea $\displaystyle \left\{ x_{n}\right\} _{n\in\N}\subset \R^{+} \footnote{ $\displaystyle \R^{+} = \left(0, \infty\right) $.}  $, y supongamos que $\displaystyle x_{n}\to x $ y $\displaystyle x > 0 $. Sea $\displaystyle m \in \N $, entonces 
	\[x_{n}^{\frac{1}{m}} \to x^{\frac{1}{m}} \iff \lim_{n \to \infty}x_{n}^{\frac{1}{m}} = \left(\lim_{n \to \infty}x_{n}\right)^{\frac{1}{m}}.\]
\footnote{Si aplicamos esta propiedad junto a la del exponente, lo podemos demostrar para $\displaystyle q \in \Q $.} 	
\end{ftheorem}

\begin{proof}
Sea $\displaystyle a = x^{\frac{1}{m}} \iff a^{m} = x $ y $\displaystyle a_{n} = x_{n}^{\frac{1}{m}} \iff a_{n}^{m} = x_{n} $. Sabemos que 
\[a^{m}-a_{n}^{m} = \left(a-a_{n}\right)\left(a^{m-1} + \cdots + a^{m-1}_{n}\right) .\]
\[ \Rightarrow \frac{a^{m}-a_{n}^{m}}{a^{m-1} + \cdots + a^{m-1}_{n}} = a - a_{n} .\]
Entonces tenemos que 
\[ \frac{ \left|a^{m}-a^{m}_{n}\right|}{a^{m-1}}\geq \frac{ \left|a^{m}-a_{n}^{m}\right|}{ \left|a^{m-1} + \cdots + a^{m-1}_{n}\right|} = \left|a-a_{n}\right| .\]
Por tanto, 
\[0 \leq \left|x^{\frac{1}{m}}-x_{n}^{\frac{1}{m}}\right|\leq \underbrace{\frac{1}{a^{m-1}} \left|x-x_{n}\right|}_{\to 0} .\]
Entonces, $\displaystyle \left|x^{\frac{1}{m}}-x_{n}^{\frac{1}{m}}\right| \to 0 $.
\end{proof}

\section{Criterios de Convergencia}

\begin{fprop}[]
\normalfont Si $\displaystyle 0 < r < 1 $, entonces $\displaystyle r^{n} \to 0 $.
\end{fprop}

\begin{proof}
Sea $\displaystyle R = \frac{1}{r} > 1 $. Queremos ver que $\displaystyle R^{n} \to \infty $. Como $\displaystyle R > 0 $, por la desigualdad de Bernouilli tenemos que
\[ R^{n} = \left(1 + x\right)^{n} \geq 1 + nx .\]
Por tanto, 
\[0 < r^{n} = \frac{1}{R^{n}} = \frac{1}{\left(1 +x\right)^{n}} \leq \frac{1}{1 + nx} < \frac{1}{nx} .\]
Como $\displaystyle \frac{1}{x} \to \frac{1}{x} $ y $\displaystyle \frac{1}{n} \to 0 $, tenemos que $\displaystyle \frac{1}{nx} \to 0 $. Entonces, $\displaystyle r^{n} \to 0 $.
\end{proof}

\begin{observation}
\normalfont En la demostración anterior hemos dicho que $\displaystyle r^{n} \to 0 $ es equivalente a decir que $\displaystyle R^{n} \to \infty $. Esto es porque la definición de la segunda será:
\[\forall c > 0, \exists n_{0}\in\N, \forall n \geq n_{0}, R^{n} > c .\]
Si $\displaystyle R^{n} > c $ tenemos que 
\[\frac{1}{r^{n}} > c \iff \frac{1}{r^{n}} < c .\]
Por tanto, se trata de la misma definición, solo que cambiamos $\displaystyle \epsilon  $  por $\displaystyle c $.
\end{observation}

\begin{eg}
\normalfont 
\begin{description}
\item[(i)] Si $\displaystyle c > 0 $, tenemos que $\displaystyle \lim_{n \to \infty}c^{\frac{1}{n}}=1 $. Si $\displaystyle c = 1 $ el resultado es trivial. Si $\displaystyle c > 1 $, tenemos que, $\displaystyle c^{\frac{1}{n}} > 1 $ (problema 18 de la hoja 2). Entonces, podemos encontrar $\displaystyle x_{n} \in \R $ tal que 
	\[ c^{\frac{1}{n}} = 1 + x_{n} \iff x_{n} = c^{\frac{1}{n}}-1 .\]
Entonces, utilizando la desigualdad de Bernuilli:
\[c = \left(1 + x_{n}\right)^{n} \geq 1 + nx_{n} \iff x_{n} \leq \frac{c-1}{n} .\]
Por tanto, tenemos que
\[ 0 < \left|c^{\frac{1}{n}}-1\right| = x_{n} \leq \frac{c-1}{n} .\]
Como $\displaystyle \frac{1}{n} \to 0 $ y $\displaystyle c-1 \to c-1 $, por el teorema 2.3 tenemos que $\displaystyle \frac{c-1}{n} \to 0 $. Entonces, 
\[ \left|c^{\frac{1}{n}}-1\right| \to 0 \iff c^{\frac{1}{n}} \to 1 .\]
Si $\displaystyle 0 < c < 1 $, tenemos que $\displaystyle c^{\frac{1}{n}} < 1 $. Entonces existe $\displaystyle x_{n} \in \R $ tal que 
\[ c^{\frac{1}{n}} = \frac{1}{1 + x_{n}} .\]
Utilizando la identidad de Bernuilli:
\[ c = \frac{1}{\left(1 + x_{n}\right)^{n}} \leq \frac{1}{1 + nx_{n}} < \frac{1}{nx_{n}} .\]
Entonces tenemos que, 
\[ \left|c^{\frac{1}{n}}-1\right| = \frac{x_{n}}{1 + x_{n}} < x_{n} < \left(\frac{1}{c}\right)\frac{1}{n} .\]
Utilizando el teorema 2.3, tenemos que $\displaystyle \frac{1}{nc} \to 0 $ y, consecuentemente, $\displaystyle c^{\frac{1}{n}} \to 1 $.
\item[(ii)] Similarmente, se tiene que $\displaystyle \lim_{n \to \infty}n^{\frac{1}{n}} = 1 $. Sabemos que $\displaystyle n^{\frac{1}{n}} > 1 $ para $\displaystyle n > 1 $. Entonces podemos escribir
	\[n^{\frac{1}{n}} = 1 + x_{n} .\]
Entonces, aplicando el teorema del binomio tenemos que 
\[n = \left(1 + x_{n}\right)^{n} = 1 + nk_{n}+\frac{1}{2}n\left(n-1\right)x_{n}^{2} + \cdots \geq 1 + \frac{1}{2}n\left(n-1\right)x_{n}^{2} .\]
Por tanto, tenemos que $\displaystyle x_{n}^{2} \leq \frac{2}{n} $. Entonces podemos decir que
\[ 0 < \left|n^{\frac{1}{n}}-1\right| = x_{n} \leq \left(\frac{2}{n}\right)^{\frac{1}{2}} .\]
Por el teorema 2.5 tenemos que $\displaystyle \left(\frac{1}{n}\right)^{\frac{1}{2}} \to 0^{\frac{1}{2}} = 0 $. Entonces, $\displaystyle \left(\frac{2}{n}\right)^{\frac{1}{2}} \to 0 $ y $\displaystyle n^{\frac{1}{n}} \to 1 $.
\end{description}
\end{eg}

\begin{ftheorem}[Regla del cociente]
	\normalfont Si $\displaystyle \left\{ x_{n}\right\} _{n\in\N} \subset \left(0 , \infty\right)$ y existe $\displaystyle \lim_{n \to \infty} \frac{x_{n+1}}{x_{n}} < 1 $. Entonces, $\displaystyle \lim_{n \to \infty} x_{n} = 0  $. 
\end{ftheorem}

\begin{proof}
Sea $\displaystyle l = \lim_{n \to \infty}\frac{x_{n+1}}{x_{n}} $. Como se trata de una sucesión de términos positivos, tenemos que $\displaystyle l \in \left[0,1\right) $ \footnote{Los límites no conservan estrictamente las desigualdades. Considera la sucesión $\displaystyle x_{n} = \frac{1}{n} $. Tenemos que cada elemento es mayor que 0 pero el límite es 0.}. Sea $\displaystyle \epsilon > 0 $, entonces existe $\displaystyle n_{0}\in\N $ tal que si $\displaystyle n \geq n_{0} $ 
\[ \left|\frac{x_{n+1}}{x_{n}}-l\right| < \epsilon \iff -\epsilon + l< \frac{x_{n+1}}{x_{n}} < \epsilon + l .\]
Por lo tanto, 
\[x_{n+1} < \left(l + \epsilon \right)x_{n} .\]
Consecuentemente, 
\[ x_{n+1} < \left(l + \epsilon \right)x_{n} < \left(l+\epsilon \right)^{2}x_{n-1} < \cdots < \left(l+\epsilon \right)^{n+1-n_{0}}x_{n_{0}}.\]
Sea $\displaystyle \epsilon > 0 $ tal que $\displaystyle l + \epsilon < 1 $, por ejemplo, cogemos $\displaystyle \epsilon = \frac{1 - l}{2}$. Tomamos $\displaystyle r = l + \epsilon < 1 $ y obtenemos que si $\displaystyle n \geq n_{0} $ 
\[ x_{n+1} < r^{n+1}\frac{x_{n_{0}}}{r^{n_{0}}} .\]
Por la proposición 2.4 tenemos que $\displaystyle r^{n+1} \to 0 $, por lo que $\displaystyle x_{n+1} \to 0 $ (aplicando la regla del bocadillo).
\end{proof}

\begin{observation}
\normalfont Si $\displaystyle l = 1 $ el resutlado puede ser falso, considera el caso $\displaystyle x_{n} = n $. Tenemos que
\[\frac{x_{n+1}}{x}=\frac{n+1}{n} = 1 + \frac{1}{n} \to 1 , \]
pero 
\[ \lim_{n \to \infty} n = \infty .\]
\end{observation}

\begin{eg}
\normalfont Consideramos la sucesión $\displaystyle x_{n} = \frac{n}{2^{n}} $. Aplicamos la regla del cociente. 
\[ \frac{x_{n+1}}{x_{n}} = \frac{\frac{n+1}{2^{n+1}}}{\frac{n}{2^{n}}} = \frac{n+1}{n} \cdot \frac{1}{2} \to \frac{1}{2} < 1 .\]
Entonces, $\displaystyle x_{n} \to 0 $.
\end{eg}

\begin{fdefinition}[Monotonía]
	\normalfont Sea $\displaystyle \left\{ x_{n}\right\} _{n\in\N} \subset \R$. Se dice que 
	\begin{description}
		\item[(i)] $\displaystyle \left\{ x_{n}\right\} _{n\in\N} $ es \textbf{creciente} si $\displaystyle x_{n} \leq x_{n+1} $ \footnote{Es estrictamente creciente si $\displaystyle x_{n} < x_{n+1} $. También funciona así si es estrictamente decreciente.} .
		\item[(ii)] $\displaystyle \left\{ x_{n}\right\} _{n\in\N} $ es \textbf{decreciente} si $\displaystyle x_{n}\geq x_{n+1} $.
		\item[(iii)] $\displaystyle \left\{ x_{n}\right\} _{n\in\N} $ es \textbf{monótona} si es creciente o decreciente.
	\end{description}
\end{fdefinition}

\begin{ftheorem}[]
	\normalfont Sea $\displaystyle \left\{ x_{n}\right\} _{n\in\N} \subset \R$. 
	\begin{description}
		\item[(i)] Si $\displaystyle \left\{ x_{n}\right\} _{n\in\N} $ es creciente y está acotada superiormente, entonces 
			\[\lim_{n \to \infty}x_{n} = \sup \left\{ x_{n}\right\} _{n\in\N} .\]
		\item[(ii)] Si $\displaystyle \left\{ x_{n}\right\} _{n\in\N} $ es decreciente y está acotada inferiormente, entonces
			\[\lim_{n \to \infty}x_{n} = \inf \left\{ x_{n}\right\} _{n\in\N} .\]	
	\end{description}
\footnote{Lo importante de este teorema es que si una sucesión está acotada y es monótona, entonces existe el límite.} 
\end{ftheorem}

\begin{proof}
\begin{description}
	\item[(i)] Sea $\displaystyle s = \sup \left\{ x_{n}\right\} _{n\in\N} < \infty $. Sea $\displaystyle \epsilon > 0 $, queremos ver que existe $\displaystyle n_{0} \in \N $ tal que $\displaystyle s - x_{n} < \epsilon, \forall n\geq n_{0} $. Como $\displaystyle s = \sup \left\{ x_{n}\right\} _{n\in\N} $, existe $\displaystyle n_{0} \in \N $ tal que 
		\[s - \epsilon < x_{n_{0}} .\]
Además, como se trata de una sucesión creciente, tenemos que si $\displaystyle n \geq n_{0} $, 
\[ x_{n} \geq x_{n_{0}} > s - \epsilon \iff s-x_{n} < \epsilon  .\]
\item[(ii)] Se puede demostrar de dos formas: una es similar a la anterior, mientras que la otra se parece a la demostración de la existencia del ínfimo en casos de la existencia de una cota inferior. Comenzamos con la primera. Sea $\displaystyle i = \inf \left\{ x_{n}\right\} _{n\in\N} $. Como $\displaystyle i $ es el ínfimo, sabemos que si $\displaystyle \epsilon > 0 $, existe un $\displaystyle n_{0} \in \N $ tal que 
	\[x_{n_{0}} < i + \epsilon  .\]
Además, como la sucesión es decreciente tenemos que si $\displaystyle n \geq 0 $ 
\[x_{n} < x_{n_{0}} < i + \epsilon  .\]
Entonces, si $\displaystyle n \geq n_{0} $ tenemos que
\[ x_{n}-i < \epsilon  .\]
Por tanto, $\displaystyle \lim_{n \to \infty}x_{n} = \inf \left\{ x_{n}\right\} _{n\in\N} $. \\ \\
La otra demostración comienza diciendo que si $\displaystyle \left\{ x_{n}\right\} _{n\in\N} $ es una sucesión decreciente acotada inferiormente, entonces la sucesión $\displaystyle \left\{ -x_{n}\right\} _{n\in\N} $ es una sucesión creciente acotada superiormente. Por un teorema que vimos en el capítulo anterior, sabemos que $\displaystyle \sup \left\{ -x_{n}\right\} _{n\in\N} = -\inf \left\{ x_{n}\right\} _{n\in\N} $. Aplicando el apartado anterior tenemos que 
\[\lim_{n \to \infty} x_{n} =- \lim_{n \to \infty}\left(-x_{n}\right) = \inf \left\{ x_{n}\right\} _{n\in\N} .\]

\end{description}
\end{proof}

\begin{eg}
\normalfont 
\begin{description}
\item[(i)] 
	\[\frac{n+1}{n} = 1 + \frac{1}{n} .\]
Esta sucesión decrece. Además, $\displaystyle 1+ \frac{1}{n} \geq 1 $, es decir, está acotada inferiormente. Entonces, 
\[\lim_{n \to \infty} \left(1 + \frac{1}{n}\right) = \inf \left\{ 1 + \frac{1}{n}\right\}_{n\in\N} = 1 + \inf \left\{ \frac{1}{n}\right\}_{n\in\N} = 1 .\]
\item[(ii)] $\displaystyle x_{1} = \frac{1}{2} $, $\displaystyle x_{n} = x_{n-1} + \frac{1}{2^{n}}, \; n\geq 2$. La sucesión va a ser de la forma
	\[\frac{1}{2}, \; \frac{3}{4}, \; \frac{7}{8}, \; \ldots .\]
Tenemos que el término genera será $\displaystyle x_{n} = \frac{1}{2} + \cdots + \frac{1}{2^{n}} = \frac{\frac{1}{2}-\frac{1}{2^{n+1}}}{1-\frac{1}{2}} \to 1 $, pues $\displaystyle \frac{1}{2^{n+1}}\to 0 $. 
\begin{observation}
\normalfont Si $\displaystyle r > 0 $ y $\displaystyle m > n $, 
\[\sum^{m}_{j=n}r^{j} = \frac{r^{n}-r^{m+1}}{1-r} .\]
\end{observation}
\item[(iii)] $\displaystyle x_{n} = 1 + \frac{1}{2} + \cdots + \frac{1}{n} $. Esta serie diverge. Sabemos que $\displaystyle \left\{ x_{n}\right\} _{n\in\N} $ es creciente. Si $\displaystyle \lim_{n \to \infty}x_{n} $ existe, entonces estaría acotada superioremnte. Nos vamos a fijar en los términos $\displaystyle x_{2^{n}} $. Tenemos que 
	\[x_{2^{n}} = 1 + \frac{1}{2} + \left(\frac{1}{3} + \frac{1}{4}\right) + \cdots + \left(\frac{1}{2^{n-1}+1} + \cdots + \frac{1}{2^{n}}\right) .\]
Tenemos que en el último paréntesis hay $\displaystyle 2^{n}-\left(2^{n-1}+1\right) + 1 = 2^{n-1} $ elementos. Entonces tenemos que
\[ x_{2^{n}} > 1 + \frac{1}{2} + 2 \cdot \frac{1}{4} + 4 \cdot \frac{1}{8} + \cdots + 2^{n-1} \cdot \frac{1}{2^{n}} = 1 + \frac{1}{2} + \frac{1}{2} + \cdots + \frac{1}{2} = 1 + \frac{n}{2} \to \infty .\]
Entonces, esta sucesión diverge y, por tanto, $\displaystyle \left\{ x_{n}\right\} _{n\in\N} $ no puede estar acotada superiormente. Entonces, la sucesión $\displaystyle \left\{ x_{n}\right\} _{n\in\N} $ es divergente.
\item[(iv)] Número de Euler. 
	\[e_{n} = \left(1 + \frac{1}{n}\right)^{n} .\]
Vamos a demostrar que es creciente y que está acotada superiormente. En efecto, 
\[e_{n} = \left(1 + \frac{1}{n}\right)^{n} = \sum^{n}_{i = 0}\begin{pmatrix} n \\ i \end{pmatrix} \frac{1}{n^{i}}.\]
\begin{observation}
\normalfont 
\[
\begin{split}
	\begin{pmatrix} n \\ i \end{pmatrix} \frac{1}{n^{i}} & = \frac{n\left(n-1\right) \cdots 1}{i ! \left(n-i\right)! n^{i}} 
							     = \frac{n\left(n-1\right) \cdots \left(n-i+1\right)}{i!n^{i}} 
							     = \frac{1}{i!} \cdot 1 \cdot \frac{n-1}{n} \cdots \frac{n-i+1}{n}\\
							     &= \frac{1}{i!} \left(1-\frac{1}{n}\right)\left(1 - \frac{2}{n}\right) \cdots \left(1-\frac{i-1}{n}\right) .
\end{split}
\]
\end{observation}
A partir de la observación 2.8, podemos ver que 
\[\begin{pmatrix} n \\ i \end{pmatrix} \frac{1}{n^{i}} = \frac{1}{i!} \left(1-\frac{1}{n}\right) \cdots \left(1 - \frac{i-1}{n}\right) < \frac{1}{i!}\left(1-\frac{1}{n+1}\right) \cdots \left(1 - \frac{i-1}{n+1}\right) .\]
Por lo tanto, 
\[e_{n+1} = \sum^{n+1}_{i = 0}\begin{pmatrix} n+1 \\ i \end{pmatrix} \frac{1}{\left(n+1\right)^{i}} = \sum^{n}_{i=0}\begin{pmatrix} n+1 \\ i \end{pmatrix}\frac{1}{\left(n+1\right)^{i}} + \frac{1}{\left(n+1\right)^{n+1}} > \sum^{n}_{i=0}\begin{pmatrix} n \\ i \end{pmatrix}\frac{1}{n^{i}} = e_{n}.\]
Por tanto, $\displaystyle e_{n} < e_{n+1} $, por lo que la sucesión $\displaystyle e_{n} $ es creciente. Ahora tenemos que ver que está acotada, en concreto, que $\displaystyle \forall n \in \N $, $\displaystyle e_{n} \leq 3 $. Usamos que $\displaystyle 2^{j-1} \leq j! $. Partiendo de la observación 2.8, tenemos que 
\[
\begin{split}
\begin{pmatrix} n \\ i \end{pmatrix} \frac{1}{n^{i}} < \frac{1}{i!} \leq \frac{1}{2^{i-1}} .
\end{split}
\]
De esta manera, 
\[e_{n} \leq 1 + \sum^{n}_{i = 1} \frac{1}{2^{i-1}} = 1 + \sum^{n-1}_{i = 0}\frac{1}{2^{i}}\leq 1 + 2 = 3 .\]
Por tanto, la sucesión va a converger a $\displaystyle e $, es decir, 
\[e = \lim_{n \to \infty}\left(1 + \frac{1}{n}\right)^{n} .\]
\end{description}
\end{eg}


\begin{eg}
\normalfont Si $\displaystyle r > 0 $ y $\displaystyle x_{n} = r^{\frac{1}{n}} $ entonces $\displaystyle x_{n}\to 1 $. Consideramos varios casos. Si $\displaystyle r = 1 $, es trivial. Si $\displaystyle r > 1 $, entonces vamos a ver si es creciente o decreciente. Tenemos que
\[ r > 1 \iff r^{n+1} > r^{n} \iff r^{\frac{1}{n}} > r^{\frac{1}{n+1}} .\]
Es decir, la sucesión decrece y, por tanto, decrece. Como $\displaystyle r > 1 $, entonces, $\displaystyle x_{n} = r > 1 $. Entonces la sucesión decreciente $\displaystyle \left\{ x_{n}\right\} _{n\in\N}$ está acotada inferiormente por 1. Consecuentemente, la sucesión converge. Vamos a demostrar que el límite es 1. Sea $\displaystyle x = \lim_{n \to \infty}r^{\frac{1}{n}} $. Como, $\displaystyle x_{n} > 1 $, tenemos que $\displaystyle x \geq 1 $. Supongamos que $\displaystyle x > 1 $. Tenemos que
\[x_{n} = r^{\frac{1}{n}} \geq x \Rightarrow r \geq x^{n} .\]
Si $\displaystyle n \to \infty $ tenemos que $\displaystyle x^{n} $ diverge, por lo que $\displaystyle r  = \infty $. Esto es una contradicción, por lo que debe ser que $\displaystyle \lim_{n \to \infty}x_{n} = 1 $. Ahora, supongamos que $\displaystyle 0 < r < 1 $.
\end{eg}

\section{Subsucesiones y Teorema de Bolzano-Weiestrass}

\begin{fdefinition}[]
	\normalfont Dada una sucesión $\displaystyle \left\{ x_{n}\right\} _{n\in\N}\subset \R $ y una sucesión creciente estrictamente de números naturales: $\displaystyle n_{1} < n_{2} < \ldots < n_{j}< n_{j+1} < \cdots  $. Se dice que $\displaystyle \left\{ x_{n_{j}}\right\} _{j\in\N} $ es una \textbf{subsucesión} (o \textbf{parcial}) de $\displaystyle \left\{ x_{n}\right\} _{n\in\N} $.
\end{fdefinition}

\begin{eg}
\normalfont Si $\displaystyle x_{n} = \left(-1\right)^{n} $. Una subsucesión sería $\displaystyle x_{2n} = 1 $ y otra sería $\displaystyle x_{2n -1} = -1 $.
\end{eg}

\begin{ftheorem}[]
	\normalfont Si $\displaystyle \left\{ x_{n}\right\} _{n\in\N}\subset\R $ sucesión convergente con $\displaystyle x = \lim_{n \to \infty}x_{n} $, y $\displaystyle \left\{ x_{n_{j}}\right\}_{j\in\N}  $ es una subsucesión, entonces $\displaystyle \lim_{j \to \infty}x_{n_{j}} = x $. 
\end{ftheorem}

\begin{proof}
Sea $\displaystyle \epsilon > 0 $, existe $\displaystyle n_{0} \in \N $ tal que si $\displaystyle n \geq n_{0} $ entonces $\displaystyle \left|x_{n}-x\right| < \epsilon  $. Si $\displaystyle n_{j} \geq n_{0} $ entonces $\displaystyle \left|x_{n_{j}}-x\right| < \epsilon  $.
\end{proof}

\begin{observation}
\normalfont El teorema anterior sirve para ver que algo no converge. 
\end{observation}

\begin{eg}
\normalfont 
\[x_{n} = 
\begin{cases}
0, \; n \; \text{par}\\
n, \; n \; \text{impar}
\end{cases}
.\]
Tenemos que $\displaystyle x_{2n-1} = 2n-1 $ no converge, por lo que $\displaystyle \left\{ x_{n}\right\} _{n\in\N} $ no converge.
\end{eg}

\begin{ftheorem}[]
	\normalfont Sea $\displaystyle \left\{ x_{n}\right\} _{n\in\N}\subset\R $. Son equivalentes:
	\begin{description}
		\item[(i)] $\displaystyle \left\{ x_{n}\right\} _{n\in\N} $ no converge a $\displaystyle x $.
		\item[(ii)] $\displaystyle \exists \epsilon > 0, \forall n_{0} \in \N, \exists n\geq n_{0}, \; \left|x_{n}-x\right| \geq \epsilon  $.
		\item[(iii)] $\displaystyle \exists \epsilon > 0 $ y $\displaystyle \exists \left\{ x_{n_{j}}\right\} _{j\in\N}\subset \left\{ x_{n}\right\} _{n\in\N} $ tal que $\displaystyle \left|x_{n_{j}}-x\right| \geq \epsilon , \; \forall j \in \N $.
	\end{description}
\end{ftheorem}

\begin{proof}
\begin{description}
\item[(i)] Tenemos que \textbf{(i)} $\displaystyle  \iff  $ \textbf{(ii)} son equivalentes por definición. 
\item[(ii)] Vamos a ver que \textbf{(iii)} $\displaystyle \Rightarrow  $ \textbf{(i)}. Como $\displaystyle \left\{ x_{n_{j}}\right\} _{j\in\N} $ no converge a $\displaystyle x $, tenemos que la sucesión $\displaystyle \left\{ x_{n}\right\} _{n\in\N} $ no converge a $\displaystyle x $.
\item[(iii)] Vamos a ver que \textbf{(ii)} $\displaystyle  \Rightarrow  $ \textbf{(iii)}. Dado $\displaystyle \epsilon > 0 $ de \textbf{(ii)}, sea $\displaystyle n_{1} \geq n_{0} $  tal que $\displaystyle \left|x_{n_{1}}-x\right| \geq \epsilon  $. Tomamos ahora $\displaystyle n_{2} \geq n_{1}+1 > n_{1} $ tal que $\displaystyle \left|x_{n_{2}}-x\right| \geq \epsilon  $. Iterando, obtenemos una colección de índices estrictamente crecientes: $\displaystyle n_{1} < n_{2} < \cdots < n_{j} < \cdots  $, de manera que, 
	\[ \left| x_{n_{j}} - x\right| \geq \epsilon  .\]
\end{description}
\end{proof}

\begin{eg}
	\normalfont Sea $\displaystyle x_{n} = \sin n $. Vamos a demostrar que no existe el límite. Supongamos que $\displaystyle x \in \bigcup_{k=0}^{\infty}\left[2k\pi, \left(2k+1\right)\pi\right]  $. Entonces $\displaystyle \sin x \geq 0 $. Como la longitud de los intervalos es mayor que uno, sabemos que hay un natural. Por tanto podemos decir que 
	\[n_{k} \in \left[2k\pi, \left(2k+1\right)\pi\right], \; n_{k} \in \N  .\]
	Como los intervalos son disjuntos, tenemos que $\displaystyle n_{1} < n_{2} < \cdots < n_{k} < \cdots  $. Si $\displaystyle \lim_{n \to \infty}\sin n = l$, existe $\displaystyle \lim_{k \to \infty} \sin n_{k} = l \geq 0 $. Ahora consideramos el caso $\displaystyle x < 0 $, para ello, tomamos $\displaystyle 0 < \epsilon < \frac{1}{2} $. Sea $\displaystyle I_{\epsilon } = \left[\pi+\epsilon, 2\pi-\epsilon \right]  $. Si $\displaystyle x \in I_{\epsilon } $, tenemos que 
\[\displaystyle \left|\sin x\right| \geq \left|\sin (\pi + \epsilon)\right| > 0. \]
	Entonces, la longitud del intervalo es estrictamente mayor que 1. Por tanto, 
\[\displaystyle \forall k \in \N, \exists m_{k} \in  \left[\left(2k+1\right)-\epsilon, \left(2k+2\right)\pi-\epsilon\right]  \]
	tal que $\displaystyle \sin m_{k} \leq \sin [\left(2k+1\right)\pi+\epsilon] $. Si $\displaystyle \exists \lim_{n \to \infty}\sin n $, existe el límite de esta subsucesión, es decir, $\displaystyle \exists l =\lim_{k \to \infty}\sin m_{k} \leq \sin \left(\epsilon + \pi\right) < 0 $. Por tanto, tenemos que $\displaystyle l \geq 0$ y $\displaystyle l < 0 $. Esto es una contradicción. 
\end{eg}

\begin{ftheorem}[Teorema de la sucesión monótona]
	\normalfont Sea $\displaystyle \left\{ x_{n}\right\} _{n\in\N}\subset\R $. Entonces $\displaystyle \exists \left\{ x_{n_{j}}\right\}_{j\in\N}\subset \left\{ x_{n}\right\} _{n\in\N} $ que es monótona.
\end{ftheorem}

\begin{proof}
	Diremos que $\displaystyle x_{m} $ es un \textbf{pico} de la sucesión si $\displaystyle x_{m} \geq x_{n} $, $\displaystyle \forall n \geq m $. Supongamos que en $\displaystyle \left\{ x_{n}\right\} _{n\in\N} $ hay infinitos picos. Entonces podemos crear una subsucesión descreciente (monótona), $\displaystyle \left\{ x_{m_{j}}\right\} _{j\in\N} $ con los picos:
	\[x_{m_{1}} < x_{m_{2}} < \ldots < x_{m_{j}} < \cdots .\]
	Si hay un número finito de picos, podemos ir hasta el último pico tal que a partir de este no hay más picos. Supongamos que no hay picos. Es decir, sea $\displaystyle x_{n_{1}} \in \left\{ x_{n}\right\} _{n\in\N} $ tal que $\displaystyle \forall n \geq n_{1} $, $\displaystyle x_{n} $ no es un pico. Por tanto, $\displaystyle \exists n_{2} > n_{1} $ tal que $\displaystyle x_{n_{2}} > x_{n_{1}} $. Iterando, 
	\[\exists n_{1} < n_{2} < \cdots < n_{j} < \cdots, \]
tales que
\[x_{n_{1}} < x_{n_{2}} < \cdots < x_{n_{j}} < \cdots .\]
Por tanto, la subsucesión $\displaystyle \left\{ x_{n_{j}}\right\} _{j\in\N} $ es creciente y, por tanto, monótona. 
\end{proof}

\begin{ftheorem}[Teorema de Bolzano-Weiestrass]
	\normalfont Dada $\displaystyle \left\{ x_{n}\right\} _{n\in\N}\subset\R $ acotada entonces $\displaystyle \exists \left\{ x_{n_{j}}\right\} _{j\in\N} $ subsucesión convergente.
\end{ftheorem}

\begin{proof}
	Por el teorema de la sucesión monótona, sabemos que existe $\displaystyle \left\{ x_{n_{j}}\right\} _{j\in\N} $ subsucesión es monótona. Además, como la sucesión está acotada, la subsucesión también lo está y, por tanto, $\displaystyle \exists \lim_{n \to \infty}x_{n_{j}} $.
\end{proof}

\begin{ftheorem}[]
	\normalfont Sea $\displaystyle \left\{ x_{n}\right\} _{n\in\N}\subset\R $ sucesión acotada y $\displaystyle x \in \R $. Si para toda $\displaystyle \left\{ x_{n_{j}}\right\} _{j\in\N} $ subsucesión que converge, lo hace a $\displaystyle x $, entonces $\displaystyle \lim_{n \to \infty}x_{n} = x $.
\end{ftheorem}

\begin{proof}
	Sea $\displaystyle \left|x_{n}\right| \leq M, \; \forall n \in \N $ y supongamos que para esta sucesión no existe el límite (por lo que no es igual a $\displaystyle x $). Así, $\displaystyle \exists \epsilon > 0 $ tal que $\displaystyle \exists \left\{ x_{n_{j}}\right\}_{j\in\N}  $ con $\displaystyle \left|x_{n_{j}}-x\right|\geq \epsilon, \forall j \in \N  $. Por hipótesis esta subsucesión esta acotada. Aplicando el teorema de Bolzano-Weiestrass, $\displaystyle \exists \left\{ y_{l}\right\} _{l\in\N} \subset \left\{ x_{n_{j}}\right\} _{j\in\N}$ subsucesión convergente. Entonces tenemos que esta subsucesión converge a $\displaystyle x $. Sin embargo, tenemos que 
	\[ \left|y_{l}-x\right| \geq \epsilon, \; l \geq l_{0} .\]
Esto implica que $\displaystyle y_{l} $ no converge a $\displaystyle x $. Esto es una contradicción, por lo que debe ser que $\displaystyle \lim_{n \to \infty}x_{n} = x $.
\end{proof}

\begin{eg}
	\normalfont Considera
\[x_{n} = 
\begin{cases}
n, \; n \; \text{par} \\
0, \; n \; \text{impar} 
\end{cases}
.\]
Satisface que toda subsucesión convergente tiene como límite $\displaystyle 0 $ pero la sucesión no converge (observamos que no es una sucesión acotada).
\end{eg}

\begin{eg}
\normalfont Estudiamos la convergencia de $\displaystyle x_{n} = n^{\frac{1}{n}} $. Vamos a ver que es decreciente (monótona). Tenemos que
\[x_{n+1} = \left(n+1\right)^{\frac{1}{n+1}} < x_{n} = n^{\frac{1}{n}} \iff \left(n+1\right)^{n} < n^{n+1} = n \cdot n^{n} \iff \left(\frac{n+1}{n}\right)^{n} = \left(1+\frac{1}{n}\right)^{n} < n .\]
Tenemos que
\[ \left(1 + \frac{1}{n}\right)^{n} < e \leq 3 .\]
Entonces tenemos que si $\displaystyle n\geq 4 $ (recordamos que no nos importan los primeros términos de la sucesión),
\[ \left(1 +\frac{1}{n}\right)^{n} < n .\]
Por tanto, si $\displaystyle n\geq 4 $ la sucesión es decreciente. Además, tenemos que está acotada inferioremnte por 1. Por tanto, la sucesión converge. Tenemos que
\[n^{\frac{1}{n}} \to l \geq 1 .\]
Como esta sucesión converge, sabemos que todas las subsucesiones han de converger al mismo límite, por lo que basta con calcular la convergencia de una subsucesión. Vamos a ver que
\[x_{2n} = \left(2n\right)^{\frac{1}{2n}} \to 1.\]
Tenemos que
\[x_{2n} = \left(2n\right)^{\frac{1}{2n}} = 2^{\frac{1}{2n}} \cdot n^{\frac{1}{2n}} = \left(\sqrt{2}\right)^{\frac{1}{n}} \cdot \left(n^{\frac{1}{n}}\right)^{\frac{1}{2}} .\]
Sabemos que $\displaystyle \left(\sqrt{2}\right)^{\frac{1}{n}} \to 1 $ y $\displaystyle \left(n^{\frac{1}{n}}\right)^{\frac{1}{2}} \to \sqrt{l} $. Entonces tenemos que, 
\[l = \sqrt{l} \iff l^{2} = l \iff l \in \left\{ 0,1\right\}  .\]
Podemos descartar 0 porque $\displaystyle l \geq 1 $, por tanto, $\displaystyle l = 1 $.
\end{eg}

\section{Sucesiones Cauchy}

\begin{fdefinition}[Sucesión de Cauchy]
	\normalfont Sea $\displaystyle \left\{ x_{n}\right\} _{n\in\N}\subset \R $. Se dice que $\displaystyle \left\{ x_{n}\right\} _{n\in\N} $ es una sucesión de \textbf{Cauchy} si $\displaystyle \forall \epsilon > 0, \exists n_{0} \in \N $ tal que $\displaystyle \left|x_{n}-x_{m}\right|<\epsilon  $, con $\displaystyle n,m \geq n_{0} $.
\end{fdefinition}

\begin{fprop}[]
	\normalfont Sea $\displaystyle \left\{ x_{n}\right\} _{n\in\N}\subset\R $. 
\begin{description}
	\item[(i)] Si $\displaystyle \lim_{n \to \infty}x_{n} = l $, entonces $\displaystyle \left\{ x_{n}\right\} _{n\in\N} $ es de Cauchy.
	\item[(ii)] Si $\displaystyle \left\{ x_{n}\right\} _{n\in\N} $ es de Cauchy entonces es una sucesión acotada.
\end{description}
\end{fprop}

\begin{proof}
\begin{description}
\item[(i)] Sabemos que $\displaystyle \forall \epsilon > 0, \exists n_{0} \in \N $  tal que $\displaystyle \left|x_{n}-l\right| < \frac{\epsilon }{2} $. Así, cogemos $\displaystyle m,n \geq n_{0} $. Entonces, 
	\[ \left|x_{n} - l\right|<\frac{\epsilon }{2}.\]
	\[ \left|x_{m}-l\right| < \frac{\epsilon }{2} .\]
Finalmente, 
\[
\begin{split}
	\left|x_{n}-x_{m}\right| = & \left|x_{n}-l+x_{m}-l\right| \\
	\leq & \left|x_{n}-l\right| + \left|l-x_{m}\right| \\
	< & \frac{\epsilon }{2} + \frac{\epsilon }{2} = \epsilon .
\end{split}
\]
\item[(ii)] Cogemos $\displaystyle \epsilon = 1 $, entonces existe $\displaystyle n_{0} \in \N $ tal que si $\displaystyle m,n \geq n_{0} $, $\displaystyle \left|x_{n}-x_{m}\right| < 1 $. Cogemos $\displaystyle m = n_{0} $ y $\displaystyle n\geq n_{0} $. Tenemos que
	\[ \left|x_{n}-x_{n_{0}}\right| < 1 .\]
Entonces, 
\[ \left|x_{n}\right| = \left|x_{n}-x_{n_{0}}+x_{n_{0}}\right| \leq \left|x_{n}-x_{n_{0}}\right| + \left|x_{n_{0}}\right| < 1 + \left|x_{n_{0}}\right| .\]
Si $\displaystyle n\geq n_{0} $, la sucesión está acotada por $\displaystyle 1 + \left|x_{n_{0}}\right| $, nos queda por acotar los $\displaystyle n < n_{0} $. 
\[ \left|x_{n}\right| \leq \max \left\{ \max \left\{ \left|x_{k}\right|\; : \; 1 \leq k\leq n_{0}\right\}, 1 + \left|x_{n_{0}}\right| \right\}  .\]
\end{description}
\end{proof}

\begin{observation}
\normalfont Vamos a concluir que ser de Cauchy es equivalente a converger. Esto se cumple en $\displaystyle \R $ pero no en $\displaystyle \Q $! Entonces, podemos pensar que este hecho está relacionado con el axioma de completitud.
\end{observation}

\begin{ftheorem}[Criterio de convergencia de Cauchy]
	\normalfont Una sucesión $\displaystyle \left\{ x_{n}\right\} _{n\in\N} \subset \R$ converge si y solo si $\displaystyle \left\{ x_{n}\right\} _{n\in\N} $ es de Cauchy.
\end{ftheorem}

\begin{proof}
	Basta probar que si $\displaystyle \left\{ x_{n}\right\} _{n\in\N} \subset \R$ es de Cauchy, entonces converge. Por el teorema de Bolzano-Weiestrass, como es una sucesión de Cauchy está acotada, por lo que existe $\displaystyle \left\{ x_{n_{k}}\right\} _{k\in\N} \subset \left\{ x_{n}\right\} _{n\in\N} $ convergente a $\displaystyle l $. Sabemos que $\displaystyle \forall \epsilon > 0, \exists n_{1} \in \N $ tal que si $\displaystyle n, m \geq n_{1} $,
	\[ \left|x_{n}-x_{m}\right| < \frac{\epsilon }{2} .\]
Además, $\displaystyle \exists n_{2} \in \N $ tal que si $\displaystyle n_{k} \geq n_{2} $,
\[ \left|x_{n_{2}}-l\right| < \frac{\epsilon }{2} .\]
Sea $\displaystyle n_{0} = \max \left\{ n_{1}, n_{2}\right\}  $, si $\displaystyle n \geq n_{0} $, queremos ver que
\[ \left|x_{n}-l\right|<\epsilon.\]
Sea $\displaystyle n_{k} \geq n_{0} $ y $\displaystyle n \geq n_{0} $,
\[ \left|x_{n}-l\right| = \left|x_{n}-x_{n_{k}}+x_{n_{k}}-l\right| \leq \left|x_{n}-x_{n_{k}}\right| + \left|x_{n_{k}}-l\right| < \frac{\epsilon }{2} + \frac{\epsilon }{2} = \epsilon .\]
\end{proof}

\begin{eg}
\normalfont $\displaystyle x_{1} = 1 $, $\displaystyle x_{2} = 2 $, 
\[x_{n} = \frac{x_{n-1}+x_{n-2}}{2}, \; n\geq 3 .\]
Tenemos que la distancia entre dos puntos sucesivos será:
\[ \left|x_{n+1}-x_{n}\right| = \frac{1}{2^{n-1}}.\]
Entonces, 
\[
\begin{split}
	\left|x_{m}-x_{n}\right| = & \left|x_{m}-x_{m -1} + x_{m -1} + \cdots + x_{n+1} + x_{n}\right| \\
	\leq & \left|x_{m}-x_{m -1}\right| + \left|x_{m -1} - x_{m -2}\right| + \cdots + \left|x_{n+1}-x_{n}\right| \\
	= & \frac{1}{2^{m -2}} + \frac{1}{2^{m -3}}+ \cdots + \frac{1}{2^{n-1}} \\
	= & \frac{\frac{1}{2^{n-1}}-\frac{1}{2^{m -1}}}{1 - \frac{1}{2}} \to 0, \; \text{si} \; n,m \to \infty.
\end{split}
\]
Así, $\displaystyle \left\{ x_{n}\right\} _{n\in\N} $ es de Cauchy. En efecto, $\displaystyle x_{n} \to \frac{5}{3} $.
\end{eg}

\section{Otros teoremas}

\begin{observation}
\normalfont 
\begin{itemize} Si $\displaystyle \left\{ x_{n}\right\} _{n\in\N}, \left\{ y_{n}\right\} _{n\in\N}\subset\R $,
\item Recordamos que $\displaystyle \lim_{n \to \infty}x_{n} = \infty $ si $\displaystyle \forall c > 0, \exists n_{0} \in \N, \forall n\geq n_{0}$ tal que $\displaystyle x_{n} > c $. 
\item Si $\displaystyle x_{n} \leq y_{n} $ y $\displaystyle \lim_{n \to \infty}x_{n} = \infty $, entonces $\displaystyle \lim_{n \to \infty}y_{n} = \infty $. 
\item Si $\displaystyle x_{n} \leq y_{n} $ y $\displaystyle \lim_{n \to \infty} y_{n} = - \infty $, entonces $\displaystyle \lim_{n \to \infty}x_{n} = - \infty $.
\end{itemize}
\end{observation}

\begin{fcolorary}[]
	\normalfont Si $\displaystyle \left\{ x_{n}\right\} _{n\in\N}, \left\{ y_{n}\right\} _{n\in\N} \subset \R $ tales que $\displaystyle \lim_{n \to \infty}\frac{x_{n}}{y_{n}} = L \in \left(0, \infty\right) $ y $\displaystyle \lim_{n \to \infty}x_{n} = \infty  $, entonces, $\displaystyle \lim_{n \to \infty} y_{n} =\infty  $.
\end{fcolorary}

\begin{proof}
Dado $\displaystyle \epsilon = 1 $, tenemos que existe $\displaystyle n_{1} \in \N $ tal que si $\displaystyle n \geq n_{1} $ 
\[ \left|\frac{x_{n}}{y_{n}}-L\right|<1 .\]
Podemos suponer sin pérdida de generalidad que $\displaystyle x_{n}, y_{n} > 0 $. Entonces, tenemos que
\[ - 1 < \frac{x_{n}}{y_{n}} - L < 1 \iff -y_{n} < x_{n} - y_{n}L < y_{n} .\]
Así, $\displaystyle x_{n} < y_{n}\left(1 + L\right) $. Tenemos que $\displaystyle x_{n} \to \infty $, por lo que $\displaystyle y_{n}\left(1+L\right) \to \infty $ por lo que $\displaystyle y_{n} \to \infty $.
\end{proof}

\begin{eg}
\normalfont 
\[\lim_{n \to \infty}\frac{3n^{2}+5n}{n^{2}+1} = 3 .\]
Como $\displaystyle n^{2} + 1 \to \infty $ tenemos que $\displaystyle 3n^{2} + 5n \to \infty $ y viceversa. 
\end{eg}

\begin{ftheorem}[Criterio de Stolz]
	\normalfont Sean $\displaystyle \left\{ a_{n}\right\} _{n\in\N}, \left\{ b_{n}\right\} _{n\in\N}\subset \R $ tales que 
	\[b_{1} < b_{2} < \cdots < b_{n} < \cdots  ,\]
	y supongamos que $\displaystyle \lim_{n \to \infty}b_{n}= \infty $. Si existe 
	\[\lim_{n \to \infty}\frac{a_{n+1}-a_{n}}{b_{n+1}-b_{n}} = l , \]
	entonces $\displaystyle \lim_{n \to \infty}\frac{a_{n}}{b_{n}} = l $.
\end{ftheorem}

\begin{proof}
Sea $\displaystyle \epsilon > 0 $, entonces existe $\displaystyle n_{0} \in \N $ tal que $\displaystyle \forall n \geq n_{0} $, 
\[ -\frac{\epsilon }{2} < \frac{a_{n+1}-a_{n}}{b_{n+1}-b_{n}} - l < \frac{\epsilon }{2} .\]
\[
\begin{split}
 \iff\left(l  - \frac{\epsilon }{2}\right)\left(b_{n+1}-b_{n}\right) < a_{n+1}-a_{n} < \left(l + \frac{\epsilon }{2}\right)\left(b_{n+1}-b_{n}\right), \; \forall n\geq n_{0}.
\end{split}
\]
Así, para $\displaystyle n_{0} $,
\[\left(l  - \frac{\epsilon }{2}\right)\left(b_{n_{0}+1}-b_{n_{0}}\right) < a_{n_{0}+1}-a_{n_{0}} < \left(l + \frac{\epsilon }{2}\right)\left(b_{n_{0}+1}-b_{n_{0}}\right) .\]
\[\vdots \]
\[\left(l-\frac{\epsilon }{2}\right)\left(b_{n_{0}+m}-b_{n_{0}+m -1}\right) < a_{n_{0}+m} - a_{n_{0}+m -1}<\left(l+\frac{\epsilon }{2}\right)\left(b_{n_{0}+m}-b_{n_{0}+m -1}\right), \; m \in \N .\]
Si sumamos todas las desigualdades y usando la propiedad telescópica, 
\[\left(l - \frac{\epsilon }{2}\right)\left(b_{n_{0}+m}-b_{n_{0}}\right) < a_{n_{0}+m} - a_{n_{0}}< \left(l + \frac{\epsilon }{2}\right)\left(b_{n_{0}+m}-b_{n_{0}}\right) .\]
Dividimos todo por $\displaystyle b_{n_{0}+m} \neq 0 $ ($\displaystyle b_{n_{0}+m} > 0 $), 
\[ \left(l-\frac{\epsilon }{2}\right)\left(1-\frac{b_{n_{0}}}{b_{n_{0}+m}}\right) < \frac{a_{n_{0}+m}}{b_{n_{0}+m}}-\frac{a_{n_{0}}}{b_{n_{0}+m}}<\left(l+\frac{\epsilon }{2}\right)\left(1 - \frac{b_{n_{0}}}{b_{n_{0}+m}}\right).\]
Así, tenemos que 
\[l - \frac{\epsilon }{2} = \lim_{m \to \infty}\inf\left(l-\frac{\epsilon }{2}\right)\left(1 - \frac{b_{n_{0}}}{b_{n_{0}+m}}\right) \leq \lim_{m \to \infty}\inf \frac{a_{n_{0}+m}}{b_{n_{0}+m}} - 0.\]
Además, tenemos que 
\[\lim_{m \to \infty}\sup \frac{a_{n_{0}+m}}{b_{n_{0}+m}} - 0 \leq \lim_{m \to \infty}\sup\left(l+\frac{\epsilon }{2}\right)\left(1-\frac{b_{n_{0}}}{b_{n_{0}+m}}\right) = l + \frac{\epsilon }{2} .\]
Así, $\displaystyle \forall \epsilon > 0 $, $\displaystyle \exists n_{0} \in \N $ tal que $\displaystyle \forall m \in \N $, 
\[\lim_{m \to \infty}\frac{a_{n_{0}+m}}{b_{n_{0}+m}} \leq \lim_{m \to \infty}\sup\frac{a_{n_{0}+m}}{b_{n_{0}+m}} \leq l + \frac{\epsilon }{2} .\]
Por lo tanto, $\displaystyle \forall \epsilon > 0 $:
\[l - \frac{\epsilon }{2} \leq \lim_{m \to \infty}\inf \frac{a_{m}}{b_{m}} \leq \lim_{m \to \infty}\sup \frac{a_{m}}{b_{m}} \leq l + \frac{\epsilon }{2}.\]
Por lo que
\[l \leq \lim_{m \to \infty}\inf \frac{a_{m}}{b_{m}} \leq \lim_{m \to \infty}\sup \frac{a_{m}}{b_{m}} \leq l .\]
Por tanto tenemos que $\displaystyle \lim_{m \to \infty}\frac{a_{m}}{b_{m}} = l $.
\end{proof}

\begin{observation}
\normalfont 
\begin{description}
\item[(i)] En el criterio de Stolz, si $\displaystyle l = \pm \infty $, entonces $\displaystyle \lim_{m \to \infty}\frac{a_{m}}{b_{m}} = \pm \infty $. 
\item[Segundo criterio de Stolz.] Si tenemos $\displaystyle \left\{ a_{n}\right\} _{n\in\N}, \left\{ b_{n}\right\} _{n\in\N}\subset\R $ tales que $\displaystyle a_{n} \to 0 $ y $\displaystyle b_{n} \to 0 $ y $\displaystyle b_{n} < b_{n+1} $ \footnote{Realmente solo importa que sea monotona, no importa que sea creciente o decreciente.}, entonces, si $\displaystyle \lim_{n \to \infty}\frac{a_{n+1}-a_{n}}{b_{n+1}-b_{n}} = l $, entonces $\displaystyle \lim_{n \to \infty}\frac{a_{n}}{b_{n}} = l $.  
\end{description}
\end{observation}

\begin{observation}
\normalfont Propiedades de los límites superiores e inferiores:
\begin{description}
\item[(i)] Si $\displaystyle x_{n} \leq y_{n} $, entonces $\displaystyle \lim_{n \to \infty}\inf x_{n} \leq \lim_{n \to \infty}\inf y_{n} $ y $\displaystyle \lim_{n \to \infty}\sup x_{n} \leq \lim_{n \to \infty} \sup y_{n} $. 
\item[(ii)] $\displaystyle \lim_{n \to \infty} \inf x_{n} = \lim_{n \to \infty}\sup x_{n} = l \iff \lim_{n \to \infty}x_{n} = l $.
\item[(iii)] Si $\displaystyle \lim_{n \to \infty}x_{n} = l $, tenemos que 
	\[\lim_{n \to \infty}\sup\left(x_{n}+y_{n}\right) = l + \lim_{n \to \infty}\sup y_{n} .\]
	\[\lim_{n \to \infty}\inf\left(x_{n}+y_{n}\right) = l + \lim_{n \to \infty}\inf y_{n} .\]
\end{description}
\end{observation}

\begin{eg}
\normalfont Sea $\displaystyle x_{n} = \left(-1\right)^{n} $ y $\displaystyle y_{n} = \left(-1\right)^{n}+1 $, tenemos que $\displaystyle x_{n} \leq y_{n} $. Además, 
\[\lim_{n \to \infty}\inf x_{n} = -1 \leq \lim_{n \to \infty } \inf y_{n} = 0 .\]
Similarmente, 
\[\lim_{n \to \infty}\sup x_{n} = 1 \leq \lim_{n \to \infty}\sup y_{n} = 2 .\]
\end{eg}

\begin{eg}
\normalfont Sabemos que si $\displaystyle a_{n} = 1 + \frac{1}{2} + \cdots + \frac{1}{n} $, entonces $\displaystyle a_{n} \to \infty $. Veamos a qué tiende \footnote{A partir de ahora $\displaystyle \log = \ln $, no es en base 10.} 
\[n\geq 2, \; \frac{1 + \frac{1}{2} + \cdots + \frac{1}{n}}{\log n} .\]
Por el criterio de Stolz, basta estudiar el límite, 
\[ \frac{a_{n+1}-a_{n}}{\log\left(n+1\right)-\log n} = \frac{\frac{1}{n+1}}{\log\left(1 + \frac{1}{n}\right)} = \frac{1}{\frac{n+1}{n}\log\left(1+\frac{1}{n}\right)^{n}} .\]
Tenemos que $\displaystyle \log\left(1 + \frac{1}{n}\right)^{n} \to \log e = 1 $ y $\displaystyle  \frac{n+1}{n} \to 1$, por lo que 
\[\frac{a_{n+1}-a_{n}}{\log\left(n+1\right)-\log\left(n\right)} \to 1 .\]
Es decir, la serie armónica, en el infinito, se aproxima al logaritmo neperiano.
\end{eg}

\section{Series numéricas}

\begin{fdefinition}[Suma parcial]
\normalfont Sea $\displaystyle \left\{ a_{n}\right\} _{n\in\N}\subset\R $. Sea llama \textbf{suma parcial n-ésima} 
\[S_{n} = a_{1} + \cdots + a_{n} .\]
\end{fdefinition}

\begin{fdefinition}[]
	\normalfont Dada la sucesión $\displaystyle \left\{ a_{n}\right\} _{n\in\N} $, diremos que la \textbf{serie} $\displaystyle \sum_{n\in\N}a_{n} $ converge a $\displaystyle S $ si $\displaystyle \lim_{n \to \infty}S_{n} = S $.
\end{fdefinition}

\begin{eg}
\normalfont 
\begin{description}
\item[(i)] $\displaystyle a_{n} = 2^{-n}, \; n \in \N $. Tenemos que 
\[S_{n} = \frac{1}{2} + \frac{1}{2^{2}} + \cdots + \frac{1}{2^{n}} = \frac{\frac{1}{2}-\frac{1}{2^{n+1}}}{1 -\frac{1}{2}} .\]
De esta manera, 
\[\sum_{n\in\N}\frac{1}{2^{n}} = \sum^{\infty}_{n=1}\frac{1}{2^{n}} = \lim_{n \to \infty}S_{n} = 1 .\]
\item[(ii)] $\displaystyle a_{n} = \left(-1\right)^{n} $. Tenemos que $\displaystyle S_{1} = -1 $, $\displaystyle S_{2}= 0 $, entonces, no existe el límite de $\displaystyle S_{n} $. Por tanto, la serie $\displaystyle \sum^{\infty}_{n=1}\left(-1\right)^{n} $ no converge.
\item[(iii)] $\displaystyle a_{n} = \frac{1}{n} $. Tenemos que 
	\[S_{n} = \frac{1}{1} + \frac{1}{2} + \cdots + \frac{1}{n} .\]
Tenemos que $\displaystyle \lim_{n \to \infty}S_{n} = \infty $, por lo que la serie armónica es divergente.
\item[(iv)] $\displaystyle \sum^{\infty}_{n = 1}\frac{1}{n^{2}+n} $. Tenemos que 
	\[\frac{1}{n^{2}+n} = \frac{1}{n} -\frac{1}{n+1} .\]
Entonces tenemos que, 
\[
\begin{split}
	S_{1} & = \frac{1}{2} \\
	S_{2} & = \frac{1}{2} + \frac{1}{6} = \left(1 - \frac{1}{2}\right) + \left(\frac{1}{2} - \frac{1}{3}\right) \\
& \vdots \\
	S_{n} & = \left(1 - \frac{1}{2}\right) + \left(\frac{1}{2} - \frac{1}{3}\right) + \cdots + \left(\frac{1}{n}-\frac{1}{n+1}\right) = 1 - \frac{1}{n+1}.
\end{split}
\]
De esta manera, $\displaystyle \sum^{\infty}_{n = 1}\frac{1}{n^{2}+n} = \lim_{n \to \infty}S_{n} = 1 $.
\item[(v)] $\displaystyle \sum^{\infty}_{n = 1} \frac{1}{n^{2}} = \frac{\pi^{2}}{6} $.
\end{description}
\end{eg}

\begin{ftheorem}[]
	\normalfont Si $\displaystyle \left\{ a_{n}\right\} _{n\in\N}\subset\R $ y la serie $\displaystyle \sum^{\infty}_{n = 1}a_{n} $ converge, entonces $\displaystyle \lim_{n \to \infty}a_{n} = 0 $.
\end{ftheorem}

\begin{proof}
Sea $\displaystyle S_{n} = \sum^{n}_{k=1}a_{k} $. Como $\displaystyle S_{n} \to l $, entonces $\displaystyle S_{n-1} \to l $. Así, $\displaystyle S_{n}-S_{n-1} \to 0 $. Entonces tenemos que $\displaystyle S_{n}-S_{n-1} = a_{n} \to 0 $. 
\end{proof}

\begin{ftheorem}[]
	\normalfont Si $\displaystyle \left\{ a_{n}\right\} _{n\in\N}\subset\R^{+} $, entonces $\displaystyle \sum^{\infty}_{n = 1}a_{n}  $ converge si y solo si $\displaystyle \left\{ S_{n}\right\} _{n\in\N} $ está acotada superioremente.
\end{ftheorem}

\begin{proof}
	Como los $\displaystyle a_{n} \geq 0 $, tenemos que $\displaystyle S_{n} \leq S_{n+1} $ (es monótona creciente). Así, $\displaystyle S_{n} \to l \iff S_{n} $ está acotada superiormente.
\end{proof}

\begin{observation}
\normalfont Si $\displaystyle \sum^{\infty}_{n = 1}a_{n} = a $ y $\displaystyle \sum^{\infty}_{n = 1}b_{n} = b $, entonces 
\[\sum^{\infty}_{n = 1}\left(a_{n}+b_{n}\right) = a + b .\]
Es decir, podemos extrapolar todas las propiedades de los límites a las series.
\end{observation}

\begin{ftheorem}[Criterio de comparación]
\normalfont Si $\displaystyle \left\{ a_{n}\right\} _{n\in\N}, \left\{ b_{n}\right\} _{n\in\N} \subset \R^{+} $ tales que $\displaystyle a_{n} \leq b_{n} $ para $\displaystyle n \geq k $. Si, $\displaystyle \sum^{\infty}_{n = 1} b_{n} $ converge, entonces $\displaystyle \sum^{\infty }_{n = 1}a_{n} $ converge.
\end{ftheorem}

\begin{proof}
Sea $\displaystyle n \geq k $, entonces
\[\sum^{n}_{k = 1}a_{k} \leq \underbrace{a_{1} + \cdots + a_{k-1}}_{a \in \R} + \sum^{n}_{j=k}b_{j} \leq \underbrace{a + \sum^{\infty}_{n = 1}b_{n}}_{C} < \infty .\]
Entonces, tenemos que $\displaystyle S_{n} \leq C, \; \forall n \in \N$ y, por el teorema anterior, $\displaystyle \sum^{\infty}_{n = 1}a_{n} $ converge. 
\end{proof}

\begin{eg}
\normalfont Vamos a demostrar que $\displaystyle \sum^{\infty}_{n = 1} \frac{1}{n^{2}} $ converge. Tenemos que $\displaystyle \forall n \in \N $, 
\[n \leq n^{2} \iff \frac{1}{n^{2}} \leq \frac{2}{n^{2}+n} .\]
Entonces, sea $\displaystyle a_{n} = \frac{1}{n^{2}} $ y $\displaystyle b_{n} = \frac{2}{n^{2}+n} $. Entonces tenemos que $\displaystyle 0 \leq a_{n} \leq b_{n} $. De un ejemplo anterior deducimos que 
\[\sum^{\infty}_{n=1}b_{n} = 2 \cdot 1 = 2 .\]
Entonces, por el teorema anterior tenemos que \footnote{Si la sucesión es de términos positivos, entonces decir $\displaystyle \lim_{n \to \infty}a_{n} < \infty $ es lo equivalente a decir que converge. Si no son términos positivos no lo podemos afirmar. Considera $\displaystyle a_{n} = \left(-1\right)^{n} $.} 
\[\sum^{\infty}_{n = 1}\frac{1}{n^{2}} < \infty .\]
\end{eg}
\begin{eg}
\normalfont Si $\displaystyle p \in \R $ y $\displaystyle a_{n} = \frac{1}{n^{p}} $:
\begin{itemize}
\item Si $\displaystyle p \leq 0 $, tenemos que 
	\[a_{n} \to 
	\begin{cases}
	\infty, \; p\neq 0 \\
	1, \; p = 0
	\end{cases}
	.\]
Así, como $\displaystyle a_{n} $ no tiende a $\displaystyle 0 $, la serie $\displaystyle \sum^{\infty}_{n = 1}\frac{1}{n^{p}} = \infty $.
\item Si $\displaystyle 0 \leq p \leq 1 $, tenemos que $\displaystyle n^{p} \leq n $. Por tanto, 
	\[\frac{1}{n} \leq \frac{1}{n^{p}} \Rightarrow \sum^{\infty}_{n=1}\frac{1}{n^{p}} = \infty .\]
\item Si $\displaystyle p \geq 2 $, tenemos que $\displaystyle n^{p} \geq n^{2} $, por lo que $\displaystyle \frac{1}{n^{p}} \leq \frac{1}{n^{2}} $. Como $\displaystyle \sum^{\infty}_{n=1}\frac{1}{n^{2}} $ converge, por el criterio de comparación tenemos que 
	\[\sum^{\infty}_{n = 1}\frac{1}{n^{p}} < \infty .\]
\item Si $\displaystyle 1 < p < 2 $, tenemos que $\displaystyle \sum^{\infty}_{n = 1}\frac{1}{n^{p}}<\infty $ por el \textbf{criterio de la integral} (lo veremos en el segundo semestre).
\end{itemize}
\end{eg}

\begin{observation}
\normalfont Así, podemos concluir que 
\[\sum^{\infty}_{n = 1}\frac{1}{n^{p}} < \infty \iff p > 1 .\]
\end{observation}

\begin{eg}
\normalfont Demostramos que $\displaystyle \sum^{\infty}_{n = 1}\frac{1}{n^{3}+2n+1} < \infty $.
\[\lim_{n \to \infty}\frac{\frac{1}{n^{3}}}{\frac{1}{n^{3}+2n+1}} = 1 .\]
Así pues, 
\[\sum^{\infty}_{n = 1}\frac{1}{n^{3}+2n+1} \approx \sum^{\infty}_{n = 1}\frac{1}{n^{3}} < \infty .\]
\footnote{El símbolo $\displaystyle \approx $ lo utilizamos para decir que cuando $\displaystyle n \to \infty $ se comportan prácticamente igual.} 
\end{eg}

\begin{ftheorem}[]
	\normalfont Si $\displaystyle \left\{ a_{n}\right\} _{n\in\N}, \left\{ b_{n}\right\} _{n\in\N}\subset\R^{+} $ y $\displaystyle \lim_{n \to \infty}\frac{a_{n}}{b_{n}} = l > 0 $, entonces
	\[\sum^{\infty}_{n = 1}a_{n} < \infty \iff \sum^{\infty}_{n = 1}b_{n} < \infty .\]
\end{ftheorem}

\begin{proof}
Sea $\displaystyle \epsilon = \frac{l}{2} > 0 $, entonces existe $\displaystyle n_{0} \in \N $ tal que $\displaystyle \forall n \geq n_{0} $, 
\[ \left|\frac{a_{n}}{b_{n}}-l\right|<\epsilon \iff l- \epsilon < \frac{a_{n}}{b_{n}} < \epsilon + l \iff \frac{l}{2}< \frac{a_{n}}{b_{n}} < \frac{3l}{2} .\]
Entonces, si $\displaystyle \sum^{\infty}_{n =1}a_{n} $ converge, tenemos que 
\[\frac{l}{2}b_{n} < a_{n} ,\]
por lo que $\displaystyle \sum^{\infty}_{n = 1}b_{n} $ también converge. Recíprocamente, si $\displaystyle \sum^{\infty}_{n = 1}b_{n} < \infty $ tenemos que $\displaystyle a_{n} < \frac{3l}{2}b_{n} $, por lo que $\displaystyle \sum^{\infty}_{n = 1}a_{n} < \infty $.
\end{proof}

\begin{ftheorem}[Criterio de la raíz]
	\normalfont Sea $\displaystyle \left\{ a_{n}\right\} _{n\in\N}\subset\R^{+} $ y supongamos que existe $\displaystyle \lim_{n \to \infty}\sqrt[n]{a_{n}} = a \geq 0 $.
\begin{itemize}
\item Si $\displaystyle a < 1 $, entonces $\displaystyle \sum^{\infty}_{n =1}a_{n} $ converge.
\item Si $\displaystyle a > 1 $, entonces $\displaystyle \sum^{\infty}_{n = 1}a_{n} $ diverge.
\item Si $\displaystyle a = 1 $, entonces no sabemos.
\end{itemize}
\end{ftheorem}

\begin{proof}
\begin{description}
\item[(i)] Si $\displaystyle a <1  $, sea $\displaystyle a < r < 1 $. Sea $\displaystyle \epsilon > 0 $ tal que $\displaystyle a - \epsilon > 0 $ y $\displaystyle a+\epsilon < r $. Entonces, existe un $\displaystyle n_{0} \in \N $ tal que $\displaystyle \forall n \geq n_{0} $, 
	\[ a - \epsilon < \sqrt[n]{a_{n}} < a+\epsilon < r .\]
Entonces tenemos que si $\displaystyle n \geq n_{0} $, 
\[\left(a-\epsilon \right)^{n} < a_{n} < \left(a+\epsilon \right)^{n} < r^{n} .\]
Por el criterio de comparación, como 
\[\sum^{\infty}_{n = 1}r^{n} = \frac{r}{1-r} < \infty, \; 0 \leq r < 1 ,\]
es convergente, tenemos que $\displaystyle \sum^{\infty}_{n=1}a_{n} $ también es convergente.
\item[(ii)] Análogamente, si $\displaystyle a > 1 $, tomamos $\displaystyle 1 < r < a $ y $\displaystyle \epsilon  $ tal que $\displaystyle r < a-\epsilon  $. Así, existe un $\displaystyle n_{0}\in\N $ tal que $\displaystyle \forall n \geq n_{0} $, 
	\[ \left|\sqrt[n]{a_{n}}-a\right|<\epsilon \iff r < -\epsilon + a < \sqrt[n]{a_{n}}<\epsilon + a .\]
Es decir, si $\displaystyle n \geq n_{0} $,
\[ r^{n} < a_{n} .\]
Como $\displaystyle r > 1 $, tenemos que 
\[\sum^{\infty}_{n = 1}r^{n} = \lim_{n \to \infty}\frac{r-r^{n+1}}{1-r} = \infty .\]
Por el criterio de comparación, como $\displaystyle \sum^{\infty}_{n = 1}a_{n} > \sum^{\infty}_{n=1}r^{n} = \infty $, tenemos que la serie de $\displaystyle a_{n} $ diverge.
\end{description}
\end{proof}

\begin{eg}
\normalfont Ejemplos del caso $\displaystyle a = 1 $. 
\begin{description}
\item[(i)] $\displaystyle \sum^{\infty}_{n =1}\frac{1}{n} = \infty $ y 
	\[\lim_{n \to \infty}\sqrt[n]{\frac{1}{n}} = \lim_{n \to \infty}\frac{1}{n^{\frac{1}{n}}} = 1 .\]
\item[(ii)] Considera $\displaystyle \sum^{\infty}_{n = 1}\frac{1}{n^{2}} $ converge y 
	\[\lim_{n \to \infty}\sqrt[n]{\frac{1}{n^{2}}} = \lim_{n \to \infty}\frac{1}{n^{\frac{2}{n}}} = \lim_{n \to \infty}\left(\frac{1}{n^{\frac{1}{n}}}\right)^{2} = 1 .\]
\end{description}
\end{eg}

\begin{eg}
\normalfont \textbf{Ejercicio 76(f).} $\displaystyle \sum^{\infty}_{n = 1}\left(\frac{n+1}{n}\right)^{n^{2}}3^{-n} $. Llamamos $\displaystyle a_{n} $ al término general. Entonces, 
\[\sqrt[n]{a_{n}} = \left(\frac{n+1}{n}\right)^{n} 3^{-1} .\]
Si $\displaystyle n \to \infty $, tenemos que 
\[\left(\frac{n+1}{n}\right)^{n} = \left(1+\frac{1}{n}\right)^{n} \to e .\]
Como $\displaystyle 0< e < 3 $, tenemos que $\displaystyle \frac{e}{3} < 1 $. Por el criterio de la raíz, la serie converge.
\end{eg}

\begin{ftheorem}[Criterio del cociente]
	\normalfont Si $\displaystyle \left\{ a_{n}\right\} _{n\in\N}\subset\R^{+} $ tal que existe $\displaystyle \lim_{n \to \infty}\frac{a_{n+1}}{a_{n}} = a $, entonces:
\begin{itemize}
\item Si $\displaystyle a < 1 $, tenemos que $\displaystyle \sum^{\infty}_{n = 1}a_{n} $ converge.
\item Si $\displaystyle a > 1 $, tenemos que $\displaystyle \sum^{\infty}_{n = 1}a_{n} $ diverge.
\item Si $\displaystyle a = 1 $, no sabemos.
\end{itemize}
\end{ftheorem}

\begin{proof}
Una manera de demostrarlo es ver que este criterio implica el de la raíz. Otra manera de hacerlos es recurriendo a series geométricas. 
\begin{description}
\item[(i)] Si $\displaystyle a < 1 $, tomamos $\displaystyle \epsilon > 0 $ tal que si $\displaystyle n \geq n_{0} $, y tomamos $\displaystyle a < r < 1 $,
	\[ \frac{a_{n+1}}{a_{n}}<a+\epsilon < r .\]
Entonces tenemos que si $\displaystyle n \geq n_{0} $, 
\[a_{n+1} < ra_{n} < \cdots < r^{n+1-n_{0}}a_{n_{0}} = r^{n+1}\frac{a_{n_{0}}}{r^{n_{0}}}.\]
Entonces, tenemos que 
\[\sum^{\infty}_{n =1}a_{n+1} < \frac{a_{n_{0}}}{r^{n_{0}}}\sum^{\infty}_{n = 1}r^{n+1} .\]
Como $\displaystyle r < 1 $, tenemos que 
\[\sum^{\infty}_{n = 1}r^{n+1} <\infty .\]
Por el criterio de comparación, tenemos que $\displaystyle \sum^{\infty}_{n = 1}a_{n} $ converge.
\item[(ii)] Si $\displaystyle a > 1 $ es análogo.
\end{description}
\end{proof}

\begin{eg}
\normalfont Ejemplos de $\displaystyle a = 1 $. 
\begin{description}
\item[(i)] Sea $\displaystyle a_{n} = \frac{1}{n} $. Tenemos que 
	\[\lim_{n \to \infty}\frac{a_{n+1}}{a_{n}} = \lim_{n \to \infty}\frac{n}{n+1} = 1 .\]
\item[(ii)] Sea $\displaystyle a_{n} = \frac{1}{n^{2}} $. 
	\[\lim_{n \to \infty}\frac{a_{n+1}}{a_{n}} = \lim_{n \to \infty}\left(\frac{n^{2}}{\left(n+1\right)^{2}}\right) = \lim_{n \to \infty}\left(\frac{n}{n+1}\right)^{2} = 1 .\]
\item[(iii)] Sea $\displaystyle k \in \N $ y sea $\displaystyle a_{n} = \frac{n^{k}}{2^{n}} $. Usamos el criterio del cociente:
	\[\lim_{n \to \infty}\frac{a_{n+1}}{a_{n}} = \lim_{n \to \infty}\frac{\frac{\left(n+1\right)^{k}}{2^{n+1}}}{\frac{n^{k}}{2^{n}}} = \lim_{n \to \infty} \frac{2^{n}}{2^{n+1}}\left(\frac{n+1}{n}\right)^{k} = \frac{1}{2}<1 .\]
Por el criterio del cociente, $\displaystyle \sum^{\infty}_{n = 1}\frac{n^{k}}{2^{n}} $ converge.
\end{description}
\end{eg}

\begin{ftheorem}[Criterio de Leibniz]
	\normalfont Supongamos que $\displaystyle \left\{ a_{n}\right\} _{n\in\N}\subset\R^{+} $ decreciente, es decir, $\displaystyle a_{n+1} \leq a_{n} $ y tal que $\displaystyle \lim_{n \to \infty}a_{n} = 0 $. Entonces, la serie alternada converge:
	\[\sum^{\infty}_{n = 1}\left(-1\right)^{n}a_{n} \quad \text{converge}.\]
\end{ftheorem}

\begin{eg}
\normalfont Considera $\displaystyle a_{n} = \frac{1}{n} $. Entonces tenemos que $\displaystyle \sum^{\infty}_{n = 1} \frac{\left(-1\right)^{n}}{n} $ converge, mientras que $\displaystyle \sum^{\infty}_{n=1} \frac{1}{n} = \infty $.
\end{eg}

\begin{fdefinition}[]
\normalfont Se dice que una serie es \textbf{absolutamente convergente} si $\displaystyle \sum^{\infty}_{n = 1} \left|a_{n}\right| < \infty $.
\end{fdefinition}

\begin{observation}
\normalfont $\displaystyle \sum^{\infty}_{n = 1}\frac{\left(-1\right)^{n}}{n} $ no es absolutamente convergente.
\end{observation}

\begin{ftheorem}[]
\normalfont Si $\displaystyle \sum^{\infty}_{n = 1}a_{n} $ es absolutamente convergente (i.e. $\displaystyle \sum^{\infty}_{n = 1} \left|a_{n}\right| < \infty $), entonces $\displaystyle \sum^{\infty}_{n = 1}a_{n} $ es convergente.
\end{ftheorem}

\begin{proof}
	Sea $\displaystyle S_{n} = \sum^{n}_{k=1}a_{k} $. Queremos ver que $\displaystyle \left\{ S_{n}\right\} _{n\in\N} $ es una sucesión convergente. Sea $\displaystyle S^{*}_{n} = \sum^{n}_{k=1} \left|a_{k}\right| $. Sabemos que $\displaystyle \left\{ S^{*}_{n}\right\} _{n\in\N} $ converge. Entonces, tenemos que $\displaystyle \left|S_{n}\right| \leq S^{*}_{n} $. Sabemos que $\displaystyle \left\{ S_{n}\right\} _{n\in\N} $ es convergente si y solo si es de Cauchy. Sea $\displaystyle m > n $, 
	\[0\leq |S_{m} - S_{n} | = |a_{n+1}+ \cdots + a_{n}| \leq \left|a_{n+1}\right| + \cdots + \left|a_{m}\right| = S^{*}_{m}-S^{*}_{n}.\]
Si $\displaystyle m,n \to \infty $, tenemos que $\displaystyle S^{*}_{m} - S^{*}_{n} \to 0 $. Entonces, $\displaystyle \left|S_{m}-S_{n}\right| \to 0 $, por lo que es de Cauchy y, consecuentemente, converge. 
\end{proof}

\begin{eg}
\normalfont Estudiar la convergencia de las siguientes series.
\begin{description}
\item[(i)] $\displaystyle \sum^{\infty}_{n = 1}\frac{n!}{n^{n}} $. Usamos el criterio del cociente.
	\[\frac{\frac{\left(n+1\right)!}{\left(n+1\right)^{n+1}}}{\frac{n!}{n^{n}}} = \frac{n+1}{\left(n+1\right)\left(\frac{n+1}{n}\right)^{n}} \to \frac{1}{e} < 1 .\]
	Por tanto, la serie converge, es decir, $\displaystyle \sum^{\infty}_{n = 1}\frac{n!}{n^{n}} < \infty $.
\item[(ii)] $\displaystyle \sum^{\infty}_{n = 1}\frac{\sin n\theta}{n^{2}} $. Tenemos que 
	\[\sum^{\infty}_{n = 1} \frac{ \left|\sin n\theta\right|}{n^{2}} \leq \sum^{\infty}_{n = 1}\frac{1}{n^{2}} < \infty .\]
	Por el criterio de comparación, $\displaystyle \sum^{\infty}_{n = 1}\frac{ \left|\sin n\theta\right|}{n^{2}} < \infty $. Es decir, la serie inicial converge absolutamente por lo que converge.
\item[(iii)] $\displaystyle \sum^{\infty}_{n = 1}\frac{x^{n}}{n!} $ con $\displaystyle x \in \R $. Tomamos la serie en valor absoluto para ver que converge absolutamente. Empleamos el criterio del cociente: 
	\[ \frac{ \frac{ \left|x\right|^{n+1}}{\left(n+1\right)!}}{\frac{ \left|x\right|^{n}}{n!}}=\frac{ \left|x\right|}{n+1} \to 0.\]
Como $\displaystyle \sum^{\infty}_{n = 1} \frac{ \left|x\right|^{n}}{n!} < \infty $, la serie inicial converge absolutamente, por lo que converge.	
\end{description}
\end{eg}

\begin{observation}
\normalfont  
\[\sum^{\infty}_{ n=1}\frac{1}{n!} = e .\]
\end{observation}

\begin{ftheorem}[Sumación por partes de Abel]
	\normalfont Sean $\displaystyle \left\{ a_{n}\right\} _{n\in\N}, \left\{ b_{n}\right\} _{n\in\N}\subset\R $, sea $\displaystyle A_{n} = \sum^{n}_{k=1}a_{k} $ y sea $\displaystyle m > n $. Entonces, 
	\[\sum^{m}_{k=n+1}a_{k}b_{k} = \sum^{m -1}_{k=n+1}A_{k}\left(b_{k}-b_{k+1}\right) + A_{m}b_{m}-A_{n}b_{n+1} .\]
\end{ftheorem}

\begin{proof}
Tenemos que 
\[
\begin{split}
	\sum^{m}_{k=n+1}a_{k}b_{k} = & \sum^{m}_{k=n+1}\left(A_{k}-A_{k-1}\right)b_{k} = \sum^{m}_{k=n+1}A_{k}b_{k} - \sum^{m - 1}_{k=n}A_{k}b_{k+1}\\ = &A_{m}b_{m} - A_{n}b_{n+1} + \sum^{m - 1}_{k = n + 1}A_{k}\left(b_{k}-b_{k+1}\right).
\end{split}
\]
\end{proof}

\begin{ftheorem}[Criterio de Dirichlet]
	\normalfont Sean $\displaystyle \left\{ a_{n}\right\} _{n\in\N}, \left\{ b_{n}\right\} _{n\in\N} \subset \R $ tal que si $\displaystyle A_{n} = \sum^{n}_{k=1}a_{k} $ entonces $\displaystyle \left|A_{n}\right| \leq M, \; \forall n \in \N $ y $\displaystyle b_{n} $ decrece y converge a 0. Entonces, 
	\[\sum^{\infty}_{n = 1}a_{n}b_{n} \quad \text{converge.} \]
\end{ftheorem}

\begin{proof}
Vamos a ver que es sucesión de Cauchy, es decir, $\displaystyle \sum^{m}_{k=n+1}a_{k}b_{k} \to 0 $ si $\displaystyle m \geq n +1 $, cuando $\displaystyle n, m \to \infty $. Utilizamos el teorema anterior:
\[
\begin{split}
	\left|\sum^{m}_{k=n+1}a_{k}b_{k}\right| = & \left|\sum^{m - 1}_{k=n+1}A_{k}\left(b_{k}-b_{k+1}\right)+A_{m}b_{m} - A_{n}b_{n+1}\right| .
\end{split}
\]
Sea $\displaystyle \epsilon > 0 $ y sea $\displaystyle n_{0}\in\N $ tal que si $\displaystyle n \geq n_{0} $, 
\[ \left|b_{n}\right| < \frac{\epsilon }{3M} ,\]
y $\displaystyle \forall m > n \geq n_{0} $ tal que 
\[ \left|b_{m}-b_{n}\right| < \frac{\epsilon }{3M} .\]
Seguimos con lo anterior:
\[
\begin{split}
	\left|\sum^{m}_{k=n+1}a_{k}b_{k}\right| = & \left|\sum^{m - 1}_{k=n+1}A_{k}\left(b_{k}-b_{k+1}\right)+A_{m}b_{m} - A_{n}b_{n+1}\right| \\
	\leq & M \sum^{m -1}_{k = n+1} \left|b_{k}-b_{k+1}\right| + M \left|b_{m}\right| + M \left|b_{n + 1}\right| \\
	\leq & M\sum^{m - 1}_{k = n+1}\left(b_{k}-b_{k+1}\right) + M\frac{\epsilon }{3M} + M\frac{\epsilon }{3M} = M \left(b_{n+1}-b_{m}\right) + \frac{2\epsilon }{3} \\
	    = & M\frac{\epsilon }{3M} + \frac{2\epsilon }{3} = \epsilon.
\end{split}
\]
\end{proof}

\begin{ftheorem}[Criterio de Leibniz]
	\normalfont Sea $\displaystyle \left\{ a_{n}\right\} _{n\in\N}\subset\R $ decreciente que converge a 0, entonces la serie alternada $\displaystyle \sum^{\infty}_{n = 1}\left(-1\right)^{n}a_{n} $ converge.
\end{ftheorem}
\begin{proof}
Tenemos que 
\[\sum^{n}_{k=1}\left(-1\right)^{k} = 
\begin{cases}
-1, \; n \; \text{impar}\\
0, \; n \; \text{par} 
\end{cases}
.\]
Entonces, las sumas parciales están acotadas superiormente por $\displaystyle M = 1 $. Por el criterio de Dirichlet, tenemos que 
\[\sum^{\infty}_{n = 1}\left(-1\right)^{n}a_{n} ,\]
converge \footnote{Este teorema es realmente un corolario del teorema anterior.} .
\end{proof}
\begin{eg}
\normalfont 
\begin{description}
\item[(i)] $\displaystyle \sum^{\infty}_{n = 1}\frac{\left(-1\right)^{n}}{n} $ converge.
\item[(ii)] $\displaystyle \sum^{\infty}_{n = 1}\frac{\left(-1\right)^{n}}{\sqrt{n}} $ converge.
\item[(iii)] Demostrar que (usar inducción)
	\[ \left|\sum^{m}_{n = 1}\cos nx\right| = \left|\frac{\sin\left(m + \frac{1}{2}\right)x}{2\sin \frac{x}{2}}-\frac{1}{2}\right| .\]
\end{description}
\end{eg}

\section{Exponentes reales.}
Recordamos que si $\displaystyle x > 0 $ y $\displaystyle n \in \N $:
\[x^{\frac{1}{n}} = \sup \left\{ r \in \Q \; : \; r^{n} < x\right\}  .\]
En general, si $\displaystyle n \in \Z, \; n < 0 $, definimos $\displaystyle x^{0} = 1 $ y
\[ x^{\frac{1}{n}} = \left(\frac{1}{x}\right)^{-\frac{1}{n}}.\]
Pasamos al caso $\displaystyle p = \frac{m}{n}\in \Q $, 
\[x^{p} = \left(x^{\frac{1}{n}}\right)^{m} .\]
Queremos ver que pasa en el caso $\displaystyle x^{\alpha } $ con $\displaystyle \alpha \in \R $. 
\begin{flema}[]
\normalfont Dado $\displaystyle \epsilon > 0 $, existe $\displaystyle n_{0} \in \N $ tal que 
\[ \left|x^{r}-1\right| < \epsilon, \; 0 < r < \frac{1}{n_{0}}, \; r \in \Q .\]
\end{flema}

\begin{proof}
Sabemos que $\displaystyle x^{\frac{1}{n}} \to 1 $ si $\displaystyle n \to \infty $. Así, dado $\displaystyle \epsilon > 0 $, existe $\displaystyle n_{0} \in \N $ tal que si $\displaystyle n \geq n_{0} $, 
\[ \left|x^{\frac{1}{n}}-1\right|< \epsilon  .\]
Sea $\displaystyle 0 < r < \frac{1}{n_{0}}, \; r \in \Q $. Si $\displaystyle x > 1 $, tenemos que $\displaystyle  0 < x^{r}-1 < x^{\frac{1}{n_{0}}}-1<\epsilon$. Si $\displaystyle 0 < x < 1 $, tenemos que $\displaystyle x^{r} > x^{\frac{1}{n_{0}}} $ por lo que $\displaystyle 0 < 1 - x^{r} < 1 - x^{\frac{1}{n_{0}}} < \epsilon  $. El caso $\displaystyle x = 1 $ es trivial. 
\end{proof}

\begin{fcolorary}[]
\normalfont Si $\displaystyle r_{n} \to 0 $ y $\displaystyle r_{n} \in \Q^{+} $, con $\displaystyle x > 0 $, entonces $\displaystyle x^{r_{n}} \to 1 $ si $\displaystyle n \to \infty $.
\end{fcolorary}

\begin{ftheorem}[]
\normalfont Si $\displaystyle x > 0 $, y $\displaystyle \alpha \in \R $, entonces
\begin{description}
	\item[(i)] Si $\displaystyle r_{n} \in\Q $ tales que $\displaystyle \lim_{n \to \infty}r_{n} = \alpha  $, entonces $\displaystyle \left\{ x^{r_{n}}\right\} _{n\in\N} $ converge. 
	\item[(ii)] Si $\displaystyle s_{n} \in \Q $ tal que $\displaystyle \lim_{n \to \infty}s_{n}=\alpha  $, entonces $\displaystyle \lim_{n \to \infty}x^{s_{n}} = \lim_{n \to \infty}x^{r_{n}} $.
\end{description}
\end{ftheorem}

\begin{proof}
\begin{description}
	\item[(i)] Tenemos $\displaystyle r_{n}\in\Q $ con $\displaystyle \lim_{n \to \infty}r_{n}=\alpha  $. Sin pérdida de generalidad tomamos $\displaystyle \alpha > 0 $. Vamos a demostrar que es de Cauchy. Como $\displaystyle \left\{ r_{n}\right\} _{n\in\N}$ converge, entonces está acotada, por lo que existe $\displaystyle K $ tal que $\displaystyle 0 \leq r_{n} < K \in \N $ (descartamos los primeros términos negativos). Hacemos primero el caso $\displaystyle x > 1 $ (el caso $\displaystyle x < 1 $ es análogo). Supongamos que $\displaystyle r_{m} \geq r_{n} $. 
	\[ \left|x^{r_{m}}-x^{r_{n}}\right| = \left|x^{r_{n}}\left(x^{r_{m}-r_{n}}-1\right)\right| \leq x^{K} \left|x^{r_{m}-r_{n}}-1\right| \to 0 .\]
En el paso anterior hemos utilizado el lema y corolario anterior. Así, tenemos que $\displaystyle \lim_{n \to \infty}x^{r_{n}} $ converge. 
\item[(ii)] Si $\displaystyle s_{n} \to \alpha  $ y $\displaystyle r_{n} \to \alpha  $, tenemos que $\displaystyle s_{n}-r_{n} \to 0 $. Supongamos que $\displaystyle r_{n} \geq s_{n} $,
\[ x^{r_{n}}-x^{s_{n}} = x^{s_{n}}\left(x^{r_{n}-s_{n}}-1\right) .\]
Entonces, tenemos que
\[ \left|x^{r_{n}}-x^{s_{n}}\right| \leq x^{K} \left|x^{r_{n}-s_{n}}-1\right| \to 0 .\]
Como la diferencia converge a 0, tenemos que $\displaystyle \lim_{n \to \infty}x^{r_{n}} = \lim_{n \to \infty}x^{s_{n}} $. \footnote{Esta afirmación solo la podemos hacer si las sucesiones convergen. Sino, no, considera $\displaystyle x_{n} = n $ y $\displaystyle y_{n} = n $.} 
\end{description}
\end{proof}

\begin{fdefinition}[]
\normalfont Dado $\displaystyle x > 0 $ y $\displaystyle \alpha \in\R $ se define $\displaystyle x^{\alpha } = \lim_{n \to \infty}x^{r_{n}} $ donde $\displaystyle r_{n} \in \Q\to \alpha  $.
\end{fdefinition}

\begin{fprop}[]
	\normalfont Si $\displaystyle \left\{ x_{n}\right\} _{n\in\N}, \left\{ y_{n}\right\} _{n\in\N} \subset\R $ con $\displaystyle x_{n}, y_{n} > 0 $ , con $\displaystyle x_{n} \to x > 0 $ y $\displaystyle y_{n} \to y > 0 $, entonces
	\[x_{n}^{y_{n}} \to x^{y} .\]
\end{fprop}

\begin{proof}
	Dado $\displaystyle \epsilon > 0 $, sea $\displaystyle \epsilon ' = \left[\left(\frac{\epsilon }{x^{y}}+1\right)^{\frac{1}{y}}-1\right] \cdot x > 0 $. Entonces, existe $\displaystyle n_{0} \in \N $ tal que si $\displaystyle n \geq n_{0} $, $\displaystyle x_{n} < x + \epsilon ' $. Así, tenemos que 
\[x_{n}^{y_{n}} < \left(x + \epsilon'\right)^{y_{n}} .\]
Entonces, tenemos que 
\[\lim_{n \to \infty}\sup x_{n}^{y_{n}} \leq \lim_{n \to \infty}\sup\left(x+\epsilon'\right)^{y_{n}}=\left(x+\epsilon'\right)^{y} = x^{y}+\epsilon, \; \forall \epsilon > 0.\]
Así, tenemos que $\displaystyle \lim_{n \to \infty}\sup x_{n}^{y_{n}} \leq x^{y} $. Ahora, dado $\displaystyle \epsilon > 0 $, tomamos $\displaystyle \epsilon '' = x - \left(x^{y}-\epsilon \right)^{\frac{1}{y}} > 0 $. Así, obtenemos que para $\displaystyle n \geq n_{0}\in\N $, $\displaystyle x-\epsilon '' < x_{n} $:
\[\left(x - \epsilon ''\right)^{y} = x^{y}-\epsilon \leq \lim_{n \to \infty}\inf x_{n}^{y_{n}}, \; \forall \epsilon > 0 .\]
Así, $\displaystyle x^{y} \leq \lim_{n \to \infty}\inf x_{n}^{y_{n}} $.
\end{proof}
