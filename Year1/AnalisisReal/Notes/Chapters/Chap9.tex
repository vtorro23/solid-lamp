\chapter{Aproximación polinómica de funciones}
\section{Polinomios de Taylor}
\begin{fdefinition}[Polinomio de Taylor]
\normalfont Dada $\displaystyle f : \left(a-\delta, a + \delta \right)\to \R $, con $\displaystyle \delta > 0 $, tal que existe $\displaystyle f\left(a\right), f'\left(a\right), \ldots, f^{\left(n\right)}\left(a\right) $, se llama \textbf{polinomio de Taylor} de la función $\displaystyle f $ centrado en $\displaystyle a $ y de grado $\displaystyle n $ al polinomio
\[P_{a,n}\left(x\right) = \sum^{n}_{k=0}\frac{f^{\left(k\right)}\left(a\right)}{k!}\left(x-a\right)^{k} .\]
\end{fdefinition}
\begin{eg}
\normalfont Consideremos la función polinómica
\[f\left(x\right) = a_{n}\left(x-a\right)^{n} + \cdots + a_{1}\left(x-a\right) + a_{0} .\]
Tenemos que $\displaystyle P_{f,a,n}\left(x\right) = f\left(x\right) $.
\end{eg}
\begin{observation}
\normalfont Tenemos que dada $\displaystyle f : \left(a-\delta, a + \delta \right)\to \R $, el polinomio de Taylor de grado 1 centrado en $\displaystyle a $ será la recta tangente a la curva en $\displaystyle \left(a,f\left(a\right)\right) $. Es decir,
\[P_{a,1}\left(x\right) = f\left(a\right) + f'\left(a\right)\left(x-a\right) .\]
Además, sabemos que
\[\lim_{x \to a}\frac{f\left(x\right)-P_{a,1}\left(x\right)}{x-a} = 0 .\]
\end{observation}
\begin{observation}
\normalfont Tenemos que si $\displaystyle f : \left(a-\delta, a + \delta \right)\to \R $, 
\[ P_{a,2}\left(x\right) = f\left(a\right) + f'\left(a\right)\left(x-a\right) + f''\left(a\right)\left(x-a\right)^{2} .\]
Entonces, podemos ver que
\[f\left(a\right) = P_{a,2}\left(a\right), \; f'\left(a\right) = P_{a,2}'\left(a\right), \; f''\left(a\right) = P''_{a,2}\left(a\right) .\]
En general, $\displaystyle f^{\left(k\right)}\left(a\right) = P^{\left(k\right)}_{a,n}\left(a\right) $, para $\displaystyle k = 0, 1, \ldots, n $.
\end{observation}
\begin{ftheorem}[]
\normalfont Sea $\displaystyle f: \left(a - \delta, a + \delta \right) \to \R $ derivable $\displaystyle n $ veces en $\displaystyle \left(a - \delta, a + \delta \right) $ con $\displaystyle f^{\left(n\right)} $ continua. Entonces
\[ \lim_{x \to a}\frac{f\left(x\right)-P_{a,n}\left(x\right)}{\left(x-a\right)^{n}} .\]
\end{ftheorem}
\begin{proof}
Aplicando L'Hôpital $\displaystyle n $ veces:
\[ \lim_{x \to a}\frac{f^{\left(n\right) } \left(x\right)- P^{\left(n\right)}_{a,n}\left(x\right)}{n!} = 0 .\]
\end{proof}
\begin{fcolorary}[]
\normalfont Sea $\displaystyle a \in \left(a - \delta, a + \delta \right) \subset \dom\left(f\right) $ de modo que existe $\displaystyle f' $ y $\displaystyle f'' $ en $\displaystyle \left(a - \delta, a + \delta \right) $ y $\displaystyle f'\left(a\right) = 0 $.
\begin{description}
\item[(a)] Si $\displaystyle f''\left(a\right) > 0 $, entonces $\displaystyle a $ es un mínimo local de $\displaystyle f $.
\item[(b)] Si $\displaystyle f''\left(a\right) < 0 $, entonces $\displaystyle a $ es un máximo local de $\displaystyle f $.
\end{description}
\end{fcolorary}
\begin{proof}
La demostración de \textbf{(b)} es análoga a la de \textbf{(a)}. Por el teorema anterior tenemos que
\[\lim_{x \to a}\frac{f\left(x\right)-P_{a,2}\left(x\right)}{\left(x-a\right)^{2}} = 0 .\]
Ahora, tenemos que
\[\frac{f\left(x\right)-P_{a,2}\left(x\right)}{\left(x-a\right)^{2}} = \frac{f\left(x\right)-f\left(a\right)-\frac{f''\left(a\right)}{2}\left(x-a\right)^{2}}{\left(x-a\right)^{2}} = \frac{f\left(x\right)-f\left(a\right)}{\left(x-a\right)^{2}} -\frac{f''\left(a\right)}{2} \to 0.\]
Por tanto tenemos que 
\[\lim_{x \to a}\frac{f\left(x\right)-f\left(a\right)}{\left(x-a\right)^{2}} = \frac{f''\left(a\right)}{2} > 0 .\]
Como $\displaystyle \left(x-a\right)^{2} > 0 $, tiene que ser que $\displaystyle f\left(x\right) - f\left(a\right) > 0 $ en un entorno de $\displaystyle a $. Por definición, se tiene que $\displaystyle a $ es un mínimo local. 
\end{proof}
\begin{fcolorary}[]
\normalfont Sea $\displaystyle f : \left(a-\delta, a + \delta \right)\to \R $ tal que existen $\displaystyle f', \ldots, f^{\left(n\right)} $ y $\displaystyle f'\left(a\right) = \cdots = f^{\left(n-1\right)}\left(a\right) = 0$ y $\displaystyle f^{\left(n\right)}\left(a\right) \neq 0 $.
\begin{description}
\item[(a)] Si $\displaystyle n $ es par y $\displaystyle f^{\left(n\right)}\left(a\right) > 0 $, entonces $\displaystyle a $ es un mínimo local de $\displaystyle f $.
\item[(b)] Si $\displaystyle n $ es par y $\displaystyle f^{\left(n\right)}\left(a\right) < 0 $, entonces $\displaystyle a $ es un máximo local de $\displaystyle f $.
\item[(c)] Si $\displaystyle n $ es impar, entonces $\displaystyle f $ no tiene un mínimo ni un máximo local en $\displaystyle a $.
\end{description}
\end{fcolorary}
\begin{proof}
Las demostraciones de \textbf{(a)} y \textbf{(b)} son similares a las del corolario anterior. Por ello, sólo demostraremos \textbf{(c)}. Por el teorema tenemos que 
\[0 = \lim_{x \to a}\frac{f\left(x\right)-P_{a,n}\left(x\right)}{\left(x-a\right)^{n}} = \lim_{x \to a}\left[\frac{f\left(x\right)-f\left(a\right)}{\left(x-a\right)^{n}}-\frac{f^{\left(n\right)}\left(a\right)}{n!}\right]  .\]
Ahora si $\displaystyle n $ es impar y, sin pérdida de generalidad, $\displaystyle f^{\left(n\right)}\left(a\right) > 0 $, entonces
\begin{itemize}
\item si $\displaystyle x > a $, $\displaystyle \left(x-a\right)^{n} > 0 $, por lo que $\displaystyle f\left(x\right)-f\left(a\right) > 0 $.
\item si $\displaystyle x < a $, $\displaystyle \left(x-a\right)^{n} < 0 $, por lo que $\displaystyle f\left(x\right)-f\left(a\right) < 0 $.
\end{itemize}
Por tanto, $\displaystyle a $ no puede ser un máximo ni un mínimo local.
\end{proof}
\begin{eg}
\normalfont 
\begin{itemize}
\item $\displaystyle f\left(x\right) = e^{x} $ en $\displaystyle a = 0 $. Tenemos que $\displaystyle \forall k \in \N $, $\displaystyle \left(e^{x}\right)^{\left(k\right)} = e^{0} = 1 $. Por tanto,
	\[P_{0,n}\left(x\right) =\sum^{n}_{k=0}\frac{x^{k}}{k!} .\]
\item Si $\displaystyle f\left(x\right) = \cos x $, tenemos que $\displaystyle f'\left(x\right) = -\sin x $, $\displaystyle f'''\left(x\right)=-\cos x $, $\displaystyle f'''\left(x\right) = \sin x $ y $\displaystyle f^{\left(4\right)}\left(x\right) = \cos x $. Así,
	\[f\left(0\right) = 1, \; f'\left(0\right) = 0, \; f''\left(0\right) = -1, \; f'''\left(0\right) = 0 .\]
Por tanto, tenemos que
\[P_{0,2n}\left(x\right) = \sum^{n}_{k=0}\frac{\left(-1\right)^{k}x^{2k}}{\left(2k\right)!} .\]
Análogamente, si $\displaystyle f\left(x\right) = \sin x $ se tiene que 
\[P_{0,2n +1}\left(x\right) = \sum^{n}_{k=0}\frac{\left(-1\right)^{k}x^{2k+1}}{\left(2k+1\right)!} .\]
\item Si $\displaystyle f\left(x\right) = \ln x $ tenemos que
	\[f'\left(x\right) = \frac{1}{x}, \; f''\left(x\right) = -\frac{1}{x^{2}}, \; f'''\left(x\right) = \frac{2}{x^{3}}, \; \cdots, \; f^{\left(k\right)}=\frac{\left(-1\right)^{k-1}\left(k-1\right)!}{x^{k}} .\]
Por tanto,
\[f^{\left(k+1\right)}\left(x\right) = \frac{\left(-1\right)^{k}k!}{x^{k+1}} .\]
Así, ha sido demostrado por inducción que 
\[f^{\left(k\right)}\left(x\right) = \frac{\left(-1\right)^{k-1}\left(k-1\right)!}{x^{k}} .\]
Por tanto, $\displaystyle f^{\left(k\right)}\left(1\right) = \left(-1\right)^{k-1}\left(k-1\right)! $. Por tanto, el polinomio de Taylor que buscamos es
\[P_{1,n}\left(x\right) = \sum^{n}_{k=1}\frac{\left(-1\right)^{k-1}}{k}\left(x-1\right)^{k} .\]
\end{itemize}
\end{eg}
\begin{fdefinition}[]
\normalfont Dadas $\displaystyle f,g : \left(a - \delta, a+ \delta \right) \to \R $, se dice que $\displaystyle f $ y $\displaystyle g $ son iguales en el punto $\displaystyle a $ hasta el orden $\displaystyle n $, con $\displaystyle n \in \N $, si 
\[\lim_{x \to a}\frac{f\left(x\right)-g\left(x\right)}{\left(x-a\right)^{n}} = 0 .\]
\end{fdefinition}
\begin{eg}
\normalfont Si $\displaystyle f : \left(a-\delta, a + \delta \right)\to \R $, que tiene $\displaystyle f', \ldots, f^{\left(n\right)} $ continuas en $\displaystyle \left(a-\delta, a + \delta \right) $, entonces $\displaystyle f $ y $\displaystyle P_{a,n}\left(x\right) $ son iguales hasta el orden $\displaystyle n $.
\end{eg}
\begin{observation}
\normalfont Consideremos que $\displaystyle \left|x-a\right| \leq 1 $. Entonces, para $\displaystyle k = 0, 1, \ldots, n $ se tiene que 
\[ \left|x-a\right|^{n} \leq \left|x-a\right|^{k} \leq \left|x-a\right| \leq 1 .\]
Entonces, se tiene que
\[ 1 \leq \frac{1}{ \left|x-a\right|} \leq \frac{1}{ \left|x-a\right|^{k}} \leq \frac{1}{ \left|x-a\right|^{n}} .\]
Por tanto, se tiene que
\[ \left|f\left(x\right)-g\left(x\right)\right| \leq \frac{ \left|f\left(x\right)-g\left(x\right)\right|}{ \left|x-a\right|^{k}} \leq \frac{ \left|f\left(x\right)-g\left(x\right)\right|}{ \left|x-a\right|^{n}} .\]
Así, se tiene que
\[0 = \lim_{x \to a}\frac{f\left(x\right)-g\left(x\right)}{ \left(x-a\right)^{n}} \iff \lim_{x \to a} \left|\frac{f\left(x\right)-g\left(x\right)}{\left(x-a\right)^{n}}\right| =0 .\]
Por tanto, $\displaystyle \lim_{x \to a}\frac{f\left(x\right)-g\left(x\right)}{ \left(x-a\right)^{k}} = 0 $, $\displaystyle \forall k = 0, 1, \ldots, n $. En particular $\displaystyle f\left(a\right) = g\left(a\right) $.
\end{observation}
\begin{flema}[]
\normalfont Sean $\displaystyle P $ y $\displaystyle Q $ dos polinomios de grado menor o igual que $\displaystyle n $. Si ambos son iguales en $\displaystyle a $ hasta el orden $\displaystyle n $, entonces $\displaystyle P = Q $.
\end{flema}
\begin{proof}
Tenemos que 
\[P\left(x\right) = a_{n}x^{n} + \cdots + a_{1}x + a_{0} = b_{n}\left(x-a\right)^{n} + \cdots + b_{1}\left(x-a\right) + b_{0} .\]
Donde $\displaystyle b_{k} = \frac{P^{\left(k\right)}\left(a\right)}{k!} $ y $\displaystyle a_{k} = \frac{P^{\left(k\right)}\left(0\right)}{k!} $. Similarmente,
\[Q\left(x\right) = c_{n}\left(x-a\right)^{n} + \cdots + c_{1}\left(x-a\right) + c_{0} .\]
Sea $\displaystyle R\left(x\right) = P\left(x\right)-Q\left(x\right) = d _{n}\left(x-a\right)^{n} + \cdots + d _{1}\left(x-a\right) + d _{0} $. Por hipótesis, tenemos que
\[0 = \lim_{x \to a}\frac{P\left(x\right)-Q\left(x\right)}{\left(x-a\right)^{n}} = \lim_{x \to a}\frac{R\left(x\right)}{\left(x-a\right)^{n}} .\]
Por tanto, para $\displaystyle k= 0 $,
\[0 = \lim_{x \to a}\frac{R\left(x\right)}{\left(x-a\right)^{0}} = \lim_{x \to a}R\left(x\right) = d _{0} .\]
Además, para $\displaystyle k= 1 $,
\[0 = \lim_{x \to a} \frac{R\left(x\right)}{x-a} = \lim_{x \to a}\frac{d _{n}\left(x-a\right)^{n} + \cdots + d _{1}\left(x-a\right)}{x-a} = d _{1}.\]
Así, por inducción se puede ver que $\displaystyle d _{k} = 0 $, $\displaystyle \forall k= 0, \ldots, n $, por lo que $\displaystyle R\left(x\right)=0 $ y $\displaystyle P\left(x\right) = Q\left(x\right) $.
\end{proof}
\begin{ftheorem}[]
\normalfont Sea $\displaystyle f : \left(a -\delta, a + \delta \right)\to \R $ tal que existen $\displaystyle f', \ldots, f^{\left(n\right)} $ continuas en $\displaystyle \left(a-\delta, a + \delta \right) $. Si $\displaystyle P $ es un polinomio de grado menor o igual que $\displaystyle n $, con $\displaystyle P $ igual a $\displaystyle f $ hasta el orden $\displaystyle n $, entonces $\displaystyle P = P_{a,n}\left(x\right) $.
\end{ftheorem}
\begin{proof}
Por hipótesis y por resultados anteriores tenemos que 
\[\lim_{x \to a}\frac{f\left(x\right)-P\left(x\right)}{\left(x-a\right)^{n}} = 0, \quad \lim_{x \to a}\frac{f\left(x\right)-P_{a,n}\left(x\right)}{\left(x-a\right)^{n}} = 0 .\]
Por tanto, calculamos el límite
\[
\begin{split}
	\lim_{x \to a}\frac{P\left(x\right)-P_{a,n}\left(x\right)}{\left(x-a\right)^{n}} = & \lim_{x \to a}\frac{P\left(x\right)-f\left(x\right) + f\left(x\right)-P_{a,n}\left(x\right)}{\left(x-a\right)^{n}} \\
	= &  \lim_{x \to a}\frac{P\left(x\right)-f\left(x\right)}{\left(x-a\right)^{n}} + \lim_{x \to a}\frac{f\left(x\right)-P_{a,n}\left(x\right)}{\left(x-a\right)^{n}} = 0 .
\end{split}
\]
Por el lema anterior, tenemos que $\displaystyle P\left(x\right) = P_{a,n}\left(x\right) $.
\end{proof}
\begin{observation}
\normalfont Este resultado nos permite dar otra definición del polinomio de Taylor: $\displaystyle P_{a,n} $ es el único polinomio centrado en $\displaystyle a $ de orden $\displaystyle n $ igual a $\displaystyle f $ hasta el orden $\displaystyle n $ en el punto $\displaystyle a $.
\end{observation}
\begin{eg}
\normalfont 
Consideremos $\displaystyle f\left(x\right) = \frac{1}{1 + x^{2}} $. Tenemos que 
	\[
	\begin{split}
		f\left(x\right) = & \frac{1}{1 + x^{2}} = \frac{1 +x^{2}-x^{2}}{1 + x^{2}} = 1 - \frac{x^{2}}{1 +x^{2}} = 1 - \frac{x^{2}+x^{4}-x^{4}}{1 +x^{2}} = 1 -x^{2} + \frac{x^{4}+x^{6}-x^{6}}{1 +x^{2}} \\
		= & 1 -x^{2}+x^{4}-\frac{x^{6}}{1 + x^{2}} = \cdots = \sum^{n}_{k=0}\left(-1\right)^{k}x^{2k} + \frac{\left(-1\right)^{n+1}x^{2n+2}}{1 +x^{2}}.
	\end{split}
	\]
Vamos a ver que $\displaystyle P_{0,2n}\left(x\right) = \sum^{n}_{k=0}\left(-1\right)^{k}x^{2k} $.
\[\lim_{x \to 0}\frac{\frac{1}{1+x^{2}}-\sum^{2n}_{k=0}\left(-1\right)^{k}x^{2k}}{x^{2n}} = \lim_{x \to 0}\frac{ \left(-1\right)^{n+1}\frac{x^{2n+2}}{1 +x^{2}}}{x^{2n}} = \lim_{x \to 0}\frac{\left(-1\right)^{n+1}x^{2}}{1+x^{2}} = 0 .\]
Por el teorema anterior, se tiene que $\displaystyle \sum^{n}_{k=0}\left(-1\right)^{k}x^{2k} $ aproxima a $\displaystyle \frac{1}{x^{2}+1} $ en cero hasta el orden $\displaystyle 2n $, por lo que es su polinomio de Taylor.
\end{eg}
\begin{eg}
\normalfont Sea $\displaystyle f\left(x\right) = \sum^{\infty}_{n=0}a_{n}\left(x-a\right)^{n}$. Supongamos que $\displaystyle \exists f\left(n\right)  $, $\displaystyle \forall n \in \N $. Vamos a ver que $\displaystyle P_{a,N} = \sum^{N}_{n=0}a_{n}\left(x-a\right)^{n} $. En efecto, tenemos que
\[\lim_{x \to a}\frac{f\left(x\right)-\sum^{N}_{n = 0}a_{n}\left(x-a\right)^{n}}{\left(x-a\right)^{N}} = \lim_{x \to a}\frac{\sum^{\infty}_{n = N+1}a_{n}\left(x-a\right)^{n}}{\left(x-a\right)^{N}} = \lim_{x \to a}\sum^{\infty}_{n = N+1}a_{n}\left(x-a\right)^{n - N} \]
Si $\displaystyle j = n - N $,
\[= \lim_{x \to a}\sum^{\infty}_{j=1}a_{N+j}\left(x-a\right)^{j} \stackrel{?}{=} 0 .\]
Esto lo demostraremos más adelante.
\end{eg}
\begin{fdefinition}[]
\normalfont Dada $\displaystyle f : \left(a-\delta, a + \delta \right)\to \R $, $\displaystyle n $ veces derivable en $\displaystyle x = a $, se define el \textbf{resto} de $\displaystyle f $ de grado $\displaystyle n $ centrado en $\displaystyle a $ 
\[R_{a,n}\left(x\right) = f\left(x\right)-P_{a,n}\left(x\right) .\]
\end{fdefinition}
\begin{observation}
\normalfont El error que se comete al tomar $\displaystyle P_{a,n}\left(x\right) $ en lugar de $\displaystyle f\left(x\right) $ es $\displaystyle \left|R_{a,n}\left(x\right)\right| $.
\end{observation}
\begin{eg}
\normalfont Como se vio anteriormente, 
\[\frac{1}{1 + x^{2}} = \underbrace{\sum^{n}_{k=0}\left(-1\right)^{k}x^{2k}}_{P_{0,2n}} + \underbrace{\frac{\left(-1\right)^{n+1}x^{2n+2}}{1 + x^{2}}}_{R_{0,2n}} .\]
\end{eg}
\begin{ftheorem}[Teorema de Taylor]
	\normalfont Sea $\displaystyle f : \left[a,x\right] \to \R $ tal que existen $\displaystyle f', \ldots, f^{\left(n+1\right)} $ en $\displaystyle \left[a,x\right]  $.
	\begin{description}
		\item[(a)] \textbf{Fórmula del resto de Cauchy.} Para algún $\displaystyle t \in \left[a,x\right]  $,
		\[ R_{a,n}\left(x\right) = \frac{f^{\left(n+1\right)} \left(t\right)}{n!}\left(x-t\right)^{n}\left(x-a\right) .\]
	\item[(b)] \textbf{Fórmula del resto de Lagrange.} Para algún $\displaystyle t \in \left[a,x\right]  $, 
		\[R_{a,n}\left(x\right) = \frac{f^{\left(n+1\right)}\left(t\right)}{\left(n+1\right)!} \left(x - a\right)^{n+1}.\]
	\item[(c)] \textbf{Fórmula integral del resto.} Si $\displaystyle f^{\left(n+1\right)} $ es integrable en $\displaystyle \left[a,x\right]  $, entonces
		\[R_{a,n}\left(x\right) = \int^{x}_{a} \frac{f^{\left(n+1\right)}\left(t\right)}{n!}\left(x-t\right)^{n} \; dt .\]
	\end{description}
\end{ftheorem}
\begin{proof}
	Para $\displaystyle t \in \left[a,x\right]  $, definimos
	\[S\left(t\right) = R_{t,n}\left(x\right) = f\left(x\right) - P_{t,n}\left(x\right) = f\left(x\right)-\sum^{n}_{k=0}\frac{f^{\left(k\right)}}{k!}\left(x-t\right)^{k}, \; t \in \left[a,x\right]  .\]
Ahora, consideremos 
\[g\left(t\right) = \left(x-t\right)^{n+1}, \; t \in \left[a,x\right]  .\]
Derivamos $\displaystyle S\left(t\right) $ y tomamos $\displaystyle j = k-1 $,
\[
\begin{split}
	S'\left(t\right) = & - \sum^{n}_{k=0}\frac{f^{\left(k+1\right)}\left(t\right)}{k!}\left(x-t\right)^{k} + \sum^{n}_{k=1}\frac{f^{\left(k\right)}\left(t\right)}{k!}k\left(x-t\right)^{k-1} = -\sum^{n}_{k=0}\frac{f^{\left(k+1\right)}\left(t\right)}{k!}\left(x-t\right)^{k}+\sum^{n-1}_{j=0}\frac{f^{\left(j+1\right)}\left(t\right)}{j!}\left(x-t\right)^{j} \\
	= & -\frac{f^{\left(n+1\right)}\left(t\right)}{n!}\left(x-t\right)^{n}.
\end{split}
\]
Ahora derivamos $\displaystyle g $,
\[g'\left(t\right) = -\left(n+1\right)\left(x-t\right)^{n} .\]
\begin{description}
	\item[(a)] Por el teorema del valor medio, tenemos que existe $\displaystyle t \in \left(a,x\right)  $ tal que 
		\[S\left(x\right)-S\left(a\right) = S'\left(t\right)\left(x-a\right)=-\frac{f^{\left(n+1\right)}\left(t\right)}{n!}\left(x-t\right)^{n}\left(x-a\right) .\]
Tenemos que $\displaystyle S\left(x\right) = 0 $ y $\displaystyle S\left(a\right) = R_{a,n}\left(x\right) $. Así, nos queda que
\[ R_{a,n}\left(x\right) = - \frac{f^{\left(n+1\right)}\left(t\right)}{n!}\left(x-t\right)^{n}\left(x-a\right).\]
\item[(b)] Por el teorema del valor medio de Cauchy, tenemos que existe $\displaystyle t \in \left(a,x\right)  $ tal que
	\[\frac{S\left(x\right)-S\left(a\right)}{g\left(x\right)-g\left(a\right)}= \frac{S'\left(t\right)}{g'\left(t\right)} \iff \frac{S\left(a\right)}{g\left(a\right)} = \frac{-\frac{f^{\left(n+1\right)}\left(t\right)}{n!}\left(x-t\right)^{n}}{-\left(n+1\right)\left(x-t\right)^{n}}=\frac{f^{\left(n+1\right)}\left(t\right)}{\left(n+1\right)!}.\]
	Hemos usado que $\displaystyle S\left(x\right)= g\left(x\right) = 0 $. Por tanto, se tiene que 
	\[S\left(a\right) = R_{a,n}\left(x\right) = \frac{f^{\left(n+1\right)}\left(t\right)}{\left(n+1\right)!} \left(x - a\right)^{n+1} .\]
\item[(c)] Si $\displaystyle f^{\left(n+1\right)}\left(t\right) $ es integrable en $\displaystyle \left[a,x\right]  $ (es decir, $\displaystyle S' $ es integrable), por el segundo teorema de Barrow tenemos que 
	\[S\left(x\right)-S\left(a\right) = \int^{x}_{a} S'\left(t\right) \; dt .\]
	Dado que $\displaystyle S\left(x\right) = 0 $ y $\displaystyle S\left(a\right) = R_{a,n}\left(x\right) $, 
\[R_{a,n}\left(x\right) = \int^{x}_{a} \frac{f^{\left(n+1\right)}\left(t\right)}{n!}\left(x-t\right)^{n} \; dt .\]	
\end{description}
\end{proof}
\begin{observation}
	\normalfont Existe un resultado análogo para $\displaystyle x < a $, en $\displaystyle \left[x,a\right]  $.
\end{observation}
\begin{eg}
\normalfont Vamos a calcular $\displaystyle \sin 1 $. Recordamos que $\displaystyle \sin x = P_{0,2n+1}\left(x\right) + R_{0,2n+1} $, así
\[\sin x = \sum^{n}_{k=0}\frac{\left(-1\right)^{k}x^{2k+1}}{\left(2k+1\right)!}+\int^{x}_{0} \frac{\left(-1\right)^{n}\sin ^{\left(2n+2\right)}\left(t\right)}{\left(2n+1\right)!}\left(x-t\right)^{2n+1} \; dt .\]
Así, obtenemos que
\[\sin 1 = \sum^{n}_{k=0}\frac{\left(-1\right)^{k}}{\left(2k+1\right)!} + \int^{1}_{0} \frac{\left(-1\right)^{n}\sin ^{\left(2n+2\right)}\left(t\right)}{\left(2n+1\right)!}\left(1-t\right)^{2n+1} \; dt.\]
Para calcular el error,
\[ \left|\int^{x}_{0} \frac{\left(-1\right)^{n}\sin ^{\left(2n+2\right)}\left(t\right)}{\left(2n+1\right)!}\left(x-t\right)^{2n+1} \; dt \right| \leq \int^{x}_{0} \frac{ \left|x-t\right|^{2n+1}}{\left(2n+1\right)!} \; dt  = \left[\frac{ \left|x-t\right|^{2n+2}}{\left(2n+2\right)!}\right] ^{x}_{0}= \frac{ \left|x\right|^{2n+2}}{\left(2n+2\right)!} .\]
Si $\displaystyle x = 1 $ se tiene que el error será menor que $\displaystyle \frac{1}{\left(2n+2\right)!} $. Así, si queremos que el error sea menor que $\displaystyle 10^{-2} $, encontramos $\displaystyle n \in \N$ tal que $\displaystyle \frac{1}{\left(2n+2\right)!} < 10^{-2} $. En este caso, basta coger $\displaystyle n=5 $. Así,
\[\sin 1 \approx 1 - \frac{1}{3!} + \frac{1}{5!} .\]
\end{eg}
\section{Series de Taylor}
\begin{fdefinition}[Serie de Taylor]
\normalfont Sea $\displaystyle f : \left(a-\delta, a + \delta \right)\to \R $ tal que existe $\displaystyle f^{\left(n\right)}\left(a\right) $, $\displaystyle \forall n \in \N $. Se llama \textbf{serie de Taylor} de $\displaystyle f $ centrada en $\displaystyle a $ a la serie $\displaystyle \sum^{\infty}_{k = 0}\frac{f^{\left(k\right)}\left(a\right)}{k!}\left(x-a\right)^{k} $.
\end{fdefinition}
\begin{observation}
\normalfont Por definición, tenemos que 
\[\sum^{\infty}_{n=0}\frac{f^{\left(n\right)}\left(a\right)}{n!}\left(x-a\right)^{n}=\lim_{n \to \infty}\sum^{n}_{k=0}\frac{f^{\left(k\right)}\left(a\right)}{k!}\left(x-a\right)^{k} = \lim_{n \to \infty}P_{a,n}\left(x\right) .\]
Recordamos que $\displaystyle f\left(x\right)-P_{a,n}\left(x\right) = R_{a,n}\left(x\right) $.
\end{observation}
\begin{fprop}[]
\normalfont Sea $\displaystyle f : \left(a-\delta, a + \delta \right)\to \R $ tal que existe $\displaystyle f^{\left(n\right)} \left(a\right)$, $\displaystyle \forall n \in \N $. Si $\displaystyle \lim_{n \to \infty}R_{a,n}\left(x\right) = 0 $, para $\displaystyle x \in \left(a-\delta, a + \delta \right) $, entonces $\displaystyle f\left(x\right)= \sum^{\infty}_{k=0}\frac{f^{\left(k\right)}\left(a\right)}{k!}\left(x-a\right)^{k} $.
\end{fprop}
\begin{proof}
Como $\displaystyle P_{a,n}\left(x\right) = f\left(x\right)-R_{a,n}\left(x\right) $, tomando límites
\[\lim_{n \to \infty}P_{a,n}\left(x\right) = \lim_{n \to \infty}f\left(x\right)-R_{a,n}\left(x\right) =f\left(x\right) .\]
\end{proof}
\begin{eg}
\normalfont Consideremos $\displaystyle f\left(x\right)= e^{x} $ centrada en 0. Su serie de Taylor es $\displaystyle \sum^{\infty}_{k=0}\frac{x^{k}}{k!} $. Así, tenemos que
\[f\left(x\right) = \sum^{n}_{k=0}\frac{x^{k}}{k!}+R_{0,n}\left(x\right) = \sum^{n}_{k=0}\frac{x^{k}}{k!}+\int^{x}_{0} \frac{f^{\left(n+1\right)}\left(s\right)}{n!}\left(x-s\right)^{n} \; d s .\]
Tenemos que 
\[ \left|\int^{x}_{0} \frac{f^{\left(n+1\right)}\left(s\right)}{n!}\left(x-s\right)^{n} \; d s\right| \leq \int^{x}_{0} \frac{e^{s}}{n!} \left|x-s\right|^{n} \; d s \leq \max \left\{ 1,e^{x}\right\} \int^{x}_{0} \frac{ \left|x-s\right|^{n}}{n!} \; d s = \max \left\{ 1,e^{x}\right\} \frac{ \left|x\right|^{n+1}}{\left(n+1\right)!} .\]
Así, si $\displaystyle x \in \left[-M,M\right]  $ para cualquier $\displaystyle M > 0 $ se tiene que
\[ \left|R_{a,n}\left(x\right)\right| \leq \max \left\{ 1,e^{x}\right\} \frac{ \left|x\right|^{n+1}}{\left(n+1\right)!} \leq \max \left\{ 1,e^{M}\right\} \frac{M^{n+1}}{\left(n+1\right)!} \to 0 .\]
 Aplicando el criterio del cociente tenemos que $\displaystyle \sum^{\infty}_{n=0}\frac{M^{n+1}}{\left(n+1\right)!} < \infty $. Así, tenemos que $\displaystyle R_{a,n}\left(x\right) \to 0 $ y, consecuentemente, 
\[e^{x} = \sum^{\infty}_{k=0}\frac{x^{k}}{k!} .\]
\end{eg}
\begin{eg}
\normalfont De forma análoga, se ve que $\displaystyle \forall x \in \R $ 
\[\cos x = \sum^{\infty}_{k=0}\frac{\left(-1\right)^{k}x^{2k}}{\left(2k\right)!} \quad \text{y} \quad \sin x = \sum^{\infty}_{k=0}\frac{\left(-1\right)^{k}x^{2k+1}}{\left(2k+1\right)!} .\]
\end{eg}
\begin{eg}
\normalfont Consideremos la función 
\[f\left(x\right) = 
\begin{cases}
e^{-\frac{1}{x^{2}}}, \; x \neq 0 \\
0, \; x = 0
\end{cases}
.\]
En un ejercicio demostramos que $\displaystyle \forall n \in\N $, $\displaystyle \exists f^{\left(n\right)}\left(0\right) = 0 $. Así, su serie de Taylor será, $\displaystyle \sum^{\infty}_{k=0}\frac{f^{\left(k\right)}\left(0\right)}{k!}x^{k} = 0 $. Es decir, hemos encontrado una función $\displaystyle f $ cuyo polinomio de Taylor solo coincide con $\displaystyle f $ en el punto $\displaystyle a = 0 $. Es decir, es una función cuya serie de Taylor no nos aporta información.
\end{eg}
\begin{eg}
\normalfont Consideremos la serie de Taylor de $\displaystyle f\left(x\right)=e^{x} $ centrada en $\displaystyle 1 $. Podemos hacer,
\[e^{x} = e^{\left(x-1\right)+1} = e e^{x-1} = e \sum^{\infty }_{k=0}\frac{\left(x-1\right)^{k}}{k!} = \sum^{\infty}_{k=0}\frac{e\left(x-1\right)^{k}}{k!} .\]
Para ver que se trata efectivamente de la serie de Taylor, basta con ver que 
\[\lim_{x \to 1}\frac{e^{x}-\sum^{n}_{k=0}\frac{e\left(x-1\right)^{k}}{k!}}{\left(x-1\right)!} = 0 .\]
Así, podemos identificar su polinomio de Taylor y, consecuentemente, su serie de Taylor.
\end{eg}
\begin{eg}
\normalfont Sea $\displaystyle g\left(x\right) = \left(x-1\right)^{5}e^{x} $. Vamos a calcular su serie de Taylor centrada en 1. Tenemos que 
\[g\left(x\right) = \left(x-1\right)^{5}e^{x} = \left(x-1\right)^{5}\sum^{\infty}_{k = 0}\frac{e\left(x-1\right)^{k}}{k!} = \sum^{\infty}_{k=0}\frac{e\left(x-1\right)^{k+5}}{k!} .\]
Si tomamos $\displaystyle j= k + 5 $ se tiene que 
\[g\left(x\right)=\sum^{\infty}_{j=5}\frac{e\left(x-1\right)^{j}}{\left(j-5\right)!} .\]
\end{eg}
\begin{eg}
\normalfont Consideremos $\displaystyle f\left(x\right) = \frac{1}{1 + x^{2}} $. Recordamos que
\[ \frac{1}{1 + x^{2}} = \sum^{n}_{k=0}\left(-1\right)^{k}x^{2k} + \left(-1\right)^{n+1}\frac{x^{2n+2}}{1+x^{2}} .\]
Entonces tenemos que, si $\displaystyle \left|x\right| < 1 $,
\[\lim_{n \to \infty}\left(-1\right)^{n+1}\frac{x^{2n+2}}{1+x^{2}} = 0.\]
Así, tenemos que 
\[\frac{1}{1 + x^{2}} = \sum^{\infty}_{k=0}\left(-1\right)^{k}x^{2k}, \; x \in \left(-1,1\right) .\]
\end{eg}
\begin{eg}
\normalfont Calculemos la serie de Taylor de $\displaystyle f\left(x\right) = \arctan x $. Tenemos que $\displaystyle \arctan x = \int^{x}_{0} \frac{1}{1 + s^{2}} \; d s $. Por tanto,
\[\arctan x = \int^{x}_{0} \sum^{n}_{k=0}\left(-1\right)^{k}s^{2k} + \left(-1\right)^{n+1}\frac{s^{2n+2}}{1+s^{2}} \; d s = \sum^{n}_{k=0}\left(-1\right)^{k}\frac{x^{2k+1}}{2k+1} + \int^{x}_{0} \left(-1\right)^{n+1}\frac{s^{2n+2}}{1+s^{2}} \; d s .\]
Para ver que el sumatorio resultante es el polinomio de Taylor habría que comprobarlo con el límite. Ahora, tenemos que si $\displaystyle \left|x\right| < 1 $,
\[ \left|\int^{x}_{0} \frac{\left(-1\right)^{n+1}s^{2n+2}}{1+s^{2}} \; d s\right| \leq \int^{x}_{0} \left|s\right|^{2n+2} \; d s = \frac{x^{2n+3}}{2n+3} \to 0.\]
En caso contrario, $\displaystyle \lim_{n \to \infty}R_{0,n} = \infty $. Así, se obtiene que 
\[\arctan x = \sum^{\infty}_{k=0}\left(-1\right)^{k}\frac{x^{2k+1}}{2k+1}, \; x \in \left(-1,1\right) .\]
\end{eg}
\section{Método de las tangentes de Newton}
Sea la ecuación $\displaystyle f\left(x\right) = 0 $. Queremos encontrar $\displaystyle x = f^{-1}\left(0\right) $. Sea $\displaystyle f : \R \to \R $ derivable. Sea $\displaystyle x_{0} \in \dom\left(f\right) $ y sea $\displaystyle y = f'\left(x_{0}\right)\left(x-x_{0}\right) + f\left(x_{0}\right) $ la recta tangente. Calculamos el corte de la recta tangente con la recta $\displaystyle y = 0 $. Es decir, tenemos la ecuación 
\[ 0 = f'\left(x_{0}\right)\left(x-x_{0}\right) + f\left(x_{0}\right) \Rightarrow x = -\frac{f\left(x_{0}\right)}{f'\left(x_{0}\right)}+x_{0} .\]
Tomamos $\displaystyle x_{1} = -\frac{f\left(x_{0}\right)}{f'\left(x_{0}\right)}+x_{0} $. Si $\displaystyle x_{1} \in \dom\left(f\right) $, repetimos el procedimiento, obteniendo la sucesión recurrente
\[x_{n} = x_{n-1}-\frac{f\left(x_{n-1}\right)}{f'\left(x_{n-1}\right)} .\]
Tenemos que ver que $\displaystyle x_{n} \to r $ tal que $\displaystyle f\left(r\right) = 0 $.
\begin{ftheorem}[Método de Newton]
	\normalfont Sea $\displaystyle f : \left[a,b\right] \to \R $ dos veces derivable y con $\displaystyle f\left(a\right)f\left(b\right)<0 $, tal que existen $\displaystyle m,M > 0 $ tales que
	\[ 0 < m \leq \left|f'\left(x\right)\right| \; \text{y} \; \left|f''\left(x\right)\right|\leq M, \; \forall x \in \left[a,b\right]  .\]
	Sea $\displaystyle r \in \left[a,b\right]  $ la única raíz de $\displaystyle f\left(x\right) = 0 $ en $\displaystyle \left[a,b\right]  $. Existe un subintervalo $\displaystyle I \subset \left[a,b\right]  $ con $\displaystyle r \in I $ tal que $\displaystyle \forall x_{0} \in I $ la sucesión recurrente definida por
	\[x_{n+1} = x_{n} - \frac{f\left(x_{n}\right)}{f'\left(x_{n}\right)},\]
	converge a $\displaystyle r $. En particular,
	\[ \left|x_{n+1}-r\right| \leq \frac{M}{2m} \left|x_{n}-r\right|^{2}, \; \forall n \in \N .\]
\end{ftheorem}
\begin{proof}
	Como $\displaystyle f\left(a\right)f\left(b\right) < 0 $, el teorema de Bolzano nos dice que existe $\displaystyle r \in \left(a,b\right) $ tal que $\displaystyle f\left(r\right) = 0 $. Además, como $\displaystyle f' $ es continua por existir $\displaystyle f'' $ y $\displaystyle f' \neq 0 $, se sigue que en una vecindad de $\displaystyle r $, $\displaystyle f $ es inyectiva en $\displaystyle \left[a,b\right]  $. \\ 
	Sea $\displaystyle K = \frac{M}{2m} $ y sea $\displaystyle \delta > 0 $ tal que $\displaystyle \delta < \frac{1}{K} $ e $\displaystyle I = \left[r-\delta, r + \delta \right] \subset \left[a,b\right] $.
	Ahora, si $\displaystyle x_{0} \in I $, la fórmula del resto de Lagrange del teorema de Taylor nos dice que existe $\displaystyle c \in \left(x_{0}, r\right) $ tal que
	\[f\left(r\right) = f\left(x_{0}\right) + f'\left(x_{0}\right)\left(r-x_{0}\right) + \frac{1}{2}f''\left(c\right)\left(r-x_{0}\right)^{2} .\]
	Como $\displaystyle f\left(r\right) = 0 $, tenemos que
	\[ -f\left(x_{0}\right) = f'\left(x_{0}\right)\left(r-x_{0}\right)+\frac{1}{2}f''\left(c\right)\left(r-x_{0}\right)^{2} .\]
	Por tanto, se deduce que 
	\[-\frac{f\left(x_{0}\right)}{f'\left(x_{0}\right)} = \left(r-x_{0}\right) + \frac{f''\left(c\right)}{2f'\left(x_{0}\right)}\left(r-x_{0}\right)^{2} .\]
Ahora, consideramos $\displaystyle x_{1} = x_{0} - \frac{f\left(x_{0}\right)}{f'\left(x_{0}\right)} $, entonces 
\[x_{1} = x_{0} + \left(r-x_{0}\right)+\frac{f''\left(c\right)}{2f'\left(x_{0}\right)}\left(r-x_{0}\right)^{2} .\]
Tomando valores absolutos y acotando se tiene que
\[ \left|x_{1}-r\right| = \left|\frac{f''\left(c\right)}{2f'\left(x_{0}\right)}\left(r-x_{0}\right)^{2}\right| \leq \frac{M}{2m}\delta ^{2} < \delta  .\]
En la última desigualdad hemos utilizado el hecho de que $\displaystyle K = \frac{M}{2m} < \frac{1}{\delta } $. Así, tenemos que $\displaystyle x_{1} \in I $. Similarmente, para cualquier $\displaystyle x_{n} \in I $ se tiene que $\displaystyle x_{n} -\frac{f\left(x_{n}\right)}{f'\left(x_{n}\right)} \in I $. Así, si $\displaystyle x_{0} \in I $, la sucesión recurrente $\displaystyle \left\{ x_{n}\right\} _{n\in\N} $ con $\displaystyle x_{n+1} = x_{n} -\frac{f\left(x_{n}\right)}{f'\left(x_{n}\right)} $ está dentro del intervalo $\displaystyle I $.
Luego, con la acotación de arriba e iterando, dado que $\displaystyle K\delta < 1 $, obtenemos que \footnote{La desigualdad anterior se justifica de la siguiente manera: $\displaystyle K\delta \left|x_{n}-r\right| \leq \left(K\delta \right)K \left|x_{n-1}-r\right|^{2} \leq \left(K\delta \right)^{2} \left|x_{n}-r\right| $. } 
\[ K\delta \left|x_{n}-r\right| \leq \left(K\delta \right)^{2} \left|x_{n-1}-r\right| \leq \cdots \leq \left(K\delta \right)^{n} \left|x_{1}-r\right| \to 0 .\]
Es decir, $\displaystyle x_{n} \to r $.
\end{proof}
\begin{observation}
\normalfont El problema de este teorema es que para encontrar $\displaystyle I $ donde tomar el primer término de la sucesión, necesitamos conocer primero $\displaystyle r $, la raíz que buscamos.
\end{observation}
\begin{fdefinition}[Función contractiva]
	\normalfont Se dice que $\displaystyle f : \left[a,b\right] \to \R $ es \textbf{contractiva} si $\displaystyle \exists K \in \left(0,1\right) $ tal que $\displaystyle \left|f\left(x\right)-f\left(y\right)\right| \leq K \left|x-y\right| $, $\displaystyle \forall x,y \in \left[a,b\right]  $.
\end{fdefinition}
\begin{flema}[Teorema de punto fijo de Banach]
	\normalfont Sea $\displaystyle f : \left[a,b\right] \to \left[a,b\right]  $ una función contractiva de modo que la sucesión $\displaystyle x_{n+1} = f\left(x_{n}\right) $, con $\displaystyle n \geq 1 $ y $\displaystyle x_{1} \in \left[a,b\right]  $, tiene a $\displaystyle x^{*} $ por límite, es decir, $\displaystyle \lim_{n \to \infty}x_{n} = x^{*} $ y además \footnote{Aquí, $\displaystyle K $ procede de la definición de contractividad.} 
	\[ \left|x_{n}-x^{*}\right| \leq \frac{K^{n}}{1-K} \left|x_{0}-x_{1}\right| .\]
\end{flema}
\begin{observation}
\normalfont Si $\displaystyle f $ es una función con $\displaystyle \left|f'\left(x\right)\right|< K $, $\displaystyle \forall x \in \dom\left(f\right) $, entonces por el teorema del valor medio existe $\displaystyle \xi \in \left(x,y\right) $ tal que 
\[ \left|f\left(x\right)-f\left(y\right)\right| = \left|f'\left(\xi\right)\right| \left|x-y\right| \leq K \left|x-y\right| .\]
Por tanto, se tiene que $\displaystyle f $ es contractiva.
\end{observation}
\begin{proof}
	Sea $\displaystyle x_{0} \in \left[a,b\right]  $ y se define $\displaystyle x_{n+1} = f\left(x_{n}\right) $ para $\displaystyle n \in \N $. Vamos a ver que $\displaystyle \left\{ x_{n}\right\} _{n\in\N} $ es de Cauchy. Sea $\displaystyle m > n $, 
	\[ \left|x_{m}-x_{n}\right| = \left|f\left(x_{m - 1} \right)- f\left(x_{n-1}\right)\right| \leq K \left|x_{m - 1} - x_{n-1}\right| .\]
	Iterando, obtenemos que 
\[
\begin{split}
	\left|x_{m}-x_{n}\right| & \leq K^{n} \left|x_{m - n}-x_{0}\right| \leq K^{n} \left( \left|x_{m - n}-x_{m - n - 1}\right| + \cdots + \left|x_{2}-x_{1}\right| + \left|x_{1}-x_{0}\right|\right) .\\
	\leq & K^{n} \left(K^{m - n - 1} \left|x_{0}-x_{1}\right| + K^{m - n - 2} \left|x_{0}-x_{1}\right| + \cdots + K \left|x_{1}-x_{0}\right| + \left|x_{1}-x_{0}\right|\right) \\
	= & K^{n} \left|x_{1}-x_{0}\right| \left(K^{m - n - 1} + \cdots + K + 1\right) \leq K^{n} \left|x_{1}-x_{0}\right| \frac{1}{1- K} \to 0.
\end{split}
\]
Hemos usado que $\displaystyle K \in \left(0,1\right) $. Así, tenemos que si $\displaystyle \epsilon > 0 $, $\displaystyle \exists n_{0} \in \N $ tal que si $\displaystyle n \geq n_{0} $, entonces
\[ \left|K^{n} \left|x_{1}-x_{0}\right|\frac{1}{1-K}\right| < \epsilon .\]
Así, si $\displaystyle m,n \geq n_{0} $, 
\[ \left|x_{m}-x_{n}\right| \leq K^{n} \left|x_{1}-x_{0}\right|\frac{1}{1-K} < \epsilon  .\]
Por tanto, la sucesión es de Cauchy. Así, tenemos que $\displaystyle \exists x^{*} \in \left[a,b\right]  $ tal que $\displaystyle x_{n} \to x^{*} $. Fijando $\displaystyle x_{m} \to x^{*} $ si $\displaystyle m \to \infty $, tenemos que
\[ \left|x_{n}-x^{*}\right| \leq K^{n} \left|x_{1}-x_{0}\right|\frac{1}{1-K} .\]
Por ser contractiva, tenemos que $\displaystyle f $ es continua, por lo que si $\displaystyle x_{n} \to x^{*} $, entonces $\displaystyle f\left(x_{n}\right) \to f\left(x^{*}\right) $. Así, tenemos que $\displaystyle x^{*} = f\left(x^{*}\right) $.
\end{proof}
\begin{observation}
\normalfont No puede haber otro $\displaystyle y^{*} \neq x^{*}$ que cumpla las mismas propiedades. En efecto, si lo hubiera se tendría que
\[ \left|x^{*}-y^{*}\right| = \left|f\left(x^{*}\right)-f\left(y^{*}\right)\right| \leq K \left|x^{*}-y^{*}\right| .\]
Esto es imposible puesto que $\displaystyle K \in\left(0,1\right) $.
\end{observation}
\begin{ftheorem}[]
	\normalfont Sea $\displaystyle f: \left[a,b\right] \to \R $ derivable tal que $\displaystyle f\left(a\right)f\left(b\right) < 0 $. Además, existen $\displaystyle m,M > 0 $ tales que 
	\[ 0 < m < \left|f\left(x\right)\right| < M , \; \forall x \in \left[a,b\right]  .\]
	Entonces, dado $\displaystyle x_{0} \in \left[a,b\right]  $, la sucesión $\displaystyle x_{n+1} = x_{n}-\frac{f'\left(x_{n}\right)}{M} $ converge a la única raíz $\displaystyle x^{*} \in \left[a,b\right]  $ de $\displaystyle f $ ($\displaystyle f\left(x^{*}\right) = 0 $) y además
	\[ \left|x^{*}-x_{n}\right| \leq \frac{ \left|f\left(x_{0}\right)\right|}{m}\left(1-\frac{m}{M}\right)^{n} .\]
\end{ftheorem}
\begin{proof}
Supongamos que $\displaystyle f' > 0 $ y $\displaystyle f $ es creciente. Como $\displaystyle f\left(a\right)f\left(b\right) < 0 $, tenemos que $\displaystyle f\left(a\right) < 0 $. Si $\displaystyle f' < 0 $, $\displaystyle f $ es decreciente y $\displaystyle f\left(a\right) > 0 $ y $\displaystyle f\left(b\right) < 0 $. Además, como $\displaystyle f' \neq 0 $, tenemos que $\displaystyle f $ es inyectiva. Por el teorema de Bolzano, existe $\displaystyle x^{*} $ tal que $\displaystyle f\left(x^{*}\right) = 0 $. Por ser $\displaystyle f $ inyectiva, este punto es único.\\ 
Sea la función 
\[
\begin{split}
	\varphi : \left[a,b\right] & \to \R\\
	x & \to \varphi\left(x\right) = x - \frac{f\left(x\right)}{M}.
\end{split}
\]
Tenemos que 
\[\varphi'\left(x\right) = 1 - \frac{f'\left(x\right)}{M} < 1 - \frac{m}{M} = K \in \left(0,1\right) .\]
Si $\displaystyle f' > 0 $, tenemos que $\displaystyle \varphi $ es creciente y contractiva. En efecto, por el teorema del valor medio existe $\displaystyle \xi \in \left(x,y\right) $ tal que
\[ \left|\varphi\left(x\right)-\varphi\left(y\right)\right| \leq \left|\varphi'\left(\xi\right)\right| \left|x-y\right| < K \left|x-y\right|.\]
Ahora, tenemos que 
\[a < a - \frac{f\left(a\right)}{M} = \varphi\left(a\right) < \varphi\left(x\right) < \varphi\left(b\right) = b - \frac{f\left(b\right)}{M} < b \Rightarrow \varphi\left(x\right) \in \left[a,b\right]  .\]
Por el lema anterior, tenemos que dado $\displaystyle x_{0} \in \left[a,b\right]  $,
\[x_{n+1} = \varphi\left(x_{n}\right) = x_{n}-\frac{f\left(x_{n}\right)}{M} \to x^{*} = x^{*} - \frac{f\left(x^{*}\right)}{M} \Rightarrow f\left(x^{*}\right) = 0.\]
Además, 
\[ \left|x_{n}-x^{*}\right| \leq \frac{K^{n}}{1-K} \left|x_{1}-x_{0}\right| = \frac{\left(1-\frac{m}{M}\right)^{n}}{\frac{m}{M}} \left|x_{0}-\frac{f\left(x_{0}\right)}{M}-x_{0}\right| = \left(1-\frac{m}{M}\right)^{n}\frac{ \left|f\left(x_{0}\right)\right|}{m} .\]
\end{proof}

