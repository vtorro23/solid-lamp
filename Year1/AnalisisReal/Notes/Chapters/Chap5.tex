\chapter{La Derivada}
Recordamos que la ecuación de la recta que pasa por dos puntos $\displaystyle \left(x_{1}, y_{1}\right), \left(x_{2}, y_{2}\right) \in \R^{2} $ será
\[ y = \frac{y_{2}-y_{1}}{x_{2}-x_{1}}\left(x - x_{1}\right) + y_{1} .\]
\begin{fdefinition}[Derivada]
\normalfont Sea $\displaystyle f : A \subset \R \to \R $ y sea $\displaystyle a \in \dom\left(f\right) $ tal que $\displaystyle \exists r > 0 $ tal que $\displaystyle \left(a-r, a + r\right) \subset \dom\left(f\right) $. Se dice que $\displaystyle f $ es \textbf{derivable} en el punto $\displaystyle x = a $ si existe 
\[\lim_{h \to 0}\frac{f\left(a + h\right)-f\left(a\right)}{h} = f'\left(a\right) .\]
Entonces, se dice que $\displaystyle f $ es \textbf{derivable} en $\displaystyle a $. 
\end{fdefinition}
\begin{observation}
\normalfont Tenemos que la derivabilidad, al igual que la continuidad, es una propiedad local. Se dice que $\displaystyle f $ es derivable en $\displaystyle A \subset \dom\left(f\right) $ si $\displaystyle f $ es derivable en cada $\displaystyle a \in A $.
\end{observation}
\begin{observation}
\normalfont 
Se llama función derivada de $\displaystyle f $ a la función 
\[
\begin{split}
	f' : \R \to & \R \\
	x \to & f'\left(x\right).
\end{split}
\]
Tenemos que $\displaystyle \dom\left(f'\right) = \left\{ x \; : \; f\left(x\right) \; \text{derivable en }x\right\}  $.
\end{observation}
\begin{eg}
\normalfont 
\begin{itemize}
\item Si consideramos $\displaystyle f\left(x\right) = a \in \R $, tenemos que 
	\[f'\left(x_{0}\right) = \lim_{h \to 0}\frac{f\left(x_{0}+h\right)-f\left(x_{0}\right)}{h} = \lim_{h \to 0}\frac{a - a}{h} = 0 .\]
\item Consideremos $\displaystyle f\left(x\right) = x $. Tenemos que
	\[\lim_{h \to 0}\frac{f\left(x_{0}+h\right)-f\left(x_{0}\right)}{h} = \lim_{h \to 0}\frac{x_{0}+h-x_{0}}{h} = 1 .\]
\item La función $\displaystyle f\left(x\right) = \left|x\right| $ no es derivable en $\displaystyle x = 0 $.
	\[ \lim_{h \to 0}\frac{ \left|0 + h\right|- \left|0\right|}{h} = \lim_{h \to 0}\frac{ \left|h\right|}{h} = 
	\begin{cases}
	1, \; h \to 0 ^{+} \\
	-1, \; h \to 0^{-}
	\end{cases}
	.\]
\end{itemize}
\end{eg}
\begin{observation}
\normalfont Si $\displaystyle x = x_{0} +h $, tenemos que $\displaystyle h = x - x_{0} $, así
\[f'\left(x\right) = \lim_{h \to 0}\frac{f\left(x_{0}+h\right)-f\left(x_{0}\right)}{h} = \lim_{x \to x_{0}}\frac{f\left(x\right)-f\left(x_{0}\right)}{x - x_{0}} .\]
\end{observation}
\begin{fdefinition}[Recta Tangente]
\normalfont Sea $\displaystyle f : A \subset \R \to \R $ derivable en $\displaystyle a \in A $. Se llama \textbf{recta tangente} de la gráfica de $\displaystyle f $ en el punto $\displaystyle \left(a,f\left(a\right)\right) $ a la recta
\[r\left(x\right) = f'\left(a\right)\left(x-a\right) + f\left(a\right) .\]
\end{fdefinition}
\begin{ftheorem}[]
\normalfont Sea $\displaystyle r\left(x\right) = f'\left(a\right)\left(x-a\right) + f\left(a\right) $ la recta tangente a la gráfica de $\displaystyle f $ en el punto $\displaystyle \left(a,f\left(a\right)\right) $. Entonces se cumple que:
\begin{description}
\item[(a)] La recta $\displaystyle r $ pasa por $\displaystyle \left(a, f\left(a\right)\right) $.
\item[(b)] $\displaystyle \lim_{x \to a} \frac{f\left(x\right)-r\left(x\right)}{x - a} = 0 $.
\item[(c)] Si $\displaystyle s\left(x\right) $ es una recta que pasa por $\displaystyle \left(a, f\left(a\right)\right) $ y $\displaystyle \lim_{x \to a}\frac{f\left(x\right)-s\left(x\right)}{x-a} = 0 $, entonces $\displaystyle r\left(x\right) = s\left(x\right) $.
\item[(d)] $\displaystyle r'\left(a\right) = f'\left(a\right) $.
\end{description}
\end{ftheorem}
\begin{proof}
\begin{description}
\item[(a)] Tenemos que $\displaystyle r\left(a\right) = f\left(a\right) $.
\item[(b)] 
	\[ .\]
	\[
	\begin{split}
		\lim_{x \to a}\frac{f\left(x\right)-r\left(a\right)}{x-a} = & \lim_{x \to a}\frac{f\left(x\right)-\left(f'\left(a\right)\left(x-a\right) + f\left(a\right)\right)}{x-a} = \lim_{x \to a}\frac{f\left(x\right)-f\left(a\right)}{x-a}-f'\left(a\right) \\
		= &  f'\left(a\right)-f'\left(a\right) = 0 .
	\end{split}
	\]
\footnote{Esto significa que $\displaystyle f\left(x\right) -r\left(x\right) \to 0 $ mucho más rápido que $\displaystyle x - a \to 0 $, lo que nos asegura que la tangente es una buena aproximación a la curva.} 	
\item[(c)] Sea $\displaystyle s\left(x\right) = d\left(x-a\right) + f\left(a\right) $ y $\displaystyle \lim_{x \to a} \frac{f\left(x\right)-s\left(x\right)}{x-a}= 0 $. Entonces,
	\[ .\]
	\[
	\begin{split}
		0 = \lim_{x \to a}\frac{f\left(x\right)-r\left(x\right)}{x-a} = & \lim_{x \to a}\frac{f\left(x\right)-s\left(x\right)+s\left(x\right)-r\left(x\right)}{x-a} = \lim_{x \to a}\frac{s\left(x\right)-r\left(x\right)}{x-a} \\
		= & \lim_{x \to a}\frac{d\left(x-a\right)+f\left(a\right)-f'\left(a\right)\left(x-a\right)-f\left(a\right)}{x-a} = \lim_{x \to a}\left(d-f'\left(a\right)\right) = d - f'\left(a\right) .
	\end{split}
	\]
	Así, $\displaystyle d = f'\left(a\right) $ y $\displaystyle r\left(x\right) = s\left(x\right) $.
\item[(d)] Tenemos que
	\[
	\begin{split}
	r'\left(a\right) = \lim_{x \to a}\frac{r\left(x\right)-r\left(a\right)}{x-a} = \lim_{x \to a}\frac{f'\left(a\right)\left(x-a\right)+f\left(a\right)-f\left(a\right)}{x-a} = f'\left(a\right) .
	\end{split}
	\]
\end{description}
\end{proof}
\begin{eg}
\normalfont Consideremos la función $\displaystyle f\left(x\right) = x^{3} $, que es continua en $\displaystyle \R^{3} $ y monótona creciente. Sabemos que $\displaystyle f\left(x\right) < 0 $ si $\displaystyle x < 0 $ y $\displaystyle f\left(x\right) > 0 $ si $\displaystyle x > 0 $. Tenemos que $\displaystyle \lim_{x \to -\infty}f\left(x\right) = -\infty $ y $\displaystyle \lim_{x \to \infty}f\left(x\right) = \infty $. Calculamos la derivada en un punto $\displaystyle x \in \R $,
\[ f'\left(x\right) = \lim_{h \to 0}\frac{\left(x+h\right)^{3}-x^{3}}{h} = \lim_{h \to 0}\frac{3x^{2}h+3xh^{2}+h^{3}}{h} = \lim_{h \to 0}\left(3x^{2}+3xh+h^{2}\right) = 3x^{2} .\]
Así, la ecuación de la tangente en un punto $\displaystyle x_{0} \in \R $ será
\[y = 3x_{0}^{2}\left(x-x_{0}\right) + x_{0}^{3} .\]
\end{eg}
\begin{ftheorem}[]
\normalfont Sea $\displaystyle f : I \subset \R \to \R $, donde $\displaystyle I $ es un intervalo o semirrecta, y $\displaystyle a \in I $. Si existe $\displaystyle f'\left(a\right) $, entonces $\displaystyle f $ es continua en $\displaystyle a $.
\end{ftheorem}
\begin{proof}
Tenemos que $\displaystyle \lim_{x \to a}\frac{f\left(x\right)-f\left(a\right)}{x-a} = f'\left(a\right) \in \R $. Tenemos que $\displaystyle f $ es continua en $\displaystyle a $ si $\displaystyle \exists\lim_{x \to a}f\left(x\right) = f\left(a\right) $, es decir, si $\displaystyle \lim_{x \to a}f\left(x\right)-f\left(a\right) = 0 $. Supongamos que $\displaystyle \lim_{x \to a}f\left(x\right)-f\left(a\right) \neq 0 $. 
Entonces, existe $\displaystyle \epsilon >0 $ y $\displaystyle x_{n} \to a $ tal que $\displaystyle \left|f\left(x_{n}\right)-f\left(a\right)\right| \geq \epsilon  $. Así,
\[\lim_{n \to \infty}\frac{f\left(x_{n}\right)-f\left(a\right)}{x_{n}-a} = \pm \infty .\]
Esto es una contradicción, puesto que $\displaystyle f'\left(a\right) \in \R $.
\end{proof}
\begin{eg}
\normalfont El recíproco no es cierto. Recordemos que $\displaystyle f\left(x\right) = \left|x\right| $ es continua en $\displaystyle x = 0 $ pero no es derivable en este punto.
\end{eg}
\begin{ftheorem}[]
\normalfont Sea $\displaystyle f,g : I \subset \R \to\R $ con $\displaystyle I $ invervalo o semirrecta y $\displaystyle a \in I $, tales que existen $\displaystyle f'\left(a\right) $ y $\displaystyle g'\left(a\right) $. Sea $\displaystyle \lambda > 0 $.
\begin{description}
\item[(a)] $\displaystyle \exists \left(f+g\right)'\left(a\right) = f'\left(a\right) +g'\left(a\right) $.
\item[(b)] $\displaystyle \exists \left(\lambda f\right)'\left(a\right) = \lambda f'\left(a\right) $.
\item[(c)] $\displaystyle \exists \left(f \cdot g\right)'\left(a\right) = f'\left(a\right)g\left(a\right) + f\left(a\right)g'\left(a\right)$.
\item[(d)] Si $\displaystyle g\left(a\right) \neq 0 $, $\displaystyle \exists \left(\frac{1}{g}\right)\left(a\right) = - \frac{g'\left(a\right)}{g^{2}\left(a\right)} $.
\item[(e)] Si $\displaystyle g\left(a\right) \neq 0 $, $\displaystyle \exists \left(\frac{f}{g}\right)'\left(a\right) = \frac{f'\left(a\right)g\left(a\right)-f\left(a\right)g'\left(a\right)}{g^{2}\left(a\right)}$.
\end{description}
\end{ftheorem}
\begin{proof}
\begin{description}
\item[(a)] Tenemos que
	\[ .\]
	\[
	\begin{split}
		\left(f+g\right)'\left(a\right) = & \lim_{x \to a}\frac{f\left(x\right)+g\left(x\right)-\left(f\left(a\right)+g\left(a\right)\right)}{x-a} = \lim_{x \to a}\frac{f\left(x\right)-f\left(a\right)}{x-a} + \lim_{x \to a}\frac{g\left(x\right)-g\left(a\right)}{x-a} \\
		= &  f'\left(a\right) + g'\left(a\right) .
	\end{split}
	\]
\item[(b)] Tenemos que 
	\[
	\begin{split}
		\left(\lambda f\right)'\left(a\right) = & \lim_{x \to a}\frac{\lambda f\left(x\right)-\lambda f\left(a\right)}{x-a} = \lambda \left(\lim_{x \to a}\frac{f\left(x\right)-f\left(a\right)}{x-a}\right) = \lambda f'\left(a\right) .
	\end{split}
	\]
	
\item[(c)] Tenemos que 
	\[
	\begin{split}
		\left(f \cdot g\right)'\left(a\right) = & \lim_{x \to a}\frac{f\left(x\right)g\left(x\right) - f\left(a\right)g\left(a\right)}{x - a} = \lim_{x \to a}\frac{f\left(x\right)g\left(x\right)-f\left(a\right)g\left(x\right)+f\left(a\right)g\left(x\right)-f\left(a\right)g\left(a\right)}{x-a} \\
		= & \lim_{x \to a} g\left(x\right) \frac{f\left(x\right)-f\left(a\right)}{x-a} + f\left(a\right) \frac{g\left(x\right)-g\left(a\right)}{x-a} = g\left(a\right)f'\left(a\right) + f\left(a\right)g'\left(a\right).
	\end{split}
	\]
\item[(d)] Tenemos que
	\[
	\begin{split}
		\left(\frac{1}{g}\right)'\left(a\right) = & \lim_{x \to a}\frac{\frac{1}{g}\left(x\right)-\frac{1}{g}\left(a\right)}{x - a} = \lim_{x \to a}\frac{1}{x -a}\frac{g\left(a\right)-g\left(x\right)}{g\left(a\right)g\left(x\right)} \\
		= & \lim_{x \to a}-\frac{g\left(x\right)-g\left(a\right)}{x-a} \cdot \frac{1}{g\left(x\right)g\left(a\right)} = -\frac{g'\left(a\right)}{g^{2}\left(a\right)} .
	\end{split}
	\]
\item[(e)] Podemos aplicar \textbf{(c)} y \textbf{(d)} para deducirla.
	\[
	\begin{split}
		\left(\frac{f}{g}\right)'\left(a\right) = & \left(f \cdot \frac{1}{g}\right)'\left(a\right) = f'\left(a\right) \frac{1}{g\left(a\right)} + f\left(a\right)\left(-\frac{g'\left(a\right)}{g^{2}\left(a\right)}\right) = \frac{f'\left(a\right)g\left(a\right)-f\left(a\right)g'\left(a\right)}{g^{2}\left(a\right)}.
	\end{split}
	\]
	
\end{description}
\end{proof}
\begin{eg}
\normalfont \textbf{Hoja 3 - Problema 7}. Si $\displaystyle g\left(0\right) = g'\left(0\right) = 0 $, se define
\[f\left(x\right) = 
\begin{cases}
g\left(x\right)\sin\left(\frac{1}{x}\right), \; x \neq 0\\
0, \; x = 0
\end{cases}
.\]
Vamos a ver si $\displaystyle f $ es continua. Tenemos que (esto fue demostrado en un ejercicio de las hojas)
\[ \left|f\left(x\right)\right| = \left|g\left(x\right)\sin\left(\frac{1}{x}\right)\right| \leq \left|g\left(x\right)\right| \to 0 .\]
Por tanto, $\displaystyle \left|f\left(x\right)\right| \to 0 $ y es continua en $\displaystyle x = 0 $. Entonces, tenemos que 
\[
\begin{split}
f'\left(0\right) = \lim_{x \to 0}\frac{f\left(x\right)-f\left(0\right)}{x - 0} = \lim_{x \to 0}\frac{g\left(x\right) \sin\left(\frac{1}{x}\right)}{x} = \lim_{x \to 0}\frac{g\left(x\right)}{x}\sin\left(\frac{1}{x}\right) = 0 .
\end{split}
\]
El límite anterior se deduce de que
\[ \left|\frac{g\left(x\right)}{x}\sin\left(\frac{1}{x}\right)\right| \leq \left|\frac{g\left(x\right)}{x}\right| \to 0 .\]
\end{eg}
\begin{ftheorem}[]
\normalfont Sea $\displaystyle f\left(x\right) = x^{n} $ con $\displaystyle n \in \N $. Entonces, existe $\displaystyle f'\left(x\right) = n x^{n-1} $.
\end{ftheorem}
\begin{proof}
Se demuestra por inducción. Si $\displaystyle n = 1 $ es trivial. Suponemos que es cierto para $\displaystyle n $. Entonces, tenemos que si $\displaystyle f\left(x\right) = x^{n+1} = x^{n} \cdot x $, 
\[ f'\left(x\right) = \left(x^{n}\right)' \cdot x + x^{n}\left(x\right)' = nx^{n-1} \cdot x + x^{n} = nx^{n}+x^{n} = \left(n+1\right)x^{n} .\]
\end{proof}
\begin{fcolorary}[]
\normalfont Sea $\displaystyle f\left(x\right) = \frac{1}{x^{n}} = x^{-n} $, con $\displaystyle n \in \N $. Entonces, si $\displaystyle x \neq 0 $, $\displaystyle f'\left(x\right) = -nx^{-n-1} $.
\end{fcolorary}
\begin{proof}
Aplicando el apartado \textbf{(c)} del \textbf{Teorema 5.3},
\[f'\left(x\right) = \frac{-nx^{n-1}}{x^{2n}} = -n x^{\left(n-1\right)-2n} = -nx^{-n-1} .\]
\end{proof}
\begin{eg}
\normalfont Consideremos $\displaystyle f\left(x\right) = \frac{x^{3}+x+1}{x^{2}-1} $. Entonces, tenemos que si $\displaystyle x \neq \pm 1 $,
\[ f'\left(x\right) = \frac{\left(3x^{2}+1\right)\left(x^{2}-1\right)-\left(x^{3}+x+1\right)\left(2x\right)}{\left(x^{2}-1\right)^{2}} .\]
\end{eg}
\begin{ftheorem}[Regla de la cadena]
\normalfont Sea $\displaystyle f,g : \R \to \R $ con $\displaystyle \Imagen\left(f\right) \subset \dom\left(g\right) $ y $\displaystyle a \in \dom\left(f\right) $, tal que existe $\displaystyle f'\left(a\right) $ y existe $\displaystyle g'\left(f\left(a\right)\right) $. Entonces, existe $\displaystyle \left(g \circ f\right)'\left(a\right) = g'\left(f\left(a\right)\right) f'\left(a\right) $.
\end{ftheorem}
\begin{proof}
Tenemos que
\[
\begin{split}
	\left(g\circ f\right)'\left(a\right) =& \lim_{x \to a}\frac{g\circ f\left(x\right) - g\circ f\left(a\right)}{x-a} = \lim_{x \to a}\frac{g\left(f\left(x\right)\right)-g\left(f\left(a\right)\right)}{x-a} \cdot \frac{f\left(x\right)-f\left(a\right)}{f\left(x\right)-f\left(a\right)} \\
	= & \lim_{x \to a} \frac{g\left(f\left(x\right)\right)-g\left(f\left(a\right)\right)}{f\left(x\right)-f\left(a\right)} \cdot \frac{f\left(x\right)-f\left(a\right)}{x - a} .
\end{split}
\]
Sea $\displaystyle h = f\left(x\right)-f\left(a\right) $, entonces $\displaystyle h\to0 $ cuando $\displaystyle x \to a $. Así,
\[\lim_{h \to 0} \frac{g\left(f\left(a\right)+h\right)-g\left(f\left(a\right)\right)}{h} = g'\left(f\left(a\right)\right) .\]
Por otro lado, es trivial que $\displaystyle \lim_{x \to a}\frac{f\left(x\right)-f\left(a\right)}{x-a} = f'\left(a\right) $. Así, hemos obtenido que $\displaystyle \left(g\circ f\right)'\left(a\right) = g'\left(f\left(a\right)\right) f'\left(a\right) $.
\end{proof}
\begin{eg}
\normalfont Consideremos la función $\displaystyle f\left(x\right) = \left(x^{2}-3\right)^{27} $. Aplicando el teorema anterior,
\[f'\left(x\right) = 27\left(x^{2}-3\right)^{26} \cdot 2x = 54x\left(x^{2}-3\right)^{26} .\]
\end{eg}
\begin{ftheorem}[Teorema de la función inversa]
\normalfont Sea $\displaystyle f: \left(\alpha, \beta \right) \to \R $ derivable en $\displaystyle \left(\alpha, \beta \right) $ y $\displaystyle f' $ es continua en $\displaystyle \left(\alpha, \beta \right) $ y para $\displaystyle a \in \left(\alpha, \beta \right) $, $\displaystyle f'\left(a\right) \neq 0 $. Entonces, en un intervalo centrado en $\displaystyle f\left(a\right) $ existe $\displaystyle f^{-1} $, que es derivable y $\displaystyle \left(f^{-1} \right)' \left(f\left(a\right)\right)= \frac{1}{f'\left(a\right)}$.
\end{ftheorem}
\begin{proof}
Si $\displaystyle f'\left(a\right) \neq 0 $ tenemos que $\displaystyle f'\left(a\right) > 0 $ o $\displaystyle f'\left(a\right) < 0 $. Sin pérdida de generalidad, asumamos que $\displaystyle f'\left(a\right) > 0 $ (como $\displaystyle f' $ es continua, en un intervalo $\displaystyle \left(a-\delta, a + \delta \right) $, $\displaystyle f'\left(x\right) >0$), entonces podemos demostrar que $\displaystyle f $ es inyectiva en $\displaystyle \left(a - \delta, a + \delta \right) $ (por el teorema del valor medio). Tenemos que $\displaystyle f $ es continua (por ser derivable) e inyectiva, por lo que $\displaystyle f $ es estrictamente monótona en $\displaystyle \left(a-\delta, a + \delta \right) $ y, por tanto, existe $\displaystyle f^{-1} $. Consideremos la aplicación
\[ f: \left(a - \delta, a + \delta \right) \to \left(f\left(a-\delta \right), f\left(a+\delta \right)\right) .\]
 Tenemos que $\displaystyle f^{-1} $ es continua en $\displaystyle \left(f\left(a-\delta \right), f\left(a+\delta \right)\right) $. Entonces, si $\displaystyle y = f\left(a\right) $, $\displaystyle f^{-1}\left(y\right) = f^{-1}\left(f\left(a\right)\right) = a $. Entonces, tenemos que 
\[\lim_{y \to f\left(a\right)}\frac{f^{-1}\left(y\right)-f^{-1}f\left(a\right)}{y-f\left(a\right)} = \lim_{x \to a}\frac{1}{\frac{f\left(x\right)-f\left(a\right)}{x-a}} = \frac{1}{f'\left(a\right)} .\]
\end{proof}
\begin{observation}
\normalfont Sea $\displaystyle f\left(x\right) = x^{\frac{1}{n}} = \sqrt[n]{x} $ con $\displaystyle n \in \N $. Entonces, $\displaystyle f^{-1}\left(x\right) = x^{n} $. Así, tenemos que
	\[f'\left(x\right) = \frac{1}{n\left(x^{\frac{1}{n}}\right)^{n-1}} = \frac{1}{n}x^{-\frac{n-1}{n}} = \frac{1}{n}x^{\frac{1}{n}-1} .\]
En general, si $\displaystyle n,m \in \N $ y$\displaystyle f\left(x\right) = x^{\frac{n}{m}} = \left(x^{\frac{1}{m}}\right)^{n} $, entonces aplicando la regla de la cadena, 
\[f'\left(x\right) = n\left(x^{\frac{1}{m}}\right)^{n-1} \cdot \frac{1}{m} x^{\frac{1}{m}-1} = \frac{n}{m}x^{\frac{n}{m}-1} .\]
\end{observation}
\begin{eg}
\normalfont Consideremos la función $\displaystyle f\left(x\right) = \sqrt{ \frac{x}{x^{2}+1}} $:
\[f'\left(x\right) = \frac{1}{2} \frac{1}{\sqrt{\frac{x}{x^{2}+1}}} \cdot \frac{\left(x+1\right)^{2}-2x^{2}}{\left(x^{2}+1\right)^{2}} .\]
\end{eg}
Consideremos ahora las funciones $\displaystyle \sin\left(x\right), \cos\left(x\right) $ y $\displaystyle \tan\left(x\right) $. Más adelante se demostrarán la siguiente igualdade:
\[ \left(\sin x\right)' = \cos x .\]
De esta se pueden deducir las otras derivadas. Tenemos que $\displaystyle \cos x= \sqrt{1 - \sin ^{2}x} $, por tanto, derivando a ambos lados:
\[\left(\cos x\right)' = \frac{-2\sin x\cos x}{2\sqrt{1 - \sin ^{2}x}} = - \sin x.\]
Por definición, tenemos que $\displaystyle \tan x= \frac{\sin x}{\cos x} $, por tanto,
\[\left(\tan x\right)' = \left(\frac{\sin x}{\cos x}\right)' = \frac{ \cos^{2} x + \sin ^{2} x }{\cos^{2}x} = \frac{1}{\cos^{2} x} = 1 + \tan ^{2} x\]
Vamos a estudiar ahora las funciones inversas: $\displaystyle \arcsin x, \arccos x, \arctan x $. Si $\displaystyle f\left(x\right) = \arctan x $, tenemos que $\displaystyle f^{-1}\left(x\right) = \tan x $. Aplicando el teorema de la función inversa,
\[f'\left(x\right) = \frac{1}{1 + \tan^{2}\left(\arctan x\right)} = \frac{1}{1 + x^{2}} .\]
Sea $\displaystyle f\left(x\right) = e^{x} $. Más adelante se demostrará que $\displaystyle f'\left(x\right) = e^{x} $. Lo haremos introduciendo la función logarítmica, que es la inversa de la exponencial.
\begin{fdefinition}[Funciones hiperbólicas]
\normalfont Sea llama \textbf{coseno hiperbólico} a la función $\displaystyle \cosh x = \frac{e^{x}+e^{-x}}{2} $. Similarmente, se llama \textbf{seno hiperbólico} a la función $\displaystyle \sinh x = \frac{e^{x}-e^{-x}}{2} $. Finalmente, se define \textbf{tangente hiperbólica} a la función $\displaystyle \tanh x = \frac{\sinh x}{\cosh x} $.
\end{fdefinition}
Calculamos sus derivadas.
\[
\begin{split}
	\left(\cosh x\right)' = & \frac{e^{x}-e^{-x}}{2} = \sinh x \\
	\left(\sinh x\right)' = & \frac{e^{x}+e^{-x}}{2} = \cosh x \\
	\left(\tanh x\right)' = & \frac{\cosh^{2}x - \sinh^{2}x}{\cosh ^{2}x} = 1 - \tanh^{2}x.
\end{split}
\]
\begin{observation}
\normalfont Podemos observar que 
\[ \boxed{1 = \cosh^{2}x - \sinh^{2}x} \]
\end{observation}
Ahora podemos calcular las derivadas de sus inversas aplicando el teorema de la función inversa.
\[
\begin{split}
	\left(\arcsinh x\right)' = & \frac{1}{\cosh\left(\arcsinh x\right)} = \frac{1}{\sqrt{1 + \sinh^{2}\left(\arcsinh x\right)}} = \frac{1}{\sqrt{1+x^{2}}} \\
	\left(\arccosh x\right)' = & \frac{1}{\sinh\left(\arccosh x\right)} = \frac{1}{\sqrt{\cosh^{2}\left(\arccos x\right) - 1}} = \frac{1}{\sqrt{x^{2}-1}}\\
	\left(\arctanh x\right)' = & \frac{1}{1 - \tanh^{2}\left(\arctanh x\right)} = \frac{1}{1 - x^{2}}.
\end{split}
\]
\begin{fdefinition}[Extremos relativos]
\normalfont Sea $\displaystyle f : A \subset \R \to \R $. 
\begin{description}
\item[(a)] Se dice que $\displaystyle x_{0} $ es un \textbf{máximo relativo} si existe $\displaystyle r > 0 $ tal que $\displaystyle \left(x_{0}-r, x_{0}+r\right) \subset A $, y allí $\displaystyle f\left(x_{0}\right) \geq f\left(x\right) $ para $\displaystyle \forall x \in \left(x_{0}-r, x_{0}+r\right) $.
\item[(b)] Se dice que $\displaystyle x_{0} $ es un \textbf{mínimo relativo} si existe $\displaystyle r > 0 $ tal que $\displaystyle \left(x_{0}-r, x_{0}+r\right) \subset A $, y allí $\displaystyle f\left(x_{0}\right) \leq f\left(x\right) $ para $\displaystyle \forall x \in \left(x_{0}-r, x_{0}+r\right) $.
\end{description}
\end{fdefinition}
\begin{observation}
\normalfont Los máximos y mínimos absolutos de una función definida en un intervalo o semirrecta abierta son automáticamente extremos relativos. Sin embargo, el recíproco no tiene por que ser cierto.
\end{observation}

\begin{eg}
	\normalfont Si consideramos la función $\displaystyle f\left(x\right) = \left|x\right| $ definida en $\displaystyle \left[-1,1\right]  $, tenemos que $\displaystyle f $ tiene un mínimo relativo en $\displaystyle x = 0 $, que también es un mínimo absoluto.
\end{eg}
\begin{ftheorem}[]
\normalfont Sea $\displaystyle f : I \subset \R \to \R  $ es un intervalo o semirrecta, con $\displaystyle f $ derivable en $\displaystyle I $. Sea $\displaystyle x_{0} \in I $ un extremo relativo. Entonces, $\displaystyle f'\left(x_{0}\right) = 0 $.
\end{ftheorem}
\begin{proof}
Sin pérdida de generalidad, se $\displaystyle x_{0} $ un máximo local. Entonces existe $\displaystyle r > 0 $ tal que $\displaystyle \left(x_{0}-r, x_{0}+r\right) \subset I $. 
\begin{itemize}
\item Si $\displaystyle x > x_{0} $ tenemos que, dado que $\displaystyle f $ es derivable en $\displaystyle I $,
	\[ \frac{f\left(x\right)-f\left(x_{0}\right)}{x-x_{0}} \leq 0 \Rightarrow \lim_{x \to x_{0}^{+}}\frac{f\left(x\right)-f\left(x_{0}\right)}{x-x_{0}} = f'\left(x_{0}\right) \leq 0 .\]
\item Si $\displaystyle x < x_{0} $, tenemos que
	\[ \frac{f\left(x\right)-f\left(x_{0}\right)}{x-x_{0}} \geq 0 \Rightarrow \lim_{x \to x_{0}^{-}}\frac{f\left(x\right)-f\left(x_{0}\right)}{x-x_{0}} = f'\left(x_{0}\right) \geq 0 .\]
\end{itemize}
Así, como $\displaystyle \exists f'\left(x_{0}\right) $, debe ser que $\displaystyle f'\left(x_{0}\right) = 0 $.
\end{proof}
\begin{eg}
\normalfont 
\begin{description}
\item[(i)] Consideremos la función $\displaystyle f\left(x\right) = \left|x\right| $ definida en $\displaystyle \left(-1, 1\right) $. Tiene un mínimo local en $\displaystyle x= 0 $ pero no es derivable en este punto.
\item[(ii)] El recíproco del teorema anterior no es cierto. En efecto, consideremos la función $\displaystyle f\left(x\right) = x^{3} $ definida en $\displaystyle \left(-1,1\right) $. Tenemos que $\displaystyle f'\left(0\right) = 0 $, pero no tiene un extremo relativo en ese punto.
\end{description}
\end{eg}
\begin{observation}
\normalfont Para buscar máximos y mínimos de una función buscamos en:
\begin{itemize}
\item Los extremos del dominio.
\item Los puntos donde no es continua o derivable.
\item Los puntos en los que se anula la derivada.
\end{itemize}
\end{observation}
\begin{ftheorem}[Teorema de Rolle]
	\normalfont Sea $\displaystyle f:[a,b] \to \R $ continua en $\displaystyle [a,b] $ y derivable en $\displaystyle \left(a,b\right) $. Si $\displaystyle f\left(a\right) = f\left(b\right) $, entonces existe $\displaystyle c \in \left(a,b\right) $ tal que $\displaystyle f'\left(c\right) = 0 $.
\end{ftheorem}
\begin{proof}
	Si $\displaystyle f $ es constante, es trivial puesto que $\displaystyle \forall x \in \left(a,b\right) $ se tiene que $\displaystyle f'\left(x\right) = 0 $. En caso contrario, dado que $\displaystyle f $ es continua en $\displaystyle \left[a,b\right]  $ existe $\displaystyle x_{0} \in \left(a,b\right) $ máximo o mínimo local con $\displaystyle f\left(x_{0}\right) > f\left(a\right) $ o $\displaystyle f\left(x_{0}\right) < f\left(a\right) $. Por el teorema anterior, dado que $\displaystyle x_{0} $ es un extremo local y $\displaystyle f $ es derivable en $\displaystyle \left(a,b\right) $, debe ser que $\displaystyle f'\left(x_{0}\right) = 0 $.
\end{proof}
\begin{ftheorem}[Teorema del valor medio]
	\normalfont Sea $\displaystyle f : \left[a,b\right]  \to \R $ continua en $\displaystyle \left[a,b\right]  $ y derivable en el intervalo $\displaystyle \left(a,b\right) $. Entonces, existe $\displaystyle c \in \left(a,b\right) $ tal que $\displaystyle f'\left(c\right) = \frac{f\left(b\right)-f\left(a\right)}{b-a} $.
\end{ftheorem}
\begin{proof}
	Sea $\displaystyle g\left(x\right) = f\left(x\right) - \frac{f\left(b\right)-f\left(a\right)}{b-a}\left(x-a\right) - f\left(a\right) $. Tenemos que $\displaystyle g $ es continua en $\displaystyle \left[a,b\right]  $ y derivable en $\displaystyle \left(a,b\right) $. Además, dado que $\displaystyle g\left(a\right) = g\left(b\right) = 0 $, por el teorema de Rolle, existe $\displaystyle c \in \left(a,b\right) $ tal que $\displaystyle g'\left(c\right) = 0 $. Es decir,
	\[0 = g'\left(c\right) = f'\left(c\right) - \frac{f\left(b\right)-f\left(a\right)}{b-a} \iff f'\left(c\right) = \frac{f\left(b\right)-f\left(a\right)}{b-a} .\]
\end{proof}
\begin{fcolorary}[]
	\normalfont 
	\begin{description}
	\item[(a)] Sea $\displaystyle f: [a,b] \to \R $ con $\displaystyle f $ continua en $\displaystyle [a,b] $ y derivable en $\displaystyle \left(a,b\right) $. Si $\displaystyle f'\left(x\right) = 0 $, $\displaystyle \forall x \in \left(a,b\right) $, entonces $\displaystyle f $ es constante.
	\item[(b)] Sean $\displaystyle f,g : [a,b] \to \R $, ambas continuas en $\displaystyle [a,b] $ y derivables en $\displaystyle \left(a,b\right) $. Si $\displaystyle f'\left(x\right) = g'\left(x\right) $, $\displaystyle \forall x \in \left(a,b\right) $, entonces existe $\displaystyle k \in \R $ tal que $\displaystyle f\left(x\right) = g\left(x\right) + k $, $\displaystyle \forall x \in [a,b] $. 
	\item[(c)] Sea $\displaystyle f : [a,b] \to \R$ es continua en $\displaystyle \left[a,b\right]  $ y derivable en $\displaystyle \left(a,b\right) $ tal que $\displaystyle f'\left(x\right) \neq 0 $, $\displaystyle \forall x \in \left(a,b\right) $, entonces $\displaystyle f $ es inyectiva \footnote{El recíproco también es cierto.} .
	\end{description}
\end{fcolorary}
\begin{proof}
\begin{description}
	\item[(a)] Sean $\displaystyle x,y \in [a,b]$, entonces $\displaystyle f $ es continua en $\displaystyle [x,y] $  y derivable en $\displaystyle \left(x,y\right) $. Por el teorema del valor medio, existe $\displaystyle c \in \left(x,y\right) $ con $\displaystyle f\left(x\right)-f\left(y\right) = f'\left(c\right)\left(x-y\right) = 0 $, así, $\displaystyle f\left(x\right) = f\left(y\right) $, $\displaystyle \forall x,y \in \left[a,b\right]  $.
	\item[(b)] Consideremos la función $\displaystyle h\left(x\right) = f\left(x\right)-g\left(x\right) $. Entonces, $\displaystyle h $ es continua en $\displaystyle [a,b] $ y derivable en $\displaystyle \left(a,b\right) $. Además, como $\displaystyle h'\left(x\right) = f'\left(x\right)-g'\left(x\right) = 0 $, tenemos, por \textbf{(a)}, que $\displaystyle h\left(x\right) = k \in \R $, por lo que $\displaystyle f\left(x\right) = g\left(x\right) + k $.
	\item[(c)] Si $\displaystyle x,y \in [a,b] $ (con $\displaystyle x \neq y $), entonces $\displaystyle f : [x,y] \to \R $ es continua en $\displaystyle [x,y] $ y derivable en $\displaystyle \left(x,y\right) $. Por el teorema del valor medio, existe $\displaystyle c \in \left(x,y\right) $ con $\displaystyle f\left(x\right)-f\left(y\right) = f'\left(c\right)\left(x-y\right) \neq 0 $, por lo que debe ser que $\displaystyle f\left(x\right) \neq f\left(y\right) $.
\end{description}
\end{proof}
\begin{eg}
\normalfont El teorema del valor medio sirve para estimar errores. Consideremos $\displaystyle f\left(x\right) = \sqrt{x} $ y que conocemos $\displaystyle \sqrt{x} $ pero no conocemos $\displaystyle \sqrt{y} $. Podemos entonces, estimar el valor de $\displaystyle \sqrt{y} $ con la distancia $\displaystyle \left|\sqrt{x}-\sqrt{y}\right| $. Sabemos que existe $\displaystyle c \in \left(x,y\right) $ tal que 
\[ f'\left(c\right)\left(x-y\right) = f\left(x\right)-f\left(y\right) .\]
Si $\displaystyle x,y > 1 $, entonces tenemos que 
\[ \left|\sqrt{x}-\sqrt{y}\right| = f'\left(c\right) = \frac{1}{2\sqrt{c}} \left|x-y\right| \leq \frac{1}{2} \left|x-y\right| .\]
Este procedimiento también sirve para ver que esta función es uniformemente continua en $\displaystyle (1,\infty) $, puesto que acabamos de ver que es de Lipschitz.
\end{eg}
\section{Regla de L'Hôpital}
\begin{fdefinition}[Curvas paramétricas]
	\normalfont Sean $\displaystyle f,g : \left[a,b\right]  \to \R $ continuas en $\displaystyle \left[a,b\right]  $. Se llama \textbf{curva paramétrica} a los puntos $\displaystyle C = \left\{ \left(f\left(t\right), g\left(t\right)\right) \; : \; t \in \left[a,b\right] \right\}  $.
\end{fdefinition}
\begin{observation}
\normalfont Si $\displaystyle f\left(t\right) = t $, tenemos una gráfica (podemos definir una función, puesto que a cada valor de $\displaystyle x $ se le asigna sólo uno de $\displaystyle y $).
\end{observation}
\begin{observation}
\normalfont Con tres funciones $\displaystyle f\left(t\right),g\left(t\right),h\left(t\right) $, podemos definir curvas en el espacio.
\end{observation}
Vamos a estudiar la tangente de una curva paramétrica. Si cogemos dos puntos, $\displaystyle t_{0} $ y $\displaystyle t $, podemos calcular la ecuación de la recta que pasa por esos puntos:
\[y\left(t\right) =  \frac{g\left(t\right)-g\left(t_{0}\right)}{f\left(t\right)-f\left(t_{0}\right)}\left(x-f(t_{0})\right) + g\left(t_{0}\right)   .\]
Tenemos que 
\[\lim_{t \to t_{0}}\frac{g\left(t\right)-g\left(t_{0}\right)}{f\left(t\right)-f\left(t_{0}\right)}= \lim_{t \to t_{0}} \left(\frac{g\left(t\right)-g\left(t_{0}\right)}{t-t_{0}} \cdot \frac{t - t_{0}}{f\left(t\right)-f\left(t_{0}\right)}\right) = \frac{g'\left(t_{0}\right)}{f'\left(t_{0}\right)} .\]
Entonces, concluimos que si $\displaystyle C $ es una curva del plano y $\displaystyle \exists f' $ y $\displaystyle g' $ entonces la pendiente de la recta tangente por el punto $\displaystyle \left(f\left(t_{0}\right), g\left(t_{0}\right)\right) $ será $\displaystyle \frac{g'\left(t_{0}\right)}{f'\left(t_{0}\right)} $. Es decir, esta recta tiene como vector director $\displaystyle \left(f'\left(t_{0}\right), g'\left(t_{0}\right)\right) $. \\ \\ 
Si tenemos una curva $\displaystyle C $ que empieza en $\displaystyle \left(f\left(a\right), g\left(a\right)\right) $ y termina en $\displaystyle \left(f\left(b\right), g\left(b\right)\right) $, nos gustaría decir que si $\displaystyle c \in \left(a,b\right) $, entonces
\[ \frac{g'\left(c\right)}{f'\left(c\right)} = \frac{g\left(b\right)-g\left(a\right)}{f\left(b\right)-f\left(a\right)} .\]
Esto se enuncia en el siguiente teorema.
\begin{ftheorem}[Teorema del valor medio de Cauchy]
	\normalfont Sean $\displaystyle f,g : \left[a,b\right] \to \R $ continuas en $\displaystyle \left[a,b\right]  $ y derivables en $\displaystyle \left(a,b\right) $. Entonces, existe $\displaystyle c \in \left(a,b\right) $ tal que 
	\[\frac{g'\left(c\right)}{f'\left(c\right)} = \frac{g\left(b\right)-g\left(a\right)}{f\left(b\right)-f\left(a\right)} .\]
\end{ftheorem}
\begin{proof}
	Sea $\displaystyle h\left(x\right) = f\left(x\right)\left(g\left(b\right)-g\left(a\right)\right) - g\left(x\right)\left(f\left(b\right)-f\left(a\right)\right) $. Dado que $\displaystyle f $ y $\displaystyle g $ son continuas y derivables, tenemos que $\displaystyle h: [a,b] \to \R $ es continua en $\displaystyle [a,b] $ y derivable en $\displaystyle \left(a,b\right) $.
Tenemos que $\displaystyle h\left(a\right)=h\left(b\right) = f\left(a\right)g\left(b\right)-g\left(a\right)f\left(b\right) $. Aplicando el teorema de Rolle, tenemos que existe $\displaystyle c \in \left(a,b\right) $ tal que $\displaystyle h'\left(c\right) = 0 $. Es decir, 
\[ h'\left(c\right) = f'\left(c\right)\left(g\left(b\right)-g\left(a\right)\right) - g'\left(c\right)\left(f\left(b\right)-f\left(a\right)\right) \iff f'\left(c\right)\left(g\left(b\right)-g\left(a\right)\right) = g'\left(c\right)\left(f\left(b\right)-f\left(a\right)\right) .\]
\end{proof}

\begin{observation}
\normalfont Si $\displaystyle f\left(b\right) \neq f\left(a\right) $ y $\displaystyle f'\left(c\right) \neq 0 $, el teorema anterior es equivalente a decir que
\[ f'\left(c\right)\left(g\left(b\right)-g\left(a\right)\right) = g'\left(c\right)\left(f\left(b\right)-f\left(a\right)\right) .\]
\end{observation}
\begin{observation}
\normalfont Podemos ver que el teorema del valor medio es un caso particular de este teorema. En efecto, si tenemos que $\displaystyle f\left(t\right) = t $, entonces obtenemos el teorema del valor medio.
\end{observation}
\begin{ftheorem}[Regla de L'Hôpital]
	\normalfont Sean $\displaystyle f,g : \left[a,b\right] \to \R $, con $\displaystyle f'\left(x\right) \neq 0 $, continuas en $\displaystyle \left[a,b\right] / \left\{ x_{0}\right\}  $ y derivables en $\displaystyle \left(a,b\right) / \left\{ x_{0}\right\}  $, tales que $\displaystyle \lim_{x \to x_{0}}f\left(x\right) = \lim_{x \to x_{0}}g\left(x\right) = 0 $. Si existe $\displaystyle \lim_{x \to x_{0}} \frac{g'\left(x\right)}{f'\left(x\right)} $, entonces existe $\displaystyle \lim_{x \to x_{0}}\frac{g\left(x\right)}{f\left(x\right)} = \lim_{x \to x_{0}}\frac{g'\left(x\right)}{f'\left(x\right)} $. 
	\footnote{Cuando decimos $\displaystyle \lim_{x \to x_{0}} $, realmente vale también para $\displaystyle x_{0}^{-}, x_{0}^{+} $ y $\displaystyle x_{0} $.} 
\end{ftheorem}
\begin{proof}
Vamos a estudiar los límites laterales (para demostrar el límite, basta con comprobar que los límites laterales coinciden). Sin pérdida de generalidad, estudiamos $\displaystyle \lim_{x \to x_{0}^{+}}\frac{g\left(x\right)}{f\left(x\right)} $. Definimos $\displaystyle g\left(x_{0}\right) = \lim_{x \to x_{0}^{+}}g\left(x\right) $ y $\displaystyle f\left(x_{0}\right) = \lim_{x \to x_{0}^{+}}f\left(x\right) $. Entonces, tenemos que 
\[ \lim_{x \to x_{0}^{+}}\frac{g\left(x\right)}{f\left(x\right)} = \lim_{x \to x_{0}^{+}}\frac{g\left(x\right)-g\left(x_{0}\right)}{f\left(x\right)-f\left(x_{0}\right)} .\]
Tenemos que $\displaystyle f,g : \left[x_{0}, x\right] \to \R $ son continuas en $\displaystyle \left[x_{0}, x\right]  $ y derivables en $\displaystyle \left(x_{0}, x\right) $. Aplicando el teorema del valor medio de Cauchy, tenemos que existe $\displaystyle c_{x} \in \left(x_{0}, x\right) $ tal que
\[ \lim_{x \to x_{0}^{+}}\frac{g\left(x\right)-g\left(x_{0}\right)}{f\left(x\right)-f\left(x_{0}\right)} = \lim_{x \to x_{0}^{+}}\frac{g'\left(c_{x}\right)}{f'\left(c_{x}\right)} = \lim_{x \to x_{0}^{+}}\frac{g'\left(x\right)}{f'\left(x\right)} .\]
En efecto, si $\displaystyle x \to x_{0}^{+} $ tenemos que $\displaystyle c_{x} \to x_{0}^{+} $, por lo que, dado que existe $\displaystyle \lim_{x \to x_{0}^{+}}\frac{g'\left(x\right)}{f'\left(x\right)} $ tenemos que
\[\lim_{x \to x_{0}^{+}}\frac{g'\left(c_{x}\right)}{f'\left(c_{x}\right)} = \lim_{y \to x_{0}^{+}}\frac{g'\left(y\right)}{f'\left(y\right)} .\]
\end{proof}
\begin{proof}
	Esta es una demostración alternativa. Al igual que antes, definimos $\displaystyle f\left(x_{0}\right) = \lim_{x \to x_{0}^{+}}f\left(x\right) $ y $\displaystyle g\left(x_{0}\right) = \lim_{x \to x_{0}^{+}}g\left(x\right) $. Dado que exsiste $\displaystyle \lim_{x \to x_{0}^{+}} \frac{g'\left(x\right)}{f'\left(x\right)} = l \in \R$, si $\displaystyle \epsilon > 0 $ tenemos que existe un $\displaystyle \delta > 0 $ tal que si $\displaystyle 0 < x - x_{0} < \delta  $ entonces $\displaystyle \left|\frac{g'\left(x\right)}{f'\left(x\right)} - l\right| < \epsilon  $. Tenemos que, dado que $\displaystyle f $ y $\displaystyle g $ son continuas en $\displaystyle \left[x_{0}, x_{0}+\delta \right]  $ y derivables en $\displaystyle \left(x_{0}, x_{0}+\delta \right) $, entonces para cada $\displaystyle x \in (x_{0}, x_{0}+\delta) $  existe $\displaystyle c_{x} \in \left(x_{0}, x \right) $ tal que
	\[ \frac{g\left(x\right)-g\left(x_{0}\right)}{f\left(x\right)-f\left(x_{0}\right)} = \frac{g\left(x\right)}{f\left(x\right)} = \frac{g'\left(c_{x}\right)}{f'\left(c_{x}\right)} .\]
	Dado que $\displaystyle c_{x} \in \left(x_{0}, x_{0}+ \delta \right) $, tenemos que
	\[ \left|\frac{g\left(x\right)}{f\left(x\right)}-l\right| = \left|\frac{g'\left(c_{x}\right)}{f'\left(c_{x}\right)}-l\right|<\epsilon .\]
\end{proof}
\begin{observation}
\normalfont Si existieran $\displaystyle f'\left(x_{0}\right) \neq 0 $ y $\displaystyle g'\left(x_{0}\right) $, la prueba sería muy sencilla. En efecto, tendríamos que serían continuas en $\displaystyle x_{0} $ y por tanto
\[ \lim_{x \to x_{0}}\frac{g\left(x\right)}{f\left(x\right)} = \lim_{x \to x_{0}}\frac{g\left(x\right)-g\left(x_{0}\right)}{f\left(x\right)-f\left(x_{0}\right)} = \lim_{x \to x_{0}} \frac{g\left(x\right)-g\left(x_{0}\right)}{x - x_{0}} \cdot \frac{x - x_{0}}{f\left(x\right)-f\left(x_{0}\right)} = \frac{g'\left(x_{0}\right)}{f'\left(x_{0}\right)} .\]
\end{observation}
\begin{eg}
\normalfont 
\[\lim_{x \to 0}\frac{\sin x}{x} = \lim_{x \to 0} \frac{\cos x}{1} = 1 .\]
\[ \lim_{x \to 0}\frac{x^{2}\sin\left(\frac{1}{x}\right)}{\sin x} = \lim_{x \to 0}\frac{x \sin\left(\frac{1}{x}\right)}{\frac{\sin x}{x}} =  0 .\]
En este segundo caso no podemos aplicar L'Hôpital, puesto que $\displaystyle \lim_{x \to 0}\frac{g'\left(x\right)}{f'\left(x\right)} $ no existe.
\end{eg}
\begin{ftheorem}[Relga de L'Hôpital general]
\normalfont Sean $\displaystyle f,g : \left(a,b\right) \to \R $ \footnote{El intervalo $\displaystyle \left(a,b\right) $ puede ser una semirrecta}  dos funciones derivables en $\displaystyle \left(a,b\right) $ salvo quizás en $\displaystyle x_{0} $, donde $\displaystyle x_{0} \in \left(a,b\right) $. Supongamos que $\displaystyle f'\left(x\right) \neq 0 $ si $\displaystyle x \neq x_{0} $ y que existen
\[\lim_{x \to x_{0}}f\left(x\right) = \lim_{x \to x_{0}}g\left(x\right) = 0 ,\]
o bien \footnote{Cuando escribimos $\displaystyle x_{0} $ en el límite, también podemos escribir $\displaystyle x_{0}^{+}, x_{0}^{-}, \infty $ o $\displaystyle -\infty $.} 
\[ \lim_{x \to x_{0}}f\left(x\right) = \pm \infty \quad \text{y} \quad \lim_{x \to x_{0}}g\left(x\right) = \pm \infty .\]
Si, existe $\displaystyle \lim_{x \to x_{0}}\frac{g'\left(x\right)}{f'\left(x\right)} $, entonces 
\[ \lim_{x \to x_{0}}\frac{g\left(x\right)}{f\left(x\right)} = \lim_{x \to x_{0}}\frac{g'\left(x\right)}{f'\left(x\right)} .\]
\end{ftheorem}
\begin{proof}
\begin{description}
\item[(i)] Supongamos que $\displaystyle \lim_{x \to \pm \infty}f\left(x\right) = \lim_{x \to \pm \infty}g\left(x\right) = 0 $. Entonces, tenemos que,
	\[ \lim_{x \to 0^{+}}f\left(\frac{1}{x}\right) = \lim_{x \to 0^{+}}g\left(\frac{1}{x}\right) = 0 .\]
Así, podemos aplicar la regla de L'Hôpital a estas funciones, de forma que
\[ \lim_{x \to \infty}\frac{g\left(x\right)}{f\left(x\right)} = \lim_{x \to 0^{+}} \frac{g\left(\frac{1}{x}\right)}{f\left(\frac{1}{x}\right)} = \lim_{x \to 0^{+}}\frac{g'\left(\frac{1}{x}\right)\left(-\frac{1}{x^{2}}\right)}{f'\left(\frac{1}{x}\right)\left(-\frac{1}{x^{2}}\right)} = \lim_{x \to 0^{+}} \frac{g'\left(\frac{1}{x}\right)}{f'\left(\frac{1}{x}\right)} = \lim_{x \to \infty} \frac{g'\left(x\right)}{f'\left(x\right)}.\]
\item[(ii)] Supongamos que $\displaystyle \lim_{x \to x_{0}}f\left(x\right) = \lim_{x \to x_{0}}g\left(x\right) = \pm \infty $ y existe $\displaystyle \lim_{x \to x_{0}^{+}}\frac{g'\left(x\right)}{f'\left(x\right)} = l \in \R $. Sea $\displaystyle \epsilon > 0 $. Por hipótesis, tenemos que existe $\displaystyle \delta > 0 $ tal que si $\displaystyle 0 < x - x_{0} < \delta  $, entonces 
	\[ \left|\frac{g'\left(x\right)}{f'\left(x\right)} - l\right| < \epsilon .\]
Cogemos $\displaystyle x_{1} \in \left(x_{0}, x_{0} + \delta \right) $ y, dado que $\displaystyle f $ tiene límite por la derecha infinito en $\displaystyle x_{0} $, podemos coger $\displaystyle x_{2} \in \left(x_{0}, x_{1}\right) $ tal que  $\displaystyle f\left(x\right) \neq f\left(x_{1}\right) $ si $\displaystyle x \in \left(x_{0}, x_{2}\right) $. Así, definimos la función
\[ F\left(x\right) = \frac{1 - \frac{f\left(x_{1}\right)}{f\left(x\right)}}{1 - \frac{g\left(x_{1}\right)}{g\left(x\right)}}, \; x \in \left(x_{0}, x_{2}\right) .\]
Tenemos que $\displaystyle \lim_{x \to x_{0}^{+}}F\left(x\right) = 1 $. Por tanto, existe $\displaystyle x_{3} \in \left(x_{0}, x_{2}\right) $ tal que si $\displaystyle 0 < x - x_{0} < x - x_{3} $, $\displaystyle \left|F\left(x\right)-1\right|<\epsilon $. Así, tenemos que si $\displaystyle \epsilon < \frac{1}{2} $ 
\[ \frac{1}{ \left|F\left(x\right)\right|} < \frac{1}{1-\epsilon } < 2.\]
Entonces, tenemos que 
\[ \frac{g\left(x\right)}{f\left(x\right)} = \frac{g\left(x\right)}{f\left(x\right)} \cdot \frac{F\left(x\right)}{F\left(x\right)} = \frac{g\left(x\right)-g\left(x_{1}\right)}{f\left(x\right)-f\left(x_{1}\right)} \cdot \frac{1}{ F\left(x\right)} .\]
Por el teorema del valor medio de Cauchy existe $\displaystyle \xi \in \left(x_{0}, x_{1}\right) $ tal que
\[ \frac{g\left(x\right)}{f\left(x\right)} = \frac{g'\left(\xi\right)}{f'\left(\xi\right)} \cdot \frac{1}{F\left(x\right)} .\]
Así, si $\displaystyle x_{0} < x < x_{3} < x_{2} < x_{1}< x_{0} + \delta $,
\[
\begin{split}
	\left|\frac{g\left(x\right)}{f\left(x\right)}-l\right| = & \left|\frac{g'\left(\xi\right)}{f'\left(\xi\right)} \cdot \frac{1}{F\left(x\right)} - l\right| 
	=  \left|\frac{g'\left(\xi\right)}{f'\left(\xi\right)}-lF\left(x\right)\right| \cdot \frac{1}{ \left|F\left(x\right)\right|} \\
	\leq & \left\{ \left|\frac{g'\left(\xi\right)}{f'\left(\xi\right)} -l\right| + \left|l - lF\left(x\right)\right|\right\} \cdot \frac{1}{ \left|F\left(x\right)\right|} \\
	\leq & \left(\epsilon + \left|l\right|\epsilon \right)2 = \left(2  \left(1 + \left|l\right|\right)\right) \epsilon .
\end{split}
\]
\end{description}
\end{proof}
\begin{eg}
\normalfont Consideremos el límite $\displaystyle \lim_{x \to 0^{+}}x\ln x $. Aplicando la regla de L'Hôpital:
\[ \lim_{x \to 0^{+}} x\ln x= \lim_{x \to 0^{+}}\frac{\ln x}{\frac{1}{x}} = \lim_{x \to 0^{+}} \frac{\frac{1}{x}}{-\frac{1}{x^{2}}} = \lim_{x \to 0^{+}} \left(- x\right) = 0 .\]
\end{eg}
\section{Crecimiento y decrecimiento}
\begin{notation}
\normalfont 
\begin{itemize}
\item Se denomina $\displaystyle f'\left(x^{+}\right) = \lim_{x \to x_{0}^{+}}\frac{f\left(x\right)-f\left(x_{0}\right)}{x - x_{0}} $, a la derivada por la derecha.
\item Se denomina $\displaystyle f'\left(x^{-}\right) = \lim_{x \to x_{0}^{-}}\frac{f\left(x\right)-f\left(x_{0}\right)}{x-x_{0}} $, a la derivada por la izquierda.
\end{itemize}
\end{notation}
\begin{observation}
\normalfont Tenemos que existe $\displaystyle f'\left(x_{0}\right) $ si $\displaystyle f'\left(x_{0}^{+}\right) = f'\left(x_{0}^{-}\right) $.
\end{observation}
\begin{fcolorary}[]
	\normalfont Sea $\displaystyle h: \left[a,b\right] \to \R $ continua en $\displaystyle \left[a,b\right] / \left\{ s\right\}  $ y derivable en $\displaystyle \left(a,b\right) / \left\{ s\right\}  $. Si existe $\displaystyle \lim_{x \to s^{*}}h\left(x\right) = r $ y existe $\displaystyle \lim_{x \to s^{*}}h'\left(x\right) $, entonces existe $\displaystyle h'\left(s^{*}\right) = \lim_{x \to s^{*}}h'\left(x\right) $. \footnote{Aquí, $\displaystyle s^{*} $ significa $\displaystyle s, s^{+} $ o $\displaystyle s^{-} $.} 
\end{fcolorary}
\begin{proof}
Sea $\displaystyle g\left(x\right) = h\left(x\right)-r $ y $\displaystyle f\left(x\right) = x - s $. Entonces, tenemos que $\displaystyle \lim_{x \to s}g\left(x\right) = \lim_{x \to s}\left(h\left(x\right)-r\right) = 0 $ y $\displaystyle \lim_{x \to s}f\left(x\right) = \lim_{x \to s}\left(f\left(x\right)-s\right) = 0 $. Entonces tenemos que
\[\lim_{x \to s}\frac{h\left(x\right)-r}{x-s}=\lim_{x \to s}\frac{g\left(x\right)}{f\left(x\right)} = \lim_{x \to s}\frac{g'\left(x\right)}{f'\left(x\right)} = \lim_{x \to s}\frac{h'\left(x\right)}{1} .\]
Este último término existe por hipótesis.
\end{proof}
\begin{observation}
\normalfont Esto nos dice que la derivada de una función continua no puede tener discontinuidad de salto.
\end{observation}
Dada $\displaystyle f : A \subset \R \to \R $ puede existir $\displaystyle f' $ y puede existir $\displaystyle \left(f'\right)' = f'' $. 
\begin{observation}
\normalfont En física, $\displaystyle f $ es un proceso, $\displaystyle f' $ es la velocidad y $\displaystyle f'' $ es la aceleración. Por ejemplo, la segunda ley de Newton se puede expresar de la forma 
\[ F = m x'' .\]
\end{observation}
A través de $\displaystyle f' $ y $\displaystyle f'' $ se pueden conocer propiedades de $\displaystyle f $. 
\begin{ftheorem}[]
	\normalfont Sea $\displaystyle f : \left(a,b\right) \to \R $ continua en $\displaystyle [a,b] $ y derivable en $\displaystyle \left(a,b\right) $. 
\begin{description}
	\item[(a)] 
		\begin{itemize}
			\item Si $\displaystyle f'\left(c\right) > 0 $, $\displaystyle \forall c \in \left(a,b\right) $, entonces $\displaystyle f $ es creciente en $\displaystyle [a,b] $.
			\item Si $\displaystyle f $ es creciente en $\displaystyle [a,b] $, entonces $\displaystyle f'\left(c\right) \geq 0 $, $\displaystyle \forall c \in \left(a,b\right) $.
		\end{itemize}
		
	\item[(b)] 
		\begin{itemize}
			\item Si $\displaystyle f'\left(c\right) < 0 $, $\displaystyle \forall c \in \left(a,b\right) $, entonces $\displaystyle f $ es decreciente en $\displaystyle [a,b] $.
			\item Si $\displaystyle f $ es decreciente en $\displaystyle [a,b] $, entonces $\displaystyle f'\left(c\right) \leq 0 $, $\displaystyle \forall c \in \left(a,b\right) $.
		\end{itemize}
		
\end{description}
\end{ftheorem}
\begin{proof} Sin pérdida de generalidad, demostramos solo \textbf{(a)}, pues la demostración de \textbf{(b)} es análoga.
\begin{description}
	\item[(i)] Sean $\displaystyle x,y \in [a,b] $ con $\displaystyle x < y $. Por el teorema del valor medio, existe $\displaystyle c \in \left(x,y\right) $ tal que 
		\[ f\left(y\right)-f\left(x\right) = f'\left(c\right)\left(y-x\right) .\]
		Tenemos que $\displaystyle f'\left(c\right) \geq0 $, por lo que también tenemos que $\displaystyle f\left(y\right)-f\left(x\right) \geq 0  $, es decir, $\displaystyle f\left(y\right) \geq f\left(x\right) $.
	\item[(ii)] Sea $\displaystyle c \in \left(a,b\right) $. Si $\displaystyle f $ es creciente, tenemos que si $\displaystyle x > c $, 
		\[ \frac{f\left(x\right)-f\left(c\right)}{x-c} \geq 0 .\]
		Similarmente, si $\displaystyle x < c $ tenemos que 
		\[ \frac{f\left(x\right)-f\left(c\right)}{x-c} \geq 0 .\]
	Por tanto, tenemos que $\displaystyle f'\left(c\right) = \lim_{x \to c}\frac{f\left(x\right)-f\left(c\right)}{x-c} \geq 0 $.
\end{description}
\end{proof}
\begin{ftheorem}[]
\normalfont Sea $\displaystyle f : \left(a,b\right) \to \R $ derivable en $\displaystyle \left(a,b\right) $. Si existe $\displaystyle x_{1}, x_{2} \in \left(a,b\right) $ y $\displaystyle \lambda \in \R $ tal que $\displaystyle f'\left(x_{1}\right) < \lambda < f'\left(x_{2}\right) $, entonces existe $\displaystyle x_{0} \in \left(a,b\right) $ tal que $\displaystyle f'\left(x_{0}\right) = \lambda  $.
\end{ftheorem}
\begin{proof}
	Sea $\displaystyle g\left(x\right) = f\left(x\right)-\lambda x $. Tenemos que $\displaystyle g $ es continua en $\displaystyle [a,b] $ y derivable en $\displaystyle \left(a,b\right) $. Además, tenemos que $\displaystyle g'\left(x\right) = f'\left(x\right) - \lambda  $. Si $\displaystyle g'\left(x\right) \neq 0  $, $\displaystyle \forall x \in \left(a,b\right) $, es decir, $\displaystyle f'\left(x\right) \neq \lambda  $, entonces $\displaystyle g $ es inyectiva, por lo que debe ser monótona.
\begin{itemize}
\item Si $\displaystyle g $ es creciente y $\displaystyle g'\left(x\right) \neq 0 $, tenemos que $\displaystyle g'\left(x\right) > 0 $, $\displaystyle \forall x \in \R $. Entonces tenemos que 
	\[ f'\left(x_{1}\right)-\lambda = g'\left(x_{1}\right) < 0 .\]
\item Si $\displaystyle g $ es decreciente y $\displaystyle g'\left(x\right) \neq 0 $, tenemos que $\displaystyle g'\left(x\right) < 0 $, $\displaystyle \forall x \in \R $. Entonces tenemos que
	\[ f'\left(x_{2}\right) - \lambda = g'\left(x_{2}\right) > 0 .\] 
\end{itemize}
\end{proof}
\begin{observation}
\normalfont Recordamos que una función derivada no puede dar saltos.
\end{observation}
\begin{eg}
\normalfont No todas las funciones derivadas son continuas. Consideremos la función 
\[ f\left(x\right) = 
\begin{cases}
x^{2}\sin\left(\frac{1}{x}\right), \; x \neq 0 \\
0, \; x = 0
\end{cases}
.\]
Teienmos que $\displaystyle f $ es continua en $\displaystyle \R $. En efecto, 
\[ \left|x^{2}\sin\left(\frac{1}{x}\right)\right| \leq x^{2} \to 0, \; x \to 0 .\]
Sin embargo, no es derivable en $\displaystyle x = 0 $:
\[ f'\left(x\right) = 
\begin{cases}
	2\sin \frac{1}{x} - \cos \frac{1}{x}, \; x \neq 0 \\
	0, \; x = 0
\end{cases}
.\]
Dado que $\displaystyle \exists\lim_{x \to 0}f'\left(x\right) $, $\displaystyle f' $ no es continua en $\displaystyle x = 0 $.
\end{eg}
\section{Concavidad y convexidad}
Procedemos a estudiar las derivadas segundas, con el objetivo de estudiar la convexidad y la concavidad.
\begin{flema}[]
	\normalfont
	\[ \left[a,b\right] = \left\{ x \in \R \; : \; x = \alpha a + \left(1- \alpha \right)b, \; \alpha \in \left[0,1\right] \right\}  .\]
\end{flema}
\begin{proof}
\begin{description}
	\item[(i)] Si $\displaystyle x = \alpha a + \left(1- \alpha \right)b = a + \left(1 -\alpha \right)\left(b - a\right) \geq a $. Además, $\displaystyle x = \alpha a + \left(1- \alpha \right)b = b + \alpha \left(a- b\right) \leq b $, por lo que $\displaystyle x \in [a,b] $.
	\item[(ii)] Si $\displaystyle c \in \left[a,b\right]  $, tenemos que 
		\[ \frac{c-a}{b - a} \leq 1, \quad \frac{b-c}{b-a} \leq 1 .\]
		Tomamos $\displaystyle \alpha = \frac{b-c}{b-a} $. Entonces, tenemos que 
		\[c = \frac{b-c}{b-a}a + \frac{c-a}{b-a}b = \alpha a + \left(1-\alpha \right)b .\]		
\end{description}
\end{proof}
\begin{observation}
\normalfont A esta expresión se la llama \textbf{combinación convexa}  de $\displaystyle a $ y $\displaystyle b $.
\end{observation}
\begin{fdefinition}[]
	\normalfont Sea $\displaystyle f : [a,b] \to \R $, se dice que $\displaystyle f $ es \textbf{convexa} en $\displaystyle [a,b]  $ si para $\displaystyle \forall x,y \in [a,b] $, $\displaystyle x < y $ se verifica que 
	\[f\left(\alpha x + \left(1-\alpha \right)y \right) \leq \alpha f\left(x\right) + \left(1-\alpha \right)f\left(y\right), \; \forall \alpha \in \left[0,1\right]  .\]
\end{fdefinition}
\begin{observation}
\normalfont Para entender la definición de convexidad
\[
\begin{split}
	f\left(\alpha x + \left(1-\alpha \right)y\right) \leq & \alpha f\left(x\right) + \left(1-\alpha \right)f\left(y\right) =  \alpha \left(f\left(x\right)-f\left(y\right)\right) + f\left(y\right) \\
	= & \frac{f\left(x\right)-f\left(y\right)}{x-y}\alpha\left(x-y\right) + f\left(y\right) \\
	= & \frac{f\left(x\right)-f\left(y\right)}{x-y}\left(\alpha\left(x-y\right)+y-y\right) + f\left(y\right) \\
	= & \frac{f\left(x\right)-f\left(y\right)}{x-y}\left( \left[\alpha x + \left(1-\alpha \right)y\right] -y\right) + f\left(y\right) = r\left(\alpha x + \left(1-\alpha\right)y\right).
\end{split}
\]
Donde $\displaystyle r $ es la recta que une los puntos $\displaystyle \left(x,f\left(x\right)\right) $ e $\displaystyle \left(y,f\left(y\right)\right) $.
Es decir, la gráfica de $\displaystyle f|_{\left[x,y\right] } $ queda por debajo de la gráfica de la recta que pasa por $\displaystyle \left(x, f\left(x\right)\right) $ e $\displaystyle \left(y, f\left(y\right)\right) $.
\end{observation}
\begin{fdefinition}[]
	\normalfont Se dice que $\displaystyle f:\left[a,b\right] \to \R $ es \textbf{cóncava} si $\displaystyle \forall x,y \in [a,b] $, $\displaystyle x<y $, y $\displaystyle \forall \alpha \in [0,1] $, 
	\[ f\left(\alpha x + \left(1-\alpha \right)y\right) \geq \alpha f\left(x\right) + \left(1-\alpha \right)f\left(y\right) .\]
\end{fdefinition}
Es decir, la gráfica de $\displaystyle f|_{[x,y]} $ queda por encima de la recta que pasa por los puntos $\displaystyle \left(x,f\left(x\right)\right) $ e $\displaystyle \left(y,f\left(y\right)\right) $.
\begin{fprop}[]
\normalfont Sea $\displaystyle f: I \subset \R \to \R $ una función definida sobre un intervalo o semirrecta. Son equivalentes:
\begin{description}
\item[(a)] $\displaystyle f $ es convexa sobre $\displaystyle I $ si y solo si $\displaystyle \forall a,b \in I $, $\displaystyle \forall x \in \left(a,b\right) $ se tiene que
	\[ \frac{f\left(x\right)-f\left(a\right)}{x-a} \leq \frac{f\left(b\right)-f\left(a\right)}{b-a} \iff \frac{f\left(b\right)-f\left(x\right)}{b-x} \geq \frac{f\left(b\right)-f\left(a\right)}{b-a}.\]
	
\item[(b)] $\displaystyle f $ es cóncava sobre $\displaystyle I $ si y solo si $\displaystyle \forall a,b \in I $, $\displaystyle \forall x \in \left(a,b\right) $ se tiene que 
	\[\frac{f\left(x\right)-f\left(a\right)}{x-a} \geq \frac{f\left(b\right)-f\left(a\right)}{b-a} \iff \frac{f\left(b\right)-f\left(x\right)}{b-x} \leq \frac{f\left(b\right)-f\left(a\right)}{b-a} .\]	
\end{description}
\end{fprop}
\begin{proof}
La demostración de \textbf{(b)} es análoga a \textbf{(a)}. Tenemos que si $\displaystyle f $ es convexa, 
\[ f\left(x\right) \leq \frac{f\left(b\right)-f\left(a\right)}{b-a}\left(x-a\right) + f\left(a\right) = r\left(x\right) .\]
Operando, obtenemos que 
\[\frac{f\left(x\right)-f\left(a\right)}{x-a} \leq \frac{f\left(b\right)-f\left(a\right)}{b-a} .\]
El recíproco se demuestra haciendo las mismas cuentas al revés.
\end{proof}
\begin{fprop}[]
\normalfont Sea $\displaystyle f : I \subset \R \to \R $ una función sobre un intervalo o semirrecta.
\begin{description}
\item[(a)] Si $\displaystyle f $ es convexa en $\displaystyle I $ y derivable en $\displaystyle I $, entonces $\displaystyle f' $ es creciente.
\item[(b)] Si $\displaystyle f $ es cóncava en $\displaystyle I $ y derivable $\displaystyle I $, entonces $\displaystyle f' $ es decreciente.
\end{description}
\end{fprop}
\begin{proof}
La demostración de \textbf{(b)} es análoga a la de \textbf{(a)}. Sea $\displaystyle a \in I $ con $\displaystyle a < b $. Dada $\displaystyle a, a + h_{1} $ y $\displaystyle a + h_{2} $ con $\displaystyle 0 < h_{1} < h_{2} $, por la proposición anterior tenemos que, dado que son cocientes crecientes,
\[  \frac{f\left(a + h_{1}\right)-f\left(a\right)}{h_{1}} \leq \frac{f\left(a + h_{2}\right)-f\left(a\right)}{h_{2}} .\]
Es decir, las pendientes $\displaystyle \frac{f\left(a+h\right)-f\left(a\right)}{h} $, decrecen cuando $\displaystyle h > 0 $ tiende a $\displaystyle 0 $. Así, como $\displaystyle f $ es derivable en $\displaystyle a $ tenemos que
\[ f'\left(a\right) = f'\left(a^{+}\right) =\lim_{h \to 0^{+}}\frac{f\left(a+h\right)-f\left(a\right)}{h} \leq \frac{f\left(a+h\right)-f\left(a\right)}{h} .\]
Por el otro lado, si $\displaystyle h_{2} < h_{1} < 0 $, por la proposición anterior tenemos que 
\[ \frac{f\left(b\right)-f\left(b + h_{2}\right)}{h_{2}} \leq \frac{f\left(b\right)-f\left(b+h_{2}\right)}{h_{2}}  .\]
Es decir, las pendientes $\displaystyle \frac{f\left(b+h\right)-f\left(b\right)}{h} $ decrecen cuando $\displaystyle h\to 0^{- } $. Así, dado que $\displaystyle f $ es derivable en $\displaystyle b $ tenemos que
\[ f'\left(b\right) = f'\left(b^{-}\right) = \lim_{h \to 0^{-}}\frac{f\left(b+h\right)-f\left(b\right)}{h} \geq \frac{f\left(b+h\right)-f\left(b\right)}{h} .\]
Si $\displaystyle a < x < b $, por lo visto anteriormente tenemos que
\[ f'\left(a\right) \leq \frac{f\left(x\right)-f\left(a\right)}{x-a} \leq \frac{f\left(b\right)-f\left(a\right)}{b-a} \leq \frac{f\left(b\right)-f\left(x\right)}{b-x} \leq f'\left(b\right) .\]
Por tanto, $\displaystyle f' $ es creciente.
\end{proof}
\begin{fcolorary}[]
\normalfont Sea $\displaystyle f : I \subset \R \to \R $, donde $\displaystyle I $ es un intervalo o semirrecta y $\displaystyle f $ es derivable dos veces en $\displaystyle I $.
\begin{description}
\item[(a)] Si $\displaystyle f $ es convexa, $\displaystyle f'' \geq 0 $.
\item[(b)] Si $\displaystyle f $ es cóncava, $\displaystyle f'' \leq 0 $.
\end{description}
\end{fcolorary}
\begin{proof}
La demostración de \textbf{(b)} es análoga a la de \textbf{(a)}. Si $\displaystyle f $ es convexa y como existe $\displaystyle f'' $, tenemos que por la proposición anterior, si $\displaystyle f' $ es creciente, $\displaystyle f'' \geq 0$.
\end{proof}
\begin{flema}[]
	\normalfont Sea $\displaystyle f : [a,b] \to \R $ derivable en $\displaystyle \left[a,b\right] $ tal que $\displaystyle f\left(a\right) = f\left(b\right) $.
	\begin{description}
		\item[(a)] Si $\displaystyle f' $ es creciente, entonces $\displaystyle f\left(x\right) \leq f\left(a\right) $, $\displaystyle \forall x \in [a,b] $.
		\item[(b)] Si $\displaystyle f' $ es decreciente, entonces $\displaystyle f\left(x\right)\geq f\left(a\right) $, $\displaystyle \forall x \in [a,b] $.
	\end{description}
\end{flema}
\begin{proof}
	Supongamos que existe $\displaystyle x \in [a,b] $ con $\displaystyle f\left(x\right) > f\left(a\right) $. Como $\displaystyle f|_{[a,b]} $ es continua, existe $\displaystyle x_{0} \in \left(a,b\right) $ tal que $\displaystyle f\left(x_{0}\right) \geq f\left(x\right) $, $\displaystyle \forall x \in [a,b] $, en concreto $\displaystyle f\left(x_{0}\right) > f\left(a\right) $, y además $\displaystyle f'\left(x_{0}\right) =0 $.
	Por el teorema del valor medio, existe $\displaystyle x_{1} \in \left(a,x_{0}\right) $ tal que 
	\[f'\left(x_{1}\right) = \frac{f\left(x_{0}\right)-f\left(a\right)}{x_{0}-a} > 0 .\]
	Entonces, tenemos que $\displaystyle f'\left(x_{1}\right) > f'\left(x_{0}\right) $, lo cual es una contradicción, pues habíamos dicho que la derivada era creciente.
\end{proof}
\begin{fprop}[]
\normalfont Sea $\displaystyle f : I \subset \R \to \R $ con $\displaystyle I $ intervalo o semirrecta, y $\displaystyle f $ derivable. 
\begin{description}
\item[(a)] Si $\displaystyle f' $ es creciente, entonces $\displaystyle f $ es convexa.
\item[(b)] Si $\displaystyle f' $ es decreciente, entonces $\displaystyle f $ es cóncava.
\end{description}
\end{fprop}
\begin{observation}
\normalfont Si existe $\displaystyle f'' $, entonces 
\begin{description}
\item[(a)] $\displaystyle f'' \geq 0  $ implica que $\displaystyle f' $ es creciente, lo cual implica que $\displaystyle f $ es convexa.
\item[(b)] $\displaystyle f'' \leq 0 $ implica que $\displaystyle f' $ es decreciente, lo cual implica que $\displaystyle f $ es cóncava.
\end{description}
\end{observation}
\begin{proof}
	Sean $\displaystyle a,b \in I $ y $\displaystyle x \in \left(a,b\right) $. Definimos la función $\displaystyle g : [a,b] \to \R $  
\[ g\left(x\right) = f\left(x\right)-\frac{f\left(b\right)-f\left(a\right)}{b-a}\left(x-a\right)-f\left(a\right) .\]
Tenemos que $\displaystyle g $ es derivable. En efecto:
\[g'\left(x\right) = f'\left(x\right)-\frac{f\left(b\right)-f\left(a\right)}{b-a} .\]
Dado que $\displaystyle f' $ es creciente, $\displaystyle g' $ es creciente. Tenemos que $\displaystyle g\left(a\right) = g\left(b\right) = 0 $. Por el lema anterior sabemos que $\displaystyle g\left(x\right) \leq g\left(a\right) = 0 $, $\displaystyle \forall x \in \left[a,b\right] $. Así, tenemos que 
\[ f\left(x\right) \leq \frac{f\left(b\right)-f\left(a\right)}{b-a}\left(x-a\right) + f\left(a\right), \; \forall x \in [a,b] .\]
\end{proof}
Todo lo anterior prueba el siguiente teorema.
\begin{ftheorem}[]
\normalfont Sea $\displaystyle f: I \subset \R \to \R  $ con $\displaystyle I $ intervalo o semirrecta, con $\displaystyle f $ derivable.
\begin{description}
\item[(a)] $\displaystyle f $ es convexa en $\displaystyle I $ si y solo si $\displaystyle f' $ es creciente. Si existe $\displaystyle f'' $, entonces $\displaystyle f $ es convexa si y solo si $\displaystyle f'' \geq 0 $.
\item[(b)] $\displaystyle f $ es cóncava en $\displaystyle I $ si y solo si $\displaystyle f' $ es decreciente. Si existe $\displaystyle f'' $, entonces $\displaystyle f $ es cóncava si y solo si $\displaystyle f'' \leq 0 $.
\end{description}
\end{ftheorem}
\section{Puntos críticos}
\begin{fdefinition}[Punto crítico]
\normalfont Sea $\displaystyle f : A \subset \R \to \R $ derivable. Se dice que $\displaystyle f $ tiene un \textbf{punto crítico} en $\displaystyle a \in A $ si $\displaystyle f'\left(a\right) = 0 $. \footnote{Puede considerarse también punto crítico si $\displaystyle f''\left(a\right) = 0 $, en el caso en el que exista $\displaystyle f'' $.} 
\end{fdefinition}
\begin{observation}
\normalfont Los puntos críticos de una función son candidatos a máximos y mínimos relativos de la función y, por tanto, puntos donde puede cambiar el crecimiento de una función. Por otro lado, los candidatos a puntos de inflexión son los puntos críticos de la función derivada.
\end{observation}
\begin{ftheorem}[]
\normalfont Sea $\displaystyle f : \left(a,b\right) \to \R $ derivable dos veces en $\displaystyle \left(a,b\right) $ y sea $\displaystyle x_{0} \in \left(a,b\right) $ un punto crítico. 
\begin{description}
\item[(a)] Si $\displaystyle f'\left(x_{0}\right) = 0 $ y $\displaystyle f''\left(x_{0}\right) < 0 $, entonces $\displaystyle x_{0} $ es un máximo local.
\item[(b)] Si $\displaystyle f'\left(x_{0}\right) = 0 $ y $\displaystyle f''\left(x_{0}\right) > 0 $, entonces $\displaystyle x_{0} $ es un mínimo local.
\end{description}
\end{ftheorem}
\begin{proof}
La demostración se verá en unas semanas.
\end{proof}
\begin{fdefinition}[Punto de inflexión]
\normalfont Sea $\displaystyle f : \left(a,b\right) \to \R $. Se dice que $\displaystyle x_{0} \in \left(a,b\right) $ es un \textbf{punto de inflexión} si para $\displaystyle \delta > 0 $, en $\displaystyle \left(x_{0}-\delta, x_{0}\right) $ es cóncava y en $\displaystyle \left(x_{0}, x_{0} + \delta \right) $ es convexa (o viceversa).
\end{fdefinition}
\begin{ftheorem}[]
\normalfont Sea $\displaystyle f : \left(a,b\right) \to \R $ tal que $\displaystyle \exists f''\left(x\right) $, $\displaystyle \forall x \in \left(a,b\right) $. Si $\displaystyle x_{0} \in \left(a,b\right) $ es un punto de inflexión, entonces $\displaystyle f''\left(x_{0}\right) = 0 $.
\end{ftheorem}
\begin{proof}
Sabemos que $\displaystyle f' $ es continua, puesto que $\displaystyle f'' $ existe. Si $\displaystyle f $ es convexa en $\displaystyle \left(x_{0}-\delta, x_{0}\right) $, entonces $\displaystyle f' $ es creciente y, al ser continua, tenemos que 
\[ f'\left(x_{0}\right) = \sup \left\{ f'\left(x\right) \; : \; x \in \left(x_{0}-\delta, x_{0} \right)\right\}  .\]
Similarmente, si $\displaystyle f $ es cóncava en $\displaystyle \left(x_{0}, x_{0} + \delta \right) $, dado que $\displaystyle f' $ es decreciente y es continua, tenemos que 
\[ f'\left(x_{0}\right) = \inf \left\{ f'\left(x\right) \; : \; x \in \left(x_{0}, x_{0} + \delta \right)\right\}  .\]
Así, tenemos que $\displaystyle f' $ tiene un máximo local en $\displaystyle x_{0} $ y, dado que existe $\displaystyle f''\left(x_{0}\right) $, tiene que ser necesariamente que $\displaystyle f''\left(x_{0}\right) =0 $.
\end{proof}
\begin{eg}
\normalfont El recíproco no es cierto. Consideremos $\displaystyle f\left(x\right) = x^{4} $. Tenemos que $\displaystyle f'\left(x\right) = 4x^{3} $ y $\displaystyle f''\left(x\right) = 12x^{2} \geq 0 $. Tenemos que $\displaystyle f''\left(x\right) = 0 \iff x = 0 $. Si embargo, la función no tiene un punto de inflexión en $\displaystyle x = 0 $, sino que se trata de un mínimo.
\end{eg}
\begin{eg}
	\normalfont Consideremos $\displaystyle f\left(x\right) = \frac{x^{3}-x^{2}-x}{x^{2}-x} $. Vamos a estudiar su gráfica. Tenemos que $\displaystyle \dom\left(f\right) = \R / \left\{ 0,1\right\}  $. Tenemos que en $\displaystyle x = 0 $ tiene una discontinuidad evitable. Estudiamos los límites:
	\[ \lim_{x \to -\infty}\frac{x^{2}-2x-1}{x-1} = - \infty, \quad \lim_{x \to \infty}\frac{x^{2}-2x-1}{x-1} = \infty .\]
	También 
	\[\lim_{x \to 1^{+}}f\left(x\right) = - \infty, \quad \lim_{x \to 1^{-}}f\left(x\right) = \infty .\]
	Es decir, $\displaystyle f $ tiene una asíntota vertical en $\displaystyle x = 1 $. Además, también tiene una asíntota oblicua. En efecto,
	\[ \lim_{x \to \infty} \frac{f\left(x\right)}{x} = \lim_{x \to \infty}\frac{x^{2}-2x-1}{x^{2}-x} = 1  .\]
\[ \lim_{x \to \infty}\left(f\left(x\right)-x\right) = \lim_{x \to \infty}\frac{x^{2}-2x-1-x\left(x-1\right)}{x-1} = \lim_{x \to \infty} \frac{-x -1}{x-1} = -1.\]
Así, la ecuación de la asíntota oblicua será $\displaystyle y = x -1 $.
	Estudiamos la derivada. Si $\displaystyle x \neq 0, 1 $,
\[ f'\left(x\right)= \frac{x^{2}-2x+3}{\left(x-1\right)^{2}} = \frac{\left(x-1\right)^{2}+3}{\left(x-1\right)^{2}}\geq 0.\]
Por tanto, tenemos que la función es creciente.
\end{eg}

