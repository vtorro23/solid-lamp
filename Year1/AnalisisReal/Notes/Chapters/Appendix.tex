\chapter{Construcción de $\displaystyle \R $}

Sabemos qué son los números naturales, $\displaystyle \N $. Con ellos definimos los números enteros, $\displaystyle \Z $, y a partir de estos se introduce $\displaystyle \Q $, el cuerpo de los números racionales. A partir de aquí, existen diversas construcciones de los números reales $\displaystyle \R $. Veamos una que está basada en la siguiente idea: vamos a interpretar un número real como una sucesión de Cauchy. \\ \\
Sea $\displaystyle \mathcal{C} = \left\{ \left\{ x_{n}\right\} _{n\in\N}\subset\Q\; : \; \left\{ x_{n}\right\} _{n\in\N} \; \text{es de Cauchy}\right\}  $. Es decir, $\displaystyle \forall \epsilon > 0 $  con $\displaystyle \epsilon \in \Q $, $\displaystyle \exists n_{0} \in \N $ tal que si $\displaystyle m, n \geq n_{0} $, $\displaystyle \left|x_{n}-x_{m}\right|<\epsilon  $. En $\displaystyle \mathcal{C} $ se define la siguiente relación de equivalencia $\displaystyle \mathcal{R} $:
\[ \left\{ x_{n}\right\} _{n\in\N} \mathcal{R} \left\{ y_{n}\right\} _{n\in\N} \iff \lim_{n \to \infty}\left(x_{n}-y_{n}\right)= 0 .\]
A partir de esto, definimos 
\[\R:= \mathcal{C}/\mathcal{R} = \left\{ [ \left\{ x_{n}\right\} _{n\in\N}] \; : \; \left\{ x_{n}\right\} _{n\in\N}\in\mathcal{C}\right\}  .\]

En $\displaystyle \R $ definimos la suma:
\[[ \left\{ x_{n}\right\} _{n\in\N}] + [ \left\{ y_{n}\right\} _{n\in\N}] = [ \left\{ x_{n}+y_{n}\right\} _{n\in\N}].\]
Esta definición es coherente con $\displaystyle \mathcal{R} $, pues si $\displaystyle \left\{ x'_{n}\right\} _{n\in\N} \in [ \left\{ x_{n}\right\} _{n\in\N}] $ y $\displaystyle \left\{ y'_{n}\right\} _{n\in\N}\in[ \left\{ y_{n}\right\} _{n\in\N}]  $. Entonces tenemos que
\[\lim_{n \to \infty}\left(x'_{n}+y'_{n} - \left(x_{n}+y_{n}\right)\right) = 0  \iff [ \left\{ x'_{n}+y'_{n}\right\}] = [ \left\{ x_{n}+y_{n}\right\}]   .\]
Análogamente, definimos el producto de la siguiente manera:
\[[ \left\{ x_{n}\right\}] \cdot [ \left\{ y_{n}\right\}] = [ \left\{ x_{n} y_{n}\right\}]  .\]
Primero vamos a ver que el producto de sucesiones de Cauchy también es sucesión de Cauchy, para poder hablar de su clase de equivalencia.
\[x_{n}y_{n}-x_{m}y_{m} = x_{n}y_{n}-x_{m}y_{m} + x_{n}y_{m}-x_{n}y_{m} = x_{n}\left(y_{n}-y_{m}\right)+y_{m}\left(x_{n}-x_{m}\right) \to 0 .\]
Si $\displaystyle \left\{ x'_{n}\right\} \in [ \left\{ x_{n}\right\}]  $ y $\displaystyle \left\{ y'_{n}\right\} \in [ \left\{ y_{n}\right\}]  $. Queremos ver que $\displaystyle [ \left\{ x'_{n}y'_{n}\right\}]= [ \left\{ x_{n}y_{n}\right\}]  $. 
\[x'_{n}y'_{n}-x_{n}y_{n} = x'_{n}y'_{n} - x_{n}y'_{n} + x_{n}y'_{n} - x_{n}y_{n} = y'_{n}\left(x'_{n}-x_{n}\right)+x_{n}\left(y'_{n}-y_{n}\right) \to 0 .\]
Por tanto, el producto está bien definido. 
\begin{fdefinition}[]
\normalfont Tenemos que $\displaystyle \Q\subset\R $ tal que si $\displaystyle x \in \Q $, entonces  $\displaystyle x \to [ \left\{ x\right\}]  $, donde 
\[ \left\{ x\right\} _{n\in\N} = \left\{ x, x, \ldots, x, \ldots\right\}  .\]
\end{fdefinition}
Así, definimos $\displaystyle 0 = [ \left\{0\right\}]  $ y $\displaystyle 1 = [ \left\{1\right\}]  $. Si $\displaystyle x \in \R $, con $\displaystyle x = [ \left\{ x_{n}\right\}]  $, definimos 
\[-x= [ \left\{- x_{n}\right\}]  .\]
Así, hemos definido el elemento neutro de la suma y del producto, así como el inverso de la suma. Ahora tenemos que encontrar el inverso del producto. Veamos que existe $\displaystyle \left\{ x'_{n}\right\} _{n\in\N} \in [ \left\{ x_{n}\right\}] \in \R/ \left\{ 0\right\}  $ tal que para algún $\displaystyle \epsilon \in \Q $, $\displaystyle \epsilon > 0 $, o bien $\displaystyle x'_{n} > \epsilon, \; \forall n \in \N $ o $\displaystyle x'_{n} < - \epsilon  $, $\displaystyle \forall n \in \N $. En efecto, como $\displaystyle x_{n} $ no tiende a $\displaystyle 0 $, existe $\displaystyle \epsilon > 0 $ con $\displaystyle \epsilon \in \Q $ tal que $\displaystyle \forall n_{0} \in \N $, $\displaystyle \exists n\geq n_{0} $ tal que $\displaystyle \left|x_{n}\right|>2\epsilon  $.
Tomando $\displaystyle k \in \N $, exiset $\displaystyle n_{k} \in \N $ tal que $\displaystyle \left|x_{n_{k}}\right| > 2\epsilon  $. Como la sucesión es de Cauchy, dado $\displaystyle \epsilon > 0 $, existe $\displaystyle j \in \N $ tal que $\displaystyle \forall m,n \geq j $, $\displaystyle \left|x_{n}-x_{m}\right| < \epsilon  $. Sea $\displaystyle k \in \N $ tal que $\displaystyle n_{k} \geq j $. Por tanto, $\displaystyle \forall m \geq j $, $\displaystyle \left|x_{n_{k}}-x_{m}\right| < \epsilon  $, por lo que $\displaystyle x_{m} > \epsilon  $. Definimos ahora
\[x'_{n} = 
\begin{cases}
1, \; 1 \leq n \leq j \\
x_{n}, \; n > j
\end{cases}
.\]
Así, $\displaystyle [ \left\{ x'_{n}\right\}] = [ \left\{ x_{n}\right\}]  $ y ahora podemos definir
\[[ \left\{ x_{n}\right\}] ^{-1} = \frac{1}{[ \left\{ x_{n}\right\}] } = \left[ \left\{ \frac{1}{x'_{n}}\right\}\right]  .\]
Entonces, claramente se cumple que $\displaystyle [ \left\{ x_{n}\right\}] \cdot [ \left\{ x_{n}\right\}] ^{-1} = 1 $. Sólamente queda ver que $\displaystyle [ \left\{ x_{n}\right\}] ^{-1} $ es sucesión de Cauchy.
Así, podemos probar que $\displaystyle \left(\R, + \cdot\right) $ es un cuerpo abeliano.

\begin{fdefinition}[]
\normalfont $\displaystyle [ \left\{ x_{n}\right\}] > 0 $ si y solo si existe $\displaystyle \epsilon > 0 $ con $\displaystyle \epsilon \in \Q $, para el que existe $\displaystyle n_{0} \in \N $ tal que $\displaystyle \forall n \geq n_{0} $, 
\[
\begin{cases}
x_{n} > \epsilon, \; \forall n\geq n_{0} \\
\text{o} \\
x_{n} < -\epsilon, \; \forall n\geq n_{0}.
\end{cases}
\]
\end{fdefinition}

Así, tenemos que $\displaystyle \left(\R, + , \cdot, >\right) $ es un cuerpo abeliano totalmente ordenado y arquimediano. 
\chapter{Monstruo de Weirstrass}
\section{Monstruo de Weirstrass}
Para $\displaystyle x \in \R $ definimos la distancia de $\displaystyle x $ a $\displaystyle \Z $ por 
\[ \left[x\right] = d\left(x,\Z\right)= \min \left\{ \left|x-z\right| \; : \; z \in \Z\right\}  .\]
Esta función es continua en todo $\displaystyle \R $ y no es derivable en los puntos de la forma $\displaystyle x = \frac{k}{2} $, $\displaystyle k \in \Z $. Definimos la siguiente sucesión de funciones:
\[f_{n}\left(x\right) = \frac{1}{10^{n}}\left[10^{n}x\right], \; \forall x \in \R, \; \forall n \in \N .\]
Nuevamente, son funciones continuas en todo $\displaystyle \R $ y no derivables en los puntos de la forma $\displaystyle x = \frac{k}{2}\frac{1}{10^{n}} $, $\displaystyle k \in \Z $.
\begin{ftheorem}[]
\normalfont La función $\displaystyle f\left(x\right) = \sum^{\infty}_{n = 1}f_{n}\left(x\right) = \sum^{\infty}_{n = 1}\frac{1}{10^{n}}\left[10^{n}x\right] $ 
es continua en todo $\displaystyle \R $ y no es derivable en ningún punto de $\displaystyle \R $.
\end{ftheorem}
\begin{proof}
Aplicando la prueba M de Weirstrass:
\[ \left|f_{n}\left(x\right)\right| < \frac{1}{10^{n}}, \; \forall x \in \R .\]
Como $\displaystyle \sum^{\infty}_{n = 1}\frac{1}{10^{n}}<\infty $, tenemos que la serie de funciones converge uniformemente a su límite puntual $\displaystyle f $ en todo $\displaystyle \R $. De la convergencia uniforme deduciemos que $\displaystyle f $ es continua en todo $\displaystyle \R $. \\
Observemos que todas las funciones $\displaystyle f_{n} $ son 1-periódicas y por tanto $\displaystyle f $ también lo es. Ahora, para ver que $\displaystyle f $ no es derivable en ningún punto de $\displaystyle \R $ fijamos $\displaystyle a \in \R $ y encontramos una sucesión $\displaystyle \left\{ h_{m}\right\} _{m\in\N}\subset \R $ tal que $\displaystyle h_{m} \to 0 $ y no existe 
\[\lim_{m \to \infty}\frac{f\left(a+h_{m}\right)-f\left(a\right)}{h_{m}} .\]
Como $\displaystyle f $ es 1-periódica, basta con tomar $\displaystyle a \in (0,1] $. Tenemos que $\displaystyle a $ se puede escribir en forma decimal, 
\[ a = \sum^{\infty}_{m = 1}\frac{a_{m}}{10^{m}} .\]
Podemos ver que 
\[\frac{1}{2} = 0,5 = 0,49999 \ldots .\]
Definimos la sucesión $\displaystyle \left\{ h_{m}\right\} _{m\in\N} $ por:
\[h_{m} =
\begin{cases}
	\frac{1}{10^{m}}, \; a_{m} \not\in \left\{ 4,9\right\} \\
	-\frac{1}{10^{m}}, \; a_{m} \in \left\{ 4,9\right\} 
\end{cases}
.\]
Claramente, $\displaystyle h_{m} \to 0 $. Además, 
\[
\begin{split}
	\frac{f\left(a+h_{m}\right)-f\left(a\right)}{h_{m}} = & \sum^{\infty}_{n = 1}\frac{1}{10^{n}}\frac{\left[10^{n}\left(a+h_{m}\right)\right] -\left[10^{n}a\right] }{\pm10^{-m}} 
	=  \sum^{\infty}_{n = 1}\pm10^{m - n}\left(\left[10^{n}\left(a+h_{m}\right)\right] -\left[10^{n}a\right] \right) \\
	= & \sum^{m - 1}_{n = 1}\pm 10^{m - n}\left(\left[10^{n}\left(a+h_{m}\right)\right] -\left[10^{n}a\right] \right).
\end{split}
\]
La última igualdad iene de que si $\displaystyle n \geq m $, entonces $\displaystyle 10^{n}h_{m} $ es entero, como la función $\displaystyle \left[x\right]  $ es 1-periódica, se tiene que $\displaystyle \left[10^{n}\left(a+h_{m}\right)\right] =\left[10^{n}a\right]  $. Por otra parte, para $\displaystyle n < m $, podemos escribir
\[10^{n}a = l + 0,a_{n+1}a_{n+2} \ldots a_{m} \ldots .\]
\[10^{n}\left(a+h_{m}\right) = l + 0,a_{n+1}a_{n+2} \ldots \left(a_{m}\pm1\right) \ldots, \; l \in \Z .\]
Supongamos que 
\[0,a_{n+1}a_{n+2} \ldots a_{m} \ldots \leq \frac{1}{2} = 0,4999 \ldots .\]
Entonces, por la elección de $\displaystyle h_{m} $,
\[0,a_{n+1}a_{n+2} \ldots \left(a_{m}\pm 1\right) \ldots \leq \frac{1}{2} = 0,4999 \ldots .\]
Así, tenemos que $\displaystyle \left[10^{n}\left(a+h_{m}\right)\right] -\left[10^{n}a\right] = \pm 1 $, por lo que $\displaystyle 10^{m - n}\left(\left[10^{n}\left(a+h_{m}\right)\right] -\left[10^{n}a\right] \right) = \pm 1 $. Análogamente, si $\displaystyle 0,a_{n+1}a_{n+2} \ldots a_{m} \ldots > \frac{1}{2} = 0,4999 \ldots $, entonces, por la elección de $\displaystyle h_{m} $,
\[0,a_{n+1}a_{n+2} \ldots \left(a_{m}\pm 1\right) \ldots > \frac{1}{2} = 0,4999 \ldots .\]
Así, tenemos que $\displaystyle \left[10^{n}\left(a+h_{m}\right)\right] -\left[10^{n}a\right] = \pm 10^{n-m} $, por lo que $\displaystyle 10^{m - n}\left(\left[10^{n}\left(a+h_{m}\right)\right] -\left[10^{n}a\right] \right) = \pm 1 $. Así, acabamos de ver que 
\[\frac{f\left(a+h_{m}\right)-f\left(a\right)}{h_{m}} \]
es la suma de $\displaystyle m - 1 $ números, cada uno de los cuales es 1 o $\displaystyle - 1 $. Ahora, al sumar a un entero un 1 o $\displaystyle - 1 $ cambia la paridad del número, por tanto el cociente anterior no puede converger porque es una sucesión de enteros alternativamente pares e impares.
\end{proof}

