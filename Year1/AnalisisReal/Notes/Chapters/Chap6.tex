\chapter{La Integral}
\section{Definición de la integral}
\begin{fdefinition}[Partición]
	\normalfont Sea $\displaystyle \left[a,b\right]  $ un intervalo cerrado de $\displaystyle \R $. Se llama \textbf{partición} $\displaystyle P $ de $\displaystyle [a,b] $ a todo subconjunto finito de $\displaystyle [a,b] $ que contiene a $\displaystyle a $ y $\displaystyle b $. 
\end{fdefinition}
\begin{observation}
\normalfont Las particiones tendrán la forma
\[ P = \left\{ a = t_{0}, t_{1}, \ldots, t_{n} = b\right\}, \; t_{i} < t_{i+1}, \; \forall i = 0,1, \ldots, n-1 .\]
\begin{notation}
\normalfont Se llama $\displaystyle P\left([a,b]\right) = \left\{ P \; \text{partición de } [a,b]\right\} $.
\end{notation}
\end{observation}
\begin{fdefinition}[]
	\normalfont Dadas $\displaystyle P, P' \in P\left([a,b]\right) $, decimos que $\displaystyle P' $ es \textbf{más fina} que $\displaystyle P $ si $\displaystyle P \subsetneq P' $.
\end{fdefinition}
\begin{eg}
	\normalfont Consideremos el intervalo $\displaystyle [1,2] $ y las particiones $\displaystyle P = \left\{ 1, \frac{3}{2}, 2\right\} \subsetneq P' = \left\{ 1, \sqrt{2},\frac{3}{2}, 2\right\}  $.
\end{eg}
\begin{fdefinition}[]
	\normalfont Sea $\displaystyle f: [a,b] \to \R $ \footnote{De momento estamos solo considerando funciones cuyas imágenes son positivas.} acotada y sea $\displaystyle P = \left\{ t_{0}= a, t_{1}, \ldots, t_{n-1}, t_{n} = b\right\}  $ una partición. Se definen
	\[ M_{i} = \sup \left\{ f\left(t\right) \; : \; t \in [t_{i}, t_{i+1}]\right\}, \; i = 0, \ldots, n-1  .\]
	\[ m_{i} = \inf \left\{ f\left(t\right) \; : \; t \in [t_{i}, t_{i+1}]\right\}, \; i = 0, \ldots, n-1  .\]
\footnote{Sabemos que existen (no tienen por qué alcanzarse) porque $\displaystyle f $ está acotada.} 	
\end{fdefinition}
\begin{fdefinition}[]
	\normalfont Sea $\displaystyle f : [a,b] \to \R $ acotada y sea $\displaystyle P = \left\{ t_{0}= a, t_{1}, \ldots, t_{n-1}, t_{n} = b\right\}  $ una partición. Se llama \textbf{suma superior} de $\displaystyle f $ respecto de $\displaystyle P $ y se escribe 
	\[ S\left(f, P\right) = \sum^{n-1}_{i = 0}M_{i}\left(t_{i+1}-t_{i}\right) .\]
	Similarmente, se llama \textbf{suma inferior} de $\displaystyle f $ respecto de $\displaystyle P $ a
	\[ I\left(f,P\right)= \sum^{n-1}_{i = 0}m_{i}\left(t_{i+1}-t_{i}\right) .\]
\end{fdefinition}
\begin{observation}
\normalfont Tenemos que $\displaystyle I\left(f,P\right) \leq S\left(f,P\right) $, pues $\displaystyle m_{i} \leq M_{i} $, $\displaystyle \forall i = 0, \ldots, n-1 $.
\end{observation}
\begin{flema}[]
	\normalfont Sea $\displaystyle f : \left[a,b\right] \to \R $ acotada y sean $\displaystyle P,P' \in P\left([a,b]\right) $ con $\displaystyle P' $ más fina que $\displaystyle P $. Entonces, 
	\[ I\left(f,P\right) \leq I\left(f,P'\right) \leq S\left(f,P'\right) \leq S\left(f,P\right) .\]
\end{flema}
\begin{proof}
	Supongamos que $\displaystyle P' = P \cup \left\{ u\right\}  $, donde $\displaystyle t_{i} < u < t_{i+1} $. Observemos que 
	\[ m_{i} = \inf \left\{ f\left(t\right) \; : \; t \in \left[t_{i}, t_{i+1}\right] \right\} \leq \min \left\{ m_{[t_{i}, u]}, m_{[u, t_{i+1}]}\right\} 
	.\]
\[ M_{i} = \sup \left\{ f\left(t\right) \; : \; t \in \left[t_{i}, t_{i+1}\right] \right\} \geq \max \left\{ M_{[t_{i}, u]}, M_{[u, t_{i+1}]}\right\}.\]
Así, tenemos que 
\[ I\left(f, P\right)= \sum^{n-1}_{j = 0}m_{j}\left(t_{j+1}-t_{j}\right) \leq \sum^{n-1}_{j=0, i \neq j} m_{j}\left(t_{j+1}-t_{j}\right) + m_{[t_{i},u]}\left(u-t_{i}\right) + m_{[u,t_{i+1}]}\left(t_{i+1}-u\right) = I\left(f,P'\right).\]
Similarmente, tenemos que 
\[ S\left(f,P'\right) = \sum^{n-1}_{j=0,i\neq j}M_{j}\left(t_{j+1}-t_{j}\right) + M_{[t_{i}, u]}\left(u-t_{i}\right) + M_{[u,t_{i+1}]}\left(t_{i+1}-u\right) \leq \sum^{n-1}_{j=0}M_{j}\left(t_{j+1}-t_{j}\right) = S\left(f,P\right) .\]
Así, tenemos que $\displaystyle I\left(f,P\right) \leq I\left(f,P'\right) \leq S\left(f,P'\right) \leq S\left(f,P\right) $. \footnote{Las desigualdades anteriores se basan en que $\displaystyle m_{i}\left(t_{i+1}-t_{i}\right) = m_{i}\left(u - t_{i}\right) + m_{i}\left(t_{i+1}-u\right) \leq m_{[u,i+1]}\left(t_{i+1}-u\right) + m_{[i,u]}\left(u-i\right) $. } 
Repetimos el proceso hasta que $\displaystyle P \cup \left\{ u_{1}, \ldots, u_{k}\right\} = P' $.
\end{proof}
\begin{flema}[]
	\normalfont Sea $\displaystyle f: [a,b] \to \R$ acotada y $\displaystyle P,P' \in P\left([a,b]\right) $. Entonces, $\displaystyle I\left(f, P\right) \leq S\left(f, P'\right) $.
\end{flema}
\begin{proof}
Cogemos $\displaystyle P'' = P \cup P' $, así $\displaystyle P'' $ es más fina que $\displaystyle P $ y $\displaystyle P' $. Aplicando el lema anterior:
\[ I\left(f,P\right) \leq I\left(f,P''\right) \leq S\left(f,P''\right) \leq S\left(f,P'\right) .\]
\end{proof}
\begin{fdefinition}[]
	\normalfont Sea $\displaystyle f: [a,b] \to \R  $ acotada.
	\begin{description}
		\item[(a)] Se define la \textbf{integral inferior } de $\displaystyle f $ en $\displaystyle [a,b] $ 
			\[ \sup\left\{ I\left(f,P\right) \; : \; P \in P\left([a,b]\right) \right\} = \underline{\int^{b}_{a}} f   .\]
		\item[(b)] Se define la \textbf{integral superior} de $\displaystyle f $ en $\displaystyle [a,b] $ 
			\[ \inf \left\{ S\left(f,P\right) \; : \; P \in P\left([a,b]\right)\right\} = \overline{\int^{b}_{a}} f .\]
		\item[(c)] Decimos que $\displaystyle f $ es \textbf{integrable} en $\displaystyle [a,b] $ si $\displaystyle \underline{\int^{b}_{a}} f = \overline{\int^{b}_{a}} f $. A este valor se le llama la integral de $\displaystyle f $ en $\displaystyle [a,b] $:
			\[ \underline{\int^{b}_{a}} f = \overline{\int^{b}_{a}} f  = \int^{b}_{a} f  .\]
	\end{description}
\end{fdefinition}
\begin{observation}
	\normalfont Tenemos que la integral inferior siempre es menor que la superior: $\displaystyle \underline{\int^{b}_{a}} f  \leq \overline{\int^{b}_{a}} f $.
\end{observation}
\begin{eg}
\normalfont Consideremos la función 
\[f\left(x\right) = 
\begin{cases}
	1, \; x \in [0,1] \cap \Q \\
	0, \; x \in [0,1] / \Q
\end{cases}
.\]
Tenemos que $\displaystyle I\left(f, P\right) = 0 $ y $\displaystyle S\left(f,P\right) = 1 $, por lo que 
\[ \underline{\int^{1}_{0}}  f = 0 \neq 1 = \overline{\int^{1}_{0}} f .\]
Por tanto, la función de Dirichlet no es integrable.
\end{eg}
\begin{fdefinition}[Área]
	\normalfont Sea $\displaystyle f : [a,b] \to \R$ acotada y $\displaystyle f \geq 0 $. 
	\begin{description}
	\item[(a)] Se llama \textbf{recinto por debajo de la gráfica de $\displaystyle f $} al subconjunto de $\displaystyle \R^{2} $ 
		\[ A_{f} = \left\{ \left(x,y\right) \in \R^{2}\; : \; x \in [a,b], y \in [0,f\left(x\right)]\right\}  .\]
	\item[(b)] Si $\displaystyle f $ es integrable en $\displaystyle [a,b] $, se define el \textbf{área} de $\displaystyle A_{f} $ 
		\[ \text{Área }A_{f} = \int^{b}_{a} f .\]
	\end{description}
\end{fdefinition}
\section{Funciones integrables}
\begin{ftheorem}[Criterio de integrabilidad de Riemann]
	\normalfont Sea $\displaystyle f : [a,b] \to \R$ acotada. Son equivalentes:
	\begin{description}
		\item[(a)] $\displaystyle f $ es integrable en $\displaystyle [a,b] $.
		\item[(b)] $\displaystyle \forall \epsilon > 0 $, $\displaystyle \exists P \in P\left([a,b]\right) $ tal que $\displaystyle S\left(f,P\right)-I\left(f,P\right) < \epsilon  $.
	\end{description}
\end{ftheorem}
\begin{proof}
\begin{description}
	\item[(i)] Sea $\displaystyle \epsilon > 0 $. Tenemos que existe $\displaystyle P_{1} \in P\left(\left[a,b\right] \right) $ tal que 
		\[ \underline{\int^{b}_{a}} f  - \frac{\epsilon }{2} < I\left(f,P_{1}\right) .\]
		Similarmente, existe $\displaystyle P_{2} \in P\left(\left[a,b\right] \right) $ tal que 
		\[ \overline{\int^{b}_{a}} f + \frac{\epsilon }{2} > S\left(f,P_{2}\right)  .\]
	Sea $\displaystyle P_{3} = P_{1} \cup P_{2} $, por lo que $\displaystyle P_{3} $ es más fina que $\displaystyle P_{1} $ y $\displaystyle P_{2} $. Así, tenemos que 
	\[ S\left(f,P_{3}\right)-I\left(f,P_{3}\right) \leq S\left(f,P_{2}\right)-I\left(f,P_{1}\right) \leq \overline{\int^{b}_{a}} f + \frac{\epsilon }{2} - \left(\underline{\int^{b}_{a}} f - \frac{\epsilon }{2}\right) = \epsilon   .\]
\item[(ii)] Sabemos que si $\displaystyle \epsilon > 0 $ existe $\displaystyle P \in P\left(\left[a,b\right] \right) $ tal que 
	\[ \overline{\int^{b}_{a}}  f - \underline{\int^{b}_{a}} f \leq S\left(f,P\right) - I\left(f,P\right) < \epsilon .\]
	Así, tenemos que $\displaystyle \forall \epsilon > 0 $,
	\[ \overline{\int^{b}_{a}} f -\underline{\int^{b}_{a}} f < \epsilon \iff \overline{\int^{b}_{a}} f = \underline{\int^{b}_{a}} f .\]
\end{description}
\end{proof}
\begin{fprop}[]
	\normalfont Sea $\displaystyle f : \left[a,b\right] \to \R $ acotada. Sea 
	\[ P_{n} = \left\{ a, a + \frac{b-a}{n}, a + \frac{2\left(b-a\right)}{n}, \ldots, a + \frac{\left(n-1\right)\left(b-a\right)}{n}, b\right\} .\]
	Si existe $\displaystyle \lim_{n \to \infty}I\left(f,P_{n}\right) = \lim_{n \to \infty}S\left(f,P_{n}\right) $, entonces $\displaystyle f $ es integrable en $\displaystyle [a,b] $ y 
	\[ \int^{b}_{a} f = \lim_{n \to \infty}I\left(f,P_{n}\right) = \lim_{n \to \infty}S\left(f,P_{n}\right) .\]
\end{fprop}
\begin{proof}
Sea $\displaystyle \alpha = \lim_{n \to \infty}I\left(f,P_{n}\right) = \lim_{n \to \infty}S\left(f,P_{n}\right) $. Tenemos que existe $\displaystyle n_{1} \in \N $ tal que si $\displaystyle n \geq n_{1} $, $\displaystyle \left|I\left(f,P_{n}\right)-\alpha \right| < \frac{\epsilon }{2} $ y existe $\displaystyle n_{2} \in \N $ tal que si $\displaystyle n \geq n_{2} $, $\displaystyle \left|S\left(f,P_{n}\right)-\alpha \right|< \frac{\epsilon }{2} $. 
Así, si $\displaystyle n_{0} = \max \left\{ n_{1}, n_{2}\right\}  $, para $\displaystyle n \geq n_{0} $ tenemos que $\displaystyle I\left(f,P_{n}\right), S\left(f,P_{n}\right) \in \left(\alpha - \frac{\epsilon }{2}, \alpha + \frac{\epsilon }{2}\right) $. Así, tenemos que 
\[ 0 \leq S\left(f,P_{n}\right) - I\left(f,P_{n}\right) < \epsilon  .\]
Así, por el criterio de integrabilidad Riemann, tenemos que 
\[ \exists \int^{b}_{a} f = \lim_{n \to \infty}I\left(f,P_{n}\right) .\]
En efecto, tenemos que  
\[ 0 < S\left(f,P_{n}\right) - \int^{b}_{a} f \leq S\left(f,P_{n}\right)-I\left(f,P_{n}\right) \to 0 .\]
\end{proof}
\begin{observation}
	\normalfont En general, si $\displaystyle \left\{ P_{n} \; : \; n \in \N\right\}  $ es una sucesión de particiones y $\displaystyle \lim_{n \to \infty}S\left(f,P_{n}\right)-I\left(f,P_{n}\right)=0 $, tenemos que $\displaystyle f $ es integrable en $\displaystyle \left[a,b\right]  $ y $\displaystyle \int^{b}_{a} f = \lim_{n \to \infty}S\left(f,P_{n}\right) $.
\end{observation}
\begin{eg}
	\normalfont Calculamos el área de un triángulo. Sea $\displaystyle f\left(x\right) = rx $ donde $\displaystyle x \in [0,a] $. Tenemos que el área será
	\[ S = \frac{ra^{2}}{2} .\]
Tomamos la partición $\displaystyle P_{n} $ de la proposición anterior. Tenemos que 
\[ I\left(f,P_{n}\right) = \frac{ra^{2}}{n^{2}}\sum^{n-1}_{i = 0}i = \frac{ra^{2}}{n^{2}} \cdot \frac{n\left(n-1\right)}{2} \to \frac{ra^{2}}{2}.\]
\[S\left(f,P\right) = \frac{ra^{2}}{n^{2}}\sum^{n-1}_{i = 0}\left(i + 1\right) = \frac{ra^{2}}{n^{2}} \cdot \frac{n\left(n+1\right)}{2} \to \frac{ra^{2}}{2} .\]
Así, $f $ es integrable y queda que 
\[\int^{a}_{0} rx \; dx = \frac{ra^{2}}{2} .\]
\end{eg}
\begin{ftheorem}[]
	\normalfont Si $\displaystyle f : [a,b] \to \R$ es continua en $\displaystyle [a,b] $, entonces existe $\displaystyle \int^{b}_{a} f $.
\end{ftheorem}
\begin{proof}
Sea $\displaystyle b -a  $ la distancia de $\displaystyle a $ a $\displaystyle b $. Hacemos partes iguales de longitud $\displaystyle \frac{b-a}{n} $. Así, obtenemos la partición
\[ P_{n} = \left\{ a, a + \frac{b-a}{n}, \ldots, a + \frac{\left(n-1\right)\left(b-a\right)}{n}, b\right\}  .\]
Dado que $\displaystyle f $ es continua en un intervalo cerrado, tenemos que es uniformemente continua en este mismo intervalo \footnote{Podemos aplicar la continuidad uniforme porque, dado que $\displaystyle f $ es continua, tenemos que $\displaystyle M_{i}, m_{i} \in \Imagen\left(f\right) $ en cada intervalo $\displaystyle [t_{i}, t_{i+1}] $.} . Así, si $\displaystyle \epsilon > 0 $, existe $\displaystyle \delta > 0 $ tal que si $\displaystyle \left|x-y\right| < \delta  $, entonces $\displaystyle \left|f\left(x\right)-f\left(y\right)\right| < \frac{\epsilon }{b-a} $.
Además, como $\displaystyle \frac{b-a}{n} \to 0 $, existe $\displaystyle n_{0} \in \N $ tal que si $\displaystyle n \geq n_{0} $ tenemos que $\displaystyle \frac{b-a}{n} < \delta  $. Así, tenemos que si $\displaystyle n \geq n_{0} $ 
\[ S\left(f,P_{n}\right)-I\left(f,P_{n}\right) = \frac{b-a}{n}\sum^{n-1}_{i=0}\left(M_{i}-m_{i}\right) \leq \frac{b-a}{n}\sum^{n-1}_{i=0}\frac{\epsilon }{b-a} = \epsilon .\]
Por el criterio de integrabilidad tenemos que $\displaystyle f $ es integrable en $\displaystyle [a,b] $. Además, tenemos que $\displaystyle \lim_{n \to \infty}S\left(f,P_{n}\right)-I\left(f,P_{n}\right) = 0 $. Por lo visto en la observación anterior, tenemos que 
\[ \int^{b}_{a} f =\lim_{n \to \infty}S\left(f,P_{n}\right) = \lim_{n \to \infty}I\left(f,P_{n}\right) .\]
\end{proof}
\begin{fcolorary}[]
	\normalfont Sea $\displaystyle f : [a,b] \to \R $ continua y consideremos la partición
	\[ P_{n} = \left\{ a, a + \frac{b-a}{n}, \ldots, a + \frac{\left(n-1\right)\left(b-a\right)}{n}, b\right\}  .\]
	\begin{description}
		\item[(a)] Sea $\displaystyle x_{i} \in \left[a + \frac{i\left(b-a\right)}{n}, a + \frac{\left(i+1\right)\left(b-a\right)}{n}\right]  $ con $\displaystyle i = 1, \ldots, n-1 $, entonces
			\[\int^{b}_{a} f = \lim_{n \to \infty}\frac{b-a}{n}\sum^{n-1}_{i = 0}f\left(x_{i}\right) .\]
		\item[(b)] En particular, 
			\[\int^{b}_{a} f = \lim_{n \to \infty}\frac{b-a}{n}\sum^{n-1}_{i = 0}f\left(a + \frac{i\left(b-a\right)}{n}\right) .\]
		\item[(c)] En particular,
			\[\int^{b}_{a} f = \lim_{n \to \infty}\frac{b-a}{n}\sum^{n-1}_{i = 0}f\left(a + \frac{\left(i+1\right)\left(b-a\right)}{n}\right) .\]
		\item[(d)] Si $\displaystyle [a,b] = [0,1]$, 
			\[\int^{1}_{0} f = \lim_{n \to \infty}\frac{1}{n}\sum^{n-1}_{ i= 0}f\left(\frac{i}{n}\right) .\]
	\end{description}
\end{fcolorary}
\begin{proof}
Demostramos sólamente \textbf{(d)}, pues el resto de casos son análogos. Por el teorema anterior tenemos que
	\[
	\begin{split}
		\int^{1}_{0} f = & \lim_{n \to \infty}I\left(f,P_{n}\right) = \lim_{n \to \infty}\sum^{n-1}_{i = 0}m_{i} \frac{1}{n} .
	\end{split}
	\]
Dado que $\displaystyle f $ es continua tenemos que existe $\displaystyle x_{i} \in \left[\frac{i}{n}, \frac{i + 1}{n}\right]  $ con $\displaystyle f\left(x_{i}\right) = m_{i} $. Dado que $\displaystyle f $ es uniformemente continua si $\displaystyle \epsilon > 0 $ tenemos que $\displaystyle \exists \delta > 0 $ tal que si $\displaystyle \left|x-y\right|<\delta  $ se tiene que $\displaystyle \left|f\left(x\right)-f\left(y\right)\right| < \epsilon  $. 
Ahora, como $\displaystyle \frac{1}{n} \to 0 $, cogemos $\displaystyle n_{0} \in \N $ tal que $\displaystyle \forall n \geq n_{0} $ se tenga que $\displaystyle \frac{1}{n} < \delta  $:
\[ 0 \leq \sum^{n-1}_{i = 0}f\left(\frac{i}{n}\right)\frac{1}{n} - \sum^{n-1}_{i=0}m_{i}\frac{1}{n} = \frac{1}{n}\sum^{n-1}_{i=0}\left[f\left(\frac{i}{n}\right)-f\left(x_{i}\right)\right] \leq \frac{1}{n}\sum^{n-1}_{i = 0}\epsilon  = \epsilon.\]
	Así, hemos demostrado que 
	\[\lim_{n \to \infty}\frac{1}{n}\sum^{n-1}_{i = 0}f\left(\frac{1}{n}\right) - \frac{1}{n}\sum^{n-1}_{i = 0}m_{i} = 0 \iff \lim_{n \to \infty}\frac{1}{n}\sum^{n-1}_{i = 0}f\left(\frac{1}{n}\right) = \int^{1}_{0} f .\]
\end{proof}
\begin{eg}
\normalfont Vamos a calcular $\displaystyle \lim_{n \to \infty}\frac{1}{n^{3}}\left(\sum^{n}_{k=1}\left(k+n\right)\left(k-n\right)\right) $. Tenemos que 
\[
\begin{split}
	\lim_{n \to \infty}\frac{1}{n^{3}}\left(\sum^{n}_{k=1}\left(k+n\right)\left(k-n\right)\right) = & \lim_{n \to \infty}\frac{1}{n^{3}}\sum^{n}_{k=1}\left(k^{2}-n^{2}\right) \\
= & \lim_{n \to \infty}\frac{1}{n}\sum^{n}_{k=1}\frac{1}{n^{2}}\left(k^{2}-n^{2}\right) = \lim_{n \to \infty}\frac{1}{n}\sum^{n}_{k=1}\left(\frac{k}{n}\right)^{2}-1 \\
= &  \lim_{n \to \infty}\frac{1}{n}\left(\sum^{n-1}_{k=0}\left(\frac{k}{n}\right)^{2}+1\right) = \int^{1}_{0} \left(x^{2}-1\right) \; dx + \lim_{n \to \infty}\frac{1}{n} \\
= & \int^{1}_{0} \left(x^{2}-1\right) \; dx.
\end{split}
\]
\end{eg}
\section{Propiedades de la integral}
\begin{ftheorem}[]
	\normalfont Sea $\displaystyle f : \left[a,b\right]  \to \R $ integrable y $\displaystyle c \in \left(a,b\right) $. Entonces, existen $\displaystyle \int^{c}_{a} f  $ y $\displaystyle \int^{b}_{c} f  $ y además
	\[ \int^{b}_{a} f = \int^{c}_{a} f + \int^{b}_{c} f .\]
\end{ftheorem}
\begin{proof}
	La parte de existencia se puede demostrar utilizando el criterio de integrabilidad. Sabemos que si $\displaystyle \epsilon > 0 $, existe $\displaystyle P \in P\left([a,b]\right) $ tal que $\displaystyle S\left(f,P\right)-I\left(f,P\right) < \epsilon  $. Podemos suponer que $\displaystyle c \in P $, es decir, existe $\displaystyle t_{k_{0}} =c $. Si no ocurriese, $\displaystyle P' = P \cup \left\{ c\right\}  $ sería más fina que $\displaystyle P $ y 
	\[ S\left(f,P'\right)-I\left(f,P'\right) \leq S\left(f,P\right)-I\left(f,P\right) < \epsilon  .\]
Así, tenemos que 
\[S\left(f,P\right)-I\left(f,P\right) = \sum^{k-1}_{i=0}\left(M_{i}-m_{i}\right)\left(t_{i+1}-t_{i}\right) + \sum^{n-1}_{i=k}\left(M_{i}-m_{i}\right)\left(t_{i+1}-t_{i}\right)<\epsilon  .\]
Definimos la partición $\displaystyle P''= \left\{ t_{0}= a, t_{1}, \ldots, t_{k_{0}}=c\right\}  $. Tenemos que 
\[
\begin{split}
S\left(f,P''\right)-I\left(f,P''\right) \leq S\left(f,P\right) - I\left(f,P\right) < \epsilon .
\end{split}
\]
Por tanto, existe $\displaystyle \int^{c}_{a} f $. Análogamente, se puede ver que existe $\displaystyle \int^{b}_{c} f $. \\ 
Ahora, vamos a demostrar la igualdad. Si $\displaystyle \epsilon > 0 $, tenemos que, si asumimos (sin pérdida de generalidad), que $\displaystyle \int^{b}_{a} f \geq \int^{c}_{a} f +\int^{b}_{c} f $,
\[ 0\leq \int^{b}_{a} f -\left(\int^{c}_{a} f +\int^{b}_{c} f \right) \leq S\left(f,P\right) - I\left(f,P\right) < \epsilon.\]
Por tanto, como esto es cierpo $\displaystyle \forall \epsilon > 0 $, se tiene que 
\[ \int^{b}_{a} f = \int^{c}_{a} f +\int^{b}_{c} f .\]
\end{proof}
\begin{observation}
\normalfont Si convenimos para que $\displaystyle \int^{b}_{a} f = - \int^{a}_{b} f $, tenemos que el teorema anterior es cierto para todo $\displaystyle a,b,c \in \R $, siempre que las integrales que aparecen existan.
\end{observation}
\begin{observation}
	\normalfont Es fácil ver que si $\displaystyle f $ es integrable en $\displaystyle \left[a,c\right]  $ y $\displaystyle \left[c,b\right]  $, también lo será en $\displaystyle \left[a,b\right]  $. En efecto, si $\displaystyle f $ es integrable en $\displaystyle \left[a,c\right]  $ y $\displaystyle \left[c,b\right]  $, tenemos que si $\displaystyle \epsilon > 0 $ existen $\displaystyle P_{1} \in P\left(\left[a,c\right] \right) $ y $\displaystyle P_{2} \in P\left(\left[c,b\right] \right) $ tales que
	\[ S\left(f,P_{1}\right)-I\left(f,P_{1}\right) < \frac{\epsilon }{2} \quad \text{y} \quad S\left(f,P_{2}\right)-I\left(f,P_{2}\right) < \frac{\epsilon }{2} .\]
	Así, si $\displaystyle P_{3} = P_{1} \cup P_{2} $, se tiene que
	\[ S\left(f,P_{3}\right)-I\left(f,P_{3}\right) = S\left(f,P_{1}\right) + S\left(f,P_{2}\right)-\left(I\left(f,P_{1}\right) + I\left(f,P_{2}\right)\right) < \frac{\epsilon }{2} + \frac{\epsilon }{2} = \epsilon.\]
\end{observation}
\begin{ftheorem}[]
	\normalfont Sean $\displaystyle f,g : \left[a,b\right]  \to \R $ dos funciones acotadas e integrables y sea $\displaystyle \lambda \in \R $. Entonces,
	\begin{description}
	\item[(a)] $\displaystyle \exists \int^{b}_{a} f + g = \int^{b}_{a} f + \int^{b}_{a} g $.
	\item[(b)] $\displaystyle \exists \int^{b}_{a} \lambda f = \lambda \int^{b}_{a} f $.
	\end{description}
\end{ftheorem}
\begin{proof}
\begin{description}
\item[(a)] Sea $\displaystyle \epsilon > 0 $, tenemos que existe $\displaystyle P \in P\left([a,b]\right) $ tal que 
	\[ S\left(f,P\right)-I\left(f,P\right) < \frac{\epsilon }{2} \quad \text{y} \quad S\left(g,P\right)-I\left(g,P\right) < \frac{\epsilon }{2} .\]
	Tenemos que si $\displaystyle f $ y $\displaystyle g $ están acotadas, $\displaystyle f+g $ también lo estará. Si $\displaystyle \left|f\right| \leq M $ y $\displaystyle \left|g\right|\leq M $, tenemos que 
	\[ \left|f+g\right| \leq \left|f\right| + \left|g\right| \leq 2M .\]
Así, tenemos que 
\[ M_{f+g,i} = \sup \left\{ \left(f+g\right)\left(t\right) \; : \; t \in \left[t_{i}, t_{i+1}\right] \right\} \leq M_{f,i} + M_{g,i}.\]
Similarmente, obtenemos que $\displaystyle m_{f+g,i} \geq m_{f,i} + m_{g,i} $. Entonces, tenemos que 
\[
\begin{split}
	S\left(f+g,P\right)-I\left(f+g,P\right) = & \sum^{n-1}_{i=0}M_{f+g,i}\left(t_{i+1}-t_{i}\right) + \sum^{n-1}_{i =0}m_{f+g,i}\left(t_{i+1}-t_{i}\right) \\
	\leq & \sum^{n-1}_{i = 0}\left(M_{f,i}+M_{g,i}\right)\left(t_{i+1}-t_{i}\right) - \sum^{n-1}_{i = 0}\left(m_{f,i}+m_{g,i}\right)\left(t_{i+1}-t_{i}\right)\\
	= & S\left(f,P\right)-I\left(f,P\right)+S\left(g,P\right)-I\left(g,P\right) < \frac{\epsilon }{2} + \frac{\epsilon }{2} = \epsilon  .
\end{split}
\]
Así, hemos demostrado que la integral existe. Ahora supongamos que (en el caso contrario se procede de forma análoga)
\[
\begin{split}
	0 \leq \int^{b}_{a} f +g - \left(\int^{b}_{a} f +\int^{b}_{a} g \right) \leq & S\left(f+g,P\right)-\left(I\left(f,P\right)+I\left(g,P\right)\right) \\ \leq & S\left(f,P\right)+S\left(g,P\right)-I\left(f,P\right)-I\left(g,P\right) < \frac{\epsilon }{2} + \frac{\epsilon }{2} = \epsilon .
\end{split}
\]
\item[(b)] Si $\displaystyle \lambda = 0 $ es trivial. Consideremos que $\displaystyle \lambda > 0 $, pues el caso $\displaystyle \lambda < 0 $ se hace de forma análoga. Tenemos que $\displaystyle \sup\left(\Imagen\left(\lambda f\right)\right) = \lambda \sup \left(\Imagen\left(f\right)\right) $ y $\displaystyle \inf\left(\Imagen\left(\lambda f\right)\right) = \lambda \inf\left(\Imagen\left(f\right)\right) $. Dado que $\displaystyle f $ es integrable, si $\displaystyle \epsilon > 0 $, tenemos que existe $\displaystyle P \in P\left(\left[a,b\right] \right) $ tal que 
	\[ S\left(f,P\right) - I\left(f,P\right) < \frac{\epsilon }{2\lambda }.\]
	Así, tenemos que 
	\[
	\begin{split}
		S\left(\lambda f,P\right) -I\left(\lambda f, P\right) = & \sum^{n-1}_{i=0}M_{\lambda f,i}\left(t_{i+1}-t_{i}\right) - \sum^{n-1}_{i = 0}m_{\lambda f,i}\left(t_{i+1}-t_{i}\right) \\
		= & \lambda\left[\sum^{n-1}_{i=0}M_{f,i}\left(t_{i+1}-t_{i}\right)-\sum^{n-1}_{i=0}m_{f,i}\left(t_{i+1}-t_{i}\right)\right] \\
		< & \lambda \cdot \frac{\epsilon }{\lambda} = \epsilon.
	\end{split}
	\]
	En el caso de $\displaystyle \lambda < 0 $, ha de tenerse en cuenta que $\displaystyle \sup\left(\lambda A\right) = \lambda \inf\left(A\right) $ y $\displaystyle \inf\left(\lambda A\right) = \lambda \sup\left(A\right)$. Así, tenemos que $\displaystyle \forall P \in P\left(\left[a,b\right] \right) $,
	\[ S\left(\lambda f,P\right) = \lambda I\left(f,P\right) \quad \text{y} \quad I\left(\lambda f,P\right) = \lambda S\left(f,P\right) .\]
\end{description}
\end{proof}
\begin{observation}
	\normalfont Consideremos el conjunto $\displaystyle \mathcal{C}\left[a,b\right]  $, que es conjunto de las funciones continuas en un intervalo. Tenemos que es un espacio vectorial. Definimos la aplicación
	\[
	\begin{split}
	T : \mathcal{C}\left[a,b\right]  \to & \R \\
	f \to & \int^{b}_{a} f .
	\end{split}
	\]
Tenemos que $\displaystyle T $ es una aplicación lineal, en concreto es una forma lineal. 
\end{observation}
\begin{ftheorem}[]
	\normalfont Sean $\displaystyle f,g : \left[a,b\right] \to \R $ integrables.
	\begin{description}
		\item[(a)] Si $\displaystyle f \leq g $, $\displaystyle \forall x \in \left[a,b\right]$, entonces
			\[ \int^{b}_{a} f \leq \int^{b}_{a} g .\]
		\item[(b)] Si $\displaystyle m \leq f \leq M $ con $\displaystyle m,M \in \R $, entonces
			\[\left(b-a\right)m \leq \int^{b}_{a} f \leq M\left(b-a\right) .\]
		\item[(c)] La función $\displaystyle \left|f\right| $ es integrable y además,
			\[ \left|\int^{b}_{a} f \right| \leq \int^{b}_{a} \left|f\right| .\]
	\end{description}
\end{ftheorem}
\begin{proof}
\begin{description}
	\item[(a)] Si $\displaystyle f \leq g $, tenemos que $\displaystyle M_{f,i} \leq M_{g,i} $ y $\displaystyle m_{f,i} \leq m_{g,i} $. Así $\displaystyle \forall P \in P\left(\left[a,b\right] \right) $ se tiene que,
	\[ S\left(f,P\right) \leq S\left(g,P\right) \quad \text{y} \quad I\left(f,P\right) \leq I\left(g,P\right) .\]
	Por tanto, 
	\[ \int^{b}_{a} f = \overline{\int^{b}_{a}} f \leq \overline{\int^{b}_{a}} g = \int^{b}_{a} g  .\]
\item[(b)] Tomamos $\displaystyle g = M $ y $\displaystyle h = m $. Dado que se tiene que $\displaystyle h \leq f \leq g $, por el apartado \textbf{(a)} se tiene que 
	\[ \int^{b}_{a} h \leq \int^{b}_{a} f \leq \int^{b}_{a} g  \iff \left(b-a\right)m \leq \int^{b}_{a} f \leq M\left(b-a\right).\]
\item[(c)] Primero demostramos que $\displaystyle \left|f\right| $ es integrable. Definimos
	\[f^{+} = 
	\begin{cases}
	f\left(x\right), \; f\left(x\right) \geq 0 \\
	0, \; f\left(x\right) < 0
	\end{cases}
	, \quad f^{-} = 
\begin{cases}
0, \; f\left(x\right) \geq 0 \\
-f\left(x\right), \; f\left(x\right) < 0
\end{cases}
.\]
Se ve fácilmente que $\displaystyle f = f^{+} - f^{-} $ y $\displaystyle \left|f\right| = f^{+} + f^{-} $. Vamos a ver que $\displaystyle f^{+} $ y $\displaystyle f^{-} $ son integrables. Sin pérdida de generalidad, vamos a ver que $\displaystyle f^{+} $ es integrable. Sea $\displaystyle \epsilon > 0 $, existe $\displaystyle P \in P\left(\left[a,b\right] \right) $ tal que 
\[
\begin{split}
	S\left(f,P\right)-I\left(f,P\right) = & \sum^{n-1}_{i=0}\left(M_{i}-m_{i}\right)\left(t_{i+1}-t_{i}\right) \\
	= & \left[\sum^{}_{i \in \left\{ j \; : \; M_{j} > 0\right\}} M_{i}\left(t_{i+1}-t_{i}\right)-\sum^{}_{i \in \left\{ j \; : \; m_{j} > 0\right\} }m_{i}\left(t_{i+1}-t_{i}\right) \right] \\
	+ & \left[\sum^{}_{i \in \left\{ j \; : \; M_{j} \leq 0\right\}} M_{i}\left(t_{i+1}-t_{i}\right)-\sum^{}_{i \in \left\{ j \; : \; m_{j} \leq 0\right\} }m_{i}\left(t_{i+1}-t_{i}\right) \right] \\
	< & \epsilon .
\end{split}
\]
Tenemos que $\displaystyle \left\{ j \; : \; M_{j} \leq 0\right\} \subset \left\{ j \; : \; m_{j} \leq 0\right\}  $ y siempre se tiene que $\displaystyle M_{i}-m_{i} \geq 0 $. Además, tenemos que 
\[ M_{j}-m_{j} \geq M_{j}, \; M_{j} > 0, m_{j} \leq 0 .\]
Similarmente, tenemos que $\displaystyle \left\{ j \; : \; m_{j} \geq 0\right\} \subset \left\{ j \; : \; M_{j} \geq 0\right\}  $. Así, tenemos que
\[
\begin{split}
	S\left(f^{+},P\right)-I\left(f^{+}, P\right) = \left[\sum^{}_{i \in \left\{ j \; : \; M_{j} > 0\right\}} M_{i}\left(t_{i+1}-t_{i}\right)-\sum^{}_{i \in \left\{ j \; : \; m_{j} > 0\right\} }m_{i}\left(t_{i+1}-t_{i}\right) \right] < \epsilon .
\end{split}
\]
Así, hemos visto que $\displaystyle f^{+} $ es integrable, y $\displaystyle f^{-} $, de forma análoga también es integrable, por lo que $\displaystyle \left|f\right| $ es integrable. \\
	Una vez demostrado que $\displaystyle \left|f\right| $ es integrable, tenemos que 
	\[
	\begin{split}
	\left|\int^{b}_{a} f \right| = \left|\int^{b}_{a} f^{+}-f^{-}\right|= \left|\int^{b}_{a} f^{+} -\int^{b}_{a} f^{-}\right| \leq \int^{b}_{a} f^{+} + \int^{b}_{a} f^{-} = \int^{b}_{a} \left|f\right| .
	\end{split}
	\]
\end{description}
\end{proof}
\section{Teorema Fundamental del Cálculo}
\begin{fdefinition}[]
	\normalfont Dada $\displaystyle f :\left[a,b\right]  \to \R $ integrable, se define la función
	\[ F\left(x\right) = \int^{x}_{a} f\left(t\right) \; dt, \; x \in \left[a,b\right]  .\]
\end{fdefinition}
\begin{eg}
\normalfont Consideremos la función 
\[f\left(x\right) = 
\begin{cases}
	\frac{1}{5}, \; x \in \left[0,\frac{1}{2}\right] \\
	\frac{6}{5}, \; x \in \left(\frac{1}{2},1\right] \\
	\frac{3}{10}, \; x \in (1,2]
\end{cases}
.\]
Tenemos que $\displaystyle f $ es integrable y 
\[ F\left(x\right)= 
\begin{cases}
	\frac{1}{5}x, \; x \in \left[0,\frac{1}{2}\right] \\
	\frac{1}{10} + \left(x-\frac{1}{2}\right) \cdot \frac{6}{5}, \; x \in \left(\frac{1}{2},1\right] \\
	\frac{7}{10} + \left(x-1\right) \cdot \frac{3}{10}, \; x \in (1,2]
\end{cases}
.\]
\end{eg}
\begin{ftheorem}[]
	\normalfont Si $\displaystyle f : \left[a,b\right]  \to \R $ es integrable, tenemos que $\displaystyle F\left(x\right)= \int^{x}_{a} f\left(t\right) \; dt $ es continua en $\displaystyle \left[a,b\right]  $.
\end{ftheorem}
\begin{proof}
	Sea $\displaystyle c \in \left[a,b\right]  $. Tenemos que
	\[
	\begin{split}
		F\left(c + h\right)- F\left(c\right) = & \int^{c + h}_{a} f\left(t\right) \; dt - \int^{c}_{a} f\left(t\right) \; dt = \int^{c + h}_{c} f\left(t\right) \; dt .
	\end{split}
	\]
	Por ser $\displaystyle f $ integrable en $\displaystyle \left[a,b\right]  $ también está acotada, es decir, existe $\displaystyle M > 0 $ tal que $\displaystyle \forall t \in [a,b] $, $\displaystyle \left|f\left(t\right)\right| \leq M $. Así, tenemos que
	\[
	\begin{split}
	\left|F\left(c+h\right)-F\left(c\right)\right| = \left|\int^{c + h}_{c} f\left(t\right) \; dt\right| \leq \int^{c + h}_{c} \left|f\left(t\right)\right| \; dt \leq \int^{c + h}_{c} M = \left|h\right| M \to 0 .
	\end{split}
	\]
	Por tanto, tenemos que $\displaystyle \lim_{h \to 0}F\left(c+h\right)-F\left(c\right) = 0 $, por lo que $\displaystyle \lim_{x \to c}F\left(x\right)= F\left(c\right) $.
\end{proof}
\begin{ftheorem}[Teorema Fundamental del Cálculo]
	\normalfont Sea $\displaystyle f : [a,b] \to \R$ continua. La función $\displaystyle F\left(x\right) = \int^{x}_{a} f\left(t\right) \; dt $ es derivable en $\displaystyle c \in \left[a,b\right]  $ y $\displaystyle F'\left(c\right) = f\left(c\right) $.  
\end{ftheorem}
\begin{proof}
Sin pérdida de generalidad, supongamos que $\displaystyle h > 0 $. Como hemos visto en la demostración anterior,
\[F\left(c + h\right)-F\left(c\right) = \int^{c + h}_{c} f\left(t\right) \; dt .\]
Sea $\displaystyle m_{h} = \inf \left\{ f\left(t\right) \; : \; t  \in \left[c, c + h\right]\right\}  $ y $\displaystyle M_{h} = \sup \left\{ f\left(t\right) \; : \; t \in \left[c, c+h\right]\right\}  $. Entonces, tenemos que
\[hm_{h} \leq \int^{c + h}_{c } f\left(t\right) \; dt \leq M_{h}h .\]
Por lo que tenemos que
\[m_{h} \leq \frac{1}{h}\int^{c+h}_{c} f\left(t\right) \; dt = \frac{F\left(c+h\right)-F\left(c\right)}{h} \leq M_{h} .\]
Dado que $\displaystyle f $ es continua, tenemos que si $\displaystyle h\to0 $, $\displaystyle m_{h}, M_{h} \to f\left(c\right) $, por lo que $\displaystyle F'\left(c\right) = f\left(c\right) $.
\end{proof}
\begin{observation}
	\normalfont Una demostración alternativa es la siguiente. Si $\displaystyle f $ es continua en $\displaystyle c \in \left[a,b\right]  $ entonces si $\displaystyle \epsilon > 0 $ existe $\displaystyle \delta > 0 $ tal que si $\displaystyle \left|h\right| < \delta  $ se tiene que $\displaystyle \left|f\left(c + h\right)-f\left(c\right)\right|<\epsilon  $. Así,
\[
\begin{split}
	\left|\frac{F\left(c+h\right)-F\left(c\right)}{h}-f\left(c\right)\right| = & \left|\frac{1}{h}\int^{c+h}_{c} f\left(t\right) \; dt - \frac{f\left(c\right)}{h}\int^{c+h}_{c}  \; dt\right| = \frac{1}{ \left|h\right|} \left|\int^{c+h}_{c} f\left(t\right)-f\left(c\right) \; dt\right|\\
	\leq & \frac{1}{ \left|h\right|}\int^{c+h}_{c} \left|f\left(t\right)-f\left(c\right)\right| \; dt < \frac{1}{ \left|h\right|} \epsilon \left|h\right| = \epsilon.
\end{split}
\]
	
\end{observation}
\begin{fcolorary}[Regla de Barrow]
	\normalfont Sea $\displaystyle f : \left[a,b\right] \to \R $ continua. Si $\displaystyle g : \left[a,b\right] \to \R $ verifica que $\displaystyle g'\left(x\right) = f\left(x\right) $, $\displaystyle \forall x \in \left[a,b\right]  $, entonces
	\[ \int^{b}_{a} f\left(t\right) \; dt = g\left(b\right)-g\left(a\right) .\]
\end{fcolorary}
\begin{proof}
	Sea $\displaystyle F\left(x\right) = \int^{x}_{a} f\left(t\right) \; dt $. Por el teorema fundamental del cálculo, tenemos que $\displaystyle F'\left(x\right) = f\left(x\right) = g'\left(x\right) $. Así, tenemos que $\displaystyle F $ y $\displaystyle g $ tienen la misma derivada. Podemos recordar que, por el teorema del valor medio, $\displaystyle F\left(x\right) = g\left(x\right) + K $, $\displaystyle \forall x \in [a,b] $. 
	En particular,
	\[ F\left(a\right) =\int^{a}_{a} f\left(t\right) \; dt = 0 \Rightarrow g\left(a\right) + K = 0.\]
	Por tanto, tenemos que $\displaystyle F\left(x\right) = g\left(x\right)-g\left(a\right) $ y, por tanto,
	\[F\left(b\right) = \int^{b}_{a} f\left(t\right) \; dt = g\left(b\right)-g\left(a\right) .\]
\end{proof}
\begin{eg}
\normalfont Consideremos $\displaystyle \int^{a}_{0} rx \; dx $. Antes lo tuvimos que resolver mediante el límite $\displaystyle \lim_{n \to \infty}\frac{a}{n}\sum^{n-1}_{i=0}r\frac{ia}{n} $. Esto se complica si consideramos, por ejemplo, $\displaystyle f\left(x\right) = x^{5} $. Con la regla de Barrow, los cálculos se simplifican considerablemente:
\[ \int^{1}_{0} x^{5} \; dx = \left[\frac{x^{6}}{6}\right] ^{1}_{0} = \frac{1}{6} .\]
\end{eg}
\begin{eg}
\normalfont Sea $\displaystyle F\left(x\right) = \int^{\ln\left(x+1\right)}_{x^{2}} \sqrt{1 + t ^{2}} \; dt $. Vamos a calcular la derivada. Tenemos que $\displaystyle f\left(t\right) = \sqrt{1 + t ^{2}} $ es continua en $\displaystyle \R $, por lo que es integrable. Podemos ver que 
\[ F\left(x\right) = \int^{\ln\left(x+1\right)}_{x^{2}} \sqrt{1 + t ^{2}} \; dt = - \int^{x^{2}}_{0} \sqrt{1 + t ^{2}} \; dt + \int^{\ln\left(x+1\right)}_{0} \sqrt{1 + t ^{2}} \; dt .\]
Definimos $\displaystyle G\left(x\right) = \int^{x}_{0} \sqrt{1 + t ^{2}} \; dt $, por lo que $\displaystyle F\left(x\right) = G\left(\ln\left(x+1\right)\right)-G\left(x^{2}\right) $. Así, tenemos que
\[ F'\left(x\right) = -\sqrt{1 + \left(x^{2}\right)^{2}} \cdot2x + \sqrt{1 + \left(\ln\left(x+1\right)\right)^{2}}\frac{1}{x+1}.\]
\end{eg}
\begin{ftheorem}[Regla de Barrow II]
	\normalfont Si $\displaystyle f : \left[a,b\right] \to \R $ es integrable y $\displaystyle g'\left(x\right) = f\left(x\right) $, $\displaystyle \forall x \in \left[a,b\right]  $, entonces 
	\[ \int^{b}_{a} f\left(t\right) \; dt = g\left(b\right)-g\left(a\right) .\]
\end{ftheorem}
\begin{proof}
	Sea $\displaystyle P \in P\left(\left[a,b\right] \right) $. Por el teorema del valor medio, tenemos que existe $\displaystyle \xi_{i} \in \left(t_{i}, t_{i+1}\right) $ tal que 
	\[ m_{i}\left(t_{i+1}-t_{i}\right) \leq g\left(t_{i+1}\right)-g\left(t_{i}\right) = g'\left(\xi_{i}\right)\left(t_{i+1}-t_{i}\right) = f\left(\xi_{i}\right)\left(t_{i+1}-t_{i}\right) \leq M_{i}\left(t_{i+1}-t_{i}\right) .\]
	Así, tenemos que
	\[
	\begin{split}
	I\left(f,P\right) = \sum^{n-1}_{i = 0}m_{i}\left(t_{i+1}-t_{i}\right) \leq \sum^{n-1}_{i=0}g\left(t_{i+1}\right)-g\left(t_{i}\right) = g\left(b\right)-g\left(a\right) \leq \sum^{n-1}_{i=0}M_{i}\left(t_{i+1}-t_{i}\right) = S\left(f,P\right) .
	\end{split}
	\]
	Así, tenemos que $\displaystyle \forall P \in P\left([a,b]\right) $, $\displaystyle I\left(f,P\right) \leq g\left(b\right)-g\left(a\right)\leq S\left(f,P\right) $. También tenemos que $\displaystyle \forall P \in P\left(\left[a,b\right] \right) $ se cumple que
	\[ I\left(f,P\right) \leq \int^{b}_{a} f\left(t\right) \; dt \leq S\left(f,P\right) .\]
	Así, si $\displaystyle \int^{b}_{a} f\left(t\right) \; dt \geq g\left(b\right)-g\left(a\right) $ (el otro caso es análogo) y $\displaystyle \epsilon > 0 $, existe $\displaystyle P \in P\left(\left[a,b\right] \right) $ tal que,
	\[ 0 \leq \int^{b}_{a} f\left(t\right) \; dt - \left(g\left(b\right)-g\left(a\right)\right) \leq S\left(f,P\right)-I\left(f,P\right) < \epsilon .\]
Por tanto, ha de ser que $\displaystyle \int^{b}_{a} f\left(t\right) \; dt = g\left(b\right)-g\left(a\right) $.	
\end{proof}
\begin{observation}
\normalfont Otra forma de demostrar la igualdad de este último teorema es deducir que, dado que 
\[ I\left(f,P\right) \leq g\left(b\right)-g\left(a\right) \leq S\left(f,P\right) .\]
Por tanto, tenemos que $\displaystyle g\left(b\right)-g\left(a\right) \leq \overline{\int^{b}_{a}} f\left(t\right) \; dt $ y $\displaystyle \underline{\int^{b}_{a}} f\left(t\right) \; dt \leq g\left(b\right)-g\left(a\right) $. Así, dado que $\displaystyle f $ es integrable se tiene que
\[\int^{b}_{a} f\left(t\right) \; dt = \underline{\int^{b}_{a}} f\left(t\right) \; dt \leq g\left(b\right)-g\left(a\right) \leq \overline{\int^{b}_{a}} f\left(t\right) \; dt = \int^{b}_{a} f\left(t\right) \; dt .\]
Por tanto, $\displaystyle \int^{b}_{a} f\left(t\right) \; dt= g\left(b\right)-g\left(a\right) $ 
\end{observation}
\begin{eg}
\normalfont Consideremos la integral $\displaystyle \int^{\pi }_{0} \sin x \; dx $. Sabemos que $\displaystyle -\cos x = \left(\sin x\right)' $. Por tanto, tenemos que
\[\int^{\pi }_{0} \sin x \; dx = -\cos x |^{\pi }_{0} = 2.\]
\end{eg}
\section{Función logarítmica y exponencial}
\begin{eg}
\normalfont Sea $\displaystyle f:\left(0,\infty\right)\to \R $ con $\displaystyle f\left(t\right)= \frac{1}{t} $. 
\end{eg}
\begin{fdefinition}[Logaritmo neperiano]
\normalfont Para $\displaystyle x > 0 $ se define \textbf{logaritmo neperiano} de $\displaystyle x $ por 
\[ \ln x = \int^{x}_{1} \frac{1}{t} \; dt .\]
\end{fdefinition}
\begin{observation}
\normalfont Tenemos que por el teorema fundamental del cálculo, $\displaystyle \left(\ln x\right)'= \frac{1}{x}. $ 
Similarmente, tenemos que $\displaystyle \ln 1 = \int^{1}_{1} \frac{1}{t} \; dt = 0 $. Si $\displaystyle x > 1 $, tenemos que 
\[ \ln x = \int^{x}_{1} \frac{1}{t} \; dx > 0 .\]
Si $\displaystyle x < 1 $ se tiene que 
\[ \ln x = \int^{x}_{1} \frac{1}{t} \; dt = - \int^{1}_{x} \frac{1}{t} \; dt < 0 .\]
Dado que la derivada es siempre positiva, tenemos que es una función creciente y, como $\displaystyle \left(\ln x\right)'' = - \frac{1}{x^{2}} < 0 $, la función es cóncava. Vamos a calcular los límites. 
\[ \lim_{n \to \infty} \ln n = \lim_{n \to \infty}\int^{n}_{1} \frac{1}{t} \; dt \geq \lim_{n \to \infty} \sum^{n}_{k=2}\frac{1}{k} = \infty .\]
De este resultado y de que $\displaystyle f $ es creciente, se deduce que $\displaystyle \lim_{x \to \infty}\ln x = \infty $. Por otro lado, vamos a calcular $\displaystyle \lim_{x \to 0^{+}}\ln x $. 
\[ \lim_{n \to \infty}\int^{\frac{1}{n}}_{1} \frac{1}{t} \; dt = \lim_{n \to \infty}-\int^{1}_{\frac{1}{n}} \frac{1}{t} \; dt \leq \lim_{n \to \infty}-\sum^{n-1}_{k=1}\frac{1}{\frac{1}{k}}\left(\frac{1}{k}-\frac{1}{k+1}\right) = \lim_{n \to \infty}-\sum^{n-1}_{k=1}\frac{1}{k+1} = - \infty .\]
Dado que $\displaystyle f $ es creciente, tenemos que $\displaystyle \lim_{x \to 0^{+}}\ln x = -\infty $.
\end{observation}
\begin{ftheorem}[]
\normalfont 
\[ \forall x,y > 0, \;\ln xy = \ln x + \ln y.\]
\end{ftheorem}
\begin{proof}
Sea $\displaystyle y $ fijo. Definimos $\displaystyle g\left(x\right) = \ln yx $. Tenemos que $\displaystyle g'\left(x\right) = \frac{1}{yx} \cdot y = \frac{1}{x} $. Como consecuencia del teorema del valor medio, tenemos que $\displaystyle g\left(x\right) = \ln x + K $. Tenemos que $\displaystyle g\left(1\right) = \ln 1 + K $, por lo que $\displaystyle K = g\left(1\right) = \ln y$. 
Así, tenemos que $\displaystyle g\left(x\right) = \ln xy = \ln x + \ln y $.
\end{proof}
\begin{fprop}[]
\normalfont 
\begin{description}
\item[(a)] $\displaystyle \forall x > 0 $, $\displaystyle \forall n \in \N $, $\displaystyle \ln x^{n} = n \ln x $.
\item[(b)] $\displaystyle \forall x > 0 $, $\displaystyle \ln \frac{1}{x} = - \ln x $.
\item[(c)] $\displaystyle \forall x,y > 0 $, $\displaystyle \ln \frac{x}{y} = \ln x - \ln y $.
\end{description}
\end{fprop}
\begin{proof}
\begin{description}
\item[(a)] Tenemos que si $\displaystyle n \in \N $, 
	\[ \ln x^{n} = \ln\left(x + \cdots + x\right) = n \ln x .\]
\item[(b)] 
	\[ 0 = \ln \frac{x}{x} = \ln x + \ln\frac{1}{x} \iff \ln \frac{1}{x} = - \ln x.\]
\item[(c)] 
	\[ \ln \frac{x}{y} = \ln x + \ln \frac{1}{y} = \ln x - \ln y .\]
\end{description}
\end{proof}

\begin{observation}
\normalfont Tenemos que $\displaystyle \ln x $ es monótona creciente y se tiene que $\displaystyle \lim_{x \to \infty}\ln x = \infty $ y $\displaystyle \lim_{x \to 0^{+}} \ln x =- \infty$. Así, tenemos que $\displaystyle \forall \lambda \in \R $, $\displaystyle \exists x_{0} $ tal que $\displaystyle \ln x_{0} = \lambda$.
Como consecuencia del valor medio, dado que $\displaystyle \left(\ln x\right)' > 0 $, se tiene que $\displaystyle \ln x  $ es inyectiva. Por tanto, cabe definir su inversa.
\end{observation}
\begin{fdefinition}[Función exponencial]
\normalfont Definimos la función
\[
\begin{split}
	\exp : \R \to & \left(0, \infty\right) \\
	x \to & \exp\left(x\right) = \ln^{-1}x.
\end{split}
\]
\end{fdefinition}
\begin{observation}
\normalfont Es decir, $\displaystyle \exp x = y \iff \ln y = x $.
\end{observation}
Por lo calculado anteriormente, tenemos que $\displaystyle \lim_{x \to -\infty}\exp x = 0 $ y $\displaystyle \lim_{x \to \infty}\exp x = \infty $. Además tenemos que
\[ \left(\exp x\right) ' = \frac{1}{\ln' \exp x}= \exp x > 0 .\]
Por tanto, tenemos que es creciente y convexa. 
\begin{ftheorem}[]
\normalfont 
\[ \forall x,y \in \R, \; \exp\left(x+y\right) = \exp x \cdot \exp y .\]
\end{ftheorem}
\begin{proof}
Sean $\displaystyle x' = \exp x $ e $\displaystyle y' = \exp y $. Tenemos que $\displaystyle \ln x' = x $ y $\displaystyle \ln y' = y $. Así, tenemos que
\[ \exp\left(x+y\right) = \exp\left(\ln x' + \ln y'\right) = \exp\left(\ln x'y'\right) = x'y' = \exp x \cdot \exp y.\]
\end{proof}
\begin{notation}
\normalfont Se define $\displaystyle e = \exp 1 $.
\end{notation}
\begin{observation}
\normalfont Por tanto, tenemos que $\displaystyle \exp n = \exp\left(1 + \cdots + 1\right) = e^{n} $.
\end{observation}
\begin{notation}
\normalfont Se define $\displaystyle \exp x = e^{x} $.
\end{notation}
\begin{ftheorem}[]
\normalfont 
\[\lim_{x \to \infty}x\ln\left(1+\frac{1}{x}\right) = 1 .\]
\end{ftheorem}
\begin{proof}
\[\lim_{x \to \infty}x\ln\left(1+\frac{1}{x}\right) = \lim_{x \to \infty}\frac{\ln\left(1+\frac{1}{x}\right)}{\frac{1}{x}} = \lim_{x \to \infty} \frac{\frac{1}{1+\frac{1}{x}} \cdot \left(-\frac{1}{x^{2}}\right)}{-\frac{1}{x^{2}}} = 1 .\]
\end{proof}
\begin{observation}
\normalfont Por tanto, se tiene que $\displaystyle \lim_{n \to \infty}n\ln\left(1 + \frac{1}{n}\right) = 1$. Dado que el logaritmo es continuo y monótono, se tiene que \footnote{Sabemos que este límite converge a $\displaystyle l \in \R $ porque es monótono y está acotado superiormente. De aquí se deduce que $\displaystyle \lim_{n \to \infty}\ln\left(1 + \frac{1}{n}\right)^{n} = \ln l = 1 $, y por ser el logaritmo inyectivo, $\displaystyle \lim_{n \to \infty}\left(1+\frac{1}{n}\right)^{n} = e $.} 
\[ \lim_{n \to \infty}\left(1 + \frac{1}{n}\right)^{n} = \exp 1 = e .\]
\end{observation}
\begin{fdefinition}[]
\normalfont 
\begin{description}
\item[(a)] $\displaystyle \forall a > 0 $ y $\displaystyle \forall x \in \R $, se define $\displaystyle a^{x} = e^{x\ln a}  $.
\item[(b)] Si $\displaystyle a > 0 $ y $\displaystyle a \neq 1 $, se denomina la inversa de $\displaystyle a^{y} $ al $\displaystyle \log_{a}x = \left(a^{y}\right)^{-1} $.
\end{description}
\end{fdefinition}
\begin{fprop}[]
\normalfont 
\begin{description}
\item[(a)] $\displaystyle \forall a,b,c \in \R $, $\displaystyle \left(a^{b}\right)^{c} = a^{bc} $.
\item[(b)] $\displaystyle \forall a \in \R $, $\displaystyle a^{1} = a $.
\item[(c)] $\displaystyle a^{x + y}= a^{x} \cdot a^{y} $.
\item[(d)] $\displaystyle \log_{a} y= \frac{\log_{x}y}{\log_{a}y} $.
\item[(e)] $\displaystyle \left(\log_{a}x\right)' = \frac{1}{x\ln a} $.
\end{description}
\end{fprop}
\begin{proof}
\begin{description}
\item[(a)] 
	\[ \left(a^{b}\right)^{c} = e^{c\ln a^{b}} = e^{c\ln e^{b\ln a}} = e^{cb\ln a}= a^{bc} .\]
\item[(b)] 
	\[ a^{1} = e^{\ln 1} = e^{0} = 1 .\]
\item[(c)] 
	\[a^{x + y} = e^{\left(x+y\right)\ln a} = e^{x\ln a + y\ln a} = e^{x\ln a} \cdot e^{y \ln a} = a^{x}a^{y} .\]
\item[(d)] 
	\[ a^{\log_{a}x} = e^{\log_{a}x \cdot \ln a}= e^{\ln x}= x .\]
\end{description}
\end{proof}
\begin{observation}
\normalfont A partir de estas funciones se definen las funciones hiperbólicas y se pueden estudiar sus propiedades.
\end{observation}
\begin{fprop}[]
\normalfont 
\begin{description}
\item[(a)] Sea una función tal que $\displaystyle f'\left(x\right) = f\left(x\right) $, entonces existe $\displaystyle K \in \R $ tal que $\displaystyle f\left(x\right) = Ke^{x} $.
\item[(b)] $\displaystyle \lim_{x \to \infty}\frac{e^{x}}{x^{n}} = \infty $, $\displaystyle \forall n \in \N $.
\end{description}
\end{fprop}
\begin{proof}
\begin{description}
\item[(a)] Sea $\displaystyle f\left(x\right) $ tal que $\displaystyle f'\left(x\right) = f\left(x\right) $. Sea $\displaystyle g\left(x\right) = \frac{f\left(x\right)}{e^{x}} $. Tenemos que
	\[ g'\left(x\right) = \frac{f'\left(x\right)e^{x}-f\left(x\right)e^{x}}{\left(e^{x}\right)^{2}} = 0 .\]
	Por tanto, tenemos que $\displaystyle g\left(x\right) = K $. Por tanto, se obtiene que $\displaystyle f\left(x\right) = Ke^{x} $.
\item[(b)] Aplicamos L'Hôpital $\displaystyle n $ veces,
	\[ \lim_{x \to \infty}\frac{e^{x}}{x^{n}} = \lim_{x \to \infty}\frac{e^{x}}{nx^{n-1}} = \cdots = \lim_{x \to \infty}\frac{e^{x}}{n!} = \infty .\]
\end{description}
\end{proof}
\section{Criterio de integrabilidad de Lebesgue}
\begin{fdefinition}[Contenido cero]
	\normalfont Se dice que $\displaystyle A \subset \R $ tiene \textbf{contenido cero} si $\displaystyle \forall \epsilon > 0 $, existe una sucesión de intervalos abiertos \footnote{Se pueden repetir intervalos.} $\displaystyle \left\{ \left(a_{n}, b_{n}\right) \; : \; n \in \N\right\}  $ tal que 
	\[  A \subset \bigcup_{n \in \N}\left(a_{n}, b_{n}\right)  \;\; \text{y} \;\;  \sum^{\infty}_{n=1}\left(b_{n}-a_{n}\right) \leq \epsilon.\]
\end{fdefinition}
\begin{eg}
	\normalfont Sea $\displaystyle A = \left\{ a\right\} \subset \R $. Sea $\displaystyle \epsilon > 0 $, tenemos que $\displaystyle A \subset \left(a - \frac{\epsilon }{2}, a + \frac{\epsilon }{2}\right) $. Así, $\displaystyle a + \frac{\epsilon }{2}-\left(a-\frac{\epsilon }{2}\right) = \epsilon $. Tenemos que este conjunto tiene contenido cero \footnote{En este caso, hemos cogido la sucesión de intervalos que contiene un intervalo abierto $\displaystyle \left(a - \frac{\epsilon }{2}, a + \frac{\epsilon }{2}\right) $ y el resto son el intervalo $\displaystyle \left(a,a\right) = \emptyset $.} . 
\end{eg}
\begin{eg}
	\normalfont Sea $\displaystyle \emptyset \neq A \subset \R $ un conjunto finito de números reales. Vamos a ver que tiene contenido cero. Sea $\displaystyle \epsilon > 0 $. Sea $\displaystyle A = \left\{ x_{1}, \ldots, x_{n}\right\}  $. Cogemos la familia de intervalos,
	\[ \left\{ \left(x_{i}-\frac{\epsilon }{2n}, x_{i}+\frac{\epsilon }{2n}\right) \; : \; i = 1, \ldots, n\right\}  .\]
	Así, tenemos que 
\[\sum^{n}_{i=1} x_{i}+\frac{\epsilon }{2n} - \left(x_{i} - \frac{\epsilon }{2n}\right) = \sum^{n}_{i=1}\frac{\epsilon }{n} = \epsilon.\]
\end{eg}
\begin{eg}
	\normalfont Sea $\displaystyle A = \left\{ r_{n}\right\}_{n\in\N} \subset \R $. Sea $\displaystyle \epsilon > 0 $, cogemos  
	\[ \left(a_{n},b_{n}\right) = \left(r_{n}-\frac{\epsilon }{2^{n+1}}, r_{n} + \frac{\epsilon }{2^{n+1}}\right) .\]
	Así, tenemos que $\displaystyle A \subset \bigcup_{n \in \N}\left(a_{n}, b_{n}\right) $. Así, tenemos que 
	\[\sum^{\infty}_{n=1}\left[r_{n} + \frac{\epsilon }{2^{n+1}} - \left(r_{n}-\frac{\epsilon }{2^{n+1}}\right)\right] = \epsilon \sum^{\infty}_{n=1}\frac{1}{2^{n}} = \epsilon .\]
Así, $\displaystyle A $ tiene contenido cero.	
\end{eg}
\begin{observation}
\normalfont De estos ejemplos se deduce que cualquier conjunto finito o numerable de $\displaystyle \R $ tiene contenido cero.
\end{observation}
\begin{ftheorem}[Teorema de integrabilidad de Lebesgue]
	\normalfont Sea $\displaystyle f : \left[a,b\right] \to \R $. Tenemos que $\displaystyle f $ integrable en $\displaystyle \left[a,b\right]  $ si y solo si el conjunto de discontinuidades de $\displaystyle f $ tiene contenido cero.
\end{ftheorem}
\begin{fprop}[]
	\normalfont Sea $\displaystyle g : \left[a,b\right] \to \R $ monótona. Entonces existe $\displaystyle \int^{b}_{a} g\left(x\right) \; dx $.
\end{fprop}
\begin{proof}
Dado que el conjunto de discontinuidades de una función monótona es a lo más numerable, tenemos quetiene contenido cero, por lo que la función es integrable.
\end{proof}
\begin{ftheorem}[]
	\normalfont Sea $\displaystyle g : \left[a,b\right] \to \R $ y $\displaystyle f : \left[a,b\right] \to \R $ continua con $\displaystyle g\left(\left[a,b\right] \right)\subset \dom\left(f\right) $. Tenemos que $\displaystyle f \circ g $ es integrable en $\displaystyle \left[a,b\right]  $.
\end{ftheorem}
\begin{proof}
Dado que $\displaystyle g $ es integrable, tenemos que el conjunto de sus discontinuidades tiene contenido cero. Como $\displaystyle f $ es continua, $\displaystyle f \circ g $ es continua donde $\displaystyle g $ es continua. Así, el conjunto de discontinuidades de $\displaystyle g $ y de $\displaystyle f\circ g $ es el mismo, por lo que el conjunto de discontinuidades de $\displaystyle f \circ g $ tiene contenido cero y, por el teorema anterior se tiene que $\displaystyle f \circ g $ es integrable.
\end{proof}
\begin{fcolorary}[]
	\normalfont Si $\displaystyle f $ es integrable en $\displaystyle \left[a,b\right]  $, $\displaystyle f^{2} $ también lo es.
\end{fcolorary}
\begin{proof}
Si $\displaystyle h\left(x\right) = x^{2} $, se tiene que $\displaystyle f^{2} = h\circ f $. Dado que $\displaystyle h $ es continua y $\displaystyle f $ es integrable, por la proposición anterior se tiene que $\displaystyle f^{2} $ es integrable.
\end{proof}
\begin{fcolorary}[]
\normalfont Si $\displaystyle f $ y $\displaystyle g $ son integrables, tenemos que $\displaystyle fg $ es integrable.
\end{fcolorary}
\begin{proof}
Si $\displaystyle f $ y $\displaystyle g $ son integrables, tenemos que $\displaystyle f + g $ es integrable. Por el corolario anterior tenemos que $\displaystyle \left(f+g\right)^{2} $ es integrable. Por tanto, tenemos que 
\[ fg = \frac{\left(f+g\right)^{2}-f^{2}-g^{2}}{2} .\]
Por lo que $\displaystyle fg $ es integrable.
\end{proof}
\begin{eg}
	\normalfont Consideremos una función $\displaystyle f : \left[0,1\right] \to \R $ con $\displaystyle f\left(0\right) = 0 $ y $\displaystyle f\left(1\right) = 1 $ tal que
	\[ f\left(x\right) = \frac{n^{2}}{2n+1}\left[\left(n+1\right)x-1\right] \left[\left(n+1\right)x + 1\right] , \; \frac{1}{n+1} \leq x < \frac{1}{n} .\]
Si calculamos
\[\lim_{x \to \frac{1}{n}}f\left(x\right) = 1 .\]
Así, en todos los $\displaystyle \frac{1}{n} \in \left[0,1\right]  $ tenemos que el límite tiende a 1. Además tenemos que 
\[f\left(\frac{1}{n+1}\right) = 0 .\]
Entonces, tenemos que la función no es continua en los puntos donde $\displaystyle x = \frac{1}{n} $. Además, tenemos que
\[ f\left(x\right) = \frac{n^{2}}{2n+1}\left[\left(n+1\right)^{2}x^{2}-1\right]  .\]
Es decir, se parece a parábolas que tienden a 1 en los puntos de la forma $\displaystyle \frac{1}{n} $. Si fuera integrable, tendríamos que
\[\int^{1}_{0} f\left(x\right) \; dx = \sum^{\infty}_{n=1}\int^{\frac{1}{n}}_{\frac{1}{n+1}} f\left(x\right) \; dx .\]
Vamos a demostrar que es integrable con el criterio de integrabilidad de Riemann. Sea $\displaystyle \epsilon > 0 $ cogemos $\displaystyle n_{0} \in \N $ tal que $\displaystyle \frac{1}{n_{0}} < \frac{\epsilon }{2} $. Entonces, tenemos que $\displaystyle f|_{\left[\frac{1}{n_{0}}, 1\right] } $ es integrable porque continene un número finito de discontinuidades. Entonces, existe $\displaystyle P' \in P\left(\left[\frac{1}{n_{0}},1\right] \right) $, tal que 
\[S\left(f|_{\left[\frac{1}{n_{0}}, 1\right] }, P'\right) - I\left(f|_{\left[\frac{1}{n_{0}}, 1\right] },P'\right) < \frac{\epsilon }{2}.\]
Ahora, consideremos la partición $\displaystyle P = \left\{ 0\right\} \cup P' $. Tenemos que
\[ S\left(f,P\right)-I\left(f,P\right) = \frac{1}{n_{0}} + S\left(f,P'\right) - \left(0 + I\left(f,P'\right)\right) < \frac{1}{n_{0}} + \frac{\epsilon }{2} < \frac{\epsilon }{2} + \frac{\epsilon }{2} = \epsilon .\]
\end{eg}
\begin{eg}
	\normalfont Consideremos $\displaystyle f : \left[a,b\right] \to \R $ integrable. Vamos a demostrar que $\displaystyle \int^{b}_{a} f\left(t\right) \; dt = \int^{b+c}_{a + c} f\left(t-c\right) \; dt $. Sea $\displaystyle g\left(t\right) = f\left(t-c\right) $, tenemos que $\displaystyle \dom\left(g\right) = \left[a+c, b + c\right]  $. Tenemos que $\displaystyle \forall \epsilon > 0 $, $\displaystyle \exists P \in P\left(\left[a,b\right] \right) $ tal que $\displaystyle S\left(f,P\right)-I\left(f,P\right) < \epsilon  $. Tenemos que $\displaystyle P = \left\{ t_{0}=a, t_{1}, \ldots, t_{n} =b\right\}  $. Ahora consideramos la partición
	\[ P' = \left\{ t_{0}', t'_{1}, \ldots, t'_{n}\right\}, \; t'_{i} = t_{i} +c, \; \forall i = 0, \ldots, n  .\]
	Así, tenemos que $\displaystyle S\left(f,P\right)=S\left(g,P'\right) $ y $\displaystyle I\left(f,P\right) = I\left(g,P'\right) $. Por tanto, 
	\[S\left(g,P'\right)-I\left(g,P'\right) < \epsilon  .\]
Así, tenemos que $\displaystyle f\left(t-c\right) $ es integrable. Ahora demostramos la igualdad:
\[I\left(g,P'\right) \leq \int^{b}_{a} f\left(t\right) \; dt \leq S\left(g,P'\right) .\]
Por tanto, $\displaystyle \int^{b}_{a} f\left(t\right) \; dt = \int^{b + c}_{a+c} f\left(t-c\right) \; dt $.
\end{eg}
\begin{eg}
\normalfont Consideremos $\displaystyle f\left(x\right) = \log_{e^{x}}\left(\sin x\right) $. Vamos a calcular su derivada. Tenemos que 
\[f\left(x\right) = \log_{e^{x}}\left(\sin x\right) = \frac{\ln \sin x}{\ln e^{x}} = \frac{\ln \sin x}{x} .\]
Ahora resulta fácil ver que 
\[f'\left(x\right) = \frac{x\frac{\cos x}{\sin x}- \ln \sin x}{x^{2}} = \frac{\cos x - \sin x \ln \sin x}{x^{2}\sin x} .\]
\end{eg}
\begin{eg}
\normalfont Calculemos el límite
\[ \lim_{x \to 0}\frac{e^{x}-1-x-\frac{x^{2}}{2}-\frac{x^{3}}{6}}{x^{3}} .\]
Aplicamos L'Hôpital:
\[
\begin{split}
	\lim_{x \to 0}\frac{e^{x}-1-x-\frac{x^{2}}{2}-\frac{x^{3}}{6}}{x^{3}} = & \lim_{x \to 0}\frac{e^{x}-1-x-\frac{x^{2}}{2}}{3x^{2}} = \lim_{x \to 0}\frac{e^{x}-1-x}{6x} = \lim_{x \to 0}\frac{e^{x}-1}{6} = 0.
\end{split}
\]
\end{eg}
\begin{eg}
\normalfont Consideremos la función 
\[f\left(x\right) = 
\begin{cases}
e^{-\frac{1}{x^{2}}}, \; x \neq 0\\
0, \; x = 0
\end{cases}
.\]
Tenemos que calcular $\displaystyle f' $, $\displaystyle f $ y $\displaystyle f^{\left(n\right)}\left(0\right) $ y estudiar su continuidad. Empezamos por esto último. El único problema está en $\displaystyle x = 0 $:
\[ \lim_{x \to 0}e^{-\frac{1}{x^{2}}} = 0.\]
Por tanto, $\displaystyle f $ es continua en $\displaystyle \R $. Para pintar la función puede resultar útil saber que $\displaystyle \lim_{x \to \pm \infty}e^{-\frac{1}{x^{2}}} = 1 $. Ahora derivamos,
\[ f'\left(x\right) = \frac{2}{x^{3}}e^{-\frac{1}{x^{2}}} = 
\begin{cases}
< 0, \; x < 0 \\
> 0, \; x > 0
\end{cases}
.\]
Calculamos la derivada en 0:
\[f'\left(0\right) = \lim_{h \to 0}\frac{f\left(h\right)-f\left(0\right)}{h} = \lim_{h \to 0}\frac{e^{-\frac{1}{h^{2}}}}{h} = \lim_{h \to 0}\frac{\frac{1}{h}}{e^{\frac{1}{h^{2}}}} .\]
Sea $\displaystyle y = \frac{1}{h} $, si $\displaystyle h \to 0 $, $\displaystyle y \to \infty $, así,
\[f'\left(0\right) = \lim_{y \to \infty}\frac{y}{e^{y^{2}}} = \lim_{y \to \infty}\frac{1}{2ye^{y^{2}}} = 0 .\]
Así, tenemos que 
\[f'\left(x\right) = 
\begin{cases}
\frac{2}{x^{3}}e^{-\frac{1}{x^{2}}}, \; x \neq 0\\
0, \; x = 0
\end{cases}
.\]
Ahora calculamos la segunda derivada.
\[f''\left(0\right) = \lim_{h \to 0}\frac{f'\left(h\right)-f'\left(0\right)}{h} = \lim_{h \to 0} \frac{\frac{2}{h^{3}}e^{-\frac{1}{h^{2}}}}{h} = \lim_{h \to 0}\frac{\frac{2}{h^{4}}}{e^{\frac{1}{h^{2}}}} = \lim_{y \to \infty}\frac{2y^{4}}{e^{y^{2}}}=0 .\]
Para $\displaystyle x \neq 0 $,
\[f'\left(x\right) = \left(-\frac{6}{x^{4}}+\frac{4}{x^{6}}\right)e^{-\frac{1}{x^{2}}} .\]
Supongamos que 
\[ f^{\left(n\right)}\left(x\right) = 
\begin{cases}
P_{n}\left(\frac{1}{x}\right)e^{-\frac{1}{x^{2}}}, \; x \neq 0\\
0, \; x = 0
\end{cases}
.\]
Ahora consideremos el caso $\displaystyle n + 1 $. Vamos a calcular $\displaystyle f^{\left(n+1\right)} $. Si $\displaystyle x \neq 0 $, 
\[
\begin{split}
	f\left(n+1\right)\left(x\right) = & P'_{n}\left(\frac{1}{x}\right)\left(-\frac{1}{x^{2}}\right)e^{-\frac{1}{x^{2}}} + P_{n}\left(\frac{1}{x}\right)\left(-\frac{2}{x^{3}}\right)e^{-\frac{1}{x^{2}}} \\
	= &  e^{-\frac{1}{x^{2}}}P'_{n}\left(\frac{1}{x}\right)\left(-\frac{1}{x^{2}}\right) + P_{n}\left(\frac{1}{x}\right)\left(-\frac{2}{x^{3}}\right) = e^{-\frac{1}{x^{2}}}P_{n+1}\left(\frac{1}{x}\right).
\end{split}
\]
Ahora si $\displaystyle x = 0 $, 
\[
\begin{split}
	f^{\left(n+1\right)}\left(0\right) = & \lim_{h \to 0}\frac{f^{\left(n\right)}\left(h\right)-f^{\left(n\right)}\left(0\right)}{h} = \lim_{h \to 0}\frac{P_{n}\left(\frac{1}{h}\right)e^{-\frac{1}{h^{2}}}}{h}  = \lim_{h \to 0}\frac{P_{n}\left(\frac{1}{h}\right)\frac{1}{h}}{e^{-\frac{1}{h^{2}}}}= \lim_{y \to \infty}\frac{P_{n}\left(y\right)y}{e^{y^{2}}} = 0.
\end{split}
\]
\end{eg}
