\chapter{Límites de funciones}

\begin{fdefinition}[Punto de acumulación]
\normalfont Sea $\displaystyle A \subset \R $. Se dice que $\displaystyle c $ es un \textbf{punto de acumulación} de $\displaystyle A $ si $\displaystyle \forall \epsilon > 0 $, $\displaystyle \exists x \in A $ con $\displaystyle x \neq c $ tal que $\displaystyle \left|x - c\right| < \epsilon  $.
\end{fdefinition}

\begin{eg}
\normalfont Sea $\displaystyle A = \left(0,1\right) $ y sea $\displaystyle 0 < c < 1 $. Tenemos que $\displaystyle c $ es un punto de acumulación de $\displaystyle A $. Similarmente, 0 y 1 también son puntos de acumulación de $\displaystyle A $.
\end{eg}

\begin{notation}
	\normalfont $\displaystyle A' = \left\{ c \in \R \; : \; c \; \text{punto de acumulación de } \; A\right\}  $. Así, si $\displaystyle A = \left(0,1\right) $, entonces $\displaystyle A' = \left[0,1\right]  $.
\end{notation}
\begin{eg}
	\normalfont $\displaystyle A = \left\{ 0,1\right\}  $. Tenemos que $\displaystyle A' = \emptyset $.
\end{eg}

\begin{ftheorem}[]
	\normalfont Sea $\displaystyle A\subset\R $. Entonces, $\displaystyle c \in A' $ si y solo si $\displaystyle \exists \left\{ x_{n}\right\} _{n\in\N}\subset A $ con $\displaystyle x_{n} \neq c $ tales que $\displaystyle \lim_{n \to \infty}x_{n} = c $.
\end{ftheorem}

\begin{proof}
\begin{description}
\item[(i)] Supongamos que $\displaystyle A' \neq \emptyset $. Si $\displaystyle c \in A' $, tomamos $\displaystyle \epsilon = \frac{1}{n} $, entonces $\displaystyle \exists x_{n} \in A, \; x_{n} \neq c$ tal que $\displaystyle \left|x_{n}-c\right| < \frac{1}{n} $. Así, $\displaystyle \left\{ x_{n}\right\} _{n\in\N}\subset A- \left\{ c\right\}  $ y $\displaystyle \left|x_{n}-c\right| < \frac{1}{n} \to 0 $, por lo que $\displaystyle x_{n} \to c $. 
\item[(ii)] Recíprocamente, si $\displaystyle \left\{ x_{n}\right\} _{n\in\N}\subset A- \left\{ c\right\}  $ tal que $\displaystyle x_{n} \to c $. Sea $\displaystyle \epsilon > 0 $ y $\displaystyle n \in \N $ tal que $\displaystyle \left|x_{n}-c\right| < \epsilon  $. Así, $\displaystyle x_{n} \in A - \left\{ c\right\}  $ y $\displaystyle \left|x_{n}-c\right| < \epsilon  $. Por lo que $\displaystyle c \in A' $.
\end{description}
\end{proof}

\begin{eg}
\normalfont 
\begin{description}
\item[(i)]  Sea $\displaystyle \left\{ x_{n}\right\} _{n\in\N}\subset\R $ tal que $\displaystyle x_{n} \to c $ con $\displaystyle x_{n} \neq c $ y sea $\displaystyle A = \left\{ x_{n}\right\} _{n\in\N} $. Entonces, $\displaystyle A' = \left\{ c\right\}  $. Por ejemplo, si $\displaystyle A = \left\{ \frac{1}{n} \; : \; n \in \N\right\}  $, entonces $\displaystyle A' = \left\{ 0\right\} $.
\item[(ii)] Sea $\displaystyle A = \Q $, tenemos que $\displaystyle A' = \R $, pues $\displaystyle \Q $ es denso en $\displaystyle \R $. Análogamente, si $\displaystyle A = \R/\Q $, tenemos que $\displaystyle A'=\R $.
\end{description}

\end{eg}

\begin{fdefinition}[Límite]
\normalfont Sea $\displaystyle f : A \subset \R \to \R $ y $\displaystyle c \in A' $. Se dice que $\displaystyle l $ es el límite de $\displaystyle f $ cuando $\displaystyle x $ se aproxima a $\displaystyle c $ \footnote{Por lo estudiado anteriormente, podemos calcular el límite sin que $\displaystyle c \in A $, es decir, sin que $\displaystyle f $ esté definido en $\displaystyle c $.} : 
\[\lim_{x \to c}f\left(x\right) = l ,\]
si $\displaystyle \forall \epsilon > 0, \; \exists \delta > 0 $ tal que si $\displaystyle x \in A - \left\{ c\right\}  $ y $\displaystyle 0< \left|x-c\right|<\delta $, entonces $\displaystyle \left|f\left(x\right)-l\right|<\epsilon $.
\end{fdefinition}

\begin{eg}
	\normalfont Sea $\displaystyle f: (0,1] \to \R $ tal que $\displaystyle f\left(x\right) = \frac{x^{2}+x}{x} $. Si $\displaystyle A = (0,1] $, tenemos que $\displaystyle 0 \in A' = \left[0,1\right]  $. Queremos ver que $\displaystyle \lim_{x \to 0}f\left(x\right) =1 $. Sea $\displaystyle \epsilon > 0 $, quiero encontrar $\displaystyle \delta > 0 $ tal que si $\displaystyle x \in (0,1] $y $\displaystyle 0 < x < \delta  $, entonces $\displaystyle \left|\frac{x^{2}+x }{x }-1\right| < \epsilon  $. Tenemos que 
	\[ \left|\frac{x^{2}+x }{x }-1\right| = \left|x +1 - 1\right|= x < \epsilon  .\]
Cogemos $\displaystyle \delta = \frac{\epsilon }{2} < \epsilon  $. También podemos tomar $\displaystyle \delta = \epsilon  $. Entonces si $\displaystyle 0 < x< \delta  $, tenemos que $\displaystyle \left|f\left(x\right)-1\right| < \epsilon  $.
\end{eg}

\begin{eg}
\normalfont $\displaystyle f : \R \to \R $ tal que $\displaystyle f\left(x\right) = x^{2} $. Tenemos que $\displaystyle 0 \in \R' = \R $. Vamos a ver que $\displaystyle \lim_{x \to 0}x^{2} =0 $. Vamos a ver que si $\displaystyle \epsilon > 0 $ existe $\displaystyle \delta > 0 $ tal que si $\displaystyle 0 < \left|x-0\right|<\delta  $, entonces $\displaystyle \left|x^{2}-0\right| < \epsilon  $.
\[ \left|x^{2}\right| < \epsilon \iff x^{2} < \epsilon  .\]
Podemos tomar $\displaystyle \delta = \sqrt{\epsilon } $, pues entonces
\[x < \sqrt{\epsilon } \Rightarrow x^{2} < \left(\sqrt{\epsilon }\right)^{2} = \epsilon  .\]
\end{eg}

\begin{ftheorem}[Caracterización del límite por convergencia de sucesiones]
\normalfont Sea $\displaystyle f: A \subset\R \to \R $, $\displaystyle c \in A' $ y $\displaystyle l \in \R $. Entonces, son equivalentes los siguientes enunciados:
\begin{description}
\item[(i)] $\displaystyle \lim_{x \to c}f\left(x\right) = l $.
\item[(ii)] $\displaystyle \forall \left\{ x_{n}\right\} _{n\in\N}\subset A- \left\{ c\right\}  $, $\displaystyle x_{n} \to c $ si $\displaystyle n \to \infty $, entonces $\displaystyle f\left(x_{n}\right) \to l $.
\end{description}
\end{ftheorem}
\begin{proof}
\begin{description}
	\item[(i) $\displaystyle \Rightarrow $ (ii)] Queremos ver que $\displaystyle \forall \epsilon > 0 $ existe un $\displaystyle n_{0} \in \N $ tal que si $\displaystyle n \geq n_{0} $, $\displaystyle \left|f\left(x_{n}\right)-l\right|<\epsilon  $. Sabemos que dado $\displaystyle \epsilon > 0 $, $\displaystyle \exists \delta > 0 $ tal que si $\displaystyle x \in A/ \left\{ c\right\}  $ y $\displaystyle 0 < \left|x - c\right| < \delta  $, entonces $\displaystyle \left|f\left(x\right)-l\right| < \epsilon  $. Como $\displaystyle x_{n} \to c $, dado $\displaystyle \delta > 0 $ existe $\displaystyle n_{0} \in \N $ tal que $\displaystyle \forall n \geq n_{0} $, $\displaystyle \left|x_{n}-c\right| < \delta $. Recordamos que $\displaystyle x_{n} \neq c $. Por tanto, $\displaystyle \left|f\left(x_{n}\right)-l\right|<\epsilon  $.
	\item[(ii) $\displaystyle \Rightarrow $ (i)] Si $\displaystyle \left\{ x_{n}\right\} _{n\in\N}\subset A/ \left\{ c\right\}  $ y $\displaystyle x_{n}\to c $, entonces $\displaystyle f\left(x_{n}\right) \to l $. Quiero ver que $\displaystyle \forall \epsilon > 0 $ existe $\displaystyle \delta > 0 $ tal que si $\displaystyle 0 < \left|x-c\right| < \delta  $ con $\displaystyle x \in A/ \left\{ c\right\}  $, entonces $\displaystyle \left|f\left(x\right)-l\right|<\epsilon $. Supongamos lo contrario. Es decir, $\displaystyle \exists \epsilon > 0, \; \forall\delta>0, \exists 0 < \left|x-c\right|<\delta  $ con $\displaystyle x \in A $ tal que $\displaystyle \left|f\left(x\right)-l\right|\geq \epsilon  $. Tomamos $\displaystyle \delta = \frac{1}{n} $, entonces existe $\displaystyle x_{n} $ tal que $\displaystyle 0 < \left|x_{n}-c\right|<\frac{1}{n} $, con $\displaystyle x_{n} \in A $ tal que $\displaystyle \left|f\left(x_{n}\right)-l\right|\geq\epsilon $. Así, $\displaystyle x_{n}\to c $, $\displaystyle x_{n} \in A/ \left\{ c\right\}  $ pero tenemos que $\displaystyle f\left(x_{n}\right) $ no tiende a $\displaystyle l $, que contradice nuestra hipótesis. Por tanto, $\displaystyle \lim_{x \to c}f\left(x\right)=l $. \footnote{Cuando pasamos de sucesión a variable continua, lo solemos hacer por reducción al absurdo.} 
\end{description}
\end{proof}

\begin{eg}
\normalfont Consideramos la función
\[f\left(x\right) = \sig x =
\begin{cases}
1, \; x > 0 \\
0 , \; x = 0 \\
-1, \; x < 0
\end{cases}
.\]
Tenemos que $\displaystyle x_{n} = \frac{1}{n} \to 0 $ y $\displaystyle f\left(x_{n}\right) = 1 \to 1 $, pero si tomamos $\displaystyle y_{n} = -\frac{1}{n} \to 0 $, pero $\displaystyle f\left(y_{n}\right) = -1 \to -1$. Entonces, como $\displaystyle -1 \neq 1 $, tenemos que no existe $\displaystyle \lim_{x \to 0}f\left(x\right) $.
\end{eg}

\begin{eg}
\normalfont Vamos a ver que no existe el límite $\displaystyle \lim_{x \to 0}\frac{1}{x} $ si $\displaystyle x > 0 $. Basta tomar $\displaystyle x_{n} = \frac{1}{n} \to 0 $, pero $\displaystyle \frac{1}{x_{n}} = n $ diverge, por lo que no converge a un número real finito.
\end{eg}

\begin{fdefinition}[]
\normalfont Sea $\displaystyle f: A \subset \R \to \R $ y $\displaystyle x_{0} \in A' $. Diremos que $\displaystyle \lim_{x \to x_{0}}f\left(x\right) = \infty $ si $\displaystyle \forall C > 0 $, $\displaystyle \exists \delta > 0 $ tal que $\displaystyle 0 < \left|x - x_{0}\right| < \delta  $, $\displaystyle x \in A $, entonces $\displaystyle f\left(x\right) > C $. \\ 
Análogamente, diremos que $\displaystyle \lim_{x \to x_{0}}f\left(x\right)=-\infty $ si $\displaystyle \forall C < 0 $, $\displaystyle \exists \delta > 0 $ tal que si $\displaystyle 0 < \left|x-x_{0}\right| < \delta  $, $\displaystyle x \in A $, entonces, $\displaystyle f\left(x\right) < C $.
\end{fdefinition}

\begin{eg}
\normalfont Sea $\displaystyle f: \left(0, \infty\right) \to \R $ con $\displaystyle f\left(x\right) = \frac{1}{x} $. Entonces, $\displaystyle \lim_{x \to 0}f\left(x\right) = \infty $. En efecto, dada $\displaystyle C > 0 $ tomamos $\displaystyle \delta < \frac{1}{C} $ y obtenemos que 
\[ 0 < \left|x-0\right| = x < \delta < \frac{1}{C} \Rightarrow \frac{1}{x} > \frac{1}{\delta }> C .\]
\end{eg}
\begin{eg}
	\normalfont Sea $\displaystyle g: \R/ \left\{ 0\right\} \to \R $ con $\displaystyle g\left(x\right) = \frac{1}{x} $. El límite $\displaystyle \lim_{x \to 0}g\left(x\right) $ no existe. 
\end{eg}

\begin{fdefinition}[]
\normalfont Sea $\displaystyle f: \left(a, \infty\right) \to \R $, $\displaystyle a \in \R $ y sea $\displaystyle l \in \R $. Se dice que $\displaystyle \lim_{x \to \infty}f\left(x\right) = l $ si $\displaystyle \forall \epsilon > 0, \; \exists C > 0 $ tal que si $\displaystyle x > C $ entonces $\displaystyle \left|f\left(x\right)-l\right| < \epsilon  $. \\
Análogamente, si $\displaystyle f:\left(-\infty, a\right) \to \R$ $\displaystyle \lim_{x \to -\infty}f\left(x\right) = l $ si $\displaystyle \forall \epsilon > 0 $, $\displaystyle \exists C < 0 $ tal que si $\displaystyle x < C $ entonces $\displaystyle \left|f\left(x\right)-l\right| < \epsilon  $.
\end{fdefinition}

\begin{eg}
\normalfont Sea $\displaystyle f\left(x\right) = \frac{x+1}{x}, \; x > 0 $. Queremos ver que $\displaystyle \lim_{x \to \infty}\frac{x+1}{x}=1 $. Sea $\displaystyle \epsilon > 0 $, 
\[ \left|\frac{x+1}{x}-1\right| = \left|\frac{x+1-x}{x}\right| =\frac{1}{x}  .\]
Si $\displaystyle C = \frac{1}{\epsilon } $, tenemos que si $\displaystyle x > C $, 
\[x > \frac{1}{\epsilon } \iff \frac{1}{x} < \epsilon  .\]
\end{eg}

\begin{fdefinition}[]
\normalfont Sea $\displaystyle f: \left(a, \infty\right) \to \R $. Diremos que $\displaystyle \lim_{x \to \infty}f\left(x\right) = \infty $ si $\displaystyle \forall C > 0, \; \exists D > 0 $ tal que si $\displaystyle x > D $, con $\displaystyle x \in \left(a, \infty\right) $, entonces $\displaystyle f\left(x\right) > C $. \\
Análogamente, $\displaystyle \lim_{x \to \infty}f\left(x\right) = -\infty $ si $\displaystyle \forall C < 0 $, $\displaystyle \exists D > 0 $ tal que si $\displaystyle x > D $, con $\displaystyle x \in \left(a, \infty\right) $, entonces $\displaystyle f\left(x\right) < C $. \\
Análogamente, sea $\displaystyle f: \left(-\infty, a\right)\to \R $. Tenemos que $\displaystyle \lim_{x \to -\infty}f\left(x\right) = \infty $ si $\displaystyle \forall C > 0 $, $\displaystyle \exists D < 0 $ tal que si $\displaystyle x < D $, entonces $\displaystyle f\left(x\right) > C $. Finalmente, $\displaystyle \lim_{x \to -\infty}f\left(x\right)=-\infty $ si $\displaystyle \forall C < 0, \; \exists D < 0 $ tal que si $\displaystyle x < D $ entonces $\displaystyle f\left(x\right) < C $. 
\end{fdefinition}

\begin{observation}
\normalfont Análogamente al criterio de existencia de límite por convergencia de sucesiones, se puede adaptar el mismo argumento en los restantes casos de límites. Por ejemplo, si $\displaystyle \lim_{x \to \infty}f\left(x\right) = \infty $ y sea $\displaystyle x_{n} \to \infty $ si $\displaystyle n \to \infty $. Entonces tenemos que $\displaystyle f\left(x_{n}\right)\to \infty $. Queremos ver que $\displaystyle \forall C > 0 $, $\displaystyle \exists n_{0} \in \N $ tal que si $\displaystyle n \geq n_{0} $, entonces $\displaystyle f\left(x_{n}\right) > C $. Sabemos que $\displaystyle \forall C>0 $, existe $\displaystyle D > 0 $ tal que si $\displaystyle x > D $, con $\displaystyle x \in \dom f $, entonces $\displaystyle f\left(x\right) > C $. Como $\displaystyle x_{n}\to \infty$, para $\displaystyle D > 0 $, tenemos que existe $\displaystyle n_{0}\in\N $ tal que si $\displaystyle n \geq n_{0} $, entonces $\displaystyle x_{n} > D $, por lo que $\displaystyle f\left(x_{n}\right) > C $. 
\end{observation}

\begin{eg}
\normalfont Sea $\displaystyle f\left(x\right) = \frac{\sin x}{x}, \; x > 0 $. Queremos ver si $\displaystyle \lim_{x \to \infty}f\left(x\right)=0 $. Queremos ver que $\displaystyle \forall\epsilon > 0, \; \exists C>0 $ tal que si $\displaystyle x > C $ entonces $\displaystyle \left|\frac{\sin x}{x}\right| < \epsilon  $. Tenemos que
\[ \left|\frac{\sin x}{x}\right| = \frac{ \left|\sin x\right|}{x} \leq \frac{1}{x} < \epsilon  .\]
Para que esto sea cierto, cogemos $\displaystyle C = \frac{1}{\epsilon } $, así, $\displaystyle x > C \Rightarrow \frac{1}{x} < \frac{1}{C} = \epsilon  $.
\end{eg}

\begin{fdefinition}[Localmente acotada]
\normalfont Sea $\displaystyle f: A \subset \R \to \R $, $\displaystyle x_{0} \in A' $. Se dice que $\displaystyle f $ está \textbf{localmente acotada}  en $\displaystyle x_{0} $ si $\displaystyle \exists\delta>0 $ y $\displaystyle \exists C > 0 $ tal que si $\displaystyle 0 < \left|x -x_{0}\right|<\delta  $, con $\displaystyle x \in A $, entonces $\displaystyle \left|f\left(x\right)\right| < C $. \footnote{No es necesario que $\displaystyle x_{0} $ sea punto de acumulación.} 
\end{fdefinition}

\begin{eg}
	\normalfont Tenemos que $\displaystyle f\left(x\right) = x^{2} $ está localmente acotada en todo $\displaystyle x_{0} \in \R $. Basta tomar $\displaystyle \delta = 1 $ y $\displaystyle f\left(x_{0}+\delta \right) = \left(x_{0}+\delta \right)^{2} $ y $\displaystyle f\left(x_{0}-\delta \right)= \left(x_{0}- \delta \right)^{2} $ y tomar $\displaystyle C = \max \left\{ f\left(x_{0}+\delta \right), f\left(x_{0}-\delta \right)\right\}  $. Tenemos que $\displaystyle \left|f\left(x\right)\right| = x^{2} \leq C, \; x \in \left(x_{0}-\delta, x_{0}+\delta \right) $.
\end{eg}

\begin{ftheorem}[]
\normalfont Sea $\displaystyle f : A\subset\R \to \R $, $\displaystyle x_{0} \in A' $, $\displaystyle l \in \R $ y $\displaystyle \lim_{x \to x_{0}} = l $. Entonces, $\displaystyle f $ está localmente acotada en $\displaystyle x_{0} $.
\end{ftheorem}

\begin{proof}
Dado $\displaystyle \epsilon = 1 $, tenemos que existe $\displaystyle \delta > 0 $ tal que si $\displaystyle 0 < \left|x -x_{0}\right| < \delta  $, entonces $\displaystyle \left|f\left(x\right)-l\right|<1 $. Entonces, tenemos que 
\[ \left|f\left(x\right)\right| = \left|f\left(x\right)-l+l\right| \leq \left|f\left(x\right)-l\right| + \left|l\right| < 1 + \left|l\right| .\]
Cogemos $\displaystyle C = 1 + \left|l\right| $.
\end{proof}

\begin{eg}
\normalfont El recíproco no funciona. Cogemos, por ejemplo $\displaystyle f\left(x\right) = \sin \frac{1}{x} $ con $\displaystyle x \neq 0 $. Tenemos que está función está localmente acotada en $\displaystyle 0 $ por 1, pero $\displaystyle \lim_{x \to 0} f\left(x\right) $ no existe.  
\end{eg}

\begin{ftheorem}[]
	\normalfont Sean $\displaystyle f, g : A \subset \R \to \R $, $\displaystyle x_{0} \in A' $, sea $\displaystyle l = \lim_{x \to x_{0}}f\left(x\right)$ y $\displaystyle m = \lim_{x \to x_{0}}g\left(x\right) $. Entonces, 
\begin{description}
\item[(i)] $\displaystyle \lim_{x \to x_{0}}\left(f\left(x\right)+g\left(x\right)\right) = l+m $.
\item[(ii)] $\displaystyle \lim_{x \to x_{0}}\left(f\left(x\right)-g\left(x\right)\right) = l -m $.
\item[(iii)] $\displaystyle \forall a \in \R $, $\displaystyle \lim_{x \to x_{0}}\left(af\left(x\right)\right) = a \cdot l $.
\item[(iv)] $\displaystyle \lim_{x \to x_{0}}\left(f\left(x\right)g\left(x\right)\right) = l \cdot m$.
\item[(v)] Si $\displaystyle m \neq 0 $, y $\displaystyle g\left(x\right) \neq 0 $, $\displaystyle \lim_{x \to x_{0}}\frac{f\left(x\right)}{g\left(x\right)} = \frac{l}{m} $.
\end{description}
\end{ftheorem}

\begin{proof}
\begin{description}
\item[(i)] Sea $\displaystyle \epsilon > 0 $ y sea $\displaystyle \delta > 0 $ tal que $\displaystyle 0 < \left|x -x_{0}\right| < \delta  $. Entonces 
	\[ \left|f\left(x\right)-l\right| < \frac{\epsilon }{2} \quad \text{y} \quad \left|g\left(x\right)-m\right| < \frac{\epsilon }{2} .\]
	Así, 
	\[ \left|f\left(x\right)+g\left(x\right) -l-m\right| \leq \left|f\left(x\right)-l\right| + \left|g\left(x\right)-m\right| < \frac{\epsilon }{2} + \frac{\epsilon }{2} = \epsilon  .\]
si $\displaystyle 0 < \left|x - x_{0}\right| < \delta , \; x \in A $.
\item[(ii)] La demostración es análoga a \textbf{(i)}.
\item[(iii)] Aplicamos el criterio del límite por sucesiones. Como $\displaystyle \lim_{x \to x_{0}}f\left(x\right)=l $, tenemos que si $\displaystyle \left\{ x_{n}\right\} _{n\in\N}\subset A/ \left\{ x_{0}\right\}  $ con $\displaystyle x_{n} \to x_{0}$, entonces $\displaystyle f\left(x_{n}\right) \to l $. Queremos ver que $\displaystyle \left(af\right)\left(x_{n}\right) \to al $. Tenemos que
	\[\left(af\right)\left(x_{n}\right) = a f\left(x_{n}\right)  .\]
Aplicando las propiedades de los límites de sucesiones, $\displaystyle a \to a $ y $\displaystyle f\left(x_{n}\right) \to l $. Por tanto, $\displaystyle \left(af\right)\left(x_{n}\right) \to al $.
\item[(iv)] Por el criterio de límite por sucesiones, sea $\displaystyle \left\{ x_{n}\right\} _{n\in\N}\subset A/ \left\{ x_{0}\right\}  $, con $\displaystyle x_{n} \to x_{0} $. Queremos ver que $\displaystyle \left(f \cdot g\right)\left(x_{n}\right) \to l \cdot m $. Tenemos que
	\[\left(f \cdot g\right)\left(x_{n}\right) = f\left(x_{n}\right)g\left(x_{n}\right) .\]
Como $\displaystyle f $ converge y $\displaystyle x_{n} \to x_{0} $, tenemos que $\displaystyle f\left(x_{n}\right) \to l $ y $\displaystyle g\left(x_{n}\right)\to m $. Aplicando las propiedades de los límites de sucesiones, 
\[\left(f \cdot g\right)\left(x_{n}\right) = f\left(x_{n}\right)g\left(x_{n}\right) \to l \cdot m .\]
\item[(v)] Por el criterio del límite por sucesiones, sea $\displaystyle \left\{ x_{n}\right\} _{n\in\N}\subset A/ \left\{ x_{0}\right\}  $ con $\displaystyle x_{n} \to x_{0} $. Vamos a ver que $\displaystyle \left(\frac{f}{g}\right)\left(x_{n}\right) \to \frac{l}{m} $. Tenemos que
	\[\left(\frac{f}{g}\right)\left(x_{n}\right) = \frac{f\left(x_{n}\right)}{g\left(x_{n}\right)} .\]
Tenemos que $\displaystyle f\left(x_{n}\right) \to l $ y $\displaystyle g\left(x_{n}\right)\to m $, por lo que $\displaystyle \left(\frac{f}{g}\right)\left(x_{n}\right) \to \frac{l}{m} $.
\end{description}
\end{proof}

\begin{eg}
\normalfont Calculamos $\displaystyle \lim_{x \to 2}\left(x^{2}+5\right) $. 
\[\lim_{x \to 2}\left(x^{2}+5\right) = \lim_{x \to 2}x^{2} + \lim_{x \to 2}5 = \left(\lim_{x \to 2}x\right)^{2} + 5 = 2^{2} + 5 = 9.\]
\end{eg}

\begin{ftheorem}[]
\normalfont Sea $\displaystyle f : A \subset \R \to \R $, $\displaystyle x_{0} \in A' $ y supongamos que $\displaystyle \exists\lim_{x \to x_{0}}f\left(x\right)=l $. Si $\displaystyle a \leq f\left(x\right) \leq b $, entonces $\displaystyle a \leq l \leq b $.
\end{ftheorem}

\begin{proof}
Dado $\displaystyle \epsilon > 0 $ existe $\displaystyle \delta > 0 $ tal que si $\displaystyle 0 < \left|x-x_{0}\right| < \delta  $, con $\displaystyle x \in A $, entonces $\displaystyle \left|f\left(x\right)-l\right| < \epsilon  $. Entonces, tenemos que 
\[ \left|f\left(x\right)-l\right| < \epsilon \iff - \epsilon < l - f\left(x\right) < \epsilon \iff f\left(x\right)-\epsilon < l < \epsilon + f\left(x\right).\]
Así, tenemos que
\[a - \epsilon < l < \epsilon + b, \; \forall \epsilon > 0 .\]
Entonces, $\displaystyle a \leq l \leq b $ \footnote{Esta demostración se puede hacer también utilizando la caracterización del límite por convergencia de sucesiones.} .
\end{proof}

\begin{ftheorem}[Regla del bocadillo]
\normalfont Si $\displaystyle f, g, h : A\subset \R \to \R $, $\displaystyle x_{0} \in A' $. Si $\displaystyle g\left(x\right) \leq f\left(x\right) \leq h\left(x\right) $ y $\displaystyle \exists \lim_{x \to x_{0}}g\left(x\right)=\lim_{x \to x_{0}}h\left(x\right) = l $, entonces, $\displaystyle \lim_{x \to x_{0}}f\left(x\right) = l $.
\end{ftheorem}

\begin{proof}
Sea $\displaystyle \epsilon > 0 $, buscamos un $\displaystyle \delta > 0 $ tal que si $\displaystyle x \in A $ y $\displaystyle 0 < \left|x - x_{0}\right| < \delta $, entonces $\displaystyle \left|f\left(x\right)-l\right| < \epsilon  $. Por hipótesis, sabemos que existe $\displaystyle \delta > 0 $ tal que si $\displaystyle 0 < \left|x-x_{0}\right| < \delta  $ con $\displaystyle x \in A $, 
\[ \left|g\left(x\right)-l\right|< \epsilon \quad \text{y} \quad \left|h\left(x\right)-l\right|< \epsilon.\]
Así, tenemos que
\[l - \epsilon < g\left(x\right) \leq f\left(x\right) \leq h\left(x\right) < l+\epsilon .\]
Así, tenemos que $\displaystyle \left|f\left(x\right)-l\right|<\epsilon  $.
\end{proof}

\begin{eg}
\normalfont Vamos a ver $\displaystyle \lim_{x \to 0}\frac{x}{\sin x}=1 $. Geométricamente podemos ver que 
\[\sin x \leq x \leq \tan x = \frac{\sin x}{\cos x} .\]
Como estamos viendo el intervalo $\displaystyle \left(0, \frac{\pi }{2}\right] $. Entonces, podemos hacer
\[1 \leq \frac{x}{\sin x} \leq \frac{1}{\cos x} .\]
Como $\displaystyle 1\to1 $ y $\displaystyle \frac{1}{\cos x} \to 1 $, tenemos que $\displaystyle \frac{x}{\sin x} \to 1 $ por la regla del bocadillo.
\end{eg}

\begin{eg}
\normalfont 
\begin{description}
	\item[(i)] Sea $\displaystyle A = \left(0,1\right) \cup \left(1, \infty\right) $ y $\displaystyle f : A \to \R $ tal que $\displaystyle f\left(x\right) = \frac{x^{3}-x^{2}-x+1}{x\left(x-1\right)^{2}} $. Tenemos que $\displaystyle A' = [0, \infty) $. Si $\displaystyle x = 0 $, tenemos que $\displaystyle x_{n} = \frac{1}{n} \in A $ y $\displaystyle x_{n} \to 0 $, por lo que $\displaystyle 0 \in A' $. Hacemos lo mismo con el 1 y la sucesión $\displaystyle x_{n} = 1 +\frac{1}{n} \to 1 $.Si $\displaystyle x < 0 $, no existe $\displaystyle \left\{ x_{n}\right\} _{n\in\N}\subset A $ tal que $\displaystyle x_{n} \to x $. Por lo que $\displaystyle x \not\in A' $. Vamos a ver que $\displaystyle \lim_{x \to 0}f\left(x\right)= \infty $. Si $\displaystyle x > 0 $, 
\[\frac{x^{3}-x^{2}-x+1}{x\left(x-1\right)^{2}} = \frac{\left(x-1\right)^{2}\left(x+1\right)}{x\left(x-1\right)^{2}} = \frac{x+1}{x} = 1 + \frac{1}{x} .\]
Queremos ver que $\displaystyle \forall C>0, \; \exists\delta>0 $ tal que si $\displaystyle 0 < \left|x-0\right| < \delta  $, $\displaystyle x \in A $, entonces $\displaystyle f\left(x\right) > C $.Tenemos que conseguir que 
\[1 + \frac{1}{x} > C \iff \frac{1}{x} > C - 1 .\]
Entonces, tomamos $\displaystyle \delta = \min \left\{ \frac{1}{ \left|C-1\right|}, 1\right\}  $, así, si $\displaystyle 0 < x < \delta  $, entonces $\displaystyle 1 + \frac{1}{x} = f\left(x\right) > C $. \\
Ahora demostramos que $\displaystyle \lim_{x \to 1}f\left(x\right) = 2 $. Tenemos que 
\[\lim_{x \to 1}f\left(x\right) = \lim_{x \to 1}\left(1+\frac{1}{x}\right) = \lim_{x \to 1}1 + \lim_{x \to 1}\frac{1}{x} = 1 + 1 = 2 .\]
Ahora demostramos que $\displaystyle \lim_{x \to 2}f\left(x\right) $,
\[\lim_{x \to 2}f\left(x\right) = \frac{\lim_{x \to 2}\left(x^{3}-x^{2}-x+1\right)}{\lim_{x \to 2}x\left(x-1\right)^{2}} = \frac{3}{2} .\]
Finalmente, demostramos que $\displaystyle \lim_{x \to \infty}f\left(x\right) = 1 $. Tenemos que 
\[\lim_{x \to \infty}f\left(x\right) = \lim_{x \to \infty}1 + \lim_{x \to \infty}\frac{1}{x} = 1 + 0 = 1.\]
\item[(ii)] Sea $\displaystyle f : \R \to \R $ tal que $\displaystyle f\left(x+y\right) = f\left(x\right) + f\left(y\right), \; \forall x,y \in \R $, y $\displaystyle \exists\lim_{x \to 0}f\left(x\right)=l $. Vamos a demostrar que $\displaystyle l = 0 $ y $\displaystyle \forall c\in \R, \; \exists \lim_{x \to c}f\left(x\right) $. \\ 
Tenemos que $\displaystyle 0 = x + \left(-x\right) $, por lo que $\displaystyle f\left(0\right) = f\left(x\right)+f\left(-x\right) $. Tenemos que $\displaystyle \lim_{x \to 0}f\left(x\right) = l $. Ahora queremos ver que $\displaystyle \lim_{x \to 0}f\left(-x\right) = l $. Sabemos que $\displaystyle \forall\epsilon>0, \; \exists\delta>0 $ tal que si $\displaystyle 0 < \left|x-0\right| < \delta  $, entonces $\displaystyle \left|f\left(x\right)-l\right|<\epsilon  $. Queremos ver que $\displaystyle \forall\epsilon>0, \; \exists\delta>0 $ tal que si $\displaystyle 0 < \left|x-0\right| < \delta  $, entonces $\displaystyle \left|f\left(-x\right)-l\right|<\epsilon  $. Tenemos que si $\displaystyle 0 < \left|x-0\right|<\delta  $, entonces $\displaystyle 0 < \left|-x-0\right|<\delta  $. Por tanto, tenemos que $\displaystyle f\left(-x\right)\to l $ y $\displaystyle f\left(0\right) = f\left(x\right) + f\left(-x\right) = 2l $ \footnote{Tenemos que $\displaystyle \lim_{x \to 0}f\left(0\right) = \lim_{x \to 0}f\left(x\right) + \lim_{x \to 0}f\left(-x\right) $. Como $\displaystyle f\left(0\right) $ es una constante, tenemos que $\displaystyle \lim_{x \to 0}f\left(0\right)=f\left(0\right) $.} . Tenemos que
	\[x = 0 + x \Rightarrow f\left(x+0\right) = f\left(x\right) + f\left(0\right) = f\left(x\right) \Rightarrow f\left(0\right) = 0 \Rightarrow l=0 .\]
Ahora vamos a ver que $\displaystyle \forall c \in \R, \; \exists \lim_{x \to c}f\left(x\right) $.
\[f\left(x\right) = f\left(x-c+c\right)=f\left(x-c\right) + f\left(c\right) .\]
Vamos a ver que $\displaystyle \lim_{x \to c}f\left(x-c\right) = 0 $. Sea $\displaystyle x - c = y $. Entonces, $\displaystyle \forall \epsilon > 0, \; \exists \delta > 0 $ tal que si $\displaystyle 0< \left|y\right| < \delta  $, tenemos que $\displaystyle \left|f\left(y\right)\right| < \epsilon  $. Así, 
\[f\left(x\right) = f\left(x-c+c\right)=f\left(x-c\right) + f\left(c\right) \to f\left(c\right) .\]
Demostramos que $\displaystyle f $ es lineal. Tenemos que
\[f\left(x\right) = f\left(\frac{x}{n} + \cdots + \frac{x}{n}\right) = n f\left(\frac{x}{n}\right) .\]
Similarmente, 
\[f\left(nx\right) = f\left(x+\cdots + x\right) = nf\left(x\right) .\]
Así, 
\[f\left(\frac{p}{q}x\right) = \frac{p}{q}f\left(x\right) \Rightarrow f\left(\lambda x\right)= \lambda x, \; \forall \lambda \in \R .\]
Este último paso se demuestra por continuidad y con sucesiones de racionales que tiendan a números reales (como la parte de exponentes reales). Ahora, podemos ver que $\displaystyle f $ es lineal. Para que sea lineal, tenemos que $\displaystyle f\left(x\right) = cx $. Esto es fácil, pues tenemos que $\displaystyle f\left(1\right) = c $.
\item[(iii)] Sea $\displaystyle f: A \subset \R \to \R $, $\displaystyle c \in A' $ y supongamos que $\displaystyle \exists \lim_{x \to c}f\left(x\right) = l >0 $. Entonces, $\displaystyle \exists \delta > 0 $ tal que $\displaystyle \forall x \in \left(c - \delta, c + \delta \right) \cap A $, $\displaystyle f\left(x\right) > 0 $. \\ 
Sabemos que $\displaystyle \forall \epsilon > 0, \; \exists \delta > 0 $ tal que si $\displaystyle 0 < \left|x - c\right| < \delta  $, $\displaystyle x \in A $, entonces $\displaystyle \left|f\left(x\right)-l\right|<\epsilon  $. Tomamos $\displaystyle \epsilon = \frac{l}{2} $, tal que $\displaystyle l - \epsilon = \frac{l}{2} > 0 $. Entonces $\displaystyle \exists \delta > 0 $ tal que si $\displaystyle x \in \left(c-\delta, c + \delta \right)\cap A $, con $\displaystyle x \neq c $, entonces:
\[ \left|f\left(x\right)-l\right|<\epsilon \iff -\frac{l}{2} < f\left(x\right)-l<\frac{l}{2} \Rightarrow 0 < \frac{l}{2} < f\left(x\right) .\]
\item[(iv)] Sea $\displaystyle f: \R \to \R $. Vamos a ver si $\displaystyle \exists\epsilon>0, \; \forall \delta > 0 $, $\displaystyle  \left|x - c\right| < \delta \Rightarrow \left|f\left(x\right)-f\left(c\right)\right|<\epsilon  $.\\
Sea $\displaystyle x \in \R $ y sea $\displaystyle \delta = 2 \left|x - c\right|+1 > 0 $. Entonces
\[ \left|x-c\right|<\delta \Rightarrow \left|f\left(x\right)-f\left(c\right)\right|<\epsilon \iff f\left(c\right)-\epsilon < f\left(x\right) < f\left(c\right)+\epsilon  .\]
\item[(v)] Sea $\displaystyle f: \R \to \R $. El enunciado $\displaystyle \forall \delta > 0, \;\exists \epsilon > 0  $ tal que si $\displaystyle 0 < \left|x - c\right|<\epsilon \Rightarrow \left|f\left(x\right)-f\left(c\right)\right|<\delta  $ es equivalente a decir que $\displaystyle \lim_{x \to c}f\left(x\right)=f\left(c\right) $.
\item[(vi)] 
	\begin{description}
	\item[(a)] $\displaystyle f\left(x\right) = \frac{1}{x}, \; x > 0 $.
		\[\lim_{x \to 0}f\left(x\right) = \lim_{x \to 0}\frac{1}{x}=\infty .\]
	\item[(b)] $\displaystyle f\left(x\right)=\frac{1}{x}, \; x \in \R/ \left\{ 0\right\}  $.
		\[\not\exists\lim_{x \to 0}f\left(x\right) .\]
	\end{description}
\end{description}
\end{eg}

\begin{fdefinition}[Límites laterales]
	\normalfont Sea $\displaystyle f: A \subset \R \to \R $, y sea $\displaystyle c $ un punto de acumulación de $\displaystyle A \cap \left(c, \infty\right) $. Sea $\displaystyle l \in \R $. Se dice que $\displaystyle l $ es el \textbf{límite por la derecha} de $\displaystyle f $ en $\displaystyle c $: \[\displaystyle \lim_{x \to c^{+}}f\left(x\right) = l, \] si $\displaystyle \forall \epsilon >0, \; \exists \delta > 0 $ tal que si $\displaystyle 0 < x - c < \delta  $ con $\displaystyle x \in A $, entonces $\displaystyle \left|f\left(x\right)-l\right|<\epsilon  $. \\
Análogamente, si $\displaystyle c $ es un punto de acumulación de $\displaystyle A \cap \left(-\infty, c\right) $. Decimos que 
\[\lim_{x \to c^{-}}f\left(x\right) = l ,\]
si $\displaystyle \forall \epsilon > 0, \; \exists \delta > 0 $ tal que si $\displaystyle 0 < c - x < \delta$, $\displaystyle x \in A $, entonces $\displaystyle \left|f\left(x\right)-l\right|<\epsilon $. 
\end{fdefinition}

\begin{fdefinition}[]
\normalfont Sea $\displaystyle f : A\subset\R \to \R $ y sea $\displaystyle c \in \left(A \cap \left(c, \infty\right)\right)' $. 
\begin{itemize}
\item $\displaystyle \lim_{x \to c^{+}}f\left(x\right) = \infty $ si $\displaystyle \forall C > 0, \; \exists \delta > 0 $ tal que si $\displaystyle 0 < x- c < \delta  $, con $\displaystyle x\in A $, entonces $\displaystyle f\left(x\right) > C $.
\item $\displaystyle \lim_{x \to c^{+}}f\left(x\right) = -\infty $ si $\displaystyle \forall C < 0, \; \exists \delta > 0 $ tal que si $\displaystyle 0 < x - c< \delta  $ con $\displaystyle x \in A $, entonces $\displaystyle f\left(x\right) < C $.
\end{itemize}
Ahora consideramos $\displaystyle c \in \left(A \cap \left(-\infty, c\right)\right)' $.
\begin{itemize}
\item $\displaystyle \lim_{x \to c^{-}}f\left(x\right) = \infty $ si $\displaystyle \forall C > 0, \; \exists \delta > 0 $ tal que si $\displaystyle 0 < c - x < \delta  $ con $\displaystyle x \in A $, entonces $\displaystyle f\left(x\right) > C $.
\item $\displaystyle \lim_{x \to c^{-}}f\left(x\right) = -\infty $ si $\displaystyle \forall C<0, \; \exists \delta > 0 $ tal que si $\displaystyle 0 < c - x < \delta  $ con $\displaystyle x \in A $, entonces $\displaystyle f\left(x\right) < C $. 
\end{itemize}
\end{fdefinition}

\begin{eg}
\normalfont En el ejemplo anterior, tenemos que si $\displaystyle f\left(x\right) = \frac{1}{x} $ con $\displaystyle x \in \R / \left\{ 0\right\} $, 
\[\lim_{x \to 0^{+}}f\left(x\right) = \infty \quad \text{y} \quad \lim_{x \to 0^{-}}f\left(x\right) = - \infty .\]
\end{eg}

\begin{ftheorem}[]
\normalfont Sea $\displaystyle f: A \subset \R \to \R $ y sea $\displaystyle c \in \left(A \cap \left(c, \infty\right)\right)'\cap\left(A \cap \left(-\infty, c\right)\right)' $. Entonces $\displaystyle \lim_{x \to c}f\left(x\right) = l $ si  y solo si 
\[\lim_{x \to c^{+}}f\left(x\right) = \lim_{x \to c^{-}}f\left(x\right) = l .\]
\end{ftheorem}

\begin{proof}
\begin{description}
\item[(i)] Sabemos que $\displaystyle \forall \epsilon > 0, \; \exists \delta > 0 $ tal que si $\displaystyle 0<\left|x-c\right|<\delta  $ con $\displaystyle x \in A $, entonces $\displaystyle \left|f\left(x\right)-l\right|<\epsilon  $. Vamos a ver que $\displaystyle \forall \epsilon > 0, \; \exists \delta > 0 $ tal que si $\displaystyle 0 < x - c < \delta  $, entonces $\displaystyle \left|f\left(x\right)-l\right|<\epsilon  $. Es trivial, pues basta tomar el mismo $\displaystyle \delta  $ de la definición de límite. El análogo para el límite por la izquierda es también trivial.
\item[(ii)] Dado $\displaystyle \epsilon > 0 $, $\displaystyle \exists \delta_{1} > 0 $ tal que si $\displaystyle 0 < x-c<\delta_{1} $, $\displaystyle x \in A $, entonces $\displaystyle \left|f\left(x\right)-l\right|<\epsilon  $. Similarmente, $\displaystyle \exists\delta_{2} > 0 $ tal que si $\displaystyle 0 < c - x < \delta _{2} $, con $\displaystyle x \in A $, entonces $\displaystyle \left|f\left(x\right)-l\right|<\epsilon  $. Sea $\displaystyle \delta = \min \left\{ \delta_{1}, \delta_{2}\right\} >0 $ y tengo que $\displaystyle 0 < \left|x-c\right|<\delta  $, con $\displaystyle x \in A $. Si $\displaystyle x > c $,
	\[0 < x - c < \delta \leq \delta_{1}, \; x \in A\; \Rightarrow \left|f\left(x\right)-l\right|<\epsilon  .\]
Si $\displaystyle x < c$,
\[0 < c - x < \delta \leq \delta_{2}, \; x \in A\; \Rightarrow \left|f\left(x\right)-l\right|<\epsilon  .\]
\end{description}
\end{proof}

\begin{eg}
\normalfont Sea $\displaystyle f\left(x\right) = \sig\left(x\right), \; x \in \R $. Tenemos que 
\[\lim_{x \to 0^{+}}f\left(x\right) = 1 \neq \lim_{x \to 0^{-}}f\left(x\right) = -1 .\]
Por tanto, no existe $\displaystyle \lim_{x \to 0}f\left(x\right) $.
\end{eg}

\begin{eg}
\normalfont $\displaystyle f, g : \R \to \R $. Sea $\displaystyle f\left(x\right) = x + 1 $ y 
\[g\left(x\right) = 
\begin{cases}
2, \; x \neq 1 \\
0, \; x = 1
\end{cases}
.\]
\begin{description}
\item[(a)] Calcular $\displaystyle \left(g \circ f\right)\left(x\right) $ y $\displaystyle g\left(\lim_{x \to 1}f\left(x\right)\right) $.\\
\[\left(g \circ f\right)\left(x\right) = 
\begin{cases}
2, \; x \neq 0 \\
1, \; x = 0
\end{cases}
.\]
Tenemos que 
\[\lim_{x \to 1}\left(g\circ f\right)\left(x\right) = 2 .\]
Ahora, tenemos que $\displaystyle \lim_{x \to 1}f\left(x\right) = 2 $, entonces 
\[g\left(\lim_{x \to 1}f\left(x\right)\right) = g\left(2\right) = 2 .\]
\item[(b)] Calcular $\displaystyle \left(f \circ g\right)\left(x\right) $ y $\displaystyle f\left(\lim_{x \to 1}g\left(x\right)\right) $.
\[\left(f \circ g\right)\left(x\right) = 
\begin{cases}
3, \; x \neq 1\\
1, \; x = 1
\end{cases}
.\]
Tenemos que 
\[\lim_{x \to 1}\left(f\circ g\right)\left(x\right) = \lim_{x \to 1^{+}}\left(f \circ g\right) \left(x\right) = \lim_{x \to 1^{-}}\left(f \circ g\right)\left(x\right) = 3 .\]
Ahora, tenemos que $\displaystyle \lim_{x \to 1}g\left(x\right) = 2 $, entonces 
\[f\left(\lim_{x \to 1}g\left(x\right)\right) = f\left(2\right) = 3 .\]
\end{description}
\end{eg}

\begin{eg}
	\normalfont Consideremos el enunciado $\displaystyle \forall \epsilon > 0, \; \forall n_{0} \in \N, \; \exists n \geq n_{0} $ tal que $\displaystyle \left|x_{n}-l\right|<\epsilon  $. Vamos a ver si este enunciado es equivalente a que exista $\displaystyle \left\{ x_{n_{k}}\right\} _{k\in\N}\subset \left\{ x_{n}\right\} _{n\in\N} $ tal que $\displaystyle x_{n_{k}}\to l $. 
\begin{description}
	\item[(i)] Sea $\displaystyle \epsilon = 1 $. Sea $\displaystyle n_{0} = 1 $, entonces $\displaystyle \exists n_{1} \geq 1 $ tal que $\displaystyle \left|x_{n_{1}}-l\right|<\epsilon = 1  $. Sea $\displaystyle \epsilon = \frac{1}{2} $ y $\displaystyle n_{0} = n_{1} + 1 $. Entonces, $\displaystyle \exists n_{2} \geq n_{0} > n_{1} $ tal que $\displaystyle \left|x_{n_{2}}-l\right|<\frac{1}{2} $. Si tenemos $\displaystyle x_{n_{1}}, \ldots, x_{n_{k}} $ estrictamente creciente, tal que $\displaystyle \left|x_{n_{j}}-l\right|<\frac{1}{j}, \; j = 1, \ldots, k $. Ahora, sea $\displaystyle \epsilon = \frac{1}{k+1} $ y $\displaystyle n_{0} = n_{k} + 1 $. Entonces, $\displaystyle \exists n_{k+1}\geq n_{0} > n_{k} $ tal que $\displaystyle \left|x_{n_{k+1}}-l\right|<\frac{1}{k+1} $. Por tanto, construimos $\displaystyle \left\{ x_{n_{k}}\right\} _{k\in\N}\subset \left\{ x_{n}\right\} _{n\in\N} $ tal que 
	\[ \left|x_{n_{k}}-l\right|<\frac{1}{k}, \; \forall k \in \N \Rightarrow \lim_{k \to \infty}x_{n_{k}} = l .\]
\item[(ii)] Tenemos que $\displaystyle \forall \epsilon > 0, \; \exists n'_{0} \in \N  $ tal que si $\displaystyle n_{k} \geq n'_{0} $, $\displaystyle \left|x_{n_{k}}-l\right|<\epsilon  $. Dado $\displaystyle \epsilon > 0 $ y dado $\displaystyle n_{0} \in \N $, tenemos que ver que $\displaystyle \exists n\geq n_{0} $ tal que $\displaystyle \left|x_{n}-l\right|<\epsilon  $. Sea $\displaystyle n_{k} \geq \max \left\{ n_{0}, n'_{0}\right\}  $, entonces
	\[ \left|x_{n_{k}}-l\right|<\epsilon, \; n_{k} \geq n_{0} .\]
\end{description}
\end{eg}
\begin{eg}
\normalfont Vamos a demostrar que $\displaystyle \lim_{x \to \infty}\left(1 + \frac{1}{x}\right)^{x}=e $. Sabemos que $\displaystyle \lim_{n \to \infty}\left(1 + \frac{1}{n}\right)^{n} =e $. Dado $\displaystyle x \in \R $ con $\displaystyle x \geq 1 $, entonces existen $\displaystyle n_{x} = \left\lfloor x \right\rfloor \in \N $ y $\displaystyle \alpha_{x}\in[0,\infty) $ tal que $\displaystyle x = n_{x} +\alpha_{x} $. Tenemos que $\displaystyle \lim_{x \to \infty}n_{x} = \infty $. Entonces, tenemos que
\[\left(1 + \frac{1}{x}\right)^{x} = \left(1 + \frac{1}{n_{x}+\alpha_{x}}\right)^{n_{x}+\alpha_{x}}= \left(1 + \frac{1}{n_{x}+\alpha_{x}}\right)^{n_{x}} \left(1 + \frac{1}{n_{x}+\alpha_{x}}\right)^{\alpha_{x}}.\]
Vamos a ver que $\displaystyle \left(1 + \frac{1}{n_{x}+\alpha_{x}}\right)^{\alpha_{x}} \to 1 $. Tenemos que
\[1 \leq \left(1 + \frac{1}{n_{x}+\alpha_{x}}\right)^{\alpha_{x}}\leq \underbrace{ 1 + \frac{1}{n_{x}+\alpha_{x}}}_{\to1} .\]
Ahora vamos a ver que $\displaystyle \lim_{x \to \infty}\left(1+\frac{1}{n_{x}+\alpha_{x}}\right)^{n_{x}}=e $. Tenemos que 
\[\left(1+\frac{1}{n_{x}+1}\right)^{n_{x}}\leq\left(1 + \frac{1}{n_{x}+\alpha_{x}}\right)^{n_{x}} \leq \underbrace{\left(1 + \frac{1}{n_{x}}\right)^{n_{x}}}_{\to e} .\]
El término de la izquierda lo cogemos y lo multiplicamos y dividimos por $\displaystyle \left(1+\frac{1}{n_{x}+1}\right) \to 1 $. Así,
\[\left(1+\frac{1}{n_{x}+1}\right)^{n_{x}} = \left(1 + \frac{1}{n_{x}+1}\right)^{n_{x}+1}\left(1+\frac{1}{n_{x}+1}\right)^{-1} \to e .\]
Así, $\displaystyle \left(1+\frac{1}{n_{x}+\alpha_{x}}\right)^{n_{x}} \to e $, por lo que el límite original converge a $\displaystyle e $.
\end{eg}

\begin{eg}
\normalfont Si $\displaystyle g\left(x\right) \to \infty $ cuando $\displaystyle x \to \infty $ y $\displaystyle f\left(x\right) \to l $ si $\displaystyle x \to \infty $, entonces $\displaystyle \left(f \circ g\right)\left(x\right) \to l $. \\
Sabemos que $\displaystyle \forall C > 0, \; \exists K > 0 $ tal que $\displaystyle \forall x > K $, $\displaystyle g\left(x\right) > C $. Similarmente, tenemos que $\displaystyle \forall \epsilon > 0, \; \exists K > 0 $ tal que si $\displaystyle x > K $ entonces $\displaystyle \left|f\left(x\right)-l\right|<\epsilon  $. Sea $\displaystyle \epsilon > 0 $, existe $\displaystyle K > 0 $ tal que cumpla los requisitos de la convergencia de $\displaystyle f $. Ahora, puedo encontrar $\displaystyle K' > 0 $ tal que si $\displaystyle x > K' $, entonces $\displaystyle g\left(x\right) > K $. Así, $\displaystyle \left|f\left(g\left(x\right)\right)-l\right|<\epsilon  $.
\end{eg}
\begin{fcolorary}[]
\normalfont Sea $\displaystyle g\left(x\right) \to \infty $ cuando $\displaystyle x \to \infty $. Tenemos que $\displaystyle \lim_{x \to \infty}\left(1+\frac{1}{g\left(x\right)}\right)^{g\left(x\right)}=e. $ 
\end{fcolorary}

\begin{ftheorem}[]
\normalfont Sea $\displaystyle A\subset\R $, $\displaystyle f, g : A \to \R $, $\displaystyle \left(a, \infty\right)\subset A $. Supongamos que $\displaystyle g\left(x\right) \neq 0 $, $\displaystyle \forall x > a $ y existe $\displaystyle L \in \R $, $\displaystyle L\neq 0 $ tal que $\displaystyle \lim_{x \to \infty}\frac{f\left(x\right)}{g\left(x\right)} = L$. Entonces
\begin{itemize}
\item Si $\displaystyle L > 0 $: si $\displaystyle x \to \infty $, $\displaystyle f\left(x\right) \to \infty \iff g\left(x\right) \to \infty $.
\item Si $\displaystyle L < 0 $: si $\displaystyle x \to \infty $, $\displaystyle f\left(x\right) \to \pm \infty \iff g\left(x\right) \to \mp \infty $.
\end{itemize}
\end{ftheorem}

\begin{proof}
Supongamos que $\displaystyle L > 0 $ y que $\displaystyle \lim_{x \to \infty}f\left(x\right) = \infty $. Queremos ver que $\displaystyle \forall C > 0, \; \exists K > 0 $ tal que si $\displaystyle x > K $ entonces $\displaystyle f\left(x\right) > C $. Sabemos que $\displaystyle \forall C' > 0, \; \exists K' > 0 $ tal que si $\displaystyle x > K' $ entonces $\displaystyle f\left(x\right) > C' $. Similarmente, $\displaystyle \forall \epsilon > 0, \; \exists K'' > 0 $ tal que si $\displaystyle x > K'' $ entonces $\displaystyle \left|\frac{f\left(x\right)}{g\left(x\right)}-l\right|<\epsilon  $. Entonces, cogemos $\displaystyle \epsilon = \frac{L}{2} $,
\[ \frac{L}{2}g\left(x\right)< f\left(x\right) < \frac{3}{2}Lg\left(x\right)  .\]
Sea $\displaystyle C' = \frac{3CL}{2} > 0 $, entonces existe $\displaystyle K'>0 $ tal que si $\displaystyle x > K $, entonces $\displaystyle f\left(x\right) > C $. Sea $\displaystyle K = \max \left\{ K', K''\right\}  $, entonces si $\displaystyle x > K $ 
\[ g\left(x\right) > \frac{2C'}{3L} = C.\]
\end{proof}

\begin{eg}
\normalfont $\displaystyle \lim_{x \to \infty}\left(3x^{2}+x+1\right) $. Tenemos que 
\[\lim_{x \to \infty}\frac{3x^{2}+x+1}{x^{2}} = 3 > 0 .\]
Como $\displaystyle x^{2} \to \infty $, tenemos que $\displaystyle 3x^{2}+x+1 \to \infty$. 
\end{eg}

\begin{eg}
\normalfont Demostrar que existe $\displaystyle f : \left(-1,1\right) \to \R $ tal que $\displaystyle \exists \lim_{x \to 0^{+}}f\left(x\right) $ y no existe $\displaystyle \lim_{x \to 0^{-}}f\left(x\right) $. Consideremso 
\[f\left(x\right) = 
\begin{cases}
\sin\frac{1}{x}, \; x \in \left(-1,0\right) \\
0, \; x \in [0,1)
\end{cases}
.\]
Tenemos que el límite por la izquierda no existe pues podemos coger dos subsucesiones convergentes a 0 cuyas imágenes converjan a cosas distintas. Por ejemplo, 
\[x_{n} = \frac{-1}{\frac{\pi }{2}+2n\pi } \to 0 \quad \text{y} \quad f\left(x_{n}\right) = 1 \to 1 .\]
\[y_{n} = \frac{-1}{\frac{3\pi }{2} + 2n\pi } \to 0 \quad \text{y} \quad f\left(y_{n}\right) = - 1 \to - 1.\]
\end{eg}

\begin{eg}
\normalfont Sea $\displaystyle x_{n} > 0 $ con $\displaystyle x_{n} \to 0 $. Esta sucesión no tiene por qué ser decreciente pues puede dar saltitos. Considera la sucesión
\[x_{n} = \frac{1}{2}, \frac{2}{2}, \frac{1}{3}, \frac{2}{3}, \frac{1}{4}, \frac{2}{4}, \ldots.\]
\end{eg}

\begin{fprop}[]
\normalfont Tenemos que $\displaystyle x_{n} \to l  $ si y solo si la subsucesión de los pares y la de los impares convergen al mismo límite.
\end{fprop}

\begin{proof}
\begin{description}
\item[(i)] La primera implicación es trivial, pues si $\displaystyle \lim_{n \to \infty}x_{n} = l $ cualquier subsucesión converge al mismo límite.
\item[(ii)] Sea $\displaystyle \epsilon > 0 $, entonces existe $\displaystyle n_{1} \in \N $ tal que si $\displaystyle n \geq n_{1} $, $\displaystyle \left|x_{2n}-l\right|<\epsilon  $. Similarmente, existe $\displaystyle n_{2} \in \N $ tal que si $\displaystyle n \geq n_{2} $, $\displaystyle \left|x_{2n-1}-l\right|<\epsilon  $. Sea $\displaystyle n_{0} = \max \left\{ n_{1}, n_{2}\right\}  $. Si $\displaystyle  n\geq n_{0} $, 
	\[
	\begin{split}
	\left|x_{2n}-l\right|<\epsilon \\
	\left|x_{2n-1}-l\right|<\epsilon.
	\end{split}
	\]
	Entonces, $\displaystyle x_{n}\to l $.
\end{description}
\end{proof}
\begin{observation}
\normalfont Este criterio se puede extender a cualquier partición de $\displaystyle \N $.
\end{observation}

\begin{eg}
\normalfont Calcular $\displaystyle \sum^{\infty}_{j = 1}\frac{j}{2^{j}} $. Aplicando el criterio del cociente
\[\frac{j+1}{2^{n+1}}\frac{2^{n}}{j} \to \frac{1}{2} < 1 .\]
Por lo que converge. Tenemos que
\[
\begin{split}
\sum^{n}_{j=1}\frac{j}{2^{j}} = \sum^{n}_{j=1}\left(\sum^{j}_{k=1}1\right)\frac{1}{2^{j}} = \sum^{n}_{k=1}\left(\sum^{n}_{j=k}\frac{1}{2^{j}}\right) = \sum^{n}_{k=1}2\left(\frac{1}{2^{k}}-\frac{1}{2^{n+1}}\right)  = 2\sum^{n}_{k=1}\frac{1}{2^{k}}-2\frac{n}{2^{n+1}} \to 2.
\end{split}
\]

\end{eg}



