\documentclass{article}

% packages

\usepackage{graphicx} % Required for images
\usepackage[spanish]{babel}
\usepackage{mdframed}
\usepackage{amsthm}
\usepackage{amssymb}
\usepackage{fancyhdr}

% for math environments

\theoremstyle{definition}
\newtheorem{theorem}{Teorema}
\newtheorem{definition}{Definición}
\newtheorem{ej}{Ejercicio}
\newtheorem{sol}{Solución}

% for headers and footers

\pagestyle{fancy}

\fancyhead[R]{Victoria Eugenia Torroja}
% Store the title in a custom command
\newcommand{\mytitle}{}

% Redefine \title to store the title in \mytitle
\let\oldtitle\title
\renewcommand{\title}[1]{\oldtitle{#1}\renewcommand{\mytitle}{#1}}

% Set the center header to the title
\lhead{\mytitle}

% Custom commands

\newcommand{\R}{\mathbb{R}}
\newcommand{\C}{\mathbb{C}}
\newcommand{\F}{\mathbb{F}}




\begin{document}

\title{Análisis - Parcial 1 Solución}
\author{Victoria Eugenia Torroja Rubio}
\date{21/1/2025}

\maketitle

\begin{sol}
Visto en clase.
\end{sol}

\begin{sol}
\begin{description}
\item[(a)] Aplicamos el criterio del cociente para sucesiones
\[ \frac{x_{n+1}}{x_{n}} = \frac{\left(n+1\right)!}{3 \cdot 5 \cdots \left(2n+3\right)} \cdot \frac{3 \cdot 5 \cdots \left(2n+1\right)}{n!} = \frac{n+1}{2n+3} \to \frac{1}{2} < 1 .\]
Por tanto, la sucesión converge a 0.
\item[(b)] Tenemos que
\[
\begin{split}
	\left(\frac{n^{3}+n^{2}+1}{n^{3}+n}\right)^{\frac{\left(n+1\right)^{n}}{n^{n-1}}} = & \left(1 + \frac{n^{2}-n+1}{n^{3}+n}\right)^{\frac{\left(n+1\right)^{n}}{n^{n-1}}} = \left(1 + \frac{1}{\frac{n^{3}+n}{n^{2}-n+1}}\right)^{\frac{\left(n+1\right)^{n}}{n^{n-1}}}  \\ = & 
	\left[\left(1 + \frac{1}{\frac{n^{3}+n}{n^{2}-n+1}}\right)^{\frac{n^{3}+n}{n^{2}-n+1}}\right]^{\frac{\left(n+1\right)^{n}}{n^{n-1}} \cdot \frac{n^{2}-n+1}{n^{3}+n}} .
\end{split}
\]
Sea $\displaystyle C\left(n\right) = \left(1 + \frac{1}{\frac{n^{3}+n}{n^{2}-n+1}}\right)^{\frac{n^{3}+n}{n^{2}-n+1}} $. Entonces tenemos que 
\[x_{n} = C\left(n\right)^{\frac{\left(n+1\right)^{n}}{n^{n-1}} \cdot \frac{n^{2}-n+1}{n^{3}+n}} .\]
Tenemos que $\displaystyle C\left(n\right) \to e $ y
\[
\begin{split}
	\frac{\left(n+1\right)^{n}}{n^{n-1}} \cdot \frac{n^{2}-n+1}{n^{3}+n} = & \frac{1}{\left(1 + \frac{1}{n}\right)^{n}} \cdot \frac{n^{3}-n^{2} +1}{n^{3}+n} \to e \cdot 1 = e	.
\end{split}
\]
Así, $\displaystyle x_{n} \to e^{e} $.
\end{description}
\end{sol}

\begin{sol}
\begin{description}
\item[(a)] Tenemos que 
	\[\lim_{x \to x_{0}}f\left(x\right) = \lim_{x \to x_{0}} \frac{f\left(x\right)}{g\left(x\right)}g\left(x\right) = \lim_{x \to x_{0}}\frac{f\left(x\right)}{g\left(x\right)}\lim_{x \to x_{0}}g\left(x\right) = 1 \cdot 0 = 0 .\]
\item[(b)] Consideremos la función
\[f\left(x\right) =
\begin{cases}
\sin\left(\frac{1}{x}\right), \; x \neq 0 \\
0, \; x = 0
\end{cases}
.\]
Existen sucesiones $\displaystyle x_{n} = \frac{1}{\frac{\pi }{2} + 2\pi n} $ e $\displaystyle y_{n} = \frac{1}{\pi n} $, que convergen a $\displaystyle 0 $ y son estrictamente positivas, pero sus imágenes convergen a cosas distintas.
\item[(c)] 
\[\lim_{x \to \infty}\left(\sqrt{x + \sqrt{x}}-\sqrt{x}\right) = \lim_{x \to \infty}\frac{\sqrt{x}}{\sqrt{x + \sqrt{x}}+\sqrt{x}} = \lim_{x \to \infty} \frac{1}{\sqrt{1+\frac{1}{\sqrt{x}}}+1} = \frac{1}{2} .\]
\end{description}
\end{sol}

\begin{sol}
\begin{description}
\item[(a)] Visto en clase.
\item[(b)] 
	\begin{description}
	\item[(i)] Vamos a ver que converge absolutamente. Tenemos que
	\[ \left|\frac{\left(-1\right)^{n}4^{2n}}{\left(2n\right)!} + \frac{1}{n\left(\log\left(n\right)\right)^{2}}\right| \leq \frac{4^{2n}}{\left(2n\right)!} + \frac{1}{n\left(\log\left(n\right)\right)^{2}}.\]
Para el primero, aplicamos la regla del cociente.
\[\frac{4^{2n+2}}{\left(2n+2\right)!}\frac{\left(2n\right)!}{4^{2n}} = \frac{16}{\left(2n+1\right)\left(2n+2\right)}\to 0 < 1 .\]
Por tanto converge. Para la segunda aplicamos el criterio de condensación de Cauchy.
\[\sum^{\infty}_{n = 2}\frac{1}{n\left(\log\left(n\right)\right)^{2}} \approx \sum^{\infty}_{n=2}\frac{2^{n}}{2^{n}\left(n\log2\right)^{2}} \approx \sum^{\infty}_{n=2}\frac{1}{n^{2}} < \infty .\]
Por tanto, la segunda serie converge. Por el criterio de comparación, la serie original converge absolutamente, por lo que converge.
\item[(ii)] Vamos a ver que no converge absolutamente.
\[
\begin{split}
\left|\frac{\cos\left(n\pi \right)n^{n-1}}{\left(n+1\right)^{n}}\right| = \frac{n^{n-1}}{\left(n+1\right)^{n}} = \frac{1}{\left(1+\frac{1}{n}\right)^{n}} \frac{1}{n} \geq  \frac{1}{3n} .
\end{split}
\]
Así, la serie no converge absolutamente. Vamos a ver si converge. Para ello aplicamos el criterio de Leibniz, que nos dice que sí converge. Podemos usar el criterio de Leibniz puesto que $\displaystyle \sum^{n}_{j = 1}\left(-1\right)^{j} $ está acotada y el otro término es estrictamente decreciente y converge a 0.
	\end{description}
\end{description}
\end{sol}
\end{document}
