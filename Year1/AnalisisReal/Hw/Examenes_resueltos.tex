\documentclass{article}

% packages

\usepackage{graphicx} % Required for images
\usepackage[spanish]{babel}
\usepackage{mdframed}
\usepackage{amsthm}
\usepackage{amssymb}
\usepackage{fancyhdr}
\usepackage{amsmath}
\usepackage{geometry}[margin=1in]
\usepackage{pgfplots}
\usepackage{url}
\usepackage{float}

% for math environments

\theoremstyle{definition}
\newtheorem*{theorem}{Teorema}
\newtheorem*{definition}{Definición}
\newtheorem*{prop}{Proposición}
\newtheorem*{observation}{Observación}
\newtheorem{ej}{Ejercicio}
\newtheorem{sol}{Solución}

% for headers and footers

\pagestyle{fancy}

%\fancyhead[R]{Victoria Eugenia Torroja}
% Store the title in a custom command
\newcommand{\mytitle}{}

% Redefine \title to store the title in \mytitle
\let\oldtitle\title
\renewcommand{\title}[1]{\oldtitle{#1}\renewcommand{\mytitle}{#1}}

% Set the center header to the title
\lhead{\mytitle}

% Custom commands

\newcommand{\R}{\mathbb{R}}
\newcommand{\C}{\mathbb{C}}
\newcommand{\F}{\mathbb{F}}
\newcommand{\N}{\mathbb{N}}
\newcommand{\Q}{\mathbb{Q}}
\newcommand{\Z}{\mathbb{Z}}
\newcommand{\K}{\mathbb{K}}
\newcommand{\mcd}{\text{mcd}}
\newcommand{\mcm}{\text{mcm}}
\DeclareMathOperator{\Ker}{Ker}
\DeclareMathOperator{\Imagen}{Im}
\DeclareMathOperator{\ord}{ord}
\DeclareMathOperator{\GL}{GL}
\DeclareMathOperator{\Biy}{Biy}


\begin{document}

\title{Exámenes pasados}
\author{Victoria Eugenia Torroja Rubio}
\date{\today}

\maketitle

\section*{Mayo 2024}
\begin{ej}
Sea $\displaystyle f\left(x\right) = \sqrt{1-x} $. Determinar para qué puntos del dominio de $\displaystyle f $ se cumple que $\displaystyle \left|f\left(x\right)-1\right| \geq \frac{1}{2} $.
\end{ej}
\begin{sol}
Se puede hacer de dos formas:
\begin{itemize}
\item Resolver la inecuación $\displaystyle \left|\sqrt{1-x}-1\right| \geq \frac{1}{2} $. 
\item Pintar la función y verlo gráficamente.
\end{itemize}
Lo hacemos de la segunda manera. Primero calculamos el dominio de $\displaystyle f $: $\displaystyle (-\infty, 1] $. Por otro lado, tenemos que 
\[f'\left(x\right) = \frac{-1}{2\sqrt{1-x}} < 0 .\]
Por tanto, es decreciente. Además, es inyectiva. Tenemos que $\displaystyle \lim_{x \to -\infty}f\left(x\right) = \infty $ y $\displaystyle f\left(1\right) = 0 $. También sabemos que la función es continua en todo su dominio. Así, podemos pintar la función y podemos ver que 
\[ \left\{ x \; : \; \left|f\left(x\right)-1\right| \geq \frac{1}{2}\right\} = (-\infty, f^{-1}\left(\frac{3}{2}\right)] \cup [f^{-1}\left(\frac{1}{2}\right), \infty).\]
Como $\displaystyle f^{-1}\left(\frac{3}{2}\right) = -\frac{5}{4} $ y $\displaystyle f^{-1}\left(\frac{1}{2}\right) = \frac{3}{4} $, tenemos que la solución es el conjunto $\displaystyle \left(-\infty, -\frac{5}{4}\right] \cup \left[\frac{3}{4}, \infty\right) $.
\end{sol}
\begin{ej}
	Sean $\displaystyle f, g : \left[a,b\right] \to \R $ dos funciones continuas y derivables en $\displaystyle \left[a,b\right] / \left\{ s\right\}  $ con $\displaystyle s \in \left(a,b\right) $, con $\displaystyle f'\left(x\right) \neq 0 $, $\displaystyle \forall x \in \left[a,b\right] / \left\{ s\right\}  $. Si existe $\displaystyle \lim_{x \to s^{-}}f\left(x\right) = \lim_{x \to s^{-}}g\left(x\right) = 0 $ y existe
	\[\lim_{x \to s^{-}}\frac{g'\left(x\right)}{f'\left(x\right)} ,\]
	entonces se verifica que 
	\[ \lim_{x \to s^{-}}\frac{g\left(x\right)}{f\left(x\right)} = \lim_{x \to s^{-}}\frac{g'\left(x\right)}{f'\left(x\right)} .\]
\end{ej}
\begin{sol}
	Cogemos $\displaystyle f, g : [a,s) \to \R $. Definimos $\displaystyle f\left(s\right) = g\left(s\right) = \lim_{x \to s^{-}}f\left(x\right) = \lim_{x \to s^{-}}g\left(x\right) = 0 $. Así, tenemos que $\displaystyle f $ y $\displaystyle g $ son continuas en $\displaystyle \left[a,s\right]  $ y derivables en $\displaystyle \left(a,s\right) $. Además, sabemos que $\displaystyle f'\left(x\right) \neq 0 $. Por el teorema del valor medio de Cauchy, tenemos que $\displaystyle \forall x \in \left(a,s\right) $, $\displaystyle \exists \xi \in (x,s) $ tal que 
	\[\frac{g\left(x\right)-g\left(s\right)}{f\left(x\right)-f\left(s\right)} = \frac{g'\left(\xi\right)}{f'\left(\xi\right)} .\]
Además, como $\displaystyle \lim_{x \to s^{-}}\frac{g'\left(x\right)}{f'\left(x\right)} = l $, tenemos que $\displaystyle \forall \epsilon > 0 $, $\displaystyle \exists \delta > 0 $ tal que si $\displaystyle 0 < \delta - x < \delta  $, entonces $\displaystyle \left|\frac{g'\left(x\right)}{f'\left(x\right)}-l\right| < \epsilon $. Así, como $\displaystyle \xi \in \left(x,s\right) $ tenemos que 
\[ \left|\frac{g\left(x\right)}{f\left(x\right)}-l\right| = \left|\frac{g\left(x\right)-g\left(s\right)}{f\left(x\right)-f\left(s\right)}-l\right| = \left|\frac{g'\left(\xi\right)}{f'\left(\xi\right)}-l\right| < \epsilon .\]
\end{sol}
\begin{ej}
Calcula  
\[ \lim_{x \to 1^{-}}\left(1-x\right)^{\frac{1}{\sqrt{-\ln\left(1-x\right)}}}.\]
\end{ej}
\begin{sol} Se trata de una indeterminación de la forma $\displaystyle 0^{0} $. Dado que el logaritmo neperiano es una función continua y monótona, el límite existe si y solo si existe el límite
\[
\begin{split}
	\lim_{x \to 1^{-}}\ln\left(\left(1-x\right)^{\frac{1}{\sqrt{-\ln\left(1-x\right)}}}\right) = & \lim_{x \to 1^{-}}\frac{\ln \left(1-x\right)}{\sqrt{-\ln\left(1-x\right)}} = \lim_{x \to 1^{-}}\frac{- \left(-\ln\left(1-x\right)\right)}{\sqrt{-\left(1-x\right)}} = -\sqrt{-\ln\left(1-x\right)} = -\infty.
\end{split}
\]
Así, el límite original vale 0.
\end{sol}
\begin{ej}
	Encuentra una función $\displaystyle \varphi:\left[0,1\right] \to \R $ derivable en todo $\displaystyle \left[0,1\right]  $ y tal que $\displaystyle \varphi\left(0\right)= 0 $ y $\displaystyle \forall x \in \left(\epsilon , 1\right) $, para cierto $\displaystyle \epsilon > 0 $, $\displaystyle \varphi\left(x\right) = 1 $ y además $\displaystyle \int^{1}_{0} \varphi\left(x\right) \; dx < 1 $.
\end{ej}
\begin{sol}
	Necesitamos una función continua en $\displaystyle \left[0,\epsilon \right]  $ que cumpla que $\displaystyle \varphi\left(\epsilon \right) = 1 $ y $\displaystyle \varphi'\left(\epsilon \right) = 0 $. Así, podemos coger por ejemplo una parábola de la forma
	\[ \varphi\left(x\right) = k\left(x-\epsilon \right)^{2} + 1 .\]
	Para que $\displaystyle \varphi\left(0\right) = 0 $, cogemos $\displaystyle k = -\frac{1}{\epsilon^{2}} $. Así, hemos construido la función
	\[\varphi\left(x\right) =
	\begin{cases}
		-\frac{1}{\epsilon ^{2}}\left(x-\epsilon \right)^{2}+1, \; x\in [0, \epsilon] \\
		1, \; x \in \left[\epsilon ,1\right] 
	\end{cases}
	.\]
Ahora falta ver que la integral es menor a uno. En efecto, tenemos que
\[\int^{1}_{0} \varphi\left(x\right) \; dx = \int^{\epsilon }_{0} \varphi\left(x\right) \; dx + \int^{1}_{\epsilon } \varphi\left(x\right) \; dx < \int^{\epsilon }_{0} 1 \; dx + \int^{1}_{\epsilon } 1 \; dx = 1 .\]
\end{sol}
\begin{ej}
Calcular la integral
\[\int \frac{1}{x\ln ^{2}x - x \ln x^{2} + x} \; dx .\]
\end{ej}
\begin{sol}
\[
\begin{split}
\int \frac{1}{x\ln ^{2}x - x \ln x^{2} + x} \; dx = \int \frac{1}{x\ln ^{2}x - 2\ln x + x} \; dx = \int \frac{1}{x\left(\ln x - 1\right)^{2}} \; dx
\end{split}
\]
Así, cogiendo $\displaystyle u = \left(\ln x-1\right)^{2} $, $\displaystyle du = \frac{1}{x}dx $,
\[ = \int \frac{1}{u^{2}} \; du = -\frac{1}{u} + C = -\frac{1}{\ln x - 1} + C = \frac{1}{1-\ln x} + C.\]
\end{sol}
\begin{ej}
Demostrar que $\displaystyle \ln x \leq \frac{1}{e}\left(x-e\right)+1 $, $\displaystyle x > 0 $.
\end{ej}
\begin{sol}
Tenemos que la recta del lado derecho de la desigualdad es la recta tangente de la función $\displaystyle \ln x $ en $\displaystyle x = e $. Una forma de resolver el ejercicio es demostrar que si una función es cóncava, cualquier recta tangente a la función queda por encima de la función. Otra forma de hacerlo es estudiando la función $\displaystyle g\left(x\right) = \frac{1}{e}\left(x-e\right)+1-\ln x $ y viendo que $\displaystyle g\left(x\right) \geq 0 $, $\displaystyle x > 0 $. Hacemos la segunda opción:
\[g'\left(x\right) = \frac{1}{e} - \frac{1}{x} =
\begin{cases}
< 0, \; x > e \\
0, \; x = e \\
> 0, \; x < e
\end{cases}
.\]
Tenemos que $\displaystyle g\left(e\right) = 0 $, que es un mínimo. Así, $\displaystyle \forall x > 0 $ se tiene que $\displaystyle g\left(x\right) \geq 0 $.
\end{sol}
\begin{ej}
Consideramos una función $\displaystyle f : [0,\infty) \to (0,\infty) $ continua en $\displaystyle \left(0,\infty\right) $ y tal que $\displaystyle \forall n \in \N $, 
\[\int^{\frac{1}{n}}_{\frac{1}{n+1}} f\left(x\right) \; dx \quad \text{y} \quad \int^{n + \frac{1}{n}}_{n} f\left(x\right) \; dx = \frac{1}{n} .\]
Estudiar si existen $\displaystyle \int^{1}_{0} f\left(x\right) \; dx $ y $\displaystyle \int^{\infty}_{0} f\left(x\right) \; dx $.
\end{ej}
\begin{sol}
Partimos el ejercicio en dos partes. 
\begin{itemize}
	\item Primero estudiamos $\displaystyle \int^{1}_{0} f\left(x\right) \; dx $. Sabemos que $\displaystyle \forall y \in (0,1] $, $\displaystyle \exists \int^{1}_{y} f\left(x\right) \; dx $. Así, tenemos que ver si existe el límite
		\[\lim_{y \to 0^{+}}\int^{1}_{y} f\left(x\right) \; dx .\]
Consideremos $\displaystyle F\left(y\right) = \int^{1}_{y} f\left(x\right) \; dx $. Por el teorema fundamental del cálculo, tenemos que $\displaystyle F'\left(y\right) = -f\left(y\right) < 0  $.
Por tanto, $\displaystyle F $ es estrictamente decreciente. Por ser monótona, tenemos que 
\[\lim_{y \to 0^{+}}F\left(y\right) = \lim_{n \to \infty}\int^{1}_{\frac{1}{n}} f\left(x\right) \; dx = \lim_{n \to \infty}\sum^{n-1}_{k = 1}\int^{\frac{1}{k}}_{\frac{1}{k+1}} f\left(x\right) \; dx = \lim_{n \to \infty}\sum^{n-1}_{k = 1}\frac{1}{n^{2}} = \sum^{\infty}_{n = 1}\frac{1}{n^{2}} < \infty.\]
Por tanto, la integral $\displaystyle \int^{1}_{0} f\left(x\right) \; dx $ existe.
\item Para estudiar la convergencia de $\displaystyle \int^{\infty}_{0} f\left(x\right) \; dx $ hacemos algo similar. Como sabemos que existe $\displaystyle \int^{1}_{0} f\left(x\right) \; dx $, para ver que converge nuestra integral de interés basta con que estudiemos la convergencia de $\displaystyle \int^{\infty}_{1} f\left(x\right) \; dx $.
	Consideremos ahora $\displaystyle F\left(y\right) = \int^{y}_{1} f\left(x\right) \; dx $. Por el TFC tenemos que $\displaystyle F'\left(y\right) = f\left(y\right) > 0 $, por lo que es monótona, por lo que
\[
\begin{split}
	\lim_{y \to \infty}F\left(y\right) = & \lim_{y \to \infty}\int^{y}_{1} f\left(x\right) \; dx = \lim_{n \to \infty}\int^{n}_{1} f\left(x\right) \; dx = \lim_{n \to \infty}\sum^{n-1}_{k = 1}\int^{n + 1}_{n} f\left(x\right) \; dx  \\
	= & \lim_{n \to \infty}\sum^{n-1}_{k = 0}\int^{n + \frac{1}{n}}_{n} f\left(x\right) \; dx = \sum^{\infty}_{n = 1}\frac{1}{n} = \infty.
\end{split}
\]
Así, como no converge la integral $\displaystyle \int^{\infty}_{1} f\left(x\right) \; dx $, no converge $\displaystyle \int^{\infty}_{0} f\left(x\right) \; dx $.
\end{itemize}
\end{sol}
\begin{ej}
Consideremos la serie de potencias $\displaystyle \sum^{\infty}_{n = 1}r^{n}\left(x-a\right)^{n} $, $\displaystyle 0 < r < a < \frac{1}{2} $.
\begin{description}
\item[(a)] Calcular el radio de convergencia.
\item[(b)] Calcular el límite puntual.
\item[(c)] Deducir si converge uniformemente en $\displaystyle \left[-1,1\right]  $.
\end{description}
\end{ej}
\begin{sol}
\begin{description}
\item[(a)] Aplicamos el criterio del cociente:
	\[\lim_{n \to \infty}\frac{r^{n+1} \left|x-a\right|^{n+1}}{r^{n} \left|x-a\right|^{n}} = r \left|x-a\right| .\]
	Así, como $\displaystyle r \left|x-a\right| < 1 \iff \left|x-a\right| < \frac{1}{r} $, tenemos que el radio de convergencia es $\displaystyle \frac{1}{r} $.
\item[(b)] Al tratarse de una serie geométrica, si $\displaystyle x \in \left(a - \frac{1}{r}, a + \frac{1}{r}\right) $:
	\[\sum^{\infty}_{n = 1}r^{n}\left(x-a\right)^{n} = \frac{r\left(x-a\right)}{1 - r\left(x-a\right)} .\]
En $\displaystyle x = a $ la serie vale 0.
\item[(c)] Tenemos que $\displaystyle r < \frac{1}{2} $, por lo que $\displaystyle 2 < \frac{1}{r} $ y $\displaystyle -\frac{1}{r} < - 2 $. Así, $\displaystyle a + \frac{1}{r} > a + 2 > 2 $ y $\displaystyle a - \frac{1}{r} < a - 2 < -1 $. Así, tenemos que $\displaystyle \left[-1,1\right] \subset \left(a-\frac{1}{r}, a +\frac{1}{r}\right) $, por lo que la serie converge uniformemente en $\displaystyle \left[-1,1\right]  $.
\end{description}

\end{sol}
 
\end{document}
