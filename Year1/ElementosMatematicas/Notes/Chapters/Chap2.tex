\chapter{Grupos de simetrías. Mosaicos.}

\textbf{Objetivo:} Estudiar y clasificar todas las formas posibles de cubrir el plano con copias idénticas de una 'ficha' o 'pieza' (tesela) de manera períodica: mosaicos. \\ \\
\textbf{Herramientas:} Estudio de los movimientos en el plano que preservan la teselación. \\ \\
\textbf{Elementos matemáticos:} 
\begin{itemize}
\item Grupo - Subgrupo
\item Grupo de movimento en el plano
\item Simetrías
\end{itemize}
\textbf{Aplicaciones:}
\begin{itemize}
\item Retículos
\item 17 grupos cristalográficos planos
\end{itemize}

\section{Grupos. Isomorfismos y homomorfismos de grupos.}

\begin{fdefinition}[Grupo]
\normalfont Se define como \textbf{grupo} al par ordenado $\displaystyle \left(G, \cdot\right) $ formado por un conjutno $\displaystyle G $ y una operación interna
\[
\begin{split}
	\cdot : G \times G & \to G \\
\left(a,b\right) & \to a \cdot b .
\end{split}
\]
tales que
\begin{description}
\item[(1)] La operación es asociativa. 
	\[\forall a, b, c \in G, \; \left(a \cdot b\right) \cdot c = a \cdot \left(b \cdot c\right) .\]
\item[(2)] Existe un elemento neutro.
	\[\exists e \in G, \forall a \in G, \; e \cdot a = a \cdot e = a .\]
\item[(3)] Existe el inverso. 
	\[\forall a \in G, \exists a^{-1} \in G\; a \cdot a^{-1} = a^{-1} \cdot a = e .\]
Si se cumple también la propiedad conmutativa:
\[\forall a, b \in G, \; a \cdot b = b \cdot a ,\]
se denomina grupo conmutativo o \textbf{abeliano}.
\end{description}
\end{fdefinition}

\begin{eg}
\normalfont 
\begin{description}
\item[(1)] $\displaystyle \left(\Z, +\right) $ es un grupo abeliano. 
\item[(2)] $\displaystyle \left(\Q, \cdot\right) $ no es un grupo pues $\displaystyle 0 \in \Q $ no tiene inverso. 
\item[(3)] $\displaystyle \left(\Q^{*}, \cdot\right) $ es un grupo abeliano. 
\item[(4)] Tenemos que $\displaystyle \left(2\Z+1, +\right) $ no es un grupo, pues la suma de dos números impares es un número par. Similarmente, $\displaystyle \left(2\Z+1, \cdot \right) $ no es un grupo por la ausencia de inversos. 
\item[(5)] El conjunto de las matrices $\displaystyle 2 \times 2 $ es un grupo abeliano con la suma. $\displaystyle \left(M_{2\times2}, +\right) $ es un grupo abeliano. 
\item[(6)] El conjunto de las matrices $\displaystyle n \times n $ con $\displaystyle \det \neq 0 $ , $\displaystyle \left(M_{n\times n}, \cdot\right) $, es un grupo abeliano. En concreto, en el caso de las matrices $\displaystyle 2 \times 2 $, recibe el nombre de \textbf{grupo general lineal de orden 2} y se denota como $\displaystyle \GL\left(2, \R\right) $ o $\displaystyle \GL\left(2\right) $. 
\item[(7)] El conjunto de matrices $\displaystyle 2 \times 2 $, $\displaystyle \left(M_{2\times2}, \cdot\right) $, con $\displaystyle \det = 1 $ es un grupo y se denomina \textbf{grupo especial lineal}: $\displaystyle \SL\left(2\right) $.
\item[(8)] El conjunto de matrices ortogonales ($\displaystyle A^{T} = A^{-1} $) es un grupo al que se denomina \textbf{grupo ortogonal}: $\displaystyle \text{O}\left(2\right) $.
\item[(9)] Las matrices ortogonales con $\displaystyle \det = 1 $ se denomina \textbf{grupo especial ortogonal}: $\displaystyle \SO\left(n\right) $.
\item[(10)] $\displaystyle \left(\Z_{n}, +\right) $ con $\displaystyle n \in \N $ forma un grupo abeliano. 
\item[(11)] Si $\displaystyle p $ es primo, $\displaystyle \left(\Z_{p}, \cdot \right) $ también es un grupo abeliano. 
\end{description}
\end{eg}

\begin{fprop}\normalfont Sea $\displaystyle \left(G, \cdot \right) $ un grupo.
\begin{description}
\item[(1)] Propiedades cancelativas por la derecha y por la izquierda de un grupo:
\[
\begin{split}
& a \cdot c = b \cdot c \Rightarrow a = b \\
& c \cdot a = c \cdot b \Rightarrow a = b.
\end{split}
\]
\item[(2)] El elemento neutro de un grupo es único.
\item[(3)] Cada elemento de un grupo tiene un único inverso. 
\item[(4)] Si $\displaystyle a \in G $, entonces $\displaystyle \left(a^{-1}\right)^{-1} = a $.
\item[(5)] Si $\displaystyle a, b \in G $, $\displaystyle \left(a \cdot b\right)^{-1} = b^{-1} \cdot a ^{-1} $. 
\item[(6)] Son equivalentes:
	\begin{description}
	\item[(a)] $\displaystyle \left(G, \cdot\right) $ es un grupo abeliano. 
	\item[(b)] $\displaystyle \left(a \cdot b\right)^{-1} = a^{-1} \cdot b^{-1}, \forall a, b \in G $.
	\end{description}
\end{description}
\end{fprop}

\begin{proof}
\begin{description}
\item[(1)] Como $\displaystyle \left(G, \cdot\right) $ es un grupo, tenemos que si $\displaystyle a, b, c \in G $, existe $\displaystyle c^{-1} \in G $ tal que 
	\[a \cdot c = b \cdot c \iff a \cdot c \cdot c^{-1} = b \cdot c \cdot c^{-1} \iff a \cdot e = b \cdot e \iff a = b .\]
Similarmente, 
\[c \cdot a = c \cdot b \iff c^{-1} \cdot c \cdot a = c^{-1} \cdot c \cdot b \iff e \cdot a = e \cdot b \iff a = b .\]
\item[(2)] Sean $\displaystyle e, e' \in G $ dos elementos neutros de $\displaystyle \left(G, \cdot\right) $. Entonces, tenemos que 
	\[ e = e \cdot e' = e' \cdot e = e' .\]
\item[(3)] Si $\displaystyle a \in G $, sean $\displaystyle b, c \in G $ dos inversos de $\displaystyle a $. Entonces tenemos que 
	\[a \cdot b = a \cdot c = e .\]
Por \textbf{(1)} tenemos que $\displaystyle b = c $. Otra manera de demostrarlo es:
\[b = b \cdot e = b \cdot \left(a \cdot c\right) = \left(b \cdot a \right) \cdot c = e \cdot c = c .\]
\item[(4)] 
	\[\left(a^{-1}\right)^{-1} \cdot a^{-1} = e .\]
Como el inverso de $\displaystyle a^{-1} $ es $\displaystyle a $ y el inverso es único, tenemos que $\displaystyle a = \left(a^{-1}\right)^{-1} $. 
\item[(5)] Tenemos que 
	\[\left(a \cdot b\right) \cdot \left(b^{-1} \cdot a^{-1}\right) = a \cdot \left(b^{-1} \cdot b\right) \cdot a^{-1} = a \cdot e \cdot a^{-1} = a \cdot a^{-1} = e .\]
	Como el inverso de $\displaystyle a \cdot b $ es $\displaystyle \left(a \cdot b\right)^{-1} $, y el inverso es único, tenemos que $\displaystyle \left(a \cdot b\right)^{-1} = b^{-1} \cdot a^{-1} $.
\item[(6)] Si es un grupo abeliano, se cumple la propiedad conmutativa y con el resultado \textbf{(5)} la demostración es trivial. Recíprocamente, si $\displaystyle \forall a, b \in G $ tenemos que 
	\[\left(a \cdot b\right) ^{-1} = b^{-1} \cdot a^{-1} = a^{-1} \cdot b^{-1} ,\]
tenemos que 
\[ \left(a \cdot b\right) \cdot \left(a \cdot b\right)^{-1} = \left( b \cdot a\right) \cdot\left(a \cdot b\right)^{-1} .\]
Por la propiedad \textbf{(1)} tenemos que $\displaystyle a \cdot b = b \cdot a $, por lo que se cumple la propiedad conmutativa y es un grupo abeliano. 
\end{description}
\end{proof}

\section{Subgrupo. Grupos finitos. Orden}

\begin{fdefinition}[]
\normalfont Sea $\displaystyle \left(G, \cdot\right) $ un grupo y sea $\displaystyle H\subset G $. Se dice que $\displaystyle H $ es un subgrupo de $\displaystyle G $, y se escribe $\displaystyle H \leq G $, si $\displaystyle \left(H, \cdot\right) $ es un grupo.
\end{fdefinition}

\begin{eg}
\normalfont 
\begin{description}
\item[(i)] Tenemos que $\displaystyle \left(2\Z, + \right) \leq \left(\Z, +\right) \leq \left(\Q, +\right) \leq \left(\R, +\right) \leq \left(\C, +\right) $. 
\item[(ii)] $\displaystyle \SO\left(n\right) \leq \SL\left(n\right) \leq \GL\left(n\right) $.
\end{description}
\end{eg}

\begin{fdefinition}[]
\normalfont Si $\displaystyle x \in G $ con $\displaystyle \left(G, \cdot \right) $ grupo, $\displaystyle n \in \Z $, denotamos a 
\[x^{n} =
\begin{cases}
	\underbrace{x \cdots x}_{n \; \text{veces}} \; \text{si} \; n > 0\\
	e \; \text{si} \; n = 0 \\
	\underbrace{x^{-1} \cdots x^{-1}}_{n \; \text{veces}} \; \text{si} \; n < 0
\end{cases}
.\]
El conjunto $\displaystyle \langle x \rangle = \left\{ x^{n} \; : \; n \in \Z\right\}  $ es un subgrupo abeliano de $\displaystyle G $. Es un subgrupo generado por el elemento $\displaystyle x $. Si $\displaystyle \exists x \in G $, $\displaystyle G = \langle x \rangle $ se denomina \textbf{grupo cíclico} . 
\end{fdefinition}

\begin{fdefinition}[Orden]
\normalfont Si $\displaystyle G $ es un grupo finito, se llama \textbf{orden} de $\displaystyle G $ a $\displaystyle \left|G\right| $, es decir, al número de elementos de $\displaystyle G $. Si $\displaystyle G $ no es un conjunto finito diremos que tiene \textbf{orden infinito}. Si $\displaystyle x \in G $, el orden de $\displaystyle x $ es: $\displaystyle \ord\left(x\right) = \left|\langle x \rangle\right| $.
\end{fdefinition}

\begin{observation}
\normalfont El orden de un elemento $\displaystyle x $, si es finito, es el menor entero positivo $\displaystyle n $  tal que $\displaystyle x^{n} = e $.
\end{observation}

\begin{fprop}[]
\normalfont Sea $\displaystyle H $ un subconjunto no vacío de un grupo $\displaystyle \left(G, \cdot\right) $. Entonces, $\displaystyle H \leq G $ si y solo si $\displaystyle \forall x,y\in H $ se tiene que $\displaystyle x \cdot y^{-1} \in H $.
\end{fprop}

\begin{proof}
\begin{description}
\item[(i)] Es trivial, pues si $\displaystyle x, y \in H $, como $\displaystyle H \leq G $, tenemos que $\displaystyle \exists y^{-1} \in H $ y $\displaystyle x \cdot y^{-1} \in H $.
\item[(ii)] Recíprocamente, tenemos que la operación interna está cerrada. La propiedad asociativa se cumple porque se $\displaystyle \left(G, \cdot\right) $ es un grupo. Por definición, tenemos que si $\displaystyle x, y \in H $ entonces, $\displaystyle x \cdot y^{-1} \in H $. Podemos coger $\displaystyle x \cdot x^{-1} = e \in H $. Así, $\displaystyle H $ contiene al elemento neutro. Similarmente, como $\displaystyle e,x \in H $, tenemos que $\displaystyle e \cdot x^{-1} = x^{-1} \in H $. Por lo que podemos encontrar inversos para todos los elementos de $\displaystyle H $. 
\end{description}
\end{proof}

\begin{fprop}[]
\normalfont Si $\displaystyle H_{1} \leq G $ y $\displaystyle H_{2} \leq G $, entonces $\displaystyle H_{1}\cap H_{2} \leq G $.
\end{fprop}

\begin{proof}
Tenemos que ver que $\displaystyle \forall x, y \in H_{1} \cap H_{2} \Rightarrow x \cdot y^{-1} \in H_{1} \cap H_{2} $. Si $\displaystyle x, y \in H_{1} \cap H_{2} $, tenemos que $\displaystyle x,y \in H_{1} $ y $\displaystyle x,y \in H_{2} $. Así, como $\displaystyle H_{1}, H_{2} \leq G $, tenemos que $\displaystyle y^{-1} \in H_{1} \cap H_{2} $ y $\displaystyle x \cdot y^{-1} \in H_{1}\cap H_{2} $.
\end{proof}

\begin{fprop}[]
\normalfont Si $\displaystyle \left(G, \cdot\right) $ es un grupo finito y $\displaystyle x \in G $, el orden de $\displaystyle x $ coincide con el menor entero positivo $\displaystyle k $ tal que $\displaystyle x^{k} = e $. Además, 
\[\langle x \rangle = \left\{ x, x^{2}, x^{3}, \ldots, x^{k} = e\right\}  .\]
\end{fprop}

\begin{eg}
	\normalfont Consideramos el grupo $\displaystyle \left(\Z_{6}, +\right) = \left\{ 0, 1, 2, 3, 4, 5\right\} $. Entonces, tenemos que 
	\[
	\begin{split}
		\langle 0 \rangle & = \left\{ 0\right\} \Rightarrow \ord\left(0\right) = 1\\
		\langle 1 \rangle & = \left\{ 1, 2, 3, 4, 5, 0\right\} \Rightarrow \ord\left(1\right) = 6 \\
		\langle 2 \rangle & = \left\{ 2, 4, 0\right\} \Rightarrow \ord\left(2\right) = 3 \\
		\langle 3 \rangle & = \left\{ 3, 0\right\} \Rightarrow \ord\left(3\right) = 2 \\
		\langle 4 \rangle & = \left\{ 4, 2, 0\right\} \Rightarrow \ord\left(4\right) = 3\\
		\langle 5 \rangle & = \left\{ 5, 4, 3, 2, 1, 0\right\} \Rightarrow \ord\left(5\right) = 6 .
	\end{split}
	\]
\end{eg}

\section{Isomorfismos de grupos}
\begin{fdefinition}[Isomorfismo]
\normalfont Sean $\displaystyle \left(G, \cdot \right), \left(G', *\right) $ grupos. Se dice que $\displaystyle f: G \to G' $ es un \textbf{isomorfismo} (de grupos) si es biyectiva y 
\[\forall x, y \in G, \; f\left(x \cdot y\right) = f\left(x\right) * f\left(y\right) .\]
Si $\displaystyle G $ es isomorfo a $\displaystyle G' $, lo escribirmos $\displaystyle G \cong G' $.
\end{fdefinition}
\begin{eg}
\normalfont $\displaystyle f: \left(\R, +\right) \to \left(\R^{+}, \cdot \right) $ con $\displaystyle f : x \to e^{x} $.
\end{eg}

\begin{fprop}[]
\normalfont Si $\displaystyle G \cong G' $, entonces
\begin{description}
\item[(a)] $\displaystyle G $ y $\displaystyle G' $ tienen el mismo cardinal.
\item[(b)] Si $\displaystyle e $ y $\displaystyle e' $ son, respectivamente, los elementos neutros de $\displaystyle G $ y $\displaystyle G' $,
	\[f\left(e\right) = e' \quad \text{y} \quad f^{-1}\left(e'\right) = e .\]
Además, $\displaystyle f\left(x^{-1}\right) = \left(f\left(x\right)\right)^{-1} $.
\item[(c)] Si $\displaystyle H \leq G $, $\displaystyle f\left(H\right) \leq G'$.
\item[(d)] Si $\displaystyle x \in G $, $\displaystyle \ord\left(x\right) = \ord\left(f\left(x\right)\right) $.
\item[(e)] Si $\displaystyle G $ es abeliano, entonces $\displaystyle G' $ también lo es.
\end{description}
\end{fprop}

\begin{proof}
\begin{description}
\item[(a)] Es trivial, puesto que tiene que un isomorfismo es biyectivo por definición.
\item[(b)] Tenemos que $\displaystyle \forall x \in G $, 
	\[f\left(x\right) = f\left(x \cdot e\right) = f\left(x\right) * f\left(e\right) .\]
De esta manera, 
\[e' = \left(f\left(x\right)\right)^{-1}*f\left(x\right) = \left(f\left(x\right)\right)^{-1}*f\left(x\right)*f\left(e\right) = e'*f\left(e\right) = f\left(e\right) .\]
Similarmente, tenemos que 
\[f\left(x\right)*f\left(x^{-1}\right) = f\left(x \cdot x^{-1}\right) = f\left(e\right) = e' .\]
Así, tenemos que $\displaystyle f\left(x^{-1}\right) = \left(f\left(x\right)\right)^{-1} $. El hecho que $\displaystyle f^{-1}\left(e'\right)=e $ se deriva de que $\displaystyle f $ es una biyección.
\item[(c)] Sea $\displaystyle H \leq G $. Entonces si $\displaystyle f\left(x\right), f\left(y\right) \in f\left(H\right) $ tenemos que 
	\[f\left(x\right) & \left(f\left(y\right)\right)^{-1} = f\left(x \cdot y^{-1}\right) \in f\left(H\right) ,\]
	pues $\displaystyle x \cdot y^{-1} \in H $.
\item[(d)] Tenemos que $\displaystyle f\left(\langle x \rangle\right) \leq G'$ y además, 
	\[f\left(x^{n}\right) = \left(f\left(x\right)\right)^{n} .\]
Por tanto, 
\[\ord\left(x\right) = \left|\langle x \rangle \right| = \left|f\left(\langle x \rangle\right)\right| = \left|\langle f\left(x\right)\rangle \right| = \ord\left(f\left(x\right)\right) .\]
\item[(e)] Si $\displaystyle G $ es abeliano, $\displaystyle \forall x,y \in G $ tenemos que $\displaystyle x \cdot y = y \cdot x$. Así,
	\[f\left(x \cdot y\right) = f\left(x\right) * f\left(y\right) \quad \text{y} \quad f\left(y \cdot x\right) = f\left(y\right) * f\left(x\right)\]
\[\therefore f\left(x\right) * f\left(y\right) = f\left(y\right) * f\left(x\right) .\]	
\end{description}
\end{proof}

\begin{fdefinition}[Homomorfismo]
\normalfont Se dice que $\displaystyle \left(G, \cdot \right) \to \left(G', *\right) $ es \textbf{homomorfismo} si $\displaystyle \forall x,y \in G $, $\displaystyle f\left(x \cdot y\right) = f\left(x\right) * f\left(y\right) $.
\end{fdefinition}

\begin{eg}
\normalfont La siguiente aplicación es homomorfismo pero no isomorfismo.
\[
\begin{split}
	f : \left(\GL\left(n\right), \cdot \right) & \to \left(\R/ \left\{ 0\right\} , \cdot\right)\\
	M_{n\times n} & \to \det\left(M_{n \times n}\right).
\end{split}
\]
Si $\displaystyle A, B \in \GL\left(n\right) $, entonces $\displaystyle f\left(A \cdot B\right) = f\left(A\right) \cdot f\left(B\right) $, pues $\displaystyle \det\left(A \cdot B\right) = \det\left(A\right) \cdot \det \left(B\right) $. 
\end{eg}

\begin{fdefinition}[Núcleo e imagen]
\normalfont Sea $\displaystyle e' $ el elemento neutro de $\displaystyle G' $. El \textbf{núcleo} de $\displaystyle f $ se define de la siguiente manera:
\[\Ker\left(f\right) = \left\{ x \in G \; : \; f\left(x\right) = e'\right\}  .\]
Se define \textbf{imagen} de $\displaystyle f $ al conjunto
\[\Imagen\left(f\right) = \left(f\left(x\right) \in G'\; : \; x \in G\right) .\]
\end{fdefinition}

\begin{eg}
	\normalfont En la aplicación del ejemplo anterior tenemos que $\displaystyle \Ker\left(f\right) = \SL\left(n\right) $ e $\displaystyle \Imagen\left(f\right)=\R/ \left\{ 0\right\}  $.
\end{eg}

