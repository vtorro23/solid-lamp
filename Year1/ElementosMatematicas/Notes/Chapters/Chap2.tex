\chapter{Grupos de simetrías. Mosaicos.}

\textbf{Objetivo:} Estudiar y clasificar todas las formas posibles de cubrir el plano con copias idénticas de una 'ficha' o 'pieza' (tesela) de manera períodica: mosaicos. \\ \\
\textbf{Herramientas:} Estudio de los movimientos en el plano que preservan la teselación. \\ \\
\textbf{Elementos matemáticos:} 
\begin{itemize}
\item Grupo - Subgrupo
\item Grupo de movimento en el plano
\item Simetrías
\end{itemize}
\textbf{Aplicaciones:}
\begin{itemize}
\item Retículos
\item 17 grupos cristalográficos planos
\end{itemize}

\section{Grupos. Isomorfismos y homomorfismos de grupos.}

\begin{fdefinition}[Grupo]
\normalfont Se define como \textbf{grupo} al par ordenado $\displaystyle \left(G, \cdot\right) $ formado por un conjutno $\displaystyle G $ y una operación interna
\[
\begin{split}
	\cdot : G \times G & \to G \\
\left(a,b\right) & \to a \cdot b .
\end{split}
\]
tales que
\begin{description}
\item[(1)] La operación es asociativa. 
	\[\forall a, b, c \in G, \; \left(a \cdot b\right) \cdot c = a \cdot \left(b \cdot c\right) .\]
\item[(2)] Existe un elemento neutro.
	\[\exists e \in G, \forall a \in G, \; e \cdot a = a \cdot e = a .\]
\item[(3)] Existe el inverso. 
	\[\forall a \in G, \exists a^{-1} \in G\; a \cdot a^{-1} = a^{-1} \cdot a = e .\]
\end{description}
Si se cumple también la propiedad conmutativa:
\[\forall a, b \in G, \; a \cdot b = b \cdot a ,\]
se denomina grupo conmutativo o \textbf{abeliano}.
\end{fdefinition}

\begin{eg}
\normalfont 
\begin{description}
\item[(1)] $\displaystyle \left(\Z, +\right) $ es un grupo abeliano. 
\item[(2)] $\displaystyle \left(\Q, \cdot\right) $ no es un grupo pues $\displaystyle 0 \in \Q $ no tiene inverso. 
\item[(3)] $\displaystyle \left(\Q^{*}, \cdot\right) $ es un grupo abeliano. 
\item[(4)] Tenemos que $\displaystyle \left(2\Z+1, +\right) $ no es un grupo, pues la suma de dos números impares es un número par. Similarmente, $\displaystyle \left(2\Z+1, \cdot \right) $ no es un grupo por la ausencia de inversos. 
\item[(5)] El conjunto de las matrices $\displaystyle 2 \times 2 $ es un grupo abeliano con la suma. $\displaystyle \left(M_{2\times2}, +\right) $ es un grupo abeliano. 
\item[(6)] El conjunto de las matrices $\displaystyle n \times n $ con $\displaystyle \det \neq 0 $ , $\displaystyle \left(M_{n\times n}, \cdot\right) $, es un grupo abeliano. En concreto, en el caso de las matrices $\displaystyle 2 \times 2 $, recibe el nombre de \textbf{grupo general lineal de orden 2} y se denota como $\displaystyle \GL\left(2, \R\right) $ o $\displaystyle \GL\left(2\right) $. 
\item[(7)] El conjunto de matrices $\displaystyle 2 \times 2 $, $\displaystyle \left(M_{2\times2}, \cdot\right) $, con $\displaystyle \det = 1 $ es un grupo y se denomina \textbf{grupo especial lineal}: $\displaystyle \SL\left(2\right) $.
\item[(8)] El conjunto de matrices ortogonales ($\displaystyle A^{T} = A^{-1} $) es un grupo al que se denomina \textbf{grupo ortogonal}: $\displaystyle \text{O}\left(2\right) $.
\item[(9)] Las matrices ortogonales con $\displaystyle \det = 1 $ se denomina \textbf{grupo especial ortogonal}: $\displaystyle \SO\left(n\right) $.
\item[(10)] $\displaystyle \left(\Z_{n}, +\right) $ con $\displaystyle n \in \N $ forma un grupo abeliano. 
\item[(11)] Si $\displaystyle p $ es primo, $\displaystyle \left(\Z_{p}, \cdot \right) $ también es un grupo abeliano. 
\end{description}
\end{eg}

\begin{fprop}\normalfont Sea $\displaystyle \left(G, \cdot \right) $ un grupo.
\begin{description}
\item[(1)] Propiedades cancelativas por la derecha y por la izquierda de un grupo:
\[
\begin{split}
& a \cdot c = b \cdot c \Rightarrow a = b \\
& c \cdot a = c \cdot b \Rightarrow a = b.
\end{split}
\]
\item[(2)] El elemento neutro de un grupo es único.
\item[(3)] Cada elemento de un grupo tiene un único inverso. 
\item[(4)] Si $\displaystyle a \in G $, entonces $\displaystyle \left(a^{-1}\right)^{-1} = a $.
\item[(5)] Si $\displaystyle a, b \in G $, $\displaystyle \left(a \cdot b\right)^{-1} = b^{-1} \cdot a ^{-1} $. 
\item[(6)] Son equivalentes:
	\begin{description}
	\item[(a)] $\displaystyle \left(G, \cdot\right) $ es un grupo abeliano. 
	\item[(b)] $\displaystyle \left(a \cdot b\right)^{-1} = a^{-1} \cdot b^{-1}, \forall a, b \in G $.
	\end{description}
\end{description}
\end{fprop}

\begin{proof}
\begin{description}
\item[(1)] Como $\displaystyle \left(G, \cdot\right) $ es un grupo, tenemos que si $\displaystyle a, b, c \in G $, existe $\displaystyle c^{-1} \in G $ tal que 
	\[a \cdot c = b \cdot c \iff a \cdot c \cdot c^{-1} = b \cdot c \cdot c^{-1} \iff a \cdot e = b \cdot e \iff a = b .\]
Similarmente, 
\[c \cdot a = c \cdot b \iff c^{-1} \cdot c \cdot a = c^{-1} \cdot c \cdot b \iff e \cdot a = e \cdot b \iff a = b .\]
\item[(2)] Sean $\displaystyle e, e' \in G $ dos elementos neutros de $\displaystyle \left(G, \cdot\right) $. Entonces, tenemos que 
	\[ e = e \cdot e' = e' \cdot e = e' .\]
\item[(3)] Si $\displaystyle a \in G $, sean $\displaystyle b, c \in G $ dos inversos de $\displaystyle a $. Entonces tenemos que 
	\[a \cdot b = a \cdot c = e .\]
Por \textbf{(1)} tenemos que $\displaystyle b = c $. Otra manera de demostrarlo es:
\[b = b \cdot e = b \cdot \left(a \cdot c\right) = \left(b \cdot a \right) \cdot c = e \cdot c = c .\]
\item[(4)] 
	\[\left(a^{-1}\right)^{-1} \cdot a^{-1} = e .\]
Como el inverso de $\displaystyle a^{-1} $ es $\displaystyle a $ y el inverso es único, tenemos que $\displaystyle a = \left(a^{-1}\right)^{-1} $. 
\item[(5)] Tenemos que 
	\[\left(a \cdot b\right) \cdot \left(b^{-1} \cdot a^{-1}\right) = a \cdot \left(b^{-1} \cdot b\right) \cdot a^{-1} = a \cdot e \cdot a^{-1} = a \cdot a^{-1} = e .\]
	Como el inverso de $\displaystyle a \cdot b $ es $\displaystyle \left(a \cdot b\right)^{-1} $, y el inverso es único, tenemos que $\displaystyle \left(a \cdot b\right)^{-1} = b^{-1} \cdot a^{-1} $.
\item[(6)] Si es un grupo abeliano, se cumple la propiedad conmutativa y con el resultado \textbf{(5)} la demostración es trivial. Recíprocamente, si $\displaystyle \forall a, b \in G $ tenemos que 
	\[\left(a \cdot b\right) ^{-1} = b^{-1} \cdot a^{-1} = a^{-1} \cdot b^{-1} ,\]
tenemos que 
\[ \left(a \cdot b\right) \cdot \left(a \cdot b\right)^{-1} = \left( b \cdot a\right) \cdot\left(a \cdot b\right)^{-1} .\]
Por la propiedad \textbf{(1)} tenemos que $\displaystyle a \cdot b = b \cdot a $, por lo que se cumple la propiedad conmutativa y es un grupo abeliano. 
\end{description}
\end{proof}

\subsection{Subgrupo. Grupos finitos. Orden}

\begin{fdefinition}[]
\normalfont Sea $\displaystyle \left(G, \cdot\right) $ un grupo y sea $\displaystyle H\subset G $. Se dice que $\displaystyle H $ es un subgrupo de $\displaystyle G $, y se escribe $\displaystyle H \leq G $, si $\displaystyle \left(H, \cdot\right) $ es un grupo.
\end{fdefinition}

\begin{eg}
\normalfont 
\begin{description}
\item[(i)] Tenemos que $\displaystyle \left(2\Z, + \right) \leq \left(\Z, +\right) \leq \left(\Q, +\right) \leq \left(\R, +\right) \leq \left(\C, +\right) $. 
\item[(ii)] $\displaystyle \SO\left(n\right) \leq \SL\left(n\right) \leq \GL\left(n\right) $.
\end{description}
\end{eg}

\begin{fdefinition}[]
\normalfont Si $\displaystyle x \in G $ con $\displaystyle \left(G, \cdot \right) $ grupo, $\displaystyle n \in \Z $, denotamos a 
\[x^{n} =
\begin{cases}
	\underbrace{x \cdots x}_{n \; \text{veces}} \; \text{si} \; n > 0\\
	e \; \text{si} \; n = 0 \\
	\underbrace{x^{-1} \cdots x^{-1}}_{n \; \text{veces}} \; \text{si} \; n < 0
\end{cases}
.\]
El conjunto $\displaystyle \langle x \rangle = \left\{ x^{n} \; : \; n \in \Z\right\}  $ es un subgrupo abeliano de $\displaystyle G $. Es un subgrupo generado por el elemento $\displaystyle x $. Si $\displaystyle \exists x \in G $, $\displaystyle G = \langle x \rangle $ se denomina \textbf{grupo cíclico} . 
\end{fdefinition}

\begin{fdefinition}[Orden]
\normalfont Si $\displaystyle G $ es un grupo finito, se llama \textbf{orden} de $\displaystyle G $ a $\displaystyle \left|G\right| $, es decir, al número de elementos de $\displaystyle G $. Si $\displaystyle G $ no es un conjunto finito diremos que tiene \textbf{orden infinito}. Si $\displaystyle x \in G $, el orden de $\displaystyle x $ es: $\displaystyle \ord\left(x\right) = \left|\langle x \rangle\right| $.
\end{fdefinition}

\begin{observation}
\normalfont El orden de un elemento $\displaystyle x $, si es finito, es el menor entero positivo $\displaystyle n $  tal que $\displaystyle x^{n} = e $.
\end{observation}

\begin{fprop}[]
\normalfont Sea $\displaystyle H $ un subconjunto no vacío de un grupo $\displaystyle \left(G, \cdot\right) $. Entonces, $\displaystyle H \leq G $ si y solo si $\displaystyle \forall x,y\in H $ se tiene que $\displaystyle x \cdot y^{-1} \in H $.
\end{fprop}

\begin{proof}
\begin{description}
\item[(i)] Es trivial, pues si $\displaystyle x, y \in H $, como $\displaystyle H \leq G $, tenemos que $\displaystyle \exists y^{-1} \in H $ y $\displaystyle x \cdot y^{-1} \in H $.
\item[(ii)] Recíprocamente, tenemos que la operación interna está cerrada. La propiedad asociativa se cumple porque se $\displaystyle \left(G, \cdot\right) $ es un grupo. Por definición, tenemos que si $\displaystyle x, y \in H $ entonces, $\displaystyle x \cdot y^{-1} \in H $. Podemos coger $\displaystyle x \cdot x^{-1} = e \in H $. Así, $\displaystyle H $ contiene al elemento neutro. Similarmente, como $\displaystyle e,x \in H $, tenemos que $\displaystyle e \cdot x^{-1} = x^{-1} \in H $. Por lo que podemos encontrar inversos para todos los elementos de $\displaystyle H $. 
\end{description}
\end{proof}

\begin{fprop}[]
\normalfont Si $\displaystyle H_{1} \leq G $ y $\displaystyle H_{2} \leq G $, entonces $\displaystyle H_{1}\cap H_{2} \leq G $.
\end{fprop}

\begin{proof}
Tenemos que ver que $\displaystyle \forall x, y \in H_{1} \cap H_{2} \Rightarrow x \cdot y^{-1} \in H_{1} \cap H_{2} $. Si $\displaystyle x, y \in H_{1} \cap H_{2} $, tenemos que $\displaystyle x,y \in H_{1} $ y $\displaystyle x,y \in H_{2} $. Así, como $\displaystyle H_{1}, H_{2} \leq G $, tenemos que $\displaystyle y^{-1} \in H_{1} \cap H_{2} $ y $\displaystyle x \cdot y^{-1} \in H_{1}\cap H_{2} $.
\end{proof}

\begin{fprop}[]
\normalfont Si $\displaystyle \left(G, \cdot\right) $ es un grupo finito y $\displaystyle x \in G $, el orden de $\displaystyle x $ coincide con el menor entero positivo $\displaystyle k $ tal que $\displaystyle x^{k} = e $. Además, 
\[\langle x \rangle = \left\{ x, x^{2}, x^{3}, \ldots, x^{k} = e\right\}  .\]
\end{fprop}

\begin{eg}
	\normalfont Consideramos el grupo $\displaystyle \left(\Z_{6}, +\right) = \left\{ 0, 1, 2, 3, 4, 5\right\} $. Entonces, tenemos que 
	\[
	\begin{split}
		\langle 0 \rangle & = \left\{ 0\right\} \Rightarrow \ord\left(0\right) = 1\\
		\langle 1 \rangle & = \left\{ 1, 2, 3, 4, 5, 0\right\} \Rightarrow \ord\left(1\right) = 6 \\
		\langle 2 \rangle & = \left\{ 2, 4, 0\right\} \Rightarrow \ord\left(2\right) = 3 \\
		\langle 3 \rangle & = \left\{ 3, 0\right\} \Rightarrow \ord\left(3\right) = 2 \\
		\langle 4 \rangle & = \left\{ 4, 2, 0\right\} \Rightarrow \ord\left(4\right) = 3\\
		\langle 5 \rangle & = \left\{ 5, 4, 3, 2, 1, 0\right\} \Rightarrow \ord\left(5\right) = 6 .
	\end{split}
	\]
\end{eg}

\subsection{Isomorfismos de grupos}
\begin{fdefinition}[Isomorfismo]
\normalfont Sean $\displaystyle \left(G, \cdot \right), \left(G', *\right) $ grupos. Se dice que $\displaystyle f: G \to G' $ es un \textbf{isomorfismo} (de grupos) si es biyectiva y 
\[\forall x, y \in G, \; f\left(x \cdot y\right) = f\left(x\right) * f\left(y\right) .\]
Si $\displaystyle G $ es isomorfo a $\displaystyle G' $, lo escribirmos $\displaystyle G \cong G' $.
\end{fdefinition}
\begin{eg}
\normalfont $\displaystyle f: \left(\R, +\right) \to \left(\R^{+}, \cdot \right) $ con $\displaystyle f : x \to e^{x} $.
\end{eg}

\begin{fprop}[]
\normalfont Si $\displaystyle G \cong G' $, entonces
\begin{description}
\item[(a)] $\displaystyle G $ y $\displaystyle G' $ tienen el mismo cardinal.
\item[(b)] Si $\displaystyle e $ y $\displaystyle e' $ son, respectivamente, los elementos neutros de $\displaystyle G $ y $\displaystyle G' $,
	\[f\left(e\right) = e' \quad \text{y} \quad f^{-1}\left(e'\right) = e .\]
Además, $\displaystyle f\left(x^{-1}\right) = \left(f\left(x\right)\right)^{-1} $.
\item[(c)] Si $\displaystyle H \leq G $, $\displaystyle f\left(H\right) \leq G'$.
\item[(d)] Si $\displaystyle x \in G $, $\displaystyle \ord\left(x\right) = \ord\left(f\left(x\right)\right) $.
\item[(e)] Si $\displaystyle G $ es abeliano, entonces $\displaystyle G' $ también lo es.
\end{description}
\end{fprop}

\begin{proof}
\begin{description}
\item[(a)] Es trivial, puesto que tiene que un isomorfismo es biyectivo por definición.
\item[(b)] Tenemos que $\displaystyle \forall x \in G $, 
	\[f\left(x\right) = f\left(x \cdot e\right) = f\left(x\right) * f\left(e\right) .\]
De esta manera, 
\[e' = \left(f\left(x\right)\right)^{-1}*f\left(x\right) = \left(f\left(x\right)\right)^{-1}*f\left(x\right)*f\left(e\right) = e'*f\left(e\right) = f\left(e\right) .\]
Similarmente, tenemos que 
\[f\left(x\right)*f\left(x^{-1}\right) = f\left(x \cdot x^{-1}\right) = f\left(e\right) = e' .\]
Así, tenemos que $\displaystyle f\left(x^{-1}\right) = \left(f\left(x\right)\right)^{-1} $. El hecho que $\displaystyle f^{-1}\left(e'\right)=e $ se deriva de que $\displaystyle f $ es una biyección.
\item[(c)] Sea $\displaystyle H \leq G $. Entonces si $\displaystyle f\left(x\right), f\left(y\right) \in f\left(H\right) $ tenemos que 
	\[f\left(x\right) & \left(f\left(y\right)\right)^{-1} = f\left(x \cdot y^{-1}\right) \in f\left(H\right) ,\]
	pues $\displaystyle x \cdot y^{-1} \in H $.
\item[(d)] Tenemos que $\displaystyle f\left(\langle x \rangle\right) \leq G'$ y además, 
	\[f\left(x^{n}\right) = \left(f\left(x\right)\right)^{n} .\]
Por tanto, 
\[\ord\left(x\right) = \left|\langle x \rangle \right| = \left|f\left(\langle x \rangle\right)\right| = \left|\langle f\left(x\right)\rangle \right| = \ord\left(f\left(x\right)\right) .\]
\item[(e)] Si $\displaystyle G $ es abeliano, $\displaystyle \forall x,y \in G $ tenemos que $\displaystyle x \cdot y = y \cdot x$. Así,
	\[f\left(x \cdot y\right) = f\left(x\right) * f\left(y\right) \quad \text{y} \quad f\left(y \cdot x\right) = f\left(y\right) * f\left(x\right)\]
\[\therefore f\left(x\right) * f\left(y\right) = f\left(y\right) * f\left(x\right) .\]	
\end{description}
\end{proof}

\begin{fdefinition}[Homomorfismo]
\normalfont Se dice que $\displaystyle \left(G, \cdot \right) \to \left(G', *\right) $ es \textbf{homomorfismo} si $\displaystyle \forall x,y \in G $, $\displaystyle f\left(x \cdot y\right) = f\left(x\right) * f\left(y\right) $.
\end{fdefinition}

\begin{eg}
\normalfont La siguiente aplicación es homomorfismo pero no isomorfismo.
\[
\begin{split}
	f : \left(\GL\left(n\right), \cdot \right) & \to \left(\R/ \left\{ 0\right\} , \cdot\right)\\
	M_{n\times n} & \to \det\left(M_{n \times n}\right).
\end{split}
\]
Si $\displaystyle A, B \in \GL\left(n\right) $, entonces $\displaystyle f\left(A \cdot B\right) = f\left(A\right) \cdot f\left(B\right) $, pues $\displaystyle \det\left(A \cdot B\right) = \det\left(A\right) \cdot \det \left(B\right) $. 
\end{eg}
\begin{observation}
\normalfont En la proposición anterior, \textbf{(b)} y \textbf{(c)} también se cumplen para los homomorfismos.   
\end{observation}

\begin{fdefinition}[Núcleo e imagen]
\normalfont Sea $\displaystyle e' $ el elemento neutro de $\displaystyle G' $. El \textbf{núcleo} de $\displaystyle f $ se define de la siguiente manera:
\[\Ker\left(f\right) = \left\{ x \in G \; : \; f\left(x\right) = e'\right\}  .\]
Se define \textbf{imagen} de $\displaystyle f $ al conjunto
\[\Imagen\left(f\right) = f\left(G\right) = \left\{f\left(x\right) \in G'\; : \; x \in G\right\} .\]
\end{fdefinition}

\begin{eg}
	\normalfont En la aplicación del ejemplo anterior tenemos que $\displaystyle \Ker\left(f\right) = \SL\left(n\right) $ e $\displaystyle \Imagen\left(f\right)=\R/ \left\{ 0\right\}  $.
\end{eg}

\begin{fprop}[]
\normalfont Si $\displaystyle f $ es homomorfismo, tenemos que $\displaystyle \Ker\left(f\right) \leq G $ y $\displaystyle \Imagen\left(f\right) \leq G' $.
\end{fprop}

\begin{proof}
Para ver que $\displaystyle \Ker\left(f\right) \leq G $ basta ver que $\displaystyle \forall x,y \in \Ker\left(f\right) $ se verifica que 
\[f\left(xy^{-1}\right)=f\left(x\right)\left(f\left(y\right)\right)^{-1}=e^{-1} .\]
Es decir, $\displaystyle xy^{-1} \in \Ker\left(f\right) $. Por otra parte, como los homomorfismos llevan subgrupos en subgrupos, tenemos que $\displaystyle \Imagen\left(f\right) \leq G $.
\end{proof}

\begin{fprop}[]
\normalfont Sea $\displaystyle f $ es homomorfismo.
\begin{description}
	\item[(a)]  $\displaystyle f $ es inyectiva $\displaystyle \iff  $ $\displaystyle \Ker\left(f\right) = \left\{ e\right\}  $.
	\item[(b)] $\displaystyle f $ es sobreyectiva $\displaystyle \iff  $ $\displaystyle \Imagen\left(f\right) = G' $.
	\item[(c)] Si $\displaystyle f $ es inyectiva y sobreyectiva es biyectiva y, por tanto, isomorfismo.
\end{description}
\end{fprop}

\begin{proof}
\begin{description}
	\item[(a)] Sea $\displaystyle f $ inyectiva. Si $\displaystyle a \in \Ker\left(f\right) $ tenemos que $\displaystyle f\left(a\right) = f\left(e\right) = e' $, por lo que $\displaystyle a = e $. Recíprocamente, si $\displaystyle \Ker\left(f\right) = \left\{ e\right\}  $, tenemos que si $\displaystyle f\left(a\right) = f\left(b\right) $ entonces
	\[ f\left(a\right) = f\left(b\right) \iff f\left(a\right)\left(f\left(b\right)\right)^{-1} = e' \iff f\left(ab^{-1}\right) = e' .\]
Entonces, $\displaystyle ab^{-1} \in \Ker\left(f\right) $ por lo que $\displaystyle ab^{-1} = e $, es decir, $\displaystyle a = b $. Por lo que $\displaystyle f $ es inyectiva.
\item[(b)] Sale directamente de la definición de sobreyectividad.
\end{description}

\end{proof}

\subsection{Subgrupos normales. Grupo cociente.}

\begin{fdefinition}[]
\normalfont Sea $\displaystyle H \leq G $. Definimos la relación de equivalencia $\displaystyle x \mathcal{R}_{H} y \iff x \cdot y^{-1} \in H $. Entonces, las clases de equivalencia tendrán la forma
\[Hx = \left[x\right] _{H} = \left\{ hx \; : \; h \in H\right\}  .\]
Cada clase de equivalencia contiene exactamente $\displaystyle \left|H\right| $ elementos.
\end{fdefinition}

\begin{ftheorem}[Teorema de Lagrange]
\normalfont Sea $\displaystyle G $ un grupo finito y $\displaystyle H \leq G $. Entonces $\displaystyle \left|H\right| $ divide a $\displaystyle \left|G\right| $.
\end{ftheorem}

\begin{proof}
Sea $\displaystyle G/H $ el conjunto cociente de $\displaystyle G $ por la relación de equivalencia $\displaystyle \mathcal{R}_{H} $. 
\[G/H = \left\{ Hx \: : \: x \in G\right\}  .\]
Dado que los $\displaystyle Hx $ son una partición de $\displaystyle G $, cada $\displaystyle Hx $ tiene $\displaystyle \left|H\right| $ elementos y, al haber $\displaystyle \left|G/H\right| $ de ellas, tenemos que $\displaystyle \left|G\right| = \left|G/H\right| \left|H\right| $.
\end{proof}

\begin{fdefinition}[Conjunto cociente]
\normalfont Denominamos conjunto cociente al conjunto
\[G/H = \left\{ Hx \; : \; x \in G\right\}  .\]
\end{fdefinition}

\begin{fdefinition}[Subgrupo normal]
\normalfont Sea $\displaystyle H \leq G $. Decimos que $\displaystyle H $ es \textbf{subgrupo normal} $\displaystyle \left(H \triangle G\right) $ si se verifica que 
\[g \cdot h \cdot g^{-1} \in H, \; \forall g \in G, \forall h \in H .\]
\end{fdefinition}

\begin{observation}
\normalfont Si $\displaystyle G $ es abeliano, todos sus subgrupos son normales.
\end{observation}

\begin{fprop}[]
\normalfont Si $\displaystyle H \triangle G $, entonces $\displaystyle \left(G/H, \cdot \right) $ es grupo.
\end{fprop}

\begin{proof}
Podemos definir el producto de clases de la siguiente manera:
\[\left(Hx\right)\left(Hy\right) = Hxy .\]
La definición no depende del representante elegido pues
\[\left(hxh'y\right)\left(xy\right)^{-1} = hxh'yy^{-1}x^{-1}=hxh'x^{-1}=h\left(xh'x^{-1}\right) \in H .\]
Esto último se cumple porque $\displaystyle H $ es normal.
\end{proof}

\begin{fprop}[]
\normalfont Sea $\displaystyle f: G \to G' $ homomorfismo. Entonces $\displaystyle \Ker\left(f\right) \triangle G $.
\end{fprop}
\begin{proof}
Tenemos que $\displaystyle \forall x \in \Ker\left(f\right), \; \forall z \in G $, $\displaystyle \Ker\left(f\right) \leq G $. Ahora tenemos que ver que $\displaystyle z \cdot x \cdot z^{-1} \in \Ker\left(f\right) $. Es decir, $\displaystyle f\left(z xz^{-1}\right) = e' $. Como $\displaystyle f $ es homomorfismo,
\[f\left(zxz^{-1}\right) = f\left(z\right)f\left(x\right)f\left(z^{-1}\right) = f\left(z\right)f\left(z^{-1}\right) = f\left(z\right)\left(f\left(z\right)\right)^{-1} = e' .\]

\end{proof}
\section{Isometrías del plano}

\begin{fdefinition}[Isometrías del plano]
\normalfont Aplicación biyectiva $\displaystyle f: \R^{2} \to \R^{2} $ tal que $\displaystyle d\left(f\left(x\right),f\left(y\right)\right) = d\left(x,y\right) $. Es decir, $\displaystyle \left|f\left(x\right)-f\left(y\right)\right| = \left|x-y\right| $. Las isometrías preservan los ángulos.
\end{fdefinition}

\begin{fprop}[]
\normalfont Tenemos que $\displaystyle E_{2} $ (isometrías del plano), con la composición forman un grupo $\displaystyle \left(E_{2}, \circ\right) $.
\end{fprop}
\begin{proof}
\begin{description}
\item[(i)] Tenemos que la composición de isometrías es isometría:
\[ \left|\left(g\circ f\right)\left(x\right)-\left(g\circ f\right)\left(y\right)\right| = \left|g\left(f\left(x\right)\right)-g\left(f\left(y\right)\right)\right|= \left|f\left(x\right)-f\left(y\right)\right| = \left|x-y\right| .\]
A partir de esto la propiedad asociativa es trivial.
\item[(ii)] La aplicación identidad es, trivialmente una isometría.
\item[(iii)] Si $\displaystyle f $ es una isometría, tenemos que, como es biyectiva, existe $\displaystyle f^{-1} $ que cumple que
	\[ \left|f^{-1}\left(x\right)-f^{-1}\left(y\right)\right| = \left|f\left(f^{-1}\left(x\right)\right)-f\left(f^{-1}\left(y\right)\right)\right| = \left|x-y\right| .\]
\end{description}
\end{proof}

\begin{eg}
\normalfont 
\begin{description}
\item[(1)] Traslación de un vector $\displaystyle \vec{v} $. Tenemos que $\displaystyle t_{\vec{v}} : \R^{2} \to \R^{2} $ tal que $\displaystyle t_{\vec{v}}\left(x\right) = x + \vec{v} $, donde $\displaystyle x $ es un punto en $\displaystyle \R^{2} $.
\item[(2)] Las rotaciones alrededor de un punto.
\item[(3)] Las reflexiones respecto a una recta.
\end{description}
\end{eg}

\begin{fdefinition}[Aplicación lineal]
\normalfont Tenemos que $\displaystyle f: \R^{2} \to \R^{2} $ es una \textbf{aplicación lineal} si se verifica que
\begin{description}
\item[(1)] $\displaystyle f\left(x + y\right) = f\left(x\right) + f\left(y\right), \; \forall x,y \in \R^{2} $.
\item[(2)] $\displaystyle f\left(\lambda x\right) = \lambda f\left(x\right), \; \forall x\in \R^{2}, \forall\lambda \in\R $.
\end{description}
\end{fdefinition}

\begin{observation}
\normalfont De \textbf{(1)} y \textbf{(2)} podemos formular el resultado equivalente:
\[f\left(\alpha x + \beta y\right) = \alpha f\left(x\right) + \beta f\left(y\right), \; \forall x,y \in \R^{2},\forall\alpha, \beta \in \R .\]
\end{observation}

\begin{observation}
	\normalfont Toda aplicación lineal $\displaystyle f: \R^{2} \to \R^{2} $ viene determinada por las imágenes de una base. Sea $\displaystyle B = \left\{e_{1} = \begin{pmatrix} 1 \\ 0 \end{pmatrix}, e_{2} = \begin{pmatrix} 0 \\ 1 \end{pmatrix}\right\}  $ una base de $\displaystyle \R^{2} $. Entonces, si $\displaystyle f\left(e_{1}\right) = \begin{pmatrix} a \\ c \end{pmatrix} $ y $\displaystyle f\left(e_{2}\right) = \begin{pmatrix} b \\ d \end{pmatrix} $. Entonces, la matriz asociada a la aplicación lineal será
	\[M = \begin{pmatrix} a & b \\ c & d\end{pmatrix} .\]
\end{observation}

\begin{fprop}[]
\normalfont Las isometrías del plano que dejan fijo el origen son aplicaciones lineales.
\end{fprop}

\begin{proof}
Sea $\displaystyle f : \R^{2} \to \R^{2} $ una isometría que verifica que $\displaystyle f\left(0\right)=0 $. Entonces
\[ \left|f\left(y\right)\right| = \left|f\left(y\right)-f\left(0\right)\right| = \left|y-0\right|= \left|y\right|, \; \forall y \in \R^{2} .\]
\begin{description}
\item[(i)] Si $\displaystyle \lambda = 0 $ o $\displaystyle x = 0 $ es trivial. Si $\displaystyle x \in \R^{2} $ y $\displaystyle \lambda \in \R $, tenemos que, como $\displaystyle f $ conserva la alineación de tres puntos, entonces $\displaystyle x $, $\displaystyle f\left(x\right) $ y $\displaystyle \lambda f\left(x\right) $ están alineados. Entonces, $\displaystyle f\left(\lambda x\right) = \mu f\left(x\right) $ para algún $\displaystyle \mu \in \R $. Así,
	\[ \left|\mu \right| \left|x\right| = \left|\mu\right| \left|f\left(x\right)\right|= \left|\mu f\left(x\right)\right| = \left|f\left(\lambda x\right)\right| = \left|\lambda x\right| = \left|\lambda \right| \left|x\right| .\]
	Por tanto, $\displaystyle \mu = \pm \lambda  $. Si $\displaystyle f\left(\lambda x \right) = -\lambda f\left(x\right) $ tendríamos que
	\[ \left|\lambda + 1\right| \left|x\right| = \left|\left(-\lambda-1\right)f\left(x\right)\right|= \left|f\left(\lambda x \right)-f\left(x\right)\right| = \left|\lambda x -x\right| = \left|\lambda-1\right| \left|x\right| .\]
	Como $\displaystyle x \neq 0 $, tenemos que $\displaystyle \left|\lambda+1\right| = \left|\lambda -1\right| $, que solo puede pasar si $\displaystyle \lambda = 0 $, que es una contradicción.
\item[(ii)] Vamos a ver que $\displaystyle \forall x,y \in \R^{2} $, $\displaystyle f\left(x+y\right) = f\left(x\right)+f\left(y\right) $. Si $\displaystyle 0 $, $\displaystyle x $ e $\displaystyle y $ están alineados tenemos que $\displaystyle y = \mu x $ y podemos aplicar el apartado anterior. En otro caso, los puntos anteriores forman el triángulo $\displaystyle x0y $, que queda transformado en el triángulo $\displaystyle f\left(x\right)0f\left(y\right) $. Similarmente, el triángulo $\displaystyle x\left(x+y\right)y $ se transforma en $\displaystyle f\left(x\right)pf\left(y\right) $ donde $\displaystyle p = f\left(x+y\right) $. Como $\displaystyle f $ es isometría y ser $\displaystyle 0, \; x, \; x+y,\;  y $ un paralelogramo, tenemos que el cuadrilátero $\displaystyle 0, \; f\left(x\right), \; p, \; f\left(y\right) $ también es un paralelogramo. Entonces, tenemos que $\displaystyle p = f\left(x\right)+f\left(y\right) $.
\end{description}
\end{proof}

\begin{fprop}[]
\normalfont Las únicas isometrías del plano que dejan fijo el origen son las rotaciones en torno al origen y las reflexiones con eje por el origen.
\end{fprop}

\begin{proof}
Por la proposición anterior, las isometrías que dejan fijo el origen son aplicaciones lineales, por lo que tienen asociadas a ellas una matriz $\displaystyle M $. Dado que tienen que preservar la norma, 
\[M^{t}M=I_{2} .\]
Esto se puede verificar usando la igualdad
\[x^{t}x = \left|x\right|^{2} = \left|Mx\right|^{2} = \left(Mx\right)^{t}Mx, \; x \in \R^{2} .\]
Por ser $\displaystyle \GL\left(2\right) $ un grupo, tenemos que 
\[M^{t} = M^{-1} \Rightarrow M^{t}M=I_{2} .\]
Consecuentemente, si 
\[M = \begin{pmatrix} a & b \\ c & d \end{pmatrix} ,\]
tenemos que 
\[\begin{pmatrix} 1 & 0 \\ 0 & 1 \end{pmatrix} = \begin{pmatrix} a & b \\ c & d \end{pmatrix}\begin{pmatrix} a & c \\ b & d \end{pmatrix} = \begin{pmatrix} a^{2}+b^{2} & ac+bd \\ ac + bd & c^{2}+d^{2} \end{pmatrix} .\]
Por un lado, tenemos que $\displaystyle a^{2}+b^{2} = c^{2}+d^{2} = 1 $, por lo que existen $\displaystyle \theta $ y $\displaystyle \varphi $ tales que 
\[a = \cos\varphi, \; b=\sin\varphi, \; d=\cos\theta \; \text{y} \; c = \sin \theta .\]
Por otro lado, tenemos que $\displaystyle ac+bd=0 $, por lo que
\[0 = \cos\varphi\sin\theta + \sin\varphi \cos\theta = \sin\left(\theta+\varphi\right) .\]
Entonces,
\[\theta+\varphi = 0 \quad \text{o} \quad \theta+\varphi = \pi  .\]
\begin{description}
\item[(i)] Si $\displaystyle \theta + \varphi = 0 $, tenemos que $\displaystyle \cos\varphi = \cos\theta  $ y $\displaystyle \sin\varphi = -\sin\theta  $. Por tanto, 
	\[M = \begin{pmatrix} \cos\theta & -\sin\theta \\ \sin\theta & \cos\theta \end{pmatrix} .\]
Esta matriz coincide con la de uan rotación de ángulo $\displaystyle \theta $ alrededor del origen, por tanto esta transformación coincide con una rotación. Además, como $\displaystyle \det\left(M\right) = 1 $ tenemos que $\displaystyle M \in \SO\left(2\right) $.
\item[(ii)] Si $\displaystyle \theta+\varphi = \pi  $ tenemos que $\displaystyle \cos\theta = -\cos\varphi $ y $\displaystyle \sin\theta = \sin\varphi $, por lo que 
	\[M = \begin{pmatrix} \cos\varphi & \sin\varphi \\ \sin\varphi & -\cos\varphi \end{pmatrix} .\]
Por tanto, como esta matriz coincide con la de una reflexión de ángulo $\displaystyle \frac{\varphi}{2} $ con respecto al origen, la isometría es dicha reflexión. En este caso, $\displaystyle \det\left(M\right) = -1 $.
\end{description}
\end{proof}

\begin{observation}
\normalfont Tenemos que la matriz asociada a una rotación con ángulo de rotación $\displaystyle \theta  $ será
\[A_{\theta } = \begin{pmatrix} \cos\theta & -\sin\theta \\ \sin\theta & \cos\theta \end{pmatrix}, \; \left|A_{\theta }\right| = 1 .\]
\end{observation}

\begin{observation}
\normalfont Tenemos que la matriz asociada a una reflexión será
\[B_{\varphi }=\begin{pmatrix} \cos\varphi & \sin\varphi \\ \sin\varphi & -\cos\varphi  \end{pmatrix}, \; \left|B_{\varphi}\right|=-1 .\]
\end{observation}

\begin{observation}
\normalfont Tenemos que $\displaystyle A_{\theta} $ y $\displaystyle B_{\varphi} $ son matrices ortogonales. Es decir, $\displaystyle A^{-1}_{\theta} = A^{t}_{\theta} $ y $\displaystyle B^{-1}_{\varphi} = B^{t}_{\varphi} $. Una propiedad importante de las matrices ortogonales es que $\displaystyle MM^{t}=I_{2} $, es decir, $\displaystyle \det\left(M\right)= \pm 1 $. Si $\displaystyle \det\left(M\right)=-1 $, tenemos que $\displaystyle M^{2}=I_{2} $.
\end{observation}

\begin{fprop}[]
\normalfont Las únicas isometrías del plano son las rotaciones en torno al origen compuestas con las traslaciones y las reflexiones con eje por el orgien compuestas con traslaciones.
\end{fprop}

\begin{proof}
Sea $\displaystyle f $ una isometría y que $\displaystyle f\left(0\right) = v $. Sea $\displaystyle t_{v} $ la traslación de vector $\displaystyle v $, se tiene que la isometría
\[t ^{-1}_{v}\circ f = g ,\]
fija el origen. El resultado se sigue aplicando la proposición anterior.
\end{proof}
Entonces, tenemos que podemos expresar cualquier isometría del plano $\displaystyle f $ en términos de una traslación $\displaystyle t_{v} $ y una transformación ortogonal $\displaystyle f_{M} $.
\begin{description}
\item[(1)] Traslación de vector $\displaystyle v $: $\displaystyle t_{v}\left(x\right) = v + x $.
\item[(2)] Transformación ortogonal (rotación o reflexión): $\displaystyle f_{M} $ tal que $\displaystyle f_{M}\left(x\right)= Mx $, donde $\displaystyle M = A_{\theta}, B_{\varphi} $.
\end{description}
\begin{notation}
\normalfont $\displaystyle f\left(v,M\right) = t_{v} \circ t_{M} $ tal que $\displaystyle f\left(x\right) = v + Mx $. Si $\displaystyle \det\left(M\right) = 1 $, entonces se trata de rotaciones. Si $\displaystyle \det\left(M\right) = -1 $, entonces se trata de reflexiones. Además, si $\displaystyle \det\left(M\right) = 1 $ se habla de \textbf{isometrías directas} y, si no, de \textbf{isometrías inversas}.
\end{notation}
 Vamos a calcular la expresión de la composición de dos isometrías. Sean $\displaystyle f=\left(v,M\right) $ y $\displaystyle g=\left(u,N\right) $ isometrías. Tenemos que
\[
\begin{split}
	f\circ g\left(x\right) = & \left(v, M\right)\circ\left(u,N\right) = \left(t_{v}\circ f_{M}\right)\circ\left(t_{n}\circ f_{N}\right)\left(x\right) = v + M\left(u + Nx\right)= \left(v + Mu\right) + MNx \\
	= & t_{v+Mu} \circ f_{MN}\left(x\right)=\left(v + Mu, MN\right).
\end{split}
\]
Así,
\[
\boxed{\left(v,M\right)\left(u,N\right) = \left(v + Mu, MN\right)}
\]

\begin{ftheorem}[]
\normalfont Así, los casos posibles de isometría en el plano se reducen a 
\begin{description}
\item[(a)] Traslaciones de vector $\displaystyle v $. Son de la forma $\displaystyle \left(v, I_{2}\right) $.
\item[(b)] Rotaciones de centro $\displaystyle c $ y ángulo $\displaystyle \theta $. Son de la forma $\displaystyle \left(v, A_{\theta}\right) $ donde
	\[v = c - A_{\theta }c \quad \text{y} \quad A_{\theta} = \begin{pmatrix} \cos\theta & -\sin\theta \\ \sin\theta & \cos\theta \end{pmatrix} .\]
\item[(c)] Reflexiones con eje una recta $\displaystyle l $ que forma un ángulo de $\displaystyle \frac{\varphi}{2} $ con la horizontal. Son de la forma $\displaystyle \left(v, B_{\varphi}\right) $ con 
	\[v = 2a \quad \text{y} \quad B_{\varphi} =\begin{pmatrix} \cos\varphi & \sin\varphi \\ \sin\varphi & -\cos\varphi \end{pmatrix} ,\]
donde $\displaystyle a $ es el punto de $\displaystyle l $ más cercano al origen.
\item[(d)] Reflexiones con deslizamiento con eje una recta $\displaystyle l $ que forma un ángulo $\displaystyle \frac{\varphi}{2} $ con la horizontal y con vector de deslizamiento $\displaystyle b $, paralelo a $\displaystyle l $. Son de la forma $\displaystyle \left(v, B_{\varphi}\right) $ con 
	\[v = 2a+b \quad \text{y} \quad B_{\varphi} = \begin{pmatrix} \cos\varphi & \sin\varphi \\ \sin\varphi & -\cos\varphi \end{pmatrix},\]
siendo $\displaystyle a $ el punto de $\displaystyle l $ más cercano al origen.
\end{description}
\end{ftheorem}

\begin{proof}
\begin{description}
\item[(a)] Trivial.
\item[(b)] Si $\displaystyle f $ es una rotación de centro $\displaystyle c $  y ángulo $\displaystyle \theta $, tenemos que $\displaystyle f $ se puede construir como una traslación de $\displaystyle c $ al origen, el giro de ángulo $\displaystyle \theta $ respecto a este y, finalmente, la traslación del origen a $\displaystyle c $. Es decir,
	\[f\left(x\right) = c + A_{\theta}\left(x-c\right) .\]
Recíprocamente, toda isometría $\displaystyle f = \left(v, A_{\theta}\right) $ se corresponde con una rotación. En efecto, si fuera una rotación tendría que verificar que $\displaystyle f\left(c\right) = c $, por lo que 
\[v + A_{\theta}c = c \Rightarrow \left(I_{2}-A_{\theta}\right)c=v .\]
Así, $\displaystyle c $ es solución del sistema lineal
\[\begin{pmatrix} 1-\cos\theta & \sin\theta \\ -\sin\theta & 1 - \cos\theta \end{pmatrix}\begin{pmatrix} x_{1} \\ x_{2} \end{pmatrix} = \begin{pmatrix} v_{1} \\ v_{2} \end{pmatrix} .\]
Como el determinante de la matriz del sistema es
\[\left(1-\cos\theta\right)^{2}+ \sin ^{2}\theta = 1 - 2 \cos\theta + \cos^{2}\theta + \sin ^{2}\theta = 2 - 2\cos\theta = 2\left(1-\cos\theta\right) \neq 0,\]
pues $\displaystyle \theta \neq 0 $, tenemos que
\[c = \left(I_{2}-A_{\theta}\right)^{-1}v .\]
Sea $\displaystyle g $ la rotación de centro este valor $\displaystyle c $ y ángulo $\displaystyle \theta $. Entonces $\displaystyle f \equiv g $, pues
\[
\begin{split}
	g\left(p\right) = & \left(c-A_{\theta}c\right) + A_{\theta}p = \left(I_{2}-A_{\theta}\right)c + A_{\theta}p\\
	= & \left(I_{2}-A_{\theta}\right)\left(I_{2}-A_{\theta}\right)^{-1}v + A_{\theta}p \\
	= & v + A_{\theta}p = f\left(p\right), \; p \in \R^{2}.
\end{split}
\]
\item[(c) y (d)] Sea $\displaystyle m $ la recta paralela a $\displaystyle l $ por el origen y $\displaystyle f_{B_{\varphi}} $ la reflexión respecto a $\displaystyle m $. Para construir la reflexión sobre $\displaystyle l $ con deslizamiento $\displaystyle b $ basta trasladar $\displaystyle l $ hasta $\displaystyle m $, reflejar $\displaystyle m $ mediante $\displaystyle f_{B_{\varphi}} $, sumar el vector $\displaystyle b $ y, finalmente, volver a trasladar hasta $\displaystyle l $. Es decir
	\[f\left(x\right) = a + B_{\varphi}\left(x-a\right)+b = a + B_{\varphi}x-B_{\varphi}a+b = 2a + b + B_{\varphi}x ,\]
pues $\displaystyle B_{\varphi}a = - a $. Recíprocamente, si $\displaystyle f $ es una isometría del tipo $\displaystyle \left(v, B_{\varphi}\right) $, consideramos
\[b = \frac{v+B_{\varphi}v}{2} \quad \text{y} \quad a = \frac{v-B_{\varphi}v}{4} .\]
\begin{description}
\item[(i)] Si $\displaystyle b \neq 0 $. Como $\displaystyle B_{\varphi}^{2} = I_{2} $,
	\[B_{\varphi}b = \frac{B_{\varphi}v+B^{2}_{\varphi}v}{2} = \frac{B_{\varphi}v + v}{2} = b .\]
	Es decir, el vector $\displaystyle b $ queda invariante por $\displaystyle B_{\varphi} $ y, por tanto, es un vector director del eje de la reflexión $\displaystyle B_{\varphi} $ (esto es, la recta que pasa por el origen y forma un ángulo de $\displaystyle \frac{\varphi}{2} $ con la horizontal). Por otra parte, como $\displaystyle B_{\varphi}a = - a $, se verifica que $\displaystyle a $ es perpendicular a dicho eje. En efecto,
\[
\begin{split}
	a^{t}b = & \frac{1}{8}\left(v-B_{\varphi}v\right)^{t}\left(v+B_{\varphi}v\right) = \frac{1}{8}\left(v^{t}v - v^{t}B^{t}_{\varphi}v + v^{t}B_{\varphi}v-v^{t}B^{t}_{\varphi}B_{\varphi}v\right)\\ 
	= & \frac{1}{8}\left(v^{t}v - v^{t}B_{\varphi}v + v^{t}B_{\varphi}v-v^{t}I_{2}v\right) = 0.
\end{split}
\]
Así pues, la recta $\displaystyle l $ de dirección $\displaystyle b $  que pasa por el punto $\displaystyle a $ tiene la propiedad de que $\displaystyle a $ es el punto de $\displaystyle l $ más cerano al origen. Si llamamos $\displaystyle g $ a la reflexión sobre $\displaystyle l $ con deslizamiento $\displaystyle b $, por lo visto anteriormente tenemos que
\[g\left(p\right) = 2a + b + B_{\varphi}p = \frac{v-B_{\varphi}v}{2}+\frac{v+B_{\varphi}v}{2}+B_{\varphi}p = v + B_{\varphi}p = f\left(p\right), \; \forall p \in \R^{2},\]
por lo que $\displaystyle f \equiv g $, es decir, $\displaystyle f $ es una reflexión con deslizamiento.
\item[(ii)] Si $\displaystyle b = 0 $, entonces $\displaystyle B_{\varphi}v = - v $ y $\displaystyle a = \frac{v}{2} $. Así pues, el vector $\displaystyle v $ (y, por tanto, $\displaystyle a $) es perpendicular al eje de reflexión de $\displaystyle B_{\varphi} $. Razonando como antes, la recta $\displaystyle l $ paralela al eje de $\displaystyle B_{\varphi} $ que pasa por el punto $\displaystyle a $ tiene la propiedad de que $\displaystyle a $ es el punto de $\displaystyle l $ más cercano al origen. Ahora, si llamamos $\displaystyle g $ a la reflexión respecto a $\displaystyle l $ tenemos que
	\[g\left(p\right) = 2a + B_{\varphi}p = v + B_{\varphi}p = f\left(p\right), \; \forall p \in \R^{2} .\]
	Entonces, $\displaystyle g \equiv f $, es decir, $\displaystyle f $ es una reflexión.
\end{description}

\end{description}
\end{proof}

\begin{observation}
\normalfont Dada una isometría $\displaystyle \left(v,M\right) $, el prodecimiento a seguir para clasificarla es:
\begin{enumerate}
\item Si $\displaystyle M = I_{2} $ entonces es traslación.
\item Si $\displaystyle M \neq I_{2} $ y $\displaystyle \left|M\right| = 1 $ se trata de una rotación.
\item Si $\displaystyle M \neq I_{2} $ y $\displaystyle \left|M\right|=-1 $ se trata de una reflexión. Para distinguir si es reflexión o reflexión con deslizamiento:
\begin{enumerate}
\item Si $\displaystyle Mv = - v $ es una reflexión.
\item Si $\displaystyle Mv \neq - v $ es una reflexión con deslizamiento.
\end{enumerate}
\end{enumerate}
\end{observation}

\section{Grupo diédrico}

$\displaystyle G = \langle x,y \rangle = \left\{ e, x, y, x^{2}, x^{3}, x^{-1}, xy, y^{2} \ldots\right\}  $ grupo generado por $\displaystyle x $ e $\displaystyle y $.
\begin{fdefinition}[Grupo diédrico]
\normalfont El \textbf{grupo diédrico}, $\displaystyle D_{n} $, es el grupo de las simetrías de un polígono regular de $\displaystyle n $ lados (esto es, las isometrías que lo dejan fijo). Tenemos que $\displaystyle \ord\left(D_{n}\right) = 2n $.
\end{fdefinition}

\begin{eg}
\normalfont Rotaciones con respecto a su centro $\displaystyle \left(\sigma \right) $. Tenemos que 
\[
\begin{split}
i \\
\sigma \to \frac{2\pi }{n} \\
\sigma^{2} \to \frac{4\pi }{n} \\
\vdots \\
\sigma^{n-1} \to \left(n-1\right)\frac{2\pi }{n}
\end{split}
\]
Tenemos que $\displaystyle \sigma^{n} = i $.
\end{eg}

\begin{eg}
\normalfont Simetría con respecto a una recta que pasa por el centro del polígono y uno de sus vértices $\displaystyle \left(\tau\right) $.
\end{eg}

El resto de movimientos se pueden obtener de la siguiente manera. Si $\displaystyle \sigma  $ es una rotación de $\displaystyle \frac{2\pi }{3} $ radianes con respecto al centro de un polígono regular de $\displaystyle n $  lados y $\displaystyle \tau $ simetría con respecto a una recta que pasa por este centro y cualquiera de los vértices del polígono, entonces las composiciones 
\[\sigma \circ \tau, \; \sigma ^{2} \circ \tau , \; \ldots, \; \sigma^{n-1}\circ \tau ,\]
son las simetrías que faltan.
Entonces tenemos que 
\[D_{n} = \left\{ i, \sigma, \sigma^{2}, \ldots, \sigma^{n-1}, \tau, \sigma\circ\tau, \sigma^{2}\circ \tau , \ldots, \sigma ^{n-1}\tau\right\}  .\]
Por lo que $\displaystyle D_{n} = \langle \sigma, \tau \rangle  $ y $\displaystyle \sigma^{n} = i $ y $\displaystyle \tau^{2} = i $.
\begin{eg}
	\normalfont Tenemos que $\displaystyle D_{3} = \left\{ i, \sigma, \sigma^{2},\tau, \sigma\circ\tau, \sigma^{2}\circ\tau\right\}  $. Tenemos que $\displaystyle \sigma  $ es la rotación de ángulo $\displaystyle \frac{2\pi }{3} $ y así para el resto de rotaciones.
\end{eg}

\begin{fprop}[]
\normalfont 
\[\tau \circ \sigma = \sigma ^{-1}\circ \tau .\]
\end{fprop}

\begin{fprop}[]
\normalfont 
\[ \sigma^{n} = I_{2}, \quad \tau^{2} = I_{2}, \quad \sigma^{k}\tau=\tau\sigma^{n-k}, \; k = 1, 2, \ldots, n-1 .\]
\end{fprop}

