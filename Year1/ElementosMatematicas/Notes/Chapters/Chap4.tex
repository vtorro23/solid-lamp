\chapter{Dinámica Discreta}
La dinámica se encarga de estudiar los cambios en un sistema a lo largo del tiempo. En particular, la dinámica discreta estudia estos cambios en una sucesión de instantes de tiempo (días, meses, años, etc.). Se asume que cada uno de estos estados depende de los instantes anteriores. En general, buscaremos la relación causa-efecto que permite determinar cierto estado en función de los anteriores.
\begin{eg}
\normalfont Consideremos como primer ejemplo la sucesión de Fibonacci. Una pareja de conejos tarda un mes en alcanzar la edad fértil. A partir de ese mes, es capaz de engendrar una nueva pareja de conejos cada mes. Dado un mes $\displaystyle n $, buscamos calcular cuántos conejos hay.
\begin{itemize}
\item En $\displaystyle t_{0} \left(n = 0\right) $ tenemos solo los dos conejos iniciales. 
\item En $\displaystyle t_{1} \left(n = 1\right) $, la pareja es fértil, por lo que en $\displaystyle t_{2} \left(n = 2\right)$ tendremos la pareja nativa sumada a su descendencia.
\item En $\displaystyle t_{3} \left(n=3\right)$ tenemos la pareja inicial, su descendencia, y otra descendencia suya. 
\item En $\displaystyle t_{4} \left(n = 4\right) $ tenemos la pareja inicial, su descendecia primera y su segunda y la descendencia de su descendencia.
\end{itemize}
Así, en general tenemos que $\displaystyle A\left(n + 2\right) = A\left(n + 1\right) - A\left(n\right) $. Matemáticamente, tenemos que la relación que buscamos es la sucesión de Fibonacci
\[
A\left(0\right) = 1, \; A\left(1\right) = 1, \; A\left(n+2\right) = A\left(n+1\right)-A\left(n\right).\]
\end{eg}
\begin{fdefinition}[]
	\normalfont Un \textbf{sistema dinámico discreto de primer orden} es una sucesión de números $\displaystyle \left\{ A\left(n\right)\right\} _{n\in\N} $ tal que cada elemento de la sucesión se relación con el anterior mediante alguna función $\displaystyle f $ tal que $\displaystyle A\left(n+1\right) = f\left(A\left(n\right),n\right) $, con $\displaystyle n \in \N $ \footnote{Asumimos que $\displaystyle \N $ incluye el 0.} .
\end{fdefinition}
\begin{fdefinition}[]
\normalfont Dado un sistema dinámico $\displaystyle A\left(n+1\right) = f\left(A\left(n\right),n\right) $ de primer orden:
\begin{description}
	\item[(a)] Llamos \textbf{solución general} a la fórmula de sucesiones $\displaystyle \left\{ A\left(n\right)\right\} _{n\in\N} $ que satisface el sistema al solicitar los valores $\displaystyle A\left(n\right) $ y $\displaystyle A\left(n+1\right) $.
	\item[(b)] Llamamos \textbf{solución particular} del sistema con dato inicial $\displaystyle a_{0} = A\left(0\right) $ a la sucesión $\displaystyle \left\{ A\left(n\right)\right\} _{n\in\N} $ que satisface el sistema para los valores $\displaystyle A\left(n+1\right) $ y $\displaystyle A\left(n\right) $ y además, tal que $\displaystyle A\left(0\right) = a_{0} $.
\end{description}
\end{fdefinition}
\begin{fdefinition}[Sistema lineal]
\normalfont Un sistema dinámico discreto de primer orden se dice \textbf{lineal} de coeficientes constantes si es de la forma $\displaystyle A\left(n+1\right) = rA\left(n\right) + b $, donde $\displaystyle b,r \in \R $. En esta situación, se denominan
\begin{description}
\item[(a)] Sistema \textbf{homogéneo} si $\displaystyle b = 0 $.
\item[(b)] Sistema \textbf{afín} si $\displaystyle b \neq 0 $.
\end{description}
\end{fdefinition}
\begin{eg}
\normalfont Consideremos $\displaystyle A\left(n+1\right) = 3A\left(n\right) $. Vamos a encontrar su solución general. Supongamos que $\displaystyle a_{0} = A\left(0\right) $. Tenemos que 
\[A\left(1\right) = 3a_{0}, \; A\left(2\right) = 3A\left(1\right) = 3^{2}a_{0}, \; \ldots, \; A\left(n\right) = 3^{n}a_{0}.\]
Podemos observar que se trata de una sucesión geométrica. 
\end{eg}
\begin{fdefinition}[]
	\normalfont Un sistema dinámico discreto de orden dos es una sucesión $\displaystyle \left\{ A\left(n\right)\right\} _{n\in\N} $ tal que $\displaystyle A\left(n+2\right) = f\left(A\left(n+1\right), A\left(n\right), n\right) $, para cierta función $\displaystyle f $. 
\end{fdefinition}
\begin{observation}
\normalfont Esta definición es generalizable a orden $\displaystyle m $ si la función depende de $\displaystyle m $ estados independientes del sistema.
\end{observation}
\section{Sistemas dinámicos lineales de primer orden}
Estudiaremos los puntos de equilibrio, la estabilidad y análisis gráfico. 
\begin{fdefinition}[Sistemas autónomos]
\normalfont Un sistema dinámico de primer orden se dice \textbf{autonónomo} si el funcional $\displaystyle f $ no depende de $\displaystyle n $, es decir, si $\displaystyle A\left(n+1\right) = f\left(A\left(n\right)\right) $. En caso contrario, se denomina \textbf{no autónomo}.
\end{fdefinition}
\begin{eg}
\normalfont Consideremos un préstamo personal al 5\% anual con pagos anuales de 100 euros. Consideremos $\displaystyle a_{0} = A\left(0\right) $. Medimos la deuda anual:
\begin{center}
\begin{tabular}{|c|c|c|c|c|}
\hline 
$\displaystyle A\left(0\right) $ & $\displaystyle A\left(1\right) $ & $\displaystyle A\left(2\right) $ & $\displaystyle \cdots  $ & $\displaystyle A\left(n+1\right) $ \\
\hline 
$\displaystyle a_{0} $ & $\displaystyle 1,05a_{0}-100 $ & $\displaystyle 1,05a_{0}-100 $ & $\displaystyle \cdots  $ & $\displaystyle 1,05A\left(n\right)-100$ \\
\hline
\end{tabular}
\end{center}
Si $\displaystyle a_{0} = 1000 $, tenemos que la deuda está disminuyendo y si $\displaystyle a_{0} = 3000 $, la deuda está aumentando, cuando $\displaystyle n \in \left\{ 1, \ldots, 5\right\}  $. Tenemos que para $\displaystyle a_{0} = 2000 $, $\displaystyle A\left(1\right) = A\left(2\right) = \cdots = A\left(5\right) $.
\end{eg}
\begin{fdefinition}[Punto de equilibrio]
	\normalfont Un \textbf{punto de equilibrio} (o \textbf{fijo}), $\displaystyle a $ , del sistema dinámico discreto autónomo $\displaystyle A\left(n+1\right) = f\left(A\left(n\right)\right) $, con $\displaystyle n \in \N \cup \left\{ 0\right\}  $, si $\displaystyle A\left(n\right) = a $ es solución constante del sistema. Es decir, si $\displaystyle A\left(n\right) = a $, $\displaystyle \forall n \in \N\cup \left\{ 0\right\}  $. 
\end{fdefinition}
\begin{observation}
\normalfont Se tiene que cumplir que $\displaystyle A\left(0\right) = a_{0}=a $.
\end{observation}
\begin{fprop}[]
\normalfont La condición necesaria y suficiente para que $\displaystyle a \in \R $ sea punto de equilibrio del sistema $\displaystyle A\left(n+1\right) = f\left(A\left(n\right)\right) $, es que $\displaystyle a = f\left(a\right) $.
\end{fprop}
\begin{proof}
	Observar que si $\displaystyle A\left(n\right) = a \iff A\left(n+1\right) = a $, $\displaystyle \forall n \in \N \cup \left\{ 0\right\}  $.
\end{proof}
\begin{eg}
\normalfont Encontrar los puntos de equilibrio del sistema $\displaystyle A\left(n+1\right) = 2A\left(n\right)-3 $. Buscamos $\displaystyle a = f\left(a\right) $, por lo que $\displaystyle A\left(0\right) = a = 3 $. 
\end{eg}
\begin{fprop}[]
\normalfont El único punto de equilibrio del sistema lineal afín discreto de la forma $\displaystyle A\left(n+1\right) = rA\left(n\right) + b $ con $\displaystyle r \neq 1 $ y $\displaystyle r,b \in \R $, es $\displaystyle a = \frac{b}{r-1} $.
\end{fprop}
\begin{proof}
Si $\displaystyle a = A\left(n\right) $, la única solución de $\displaystyle a = ra + b $ es $\displaystyle a = \frac{b}{r-1} $.
\end{proof}
\begin{fdefinition}[]
\normalfont Se dice que un punto de equilibrio $\displaystyle a $ del sistema dinámico $\displaystyle A\left(n+1\right) = f\left(A\left(n\right)\right) $ es
\begin{description}
\item[(i)] \textbf{Estable (atractor/atractivo)} si $\displaystyle \exists \epsilon > 0 $ tal que $\displaystyle \left|A\left(0\right)-a\right|<\epsilon  $, entonces $\displaystyle \lim_{n \to \infty}A\left(n\right) = a $.
\item[(ii)] \textbf{Inestable  (repulsor/repulsivo)} si $\displaystyle \exists \epsilon > 0$, tal que si $\displaystyle 0 < \left|A\left(0\right)-a\right|<\epsilon  $, entonces $\displaystyle \lim_{n \to \infty}A\left(n\right) \neq a $.
\item[(iii)] \textbf{Neutro} si no es estable ni inestable.
\end{description}
\end{fdefinition}
\begin{eg}
\normalfont Continuamos con el ejemplo anterior, $\displaystyle A\left(n+1\right) = 2A\left(n\right)+3 $. Tenemos que el punto de equilibrio único es $\displaystyle a = A\left(0\right) = 3 $. Consideremos $\displaystyle a_{0}^{1} = 3,01 $ y $\displaystyle a_{0}^{2} = 2,99 $.
\begin{center}
\begin{tabular}{|c|c|c|}
\hline 
& $\displaystyle a_{0}^{1} $ &$ a_{0}^{2} $ \\
\hline 
	$\displaystyle A\left(0\right) $ & & \\
	\hline 
	$\displaystyle A\left(1\right) $ & $\displaystyle 3,02 $ & $\displaystyle 2,98 $ \\
	\hline 
	\vdots & \vdots & \vdots \\
	\hline 
	$\displaystyle A\left(10\right) $ & $\displaystyle 13,24 $ & $\displaystyle -7,24 $ \\
	\hline
\end{tabular}
\end{center}
Sin embargo, si cogemos ahora $\displaystyle A\left(n+1\right) = 0,8A\left(n\right) - 2 $, tenemos que el único punto de equilibrio será $\displaystyle a = - 10 $. Cogemos $\displaystyle a_{0}^{1} = -9,99 $, $\displaystyle a_{0}^{2} = -10,01 $, $\displaystyle a_{0}^{3} = -10,005 $ y $\displaystyle a_{0}^{4} = -9,995 $. 
\begin{center}
\begin{tabular}{|c|c|c|c|c|}
\hline
& $\displaystyle a_{0}^{1} $  & $\displaystyle a_{0}^{2} $ & $\displaystyle a_{0}^{3} $ & $\displaystyle a_{0}^{4} $ \\
\hline
	$\displaystyle A\left(0\right) $ & $\displaystyle -9,99 $ & $\displaystyle -10,01 $ & $\displaystyle -10,005 $ & $\displaystyle -9,995 $ \\
	\hline
	\vdots & \vdots & \vdots & \vdots & \vdots \\
	\hline
	$\displaystyle A\left(10\right) $ & $\displaystyle -9,9989 $ & $\displaystyle -10,0010 $ & $\displaystyle -10,0005 $ & $\displaystyle -9,99946 $ \\
	\hline
\end{tabular}
\end{center}
\end{eg}
\begin{ftheorem}[]
	\normalfont Sea el sistema dinámico afín $\displaystyle A\left(n+1\right) = rA\left(n\right) + b $, con $\displaystyle r \in \R / \left\{ 1\right\} $. El punto de equiilibrio $\displaystyle a = \frac{b}{r-1} $ es
	\begin{description}
	\item[(a)] Estable si $\displaystyle \left|r\right| < 1 $. Resulta además que $\displaystyle \lim_{n \to \infty}A\left(n\right) = a $, $\displaystyle \forall A\left(0\right)=a_{0} \in \R$.
	\item[(b)] Inestable si $\displaystyle \left|r\right| > 1 $. Resulta además que $\displaystyle \lim_{n \to \infty} \left|A\left(n\right)\right| = \infty$, $\displaystyle \forall A\left(0\right) = a_{0} \in \R $, con $\displaystyle A\left(0\right) \neq a $. 
	\item[(c)] Neutro si $\displaystyle r = - 1 $. En esta situación $\displaystyle \forall n \in \N\cup \left\{ 0\right\}  $ se tiene que $\displaystyle A\left(2n\right) = A\left(0\right) $ y $\displaystyle A\left(2n+1\right) = A\left(1\right) $. 
	\end{description}
\end{ftheorem}
\begin{proof}  Dado el sistema $\displaystyle A\left(n+1\right) = rA\left(n\right)+b $. Tenemos que el punto de equilibrio de equilibrio verifica que 
	\[a = f\left(a\right) \Rightarrow a = \frac{b}{r-1} .\]
	Por tanto, se tiene que $\displaystyle \forall n \in \N $, 
	\[ \left|A\left(n\right)-a\right| = \left|rA\left(n-1\right)+b-a\right| = \left|rA\left(n-1\right)+b+ar-b\right| = \left|r\right| \left|A\left(n-1\right)-a\right| .\]
	Apliquemos el procedimiento de inducción sobre $\displaystyle n $. Si $\displaystyle n = 1 $, 
	\[ \left|A\left(1\right) - a\right| = \left|r\right| \left|A\left(0\right)-a\right| .\]
	Si $\displaystyle n = 2 $, 
	\[ \left|A\left(2\right)-a\right| = \left|r\right| \left|A\left(1\right)-a\right| = r^{2} \left|A\left(0\right)-a\right| .\]
	Asumimos que es cierto para $\displaystyle n $. Ahora estudiamos el caso $\displaystyle n+1 $:
	\[ \left|A\left(n+1\right)-a\right| = \left|r\right| \left|A\left(n\right)-a\right| = \left|r\right|^{n+1} \left|A\left(0\right)-a\right| .\]
\begin{description}
	\item[(a)] Si $\displaystyle \left|r\right| < 0 $, tenemos que si $\displaystyle n \to \infty $, $\displaystyle \left|r\right|^{n} \to 0 $ y, en consecuencia, 
	\[ \left|A\left(n\right)-a\right| = \left|r\right|^{n} \left|A\left(0\right)-a\right| \to 0 .\]
\item[(b)] Si $\displaystyle \left|r\right| > 1 $, entonces
	\[\lim_{n \to \infty} \left|A\left(n\right)-a\right| = \lim_{n \to \infty} \left|r^{n}\right| \left|A\left(0\right)-a\right| = \infty .\]
	Por tanto, $\displaystyle a $ es un punto de equilibrio inestable.
\item[(c)] Si $\displaystyle r = - 1 $, tenemos que 
	\[ \left|A\left(n\right)-a\right| = \left|A\left(0\right)-a\right|, \; n \in \N\cup \left\{ 0\right\}  .\]
	Por tanto, se tiene que $\displaystyle a $ es un punto de equilibrio neutro. Además, tenemos que para todo $\displaystyle n $ se verifica que 
	\[A\left(n\right) = - A\left(n-1\right) + b = - \left(-A\left(n-2\right)+b\right)+b = A\left(n-2\right) .\]
	Así, se tiene que
	\[
	\begin{cases}
	A\left(0\right) = A\left(2\right) = \cdots = A\left(2n\right) = \cdots \\
	A\left(1\right) = A\left(3\right) = \cdots = A\left(2n+1\right) = \cdots 
	\end{cases}
	.\]
\end{description}
\end{proof}
\begin{eg}
\normalfont \textbf{Ejercicio 1, Hoja 7.} El día 1 de enero de 2010. El capital inicial en la cuenta bancaria es de 1000 euros. El interés compuesto es de 10\% anual. 
\begin{description}
\item[(a)] Obtener la relación entre la cantidad de dinero que habrá en la cuenta el día 1 de enero de cada año a partir de la que hubera el anterior.
	\begin{center}
	\begin{tabular}{c c}
		1 enero 2010 & $\displaystyle A\left(0\right) = 1000 $ \\
		1 enero 2011 & $\displaystyle A\left(1\right) = 1100 $ \\
		1 enero 2012 & $\displaystyle A\left(2\right) = 1210 $ 
	\end{tabular}
	\end{center}
	Buscamos la relación $\displaystyle A\left(n+1\right) = rA\left(n\right) + b$. Tenemos que 
	\[
	\begin{split}
		A\left(0\right) = & 1000 \\
		A\left(1\right) = & 1000 + 10\% 1000 = 1,1 A\left(0\right) \\
		A\left(2\right) = & A\left(1\right) + 10\% A\left(1\right) = 1,1A\left(1\right)=1,1^{2}A\left(0\right) \\
		\vdots \\
		A\left(n\right) = & 1,1^{n}A\left(0\right).
	\end{split}
	\]
\item[(b)] $\displaystyle A\left(n\right) = 1,1^{n}A\left(0\right) $.
\item[(c)] Calcular el dinero en la cuenta el 1 de enero de 2028.
	\[A\left(18\right) = 1,1^{18}A\left(0\right) = 5559,92 \; \text{euros} .\]	
\end{description}
\end{eg}
\begin{eg}
\normalfont \textbf{Ejercicio 2, Hoja 7.} Un cultivo celular, las células se reproducen por mitosis (de cada célula se obtienen dos nuevas en una unidad de tiempo). 
\begin{description}
\item[(a)] Si inicialmente tenemos $\displaystyle 100 $ células, cuántas habrá para $\displaystyle 6 $ unidades de tiempo y para $\displaystyle n $ unidades? \\
Es evidente que (en el examen hay que demostrarlo por inducción)
\[A\left(n\right) = 2A\left(n-1\right) = 2^{n}A\left(0\right).\]
Para $\displaystyle n = 6 $ se tiene que $\displaystyle A\left(6\right) = 2^{6} \cdot 100 = 6400 $. Para $\displaystyle n = 100 $ tenemos que $\displaystyle A\left(100\right) \approx 10^{30} $. 
\item[(b)] Modelizar el problema nuevamente asumiendo la tasa de natalidad $\displaystyle b \in \R $ y la tasa de mortalidad $\displaystyle d \in \R $. \\
	La nueva solución general será $\displaystyle A\left(n\right) = A\left(n-1\right) + bA\left(n-1\right) - dA\left(n-1\right) = \left(1+b-d\right)A\left(n-1\right) $. Así, tenemos que $\displaystyle A\left(n\right) = \left(1+b-d\right)^{n}A\left(0\right) $.
\end{description}
\end{eg}
\begin{eg}
\normalfont \textbf{Ejercicio 4, Hoja 7.} Hemos hecho un préstamo de $\displaystyle 1000 $ euros al 5\% anual con pagos de 100 euros anuales. Cuánto dinero debemos a los 5 años? \\ 
Tenemos que 
\[
\begin{split}
	A\left(0\right) = & a_{0} \\
	A\left(1\right) = & a_{0} + 5\% a_{0} -100 = 1,05a_{0} - 100 \\
	A\left(2\right) = & a_{1} + 5\% a_{1} -100 = 1,05\left(1,05a_{0}-100\right)-100 = 1,05^{2}a_{0}-1,05 \cdot 100 - 100 \\
	\vdots \\
	A\left(n\right) =& 1,05^{n}a_{0} - 100\left(1,05^{n-1} + \cdots + 1,05 + 1\right) = 1,05^{n}a_{0}-100\sum^{n-1}_{i=0}1,05^{i}
\end{split}
\]
Así, tenemos que $\displaystyle A\left(n\right) = 1,05^{n}a_{0} -100 \cdot \frac{1 - 1,05^{n-2}}{1 - 1,05} $.
\begin{description}
\item[(a)] Deuda para $\displaystyle n = 5 $ con $\displaystyle a_{0}^{1}= 1000 $, $\displaystyle a_{0}^{2} = 2000 $ y $\displaystyle a_{0}^{3} = 3000 $. Tenemos que
	\[
	\begin{split}
	a_{0}^{1} \to A\left(5\right) = 723,72 \\
	a_{0}^{2} \to A\left(5\right) = 2000 \\
	a_{0}^{3} \to A\left(5\right) = 3276,28.
	\end{split}
	\]
\item[(b)] El punto de equilibrio $\displaystyle a $ debe verificar que $\displaystyle f\left(a\right) = a $. Tenemos que 
\[ a = 1,05^{n}a - \frac{1-1,05^{n}}{1 - 1,05} \cdot 100 \Rightarrow a = 100 \cdot \frac{1 - 1,05^{n}}{\left(1-1,05^{n}\right)\left(1-1,05\right)} = 2000 .\]
\end{description}
\end{eg}
\begin{fdefinition}[Gráfica de un sistema]
	\normalfont Dado el sistema dinámico lineal autónomo de primer orden $\displaystyle A\left(n+1\right) = f\left(A\left(n\right)\right) $ con $\displaystyle n \in \N \cup \left\{ 0\right\}  $, con $\displaystyle A\left(0\right) = a_{0} $ dado. Sea la representación gráfica de la curva $\displaystyle y = f\left(x\right) $ e $\displaystyle y = x $. Entonces la sucesión de puntos  
\[ \left\{ \left(A\left(0\right), f\left(A\left(0\right)\right)\right), \ldots, \left(A\left(n\right), f\left(A\left(n\right)\right)\right)\right\}.\]
se denomina \textbf{gráfica del sistema} 
\end{fdefinition}
\begin{ftheorem}[]
\normalfont Sea la representación gráfica del sistema dinámico $\displaystyle A\left(n+1\right) = r A\left(n\right) + b $, dado por la recta $\displaystyle y = rx + b $. El punto de intersección con $\displaystyle y = x $ es $\displaystyle \left(a,a\right) $, con $\displaystyle a $  el punto de equilibrio del sistema. Entonces la sucesión de puntos de la gráfica del sistema verifica:
\begin{description}
\item[(a)] Converge a $\displaystyle \left(a,a\right) $ si $\displaystyle \left|r\right| < 1 $.
\item[(b)] Diverge de $\displaystyle \left(a,a\right) $ si $\displaystyle \left|r\right| > 1 $.
\item[(c)] Oscila en un cuadrado alrededor de $\displaystyle \left(a,a\right) $ si $\displaystyle r = -1 $.
\end{description}
\end{ftheorem}
\begin{proof}
Trivial.
\end{proof}
\begin{eg}
\normalfont Consideremos $\displaystyle A\left(n+1\right) = -0,8A\left(n\right) + 3,6 $, con $\displaystyle A\left(0\right) = 4 $; y $\displaystyle A\left(n+1\right) = -A\left(n\right)+4 $ con $\displaystyle A\left(0\right) = 6 $. En el primer caso, como $\displaystyle \left|-0,8\right| < 1 $, el punto de equilibrio $\displaystyle \left(2,2\right) $ es punto de equilibrio estable. Gráficamente, el sistema converge a $\displaystyle \left(2,2\right) $. En el segundo caso, el punto de equilibrio es también $\displaystyle \left(2,2\right) $, pero no es estable puesto que $\displaystyle \left|-1\right| = 1 $.
\end{eg}
\begin{fdefinition}[]
	\normalfont Una solución $\displaystyle A\left(n\right) $ de un sistema dinámico se dice \textbf{periódica} si existe $\displaystyle m \in \N $ tal que $\displaystyle A\left(n + m\right) = A\left(n\right) $ con $\displaystyle n \in \N \cup \left\{ 0\right\}  $. Se dice que $\displaystyle m $ es el \textbf{periódo} de la solución.
\end{fdefinition}
\subsection{Solución general y particular de los sistemas homogéneos y afines}
Recordamos que un sistema homogéneo es un sistema dinámico de la forma
\[A\left(n + 1\right) = rA\left(n\right), \; r \in \N .\]
\begin{ftheorem}[]
\normalfont La solución general del sistema dinámico lineal homogéneo es
\[ A\left(n\right) = cr^{n}, \; n \in \N \cup \left\{ 0\right\}  ,\]
con $\displaystyle c \in \R $. La solución particular del sistema con $\displaystyle A\left(0\right) = a_{0} $ es 
\[A\left(n\right) = a_{0}r^{n} .\]
\end{ftheorem}
\begin{proof}
Por un lado, tenemos que $\displaystyle \forall c \in \R $, $\displaystyle cr^{n+1} =rcr^{n} $. Por tanto, la sucesión $\displaystyle A\left(n\right) = cr^{n} $ es solución del sistema dinámico. Por otro lado, cualquier otra solución del sistema dinámico debe ser de esa forma. Razonamos por inducción. Tenemos que $\displaystyle A\left(1\right) = rA\left(0\right) $. Asumimos que $\displaystyle A\left(n\right) = r^{n}A\left(0\right) $. Entonces,
\[A\left(n+1\right) = rA\left(n\right) = r^{n+1}A\left(0\right) .\]
\end{proof}
\begin{fcolorary}[]
\normalfont La solución particular del sistema, conocido el valor $\displaystyle A\left(m\right) = a_{m} $ para cierto $\displaystyle m \in \N $ es 
\[A\left(n\right) = \frac{a_{m}}{r^{m}}r^{n} = a_{m}r^{n-m}, \; n \in \N \cup \left\{ 0\right\}  .\]
\end{fcolorary}
\begin{proof}
Tenemos que 
\[A\left(m\right) = a_{0}r^{m} = a_{m} \Rightarrow a_{0} = \frac{a_{m}}{r^{m}} .\]
Por tanto,
\[A\left(n\right) = a_{0}r^{n} = \frac{a_{m}}{r^{m}}r^{n} = a_{m}r^{n - m} .\]
\end{proof}
\begin{ftheorem}[]
\normalfont Sea el sistema dinámico lineal afín $\displaystyle A\left(n+1\right) = rA\left(n\right) + b $, con $\displaystyle r,b \in \R $. La solución general es
\[A\left(n+1\right) = 
\begin{cases}
cr^{n} - \frac{b}{1-r}, \; r \neq 1\\
c + nb, \; r = 1
\end{cases}
.\]
La solución particular con $\displaystyle a_{0} = A\left(0\right) $ es
\[A\left(n\right) =
\begin{cases}
	\left(a_{0}-\frac{b}{1-r}\right)r^{n}+\frac{b}{1-r}, \; r \neq 1 \\
	a_{0} + nb, \; r = 1
\end{cases}
.\]
\end{ftheorem}
\begin{proof}
El punto de equilibrio del sistema dinámico afín $\displaystyle A\left(n+1\right) = rA\left(n\right) + b $ es 
\[ a = ra + b \Rightarrow a = \frac{b}{1-r}, \; r \neq 1 .\]
Consideremos el sistema dinámico $\displaystyle E\left(n\right) = A\left(n\right)-a $. Así, tenemos que
\[ E\left(n+1\right) = A\left(n+1\right)-a = rA\left(n\right) + b - \left(ra + b\right) = r\left(A\left(n\right)-a\right) = rE\left(n\right).\]
Por tanto, tenemos que $\displaystyle E\left(n\right) $ es un sistema dinámico homogéneo, por lo que su solución general será
\[E\left(n\right) = cr^{n}, \; c \in \R, n \in \N \cup \left\{ 0\right\}  .\]
Por tanto, 
\[ E\left(n\right) = A\left(n\right) - a = cr^{n} \Rightarrow A\left(n\right) = cr^{n} + a = cr^{n} + \frac{b}{1-r}, \; r \neq 1.\]
Si $\displaystyle A\left(0\right) = a_{0} $, se tiene que $\displaystyle E\left(0\right) = a_{0} - a $. Así,
\[E\left(n\right) = \left(a_{0} - a\right)r^{n} = A\left(n\right) + a \Rightarrow A\left(n\right) = \left(a_{0} - a\right)r^{n} + a= \left(a_{0}-\frac{b}{1-r}\right)r^{n} + \frac{b}{1-r}.\]

Si $\displaystyle r = 1 $, se tiene que el sistema se escribe $\displaystyle A\left(n+1\right) = A\left(n\right) + b $. Por inducción \footnote{Se trata de una sucesión artimética.} obtenemos que
\[A\left(n\right) = a_{0} + nb .\]
\end{proof}
\section{Sistemas dinámicos lineales de segundo orden}
Son sistemas de la forma
\[A\left(n+2\right) = aA\left(n+1\right) + bA\left(n\right) + c ,\]
con $\displaystyle a,b,c \in \R $. En este caso también diferenciamos en sistemas homogéneos y afines:
\begin{itemize}
\item Si $\displaystyle c = 0 $ se dice que el sistema es \textbf{homogéneo}.
\item Si $\displaystyle c \neq 0 $ se dice que el sistema es \textbf{afín}.
\end{itemize}
\subsection{Sistemas dinámicos de segundo orden homogéneos}
\begin{fdefinition}[Ecuación y polinomio característicos]
\normalfont La \textbf{ecuación característica} asociada a un sistema dinámico lienal de segundo orden homogéneo $\displaystyle A\left(n+2\right) = aA\left(n+1\right) + bA\left(n\right) $, con $\displaystyle a, b \in \R $, es $\displaystyle x^{2} = ax + b $. El \textbf{polinomio característico} de este sistema es el polinomio de segundo grado $\displaystyle P\left(x\right) = x^{2}-ax - b $.
\end{fdefinition}
\begin{ftheorem}[]
\normalfont Sea el sistema lineal homogéneo de segundo orden $\displaystyle A\left(n+2\right) = aA\left(n+1\right) + bA\left(n\right) $, con $\displaystyle a,b \in \R $, y sean $\displaystyle r,s \in \R $ con $\displaystyle r \neq s $ las soluciones del polinomio característico (reales o complejas). Entonces
\begin{description}
\item[(a)] La solución general del sistema es
	\[A\left(n\right) = c_{1}r^{n} + c_{2}s^{n} ,\]
	donde $\displaystyle c_{1}, c_{2} \in \R $.
\item[(b)] La solución particular del sistema con $\displaystyle A\left(0\right) = a_{0} $ y $\displaystyle A\left(1\right) = a_{1} $ será
	\[ A\left(n\right) = \frac{a_{1}-sa_{0}}{r-s}r^{n} + \frac{a_{1}-ra_{0}}{r-s}s^{n} .\]
\end{description}
\end{ftheorem}
\begin{proof}
\begin{description}
\item[(a)] Hacemos uso de la técnica de inducción. Primero, veamos que la solución del sistema se escribe de la forma $\displaystyle A\left(n\right) = c_{1}r^{n} + c_{2}s^{n} $. Sean $\displaystyle r $ y $\displaystyle s $ las soluciones del polinomio característico con $\displaystyle r \neq s $. Es decir, verifican que
	\[ r^{2}-ar - b = 0, \quad \text{y} \quad s^{2}-as-b = 0.\]
En esta situación, $\displaystyle A\left(0\right) $ y $\displaystyle A\left(1\right) $ se puede escribir de la forma
\[A\left(n\right) =  .\]
\end{description}
\end{proof}
\begin{ftheorem}[]
\normalfont Sea el sistema dinámico lineal de segundo orden, $\displaystyle A\left(n+2\right) = aA\left(n+1\right) + bA\left(n\right) $, y sea $\displaystyle r \neq 0 $ la única raíz de $\displaystyle P\left(x\right) $ (es decir, raíz doble). Entonces,
\begin{description}
\item[(a)] La solución general del sistema dinámico es de la forma
	\[A\left(n\right) = \left(c_{1}+c_{2}n\right)r^{n} , \; n \in \N \cup \left\{ 0\right\} .\]
\item[(b)] La solución particular del sistema que satisface $\displaystyle A\left(0\right) = a_{0} $ y $\displaystyle A\left(1\right) = a_{1} $ es
	\[A\left(n\right) = \left(a_{0} + \frac{a_{1}-ra_{0}}{r}n\right)r^{n}, \; n \in \N \cup \left\{ 0\right\}  .\]
\end{description}
\end{ftheorem}
\begin{proof}
\begin{description}
\item[(a)] Dado el sistema dinámico $\displaystyle A\left(n+2\right) = aA\left(n+1\right) + bA\left(n\right) $ de polinomio característico $\displaystyle P\left(x\right) = x^{2}-ax-b $, si $\displaystyle r \neq 0 $ es la única solución posible, tenemos que
\[x^{2} - ax - b = \left(x-r\right)^{2} = x^{2}-2xr + r^{2} .\]
Así, tenemos que $\displaystyle a = 2r $ y $\displaystyle b = - r^{2} $. Por tanto, tenemos que el sistema se puede escribir de la forma
\[A\left(n+2\right) = 2rA\left(n+1\right)-r^{2}A\left(n\right) .\]
Supongamos ahora que existe un sistema dinámico $\displaystyle B\left(n\right) $ tal que $\displaystyle A\left(n\right) = B\left(n\right)r^{n} $, que debe verificar que 
\[A\left(n+2\right) = B\left(n+2\right)r^{n} = 2r\left(B\left(n+1\right)r^{n+1}\right)-r^{2}B\left(n\right)r^{n} .\]
Por tanto, $\displaystyle B\left(n+2\right) = 2B\left(n+1\right)-B\left(n\right) $. Veamos, por inducción, si $\displaystyle B\left(n\right) $ tiene solución de la forma $\displaystyle B\left(n\right) = c_{1} + c_{2}n $. Para $\displaystyle n = 0 $ y $\displaystyle n = 1 $ se tiene que, si fuera cierto,
\[
\begin{split}
	B\left(0\right) = & c_{1} + 0c_{2} \\
	B\left(1\right) = & c_{1} + 1c_{2}.
\end{split}
\]
Como este sistema tiene solución única, es claro que $\displaystyle B\left(0\right) $ y $\displaystyle B\left(1\right) $ verifican solución. Supongamos que $\displaystyle B\left(n\right)  $ y $\displaystyle B\left(n+1\right) $ también verifican la solución y comprobemos si entonces $\displaystyle B\left(n+2\right) $ también es de la forma deseada.
\[B\left(n+2\right) = 2B\left(n+1\right) - B\left(n\right) = 2\left(c_{1} + c_{2}\left(n+1\right)\right)-\left(c_{1} + c_{2}n\right) = c_{1}+2c_{2}\left(n+2\right) .\]
Así, se tendría que 
\[A\left(n\right) = B\left(n\right) r^{n} = \left(c_{1} + c_{2}n\right)r^{n} .\]
\item[(b)] Busquemos ahora la solución particular para $\displaystyle A\left(0\right) = a_{0} $ y $\displaystyle A\left(1\right) = a_{1} $. Si $\displaystyle n = 0 $ y $\displaystyle n = 1 $:
	\[
	\begin{split}
		n = 0 \Rightarrow & \left(c_{1} + 0c_{2}\right)r^{0} = c_{1} = a_{0} \\
		n = 1 \Rightarrow & \left(c_{1} + 1c_{2}\right)r^{1} = rc_{1} + rc_{2} = a_{1}.
	\end{split}
	\]
Así, tenemos que $\displaystyle c_{1} = a_{0}$ y $\displaystyle c_{2} = \frac{a_{1}-ra_{0}}{r} $, por lo que
\[A\left(n\right) = \left(a_{0} + \frac{a_{1}-ra_{0}}{r}n\right)r^{n}, \; n \in \N \cup \left\{ 0\right\}  .\]
\end{description}
\end{proof}
\subsection{Sistemas dinámicos de segundo orden afines}
\begin{ftheorem}[]
\normalfont Sea el sistema lineal afín de segundo orden $\displaystyle A\left(n+2\right) = aA\left(n+1\right) + bA\left(n\right) + c $ y sean $\displaystyle r $ y $\displaystyle s $ las soluciones del polinomio característico asociado $\displaystyle P\left(x\right) = x^{2}-ax - b $. La solución general del sistema es
\[A\left(n\right) = 
\begin{cases}
c_{1}r^{n} + c_{2}s^{n} + \frac{c}{1 - a - b}, \; r \neq s \land r,s \neq 1 \\
\left(c_{1} + c_{2}n\right)r^{n} + \frac{c}{1-a-b}, \; r = s \neq 1 \\
\left(c_{1}r^{n}+c_{2}\right) + \frac{c}{2-a}n, \; r\neq 1 \land s = 1 \\
\left(c_{1}+c_{2}n\right) + \frac{cn^{2}}{2}, \; r = s = 1
\end{cases}
.\]
\end{ftheorem}
\begin{proof}
Dadas dos soluciones $\displaystyle A\left(n\right) $ y $\displaystyle B\left(n\right) $ del sistema, tales que 
\[A\left(n+2\right) = aA\left(n+1\right) + bA\left(n\right) + c .\]
\[B\left(n+2\right) = aB\left(n+1\right) + bB\left(n\right) + c .\]
Es claro que su diferencia
\[ .\]
\[
\begin{split}
	D\left(n+2\right) = & A\left(n+2\right)-B\left(n+2\right) = a\left(A\left(n+1\right)-B\left(n+1\right)\right) + b\left(A\left(n\right)-B\left(n\right)\right) \\
	= &  a D\left(n+1\right) + bD\left(n\right).
\end{split}
\]
es un sistema homogéneo. Así, dada una solución del sistema lineal homogéneo asociado al sistema afín, podemos construir la solución general a partir de ella y una particular, con $\displaystyle A\left(n\right) $ la solución del sistema homogéneo asociado y $\displaystyle A_{p}\left(n\right) $ una solución particular del sistema afín. Para buscar $\displaystyle A_{p}\left(n\right) $, vemos en primer lugar que si una raíz de $\displaystyle P\left(x\right) = 1 $, entonces 
\[P\left(1\right) = 0 = 1 - a - b \Rightarrow b = 1 - a .\]
Si $\displaystyle s = 1 $ y $\displaystyle r = a - 1 $, se tendrán las soluciones. Si $\displaystyle r = s = 1 $, que ocurre cuando $\displaystyle a = 2  $ y $\displaystyle b = - 1 $. Otra posibilidad es que $\displaystyle r \neq 1 $ y $\displaystyle s = 1 $ si $\displaystyle  b = 1 - a $ y $\displaystyle a \neq 2 $. Si $\displaystyle r \neq 1 $ y $\displaystyle s \neq 1 $, si $\displaystyle  b=1-a $ y $\displaystyle a \neq 1 $.
\end{proof}
\begin{observation}
\normalfont Para encontrar la solución particular con datos $\displaystyle A\left(0\right) = a_{0} $ y $\displaystyle A\left(1\right) = a_{1} $, basta con resolver el sistema de ecuaciones de incógnitas $\displaystyle c_{1} $ y $\displaystyle c_{2} $.
\end{observation}
\begin{eg}
\normalfont Calcular la TAE, de una cuenta con capitalización cada $\displaystyle m $ meses de intereses con interés $\displaystyle 3\% $. Tendríamos que el sistema dinámico sería
\[A\left(n+1\right) = \left(1+\frac{I}{m}\right)A\left(n\right) + b ,\]
con $\displaystyle b $ siendo la cantidad aportar a detraer al finalizar cada período. La solución particular será
\[A\left(n\right) = \left(1 + \frac{I}{m}\right)^{n}\left(a_{0} + \frac{mb}{I}\right)-\frac{mb}{I} .\]
En este caso se tienen $\displaystyle m = 4 $ períodos de pago, $\displaystyle I = \frac{3}{100} $ $\displaystyle b = 0 $. A 12 meses,
\[A\left(4\right) = \left(1 + \frac{0,03}{4}\right)^{4} a_{0}= 1,0304 a_{0} .\]
Así, $\displaystyle \text{TAE} = 0,0304 = 3,04 \% $.
\end{eg}
\section{Sistemas no lineales}
\begin{fdefinition}[]
	\normalfont Son aquellos sistemas de la forma $\displaystyle A\left(n+1\right) = f\left(A\left(n\right), n\right) $, $\displaystyle n \in \N\cup \left\{ 0\right\}  $, con $\displaystyle f $ no lineal.
\end{fdefinition}
\begin{eg}
\normalfont Consideremos el sistema $\displaystyle A\left(n+1\right) = - A\left(n\right)^{3} + 0,8A\left(n\right) $. Calculamos el punto de equilibrio con $\displaystyle a = f\left(a\right) $:
\[a = -a^{3} -0,8a \Rightarrow a = 0 .\]
Existen otras dos raíces complejas conjugadas. Si $\displaystyle A\left(0\right) = 1,2 $, la sucesión de puntos de la gráfica converge a $\displaystyle \left(a,a\right) $, por lo que es estable. Esto no es cierto para $\displaystyle A\left(0\right)=1,4 $.
\end{eg}
\begin{ftheorem}[]
\normalfont Sea $\displaystyle a $ el punto de equilibrio de un sistema autónomo $\displaystyle A\left(n+1\right) = f\left(A\left(n\right)\right) $ no lineal, que tenga derivada continua, entonces
\begin{description}
\item[(a)] $\displaystyle a $ es estable si $\displaystyle \left|f'\left(a\right)\right| < 1 $.
\item[(b)] $\displaystyle a $ es inestable si $\displaystyle \left|f'\left(a\right)\right| > 1 $.
\end{description}
\end{ftheorem}
\begin{proof}
\begin{description}
\item[(a)] Si $\displaystyle \left|f'\left(a\right)\right|<1 $, tenemos que $\displaystyle \exists \epsilon > 0, b \in \left( \left|f'\left(a\right)\right|, 1\right) $ tal que $\displaystyle \left|f'\left(x_{0}\right)\right|<b $ para $\displaystyle x_{0} \in \left(a-\epsilon, a + \epsilon \right) $. Llamamos $\displaystyle I = \left(a - \epsilon , a + \epsilon \right) $ y elegimos $\displaystyle A\left(0\right) \in I $. Veamos que $\displaystyle \left|A\left(1\right)-a\right| < \left|A\left(0\right)-a\right| $ y que iterativamente tendremos que $\displaystyle \lim_{n \to \infty} \left|A\left(n\right)-a\right| = 0 $.
Por definición, siendo $\displaystyle a $ punto de equilibrio, entonces por definición $\displaystyle a = f\left(a\right) $. Además, $\displaystyle A\left(1\right) = f\left(A\left(0\right)\right) $. Por tanto, 
\[ \left|A\left(1\right)-a\right| = \left|f\left(A\left(0\right)\right)-f\left(a\right)\right| .\]
Por el teorema del valor medio, como $\displaystyle f $ es continua $\displaystyle \exists x_{0}  $ entre $\displaystyle a $ y $\displaystyle A\left(0\right) $ tal que 
\[ f\left(A\left(0\right)\right)-f\left(a\right) = f'\left(x_{0}\right) \left(A\left(0\right)-a\right) .\]
Habiendo elegido $\displaystyle A\left(0\right) \in I $, entonces $\displaystyle x_{0} \in I $ y, consecuentemente, $\displaystyle \left|f'\left(x_{0}\right)\right| < b $. Así
\[ \left|A\left(1\right)-a\right| = \left|f'\left(x_{0}\right)\right| \left|A\left(0\right)-a\right| < b\left|A\left(0\right)-a\right|.\]
Así, tenemos que $\displaystyle A\left(1\right) \in I $, por lo que podemos repetir el proceso para demostrar lo mismo para $\displaystyle A\left(2\right) $. Así, iterando
\[ \left|A\left(n\right)-a\right| < \cdots < b^{n-1}\left|A\left(1\right)-a\right|< b^{n}\left|A\left(0\right)-a\right| .\]
Por ser $\displaystyle b < 1 $, se tiene que $\displaystyle \lim_{n \to \infty} \left|A\left(n\right)-a\right| = 0 $.
\item[(b)] Procediendo de la misma forma que \textbf{(a)}, siendo $\displaystyle \left|f'\left(a\right)\right| > 1 $, entonces existen $\displaystyle \epsilon > 0 $ y $\displaystyle b \in \left(1, \left|f'\left(a\right)\right|\right) $ tales que $\displaystyle \left|f'\left(x\right)\right| > b $, $\displaystyle \forall x \in \left(a - \epsilon, a + \epsilon \right) $. Elegimos ahora $\displaystyle A\left(0\right) \in I = \left(a-\epsilon, a+ \epsilon \right) $.
	Existirá cierto $\displaystyle n \in \N $ con $\displaystyle A\left(n\right) \not\in I $ y $\displaystyle a $ será inestable. En efecto, siendo $\displaystyle a $ de equilibrio tenemos que $\displaystyle a = f\left(a\right) $ y $\displaystyle A\left(1\right) = f\left(A\left(0\right)\right) $. Entonces,
	\[ \left|A\left(0\right)-a\right| = \left|f\left(A\left(0\right)\right)-f\left(a\right)\right| .\]
Por el teorema del valor medio, existe $\displaystyle x_{0} \in I $ tal que 
\[ \left|A\left(1\right)-a\right| = \left|f'\left(x_{0}\right)\right| \left|A\left(0\right)-a\right| > b \left|A\left(0\right)-a\right| .\]
Ahora pueden pasar dos cosas:
\begin{itemize}
\item Si $\displaystyle A\left(1\right) \not\in I $, hemos ganado.
\item Si $\displaystyle A\left(1\right) \in I $, iteramos obteniendo $\displaystyle A\left(2\right) $ tal que
	\[ \left|A\left(2\right) - a\right| > b \left|A\left(1\right) - a\right| > b^{2} \left|A\left(0\right)-a\right| .\]
	Si $\displaystyle A\left(2\right) \not\in I $, hemos ganado. En caso contrario, seguimos iterando.
\end{itemize}
 Como $\displaystyle I $ es finito, entonces iterando el procedimiento existirá cierto $\displaystyle n \in \N $ tal que $\displaystyle A\left(n\right) \not\in I $ o bien $\displaystyle \left|A\left(n+1\right)-a\right| > b^{n+1} \left|A\left(0\right)-a\right| $. Dado que $\displaystyle \lim_{n \to \infty}b^{n}=\infty $, entonces se tendrá que existe $\displaystyle n \in \N $ tal que $\displaystyle A\left(n\right) \not\in I $ o $\displaystyle A\left(n+1\right) \not\in I $, por lo que el sistema será inestable.
\end{description}
\end{proof}
\begin{observation}
\normalfont Si $\displaystyle \left|f'\left(a\right)\right| = 1 $, el resultado no es concluyente. Una forma de obtener más información es estudiando las derivadas sucesivas, aunque aquí no lo haremos.
\end{observation}
\begin{eg}
\normalfont 
\begin{description} 
\item[(a)] Continuación de $\displaystyle A\left(n+1\right) = -A\left(n\right)^{3}-0,8A\left(n\right) $. Teníamos que $\displaystyle a=0 $ era el único punto de equilibrio y $\displaystyle f'\left(x\right) = -3x^{2}-0,8 $. Se tiene que $\displaystyle \left|f'\left(0\right)\right| = 0,8 < 1 $, por lo que el sistema es estable.
\item[(b)] Consideremos $\displaystyle A\left(n+1\right) = A\left(n\right)^{3} - A\left(n\right)^{2} + 1 $, $\displaystyle n \in \N \cup \left\{ 0\right\}  $. Calculamos los puntos de equilibrio,
	\[a = a^{3}-a^{2}+1 \Rightarrow a^{3}-a^{2}-a+1=0 \Rightarrow a = \pm 1.\]
	Tenemos que $\displaystyle f'\left(x\right) = 3x^{2}-2x $. Así, $\displaystyle \left|f'\left(1\right)\right| = 1 $, por lo que no podemos decidir, y $\displaystyle \left|f'\left(-1\right)\right| = 5 > 1 $ por lo que es inestable. Por observación gráfica, se puede ver que en $\displaystyle a = 1 $, es estable por la izquierda e inestable por la derecha, por lo que es \textbf{semiestable}. Es decir $\displaystyle a = 1 $ es un \textbf{punto de equilibrio semiestable}.
\end{description}
\end{eg}
\begin{observation}
\normalfont Cuando un punto de equilibrio es estable, es útil calcular el \textbf{intervalo maximal} $\displaystyle \left(c,d\right) $ en el que se verifique que
\[\lim_{n \to \infty}A\left(n\right) = a, \; \forall A\left(0\right) \in \left(c,d\right) .\]
\end{observation}

