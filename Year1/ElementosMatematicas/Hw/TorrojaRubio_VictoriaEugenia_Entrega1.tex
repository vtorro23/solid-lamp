\documentclass{article}

% packages

\usepackage{graphicx} % Required for images
\usepackage[spanish]{babel}
\usepackage{mdframed}
\usepackage{amsthm}
\usepackage{amssymb}
\usepackage{fancyhdr}
\usepackage{amsmath}
\usepackage{geometry}[margin=1in]
\usepackage{pgfplots}
\usepackage{url}
\usepackage{float}

% for math environments

\theoremstyle{definition}
\newtheorem*{theorem}{Teorema}
\newtheorem*{definition}{Definición}
\newtheorem*{prop}{Proposición}
\newtheorem*{observation}{Observación}
\newtheorem{ej}{Ejercicio}
\newtheorem{sol}{Solución}

% for headers and footers

\pagestyle{fancy}

%\fancyhead[R]{Victoria Eugenia Torroja}
% Store the title in a custom command
\newcommand{\mytitle}{}

% Redefine \title to store the title in \mytitle
\let\oldtitle\title
\renewcommand{\title}[1]{\oldtitle{#1}\renewcommand{\mytitle}{#1}}

% Set the center header to the title
\lhead{\mytitle}

% Custom commands

\newcommand{\R}{\mathbb{R}}
\newcommand{\C}{\mathbb{C}}
\newcommand{\F}{\mathbb{F}}
\newcommand{\N}{\mathbb{N}}
\newcommand{\Q}{\mathbb{Q}}
\newcommand{\Z}{\mathbb{Z}}
\newcommand{\K}{\mathbb{K}}
\newcommand{\mcd}{\text{mcd}}
\newcommand{\mcm}{\text{mcm}}
\DeclareMathOperator{\Ker}{Ker}
\DeclareMathOperator{\Imagen}{Im}
\DeclareMathOperator{\ord}{ord}
\DeclareMathOperator{\GL}{GL}
\DeclareMathOperator{\Biy}{Biy}


\begin{document}

\title{Elementos de Matemáticas y Aplicaciones - Entrega 1}
\author{Victoria Eugenia Torroja Rubio}
\date{\today}

\maketitle

\begin{ej}
Establece criterios de divisibilidad entre 2, 3, 4, 5, 6, 7 y 8 de un número en función de los dígitos de su expresión en base 8. \\ 
Por ejemplo, un número es múltiplo de 2 cuando la última cifra de su expresión en base 8 es 0, 2, 4 o 6.
\end{ej}

\begin{sol}
\begin{description}
\item[Criterio del 2.] El criterio de divisibilidad del 2 en base 8 nos viene dado en el enunciado, por tanto, solo tenemos que demostrarlo. Sea $\displaystyle n \in \N $ y sea 
	\[n = a_{k}8^{k} + \cdots + a_{1}8+a_{0} = \sum^{k}_{i=0}a_{i}8^{i} \]
	su expresión en base $\displaystyle 8 $. Si $\displaystyle a_{0} \in \left\{0,2,4,6\right\}  $, $\displaystyle a_{0}\equiv 0 \mod 2 $. Como $\displaystyle 8 \equiv 0 \mod 2 $,
	\[n \equiv a_{k}8^{k} + \cdots + a_{1}8+a_{0} \equiv a_{k} \cdot 0 + \cdots + a_{1} \cdot 0+ a_{0} \equiv 0 \mod 2 .\]
	Recíprocamente, si $\displaystyle a_{0} \not\in \left\{ 0,2,4,6\right\}  $ no sería divisible entre 2, pues el resto de elementos de $\displaystyle \Z_{8} $ son impares y, por ello, indivisibles entre 2.
\item[Criterio del 3.] Tenemos que $\displaystyle 8\equiv 2 \mod 3 $, $\displaystyle 8^{2} \equiv 1\mod 3 $ y, en general
	\[8^{2k} \equiv \left(8^{2}\right)^{k} \equiv 1 \mod 3 \quad \text{y} \quad 8^{2k+1} \equiv 8^{2k} \cdot 8 \equiv 2 \mod 3, \; \text{con} \; k \in \N .\]
Entonces, el criterio de divisibilidad del 3 consiste en que el doble de las cifras de posición par más la suma de las cifras de posición impar \footnote{ En todos los casos, los coeficientes de $\displaystyle n $ en base 8 son elementos de $\displaystyle \Z_{8} $. Además, las cifras de posición impar son aquellas con subíndice par y las cifras de posición par son aquellas de subíndice impar.} debe ser múltiplo de 3 (las cifras están en base 8). Es decir, asumiendo, sin pérdida de generalidad, que $\displaystyle n = a_{2k}8^{2k} + \cdots + a_{1}8^{1} + a_{0} $ es necesario y suficiente que,
\[ 
\begin{split}
n \equiv a_{2k}8^{2k} + \cdots + a_{1}8^{1} + a_{0} \equiv \sum^{k}_{i=0}a_{2i}8^{2i} + \sum^{k}_{i=1}a_{2i-1}8^{2i-1} \equiv \sum^{k}_{i=0}a_{2i} + 2\sum^{k}_{i=1}a_{2i-1} \equiv 0 \mod 3
\end{split}
\]

\item[Criterio del 4.] Tenemos que $\displaystyle 8 \equiv 0 \mod 4 $. Por tanto, para que un número $\displaystyle n $ sea divisible entre 4, su última cifra en base 8 tiene que ser 0 o 4. En efecto,si $\displaystyle a_{0} \in \left\{ 0,4\right\}  $,
	\[n \equiv a_{k}8^{k} + \cdots + a_{1}8+a_{0} \equiv a_{k} \cdot 0 + \cdots + a_{1} \cdot 0 + a_{0} \equiv 0 \mod 4 .\]
	Si $\displaystyle a_{0} \not\in \left\{ 0,4\right\}  $, entonces no sería divisible entre 4, pues en $\displaystyle \Z_{8} $ los únicos elementos divisibles entre 4 son 0 y 4.
\item[Criterio del 5.] Tenemos que $\displaystyle 8 \equiv 3 \mod 5 $, $\displaystyle 8^{2} \equiv 4 \mod 5 $, $\displaystyle 8 ^{3} \equiv 2 \mod 5 $ y $\displaystyle 8^{4} \equiv 1 \mod 5 $. En general, tenemos que si $\displaystyle k \in \N $,
	\[
	\begin{split}
	8^{4k+1} \equiv 8^{4k} \cdot 8 \equiv 3 \mod 5 \\
	8 ^{4k+2} \equiv 8^{4k} \cdot 8^{2} \equiv 4 \mod 5 \\
	8 ^{4k+3} \equiv 8^{4k} \cdot 8^{3} \equiv 2 \mod 5 \\
	8^{4k+4} \equiv 8^{4k} \cdot 8^{4} \equiv 1 \mod 5.
	\end{split}
	\]
Así, podemos deducir que el criterio de divisibilidad del 5 en base 8 consiste en que la suma de sus cifras que vayan con un exponente de 8 divisible entre 4, más el triple de la suma de las cifras que vayan acompañadas con $\displaystyle 8^{k} $ tal que $\displaystyle k \equiv 1 \mod 4 $, más cuatro veces la suma de las cifras acompañadas con un $\displaystyle 8^{k} $ con $\displaystyle k\equiv 2 \mod 4 $, más el doble de la suma de las cifras acompañadas con un $\displaystyle 8^{k} $ donde $\displaystyle k \equiv 3 \mod 4 $, sea múltiplo de $\displaystyle 5 $. Asumamos, sin pérdida de generalidad, que $\displaystyle n = a_{4k}8^{4k} + \cdots + a_{1}8 + a_{0} $, entonces debe cumplirse que
\[ 
\begin{split}
	n \equiv &  a_{4k}8^{4k} + \cdots + a_{1}8+a_{0} \equiv \sum^{k}_{i = 0}a_{4i}8^{4i} + \sum^{k}_{i=1}a_{4i-3}8^{4i-3} + \sum^{k}_{i=1}a_{4i-2}8^{4i-2} + \sum^{k}_{i=1}a_{4i-1}8^{4i-1} \\
	\equiv & \sum^{k}_{i = 0}a_{4i} + 3\sum^{k}_{i=1}a_{4i-3} + 4\sum^{k}_{i=1}a_{4i-2} + 2\sum^{k}_{i=1}a_{4i-1} \equiv 0 \mod 5.
\end{split}\]
\item[Criterio del 6.] El criterio de divisibilidad del 6 consiste en que 4 veces la suma de las cifras de posición impar más dos veces la suma de las cifras de posición par sumen múltiplo de 6. Para demostrar esto vamos a demostrar en primer lugar que $\displaystyle 8^{2k} \equiv 4 \mod 6 $, donde $\displaystyle k \in \N $. Usamos el método de inducción. Tenemos que $\displaystyle 8^{2} \equiv 64 \equiv 4 \mod 6 $. Asumimos que $\displaystyle 8^{2k} \equiv 4 \mod 6 $. En el caso $\displaystyle 2\left(k+1\right) $:
	\[8^{2\left(k+1\right)}\equiv8^{2k} \cdot 8^{2} \equiv 4 \cdot 4 \equiv 4 \mod 6 .\]
Así, $\displaystyle \forall k \in \N $, $\displaystyle 8^{2k} \equiv 4 \mod 6 $. A partir de este resultado es fácil deducir que 
\[8^{2k+1} \equiv 8^{2k} \cdot 8 \equiv 4 \cdot 8 \equiv 2 \mod 6, \; \forall k \in \N .\]
Es decir, asumamos sin pérdida de generalidad que la mayor potencia de 8 de $\displaystyle n $ sea par:
\[ n \equiv a_{2k}8^{2k} + \cdots + a_{1}8+a_{0} \equiv \sum^{k}_{i=0}a_{2i}8^{2i} + \sum^{k}_{i = 1}a_{2i-1}8^{2i-1} \equiv 4 \sum^{k}_{i=0}a_{2i} + 2\sum^{k}_{i=1}a_{2i-1} \mod 6 .\]
\item[Criterio del 7.] Dado que $\displaystyle 8 \equiv 1 \mod 7$, para que $\displaystyle n $ sea divisible entre 7 basta que la suma de sus cifras en base 8 sea múltiplo de 7. En efecto,
	\[n \equiv a_{k}8^{k} + \cdots + a_{1}8+a_{0} \equiv a_{k} + \cdots + a_{1} + a_{0} \equiv 0 \mod 7 .\]
\item[Criterio del 8.] Para que un número $\displaystyle n $ sea divisible entre 8, su última cifra en base 8 debe ser 0 (no puede ser 8 pues las cifras son elementos de $\displaystyle \Z_{8} $). En efecto, si $\displaystyle a_{0} = 0 $,
	\[n \equiv a_{k}8^{k} + \cdots + a_{1}8+a_{0} \equiv a_{k} \cdot 0 + \cdots + a_{1} \cdot 0 + a_{0} \equiv 0 \mod 8 .\]
	Si $\displaystyle a_{0} \neq 0 $, como $\displaystyle a_{0} \in \Z_{8} $ y ningún otro elemento de $\displaystyle \Z_{8} $ es divisible entre 8, tenemos que $\displaystyle n \not \equiv 0 \mod 8 $.
\end{description}

\end{sol}

\begin{ej}
El ISBN es un código numérico que identifica cada libro. 
\begin{description}
\item[(a)] Utilizando la aritmética modular, describe el algoritmo utilizado para calcular el dígito de control del ISBN de 13 dígitos.
\item[(b)] Aplicar el algoritmo anterior para calcular el dígito de control de un libro en el que se ha borrado. El resto del ISBN se ve correctamente: 978-84-131-8779-*. 
\end{description}
\end{ej}

\begin{sol}
\begin{description}
\item[(a)] El algoritmo que se utiliza para calcular el dígito de control del ISBN de 13 dígitos consiste en multiplicar las cifras con posición impar por 1 y las cifras con posición par por 3 (las posiciones se cuentan de izquierda a derecha), sumar todos estos productos (a este valor lo denotaremos $\displaystyle s $), y encontrar el primer número $\displaystyle n  \in \N$ que cumple $\displaystyle s + n \equiv 0 \mod 10 $. Es decir el número $\displaystyle n $  que buscamos cumple que
	\[s + n \equiv 0 \mod 10 \iff n \equiv - s \mod 10 .\]
	Fuente: \url{https://www.grupoalquerque.es/mate_cerca/paneles_2012/168_ISBN2.pdf}	
\item[(b)] En este caso, el valor de $\displaystyle s $ será:
	\[s = 9 + 3 \cdot 7 + 8 + 3 \cdot 8 + 4 + 3 \cdot 1 + 3 + 3 \cdot 1 + 8 + 3 \cdot 7 + 7 + 3 \cdot 9 = 138 .\]
Así, $\displaystyle n \equiv - s \equiv - 138 \equiv 2 \mod 10 $. Por tanto, el dígito que buscamos es 2.	
\end{description}
\end{sol}

\begin{ej}
\begin{description}
\item[(a)] Encriptar el mensaje: 'CARA', según el algoritmo RSA con clave pública $\displaystyle \left(n,k\right) = \left(65,5\right) $.
\item[(b)] Comprobar que se recupera el mensaje original, desencriptando el resultado obtenido en \textbf{(a)}. 
\end{description}
\end{ej}

\begin{sol}
\begin{description}
\item[(a)] En primer lugar, debemos codificar el mensaje (en este caso vamos a usar la tabla que no contiene a la 'ñ'). Entonces, tenemos que el mensaje 'CARA' se traduce a 
	\[12 \;\; 10 \;\; 27 \;\; 10 .\]
A continuación, encriptamos la codificación elevando cada número a $\displaystyle 5 $ y reduciéndolo módulo 65.
\[
\begin{split}
12^{5} \equiv \left(12^{2}\right)^{2} \cdot 12 \equiv 144^{2} \cdot 12 \equiv 14^{2} \cdot 12 \equiv 12 \mod 65 \\
10^{5} \equiv \left(10^{2}\right)^{2} \cdot 10 \equiv 100^{2} \cdot 10 \equiv 35^{2} \cdot 10 \equiv 30 \mod 65 \\
27^{5} \equiv \left(27^{2}\right)^{2} \cdot 27 \equiv 729^{2} \cdot 27 \equiv 14^{2} \cdot 27 \equiv 27 \mod 65.
\end{split}
\]
De esta manera, el texto cifrado que obtenemos es el siguiente:
\[12 \; \; 30 \; \; 27 \; \; 30 .\]
\item[(b)] Para recuperar el mensaje transmitido, debemos encontrar el inverso de 5 módulo $\displaystyle \varphi\left(65\right) $  y elevar cada uno de los elementos del cifrado a este número y reducirlo módulo 65. En primer lugar, calculamos la función de Euler para 65. Dado que $\displaystyle 65 = 13 \cdot 5 $:
	\[\varphi\left(65\right) = 65\left(1-\frac{1}{13}\right)\left(1-\frac{1}{5}\right) = 12 \cdot 4 = 48 .\]
Como $\displaystyle \mcd\left(5,48\right) = 1 $, tenemos que existe $\displaystyle 5^{-1} \mod 48 $. El número $\displaystyle x $ que buscamos cumple que 
\[5x \equiv 1 \mod 48 .\]
Para resolver la congruencia comenzamos con encontrar una identidad de Bézout para 5 y 48:
\[1 = 2 \cdot 48 + \left(-19\right) \cdot 5 \iff \left(-19\right)5-1 = \left(-2\right)48 \iff \left(-19\right)5 \equiv 1 \mod 48.\]
Entonces, $\displaystyle x\equiv5^{-1} \equiv -19 \equiv 29 \mod 48 $. Ahora, elevamos cada uno de los elementos del cifrado a 29 y lo reducimos módulo 65.
\[
\begin{split}
	12^{29} \equiv & 12^{16} \cdot 12^{8} \cdot 12^{4} \cdot 12 \equiv 14^{8} \cdot 14^{4} \cdot 14^{2} \cdot 12 \equiv 1^{4} \cdot 1^{2} \cdot 12 \equiv 12 \mod 65 \\
	30^{29} \equiv & 30^{16} \cdot 30^{8} \cdot 30^{4} \cdot 30 \equiv 55^{8} \cdot 55^{4} \cdot 55^{2} \cdot 30 \equiv 35^{4} \cdot 35^{2} \cdot 35 \cdot 30 \equiv 55^{2} \cdot 55 \cdot 35 \cdot 30 \\  \equiv & \underbrace{35^{2} \cdot 55}_{55^{2} \equiv 35 \mod65} \cdot 30 \equiv 35 \cdot 30 \equiv 10 \mod 65 \\
	27^{29} \equiv & 27^{16} \cdot 27 ^{8} \cdot 27^{4} \cdot 27 \equiv 14^{8} \cdot 14^{4} \cdot 14^{2} \cdot 27 \equiv 1^{4} \cdot 1^{2} \cdot 1 \cdot 27 \equiv 27 \mod 65.
\end{split}
\]
Así, recuperamos el mensaje dado por el enunciado, pues $\displaystyle 12 \; \; 10 \; \; 27 \; \; 10 \to $ 'CARA'.
\end{description}
\end{sol}

\end{document}

