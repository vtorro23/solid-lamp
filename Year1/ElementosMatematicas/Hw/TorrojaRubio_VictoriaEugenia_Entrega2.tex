\documentclass{article}

% packages

\usepackage{graphicx} % Required for images
\usepackage[spanish]{babel}
\usepackage{mdframed}
\usepackage{amsthm}
\usepackage{amssymb}
\usepackage{fancyhdr}

% for math environments

\theoremstyle{definition}
\newtheorem{theorem}{Teorema}
\newtheorem{definition}{Definición}
\newtheorem{ej}{Ejercicio}
\newtheorem{sol}{Solución}

% for headers and footers

\pagestyle{fancy}

\fancyhead[R]{Victoria Eugenia Torroja}
% Store the title in a custom command
\newcommand{\mytitle}{}

% Redefine \title to store the title in \mytitle
\let\oldtitle\title
\renewcommand{\title}[1]{\oldtitle{#1}\renewcommand{\mytitle}{#1}}

% Set the center header to the title
\lhead{\mytitle}

% Custom commands

\newcommand{\R}{\mathbb{R}}
\newcommand{\C}{\mathbb{C}}
\newcommand{\F}{\mathbb{F}}




\begin{document}

\title{Elementos de Matemáticas y Aplicaciones - Entrega 2}
\author{Victoria Eugenia Torroja Rubio}
\date{\today}

\maketitle

\begin{ej}
Grupo diédrico.
\begin{description}
\item[(a)] Describe los elementos del grupo diédrico $\displaystyle D_{4} $. ¿Cuál es el orden de $\displaystyle D_{4} $?
\item[(b)] Construye la tabla de $\displaystyle D_{4} $, indicando todas las operaciones.
\item[(c)] Determina todos los subgrupos de $\displaystyle D_{4} $.
\end{description}
\end{ej}

\begin{ej}
Generación de teselaciones periódicas. Grupos cristalográficos.
\begin{description}
\item[(a)] Realizar un cuadro-resumen de los grupos cristalogáficos.
\item[(b)] Buscar en la arquitectura o diseñar ejemplos de al menos 10 grupos cristalográficos diferentes que incluyan todos los tipos de retículos. Considerar para los retículos cuadrados, los mosaicos de la facultad.
\item[(c)] Justificar los ejemplos del apartado anterior, indicando las características del grupo cristalográfico.
\end{description}
\end{ej}

\end{document}
