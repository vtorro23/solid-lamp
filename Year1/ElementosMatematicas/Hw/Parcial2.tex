\documentclass{article}

% packages

\usepackage{graphicx} % Required for images
\usepackage[spanish]{babel}
\usepackage{mdframed}
\usepackage{amsthm}
\usepackage{amssymb}
\usepackage{fancyhdr}

% for math environments

\theoremstyle{definition}
\newtheorem{theorem}{Teorema}
\newtheorem{definition}{Definición}
\newtheorem{ej}{Ejercicio}
\newtheorem{sol}{Solución}

% for headers and footers

\pagestyle{fancy}

\fancyhead[R]{Victoria Eugenia Torroja}
% Store the title in a custom command
\newcommand{\mytitle}{}

% Redefine \title to store the title in \mytitle
\let\oldtitle\title
\renewcommand{\title}[1]{\oldtitle{#1}\renewcommand{\mytitle}{#1}}

% Set the center header to the title
\lhead{\mytitle}

% Custom commands

\newcommand{\R}{\mathbb{R}}
\newcommand{\C}{\mathbb{C}}
\newcommand{\F}{\mathbb{F}}




\begin{document}

\title{Elemenos - Mayo 2025}
%\author{Victoria Eugenia Torroja Rubio}
\date{12/5/2025}

\maketitle

\begin{sol}
\begin{description}
\item[(a)] Sea $\displaystyle d $ la distancia entre Croydon y Le Bourget. En primer lugar, tenemos que 
	\[\Delta \Lambda = \Lambda _{C} + \Lambda_{B} = 0,1090 + 2,4410 = 2,55 .\]
	Tenemos el siguiente triángulo esférico: Aplicamos la primera fórmula de Bessel:
	\[
	\begin{split}
		\cos d = & \cos \left(90 - \Phi_{C}\right)\cos \left(90-\Phi_{B}\right) + \sin\left(90-\Phi_{C}\right)\sin\left(90-\Phi_{B}\right)\cos\Delta\Lambda \\
		= & \sin\Phi_{C}\sin\Phi_{B} + \cos \Phi_{C}\cos\Phi_{B}\cos\Delta\Lambda \\
		= & \sin51,3720\sin48,9690 + \cos 51,3720 \cos 48,9690 \cos 2,55.
	\end{split}
	\]
\item[(b)] 
\end{description}

\end{sol}

\begin{sol}
Se trata de un sistema dinámico discreto lineal de primer orden, es decir, será de la forma $\displaystyle A\left(n+1\right) = rA\left(n\right) + b$ con $\displaystyle r,b \in \R $. En efecto, será,
\[A\left(n+1\right) = \left(1+\frac{0,03}{12}\right)A\left(n\right) - 500 .\]
Si $\displaystyle r \neq 1 $ se tiene que la fórmula particular para un $\displaystyle A\left(0\right) = a_{0} $,
\[A\left(n\right) = \left(a_{0}-\frac{b}{1-r}\right)r^{n}+\frac{b}{1-r} .\]
En este caso tenemos que $\displaystyle a_{0}= 10000 $, $\displaystyle r = 1 + \frac{0,03}{12}$ y $\displaystyle b = - 500 $. Así, la solución particular será
\[A\left(n\right) = \left(10000 + \frac{500}{-\frac{0,03}{12}}\right)\left(1+\frac{0,03}{12}\right)^{n}+\frac{500}{\frac{0,03}{12}} = 190000 \cdot  .\]

\end{sol}
\end{document}
