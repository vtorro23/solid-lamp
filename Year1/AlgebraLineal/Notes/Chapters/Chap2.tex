\chapter{Aplicaciones Lineales}

\begin{fdefinition}[Aplicación Lineal]
\normalfont Sean $\displaystyle V $ y $\displaystyle V' $ son espacios vectoriales sobre $\displaystyle \K $.Una aplicación $\displaystyle f: V \to V' $ es lineal si
\begin{description}
\item[(a)] $\displaystyle f\left(\vec{x} + \vec{y}\right) = f\left(\vec{x}\right) + f\left(\vec{y}\right), \; \forall \vec{x}, \vec{y} \in V$.
\item[(b)] $\displaystyle f\left(a\vec{x}\right) = a f\left(\vec{x}\right), \; \forall a \in \K, \forall \vec{x} \in V $.
\end{description}
\end{fdefinition}

\begin{fdefinition}[]
\normalfont Un \textbf{monomorfismo} de $\displaystyle V $ en $\displaystyle V' $ es una aplicación lineal inyectiva. Un \textbf{epimorfismo} es una aplicación lineal sobreyectiva. Un \textbf{isomorfismo} es una aplicación lineal biyectiva (es homomorfismo y epimorfismo a la vez).
\end{fdefinition}

\begin{eg}
\normalfont 
\begin{description}
\item[(a)]Tenemos que $\displaystyle L \in \mathcal{L}\left(V\right) $, 
\[
\begin{split}
	i & : L \to V \\
	  & \vec{x} \to \vec{x}.
\end{split}
\]
Esta aplicación es un monomorfismo.
\item[(b)] Si $\displaystyle L \in \mathcal{L}\left(V\right),  $ la aplicación 
	\[
	\begin{split}
		p & : V \to V/L \\
		  & \vec{x} \to \vec{x} + L
	\end{split}
	\]
Es un epimorfismo.
\item[(c)] Sean $\displaystyle \left\{ \vec{u}_{1}, \ldots, \vec{u}_{n}\right\}  $ una base de $\displaystyle V $, la aplicación
	\[
	\begin{split}
		f & : V \to \K^{n} \\
		  & \vec{x} \to \left(a^{1}, \ldots, a^{n}\right).
	\end{split}
	\]
Donde, $\displaystyle \vec{x} = a^{1}\vec{u}_{1} + \cdots + a^{n}\vec{u}_{n} $. Entonces $\displaystyle f $ es un isomorfismo. 
\end{description}
\end{eg}

\begin{fprop}[]
\normalfont Sea $\displaystyle f : V \to V' $ una aplicación lineal. 
\begin{description}
\item[(a)] Si $\displaystyle L \in \mathcal{L}\left(V\right) $
	\[f\left(L\right) = \left\{ f\left(\vec{x}\right)\in V' \; : \; \vec{x} \in L\right\} \in \mathcal{L}\left(V'\right) .\]
\item[(b)] $\displaystyle f\left(\vec{0}\right) = \vec{0} $ y $\displaystyle f\left(-\vec{x}\right)=-f\left(\vec{x}\right) $.
\item[(c)] Si $\displaystyle L' \in \mathcal{L}\left(V'\right) $,
	\[f^{-1}\left(L'\right) = \left\{ \vec{x} \in V \; : \; f\left(\vec{x}\right) \in L'\right\} \in \mathcal{L}\left(V\right) .\]
\item[(d)] $\displaystyle 0 : V \to V' $ tal que $\displaystyle \vec{x} \to \vec{0} $ es una aplicación lineal.
\end{description}
\end{fprop}

\begin{proof}
\begin{description}
\item[(a)] Queremos ver que $\displaystyle \forall \vec{x}', \vec{y}' \in f\left(L\right) \Rightarrow \exists\vec{x}, \vec{y} \in L, \; f\left(\vec{x}\right) = \vec{x'}$ y $\displaystyle f\left(\vec{y}\right) = \vec{y'} $. Tenemos que ver que la suma y el producto por escalares está bien definidas.  
	\[\vec{x'} + \vec{y'} = f\left(\vec{x}\right) + f\left(\vec{y}\right) = f\left(\vec{x} + \vec{y}\right) \in f\left(L\right) .\]
Similarmente, para el producto por escalares, si $\displaystyle a \in \K $,
\[a \vec{x'} = af\left(\vec{x}\right) = f\left(a\vec{x}\right) \in f\left(L\right).\]
\item[(b)] Tenemos que $\displaystyle f\left(\vec{0}\right) = f\left(0 \cdot \vec{x}\right) = 0 f\left(\vec{x}\right) = \vec{0} $. Similarmente, 
	\[f\left(-\vec{x}\right) =  f\left(\left(-1\right)\vec{x}\right) = - f\left(\vec{x}\right) .\]
Otra demostración es:
\[f\left(\vec{x}\right) + f\left(-\vec{x}\right) = f\left(\vec{x} + \left(-\vec{x}\right)\right) = f\left(\vec{0}\right) = \vec{0} .\]
\item[(c)] Sean $\displaystyle \vec{x}, \vec{y} \in f^{-1}\left(L'\right) $, entonces $\displaystyle f\left(\vec{x}\right), f\left(\vec{y}\right) \in L' $. Como $\displaystyle L' $ es subespacio vectorial, 
	\[f\left(\vec{x}\right) + f\left(\vec{y}\right) \in L' .\]
Por tanto, 
\[f\left(\vec{x} + \vec{y}\right) \in L' \Rightarrow \vec{x} + \vec{y} \in f^{-1}\left(L'\right) .\]
Para el producto por escalares, si $\displaystyle a\in\K $, $\displaystyle \vec{x} \in f^{-1}\left(L'\right) $, 
\[f\left(\vec{x}\right)\in L' \Rightarrow a f\left(\vec{x}\right) = f\left(a\vec{x}\right) \in L' \Rightarrow a\vec{x} \in f^{-1}\left(L\right) .\]
\end{description}
\end{proof}

\begin{fcolorary}[]
\normalfont 
\begin{description}
	\item[(a)] Imagen. $\displaystyle \Imagen\left(f\right) = f\left(V\right) = \left\{ f\left(\vec{x}\right)\; :\; \vec{x} \in V\right\}\in \mathcal{L}\left(V'\right)  $.
		\item[(b)] Núcleo. $\displaystyle \Ker\left(f\right) = f^{-1}\left(\{\vec{0}\}\right) = \left\{ \vec{x}\in V\; : \; f\left(\vec{x}\right) = \vec{0}\right\} \in \mathcal{L}\left(V\right) $.
\end{description}
\end{fcolorary}
\begin{proof}
	Como $\displaystyle V $ y $\displaystyle \left\{ \vec{0}\right\}  $ son subespacios vectoriales su imagen y preimagen, respectivamente, también serán subespacios vectoriales por la proposición 2.1. 
\end{proof}


\begin{fprop}[]
\normalfont Sea $\displaystyle f: V \to V' $ una aplicación lineal, entonces:
\begin{description}
\item[(a)] $\displaystyle f $ es epimorfismo $\displaystyle \iff  $ $\displaystyle \Imagen\left(f\right) = V' $.
\item[(b)] $\displaystyle f $ es monomorfismo $\displaystyle \iff  $ $\displaystyle \Ker\left(f\right) = \left\{ \vec{0}\right\}  $.
\end{description}
\end{fprop}

\begin{proof}
\begin{description}
\item[(a)] Es la definición de sobreyectividad.
	\item[(b)] Primera implicación. Tenemos que $\displaystyle f\left(\vec{0}\right) = \vec{0} $. Si $\displaystyle f\left(\vec{x}\right)=\vec{0} $, tenemos que $\displaystyle \vec{x} = \vec{0} $ (porque $\displaystyle f $ es inyectiva). Segunda implicación. Supongo que $\displaystyle \Ker\left(f\right) = \left\{ \vec{0}\right\}  $, entonces
		\[f\left(\vec{x}\right) = f\left(\vec{y}\right) \Rightarrow f\left(\vec{x}\right)-f\left(\vec{y}\right) = f\left(\vec{x}-\vec{y}\right)= \vec{0} \Rightarrow \vec{x}-\vec{y} \in \Ker\left(f\right) \Rightarrow \vec{x}=\vec{y} .\]
\end{description}
\end{proof}

\begin{fprop}[]
	\normalfont Sea $\displaystyle f: V \to V' $ una aplicación lineal y sea $\displaystyle \left\{ \vec{x}_{1}, \ldots, \vec{x}_{p}\right\}  $ un sistema de generadores de $\displaystyle L \in \mathcal{L}\left(V\right) $. Entonces $\displaystyle \left\{ f\left(\vec{x}_{1}\right), \ldots, f\left(\vec{x}_{p}\right)\right\}  $ es un sistema de generadores de $\displaystyle f\left(L\right) $.
\footnote{La independencia lineal no se conserva en una aplicación lineal en general.} 
\end{fprop}

\begin{proof}
	Sea $\displaystyle \vec{x'} \in f\left(L\right) $, existe $\displaystyle \vec{x} \in L $ tal que $\displaystyle f\left(\vec{x}\right) = \vec{x'} $. Como $\displaystyle \left\{ \vec{x}_{1}, \ldots, \vec{x}_{p}\right\}  $ es un sistema de generadores de $\displaystyle L $, existen $\displaystyle a^{i}\in\K $ escalares tales que 
	\[\vec{x} = a^{1}\vec{x}_{1} + \cdots + a^{p}\vec{x}_{p}.\]
Entonces, 
\[\vec{x'} = f\left(a^{1}\vec{x}_{1} + \cdots + a^{p}\vec{x}_{p}\right) = a^{1}f\left(\vec{x}_{1}\right) + \cdots + a^{p}f\left(\vec{x}_{p}\right) .\]
\end{proof}

\begin{ftheorem}[]
	\normalfont Sea $\displaystyle f : V \to V' $ una aplicación lineal. Entonces, $\displaystyle f $ es un monomorfismo si y solo si $\displaystyle \forall p \in \N, \forall \left\{ \vec{u}_{1}, \ldots, \vec{u}_{p}\right\} \subset V $ linealmente independientes, implica que $\displaystyle \left\{ f\left(\vec{u}_{1}\right), \ldots, f\left(\vec{u}_{p}\right)\right\}  $ son linealmente independientes.
\end{ftheorem}

\begin{proof}
\begin{description}
	\item[(i)] Si $\displaystyle f $ es un monomorfismo y sea $\displaystyle p \in \N $, con $\displaystyle \left\{ \vec{u}_{1}, \ldots, \vec{u}_{p}\right\} \subset V $ linealmente independientes. Cogemos $\displaystyle a^{i}\in\K $ tales que 
		\[a^{1}f\left(\vec{u}_{1}\right) + \cdots + a^{p}f\left(\vec{u}_{p}\right) = \vec{0} .\]
Como $\displaystyle f $ es una aplicación lineal, 
\[
\begin{split}
	& a^{1}f\left(\vec{u}_{1}\right) + \cdots + a^{p}f\left(\vec{u}_{p}\right) \\
	=& f\left(a^{1}\vec{u}_{1}+ \cdots + a^{p}\vec{u}_{p}\right) \\
	=&\vec{0} .
\end{split}
\]
Como $\displaystyle f $ es monomorfismo, tenemos que 
\[a^{1}\vec{u}_{1}+ \cdots + a^{p}\vec{u}_{p} = \vec{0} .\]
Como estos vectores forman una base, tenemos que $\displaystyle a^{1} = \cdots = a^{p} = 0 $.
\item[(ii)] Lo hacemos por contraposición. Suponemos que $\displaystyle f $ no es inyectiva (no es monomorfismo), por lo que existe $\displaystyle \vec{x} \neq \vec{0} $ tal que $\displaystyle \vec{x} \in \Ker\left(f\right) $. Entonces, $\displaystyle \left\{ \vec{x}\right\}  $ es linealmente independiente y $\displaystyle \left\{ f\left(\vec{x}\right)\right\}  = \left\{ \vec{0}\right\}  $ es linealmente dependiente.
\end{description}
\end{proof}

\begin{ftheorem}[]
	\normalfont Sea $\displaystyle f: V \to V' $ una aplicación lineal. Entonces $\displaystyle f $ es epimorfismo si y solo si para cada sistema de generadores $\displaystyle \left\{ \vec{u}_{1}, \ldots, \vec{u}_{n}\right\}  $ de $\displaystyle V $, $\displaystyle \left\{ f\left(\vec{u}_{1}\right), \ldots, f\left(\vec{u}_{n}\right)\right\}  $ es sistema de generadores de $\displaystyle V' $.
\end{ftheorem}

\begin{proof}
	Sabemos que si $\displaystyle \left\{ \vec{u}_{1}, \ldots , \vec{u}_{n}\right\}  $ es sistema de generadores en $\displaystyle V $, entonces $\displaystyle \left\{f\left(\vec{u}_{1}\right), \ldots, f\left(\vec{u}_{n}\right) \right\}  $ será sistema de generadores de $\displaystyle \Imagen (f)  $. Si $\displaystyle f $ es epimorfismo, entonces será base de $\displaystyle V' $ y, si es base de $\displaystyle V' $ es porque $\displaystyle \Imagen\left(f\right)= V' $, por lo que es epimorfismo.
\end{proof}

\begin{fprop}[]
\normalfont Sean $\displaystyle f: V \to V' $ y $\displaystyle g: V' \to V '' $ aplicaciones lineales. Entonces, la composición $\displaystyle g \circ f $ también es lineal. 
\end{fprop}

\begin{proof}
Sean $\displaystyle \vec{x}, \vec{y} \in V $, tenemos que  
\[ g \left( f \left(\vec{x}+\vec{y}\right)\right) = g \left(f\left(\vec{x}+\vec{y}\right)\right) = g\left(f\left(\vec{x}\right) + f\left(\vec{y}\right)\right) = g\left(f\left(\vec{x}\right)\right) + g\left(f\left(\vec{y}\right)\right) .\]
Entonces, $\displaystyle g \circ f $ es lineal respecto a la suma. Similarmente, si $\displaystyle a \in \K, \vec{x} \in V $ tenemos que
\[ g\left(f\left(a\vec{x}\right)\right) = g\left(af\left(\vec{x}\right)\right) = ag\left(f\left(\vec{x}\right)\right) .\]
Por tanto, $\displaystyle g\circ f $ es una aplicación lineal. 
\end{proof}

\begin{fprop}[]
\normalfont Sea $\displaystyle f: V \to V' $ un isomorfismo (lineal y biyectiva) \footnote{Una función es biyectiva si y sólo si tiene inversa.} . Sabemos que existe $\displaystyle f^{-1}:V'\to V $. Entonces, $\displaystyle f^{-1} $ es isomorfismo.
\end{fprop}

\begin{proof}
Solo tenemos que demostrar que es aplicación lineal, porque inversa de una biyección también es biyección. Si $\displaystyle \forall \vec{x'}, \vec{y'} \in V' $, como $\displaystyle f $ es biyectiva, $\displaystyle \exists! \vec{x}, \vec{y} \in V $ tales que $\displaystyle \vec{x'} = f\left(\vec{x}\right) \iff \vec{x} = f^{-1}\left(\vec{x'}\right) $ y $\displaystyle \vec{y'} = f\left(\vec{y}\right) \iff \vec{y} = f^{-1}\left(\vec{y'}\right) $. Entonces, 
\[f^{-1}\left(\vec{x'} + \vec{y'}\right) = f^{-1}\left(f\left(\vec{x}\right)+f\left(\vec{y}\right)\right) = f^{-1}\left(f\left(\vec{x}+\vec{y}\right)\right) = \vec{x}+\vec{y} = f^{-1}\left(\vec{x'}\right) + f^{-1}\left(\vec{y'}\right) .\]
Similarmente, si $\displaystyle a \in \K $ y $\displaystyle \vec{x'}\in V' $, $\displaystyle \exists! \vec{x} \in V $ tal que $\displaystyle f\left(\vec{x}\right) = \vec{x'} \iff f^{-1}\left(\vec{x'}\right) = \vec{x}$. Entonces tenemos que, 
\[f^{-1}\left(a\vec{x'}\right) = f^{-1}\left(af\left(\vec{x}\right)\right) = f^{-1}\left(f\left(a\vec{x}\right)\right) = a\vec{x} = a f^{-1}\left(\vec{x'}\right) .\]
\end{proof}

\begin{ftheorem}[]
	\normalfont Sea $\displaystyle \left\{ \vec{u}_{1}, \ldots, \vec{u}_{n}\right\}  $ una base de $\displaystyle V $ y sean $\displaystyle \left\{ \vec{v}_{1}, \ldots, \vec{v}_{n}\right\} \subset V' $. Entonces, $\displaystyle \exists! f : V \to V' $ lineal tal que $\displaystyle f\left(\vec{u}_{i}\right) = \vec{v}_{i} $.
\end{ftheorem}

\begin{proof}
\begin{description}
\item[(i)] Primero demostramos la unicidad, es decir, asumimos que existe y demostramos que debe ser única. Entonces, asumimos que existe $\displaystyle f : V \to V' $ linal tal que $\displaystyle f\left(\vec{u}_{i}\right) = \vec{v}_{i}, \; \forall i = 1, \ldots, n $. Sea $\displaystyle \vec{x}\in V $, entonces existen $\displaystyle a^{i} \in \K $ únicos tales que 
	\[\vec{x} = a^{1}\vec{u}_{1} + \cdots + a^{n}\vec{u}_{n} .\]
Entonces, 
\[f\left(\vec{x}\right) = f\left(a^{1}\vec{u}_{1} + \cdots + a^{n}\vec{u}_{n}\right) = a^{1}f\left(\vec{u}_{1}\right) + \cdots + a^{n}f\left(\vec{u}_{n}\right) = a^{1}\vec{v}_{1} + \cdots + a^{n}\vec{v}_{n}.\]
Si exsistese otra función $\displaystyle g : V \to V' $ tal que $\displaystyle g\left(\vec{u}_{i}\right) = \vec{v}_{i}, \; \forall i = 1, \ldots, n $, tendríamos que $\displaystyle f\left(\vec{x}\right) = g\left(\vec{x}\right), \; \forall \vec{x} \in V $.
\item[(ii)] Ahora demostramos la existencia. Sea $\displaystyle f : V \to V' $ la aplicación $\displaystyle f\left(\vec{x}\right) = x^{1}\vec{v}_{1} + \cdots + x^{n}\vec{v}_{n} $, donde $\displaystyle \vec{x} = x^{1}\vec{u}_{n} + \cdots + x^{n}\vec{u}_{n} $. Tenemos que demostrar que esta aplicación es lineal. Si $\displaystyle \vec{x}, \vec{y} \in V $ queremos ver que $\displaystyle f\left(\vec{x}+\vec{y}\right) = f\left(\vec{x}\right) + f\left(\vec{y}\right) $. Tenemos que
\[
\begin{split}
	f\left(\vec{x}+\vec{y}\right) = & f\left(\left(x^{1}\vec{u}_{n} + \cdots + x^{n}\vec{u}_{n}\right) + \left(y^{1}\vec{u}_{n} + \cdots + y^{n}\vec{u}_{n}\right)\right) \\
	= & f\left(\left(x^{1}+y^{1}\right)\vec{u}_{1} + \cdots + \left(x^{n}+y^{n}\right)\vec{u}_{n}\right) \\
	= & \left(x^{1}+y^{1}\right)\vec{v}_{1} + \cdots + \left(x^{n} + y^{n}\right) \vec{v}_{n} \\
	= & \left(x^{1}\vec{v}_{1} + \cdots + x^{n}\vec{v}_{n}\right) + \left(y^{1}\vec{v}_{1} + \cdots + y^{n}\vec{v}_{n}\right) \\
	= & f\left(\vec{x}\right) + f\left(\vec{y}\right).
\end{split}
\]
Similarmente, 
\[f\left(a\vec{x}\right) = f\left(a\left(x^{1}\vec{u}_{n} + \cdots + x^{n}\vec{u}_{n}\right)\right)= f\left(ax^{1}\vec{u}_{1} + \cdots + ax^{n}\vec{u}_{n}\right) = a f\left(\vec{x}\right) .\]
\end{description}
\end{proof}

Podemos ver que
\[f\left(\vec{u}_{i}\right) = f\left(0 \cdot \vec{u}_{1} + \cdots + 0 \cdot \vec{u}_{i-1} + 1 \cdot \vec{u}_{i} + 0 \cdot \vec{u}_{i+1} + \cdots + 0 \cdot \vec{u}_{n}\right) = 0 \cdot \vec{v}_{1} + \cdots + 1 \cdot \vec{v}_{i} + \cdots + 0 \cdot \vec{v}_{n} = \vec{v}_{i} .\]

\begin{fcolorary}[]
	\normalfont Sea $\displaystyle f: V \to V' $ lineal y $\displaystyle f\left(\vec{u}_{i}\right) = g\left(\vec{u_{i}}\right), \forall i = 1, \ldots, n $ donde $\displaystyle \left\{ \vec{u}_{1}, \ldots, \vec{u}_{n}\right\}  $ es base, entonces $\displaystyle f = g $.
\end{fcolorary}

\begin{fdefinition}[]
\normalfont Dos espacios vectoriales $\displaystyle V $ y $\displaystyle V' $ son isomorfos si existe $\displaystyle f: V \to V' $ isomorfismo. Lo expresaremos como $\displaystyle V \approx V' $.
\end{fdefinition}

\begin{ftheorem}[]
\normalfont Sea $\displaystyle f: V \to V' $ una aplicación lineal. Entonces son equivalentes:
\begin{description}
\item[(a)] $\displaystyle f $ es isomorfismo.
\item[(b)] $\displaystyle \forall \vec{u}_{1}, \ldots, \vec{u}_{n} $ base de $\displaystyle V $, $\displaystyle \left\{ f\left(\vec{u}_{1}\right), \ldots, f\left(\vec{u}_{n}\right)\right\}  $ es base de $\displaystyle V' $.
\item[(c)] $\displaystyle \exists \left\{ \vec{u}_{1}, \ldots, \vec{u}_{n}\right\}  $ base de $\displaystyle V $ tal que $\displaystyle \left\{ f\left(\vec{u}_{1}\right), \ldots, f\left(\vec{u}_{n}\right)\right\}  $ es base de $\displaystyle V' $ .
\end{description}
\end{ftheorem}

\begin{proof} Vamos a ver que (a) $\displaystyle \Rightarrow $ (b), que (b) $\displaystyle \Rightarrow $ (c) y que (c) $\displaystyle \Rightarrow $ (a). 
\begin{description}
	\item[(a) $\displaystyle \Rightarrow $ (b) ] Suponemos que $\displaystyle f $ es un isomorfismo y sea $\displaystyle \left\{ \vec{u}_{1}, \ldots, \vec{u}_{n}\right\}  $ base de $\displaystyle V $. Entonces este conjunto es sistema de generadores y son linealmente independientes. Entonces $\displaystyle \left\{ f\left(\vec{u}_{1}\right), \ldots , f\left(\vec{u}_{n}\right)\right\}  $ es sistema de generadores de $\displaystyle V' $. Además, como son linealmente independientes, $\displaystyle \left\{ f\left(\vec{u}_{1}\right), \ldots, f\left(\vec{u}_{n}\right)\right\}  $ también son linealmente independientes. 
	\item[(b) $\displaystyle \Rightarrow $ (c)] Evidente.
	\item[(c) $\displaystyle \Rightarrow $ (a)] Sea $\displaystyle f : V \to V' $ lineal, entonces $\displaystyle \exists \left\{ \vec{u}_{1}, \ldots, \vec{u}_{n}\right\}  $ base de $\displaystyle V $, entonces tenemos que $\displaystyle \left\{ f\left(\vec{u}_{1}\right), \ldots, f\left(\vec{u}_{n}\right)\right\}  $ es base de $\displaystyle V' $. Sea $\displaystyle g : V' \to V $, tal que $\displaystyle g = f^{-1} $, la única aplicación lineal tal que $\displaystyle g\left(f\left(\vec{u}_{i}\right)\right) = \vec{u}_{i}, \forall i = 1, \ldots, n$. Entonces, tenemos que $\displaystyle f $ tiene inversa, por lo que es biyectiva y, además, como es aplicación lineal, es isomorfismo.
\end{description}
\end{proof}

\begin{fcolorary}[]
\normalfont $\displaystyle V \approx V' \iff \dim V = \dim V' $ 
\end{fcolorary}

\begin{proof}
\begin{description}
\item[(i)] Si $\displaystyle V \approx V' $, existe un isomorfismo entre ellos, y podemos encontrar una base con el mismo número de elementos en $\displaystyle V' $ .
\item[(ii)]  Supongamos que $\displaystyle \dim V = \dim V' $. Sea $\displaystyle \left\{ \vec{u}_{1}, \ldots, \vec{u}_{n}\right\}  $ base de $\displaystyle V $ y $\displaystyle \left\{ \vec{v}_{1}, \ldots, \vec{v}_{n}\right\}  $ base de $\displaystyle V' $ y $\displaystyle f: V \to V' $ la única aplicación lineal tal que $\displaystyle f\left(\vec{u}_{i}\right) = \vec{v}_{i}, \forall i = 1, \ldots, n $. Como es una aplicación lineal que lleva bases en bases es un isomorfismo. 
\end{description}
\end{proof}

\begin{observation}
\normalfont Si consideramos la relación de equivalencia de que dos espacios vectoriales sean isomorfos, tenemos que el conjunto cociente tiene tantos elementos como biyecciones.
\end{observation}

\begin{ftheorem}[]
\normalfont 
\[\dim V = \dim \Ker\left(f\right) + \dim \Imagen\left(f\right) .\]
\end{ftheorem}
\begin{proof}
	Sea $\displaystyle \left\{ \vec{u}_{1}, \ldots, \vec{u}_{r}\right\}  $ base de $\displaystyle \Ker\left(f\right) $ y sea $\displaystyle \left\{ \vec{u}_{1}, \ldots, \vec{u}_{r}, \vec{u}_{r+1}, \ldots, \vec{u}_{n}\right\}  $ base de $\displaystyle V $. Entonces $\displaystyle \dim V = n $ y $\displaystyle \dim \Ker\left(f\right) = r $. Vamos a ver que $\displaystyle \left\{ f\left(\vec{u}_{r+1}\right), \ldots, f\left(\vec{u}_{n}\right)\right\}  $ es base de $\displaystyle \Imagen\left(f\right) $. Primero tenemos que ver que son sistema de generadores. Sea $\displaystyle \vec{x'} \in \Imagen\left(f\right) $, entonces existe $\displaystyle \vec{x} \in V $ tal que $\displaystyle f\left(\vec{x}\right) = \vec{x'} $. Como $\displaystyle \vec{x}\in V $, existen $\displaystyle a^{i}\in\K $ tales que
	\[ \vec{x} = a^{1}\vec{u}_{1} + \cdots + a^{n}\vec{u}_{n}.\]
Entonces tenemos que
\[\vec{x'} = f\left(\vec{x}\right) = f\left(a^{1}\vec{u}_{1} + \cdots + a^{n}\vec{u}_{n}\right) = fa^{1}\left(\vec{u}_{1}\right) + \cdots + a^{r}f\left(\vec{u}_{r}\right) + a^{r+1}f(\vec{u}_{r+1}) + \cdots + a^{n} f\left(\vec{u}_{n}\right) .\]
Tenemos que como $\displaystyle \left\{ \vec{u}_{1}, \ldots, \vec{u}_{r}\right\}  $ es una base de $\displaystyle \Ker\left(f\right) $, $\displaystyle f\left(\vec{u}_{i}\right) = \vec{0}, \forall i = 1, \ldots, r $:
\[\vec{x'} = a^{r+1}f\left(\vec{u}_{r+1}\right) + \cdots + a^{n}f\left(\vec{u}_{n}\right) .\]
Entonces, $\displaystyle \left\{ f(\vec{u}_{r+1}), \ldots, f(\vec{u}_{n})\right\}  $ es sistema de generadores. Ahora vamos a ver que son linealmente independientes. Sean $\displaystyle b^{i} \in \K $ tal es que 
\[
\begin{split}
  & b^{r+1}f\left(\vec{u}_{r+1}\right) + \cdots + b^{n}f\left(\vec{u}_{n}\right) = \vec{0} \\
 \therefore & f\left(b^{r+1}\vec{u}_{r+1} + \cdots + b^{n}\vec{u}_{n}\right) = \vec{0}
\end{split}
\]
Por tanto, $\displaystyle b^{r+1}\vec{u}_{r+1} + \cdots + b^{n}\vec{u}_{n} \in \Ker\left(f\right) $ y lo podemos poner como combinación lineal de su base (existen $\displaystyle b^{i}\in\K $ tales que):
\[b^{r+1}\vec{u}_{r+1} + \cdots + b^{n}\vec{u}_{n} = b^{1}\vec{u}_{1} + \cdots + b^{r}\vec{u}_{r} .\]
\[\therefore -b^{1}\vec{u}_{1} - \cdots - b^{r}\vec{u}_{r} + b^{r+1}\vec{u}_{r+1} + \cdots + b^{n}\vec{u}_{n} = \vec{0} .\]
Como se trata de una base de $\displaystyle V $, tenemos que son linealmente independientes y $\displaystyle b^{1} = \cdots = b^{r+1} = \cdots = b^{n} = 0 $. Por lo que todos los coeficientes son nulos, $\displaystyle \left\{ f\left(\vec{u}_{r+1}\right), \ldots, f\left(\vec{u}_{n}\right)\right\}  $ son linealmente independientes y forman una base de $\displaystyle \Imagen\left(f\right) $.
\end{proof}

\begin{observation}
\normalfont Si $\displaystyle \dim V = \dim V' $ entonces, $\displaystyle f $ es monomorfismo por lo que $\displaystyle \dim \Ker\left(f\right) = \vec{0} $. Entonces, $\displaystyle \dim V = \dim V'= \dim \Imagen\left(f\right) \iff \Imagen\left(f\right) = V' $. Es decir, monomorfismo si y solo si epimorfismo si y solo si isomorfismo, es decir, para demostrar que es un isomorfismo solo hay que demostrar que es un monomorfismo!
\end{observation}

\begin{fdefinition}[Endomorfismo]
\normalfont Un \textbf{endomorfismo} de $\displaystyle V $ es una aplicación lineal $\displaystyle f: V \to V $. 
\end{fdefinition}

\begin{fdefinition}[Automorfismo]
\normalfont Un \textbf{automorfismo}  de $\displaystyle V $ es un endomorfismo biyectivo de $\displaystyle V $.
\end{fdefinition}

\begin{ftheorem}[]
\normalfont Sea $\displaystyle f: V \to V' $ lineal, entonces $\displaystyle p : V \to V/\Ker\left(f\right) $ tal que $\displaystyle \vec{x} \to \vec{x} + \Ker\left(f\right) $ es un epimorfismo. Sea $\displaystyle i : \Imagen\left(f\right) \to V' $ tal que $\displaystyle \vec{x'} \to \vec{x'} $  es monomorfismo. Entonces, $\displaystyle \exists! b : V/\Ker\left(f\right) \to \Imagen\left(f\right) $ tal que 
\[i \circ b \circ p = f: V \to V' .\]
Además, $\displaystyle b $ es isomorfismo.
\end{ftheorem}

\begin{proof}
Definimos $\displaystyle b\left(\vec{x}+\Ker\left(f\right)\right) = f\left(\vec{x}\right) $. \\ \\
\textbf{Unicidad.} Suponemos que existe $\displaystyle b $, $\displaystyle \forall\vec{x} \in V $ tal que $\displaystyle b\left(\vec{x} + \Ker\left(f\right)\right) = f\left(\vec{x}\right) $. \\ \\
\textbf{Vemos que $\displaystyle b $ está bien definida.} Si $\displaystyle \vec{y} \in V $, con $\displaystyle \vec{x} + \Ker\left(f\right) = \vec{y} + \Ker\left(f\right) $, tenemos que
	\[\vec{x}-\vec{y} \in \Ker\left(f\right) \Rightarrow f\left(\vec{x}-\vec{y}\right) = \vec{0} .\]
Entonces, 
\[ f\left(\vec{x}-\vec{y}\right) = f\left(\vec{x}\right)-f\left(\vec{y}\right) = \vec{0} \iff f(\vec{x}) = f(\vec{y}) .\]
Entonces, 
\[b\left(\vec{x}+\Ker\left(f\right)\right) = b\left(\vec{y}+\Ker\left(f\right)\right) .\]
 \textbf{Comprobamos que $\displaystyle b $ es lineal.} Comenzamos con la suma. Si $\displaystyle \vec{x}, \vec{y} \in V $, 
\[
\begin{split}
	b \left(\left(\vec{x}+\Ker\left(f\right)\right) + \left(\vec{y}+\Ker\left(f\right)\right)\right)   = & b \left(\left(\vec{x}+\vec{y}\right)+\Ker\left(f\right)\right) \\
	= & f\left(\vec{x}+\vec{y}\right) = f\left(\vec{x}\right) + f\left(\vec{y}\right) \\
	= & b\left(\vec{x} + \Ker\left(f\right)\right) + b\left(\vec{y} + \Ker\left(f\right)\right) .
\end{split}
\]
Ahora comprobamos el producto por escalares. Sea $\displaystyle a \in \K $ y $\displaystyle \vec{x} \in V $, 
\[
\begin{split}
	b\left(a\left(\vec{x}+\Ker\left(f\right)\right)\right) = & b\left(a\vec{x}+\Ker\left(f\right)\right) \\
	= & f\left(a\vec{x}\right) = af\left(\vec{x}\right) = a\left(b\left(\vec{x}+\Ker\left(f\right)\right)\right) .
\end{split}
\]
 \textbf{Comprobamos que $\displaystyle b $ es un epimorfismo.} 
	\[\forall \vec{x'} \in \Imagen\left(f\right) \Rightarrow \exists \vec{x} \in V, \; f\left(\vec{x}\right) = \vec{x'} .\]
Por tanto, $\displaystyle f\left(\vec{x}\right) = b\left(\vec{x}+ \Ker\left(f\right)\right) = \vec{x'} $. Entonces, $\displaystyle \vec{x'} \in \Imagen\left(b\right) $. \\ \\
 \textbf{Comprobamos que $\displaystyle b $ es un monomorfismo.} Sea $\displaystyle \vec{x} + \Ker\left(f\right) \in \Ker\left(b\right) $, entonces $\displaystyle f\left(\vec{x}\right) = \vec{0} $ por lo que $\displaystyle \vec{x} \in \Ker\left(f\right) $. Concluimos que
	\[\vec{x}+\Ker\left(f\right) = \vec{0} + \Ker\left(f\right) .\]
\end{proof}
\begin{observation}
\normalfont Lo que nos interesa concluir con este teorema es que si $\displaystyle f: V \to V' $ es lineal, entonces
\[\Imagen\left(f\right) \approx V/\Ker\left(f\right) .\]
\end{observation}

\begin{fprop}[]
\normalfont Sea $\displaystyle L $ el complementario vectorial de $\displaystyle \Ker\left(f\right) $ en $\displaystyle V $, es decir, 
\[L \oplus \Ker\left(f\right) = V \Rightarrow  \dim L + \dim \Ker\left(f\right) = \dim V .\]
La aplicación 
\[f_{L}: L \to \Imagen\left(f\right) \]
\[\vec{x} \to f\left(\vec{x}\right) ,\]
es un isomorfismo.
\end{fprop}

\begin{proof}
Tenemos que $\displaystyle f_{L} $ es lineal. Además, 
\[\Ker\left(f_{L}\right) = \left\{ \vec{x} \in L\; : \; f\left(\vec{x}\right)=\vec{0}\right\} = L \cap \Ker\left(f\right) = \left\{ \vec{0}\right\}  .\]
Entonces tenemos que es monomorfismo. Por otro lado queremos ver que, 
\[\Imagen\left(f_{L}\right) = \Imagen\left(f\right) .\]
Si $\displaystyle \vec{x'} \in \Imagen\left(f\right) $, existe $\displaystyle \vec{x} \in V $ tal que $\displaystyle \vec{x'} = f\left(\vec{x}\right) $. Como $\displaystyle \vec{x} \in V $, $\displaystyle \exists \vec{y} \in \Ker\left(f\right) $ y $\displaystyle \vec{z} \in L $ tales que
\[\vec{x'} \in \Imagen\left(f_{L}\right) .\]
Es decir, 
\[\vec{x'} = f\left(\vec{x}\right) = f\left(\vec{y}+\vec{z}\right) = f\left(\vec{y}\right)+f\left(\vec{z}\right) = f\left(\vec{z}\right) = f_{L}\left(\vec{z}\right) .\]
Es trivial ver que $\displaystyle \Imagen\left(f_{L}\right) \subset \Imagen\left(f\right)$. Por tanto es epimorfismo y, consecuentemente, isomorfismo. 
\end{proof}


\begin{fcolorary}[]
\normalfont Si $\displaystyle L_{1}, L_{2} \in \mathcal{L}\left(V\right) $, 
\[\left(L_{1}+L_{2}\right) / L_{1} \approx L_{2} / L_{1} \cap L_{2} .\]
\end{fcolorary}

\begin{proof}
Si $\displaystyle L_{1}, L_{2} \in \mathcal{L}\left(V\right) $,
\[\dim \left(L_{1}+L_{2}\right) - \dim L_{1} = \dim L_{2}-\dim\left(L_{1}\cap L_{2}\right) .\]
Entonces tenemos que, 
\[ \dim\left(L_{1}+L_{2}\right)/L_{1} = \dim L_{2} / L_{1}\cap L_{2} .\]


\end{proof}

\begin{fcolorary}[]
\normalfont Si $\displaystyle L_{1} \subset L_{2} \in \mathcal{L}\left(V\right) $, entonces
\[\left(V/L_{1}\right)/\left(L_{2}/L_{1}\right) \approx V/L_{2} .\]
\end{fcolorary}

\begin{ftheorem}[]
\normalfont $\displaystyle \Hom\left(V,V'\right) $ tiene una estructura de $\displaystyle \K $-espacio vectorial.
\end{ftheorem}
\begin{proof}

Sea $\displaystyle \Hom\left(V, V'\right) = \left\{ f: V \to V' \; : \; f \; \text{lineal}\right\}  $. En $\displaystyle \Hom\left(V, V'\right) $ definimos la suma de la siguiente manera. Si $\displaystyle f, g \in \Hom\left(V, V'\right) $ definimos 
\[
\begin{split}
& f + g : V \to V' \\
& \left(f+ g\right) \left(\vec{x}\right) = f\left(\vec{x}\right) + g\left(\vec{x}\right).
\end{split}
\]
Vamos a ver que $\displaystyle f + g $ es aplicación lineal. Si $\displaystyle \vec{x}, \vec{y} \in V $, 
\[\left(f+g\right)\left(\vec{x}+\vec{y}\right) = f\left(\vec{x} + \vec{y}\right) + g\left(\vec{x}+\vec{y}\right) = f\left(\vec{x}\right) + f\left(\vec{y}\right) + g\left(\vec{x}\right) + g\left(\vec{y}\right) = \left(f+g\right)\left(\vec{x}\right) + \left(f+g\right)\left(\vec{y}\right).\]
Si $\displaystyle a \in \K  $ y $\displaystyle \vec{x} \in V $, 
\[\left(f+g\right)\left(a\vec{x}\right) = f\left(a\vec{x}\right) + g\left(a\vec{x}\right) = af\left(\vec{x}\right) + a g\left(\vec{x}\right) = a \left(f+g\right)\left(\vec{x}\right) .\]
Entonces, $\displaystyle f+g $ es aplicación lineal. Así, vamos a comprobar que $\displaystyle \left( \Hom\left(V, V'\right), +\right) $ es grupo abeliano. Comprobamos la conmutatividad. Si $\displaystyle \vec{x} \in V $, 
\[\left(f+g\right)\left(\vec{x}\right) = f\left(\vec{x}\right) + g\left(\vec{x}\right) = g\left(\vec{x}\right) + f\left(\vec{x}\right) = \left(g+f\right)\left(\vec{x}\right) .\]
Aplicamos la conmutatividad en $\displaystyle V' $ como espacio vectorial. Comprobamos la asociatividad. Si $\displaystyle f, g, h \in \Hom\left(V, V'\right) $, 
\[(\left(f+g\right) + h)\left(\vec{x}\right) = \left(f+g\right)\left(\vec{x}\right)+h\left(\vec{x}\right) = f\left(\vec{x}\right) + g\left(\vec{x}\right) + h\left(\vec{x}\right) = f\left(\vec{x}\right) + \left(g+h\right)\left(\vec{x}\right) = \left(f+\left(g+h\right)\right)\left(\vec{x}\right) .\]
Vamos a ver que $\displaystyle 0 \in \Hom\left(V, V'\right) $, definida por $\displaystyle 0\left(\vec{x}\right) = \vec{0}, \forall\vec{x} \in V $. Tenemos que si $\displaystyle \vec{x}, \vec{y} \in V $, 
\[0 \left(\vec{x} + \vec{y}\right) = \vec{0} = 0\left(\vec{x}\right) + 0\left(\vec{y}\right) .\]
Si $\displaystyle a \in \K, \vec{x} \in V $, 
\[0\left(a\vec{x}\right) = \vec{0} = a \cdot 0\left(\vec{x}\right) = a \cdot \vec{0} = \vec{0} .\]
Si $\displaystyle f \in \Hom\left(V, V'\right) $, $\displaystyle \forall \vec{x} \in V $, 
\[ \left(f+0\right)\left(\vec{x}\right) = f\left(\vec{x}\right) + 0\left(\vec{x}\right) = f\left(\vec{x}\right) .\]
Entonces, existe el elemento neutro. Vamos a ver la existencia del opuesto. Si $\displaystyle f \in \Hom\left(V, V'\right) $, tomamos $\displaystyle -f $ tal que $\displaystyle -f\left(\vec{x}\right) = \left(-f\right)\left(\vec{x}\right) $. Entonces tenemos que
\[\left(f+\left(-f\right)\right)\left(\vec{x}\right) = f\left(\vec{x}\right) + \left(-f\right)\left(\vec{x}\right) = f\left(\vec{x}\right)-f\left(\vec{x}\right) = \vec{0} .\]
Entonces, 
\[\left(f+\left(-f\right)\right)\left(\vec{x}\right) = 0\left(\vec{x}\right) .\]

Producto por escalares. Si $\displaystyle a \in \K $ y $\displaystyle f \in \Hom\left(V, V'\right) $, defino $\displaystyle \left(af\right)\left(\vec{x}\right) = af\left(\vec{x}\right) $. Vamos a ver que $\displaystyle af \in \Hom\left(V, V'\right) $.
\[
\begin{split}
	\left(af\right)\left(\vec{x}+\vec{y}\right) = af\left(\vec{x}+\vec{y}\right) = af\left(\vec{x}\right) + a f\left(\vec{y}\right) = \left(af\right)\left(\vec{x}\right)+ \left(af\right)\left(\vec{y}\right)  .
\end{split}
\]
Similarmente, 

Si $\displaystyle b \in \K $ y $\displaystyle \vec{x} \in V $, 
\[\left(af\right)\left(b\vec{x}\right) = af\left(b\vec{x}\right) = abf\left(\vec{x}\right) = b\left(af\left(\vec{x}\right)\right) = b\left(f\left(a\vec{x}\right)\right) .\]
Si $\displaystyle a \in \K $ y $\displaystyle f, g \in \Hom\left(V, V'\right) $, si $\displaystyle \vec{x}\in V $, 
\[\left(a\left(f+g\right)\right)\left(\vec{x}\right) = a\left(f+g\right)\left(\vec{x}\right) = af\left(\vec{x}\right) + a g\left(\vec{x}\right) = \left(af\right)\left(\vec{x}\right) + \left(ag\right)\left(\vec{x}\right) = \left(af+ag\right)\left(\vec{x}\right) .\]
Además, 
\[\left(1f\right)\left(\vec{x}\right) = 1 f\left(\vec{x}\right) = f\left(\vec{x}\right) .\]
\end{proof}

\section{Ejemplos de aplicaciones lineales}

\subsection{Formas Lineales}

\begin{fdefinition}[Forma lineal]
\normalfont Una \textbf{formal lineal} definida en $\displaystyle V $ es una aplicación lineal $\displaystyle \lambda : V \to \K $ \footnote{ $\displaystyle \K $ es un espacio vectorial sobre sí mismo de dimensión 1.} .
\end{fdefinition}

\begin{ftheorem}[]
\normalfont  Si $\displaystyle \lambda $ es una forma lineal no nula, entonces $\displaystyle \lambda $ es sobreyectiva (epimorfismo).  
\end{ftheorem}

\begin{proof}
Sea $\displaystyle a \in \K $. Sabemos que $\displaystyle \lambda \neq 0 $, por lo que existe $\displaystyle \vec{x}_{0} \in V $ tal que $\displaystyle \lambda\left(\vec{x}_{0}\right) = b \neq 0 $. Consideramos $\displaystyle \vec{x} = \frac{a}{b}\vec{x}_{0} \in V $, entonces
\[\lambda\left(\vec{x}\right) = \frac{a}{b}\lambda\left(\vec{x}_{0}\right) = a .\]
\end{proof}

\begin{fdefinition}[]
\normalfont Una \textbf{recta vectorial}  de $\displaystyle V $ es un subespacio vectorial de dimensión 1. Análogamente, un \textbf{plano vectorial}  de $\displaystyle V $ es un subespacio vectorial de dimensión 2. Un \textbf{hiperplano vectorial} de $\displaystyle V $ es un subespacio vectorial de dimensión $\displaystyle \dim V -1 $.
\end{fdefinition}

\begin{observation}
\normalfont Los complementarios vectoriales de los hiperplanos son rectas y viceversa.
\end{observation}

Si $\displaystyle \lambda : V \to \K $ lineal no nula, 
\[\dim \Imagen\left(\lambda\right) = 1 .\]
Por tanto, la imagen de $\displaystyle \lambda $ es una recta en $\displaystyle \K $. Además, 
\[\dim V = \dim \Imagen\left(\lambda\right) + \dim \Ker\left(\lambda\right) .\]
Si $\displaystyle \lambda \neq 0 $, tenemos que $\displaystyle \Ker\left(\lambda\right) $  es un hiperplano vectorial. 

\begin{fprop}[]
\normalfont Si $\displaystyle L \in \mathcal{L}\left(V\right)  $ es un hiperplano vectorial, existe $\displaystyle \lambda : V \to \K $ lineal no nula, tal que $\displaystyle L = \Ker\left(\lambda\right) $ \footnote{Esto nos sirve para definir los hiperplanos en dimensión infinita.} .
\end{fprop}

\begin{proof}
	Sea $\displaystyle \left\{ \vec{u}_{1}, \ldots, \vec{u}_{n-1}\right\}  $ base de $\displaystyle L $ y sea $\displaystyle \left\{ \vec{u}_{1}, \ldots, \vec{u}_{n}\right\}  $ base de $\displaystyle V $. Definimos $\displaystyle \lambda : V \to \K $ la única aplicación lineal tal que $\displaystyle \lambda\left(\vec{u}_{i}\right) = 0, \forall i = 1, \ldots n-1 $, y cogemos $\displaystyle \lambda\left(\vec{u}_{n}\right) = 1 \footnote{Vale cualquier escalar no nulo.}$. Entonces $\displaystyle \Ker\left(\lambda\right) = L $, pues su núcleo va a ser un hiperplano vectorial y este contiene a la base de $\displaystyle L $. Un hiperplano vectorial contenido en otro hiperplano vectorial es el mismo.
\end{proof}

\subsection{Homotecias vectoriales}

\begin{fdefinition}[Homotecia vectorial]
	\normalfont Sea $\displaystyle \alpha \in \K - \left\{ \vec{0}\right\}  $. La \textbf{homotecia vectorial} de razón $\displaystyle \alpha  $ es la aplicación $\displaystyle h_{\alpha }: V \to V $ tal que $\displaystyle \vec{x} \to \alpha \vec{x} $.
\end{fdefinition}

\begin{ftheorem}[]
\normalfont 
\begin{description}
	\item[(i)] Todas las homotecias vectoriales son automorfismos. Es decir, 
		\[ \forall \alpha \in \K - \left\{ \vec{0}\right\},  h_{\alpha } \in \Aut\left(V\right). \]
		\item[(ii)] La aplicación $\displaystyle \K - \left\{ \vec{0}\right\} \to H\left(V\right) $ donde $\displaystyle H\left(V\right) = \left\{ h_{\alpha }: \alpha \in \K - \left\{ \vec{0}\right\} \right\}  $ con $\displaystyle \alpha \to h_{\alpha } $, es biyectiva. 
		\item[(iii)] $\displaystyle \forall \alpha, \beta \in \K^{*} $, 
			\[h_{\alpha \beta } = h_{\alpha }\circ h_{\beta } .\]
\end{description}
\end{ftheorem}
\begin{proof}
\begin{description}
\item[(i)] Tenemos que demostrar que es aplicación lineal, pues está claro que es biyectiva. 
	\[  \forall \vec{x}, \vec{y} \in V ,  h_{\alpha }\left(\vec{x} + \vec{y}\right) = \alpha \left(\vec{x}+\vec{y}\right) = \alpha \vec{x}+\alpha\vec{y} = h_{\alpha }\left(\vec{x}\right) + h_{\alpha }\left(\vec{y}\right).\]
Si $\displaystyle a \in \K  $,
		\[h_{\alpha }\left(a\vec{x}\right) = \alpha \left(a\vec{x}\right) = a h_{\alpha }\left(\vec{x}\right) .\]
\item[(ii)] Está claro que la aplicación es sobreyectiva, tenemos que comprobar que es inyectiva: 
	Si $\displaystyle h_{\alpha } = h_{\beta } $, tenemos que $\displaystyle \forall \vec{x} \in V $,
\[\alpha \vec{x} = \beta \vec{x} .\]
Si $\displaystyle V  $ no es nulo, podemos deducir que,
\[\exists \vec{x} \neq \vec{0}, \; \left(\alpha - \beta \right) \vec{x} = \vec{0} \; \Rightarrow \; \alpha = \beta  .\]
Por lo que esta aplicación es inyectiva.
\item[(iii)]
\[\forall \vec{x} \in V, \; h_{\beta }\circ h_{\alpha }\left(\vec{x}\right) = h_{ \beta }\left(\alpha \vec{x}\right) = \beta \alpha \vec{x} = h_{\beta \alpha}\left(\vec{x}\right).\]
\end{description}
\end{proof}


\begin{ftheorem}[]
\normalfont Sea $\displaystyle f : V \to V $ lineal. Entonces $\displaystyle f $ es una homotecia vectorial si y solo si $\displaystyle \forall L \in \mathcal{L}\left(V\right) $ recta vectorial $\displaystyle f\left(L\right) = L $.
\end{ftheorem}

\begin{proof}
\begin{description}
	\item[(i)] Si $\displaystyle \alpha \in \K^{*} $ \footnote{A lo largo de esta demostración, se utila el símbolo $\displaystyle ^{*} $ para denotar que se excluye el 0.} tal que $\displaystyle f = h_{\alpha} $ , sea $\displaystyle L \in \mathcal{L}\left(V\right) $ una recta vectorial. Entonces, existe $\displaystyle \vec{x}_{0} \in L^{*}$ tal que $\displaystyle \left\{ \vec{x}_{0}\right\}  $ es base de $\displaystyle L $. Entonces, $\displaystyle L = \left\{ a\vec{x}_{0} \; : \; a \in \K\right\}  $. Además, tenemos que 
		\[h_{\alpha }\left(L\right) = \left\{ \alpha a \vec{x}_{0}\; : \; a \in \K \right\} = L .\]
	\item[(ii)] Si $\displaystyle  \vec{x} \in V^{*} $, entonces $\displaystyle L\left( \left\{ \vec{x}\right\} \right)  $ es una recta vectorial. Tenemos que
		\[f\left(L\left( \left\{ \vec{x}\right\} \right)\right) = L\left( \left\{ \vec{x}\right\} \right) .\]
		Si $\displaystyle \vec{x} \in L\left( \left\{ \vec{x}\right\} \right) $, 
		\[f\left(\vec{x}\right) \in L\left( \left\{ \vec{x}\right\} \right) \Rightarrow \exists \alpha_{\vec{x}} \in \K, \; f\left(\vec{x}\right) = \alpha_{\vec{x}}\vec{x} .\]
Tenemos que ver que $\displaystyle \alpha  $ es único. Si $\displaystyle \vec{x}, \vec{y} \in V^{*} $, tenemos que ver si son linealmente dependientes o linealmente independientes. Si son linealmente dependientes, $\displaystyle \exists a \in \K^{*} $ tal que 
\[\vec{x} = a \vec{y} .\]
Entonces, $\displaystyle f\left(\vec{x}\right) = a f\left(\vec{y}\right) = f\left(a\vec{y}\right) = \alpha_{\vec{y}} a \vec{y} = \alpha_{\vec{y}}\vec{x} $. Tenemos que
\[\left(\alpha_{\vec{x}}-\alpha_{\vec{y}}\right)\left(\vec{x}\right) = \vec{0} \Rightarrow \alpha_{\vec{x}}= \alpha_{\vec{y}} .\]
Si son linealmente independientes, $\displaystyle \vec{x} + \vec{y} \neq 0 $, 
\[f\left(\vec{x}+ \vec{y}\right) = f\left(\vec{x}\right) + f\left(\vec{y}\right) = \alpha_{\vec{x}+\vec{y}}\left(\vec{x}+\vec{y}\right) = \alpha_{\vec{x}+\vec{y}}\vec{x} + \alpha_{\vec{x}+\vec{y}}\vec{y} = \alpha_{\vec{x}}\vec{x}+\alpha_{\vec{y}}\vec{y} .\]
Entonces tenemos que
\[\left(\alpha_{\vec{x}+\vec{y}}-\alpha_{ \vec{x}}\right)\vec{x} = \left(\alpha_{y}-\alpha_{\vec{x}+\vec{y}}\right) \vec{y} \Rightarrow \alpha_{\vec{x}} = \alpha_{\vec{x}+\vec{y}} = \alpha_{\vec{y}} .\]
Sea $\displaystyle \alpha = \alpha_{\vec{x}} $, con $\displaystyle \vec{x} \in V^{*} $. Entonces si $\displaystyle \vec{y} \in V^{*} $ tenemos que $\displaystyle f\left(\vec{y}\right) = \alpha\vec{y} $ y $\displaystyle \vec{0} = \alpha \vec{0} $. Por tanto, se trata de una homotecia lineal.
\end{description}
\end{proof}

\subsection{Proyecciones}
Supongamos que $\displaystyle L_{1}, L_{2} \in \mathcal{L} \left(V\right)$ y seam $\displaystyle L_{1} \oplus L_{2} = V $. $\displaystyle \forall \vec{x} \in V , \exists \vec{x}_{1} \in L_{1}, \exists\vec{x}_{2} \in L_{2}$, $\displaystyle \vec{x} = \vec{x}_{1}+\vec{x}_{2} $.
\begin{fdefinition}[Proyección]
\normalfont La \textbf{proyección} de base $\displaystyle L_{1} $ (respecto a la base $\displaystyle L_{2} $) y dirección $\displaystyle L_{2} $ (respecto a $\displaystyle L_{1} $) es la aplicación 
\[
\begin{split}
& p_{1} : V \to V \\
& \vec{x} = \vec{x}_{1}+\vec{x}_{2} \to \vec{x}_{1}.
\end{split}
\]
Respecto a $\displaystyle p_{2}: V \to V $ tal que $\displaystyle \vec{x} = \vec{x}_{1}+\vec{x}_{2} \to \vec{x}_{2}$.
\end{fdefinition}

\begin{ftheorem}[]
\normalfont 
\begin{description}
\item[(i)] $\displaystyle p_{1}+p_{2} = id _{V}. $  
\item[(ii)] $\displaystyle p_{1}\circ p_{2} = 0 $ y $\displaystyle p_{2} \circ p_{1} = 0 $.
\item[(iii)] $\displaystyle p_{1} $ y $\displaystyle p_{2} $ son lineales.
\item[(iv)] $\displaystyle p_{1} \circ p_{1} = p_{1} $ y $\displaystyle p_{2} \circ p_{2} = p_{2} $.
\end{description}
\end{ftheorem}

\begin{proof}
\begin{description}
\item[(i)] $\displaystyle \forall \vec{x}_{1} \in L_{1}, \vec{x}_{2} \in L_{2} $,
\[\left(p_{1}+p_{2}\right)\left(\vec{x}_{1}+\vec{x}_{2}\right) = p_{1}\left(\vec{x}_{1}+\vec{x}_{2}\right) + p_{2}\left(\vec{x}_{1}+\vec{x}_{2}\right) = \vec{x}_{1}+\vec{x}_{2} .\]
\item[(ii)] 
\[p_{1} \circ p_{2}\left(\vec{x}_{1}+\vec{x}_{2}\right) = p_{1}\left(\vec{x}_{2}\right) = p_{1}\left(\vec{0}+\vec{x}_{2}\right) = \vec{0} .\]
\item[(iii)] \[ p_{1}\left(\vec{x}_{1}+\vec{x}_{2} + \vec{y}_{1}+\vec{y}_{2}\right) = p_{1}\left(\left(\vec{x}_{1}+\vec{y}_{1}\right)+\left(\vec{x}_{2}+\vec{y}_{2}\right)\right) = p_{1}\left(\vec{x}_{1}+\vec{x}_{2}\right) + p_{1}\left(\vec{y}_{1}+\vec{y}_{2}\right) .\]
\[p_{1}\left(a\vec{x}\right) = p_{1}\left(a\left(\vec{x}_{1}+\vec{x}_{2}\right)\right) = p_{1}\left(a\vec{x}_{1}+a\vec{x}_{2}\right) = a\vec{x}_{1} = ap_{1}\left(\vec{x}_{1}+\vec{x}_{2}\right) .\]
\item[(iv)] Si $\displaystyle \vec{x} \in V $,
	\[p_{1} \circ p_{1} \left(\vec{x}_{1}+\vec{x}_{2}\right) = p_{1}\left(\vec{x}_{1}\right) = \vec{x}_{1} .\]
\end{description}
\end{proof}

\begin{fdefinition}[]
\normalfont Un endomorfismo $\displaystyle  p : V \to V $ es un proyector si $\displaystyle p^{2} = p $.
\end{fdefinition}

\begin{ftheorem}[]
\normalfont Sea $\displaystyle p $ un proyector definido en $\displaystyle V $, entonces $\displaystyle \left(id_{V} - p\right) $ es un proyector y $\displaystyle V = L_{1}\oplus L_{2} $ donde $\displaystyle L_{1} = \Imagen\left(p\right) = \Ker\left(p- id _{V}\right) $ y $\displaystyle L_{2} = \Ker\left(p\right) = \Imagen\left(p-id _{V}\right) $. 
\end{ftheorem}

\begin{proof}
$\displaystyle \forall \vec{x} \in V $,
\[ \left(id _{V} - p\right)^{2}\left(\vec{x}\right) = \left(id _{V} - p\right)\left(\vec{x}-p\left(\vec{x}\right)\right) = \vec{x}-p\left(\vec{x}\right)-p\left(\vec{x}\right) + p^{2}\left(\vec{x}\right) = \vec{x}-p\left(\vec{x}\right) = \left(id _{V}- p\right)\left(\vec{x}\right).\]
Vamos a ver que $\displaystyle V = L_{1} \oplus L_{2} $ donde $\displaystyle L_{1} = \Imagen\left(p\right) $ y $\displaystyle L_{2} = \Ker\left(p\right) $.
\[\forall \vec{x} \in V, \; \vec{x} = p\left(\vec{x}\right) + \vec{x}-p\left(\vec{x}\right), \; p\left(\vec{x}-p\left(\vec{x}\right)\right) = p\left(\vec{x}\right) - p^{2}\left(\vec{x}\right) = \vec{0} .\]
Tenemos que $\displaystyle p\left(\vec{x}\right) \in \Imagen\left(p\right) $ y $\displaystyle \left(id _{V}-p\right)\left(\vec{x}\right) \in \Ker\left(p\right) $.
Entonces $\displaystyle V = L_{1}\oplus L_{2} $. \\ \\
Si $\displaystyle \vec{x} \in L_{1}\cap L_{2} $, existe $\displaystyle \vec{y} \in V $ tal que $\displaystyle p\left(\vec{y}\right) = \vec{x} $ y $\displaystyle p\left(\vec{x}\right) = \vec{0} $ . Entonces, 
\[p\left(\vec{x}\right) = \vec{0} \Rightarrow p^{2}\left(\vec{y}\right) = \vec{0} .\]
Por tanto, $\displaystyle \vec{x} = \vec{0} $ y $\displaystyle p\left(\vec{y}\right)= \vec{x} $.\\ \\
Vamos a comprobar que 
\[\Imagen\left(p\right) = \Ker\left(p-id _{V}\right) .\]
\begin{description}
\item[(i)] $\displaystyle \vec{x} \in \Imagen\left(p\right) $, entonces $\displaystyle \exists \vec{y} \in V $ tal que $\displaystyle p\left(\vec{y}\right) = \vec{x} $. 
	\[\left(p-id _{V}\right)\left(\vec{x}\right) = \left(p- id _{V}\right)\left(p\left(\vec{y}\right)\right) = \underbrace{p^{2} \left(\vec{y}\right)}_{p} -p\left(\vec{y}\right) = \vec{0} .\]
	Por tanto, $\displaystyle \vec{x} \in \Ker\left(p-id _{V}\right) $.
\item[(ii)] Si $\displaystyle \vec{x} \in \Ker\left(p- id _{V}\right) $, tenemos que
	\[\left(p- id _{V}\right)\left(\vec{x}\right) = p\left(\vec{x}\right)- \vec{x} = \vec{0} .\]
	Por tanto, $\displaystyle \vec{x} = p\left(\vec{x}\right) $ y, consecuentemente, $\displaystyle \vec{x} \in \Imagen\left(p\right) $.
\end{description}
La proposición $\displaystyle \Ker\left(p\right) = \Imagen\left(p- id _{V}\right) $ se demuestra igual.
\end{proof}

\subsection{Simetrías vectoriales}
Asumimos que el cuerpo $\displaystyle \K $ es de característica distinta de 2. 
\begin{fdefinition}[]
\normalfont Un endomorfimos $\displaystyle s $ de $\displaystyle V $ es involutivo si $\displaystyle s^{2} = id _{V} $ ($\displaystyle \iff \exists s^{-1} $ y $\displaystyle s^{-1} = s $). 
\end{fdefinition}

\begin{fdefinition}[]
\normalfont $\displaystyle s $ es la simetría vectorial de base $\displaystyle L_{1} $ y dirección $\displaystyle L_{2} $.
\end{fdefinition}

\begin{ftheorem}[]
\normalfont Sea $\displaystyle s : V \to V $ un endomorfismo involutivo. Sea $\displaystyle L_{1} = \Imagen\left(s + id _{V}\right) $ y $\displaystyle L_{2} = \Imagen\left(id _{V}-s\right) $. Entonces, $\displaystyle L_{1} \oplus L_{2} = V $ y $\displaystyle s = p_{1}- p_{2} $ ($\displaystyle p_{1}, p_{2} $ son proyecciones). Diremos que $\displaystyle s $  es la simetría de base $\displaystyle L_{1} $ y dirección $\displaystyle L_{2} $.
\end{ftheorem}
\[\Imagen\left(s - id _{V}\right) = \Ker\left(s+ id _{V}\right) \quad \text{y} \quad \Imagen\left(s + id _{V}\right) = \Ker\left(s - id _{V}\right) .\]

\begin{proof}
Sea $\displaystyle L_{1} = \Imagen\left(s + id _{ V}\right) $ y $\displaystyle L_{2} = \Imagen\left(id _{V}-s\right) $.
\[\forall \vec{x} \in V, \; \vec{x} = \frac{ \vec{x} - s\left(\vec{x}\right)}{2} + \frac{ \vec{x} + s\left(\vec{x}\right)}{2} = \left(id _{V}- s\right)\left(\frac{\vec{x}}{2}\right) + \left(id _{V} + s\right)\left(\frac{\vec{x}}{2}\right) .\]
Sea $\displaystyle \vec{x} \in L_{1} \cap L_{2} $, 
\[\vec{x}\in L_{1}, \;\exists \vec{y} \in V, \; \left(s+ id _{V}\right)\left(\vec{y}\right) = \vec{x} \Rightarrow s\left(\vec{x}\right) = s\left(\left(s+id _{V}\right)\left(\vec{y}\right)\right) = s\left(s\left(\vec{y}\right)+\vec{y}\right) = s^{2}\left(\vec{y}\right) + s\left(\vec{y}\right) = \vec{y}+s\left(\vec{y}\right) = \vec{x}.\]
\[\vec{x} \in L_{2}, \; \exists \vec{z} \in V, \; \left(id _{V}-s\right)\left(\vec{z}\right) = \vec{x} \Rightarrow s\left(\vec{x}\right) = s\left(\left(id _{V}-s\right)\left(\vec{z}\right)\right) = s\left(\vec{z}\right) - s^{2}\left(\vec{z}\right) = s\left(\vec{z}\right)-\vec{z} = - \vec{x} .\]
Entonces, si $\displaystyle  \vec{x} \in L_{1} \cap L_{2} $, 
\[ s\left(\vec{x}\right) = \vec{x} \quad \text{y} \quad s\left(\vec{x}\right) = - \vec{x} \Rightarrow \vec{x} = \vec{0} .\]
\end{proof}

\begin{observation}
\normalfont Tenemos que $\displaystyle p_{1}\left(\vec{x}\right) = \left(id _{V}+ s\right)\left(\frac{\vec{x}}{2}\right) $, y $\displaystyle p_{2}\left(\vec{x}\right) = \left(id _{V} - s\right)\left(\frac{\vec{x}}{2}\right) $. Entonces, $\displaystyle s = p_{1}-p_{2} $. Además, diremos que $\displaystyle L_{1} = \Ker\left(s-id _{V}\right) $ y $\displaystyle L_{2} = \Ker\left(s+id _{V}\right) $. Esto se puede demostrar con ambas implicaciones: $\displaystyle L_{1} \subset \Ker\left(id _{V}-s\right) $ y viceversa.
\end{observation}

\begin{figure}
\centering
\includegraphics[scale = 0.5 ]{~/Desktop/Images/simetrias_vectoriales.png}
\caption{Ejemplo de simetrías vectoriales}
\end{figure}

\section{Espacio vectorial dual}

\begin{fdefinition}[Espacio dual]
\normalfont Llamaremos \textbf{espacio dual}  de $\displaystyle V $ y lo representaremos por $\displaystyle V^{*} = \Hom\left(V, \K\right) $.
\end{fdefinition}

\begin{ftheorem}[]
	\normalfont Sea $\displaystyle \left\{ \vec{u}_{1}, \ldots, \vec{u}_{n}\right\}  $ base de $\displaystyle V $, entonces las formas lineales $\displaystyle \left\{ \omega^{1}, \ldots, \omega^{n}\right\}  $ definidas por ser las únicas formas lineales tales que 
\[
\begin{split}
 \omega^{i}\left(\vec{u}_{j}\right) = 
\begin{cases}
1, \; \text{si} \; i = j \\
0, \; \text{si} \; i \neq j
\end{cases}
\end{split}
\]
que llamaremos base dual de $\displaystyle \left\{ \vec{u}_{1}, \ldots, \vec{u}_{n}\right\}  $.
\end{ftheorem}

\begin{proof}
	Tenemos que ver que $\displaystyle \left\{ \omega^{1}, \ldots, \omega^{n}\right\}  $ son linealmente independientes. Sean $\displaystyle a_{i} \in \K $ tales que
\[
\begin{split}
a_{1}\omega^{1} + \cdots + a_{n}\omega^{n} = 0 \in V^{*} .
\end{split}
\]
Tenemos que $\displaystyle \forall i = 1, \ldots, n $,
\[
\begin{split}
0\left(\vec{u}_{i}\right) = 0 = \left(a_{1}\omega^{1} + \cdots + a_{n}\omega^{n}\right)\left(\vec{u}_{i}\right) = a_{1}\omega^{1}\left(\vec{u}_{i}\right)+ \cdots + a_{i}\omega^{i}\left(\vec{u}_{i}\right) + \cdots + a_{n}\omega^{n}\left(\vec{u}_{n}\right) = a_{i} .
\end{split}
\]
Por tanto, $\displaystyle a_{i} = 0, \; \forall i = 1, \ldots , n $. \\ \\
Ahora vamos a ver que son sistema de generadores. Si $\displaystyle \lambda \in V^{*} $, $\displaystyle \forall i = 1, \ldots, n $, 
\[
\begin{split}
	\left(\lambda\left(\vec{u}_{1}\right)\omega^{1} + \lambda\left(\vec{u}_{2}\right)\omega^{2}+ \cdots + \lambda\left(\vec{u}_{n}\right)\omega^{n}\right)\left(\vec{u}_{i}\right) = \lambda\left(\vec{u}_{i}\right) .
\end{split}
\]
Es decir, tenemos que 
\[\lambda = \lambda\left(\vec{u}_{1}\right)\omega^{1} + \cdots + \lambda\left(\vec{u}_{n}\right)\omega^{n} .\]
\end{proof}

\begin{observation}
\normalfont 
\[
\begin{split}
& \vec{x} = a^{1} \vec{u}_{1} + \cdots + a^{n}\vec{u}_{n} \\
& \forall i = 1, \ldots, n, \; w^{i}\left(\vec{x}\right) = w^{i}\left(a^{1}\vec{u}_{1} + \cdots + a^{n}\vec{u}_{n}\right) = a^{1}w^{i}\left(\vec{u}_{1}\right) + \cdots + a^{n}w^{i}\left(\vec{u}_{n}\right) = a^{i}.
\end{split}
\]
\end{observation}

\begin{fcolorary}[]
\normalfont $\displaystyle \dim V = \dim V^{*} $.
\end{fcolorary}

\begin{observation}
\normalfont $\displaystyle \left(V^{*}\right)^{*} = V^{**} $ bidual. $\displaystyle \forall \vec{x} \in V $, 
\[
\begin{split}
	\theta_{\vec{x}}: & V^{*} \to \K\\
& \lambda \to \theta_{\vec{x}}\left(\lambda \right) = \lambda\left(\vec{x}\right), \forall \lambda \in V^{*}.
\end{split}
\]
Tenemos que, $\displaystyle \theta_{\vec{x}} \in V^{**} $, además, $\displaystyle \forall\lambda_{1}, \lambda_{2} \in V^{*} $,
\[\theta_{\vec{x}} \left(\lambda_{1}+\lambda_{2}\right)= \left(\lambda_{1}+\lambda_{2}\right)\left(\vec{x}\right) = \left(\lambda_{1}+\lambda_{2}\right)\left(\vec{x}\right) = \lambda_{1}\left(\vec{x}\right) + \lambda_{2}\left(\vec{x}\right) = \theta_{\vec{x}}\left(\lambda_{1}\right) + \theta_{\vec{x}}\left(\lambda_{2}\right) .\]
Además, si $\displaystyle a \in \K $, $\displaystyle \forall\lambda \in V^{*} $, 
\[\theta_{\vec{x}}\left(a\lambda\right) = \left(a\lambda\right)\left(\vec{x}\right) = a \lambda\left(\vec{x}\right) = a \theta_{\vec{x}}\left(\vec{\lambda}\right) .\]
Entonces tenemos que $\displaystyle \theta_{\vec{x}} $ es una aplicación lineal y, por tanto, $\displaystyle \theta_{\vec{x}} \in V^{**} $ .
\end{observation}

\begin{ftheorem}[]
\normalfont La aplicación
\[
\begin{split}
	\theta : & V \to V^{**} \\
		 & \vec{x} \to \theta_{\vec{x}},
\end{split}
\]
es lineal.
\end{ftheorem}

\begin{proof}
Tenemos que $\displaystyle \forall \lambda \in V^{*} $,
\[\theta_{\vec{x}+\vec{y}}\left(\lambda \right) = \lambda\left(\vec{x}+\vec{y}\right) = \lambda\left(\vec{x}\right) + \lambda\left(\vec{y}\right) = \theta_{\vec{x}}\left(\lambda \right) + \theta_{\vec{y}}\left(\lambda \right) = \left(\theta_{\vec{x}}+\theta_{\vec{y}}\right)\left(\lambda\right) .\]
Entonces tenemos que 
\[\theta_{\vec{x}+\vec{y}} = \theta_{\vec{x}} + \theta_{\vec{y}} \iff \theta\left(\vec{x}+\vec{y}\right) = \theta\left(\vec{x}\right) + \theta\left(\vec{y}\right) .\]
Además, si $\displaystyle a \in \K $ y $\displaystyle \vec{x} \in V $:
\[\theta\left(a\vec{x}\right) = \theta_{a\vec{x}} = a \theta_{\vec{x}} = a\theta\left(\vec{x}\right) .\]
Entonces, $\displaystyle \forall \lambda \in V^{*} $,
\[\theta_{a\vec{x}}\left(\lambda\right) = \lambda\left(a\vec{x}\right) = a\lambda\left(\vec{x}\right) = \left(a\theta_{\vec{x}}\right)\left(\lambda\right) .\]
Por tanto,
\[\theta_{a\vec{x}} = a\theta_{\vec{x}} \iff \theta\left(a\vec{x}\right) = a \theta\left(\vec{x}\right) .\]
\end{proof}

\begin{fdefinition}[]
\normalfont Sea $\displaystyle V^{**} = \Hom\left(V^{*}, \K\right) $ el espacio dual de $\displaystyle V^{*} $. Entonces es el espacio \textbf{bidual} de $\displaystyle V $. Además, 
\[\dim V^{**} = \dim V^{*} = \dim V .\]
\end{fdefinition}

\begin{ftheorem}[]
\normalfont $\displaystyle \theta $ es un isomorfismo de espacios vectoriales.
\end{ftheorem}

\begin{proof}
Como tienen la misma dimensión, nos basta con demostrar que es inyectiva. Sea $\displaystyle \vec{x} \in \Ker\left(\theta \right) $, queremos ver que $\displaystyle \vec{x} = \vec{0} $. Tenemos que $\displaystyle \forall \lambda \in V^{*} $, 
\[\theta_{\vec{x}}\left(\lambda \right) = \lambda\left(\vec{x}\right) = \vec{0} .\]
Por tanto, tenemos que $\displaystyle \vec{x} = \vec{0} $ (porque todas las formas lineales devuelven 0 si insertas 0) y, consecuentemente, $\displaystyle \theta $ es isomorfismo.
\end{proof}

\begin{ftheorem}[]
	\normalfont Sea $\displaystyle \left\{ \vec{u}_{1}, \ldots, \vec{u}_{n}\right\}  $ base de $\displaystyle V $ y sea $\displaystyle \left\{ \omega^{1}, \ldots, \omega^{n}\right\}  $ su dual. Tenemos que $\displaystyle \left\{ \theta_{\vec{u}_{1}}, \ldots, \theta_{\vec{u}_{n}}\right\}  $ es base de $\displaystyle V^{**} $. Además, $\displaystyle \forall i, j= 1, \ldots, n $, 
	\[\theta_{\vec{u}_{i}}\left(\omega^{j}\right) = \omega^{j}\left(\vec{u}_{i}\right) =
	\begin{cases}
	1, \; \text{si} \; i = j \\
	0, \; \text{si} \; i \neq j
	\end{cases}
	.\]
	Es decir, tenemos que $\displaystyle \left\{ \theta_{\vec{u}_{1}}, \ldots, \theta_{\vec{u}_{n}}\right\}  $ es la base dual de $\displaystyle \left\{ \omega^{1}, \ldots, \omega^{n}\right\}  $.	
\end{ftheorem}

\begin{observation}
\normalfont Si $\displaystyle f: V \to V' $ lineal y $\displaystyle \lambda' \in V^{'*} $, tenemos que
\[\lambda'\circ f : V \to \K, \; \lambda'\circ f \in V^{*}.\]
Se tiene definida una aplicación (la aplicación dual de $\displaystyle f $) 
\[
\begin{split}
	f^{*} :& V^{'*} \to V^{*}\\
	& \lambda' \to f^{*}\left(\lambda '\right) = \lambda' \circ f
\end{split}
\]
\end{observation}

\begin{fprop}[]
\normalfont $\displaystyle f^{*} $ es lineal.
\end{fprop}

\begin{proof}
Tenemos que $\displaystyle \forall \lambda'_{1}, \lambda'_{2} \in V'^{*} $, 
\[f^{*}\left(\lambda'_{1} + \lambda'_{2}\right) = \left(\lambda'_{1} + \lambda'_{2}\right) \circ f .\]
Además, $\displaystyle \forall\vec{x} \in V $,
\[
\begin{split}
	\left(\left(\lambda_{1}+\lambda'_{2}\right)\circ f\right)\left(\vec{x}\right) = & \left(\lambda'_{1}+\lambda'_{2}\right)\left(f\left(\vec{x}\right)\right) = \lambda'_{1}\left(f\left(\vec{x}\right)\right) + \lambda'_{2}\left(f\left(\vec{x}\right)\right) = \left(\lambda'_{1}\circ f\right)\left(\vec{x}\right) + \left(\lambda'_{2}\circ f\right)\left(\vec{x}\right) \\
	= & f^{*}\left(\lambda'_{1}\right)\left(\vec{x}\right) + f^{*}\left(\lambda'_{2}\right)\left(\vec{x}\right) = \left(f^{*}\left(\lambda'_{1}\right)+f^{*}\left(\lambda'_{2}\right)\right)\left(\vec{x}\right) .
\end{split}
\]
Además, $\displaystyle \forall a \in \K, \forall \lambda'\in V'^{*} $, tenemos que 
\[f^{*}\left(a\lambda'\right) = \left(a\lambda'\right) \circ f = a f^{*}\left(\lambda'\right) .\]
Tenemos que $\displaystyle \forall \vec{x}\in V $, 
\[\left(a\lambda'\right) \circ f\left(\vec{x}\right) = a\left(\lambda' \circ f\right)\left(\vec{x}\right) = a f^{*}\left(\lambda'\right)\left(\vec{x}\right) .\]
\end{proof}

\begin{ftheorem}[]
\normalfont Se tiene definida una aplicación
\[
\begin{split}
	* :  \Hom\left(V, V'\right) & \to \Hom\left(V'^{*}, V^{*}\right)\\
	f & \to f^{*}.
\end{split}
\]
Tenemos que $\displaystyle * $ es lineal.
\end{ftheorem}

\begin{proof}
Si $\displaystyle f, g \in \Hom\left(V, V'\right) $, tenemos que $\displaystyle \forall \lambda' \in V'^{*} $, 
\[\left(f+g\right)^{*}\left(\lambda'\right) = \lambda' \circ \left(f + g\right)= f^{*}\left(\lambda'\right) + g^{*}\left(\lambda'\right) = \left(f^{*}+g^{*}\right)\left(\lambda'\right) .\]
Entonces tenemos que $\displaystyle \forall \vec{x}\in V $, 
\[\lambda'\circ\left(f+g\right)\left(\vec{x}\right) = \lambda'\left(f\left(\vec{x}\right)+g\left(\vec{x}\right)\right) = \left(f^{*}\left(\lambda'\right)+g^{*}\left(\lambda'\right)\right)\left(\vec{x}\right) .\]
Si $\displaystyle a \in \K $ y $\displaystyle f \in \Hom\left(V, V'\right) $, queremos ver que
\[ *\left(af\right) = \left(af\right)^{*} = af^{*}.\]
Tenemos que $\displaystyle \forall \lambda' \in V'^{*} $, 
\[\left(af\right)^{*}\left(\lambda'\right) = \lambda' \circ \left(af\right) = a f^{*}\left(\lambda'\right) .\]
Entonces, $\displaystyle \forall \vec{x} \in V $, 
\[
\begin{split}
	\left(\lambda' \circ \left(af\right)\right)\left(\vec{x}\right) = \lambda'\left(af\left(\vec{x}\right)\right) = a \lambda'\left(f\right)\left(\vec{x}\right) = af^{*}\left(\lambda'\right)\left(\vec{x}\right) .
\end{split}
\]
\end{proof}

\begin{fprop}[]
\normalfont Sea $\displaystyle f \in \Hom\left(V, V'\right) $ y $\displaystyle g \in \Hom\left(V', V''\right) $. Se cumple que 
\[\left(g \circ f\right)^{*} = f^{*} \circ g^{*} .\]
\end{fprop}

\begin{proof}
Tenemos que $\displaystyle \forall \lambda''\in V''^{*} $, 
\[\left(g\circ f\right)^{*}\left(\lambda''\right)=\lambda''\circ\left(g\circ f\right) = \left(\lambda''\circ g\right)\circ f = \left(g^{*}\left(\lambda''\right)\right)\circ f = f^{*}\left(g^{*}\left(\lambda''\right)\right) = \left(f^{*}\circ g^{*}\right)\left(\lambda''\right) .\]
\end{proof}

\begin{fprop}[]
\normalfont 
\[\left(id _{V}\right)^{*} = id _{V^{*}} .\]
\end{fprop}

\begin{proof}
Tenemos que $\displaystyle \forall \lambda \in V^{*} $, 
\[\left(id _{V}\right)^{*}\left(\lambda\right) = \lambda \circ id _{V} = id _{V^{*}}\left(\lambda\right) .\]
\end{proof}

\begin{fprop}[]
\normalfont Si $\displaystyle f: V \to V' $ isomorfismo, $\displaystyle f^{*} $ isomorfismo y $\displaystyle \left(f^{-1}\right)^{*} = \left(f^{*}\right)^{-1} $.
\end{fprop}

\begin{proof}
Existe $\displaystyle f^{-1}: V' \to V $ tal que $\displaystyle f^{-1}\circ f = id _{V} $ y $\displaystyle f \circ f^{-1} = id _{V'} $. Entonces tenemos que 
\[f^{*} \circ \left(f^{-1}\right)^{*} = \left(f^{-1}\circ f\right)^{*} = \left(id _{V}\right)^{*} = id _{V^{*}} .\]
Similarmente, 
\[\left(f^{-1}\right)^{*} \circ f^{*} = \left(f\circ f^{-1}\right)^{*} = \left(id _{V'}\right)^{*} = id _{V'^{*}} .\]
\end{proof}

Si $\displaystyle f \in \Hom\left(V, V'\right) $, tenemos que $\displaystyle \left(f^{*}\right)^{*} = f^{**} \in\Hom\left(V^{**}, V'^{**}\right) $. Tenemos que $\displaystyle \forall \vec{x} \in V, \theta_{\vec{x}}\in V^{**} $, entonces
\[f^{**}\left(\theta_{\vec{x}}\right) = \theta_{\vec{x}} \circ f^{*} = \theta \circ f .\]
\[ \forall \lambda'\in V'^{*}, \theta_{\vec{x}}\circ f^{*}\left(\lambda'\right) = \theta_{\vec{x}}\left(\lambda'\circ f\right) = \lambda'\circ f\left(\vec{x}\right) = \lambda'\left(f\left(\vec{x}\right)\right) = \theta_{f\left(\vec{x}\right)}\left(\lambda'\right) .\]
Queremos ver que 
\[\theta^{-1}f^{**}\theta_{\vec{x}} = f .\]

\begin{figure}
\centering
\includegraphics[width=0.5\linewidth]{~/Desktop/Images/espacio_dual.png}
\caption{Resumen del espacio dual}
\label{ enter label$}
\end{figure}

\subsection{Anulador de un subespacio}

\begin{fdefinition}[Ortogonal y anulador]
	\normalfont Si $\displaystyle \emptyset \neq A \subset V $. Llamaremos \textbf{ortogonal} a $\displaystyle A $ y la expresaremos por:
	\[ A^{\perp} = \left\{ \lambda \in V^{*} \; : \; \lambda\left(\vec{x}\right) = \vec{0}, \forall\vec{x} \in A\right\} = \left\{ \lambda \in V^{*} \; : \; A \subset \Ker\left(\lambda\right)\right\} .\]
\end{fdefinition}

\begin{fprop}[]
\normalfont 
\[A^{\perp} \in \mathcal{L}\left(V^{*}\right) .\]
\end{fprop}

\begin{proof}
Sean $\displaystyle \lambda_{1}, \lambda_{2} \in A^{\perp} $, vamos a ver que $\displaystyle \lambda_{1} + \lambda _{2} \in A^{\perp} $:
\[\forall\vec{x} \in A, \; \left(\lambda_{1}+\lambda_{2}\right)\left(\vec{x}\right) = \lambda_{1}\left(\vec{x}\right) + \lambda_{2}\left(\vec{x}\right) = 0 .\]
Por lo que, $\displaystyle \lambda_{1} + \lambda_{2} \in A^{\perp} $. Similarmente, si $\displaystyle \lambda \in A^{\perp} $ y $\displaystyle a \in \K $, vamos a ver que $\displaystyle a\lambda \in A^{\perp} $:
\[\forall\vec{x} \in A, \left(a\lambda\right)\left(\vec{x}\right) = a\lambda\left(\vec{x}\right) = a \cdot 0= 0.\]
Por lo que $\displaystyle a\lambda \in A^{\perp} $.
\end{proof}

\begin{fprop}[]
\normalfont Si $\displaystyle \emptyset \neq A \subset B \subset V $, entonces $\displaystyle B^{\perp} \subset A^{\perp} $.
\end{fprop}

\begin{proof}
Si $\displaystyle \lambda \in B^{\perp} $, tenemos que $\displaystyle \forall \vec{x} \in B $, $\displaystyle \lambda\left(\vec{x}\right) = 0 $. Como $\displaystyle A \subset B $, tenemos que $\displaystyle \forall \vec{x} \in A $, $\displaystyle \lambda\left(\vec{x}\right) = 0 $, por lo que $\displaystyle \lambda \in A^{\perp} $.
\end{proof}

\begin{fprop}[]
\normalfont 
\[A^{\perp} = L\left(A\right)^{\perp} .\]
\end{fprop}

\begin{proof}
Si $\displaystyle A \subset L\left(A\right) $ tenemos que $\displaystyle L\left(A\right)^{\perp} \subset A^{\perp} $. Sea $\displaystyle \lambda \in A^{\perp} $, 
\[\forall \vec{x}\in L\left(A\right), \exists p \in \N, \exists a^{1}, \ldots, a^{p} \in \K, \exists \vec{x}_{1}, \ldots, \vec{x}_{p} \in A .\]
\[\vec{x} = a^{1}\vec{x}_{1}+\cdots + a^{p}\vec{x}_{p} .\]
Entonces, 
\[\lambda\left(\vec{x}\right) = \lambda\left(a^{1}\vec{x}_{1}+\cdots + a^{p}\vec{x}_{p}\right) = a^{1}\lambda\left(\vec{x}_{1}\right)+\cdots + a^{p}\lambda\left(\vec{x}_{p}\right) = 0 .\]
\end{proof}

\begin{fprop}[]
\normalfont Sean $\displaystyle L_{1}, L_{2} \in \mathcal{L}\left(V\right) $, entonces tenemos que $\displaystyle \left(L_{1}+L_{2}\right)^{\perp} = L_{1}^{\perp} \cap L_{2}^{\perp} $ y $\displaystyle \left(L_{1}\cap L_{2}\right)^{\perp} = L_{1}^{\perp} + L_{2}^{\perp} $. 
\end{fprop}

\begin{proof}
\begin{description}
\item[(1.1)]  Tenemos que $\displaystyle L_{1} \subset L_{1} + L_{2} $ y $\displaystyle L_{2} \subset L_{1}+L_{2} $. Entonces sabemos que $\displaystyle \left(L_{1}+L_{2}\right)^{\perp} \subset L_{1}^{\perp} $ y $\displaystyle \left(L_{1}+L_{2}\right)^{\perp}\subset L_{2}^{\perp} $. Por tanto
\[\left(L_{1}+L_{2}\right)^{\perp} \subset L_{1}^{\perp} \cap L_{2}^{\perp} .\]
\item[(1.2)] Si $\displaystyle \lambda \in L_{1}^{\perp} \cap L_{2}^{\perp} $, tenemos que 
\[
\begin{split}
&  \forall \vec{x} \in L_{1}^{\perp}, \lambda\left(\vec{x}\right) = \vec{0} \\
& \forall \vec{y} \in L_{2}^{\perp}, \lambda\left(\vec{y}\right) = \vec{0}.
\end{split}
\]
Tenemos que 
\[\forall\vec{x} \in L_{1}+L_{2}, \exists\vec{x}_{1}\in L_{1}, \vec{x}_{2} \in L_{2}, \vec{x} = \vec{x}_{1} +\vec{x}_{2} .\]
Entonces tenemos que 
\[\lambda\left(\vec{x}\right) = \lambda\left(\vec{x}_{1}+\vec{x}_{2}\right) = \lambda\left(\vec{x}_{1}\right)+\lambda\left(\vec{x}_{2}\right) = \vec{0} + \vec{0} = \vec{0} .\]
Por tanto, $\displaystyle \lambda \in \left(L_{1}+L_{2}\right)^{\perp} $, por lo que 
\[L_{1}^{\perp} \cap L_{2}^{\perp} \subset \left(L_{1}+L_{2}\right)^{\perp} .\]
\item[(2.1)] Tenemos que $\displaystyle L_{1}\cap L_{2} \subset L_{1}, L_{2} $, por lo que 
	\[
	\begin{split}
	& L_{1}^{\perp}\subset \left(L_{1}\cap L_{2}\right)^{\perp} \in \mathcal{L}\left(V^{*}\right) \\
	& L_{2}^{\perp} \subset \left(L_{1}\cap L_{2}\right)^{\perp} \in \mathcal{L}\left(V^{*}\right).
	\end{split}
	\]
Entonces, 
\[L_{1}^{\perp} + L_{2}^{\perp} \subset \left(L_{1} \cap L_{2}\right)^{\perp} .\]
\item[(2.2)] Vamos a ver que $\displaystyle \left(L_{1}\cap L_{2}\right)^{\perp} \subset L_{1}^{\perp} + L_{2}^{\perp} $. Sea $\displaystyle \left\{ \vec{u}_{1}, \ldots, \vec{u}_{r}\right\}  $ base de $\displaystyle L_{1}\cap L_{2} $ y sean $\displaystyle \left\{ \vec{u}_{1}, \ldots, \vec{u}_{r+p}\right\}  $ y $\displaystyle \left\{ \vec{u}_{1}, \ldots, \vec{u}_{r}, \vec{w}_{r+1}, \ldots, \vec{w}_{r+q}\right\}  $ bases de $\displaystyle L_{1} $ y $\displaystyle L_{2} $, respectivamente. Sea 
	\[  \left\{ \vec{u}_{1}, \ldots, \vec{u}_{r+p}, \vec{w}_{r+1}, \ldots, \vec{w}_{r+q}, \vec{v}_{1}, \ldots, \vec{v}_{k}\right\}, \]
	base de $\displaystyle V $. Si $\displaystyle \lambda \in \left(L_{1}\cap L_{2}\right)^{\perp}$, tenemos que ver que existen $\displaystyle \lambda_{1} \in L_{1}^{\perp}, \lambda_{2}\in L_{2}^{\perp} $ tales que $\displaystyle \lambda = \lambda_{1} + \lambda_{2} $. Para $\displaystyle \forall i = 1, \ldots, r $ defino 
	\[\lambda_{1}\left(\vec{u}_{i}\right) = 0 \quad \text{y} \quad \lambda_{2}\left(\vec{u}_{i}\right) = 0 .\]
De esta manera, $\displaystyle \left(\lambda_{1}+\lambda_{2}\right)\left(\vec{u}_{i}\right) = 0 $. Para $\displaystyle i = r+1, \ldots, r+p $ y $\displaystyle \vec{u}_{i} \in L_{1} $ defino
\[\lambda\left(\vec{u}_{i}\right) \in \K, \;  \lambda_{1}\left(\vec{u}_{i}\right) = 0, \; \lambda_{2}\left(\vec{u}_{i}\right) = \lambda\left(\vec{u}_{i}\right) .\]
Similarmente, para $\displaystyle i = r+1, \ldots, r+q $, 
\[\lambda\left(\vec{w}_{i}\right) \in \K, \; \lambda_{1}\left(\vec{w}_{i}\right) = \lambda\left(\vec{w}_{i}\right), \; \lambda_{2}\left(\vec{w}_{i}\right) = 0 .\]
\end{description}
\end{proof}

\begin{ftheorem}[]
\normalfont Sea $\displaystyle L \in \mathcal{L}\left(V\right) $, entonces 
\[\dim L^{\perp} = \dim V - \dim L .\]
\end{ftheorem}

\begin{observation}
\normalfont Tenemos que 
\[\left(L^{\perp}\right)^{\perp} = \theta\left(L\right) .\]
\end{observation}

\begin{proof}
	Sea $\displaystyle \left\{\vec{u}_{1}, \ldots, \vec{u}_{r} \right\}  $ base de $\displaystyle L $, $\displaystyle \left\{ \vec{u}_{1}, \ldots, \vec{u}_{r}, \ldots, \vec{u}_{n}\right\}  $ base de $\displaystyle V $ y $\displaystyle \left\{ \omega^{1}, \ldots, \omega^{n}\right\}  $ su base dual. Entonces tenemos ya que $\displaystyle \left\{ \omega^{r+1}, \ldots, \omega^{n}\right\}  $ son linealmente independientes. Vamos a ver que son sistema de generadores, 
	\[\forall \lambda \in L^{\perp} \in \mathcal{L}\left(V\right), \exists a_{1}, \ldots, a_{n} \in \K, .\]
	\[\lambda = a_{1}\omega^{1} + \cdots + a_{n}\omega^{n} .\]
Tenemos que $\displaystyle \forall i = 1, \ldots, r $, 
\[0 = \lambda\left(\vec{u}_{i}\right) = a_{1}\omega^{1}\left(\vec{u}_{i}\right)+ \cdots + a_{i}\omega^{i}\left(\vec{u}_{i}\right) + \cdots + a_{r}\omega^{r}\left(\vec{u}_{i}\right) + \cdots + a_{n}\omega^{n}\left(\vec{u}_{i}\right) = a_{i} .\]
Así, 
\[\lambda = a_{r+1}\omega^{r+1} + \cdots + a_{n}\omega^{n} .\]
\end{proof}

\begin{observation}
\normalfont Tenemos que 
\[\dim L = \dim \theta\left(L\right) = \dim \left(L^{\perp}\right)^{\perp} .\]
\end{observation}

\begin{flema}[]
\normalfont Sea $\displaystyle L_{1}, L_{2} \in \mathcal{L}\left(V\right) $ tales que $\displaystyle V = L_{1}\oplus L_{2} $ y sea $\displaystyle f_{1} : L_{1} \to V' $ y $\displaystyle f_{2}: L_{2} \to V' $ aplicaciones lineales. Entonces, existe una única $\displaystyle f : V \to V' $ tal que 
\[f|_{L_{1}} = f_{1} \quad \text{y} \quad f|_{L_{2}} = f_{2} .\]
\footnote{ $\displaystyle f|_{L} $ es $\displaystyle f $ restringida por $\displaystyle L $, es decir, solo está definida para los vectores de $\displaystyle L $.} 
\end{flema}
\begin{proof}
Tenemos que $\displaystyle \forall\vec{x}\in V $, $\displaystyle \exists! \vec{x}_{1} \in L_{1}, \vec{x}_{2} \in L_{2} $ tales que $\displaystyle \vec{x} = \vec{x}_{1}+\vec{x}_{2} $. Queremos que $\displaystyle f\left(\vec{x}\right) = f\left(\vec{x}_{1}\right)+f\left(\vec{x}_{2}\right) $:
\[f\left(\vec{x}\right) = f\left(\vec{x}_{1}\right) + f\left(\vec{x}_{2}\right) = f|_{L_{1}}\left(\vec{x}_{1}\right) + f|_{L_{2}}\left(\vec{x}_{2}\right) = f_{1}\left(\vec{x}_{1}\right) + f_{2}\left(\vec{x}_{2}\right) .\]
Sea $\displaystyle f: V \to V' $ la aplicación definida de esta manera: 
\[f\left(\vec{x}_{1}+\vec{x}_{2}\right) = f_{1}\left(\vec{x}_{1}\right)+f_{2}\left(\vec{x}_{2}\right) .\]
Queremos ver que $\displaystyle f $ es lineal:
\[.\]
\[
\begin{split}
	\forall \vec{x}, \vec{y}\in V, \; f\left(\vec{x}+\vec{y}\right) & = f\left(\left(\vec{x}_{1}+\vec{x}_{2}\right)+\left(\vec{y}_{1}+\vec{y}_{2}\right)\right)\\
									&= f\left(\left(\vec{x}_{1}+\vec{y}_{1}\right)+\left(\vec{x}_{2}+\vec{y}_{2}\right)\right)\\
									&= f_{1}\left(\vec{x}_{1}+\vec{y}_{1}\right) +f_{2}\left(\vec{x}_{2}+\vec{y}_{2}\right)\\
									&= f_{1}\left(\vec{x}_{1}\right) + f_{2}\left(\vec{x}_{2}\right)+f_{1}\left(\vec{y}_{1}\right) + f_{2}\left(\vec{y}_{2}\right)\\
									&= f\left(\vec{x}_{1}+\vec{x}_{2}\right)+f\left(\vec{y}_{1}+\vec{y}_{2}\right) \\
									&= f\left(\vec{x}\right)+f\left(\vec{y}\right).
\end{split}
\]
Similarmente, sea $\displaystyle a \in \K $ y $\displaystyle \vec{x} \in V $, 
\[
\begin{split}
	f\left(a\vec{x}\right) = & f\left(a\left(\vec{x}_{1}+\vec{x}_{2}\right)\right) = f\left(a\vec{x}_{1}+a\vec{x}_{2}\right) 
	=  f_{1}\left(a\vec{x}_{1}\right)+f_{2}\left(a\vec{x}_{2}\right) \\
	= & a f_{1}\left(\vec{x}_{1}\right) + a f_{2}\left(\vec{x}_{2}\right) 
	=  a \left(f_{1}\left(\vec{x}_{1}\right)+f_{2}\left(\vec{x}_{2}\right)\right) \\
	= & a f\left(\vec{x}_{1}+\vec{x}_{2}\right)
	=  af\left(\vec{x}\right) .
\end{split}
\]
Si $\displaystyle \vec{x}_{1} \in L_{1} $, tenemos que 
\[f|_{L_{1}}\left(\vec{x}_{1}\right) = f\left(\vec{x}_{1}+\vec{0}\right) = f_{1}\left(\vec{x}_{1}\right)+f_{2}\left(\vec{0}\right) = f_{1}\left(\vec{x}_{1}\right) .\]
\end{proof}

\begin{ftheorem}[]
\normalfont Sea $\displaystyle f: V \to V' $ una aplicación lineal. Entonces,
\begin{description}
\item[(a)] $\displaystyle \Ker\left(f^{*}\right) = \Imagen\left(f\right)^{\perp } $.
\item[(b)] $\displaystyle \Imagen\left(f^{*}\right) = \Ker\left(f\right)^{\perp } $.
\item[(c)] $\displaystyle f $ inyectiva $\displaystyle \iff  $ $\displaystyle f^{*} $ sobreyectiva.
\item[(d)] $\displaystyle f $ sobreyectiva $\displaystyle \iff  $ $\displaystyle f^{*} $ inyectiva.
\item[(e)] $\displaystyle \dim \Imagen\left(f\right) = \dim \Imagen\left(f^{*}\right) $.
\end{description}
\end{ftheorem}

\begin{proof}
\begin{description}
\item[(a)] Tenemos que $\displaystyle \forall\lambda'\in\Ker\left(f^{*}\right) $, tenemos que 
	\[\forall\vec{x}\in V, \; \lambda' \circ f\left(\vec{x}\right) = \vec{0} .\]
Similarmente, 
\[f^{*}\left(\lambda'\right) = \lambda' \circ f = 0 .\]
En la otra dirección, 
\[\lambda' \in \Imagen\left(f\right)^{\perp } \Rightarrow \forall \vec{x} \in V, \lambda'\left(f\left(\vec{x}\right)\right) = f^{*}\left(\lambda'\right)\left(\vec{x}\right) = 0 .\]
Por tanto, $\displaystyle f^{*}\left(\lambda'\right) = 0 $ y $\displaystyle \lambda' \in \Ker\left(f^{*}\right) $.
\item[(b)] Si $\displaystyle \lambda \in \Imagen\left(f^{*}\right) $. Entonces, existe $\displaystyle \lambda'\in V'^{*} $ tal que $\displaystyle \lambda = f^{*}\left(\lambda'\right) = \lambda' \circ f $. Así, $\displaystyle \forall \vec{x} \in \Ker\left(f\right) $, $\displaystyle \lambda\left(\vec{x}\right) = \lambda'\circ f\left(\vec{x}\right) = \lambda'\left(\vec{0}\right) = 0 $. \\ 
En la otra dirección, si $\displaystyle \lambda \in \Ker\left(f\right)^{\perp } $, buscamos $\displaystyle \lambda'\in V'^{*} $ tal que $\displaystyle f^{*}\left(\lambda'\right) = \lambda' \circ f=\lambda $. Sea $\displaystyle L' \in \mathcal{L}\left(V'\right) $ tal que $\displaystyle \Imagen\left(f\right) \oplus L' = V' $ y buscamos $\displaystyle \lambda'|_{L'} = 0 $. Si $\displaystyle \vec{x'} \in \Imagen\left(f\right) $, existe un $\displaystyle \vec{x}\in V $ tal que $\displaystyle f\left(\vec{x}\right) = \vec{x'} $. Defino $\displaystyle \lambda'\left(\vec{x'}\right) = f\left(\vec{x}\right) $. Como $\displaystyle \lambda \in \Ker\left(f\right)^{\perp } $ \footnote{Ortogonal del núcleo, que no el núcleo del ortogonal.}, tenemos que si $\displaystyle f\left(\vec{x}\right) = f\left(\vec{y}\right) $, 
\[f\left(\vec{x}\right)-f\left(\vec{y}\right) = f\left(\vec{x}-\vec{y}\right) = \vec{0} \Rightarrow \vec{x}-\vec{y} \in \Ker\left(f\right) \subset \Ker\left(\lambda\right) .\]
Asi, definimos que $\displaystyle \lambda'\left(\vec{x'}\right) = \vec{x} $ con $\displaystyle \vec{x} \in V $ tal que $\displaystyle f\left(\vec{x}\right) = \vec{x'} $.
\item[(c)] Tenemos que $\displaystyle f $ es inyectiva si y solo si $\displaystyle \Ker\left(f\right) = \left\{ \vec{0}\right\}  $. Esto es equivalente a que $\displaystyle \Ker\left(f\right)^{\perp } = \Imagen\left(f^{*}\right) = V^{*} $, que es equivalente a que $\displaystyle f^{*} $ es sobreyectiva. 
\item[(d)] Tenemos que $\displaystyle f $ es sobreyectiva si y solo si $\displaystyle \Imagen\left(f\right) = V' $. Esto es equivalente a decir que $\displaystyle \Imagen\left(f\right)^{\perp } = \Ker\left(f^{*}\right)= \left\{ \vec{0}\right\}  $, que es equivalente a decir que $\displaystyle f^{*} $ sea inyectiva.
\item[(e)] 
	\[\dim \Imagen\left(f\right) = \dim V'- \dim\Imagen \left(f^{\perp}\right) = \dim V' - \dim \Ker\left(f^{*}\right) = \dim V'^{*} - \dim \Ker\left(f^{*}\right) = \dim \Imagen\left(f^{*}\right) .\]
\end{description}
\end{proof}

\begin{fdefinition}[]
\normalfont Si $\displaystyle f: V \to V' $ es lineal llamaremos $\displaystyle \ran\left(f\right) $ a $\displaystyle \dim \Imagen\left(f\right) $.
\end{fdefinition}

\subsection{Matriz de $\displaystyle f^{*} $ }
\begin{ftheorem}[]
\normalfont 
\[\mathcal{M}_{ \left\{ \sigma^{j}\right\} \left\{ \omega^{i}\right\} }\left(f^{*}\right) = \left(\mathcal{M}_{ \left\{ \vec{u}_{i}\right\} \left\{ \vec{v}_{j}\right\} }\left(f\right)\right)^{t} .\]
\end{ftheorem}
\begin{proof}
	Sea $\displaystyle f : V \to V' $ lineal y $\displaystyle \left\{ \vec{u}_{1}, \ldots, \vec{u}_{n}\right\}  $ una base de $\displaystyle V $ y $\displaystyle \left\{ \vec{v}_{1}, \ldots, \vec{v}_{m}\right\}  $ una base de $\displaystyle V' $. Entonces, tenemos que $\displaystyle \mathcal{M}_{ \left\{ \vec{u}_{i}\right\} \left\{ \vec{v}_{j}\right\} }\left(f\right) = A \in \mathcal{M}_{m \times n}\left(\K\right) $ es la matriz asociada a $\displaystyle f $. Sea 
	\[A = \begin{pmatrix} a^{1}_{1} & \cdots & a^{1}_{n} \\
	\vdots & & \vdots \\
a^{m}_{1} & \cdots & a^{m}_{n}\end{pmatrix} .\]
Así, la dual de $\displaystyle f $ está definida de la siguiente manera
\[
\begin{split}
	f^{*} : V'^{*} & \to V^{*} \\
	\lambda & \to \lambda \circ f.
\end{split}
\]
Así, si $\displaystyle \left\{ \omega^{1}, \ldots, \omega^{n}\right\}  $ es la base dual de $\displaystyle \left\{ \vec{u}_{1}, \ldots, \vec{u}_{n}\right\}  $ y $\displaystyle \left\{ \sigma^{1}, \ldots, \sigma^{m}\right\} $ es la base dual de $\displaystyle \left\{ \vec{v}_{1}, \ldots, \vec{v}_{m}\right\}  $, entonces $\displaystyle \mathcal{M}_{ \left\{ \vec{\sigma}^{i}\right\} \left\{ \vec{\omega}^{j}\right\} }\left(f^{*}\right) = B \in \mathcal{M}_{n \times m}\left(\K\right) $ es la matriz que corresponde a $\displaystyle f^{*} $. Las columnas de $\displaystyle B $ son las coordenadas de $\displaystyle f^{*}\left(\sigma^{j}\right) $ en la base $\displaystyle \left\{ \omega^{1}, \ldots, \omega^{n}\right\}  $. Así,
\[f^{*}\left(\sigma^{j}\right)\left(\vec{u}_{i}\right) = \sigma^{j}\left(f\left(\vec{u}_{i}\right)\right) = \sigma^{j}\left(a^{1}_{i}\vec{v}_{i} + \cdots + a^{j}_{i}\vec{v}_{j} + \cdots + a^{m}_{i}\vec{v}_{m}\right) = a^{j}_{i} .\]
Así, $\displaystyle B = A^{t} $.
\end{proof}

