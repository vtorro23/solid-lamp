\chapter{Reducción de Endomorfismos}
Sea $\displaystyle f : V \to V' $ lineal. Si $\displaystyle \left\{ \vec{u}_{1}, \ldots, \vec{u}_{n}\right\}  $ y $\displaystyle \left\{ \vec{v}_{1}, \ldots, \vec{v}_{m}\right\}  $ son bases de $\displaystyle V $ y $\displaystyle V' $, respectivamente, tenemos que $\displaystyle A =\mathcal{M}_{ \left\{ \vec{u}_{i}\right\} \left\{ \vec{v}_{j}\right\} }\left(f\right) \in \mathcal{M}_{m \times n} \left(\K\right) $. Si $\displaystyle \left\{ \vec{u'}_{1}, \ldots, \vec{u'}_{n}\right\}  $, $\displaystyle \left\{ \vec{v'}_{1}, \ldots, \vec{v'}_{m}\right\}  $ son bases de $\displaystyle V $ y $\displaystyle V' $, respectivamente, con $\displaystyle B =\mathcal{M}_{ \left\{ \vec{u'}_{i}\right\} \left\{ \vec{v'}_{j}\right\} }\left(f\right) $, tenemos que
\[
\begin{split}
	& \begin{pmatrix} \vec{u'}_{1} & \ldots & \vec{u'}_{n} \end{pmatrix} = \begin{pmatrix} \vec{u}_{1} & \ldots & \vec{u}_{n} \end{pmatrix} C, \; C \in \GL\left(n, \K\right) \\
	& \begin{pmatrix} \vec{v'}_{1} & \ldots & \vec{v'}_{m} \end{pmatrix} = \begin{pmatrix} \vec{v}_{1} & \ldots & \vec{v}_{m} \end{pmatrix} D, \; D \in \GL\left(m, \K\right).
\end{split}
\]
Entonces, tenemos el siguiente diagrama. 
\begin{figure}
\centering
\includegraphics[width=0.5\linewidth]{~/Desktop/Images/diagonalizacion1.png}
\caption{Semejanza de matrices}
\label{ }
\end{figure}
De aquí se deduce que $\displaystyle f = id _{V'} \circ f \circ id _{V} $, que es lo mismo que
\[B = D^{-1}AC .\]
\begin{fdefinition}[]
	\normalfont Dos matrices $\displaystyle A,B \in \mathcal{M}_{m \times n}\left(\K\right) $ son equivalentes si existe $\displaystyle C \in \GL\left(n, \K\right) $, $\displaystyle D \in \GL\left(m, \K\right) $ tales que $\displaystyle D^{-1}AC = B $.
\end{fdefinition}

\begin{observation}
\normalfont Esta es una relación de equivalencia en el cojunto de matrices $\displaystyle \mathcal{M}_{m \times n}\left(\K\right) $. En concreto las matrices equivalentes tienen el mismo rango.
\end{observation}

\begin{observation}
\normalfont Revisar el último ejercicio de aplicaciones lineales.
\end{observation}

\begin{observation}
\normalfont Si $\displaystyle f \in \End\left(V\right) $, tenemos que $\displaystyle C = D $, por lo que $\displaystyle B = C^{-1}AC $.
\end{observation}

\begin{fdefinition}[]
\normalfont Dos matrices $\displaystyle A, B \in \mathcal{M}_{n \times n}\left(\K\right) $ son semejantes si existe $\displaystyle C \in \GL\left(n, \K\right) $ tal que $\displaystyle B = C^{-1}AC $.
\end{fdefinition}

\begin{fdefinition}[]
	\normalfont El vector $\displaystyle \vec{x} \in V $ es un \textbf{vector propio} de $\displaystyle f \in \End\left(V\right) $ si existe $\displaystyle \lambda \in \K $ tal que $\displaystyle f\left(\vec{x}\right) = \lambda \vec{x} $. Similarmente, se dice que $\displaystyle \lambda \in \K $ es un \textbf{valor propio} de $\displaystyle f $ si $\displaystyle \exists \vec{x} \in V/ \left\{ \vec{0}\right\}  $ tal que $\displaystyle f\left(\vec{x}\right) = \lambda \vec{x} $.
\end{fdefinition}

Si $\displaystyle \left\{ \vec{x}_{1}, \ldots, \vec{x}_{n}\right\}  $ es base de $\displaystyle V $ formada por vectores propios de $\displaystyle f $
\[f\left(\vec{x}_{i}\right) = \lambda_{i}\vec{x}_{i} .\]

\begin{eg}
\normalfont No siempre existen los valores propios. Por ejemplo, consideremos la aplicación dada por la matriz
\[\begin{pmatrix} 0 & -1 \\ 1 & 0 \end{pmatrix}\begin{pmatrix} x \\ y \end{pmatrix}  = \lambda\begin{pmatrix} x \\ y \end{pmatrix}.\]
A partir de un sistema de ecuaciones obtenemos que $\displaystyle \lambda^{2} = -1 $. Si $\displaystyle \K = \R $, esta aplicación no tendría valores propios. Si $\displaystyle \K = \C $, sí que los tendría.
\end{eg}

\section{Polinomios}

\begin{fdefinition}[]
\normalfont Una \textbf{sucesión} definida en $\displaystyle \K $ es una aplicación $\displaystyle a\left(x\right) : \N \to \K $ tal que $\displaystyle n \to a_{n} $. Diremos que $\displaystyle a\left(x\right) = \left(a_{0}, a_{1}, \ldots, a_{n}, \ldots\right) $. Un \textbf{polinomio} con coeficientes en $\displaystyle \K $ es una sucesión $\displaystyle a\left(x\right) = \left(a_{0}, \ldots, a_{n}, \ldots\right) $ tal que existe $\displaystyle a_{m} \neq 0 $ y $\displaystyle a_{k} = 0, \; \forall k > m $. Diremos que $\displaystyle m $ es el \textbf{grado} del polinomio $\displaystyle a\left(x\right) $.
\end{fdefinition}

\begin{observation}
\normalfont Por esta definición, el polinomio $\displaystyle 0 = \left(0, \ldots, 0, \ldots\right) $ no tiene grado.
\end{observation}
Sea $\displaystyle \K[x] = \left\{ a\left(x\right) \; : \; a\left(x\right) \; \text{polinomio con coeficientes en } \; \K\right\}  $. Así, definimos la suma de polinomios
\[a\left(x\right) + b\left(x\right) = \left(a_{0}+b_{0}, a_{1}+b_{1}, \ldots, a_{n}+b_{n}, \ldots\right) .\]
\begin{observation}
	\normalfont Tenemos que con esta suma, $\displaystyle \K[x] $ es un grupo abeliano. 
\end{observation}
Ahora definimos el producto de polinomios:
\[a\left(x\right) \cdot b\left(x\right) = \left(\underbrace{a_{0}b_{0}}_{c_{0}}, \underbrace{a_{1}b_{0} + a_{0}b_{1}}_{c_{1}}, \ldots, c_{n} = \sum_{i + j = n}a_{i}b_{j}, \ldots\right) .\]
\begin{observation}
	\normalfont Tenemos que con este producto, $\displaystyle \left( \K[x], + , \cdot\right) $ es un anillo conmutativo con unidad. La unidad será $\displaystyle \left(1, 0, \ldots, 0, \ldots\right) $.  \end{observation}
\begin{observation}
\normalfont Se deduce fácilmente que
\begin{itemize}
\item  $\displaystyle \grad\left(a\left(x\right) + b\left(x\right)\right) \leq \max\left(\grad\left(a\left(x\right)\right), \grad\left(b\left(x\right)\right)\right) $ 
\item $\displaystyle \grad\left(a\left(x\right) \cdot b\left(x\right)\right) = \grad\left(a\left(x\right)\right) + \grad\left(b\left(x\right)\right) $.
\end{itemize}
\end{observation}

\begin{fprop}[]
\normalfont A partir de la segunda igualdad tenemos que
\begin{description}
\item[(a)] Si $\displaystyle a\left(x\right) \cdot b\left(x\right) = 0 $, $\displaystyle a\left(x\right) = 0 $ o $\displaystyle b\left(x\right)=0 $.
\item[(b)] Si $\displaystyle a\left(x\right) \cdot b\left(x\right) = a\left(x\right) \cdot c\left(x\right) \neq 0 $, entonces $\displaystyle b\left(x\right) = c\left(x\right) $.
\item[(c)] Los únicos elementos invertibles de $\displaystyle \K[x] $ son los de grado 0.
\end{description}
\end{fprop}
Podemos definir el producto por escalares:
\[
\begin{split}
	\cdot : \K \times \K[x] & \to \K[x] \\
	\left(a, \left(a_{0}, \ldots, a_{n}, \ldots\right)\right) & \to \left(a a_{0}, \ldots, a a_{n}, \ldots\right).
\end{split}
\]
\begin{observation}
\normalfont 
Así, $\displaystyle \K[x] $ es un $\displaystyle \K $-espacio vectorial. 
\end{observation}
Esto nos permite escribir
\[
\begin{split}
	a\left(x\right) = & \left(a_{0}, a_{1}, \ldots, a_{n}, \ldots\right) \\
	= & a_{0}\left(1, 0, \ldots, 0, \ldots\right) + a_{1} \left(0, 1, \ldots, 0, \ldots\right) + \cdots + a_{n}\left(0, \ldots, 1, \ldots\right) + \cdots .
\end{split}
\]
Así, definimos
\[
\begin{split}
& x = \left(0, 1, \ldots, 0, \ldots\right) \\
& x^{2} = \left(0, 0, 1, \ldots, 0, \ldots\right) \\
& x^{3} = \left(0, 1, \ldots, 0, \ldots\right) \cdot  \left(0, 0, 1, \ldots, 0, \ldots\right) = \left(0, 0, 0, 1, \ldots, 0, \ldots\right)
\end{split}
\]
Supongamos que $\displaystyle x^{k -1} = \left(0, 0, \ldots, 1, \ldots , 0, \ldots\right) $. Así, para $\displaystyle x^{k} $,
\[x^{k} = \left(0, 1, \ldots, 0, \ldots\right) \cdot \left(0, 0, \ldots, 1, \ldots, 0, \ldots\right) .\]
Así, tenemos que si $\displaystyle a\left(x\right) \in \K[x] $,
\[a\left(x\right) = \left(a_{0},0, \ldots, 0, \ldots\right) + a_{1}x + a_{2}x^{2} + \cdots + a_{n}x^{n} .\]
Si a cada $\displaystyle a \in \K $ le hacemos corresponder el polinomio $\displaystyle \left(a, 0, \ldots, 0, \ldots\right) $, obtenemos una aplicación inyectiva $\displaystyle \K \to \K[x] $ que conserva las operaciones anteriores. Así, podemos sustituir $\displaystyle \left(a_{0}, 0, \ldots, 0, \ldots\right) $ por $\displaystyle a_{0} $. 
\begin{ftheorem}[]
	\normalfont Sean $\displaystyle a\left(x\right), b\left(x\right) \in \K[x]/ \left\{ 0\right\}  $, entonces $\displaystyle \exists!p\left(x\right), r\left(x\right) \in \K[x] $ tales que $\displaystyle a\left(x\right) = b\left(x\right)p\left(x\right)+r\left(x\right) $. Además, $\displaystyle \grad\left(r\left(x\right)\right) < \grad\left(b\left(x\right)\right) $ o $\displaystyle r\left(x\right) = 0 $. 
\end{ftheorem}
\begin{proof}
Si $\displaystyle \grad\left(a\left(x\right)\right) < \grad\left(b\left(x\right)\right) $ tomamos $\displaystyle p\left(x\right) = 0 $ y $\displaystyle r\left(x\right) = a\left(x\right) $. Supongamos que $\displaystyle \grad\left(a\left(x\right)\right) \geq \grad\left(b\left(x\right)\right) $, entonces
\[
\begin{split}
& a\left(x\right) = a_{0} + a_{1}x + \cdots + a_{n}x^{n}, \; a_{n} \neq 0\\
& b\left(x\right) = b_{0} + b_{1}x + \cdots + b_{m}x^{m}, \; b_{m} \neq 0.
\end{split}
\]
Tenemos que 
\[a\left(x\right)-b\left(x\right)\frac{a_{n}}{b_{m}}\left(x^{n-m}\right) = c_{0} + c_{1}x + \cdots + c_{n_{1}}x^{n_{1}}, \; n_{1}\geq n .\]
Si $\displaystyle n_{1} < m $,
\[a\left(x\right) = b\left(x\right) \underbrace{\left(\frac{a_{m}}{b_{m}}\left(x^{m -n}\right)\right)}_{p\left(x\right)} + \underbrace{c\left(x\right)}_{r\left(x\right)} .\]
Si $\displaystyle n_{1} \geq m $, 
\[ c\left(x\right) - \frac{c_{n_{1}}}{b_{m}}\left(x^{n_{1}-m}\right) = d _{0} + d _{1}x + \cdots d _{n_{2}}x^{n_{2}}, \; n_{2} < n_{1} < n .\]
Si $\displaystyle n_{2} < m $,
\[a\left(x\right) = b\left(x\right)\left(\frac{a_{n}}{b_{m}}x^{n - m} + \frac{c_{n_{1}}}{b_{m}}x^{n_{1}-m}\right) + d _{n_{2}}x^{n_{2}} .\]
Si $\displaystyle n_{2} \geq m $, $\displaystyle n_{2} < n_{1} < n $ y repetimos el paso anterior. Después de un número finito de pasos, obtendremos un polinomio resto que tenga grado 0 o cuyo grado sea menor que $\displaystyle m $. Ahora demostramos la unicidad, $\displaystyle a\left(x\right) = b\left(x\right) p\left(x\right) + r\left(x\right) $ con $\displaystyle \grad\left(r\left(x\right)\right) < \grad\left(b\left(x\right)\right) $ o $\displaystyle r\left(x\right) = 0 $ y $\displaystyle a\left(x\right) = b\left(x\right) p'\left(x\right) + r'\left(x\right) $ con $\displaystyle \grad\left(r'\left(x\right)\right) < \grad\left(b\left(x\right)\right) $ o $\displaystyle r'\left(x\right) = 0 $. Tenemos que
\[b\left(x\right) \left(p\left(x\right)-p'\left(x\right)\right) = r'\left(x\right) -r\left(x\right) .\]
Si $\displaystyle r\left(x\right) \neq r'\left(x\right) $, entonces $\displaystyle \grad\left(r\left(x\right)-r'\left(x\right)\right) < \grad\left(b\left(x\right)\right) $. Así, $\displaystyle \grad\left(b\left(x\right)\left(p\left(x\right)-p'\left(x\right)\right)\right) < \grad\left(b\left(x\right)\right) $. Sin embargo, tenemos que $\displaystyle \grad\left(b\left(x\right)\left(p\left(x\right)-p'\left(x\right)\right)\right) \geq \grad\left(b\left(x\right)\right) $. Esto es una contradicción, por lo que debe ser que $\displaystyle p\left(x\right) = p'\left(x\right) $ y $\displaystyle r\left(x\right) = r'\left(x\right) $.
\end{proof}

\begin{fdefinition}[]
\normalfont Decimos que $\displaystyle p\left(x\right) $ es el \textbf{cociente} de la división de $\displaystyle a\left(x\right) $ entre $\displaystyle b\left(x\right) $ y $\displaystyle r\left(x\right) $ es el \textbf{resto}.
\end{fdefinition}

\begin{fdefinition}[]
\normalfont Si $\displaystyle r\left(x\right) = 0 $, decimos que $\displaystyle a\left(x\right) $ es múltiplo de $\displaystyle b\left(x\right) $ o que $\displaystyle b\left(x\right) $ divide a $\displaystyle a\left(x\right) $. Esto se escribe $\displaystyle b\left(x\right) | a\left(x\right) $.
\end{fdefinition}

\begin{fdefinition}[Ideal]
	\normalfont Un \textbf{ideal} $\displaystyle I $ de $\displaystyle \K[x] $ es un conjunto $\displaystyle I \neq \emptyset $ y $\displaystyle I \subset \K[x] $, que verifica que
	\begin{description}
	\item[(a)] Si $\displaystyle a\left(x\right), b\left(x\right) \in I $, $\displaystyle a\left(x\right)+b\left(x\right)\in I $.
	\item[(b)] Si $\displaystyle a\left(x\right) \in I $ y $\displaystyle p\left(x\right) \in \K[x] $, tenemos que $\displaystyle p\left(x\right)a\left(x\right) \in I $.
	\end{description}
\end{fdefinition}

\begin{observation}
\normalfont 
Dado $\displaystyle b\left(x\right) \in \K[x] $ y sea $\displaystyle \left(b\left(x\right)\right) = \left\{ p\left(x\right)b\left(x\right)\; : \; p\left(x\right) \in \K[x]\right\}  $. Tenemos que este conjunto es un ideal. 
\end{observation}
\begin{fprop}[]
	\normalfont Sea $\displaystyle I \subset \K[x] $ un ideal. Entonces, $\displaystyle \exists! b\left(x\right) \in I $ mónico, tal que $\displaystyle I = \left(b\left(x\right)\right) $.
\end{fprop}
\begin{proof}
	Si $\displaystyle I = \left\{ 0\right\}  $, entonces $\displaystyle \left(0\right) = I $. Si $\displaystyle I \neq \left\{ 0\right\}  $, sea $\displaystyle b\left(x\right) \in I $ un polinomio de menor grado entre los polinomios de $\displaystyle I $. Tenemos que $\displaystyle \left(b\left(x\right)\right) \subset I $ por las propiedades del ideal. Si $\displaystyle a\left(x\right) \in I $, tenemos que existen $\displaystyle p\left(x\right), r\left(x\right) \in \K[x] $ tales que
	\[a\left(x\right) = b\left(x\right) p\left(x\right) + r\left(x\right) .\]
Entonces, $\displaystyle r\left(x\right) = a\left(x\right)-b\left(x\right)p\left(x\right) \in I $. Como $\displaystyle \grad\left(r\left(x\right)\right) < \grad\left(b\left(x\right)\right) $ y $\displaystyle r\left(x\right) \in I $, tenemos que $\displaystyle r\left(x\right) = 0 $. Así, tenemos que $\displaystyle I \subset \left(b\left(x\right)\right) $, por lo que tenemos que $\displaystyle I = \left(b\left(x\right)\right) $.	
\end{proof}

\begin{fdefinition}[]
	\normalfont Se dice que $\displaystyle p\left(x\right) \in \K[x] $ es \textbf{mónico} si su coeficiente de mayor grado es 1.
\end{fdefinition}

\begin{observation}
\normalfont 
\begin{description}
	\item[(i)] $\displaystyle \forall k \in \K/ \left\{ 0\right\}  $, $\displaystyle \left(b\left(x\right)\right) = \left(k b\left(x\right)\right) $.
	\item[(ii)]  $\displaystyle a\left(x\right) \in \left(a\left(x\right)\right) \subset \left(b\left(x\right)\right) $  $\displaystyle \iff $ $\displaystyle b\left(x\right) | a\left(x\right) $.
	\item[(iii)] Si $\displaystyle \left(a\left(x\right)\right) = \left(b\left(x\right)\right) $, entonces $\displaystyle a\left(x\right) = kb\left(x\right) $ con $\displaystyle k \in \K $.
\end{description}
\end{observation}

\begin{fprop}[]
\normalfont Sean $\displaystyle I_{1}, \ldots, I_{i} $ ideales de $\displaystyle \K[x] $ donde $\displaystyle i \in X $, entonces $\displaystyle \bigcap_{i \in X}I_{i}  $ también es ideal.
\end{fprop}

\begin{proof}
Si $\displaystyle a\left(x\right),b\left(x\right) \in \bigcap_{i \in X}I_{i} $, entonces $\displaystyle \forall i \in X $, $\displaystyle a\left(x\right), b\left(x\right) \in I_{i} $. Así, $ \displaystyle \forall i \in X, \; a\left(x\right) + b\left(x\right) \in I_{i} $, de esta manera $ \displaystyle a\left(x\right) + b\left(x\right) \in \bigcap_{i \in X}I_{i}$
Similarmente, si $\displaystyle p\left(x\right)\in \K[x] $ y $\displaystyle a\left(x\right) \in \bigcap_{i \in X}I_{i} $, tenemos que $\displaystyle \forall i \in X, \; p\left(x\right)a\left(x\right) \in I_{i} $. Así, $\displaystyle p\left(x\right)a\left(x\right) \in \bigcap_{i \in X}I_{i} $.
\end{proof}
Sean $\displaystyle a_{1}\left(x\right), \ldots, a_{p}\left(x\right) \in \K[x]$. Entonces tenemos que 
\[\left(a_{1}\left(x\right)\right) \cap \left(a_{2}\left(x\right)\right) \cap\cdots \cap \left(a_{p}\left(x\right)\right) ,\]
es un ideal. Por tanto, $\displaystyle \exists!\left(m\left(x\right)\right) $ mónico tal que $\displaystyle \left(a_{1}\left(x\right)\right) \cap \left(a_{2}\left(x\right)\right) \cap\cdots \cap \left(a_{p}\left(x\right)\right) = \left(m\left(x\right)\right) $. 
\begin{fdefinition}[Mínimo común múltiplo]
\normalfont Decimos que $\displaystyle m\left(x\right) $ es \textbf{mínimo común múltiplo} de $\displaystyle a_{1}\left(x\right), \ldots, a_{p}\left(x\right) $.
\end{fdefinition}
Dados $\displaystyle a_{1}\left(x\right), \ldots, a_{q} \left(x\right) \in \K[x]$, sea 
\[I = \left\{ p_{1}\left(x\right)a_{1}\left(x\right) + \cdots + p_{q}\left(x\right)a_{q}\left(x\right) \; : \; p_{1}\left(x\right), \ldots, p_{q}\left(x\right) \in \K[x]\right\}  .\]
Tenemos que $\displaystyle I $ es un ideal de $\displaystyle \K[x] $. Así, $\displaystyle \exists !d\left(x\right) \in \K[x] $ mónico tal que $\displaystyle I = \left(d\left(x\right)\right) $. Así, $\displaystyle \left(a_{1}\left(x\right)\right), \ldots, \left(a_{q}\left(x\right)\right) \subset \left(d\left(x\right)\right) $. De esta manera tenemos que $\displaystyle \forall i = 1, \ldots , q $, $\displaystyle d\left(x\right) | a_{i}\left(x\right) $. 
Si $\displaystyle q\left(x\right) \in \K[x] $ tal que $\displaystyle \forall i = 1, \ldots, q $, $\displaystyle q\left(x\right) | a_{i}\left(x\right) $, tenemos que $\displaystyle q\left(x\right) | d\left(x\right) $. 
\begin{fdefinition}[Máximo común divisor]
\normalfont Se dice que $\displaystyle d\left(x\right) $ es el \textbf{máximo común divisor}.
\end{fdefinition}
\begin{fdefinition}[]
	\normalfont Si $\displaystyle a\left(x\right), b\left(x\right) \in \K[x] $ son \textbf{primos entre sí} si su máximo común divisor es 1. 
\end{fdefinition}
\begin{ftheorem}[]
\normalfont Si $\displaystyle \mcd\left(a\left(x\right), b\left(x\right)\right) = 1 $ y $\displaystyle a\left(x\right) | c\left(x\right)b\left(x\right) $, entonces $\displaystyle a\left(x\right) | c\left(x\right) $.
\end{ftheorem}
\begin{proof}
	Tenemos que $\displaystyle \exists p\left(x\right), q\left(x\right) \in \K[x] $ tales que $\displaystyle 1 = p\left(x\right)a\left(x\right) + q\left(x\right)b\left(x\right) $. Así, 
	\[c\left(x\right) = c\left(x\right)p\left(x\right)a\left(x\right) + c\left(x\right)q\left(x\right)b\left(x\right) \subset \left(a\left(x\right)\right).\]
\end{proof}
\begin{fdefinition}[]
	\normalfont Un polinomio $\displaystyle a\left(x\right) \in \K\left[x\right] $ es \textbf{irreducible} si sus únicos divisores sean $\displaystyle k $ o $\displaystyle ka\left(x\right) $, donde $\displaystyle k \in \K/ \left\{ 0\right\}  $.
\end{fdefinition}
\begin{ftheorem}[]
\normalfont Todo polinomio de grado mayor o igual que 1 es producto de polinomios irreducibles.
\end{ftheorem}
\begin{proof}
	Sea $\displaystyle a\left(x\right) \in \K[x] $. Si $\displaystyle a\left(x\right) $ es irreducible, tenemos que $\displaystyle a\left(x\right) = a\left(x\right) \cdot 1 $. Si $\displaystyle a\left(x\right) $ no es irreducible, tenemos que $\displaystyle \exists p_{1}\left(x\right) \in \K[x] $ tal que $\displaystyle p_{1}\left(x\right) | a\left(x\right) $ de grado mínimo entre los divisores de $\displaystyle a\left(x\right) $. Así, $\displaystyle a\left(x\right) = p_{1}\left(x\right)a_{1}\left(x\right) $. Tenemos que $\displaystyle p_{1}\left(x\right) $ es irreducible. Si $\displaystyle a_{1}\left(x\right) $ es irreducible, hemos ganado.
	Si $\displaystyle a_{1}\left(x\right) $ no es irreducible, tenemos que $\displaystyle \exists p_{2}\left(x\right) \in \K[x] $ tal que $\displaystyle p_{2}\left(x\right) | a_{1}\left(x\right) $ de grado mínimo entre los polinomios que dividen $\displaystyle a_{1}\left(x\right) $. Tenemos que $\displaystyle p_{2}\left(x\right) $ es irreducible. Así, $\displaystyle a_{1}\left(x\right) = p_{2}\left(x\right)a_{2}\left(x\right) $, así $\displaystyle a\left(x\right) = p_{1}\left(x\right)p_{2}\left(x\right)a_{2}\left(x\right) $.
Después de $\displaystyle n $ etapas, tenemos que $\displaystyle a\left(x\right) = p_{1}\left(x\right)p_{2}\left(x\right) \cdots p_{n}\left(x\right)a_{n}\left(x\right) $, donde $\displaystyle a_{n}\left(x\right) $ será irreducible.
\end{proof}
\begin{fprop}[]
\normalfont Sea $\displaystyle p\left(x\right) = p_{1}\left(x\right) \cdots p_{n}\left(x\right) = q_{1}\left(x\right) \cdots q_{m}\left(x\right)$, donde $\displaystyle p_{i}\left(x\right), q_{j}\left(x\right) $ son irreducibles. Entonces, tenemos que $\displaystyle n = m $ y los polinomios $\displaystyle p_{i}\left(x\right), q_{j}\left(x\right) $ coinciden salvo en orden y factores escalares.
\end{fprop}
\begin{proof}
Si $\displaystyle n = 1 $, tenemos que $\displaystyle p_{1} = q_{1}\left(x\right) q_{2}\left(x\right) \cdots q_{m}\left(x\right) $. Entonces, si $\displaystyle m \neq 1 $, tenemos que $\displaystyle p_{1}\left(x\right) $ no es irreducible, lo cual es una contradicción. Por tanto, debe ser que $\displaystyle m = 1 $ y $\displaystyle p_{1}\left(x\right) = q_{1}\left(x\right) $. Ahora, asumimos que es cierto para $\displaystyle n - 1 $. Tenemos que
\[p_{1}\left(x\right) \cdots p_{n}\left(x\right) = q_{1}\left(x\right) \cdots q_{m}\left(x\right) .\]
Así, $\displaystyle p_{n}\left(x\right) | q_{1}\left(x\right) \cdots q_{m}\left(x\right) $. Así, tenemos que $\displaystyle p_{n}\left(x\right) | q_{1}\left(x\right) $ o $\displaystyle p_{n}\left(x\right) | q_{2}\left(x\right) \cdots q_{m}\left(x\right) $. 
\begin{itemize}
	\item En el primer caso, tenemos que $\displaystyle \exists k \in \K/ \left\{ 0\right\}  $ tal que $\displaystyle p_{n}\left(x\right) = kq_{1}\left(x\right) $. Así, $\displaystyle p_{1}\left(x\right) \cdots p_{n-1}\left(x\right) = \frac{1}{k}\left(q_{1}\left(x\right) \cdots q_{m -1}\left(x\right)\right) $. Por hipótesis, tenemos que $\displaystyle n - 1 = m - 1 $ y $\displaystyle p_{i}\left(x\right) = q_{j}\left(x\right) $ salvo escalares no nulos.
	\item En el segundo caso, tenemos que $\displaystyle p_{n}\left(x\right) | q_{2} \cdots q_{m} $, donde podemos iterar el paso anterior.
\end{itemize}
Llegaremos a un $\displaystyle q_{j}\left(x\right) $ tal que $\displaystyle p_{n}\left(x\right) = kq_{j}\left(x\right) $. Así, solo nos quedan $\displaystyle n - 1 $ factores, que podemos reducir a la hipótesis de inducción, y se concluye que $\displaystyle n - 1 = m - 1 $, por lo que $\displaystyle n = m $. 
\end{proof}
Si $\displaystyle p\left(x\right) \in \K[x] $, $\displaystyle p\left(x\right) = a_{0} + a_{1}x + \cdots + a_{n}x^{n} $, definimos la aplicación 
\[
\begin{split}
	p : \K &\to \K \\
	k &\to p\left(k\right) = a_{0} + a_{1}k + a_{2}k^{2} + \cdots + a_{n}k^{n}.
\end{split}
\]
\begin{fdefinition}[]
\normalfont Se dice que $\displaystyle k \in \K $ es una \textbf{raíz} de $\displaystyle p\left(x\right) $ si $\displaystyle p\left(k\right) = 0 $.
\end{fdefinition}
\begin{fprop}[]
	\normalfont Sea $\displaystyle p\left(x\right) \in \K[x] $ de grado mayor o igual que 1, entonces $\displaystyle k \in \K $ es una raíz de $\displaystyle p\left(x\right) $ sí y sólo si $\displaystyle \left(x-k\right) | p\left(x\right) $.
\end{fprop}
\begin{proof}
\begin{description}
\item[(i)] Sean $\displaystyle p\left(x\right) = q\left(x\right)\left(x-k\right) + r\left(x\right) $, donde $\displaystyle \grad\left(r\left(x\right)\right) < 1 $ o $\displaystyle r\left(x\right) = 0 $. Así, tenemos que si $\displaystyle p\left(k\right) = 0 = q\left(k\right)\left(k-k\right) + r\left(x\right) $, entonces $\displaystyle \left(x-k\right) | p\left(x\right) $. 
\item[(ii)] Recíprocamente, si $\displaystyle p\left(x\right) = q\left(x\right)\left(x-k\right) $, tenemos que $\displaystyle p\left(k\right) = 0 $. 
\end{description}
\end{proof}
\begin{fdefinition}[]
	\normalfont Sea $\displaystyle k $ una raíz de $\displaystyle p\left(x\right) \in \K[x] $. Diremos que $\displaystyle k $ es una raíz de $\displaystyle p\left(x\right) $ de orden $\displaystyle r $ si $\displaystyle \left(x -k\right)^{r} | p\left(x\right) $ y $\displaystyle \left(x-k\right)^{r+1} \not | p\left(x\right) $. 
\end{fdefinition}
\begin{observation}
\normalfont La suma de las multiplicidades de las raíces de un polinomio de grado $\displaystyle n $ es menor o igual que $\displaystyle n $.
\end{observation}
\section{Vectores y valores propios}
Sea $\displaystyle f\in\End\left(V\right) $. Sea $\displaystyle \left\{ \vec{u}_{1}, \ldots, \vec{u}_{n}\right\}  $ base de $\displaystyle V $. Tenemos que $\displaystyle \mathcal{M}_{ \left\{ \vec{u}_{i}\right\} \left\{ \vec{u}_{i}\right\} }\left(f\right) = A \in \mathcal{M}_{n \times n}\left(\K\right) $. Si $\displaystyle \left\{ \vec{v}_{1}, \ldots, \vec{v}_{n}\right\}  $ es otra base de $\displaystyle V $. Sea $\displaystyle \mathcal{M}_{ \left\{ \vec{v}_{i}\right\} \left\{ \vec{v}_{i}\right\} }\left(f\right) = B \in \mathcal{M}_{n \times n}\left(\K\right) $. 
Tenemos que
\[ \left(\vec{v}_{1}, \ldots, \vec{v}_{n}\right) = \left(\vec{u}_{1}, \ldots, \vec{u}_{n}\right) C, \; C \in \GL\left(n,\K\right) .\]
Así, tenemos que $\displaystyle B = C^{-1} A C $.
\begin{fdefinition}[]
	\normalfont Dos matrices $\displaystyle A, B \in \mathcal{M}_{n \times n}\left(\K\right) $ son \textbf{semejantes} si existe $\displaystyle C \in \GL\left(n, \K\right) $ tal que $\displaystyle B = C^{-1}AC $.
\end{fdefinition}
Sea $\displaystyle f \in \End\left(V\right) $ (o $\displaystyle A \in \mathcal{M}_{n \times n}\left(\K\right) $).
\begin{fdefinition}[]
	\normalfont Un vector $\displaystyle \vec{x} \in V $ es un \textbf{vector propio} o \textbf{autovector} de $\displaystyle f $ si existe $\displaystyle \lambda \in \K $ tal que $\displaystyle f\left(\vec{x}\right) = \lambda \vec{x} $. Similarmente, se dice que un escalar $\displaystyle \lambda \in \K $ es un \textbf{valor propio} o \textbf{autovalor} de $\displaystyle f $ si existe $\displaystyle \vec{x} \in V/ \left\{ \vec{0}\right\}  $ tal que $\displaystyle f\left(x\right) = \lambda \vec{x} $.
\end{fdefinition}
Sea $\displaystyle \lambda \in \K $ y sea $\displaystyle L_{\lambda } = \left\{ \vec{x} \in V \; : \; f\left(\vec{x}\right) = \lambda\vec{x}\right\} = \Ker\left(f - \lambda id _{V}\right) \in \mathcal{L}\left(V\right)$. Si $\displaystyle \lambda  $ no es valor propio de $\displaystyle f $, tenemos que $\displaystyle L_{\lambda } = \left\{ \vec{0}\right\}  $.
\begin{observation}
	\normalfont En las simetrías los únicos autovalores son $\displaystyle \left\{ -1, 1\right\}  $. En efecto, tenemos que $\displaystyle V = \Ker\left(s + id _{V}\right) \oplus \Ker\left(s - id _{V}\right) $.
\end{observation}
\begin{ftheorem}[]
	\normalfont Sea $\displaystyle \lambda_{1}, \ldots, \lambda_{p} \in \K $ valores propios de $\displaystyle f $ distintos 2 a 2, y sean $\displaystyle \vec{x}_{i} \in L_{\lambda_{i}} / \left\{ \vec{0}\right\}, \forall i = 1, \ldots, p $. Entonces, $\displaystyle \left\{ \vec{x}_{1}, \ldots, \vec{x}_{p}\right\}  $ son linealmente independientes.
\end{ftheorem}
\begin{proof}
	Si $\displaystyle p = 1 $, tenemos que $\displaystyle \vec{x}_{1} \neq \vec{0} $, por lo que $\displaystyle \left\{ \vec{x}_{1}\right\}  $ es linealmente independiente. Ahora, consideremos el caso de $\displaystyle p = 2 $. Sean $\displaystyle a^{1}, a^{2} \in \K $ tales que $\displaystyle a^{1}\vec{x}_{1} + a^{2}\vec{x}_{2} = \vec{0} $. Así, tenemos que
	\[\vec{0} = f\left(\vec{0}\right) = f\left(a^{1}\vec{x}_{1} + a^{2}\vec{x}_{2}\right) = a^{1}f\left(\vec{x}_{1}\right) + a^{2}f\left(\vec{x}_{2}\right) = a^{1}\lambda_{1}\vec{x}_{1} + a^{2}\lambda_{2}\vec{x}_{2} .\]
Así, tenemos que
\[\lambda_{1} \cdot \vec{0} = \lambda_{1}\left(a^{1}\vec{x}_{1} + a^{2}\vec{x}_{2}\right) = a^{1}\lambda_{1}\vec{x}_{1} + a^{2}\lambda_{1}\vec{x}_{2} .\]
Así, restando las dos ecuaciones anteriores tenemos que
\[ \vec{0} = \left(a^{1}\lambda_{1} - a^{1}\lambda_{1}\right) \vec{x}_{1} + \left(a^{2}\lambda_{2}-a^{2}\lambda_{1}\right)\vec{x}_{2} \Rightarrow \vec{0} = a^{2}\left(\lambda_{2}-\lambda_{1}\right)\vec{x}_{2} \Rightarrow a^{2} = 0 .\]
Así, como $\displaystyle a^{1}\vec{x}_{1} = \vec{0} $, tenemos que $\displaystyle a^{1} = 0 $. Ahora, asumimos que es cierto para $\displaystyle p - 1 $. En el caso de $\displaystyle p $, sean $\displaystyle \left\{ a^{1}, \ldots, a^{p}\right\} \subset \K $ tales que $\displaystyle a^{1}\vec{x}_{1} + \cdots + a^{p}\vec{x}_{p} = \vec{0} $. Tenemos que
\[\vec{0} = \lambda_{1} \vec{0} = \lambda_{1}a^{1}\vec{x}_{1} + \cdots + \lambda_{1}a^{p}\vec{x}_{p} .\]
Por otro lado tenemos que
\[\vec{0} = f\left(\vec{0}\right) = a^{1}\lambda_{1}\vec{x}_{1} + \cdots + a^{p}\lambda_{p}\vec{x}_{p} \Rightarrow a^{2}\left(\lambda_{2}-\lambda_{1}\right)\vec{x}_{2} + \cdots + a^{p}\left(\lambda_{p}-\lambda_{1}\right)\vec{x}_{p} .\]
Por hipótesis de inducción tenemos que 
\[ a^{2}\left(\lambda_{2}-\lambda_{1}\right) = a^{3}\left(\lambda_{3}-\lambda_{1}\right) = \cdots = a^{p}\left(\lambda_{p}-\lambda_{1}\right) = 0 .\]
Por tanto, $\displaystyle a^{2} = a^{3} = \cdots = a^{p} = 0 $ y, consecuentemente $\displaystyle a^{1}\vec{x}_{1} = \vec{0} \Rightarrow a^{1} = 0 $.
\end{proof}
\begin{observation}
\normalfont Una consecuencia de este teorema es que 
\[L_{\lambda_{1}}\oplus L_{\lambda_{2}} \oplus \cdots \oplus L_{\lambda_{p}} .\]
\end{observation}
\begin{observation}
\normalfont Otra consecuencia interesante es que si $\displaystyle \dim\left(V\right) = n $ y $\displaystyle f \in \End\left(V\right) $, entonces no puede haber más de $\displaystyle n $ valores propios.
\end{observation}
\section{Polinomio característico}
Tenemos que $\displaystyle \K \subset \K[x] \subset \K\left(x\right) $, donde $\displaystyle \K\left(x\right)$ denota el cuerpo de las fracciones racionales. Podemos definir la relación de divisibilidad, que es una relación de equivalencia:
\[\forall p\left(x\right), p'\left(x\right) \in \K[x], \; \left( p\left(x\right), q\left(x\right)\right) \mathcal{R}\left( p'\left(x\right), q'\left(x\right)\right) \iff p\left(x\right)q'\left(x\right) = q\left(x\right) p'\left(x\right).\]
Sea $\displaystyle A \in \mathcal{M}_{n \times n}\left(\K\right) $, tenemos que 
\[\det\left(A - x I_{n \times n}\right) = \begin{vmatrix} a_{1}^{1}-x & a^{1}_{2} & \cdots & a^{1}_{j} & \cdots & a^{1}_{i} & \cdots & a^{1}_{n} \\
a^{2}_{1} & a^{2}_{2} - x & \cdots & a^{2}_{j} & \cdots & a^{2}_{i} & \cdots & a^{2}_{n}\\
\vdots & \vdots & \vdots & \vdots & \vdots & \vdots & \vdots & \vdots \\
a^{j}_{1} & a^{j}_{2} & \cdots & a^{j}_{j}-x & \cdots & a^{j}_{i} & \cdots & a^{j}_{n} \\
\vdots & \vdots & \vdots & \vdots & \vdots & \vdots & \vdots & \vdots \\
a^{i}_{1} & a^{i}_{2} & \cdots & a^{i}_{j} & \cdots & a^{i}_{i} - x & \cdots & a^{i}_{n} \\
\vdots & \vdots & \vdots & \vdots & \vdots & \vdots & \vdots & \vdots \\
a^{n}_{1} & a^{n}_{2} & \cdots & a^{n}_{j} & \cdots & a^{n}_{i} & \cdots & a^{n}_{n} - x
\end{vmatrix}  .\]
Así, tenemos que el valor del determinante será:
\[ \det\left(A - x I_{n \times n}\right) = \left(a^{1}_{1} - x\right)\cdots\left(a^{n}_{n}-x\right)+ q\left(x\right) = \left(-1\right)^{n}x^{n} + \left(-1\right)^{n-1}\left(a^{1}_{1}+ \cdots + a^{n}_{n}\right)x^{n-1} + q\left(x\right), \; \grad\left(q\left(x\right)\right) \leq n - 2 .\]
Dado que el término independiente será $\displaystyle \det\left(A\right) $,
\[\det\left(A - xI_{n \times n}\right) = \left(-1\right)^{n}x^{n} + \left(-1\right)^{n - 1}\left(a^{1}_{1} + \cdots + a^{n}_{n}\right)x^{n-1} + \cdots + \det\left(A\right) .\]
Sea $\displaystyle B \in \mathcal{M}_{n \times n}\left(\K\right) $ semejante a $\displaystyle A \in \mathcal{M}_{n \times n}\left(\K\right) $. Entonces, existe $\displaystyle C \in \GL\left(n, \K\right) $ tal que $\displaystyle B = C^{-1}AC $. Así, tenemos que 
\[
\begin{split}
	\det\left(B - x I_{n\times n}\right) = & \det\left(C^{-1}AC - xC^{-1}C\right) = \det\left(C^{-1}AC - C^{-1} x C\right) = \det\left(C^{-1}\left(A - x I_{n \times n}\right)C\right) \\
	= & \det\left(C^{-1}\right)\det\left(A - x I_{n \times n}\right) \det \left(C\right) = \det\left(A - x I_{n \times n}\right).
\end{split}
\]
\begin{fdefinition}[Polinomio característico]
\normalfont Decimos que $\displaystyle \det\left(A - xI_{n \times n}\right) $ es el \textbf{polinomio característico} de $\displaystyle A $. Si $\displaystyle f \in \End\left(V\right) $, el polinomio característico de $\displaystyle f $ es el polinomio característico de $\displaystyle A $, donde $\displaystyle A $ es la matriz de $\displaystyle f $ en una base. \footnote{Normalmente se designa como $\displaystyle P_{cf} $ o $\displaystyle P_{cf}\left(x\right) $. }.
\end{fdefinition}
\begin{observation}
\normalfont Dado que el polinomio característico no depende de la base, la suma de los elementos de la diagonal principal de la matriz de un endomorfismo es invariante. 
\end{observation}
\begin{fdefinition}[]
\normalfont Llamaremos \textbf{traza} de $\displaystyle f $ a $\displaystyle a^{1}_{1} + \cdots + a^{n}_{n} = \traz\left(f\right) $ 
\end{fdefinition}
\begin{observation}
\normalfont Tenemos que 
\[
\begin{split}
	\traz : \End\left(V\right)  & \to \K \\
	f & \to \traz\left(f\right),
\end{split}
\]
es lineal. Además, $\displaystyle \traz\left(f \circ g\right) = \traz\left(g \circ f\right) $.
\end{observation}
\begin{observation}
\normalfont Tenemos que si $\displaystyle \lambda \in \K $ es valor propio de $\displaystyle f $, $\displaystyle \det\left(A - x I _{n \times n}\right) = 0 $.
\end{observation}
\begin{ftheorem}[]
\normalfont $\displaystyle \lambda \in \K $ es un valor propio de $\displaystyle f $ si y solo si $\displaystyle \lambda  $ es raíz de $\displaystyle P_{cf}\left(x\right) $. Además, si $\displaystyle \lambda  $ es un valor propio, $\displaystyle \dim\left(L_{\lambda }\right)  $ es menor o igual que la multiplicidad de $\displaystyle \lambda  $ como raíz de $\displaystyle P_{cf}\left(x\right) $.
\end{ftheorem}
\begin{proof}
	Tenemos que $\displaystyle \lambda  $ es valor propio de $\displaystyle f $ si y solo si existe $\displaystyle \vec{x} \in V/ \left\{ \vec{0}\right\}  $ tal que $\displaystyle \left(f - \lambda id _{V}\right)\left(\vec{x}\right) = \vec{0} $, si y solo si $\displaystyle \Ker\left(f - \lambda id _{V}\right) \neq \left\{ \vec{0}\right\}  $, si y solo si $\displaystyle f - \lambda id _{V} $ no es inyectiva, si y solo si $\displaystyle \det\left(A - \lambda I_{n \times n}\right) = 0 $, si y solo si $\displaystyle \lambda  $ es raíz de $\displaystyle P_{cf}\left(x\right) $.
	Sea $\displaystyle \lambda \in \K $ valor propio de $\displaystyle f $ y $\displaystyle L_{\lambda } = \left\{ \vec{x} \in V \; : \; f\left(\vec{x}\right) = \lambda \vec{x}\right\}  $. Sea $\displaystyle \left\{ \vec{u}_{1}, \ldots, \vec{u}_{k}\right\}  $ base de $\displaystyle L_{\lambda } $ y $\displaystyle \left\{ \vec{u}_{1}, \ldots, \vec{u}_{n}\right\}  $ base de $\displaystyle V $.
Tenemos que $\displaystyle \forall i = 1, \ldots, k $, $\displaystyle f\left(\vec{u}_{i}\right) = \lambda \vec{u}_{i} $. Así,
\[
	\mathcal{M}_{ \left\{ \vec{u}_{i}\right\} \left\{ \vec{u}_{i}\right\} }\left(f\right) = 
	\begin{pmatrix} \lambda & 0 & \cdots & 0\\
	0 & \lambda & \cdots & 0 \\
\vdots & \vdots & \vdots & \vdots \\
0 & 0 & \cdots & \lambda  & A\\
\vdots & \vdots & \vdots & \vdots \\
0 & 0 & \cdots & 0\end{pmatrix} = \begin{pmatrix} \lambda & 0 & \cdots & 0\\
	0 & \lambda & \cdots & 0 \\
	\vdots & \vdots & \vdots & \vdots & A_{1} \\
0 & 0 & \cdots & \lambda  \\
\vdots & \vdots & \vdots & \vdots & A_{2} \\
0 & 0 & \cdots & 0\end{pmatrix}
.\]
Tenemos que $\displaystyle A $ está formada por los componentes de las imágenes de los restantes vectores de la base. Si formásemos la matriz de la ecuación característica y desarrollásemos el determinante, está claro que en el polinomio característico aparece un factor de $\displaystyle \left(\lambda - x\right)^{k} $, por lo que $\displaystyle k $ no supera la multiplicidad de la raíz $\displaystyle \lambda  $ en el polinomio característico. También se puede ver de la siguiente forma:
\[P_{cf}\left(x\right) = \begin{vmatrix} \lambda - x & 0 & \cdots & 0\\
	0 & \lambda - x& \cdots & 0  \\
	\vdots & \vdots & \vdots & \vdots & A_{1} \\
0 & 0 & \cdots & \lambda  - x \\ \hline \\
\vdots & \vdots & \vdots & \vdots & A_{2} - x I_{\left(n - k\right)\times\left(n - k\right)}  \\
0 & 0 & \cdots & 0\end{vmatrix} = \left(\lambda - x\right)^{k} \left|A_{2} - I_{\left(n - k\right) \times \left(n - k\right)}\right|
 .\]
\end{proof}
\begin{fdefinition}[]
\normalfont Sea $\displaystyle \lambda  $ un valor propio de $\displaystyle f $. La \textbf{multiplicidad geométrica} de $\displaystyle \lambda  $ como valor propio de $\displaystyle f $ es $\displaystyle \dim\left(L_{\lambda }\right) $. Similarmente, la \textbf{multiplicidad algebraica}  de $\displaystyle \lambda  $ como valor propio de $\displaystyle f $ es la multiplicidad de $\displaystyle \lambda  $ como raíz de $\displaystyle P_{cf}\left(x\right) $. Entonces
\[m.g\left(\lambda \right) \leq m. a \left(\lambda \right) .\]
\end{fdefinition}
\section{Endomorfismos diagonalizables y triangulables}
\begin{fdefinition}[]
	\normalfont Un endomorfismo $\displaystyle f $ de $\displaystyle V $ es \textbf{diagonalizable} si existe $\displaystyle \left\{ \vec{u}_{1}, \ldots, \vec{u}_{n}\right\}  $ base de $\displaystyle V $ formada por vectores propios de $\displaystyle f $.
\end{fdefinition}
\begin{ftheorem}[]
	\normalfont Un endomorfismo $\displaystyle f $ es diagonalizable si y solo si $\displaystyle P_{cf}\left(x\right) $ se descompone en factores lineales en $\displaystyle \K[x] $ y $\displaystyle \forall \lambda \in \K $ valor propio de $\displaystyle f $, $\displaystyle m.g\left(\lambda \right) = m.a\left(\lambda \right) $.
\end{ftheorem}
\begin{proof}
\begin{description}
\item[(i)] Si $\displaystyle \exists \left\{ \vec{u}_{1}, \ldots, \vec{u}_{n}\right\}  $ base de $\displaystyle V $ foramda por vectores propios de $\displaystyle f $, entonces
\[\mathcal{M}_{ \left\{ \vec{u}_{i}\right\} }\left(f\right) = \begin{pmatrix} 
	\lambda _{1} & 0 & \cdots & 0 \\
0 & \lambda _{2} & \cdots & 0 \\
\vdots & \vdots & \vdots & \vdots \\
0 & 0 & \cdots & \lambda_{n}\end{pmatrix} .\]
Por lo que 
\[P_{cf}\left(x\right) = \left(\lambda_{1}-x\right)\left(\lambda_{2}-x\right) \cdots \left(\lambda_{n}-x\right) .\]
En efecto, 
\[P_{cf}\left(\lambda \right) = \begin{pmatrix} \lambda _{1} - \lambda & 0 & \cdots & 0 \\
0 & \lambda _{2} - \lambda & \cdots & 0 \\
\vdots & \vdots & \vdots & \vdots \\
0 & 0 & \cdots & \lambda_{n} - \lambda \end{pmatrix} .\]
Sean $\displaystyle \lambda_{1}, \ldots, \lambda_{k} \in \K $ valores propios de $\displaystyle f $, donde $\displaystyle \lambda_{i} \neq \lambda_{j} $ si $\displaystyle i \neq j $. Entonces, tenemos que $\displaystyle L_{\lambda_{1}} \oplus L_{\lambda_{2}} \oplus \cdots \oplus L_{\lambda_{k}} $. Entonces, tenemos que existen $\displaystyle r_{1}, r_{2}, \ldots, r_{k} \geq 1 $, con $\displaystyle r_{1} + \cdots + r_{k}  = n $, tales que
\[
\begin{split}
	P_{cf}\left(x\right) = & \left(x - \lambda_{1}\right)^{r_{1}} \cdots \left(x - \lambda _{k}\right)^{r_{k}} .
\end{split}
\]
Entonces, tenemos que 
\[\dim\left(V\right) \leq \dim\left(L_{\lambda_{1}}\right) + \cdots + \dim\left(L_{\lambda_{k}}\right) \leq r_{1} + \cdots + r_{k} = n .\]
Así, $\displaystyle \dim\left(L_{\lambda_{i}}\right) = r_{i} $, $\displaystyle \forall i = 1, \ldots, k $. Así, hemos obtenido que $\displaystyle L_{\lambda_{1}} \oplus \cdots \oplus L_{\lambda_{k}} = V $.
\item[(ii)] Si $\displaystyle P_{cf}\left(x\right) $ se descompone en factores lineales de forma que 
	\[P\left(cf\right)\left(x\right) = \left(\lambda_{1}-x\right)^{r_{1}}\cdots\left(\lambda_{k}-x\right)^{r_{k}} .\]
	Entonces, tenemos que las raíces del polinomio son valores propios de $\displaystyle f $, por lo que existen $\displaystyle L_{\lambda_{1}} \oplus \cdots \oplus L_{\lambda_{k}} $. Además, como $\displaystyle \dim\left(L_{\lambda_{i}}\right) = r_{i} $ para $\displaystyle i = 1, \ldots, k $, tenemos que $\displaystyle L_{\lambda_{1}}\oplus\cdots\oplus L_{\lambda_{k}} = V $, por lo que existe una base de $\displaystyle V $ formada por vectores propios y $\displaystyle f $ es diagonalizable. En efecto, sea $\displaystyle \left\{ \vec{u^{i}}_{1}, \ldots, \vec{u^{i}}_{r_{i}}\right\}  $ base de $\displaystyle L_{\lambda_{i}} $ con $\displaystyle i = 1, \ldots, k $. Entonces, tenemos que $\displaystyle \bigcup_{i = 1, \ldots, k} \left\{ \vec{u^{i}}_{1}, \ldots, \vec{u^{i}}_{r_{i}}\right\}  $ es base de $\displaystyle V $. Así, la matriz de $\displaystyle f $ en esta base será
\[ \begin{pmatrix} \lambda_{1} & 0 & \cdots & 0 \\\
0 & \lambda_{2} & \cdots & 0 \\
 \vdots & \vdots & \vdots & \vdots \\
0 & 0 & \cdots & \lambda_{k}\end{pmatrix}.\]
\end{description}
\end{proof}
\begin{observation}
\normalfont Para ver si un endofmorfismo es diagonalizable, hay que resolver el polinomio característico lo que da unas raíces $\displaystyle \lambda_{1}, \ldots, \lambda_{p} $ y lo primero que hay que comprobar es si la suma de las multiplicidades coincide con la dimensión de $\displaystyle V $.
\end{observation}

\begin{fdefinition}[]
	\normalfont Sea $\displaystyle f \in \End\left(V\right) $ es \textbf{triangulable} si $\displaystyle \exists \left\{ \vec{u}_{1}, \ldots, \vec{u}_{n}\right\}  $ base de $\displaystyle V $ tal que 
	\[\mathcal{M}_{ \left\{ \vec{u}_{i}\right\} \left\{ \vec{u}_{i}\right\} }\left(f\right) =  \begin{pmatrix} a^{1}_{1} & a^{1}_{2} & \cdots & a^{1}_{n} \\
		0 & a^{2}_{2} & \cdots & a^{2}_{n} \\
0 & 0 & \cdots & a^{3}_{n} \\
\vdots & \vdots & \vdots & \vdots \\
0 & 0 & \cdots & a^{n}_{n}\end{pmatrix} .\]
\end{fdefinition}
\begin{observation}
\normalfont Tenemos que $\displaystyle P_{cf}\left(x\right) = \det\left(A - x I_{n \times n}\right)= \left(-1\right)^{n}\left(x - a^{1}_{1}\right) \cdots \left(x- a^{n}_{n}\right) $.
\end{observation}
\begin{ftheorem}[]
	\normalfont $\displaystyle f $ es triangulable si y solo si $\displaystyle P_{cf}\left(x\right) $ se descompone en factores lineales en $\displaystyle \K[x] $.
\end{ftheorem}
\begin{proof}
\begin{description}
\item[(i)] Evidente.
\item[(ii)] Consideremos el caso $\displaystyle n = 2 $. Sea $\displaystyle \lambda \in \K $ un vector propio de $\displaystyle f $ y $\displaystyle \vec{x}_{1} \in L_{\lambda }/ \left\{ \vec{0}\right\}  $. Entonces tenemos que $\displaystyle f\left(\vec{x}_{1}\right) = \lambda\vec{x}_{1} $. Si $\displaystyle \left\{ \vec{x}_{1}, \vec{x}_{2}\right\}  $ es base de $\displaystyle V $, tenemos que
	\[\mathcal{M}_{ \left\{ \vec{u}_{i}\right\} \left\{ \vec{u}_{i}\right\} }\left(f\right) = \begin{pmatrix} \lambda & a \\
	0 & b\end{pmatrix} .\]
	Asumimos que es cierto para $\displaystyle n - 1 $. Sea $\displaystyle \lambda_{1} \in \K $ un valor propio de $\displaystyle f $ y sea $\displaystyle \vec{x}_{1} \in L_{\lambda_{1}} / \left\{ \vec{0}\right\}  $. Sea $\displaystyle \left\{ \vec{x}_{1}, \ldots, \vec{ x}_{n}\right\}  $ base de $\displaystyle V $ y sea $\displaystyle L = L\left( \left\{ \vec{x}_{2}, \ldots, \vec{x}_{n}\right\} \right) $. Tenemos que $\displaystyle L \oplus L\left( \left\{ \vec{x}_{1}\right\} \right) = V $. 
	Sea $\displaystyle p $ la proyección vectorial de base $\displaystyle L $ y dirección $\displaystyle L\left( \left\{ \vec{x}_{1}\right\} \right) $, y sea $\displaystyle p \circ f \in \End\left(L\right) $. Tenemos que $\displaystyle \dim\left(L\right) = n - 1 $. Tenemos que 
		\[\mathcal{M}_{ \left\{ \vec{u}_{i}\right\} \left\{ \vec{u}_{i}\right\} }\left(f\right) = \begin{pmatrix} \lambda _{1} & a^{1}_{2} & \cdots & a^{1}_{n} \\
		0 & a^{2}_{2} & \cdots & a^{2}_{n} \\
	\vdots & \vdots & \vdots & \vdots\\
0 & a^{n}_{2} & \cdots & a^{n}_{n}\end{pmatrix} .\]
Similarmente, sabemos que si $\displaystyle \vec{x} \in V $, 
\[p\circ f\left(\vec{x}\right) = p\left(a^{1}_{i}\vec{x}_{1} + \cdots + a^{n}_{i}\vec{x}_{n}\right)= a^{2}_{i}\vec{x}_{2} + \cdots + a^{n}_{i}\vec{x}_{n} .\]
Entonces, tenemos que 
\[p\circ f \to \begin{pmatrix} a^{2}_{2} & \cdots & a^{2}_{n} \\
\vdots & & \vdots \\
a^{n}_{2} & \cdots & a^{n}_{n}\end{pmatrix} \Rightarrow P_{cf}\left(x\right) = - \left(x - \lambda_{1}\right) P_{c,p \circ f}\left(x\right) .\]
	Por hipótesis de inducción, existe $\displaystyle \left\{ \vec{u}_{2}, \ldots, \vec{u}_{n}\right\}  $ base de $\displaystyle L $ tal que $\displaystyle \mathcal{M}_{ \left\{ \vec{u}_{i}\right\} \left\{ \vec{u}_{i}\right\} }\left(p \circ f\right) = \begin{pmatrix} \lambda_{2} & b^{2}_{2} & \cdots & b^{2}_{n} \\
	0 & \lambda_{3} & \cdots & b^{3}_{n} \\
\vdots & \vdots & \vdots & \vdots \\
0 & 0 & \cdots & \lambda_{n}\end{pmatrix} $. Así, en la base $\displaystyle \left\{ \vec{x}_{1}, \vec{u}_{2}, \ldots, \vec{u}_{n}\right\}  $, la matriz de $\displaystyle f $ es
		\[\begin{pmatrix} \lambda_{1} & a^{1}_{2} & a^{1}_{3} & \cdots & a^{1}_{n}\\
		0 & \lambda_{2} & b^{2}_{3} & \cdots & b^{2}_{n} \\
	\vdots & \vdots & \vdots & \vdots & \vdots \\
0 & 0 & 0 & \cdots & \lambda_{n}\end{pmatrix} .\]
\end{description}
\end{proof}
\begin{observation}
\normalfont 
Si $\displaystyle p' $ es la proyección de base $\displaystyle L\left( \left\{ \vec{x}_{1}\right\} \right) $ y dirección $\displaystyle L $, tenemos que $\displaystyle id _{V} = p + p' $. Otra forma de decir que $\displaystyle f $ es triangulable, es decir que existe una sucesión de subespacios de $\displaystyle V $: $\displaystyle \left\{ \vec{0}\right\}  \subset L_{1} \subset \cdots \subset L_{n} = V $ tales que $\displaystyle f\left(L_{i}\right) \subset L_{i} $ para $\displaystyle i = 1, \ldots, n $.
\end{observation}

\begin{fdefinition}[]
\normalfont Un endomorfismo $\displaystyle f $ de $\displaystyle V $ es \textbf{nilpotente} si existe $\displaystyle m \in \N $ tal que $\displaystyle f^{m} = 0 $ y $\displaystyle f^{m - 1} \neq 0 $. 
\end{fdefinition}
\begin{observation}
\normalfont Si $\displaystyle \vec{x}  $ es un vector propio de $\displaystyle f $, tenemos que $\displaystyle f\left(\vec{x}\right) = \lambda \vec{x} $, \ldots, $\displaystyle f^{m - 1}\left(\vec{x}\right) = \lambda^{m - 1}\vec{x} $ y $\displaystyle f^{m}\left(\vec{x}\right) =\lambda^{m}\vec{x}=\vec{0} $. Es decir, si es nilpotente, el único posible valor propio es el 0. Es decir, en este caso la matriz triangular tiene diagonal nula.
\end{observation}
\section{Teorema de Cayley-Hamilton}
Tenemos que $\displaystyle P_{cf}\left(f\right) \in \End\left(V\right)$. Si $\displaystyle A $ es la matriz de $\displaystyle f $, tenemos que $\displaystyle P_{cA}\left(A\right) \in \mathcal{M}_{n \times n}\left(\K\right) $. Si $\displaystyle p\left(x\right), q\left(x\right) \in \K[x] $, tenemos definidas las operaciones
\[p\left(x\right) + q\left(x\right) = \left(p + q\right)\left(x\right) = \left(q + p\right)\left(x\right) = q\left(x\right) + p\left(x\right) .\]
\[p\left(x\right)q\left(x\right) = q\left(x\right) p\left(x\right) .\]
Similarmente, 
\[\left(p\left(x\right)+q\left(x\right)\right)\left(f\right) = p\left(f\right) + q\left(f\right) .\]
\[\left(p\left(x\right)q\left(x\right)\right)\left(f\right) = p\left(f\right)\circ q\left(f\right) .\]
Entonces, tenemos que
\[p\left(A\right) = \mathcal{M}_{ \left\{ \vec{u}_{i}\right\} , \left\{ \vec{u}_{i}\right\} }\left(p\left(f\right)\right) .\]
\begin{observation}
	\normalfont Si $\displaystyle p\left(x\right) \in \K[x] $ tal que $\displaystyle p\left(x\right) = a_{0} + a_{1}x + \cdots + a_{n}x^{n} $, y $\displaystyle f \in \End\left(V\right)$, se define 
	\[p\left(f\right) = a_{0} + a_{1}f + \cdots + a_{n}f^{n} \in \End\left(V\right) .\]
\end{observation}
\begin{ftheorem}[]
\normalfont 
\[P_{cf}\left(f\right) = 0 \in \End\left(V\right), \; P_{CA}\left(A\right) = 0 .\]
\end{ftheorem}
\begin{proof}
	Sea $\displaystyle  \K = \C $, entonces $\displaystyle P_{cf}\left(x\right) = \left(-1\right)^{n}\left(x - \lambda_{1}\right) \cdots \left(x - \lambda_{n}\right) $. Según el teorema 4.7, tenemos que existe una base de $\displaystyle V $, $\displaystyle \left\{ \vec{u}_{1}, \ldots, \vec{u}_{n}\right\}  $ tal que 
	\[
	\begin{split}
	& f\left(\vec{u}_{1}\right) = \lambda_{1}\vec{u}_{1} \\
	& f\left(\vec{u}_{2}\right) = a^{1}_{2}\vec{u}_{1} + \lambda_{2}\vec{u}_{2} \\
	& \vdots \\
	& f\left(\vec{u}_{n}\right) = a^{1}_{n}\vec{u}_{1} + \cdots + a^{n-1}_{n}\vec{u}_{n-1}+\lambda_{n}\vec{u}_{n}.
	\end{split}
	\]
Entonces tenemos que 
\[P_{cf}\left(f\right) = \left(-1\right)^{n}\left(f-\lambda_{1}id _{V}\right) \circ \cdots \circ \left(f - \lambda_{n}id _{V}\right). \] 
Para ver que $\displaystyle P_{cf}\left(f\right) $ es el endomorfismo nulo, tenemos que ver que se anula sobre la base anterior. Así, $\displaystyle \left(f - \lambda_{1}id _{V}\right)\left(\vec{u}_{1}\right) = \vec{0} $. Dado que $\displaystyle \left(f - \lambda_{1}id _{V}\right) \circ \left(f - \lambda_{2}id _{V}\right) = \left(f - \lambda_{2}id _{V}\right) \circ \left(f - \lambda_{1}id _{V}\right) $, tenemos que $\displaystyle \left(f - \lambda_{2}id _{V}\right) \circ \left(f - \lambda_{1}id _{V}\right)\left(\vec{u}_{1}\right) = \vec{0} $.
También tenemos que  
\[
\begin{split}
	\left(f - \lambda_{1}id _{V}\right) \circ \left(f - \lambda_{2}id _{V}\right)\left(\vec{u}_{2}\right) = \left(f - \lambda_{1}id _{V}\right) \circ \left(f - \lambda_{2}id _{V}\right)\left(a^{1}_{2}\vec{u}_{1}+\lambda_{1}\vec{u}_{2}\right) = \vec{0} .
\end{split}
\]
Se deja como ejercicio para el lector que esto definitivamente es cierto.
Suponemos que es cierto para $\displaystyle i \leq n $. Tenemos que $\displaystyle \left(f - \lambda_{i - 1}id _{V}\right)\left(\vec{u}_{j}\right) = \vec{0} $ para $\displaystyle j = 1, \ldots, i - 1 $. Además, 
\[ \left(f - \lambda_{i}id _{V}\right)\circ \left(f - \lambda_{1}id _{V}\right) \circ \cdots \circ \left(f - \lambda_{i - 1}id _{V}\right)= \left(f - \lambda_{1}id _{V}\right)\circ \cdots\circ \left(f - \lambda_{i}id _{V}\right).\]
Así, por hipótesis tenemos que para $\displaystyle j = 1, \ldots, i - 1 $,
\[
\begin{split}
\left(f - \lambda_{1}id _{V}\right)\circ \cdots\circ \left(f - \lambda_{i}id _{V}\right)\left(\vec{u}_{j}\right) = \left(f - \lambda_{i}id _{V}\right)\circ\left(f - \lambda_{1}id _{V}\right)\circ \cdots\circ \left(f - \lambda_{i-1}id _{V}\right)\left(\vec{u}_{j}\right) = \vec{0}.
\end{split}
\]
En el caso de $\displaystyle \vec{u}_{i} $, tenemos que $\displaystyle \left(f - \lambda_{i}id _{V}\right)\left(\vec{u}_{i}\right) = f\left(\vec{u}_{i}\right)-\lambda_{i}\vec{u}_{i} =  a^{1}_{i}\vec{u}_{1} + \cdots + a^{i -1}_{i}\vec{u}_{i-1} $. Si aplicamos $\displaystyle \left(f - \lambda_{1}id _{V}\right) \circ \cdots \circ \left(f - \lambda_{i - 1}id _{V}\right) $ sobre el resultado anterior, obtendremos el vactor nulo, por lo que
\[ \left(f - \lambda_{1}id _{V}\right) \circ \cdots \circ \left(f - \lambda_{i - 1}id _{V}\right)\left(f - \lambda_{i}id _{V}\right)\left(\vec{u}_{i}\right) = \vec{0} .\]
\end{proof}
Recordamos que $\displaystyle P_{cf}\left(x\right) = \det\left(A - xI_{n\times n}\right) $, donde $\displaystyle f \in \End\left(V\right) $ y $\displaystyle A $ es la matriz asociada a $\displaystyle f $. Sea $\displaystyle I = \left\{ p\left(x\right) \in \K[x] \; : \; p\left(f\right) = 0\right\}  $. Sabemos que $\displaystyle I \neq \emptyset $. 
Tenemos que $\displaystyle I $ es un ideal. En efecto, si $\displaystyle p\left(x\right), q\left(x\right) \in I $,
\[ \left(p\left(x\right)+q\left(x\right)\right)\left(f\right) = p\left(f\right) + q\left(f\right)= 0 .\]
Similarmente, si $\displaystyle p\left(x\right) \in \K[x] $ y $\displaystyle q\left(x\right) \in I $, entonces $\displaystyle p\left(f\right) \in \End\left(V\right) $ y $\displaystyle q\left(f\right) = 0 $, por lo que $\displaystyle p\left(f\right) \circ q\left(f\right) = 0 $. \\ \\
Entonces, existe $\displaystyle P_{mf}\left(x\right) \in \K[x] $ mónico tal que $\displaystyle I = \left(P_{mf}\left(x\right)\right) $.
\begin{fdefinition}[]
\normalfont A $\displaystyle P_{mf}\left(x\right) $ lo llamamos \textbf{polinomio mínimo} de $\displaystyle f $. Se dice que $\displaystyle p\left(x\right) $ es un \textbf{polinomio anulador} de $\displaystyle f $ si $\displaystyle p\left(x\right) \in I $.
\end{fdefinition}
\begin{fdefinition}[]
\normalfont Un subespacio vectorial $\displaystyle L $ de $\displaystyle V $ es \textbf{invariante} por $\displaystyle f $ si $\displaystyle f\left(L\right) \subset L $.
\end{fdefinition}
\begin{observation}
\normalfont Si $\displaystyle V = L_{1} \oplus L_{2} \oplus \cdots \oplus L_{k} $ con $\displaystyle L_{i} \in \mathcal{L}\left(V\right) $ ivariante por $\displaystyle f $ para $\displaystyle i = 1, \ldots, k $:
\[f|_{L_{i}}:L_{i} \to L_{i} .\]
Si $\displaystyle \left\{ \vec{u}^{i}_{1}, \ldots, \vec{u}^{i}_{k}\right\}  $ base de $\displaystyle L_{i} $ para $\displaystyle i = 1, \ldots, k $, entonces 
\[ \left\{ \vec{u}^{1}_{1}, \ldots, \vec{u}^{1}_{k}, \ldots, \vec{u}^{k}_{1}, \ldots, \vec{u}^{k}_{k}\right\}  ,\]
es base de $\displaystyle V $. Así, tenemos que 
\[ \mathcal{M}_{ \left\{ \vec{u}^{j}_{i}\right\} \left\{ \vec{u}^{j}_{i}\right\} }\left(f\right) = \begin{pmatrix} A_{1} & 0 & \cdots & 0 \\
0 & A_{2} & \cdots & 0 \\
\vdots & \vdots & \cdots & \vdots \\
0 & 0 & \cdots & A_{n}\end{pmatrix} .\]
Donde $\displaystyle A_{j} = \mathcal{M}_{ \left\{ \vec{u}^{j}_{i}\right\} \left\{ \vec{u}^{j}_{i}\right\} } \left(f|_{L_{j}}\right)$.
\end{observation}
Si $\displaystyle p\left(x\right) \in \K[x] $.
\begin{fprop}[]
\normalfont 
\begin{description}
\item[(a)] $\displaystyle \Ker\left(p\left(f\right)\right) $ es un subespacio vectorial invariante por $\displaystyle f $.
\item[(b)] $\displaystyle \Imagen\left(p\left(f\right)\right)\in \mathcal{L}\left(V\right) $ es un subespacio vectorial invariante por $\displaystyle f $.
\end{description}
\end{fprop}
\begin{proof}
\begin{description}
\item[(a)] Tenemos que $\displaystyle p\left(x\right) \cdot x = x \cdot p\left(x\right) $. Así, si $\displaystyle p\left(f\right) \left(\vec{x}\right) = \vec{0} $, 
	\[p\left(f\right)\left(f\left(\vec{x}\right)\right) = \left(\left(p\left(x\right) \cdot x\right)\left(f\right)\right)\left(\vec{x}\right) = f \circ p\left(f\right)\left(\vec{x}\right) = f\left(\vec{0}\right) = \vec{0} .\]
\item[(b)] Si $\displaystyle \vec{x} \in \Imagen\left(p\left(f\right)\right) $, entonces existe $\displaystyle \vec{y} \in V $ tal que $\displaystyle p\left(f\left(\vec{y}\right)\right) = \vec{x} $. Así,
	\[f\left(\vec{x}\right) = f\left(p\left(f\left(\vec{y}\right)\right)\right) = \left(f\circ p\left(f\right)\right)\left(\vec{y}\right) = p\left(f\right) \circ f\left(\vec{y}\right) = p\left(f\right)\left( f\left(\vec{y}\right) \right) \in \Imagen\left(p\left(f\right)\right).\]	
\end{description}
\end{proof}
\begin{ftheorem}[]
\normalfont Sea $\displaystyle p\left(x\right) $ un polinomio anulador de $\displaystyle f $ y sea $\displaystyle p\left(x\right) = a_{1}\left(x\right) \cdot a_{2}\left(x\right) $, con $\displaystyle a_{1}\left(x\right) $ y $\displaystyle a_{2}\left(x\right) $ primos entre sí. Entonces, $\displaystyle V = \Ker\left(a_{1}\left(f\right)\right) \oplus \Ker\left(a_{2}\left(f\right)\right) $.
\end{ftheorem}
\begin{proof}
	Tenemos que $\displaystyle \mcd\left(a_{1}\left(x\right), a_{2}\left(x\right)\right) = 1 $, por lo que existen $\displaystyle p_{1}\left(x\right), p_{2}\left(x\right) \in \K[x] $ tales que 
	\[ 1 = p_{1}\left(x\right)a_{1}\left(x\right) + p_{2}\left(x\right)a_{2}\left(x\right) .\]
Así, $\displaystyle \forall \vec{x} \in V $, 
\[id _{V}\left(\vec{x}\right) = \left(\left(p_{1}\left(x\right)a_{1}\left(x\right)\right)f\right)\left(\vec{x}\right) + \left(\left(p_{2}\left(x\right)a_{2}\left(x\right)\right)f\right)\left(\vec{x}\right) .\]
Así,
\[\vec{x} = \underbrace{\left(p_{1}\left(f\right)\circ a_{1}\left(f\right)\right)\left(\vec{x}\right)}_{\in\Ker\left(a_{2}\left(f\right)\right)}+\underbrace{\left(p_{2}\left(f\right)\circ a_{2}\left(f\right)\right)\left(\vec{x}\right)}_{\in\Ker\left(a_{1}\left(f\right)\right)} .\]
En efecto, tenemos que 
\[
\begin{split}
	& a_{2}\left(f\right) \circ \left(p_{1}\left(f\right)\circ a_{1}\left(f\right)\right))\left(\vec{x}\right) = p_{1}\left(f\right)\circ\underbrace{\left(a_{1}\left(f\right)\circ a_{2}\left(f\right)\right)}_{p\left(f\right)}\left(\vec{x}\right) = \vec{0} \\
	& a_{1}\left(f\right) \circ \left(p_{2}\left(f\right)\circ a_{2}\left(f\right)\right)\left(\vec{x}\right) = p_{2}\left(f\right) \circ \underbrace{\left(a_{1}\left(f\right)\circ a_{2}\left(f\right)\right)}_{p\left(f\right)}\left(\vec{x}\right) = \vec{0}.
\end{split}
\]
Finalmente, si $\displaystyle \vec{x}\in \Ker\left(a_{1}\left(f\right)\right)\cap\Ker\left(a_{2}\left(f\right)\right) $, tenemos que 
\[\vec{x} = \left(p_{1}\left(f\right) \circ a_{1}\left(f\right)\right)\left(\vec{x}\right) + \left(p_{2}\left(f\right)\circ a_{2}\left(f\right)\right)\left(\vec{x}\right) = p_{1}\left(f\right)\left(\vec{0}\right) + p_{2}\left(f\right)\left(\vec{0}\right) = \vec{0} .\]
\end{proof}
\begin{fprop}[]
\normalfont Si $\displaystyle p\left(x\right) = p_{1}\left(x\right)^{r_{1}}\cdots p_{k}\left(x\right)^{r_{k}} $ es la descomposición en factores irreducibles de un polinomio anulador de $\displaystyle f $, entonces
\[V = \Ker\left(p_{1}\left(f\right)^{r_{1}}\right) \oplus \cdots \oplus \Ker\left(p_{k}\left(f\right)^{r_{k}}\right) .\]
\end{fprop}
\begin{proof}
La demostración se realiza por inducción sobre $\displaystyle k $. 
\begin{description}
\item[(i)] Si $\displaystyle k=2 $, queda demostrado por el resultado anterior.
\item[(ii)] Asumimos que el resultado es cierto para $\displaystyle k - 1 $. Tenemos que los polinomios $\displaystyle p_{1}\left(x\right)^{r_{1}}, \ldots, p_{k}\left(x\right)^{r_{k}} $ son primos entre si. Entonces, tenemos que
	\[ V = \Ker\left(p_{1}\left(f\right)^{r_{1}}\right) \oplus \Ker\left(p_{2}\left(f\right)^{r_{2}} \circ \cdots \circ p_{k}\left(f\right)^{r_{k}}\right) .\]
Por hipótesis de inducción, tenemos que $\displaystyle V = \Ker\left(p_{1}\left(f\right)^{r_{1}}\right) \oplus \cdots \oplus \Ker\left(p_{k}\left(f\right)\right)^{r_{k}} $.
\end{description}
\end{proof}
\begin{fprop}[]
\normalfont Sea $\displaystyle P_{mf}\left(x\right) = p_{1}\left(x\right)p_{2}\left(x\right) $ con $\displaystyle p_{1}\left(x\right) $ y $\displaystyle p_{2}\left(x\right) $ primos entre si. Entonces, $\displaystyle V = \Ker\left(p_{1}\left(f\right)\right) \oplus \Ker\left(p_{2}\left(f\right)\right) $. Además, $\displaystyle p_{1}\left(x\right) = P_{m, f|_{\Ker\left(p_{1}\left(f\right)\right)}} $ y $\displaystyle p_{2}\left(x\right) = P_{m,f|_{\Ker\left(p_{2}\left(f\right)\right)}} $, salvo por factores escalares.
\end{fprop}
\begin{proof}
Lo único que hay que demostrar es la última afirmación. Tenemos que $\displaystyle p_{1}\left(x\right) $ es un polinomio anulador de $\displaystyle f|_{\Ker\left(p_{2}\left(f\right)\right)} $ y lo mismo sucede para $\displaystyle p_{2}\left(x\right) $. Similarmente, $\displaystyle p_{1}\left(x\right)p_{2}\left(x\right) $ es anulador de $\displaystyle f $, por lo que es producto del polinomio mínimo de $\displaystyle f $. La demostración termina utilizando el hecho de que muchas cosas son primas entre sí.
\end{proof}
\begin{fprop}[]
\normalfont Sea $\displaystyle P_{mf} = p_{1}\left(x\right)^{r_{1}}\cdots p_{k}\left(x\right)^{r_{k}} $ la descomposición en factores irreducibles del polinomio mínimo de $\displaystyle f $. Entonces, tenemos que $\displaystyle V = \Ker\left(p_{1}\left(f\right)^{r_{1}}\right) \oplus \cdots \oplus \Ker\left(p_{k}\left(f\right)^{r_{k}}\right) $.
\end{fprop}
\begin{observation}
\normalfont Si cogemos la base de los subespacios anteriores, la matriz de $\displaystyle f $ en esta base se nos quedará de la forma
\[ f \to \begin{pmatrix} A_{1} & 0 & \cdots & 0 \\
0 & A_{2} & \cdots & 0 \\
\vdots & \vdots & \vdots & \vdots \\
0 & 0 & \cdots & A_{k}\end{pmatrix} .\]
Donde $\displaystyle A_{i}  = \mathcal{M}\left(f|_{\Ker\left(p_{i}\left(f\right)^{r_{i}}\right)}\right) $, para $\displaystyle i = 1, \ldots, k $. Además, tenemos que,
\[ P_{m, f|_{\Ker\left(p_{i}\left(f\right)^{r_{i}}\right)}}\left(x\right) = p_{i}\left(x\right)^{r_{i}}.\]
\end{observation}
\begin{fdefinition}[]
\normalfont Un endomorfismo $\displaystyle f \in \End\left(V\right) $ es \textbf{nilpotente} si existe $\displaystyle m \in \N $ tal que $\displaystyle f^{m} = 0 $ pero $\displaystyle f^{m - 1} \neq 0 $.
\end{fdefinition}
\begin{observation}
	\normalfont Tenemos que $\displaystyle p_{i}\left(f\right) ^{r_{i}}\left(x\right) = P_{m,f|\Ker\left(p_{i}\left(f\right)^{r_{i}}\right)}\left(x\right) $. Si $\displaystyle P_{mf}\left(x\right) = p_{1}\left(x\right) $ se descompone en factores lineales en $\displaystyle \K[x] $ y $\displaystyle p_{1}\left(x\right) = x - \lambda_{1} $, entonces $\displaystyle p_{1}\left(f\right) = f - \lambda_{1} $.
\end{observation}
Sea $\displaystyle f $ un endomorfismo de índice de nilpotencia $\displaystyle m \in \N $. Sea $\displaystyle \vec{x} \in V $ tal que $\displaystyle f^{m - 1}\left(\vec{x}\right) \neq \vec{0} $. Sea $\displaystyle L = L\left( \left\{ \vec{x}, f\left(\vec{x}\right), \ldots, f^{m - 1}\left(\vec{x}\right)\right\} \right) $. Tenemos que $\displaystyle \left\{ \vec{x}, f\left(\vec{x}\right), \ldots, f^{m -1}\left(\vec{x}\right)\right\}  $ son linealmente independentes. En efecto, si $\displaystyle a_{0}, \ldots, a_{m - 1} \in \K $,
\[a_{0}\vec{x} + a_{1}f\left(\vec{x}\right) + \cdots + a_{m - 1}f^{m - 1}\left(\vec{x}\right) = \vec{0} .\]
Por tanto,
\[
\begin{split}
f\left(a_{0}\vec{x} + a_{1}f\left(\vec{x}\right) + \cdots + a_{m - 1}f^{m - 1}\left(\vec{x}\right)\right) = a_{0}f\left(\vec{x}\right) + \cdots a_{m -2}f^{m -1}\left(\vec{x}\right) = \vec{0}.
\end{split}
\]
Iterando, tenemos que $\displaystyle a_{0}f^{m - 1}\left(\vec{x}\right) = \vec{0} $. Como $\displaystyle \vec{x} \not\in \Ker\left(f^{m -1}\right) $, tenemos que $\displaystyle a_{0} = 0 $. Supongamos que $\displaystyle a_{0} = a_{1} = \cdots = a_{i} = 0$, con $\displaystyle i < m $. Entonces, tenemos que 
\[ a_{i}f^{i}\left(\vec{x}\right) + a_{i+1}f^{i+1}\left(\vec{x}\right) + \cdots + a_{m -1}f^{m -1}\left(\vec{x}\right) = \vec{0} .\]
Aplicando $\displaystyle f^{m -i-1} $, tenemos que
\[\vec{0} = a^{i}f^{m -1}\left(\vec{x}\right)  .\]
Como vimos anteriormente, $\displaystyle a_{i} = 0 $. Así, concluimos que $\displaystyle a_{i} = 0 $ para $\displaystyle i = 0, 1, \ldots, m -1 $. 
\begin{observation}
\normalfont Tenemos que $\displaystyle L $ es de dimensión $\displaystyle m $ y $\displaystyle f\left(L\right) \subset L $. Además, la matriz de $\displaystyle f|_{L} $ en la base anterior será de la forma:
\[ f|_{L} \to \begin{pmatrix} 0 & 0  & \cdots &0 & 0\\
	1 & 0  & \cdots & 0 & 0 \\
0 & 1  & \cdots & 0 & 0 \\
\vdots & \vdots  & \vdots & \vdots & \vdots \\
0 & 0 & \cdots & 1 & 0 \end{pmatrix} .\]
\end{observation}
\begin{ftheorem}[]
\normalfont Existe $\displaystyle L'\in \mathcal{L}\left(V\right) $ invariante por $\displaystyle f $ tal que $\displaystyle L\oplus L' = V $.
\end{ftheorem}
\begin{proof}
	Consideremos $\displaystyle f^{*} $, que es nilpotente de índice de nilpotencia $\displaystyle m $. Esto se deduce de que $\displaystyle \left(f^{*}\right)^{i} = \left(f^{i}\right)^{*} $. Dado que $\displaystyle f^{m -1} \neq 0 $, para algún $\displaystyle \vec{x} \in V $ se tiene que $\displaystyle f^{ m -1}\left(\vec{x}\right) \neq \vec{0} $. De manera similar, deducimos que existe $\displaystyle \alpha \in V^{*} $ tal que $\displaystyle \alpha\left(f^{m -1}\left(\vec{x}\right)\right) \neq \vec{0} $. Así, $\displaystyle \left\{ \alpha, f^{*}\left(\alpha \right), \ldots, \left(f^{*}\right)^{m-1}\left(\alpha\right)\right\}  $ son linealmente independientes.
Sea $\displaystyle a_{0}, \ldots, a_{m -1} \in \K $ tales que
\[a_{0}\alpha + a_{1}f^{*}\left(\alpha \right) +\cdots+a_{m -1}\left(f^{*}\right)^{m -1}\left(\alpha \right) = 0 .\]
Entonces, tenemos que 
\[
\begin{split}
	0 =&  \left(a_{0}\alpha + a_{1}f^{*}\left(\alpha \right) +\cdots+a_{m -1}\left(f^{*}\right)^{m -1}\left(\alpha \right)\right)\left(f^{m -1}\left(\vec{x}\right)\right) \\
 = &  a_{0} \alpha\left(f^{m -1}\left(\vec{x}\right)\right) + a_{1}\alpha\left(f^{m}\left(\vec{x}\right)\right) + \cdots + a_{m -1}\alpha\left(f^{2m -2}\left(\vec{x}\right)\right) \Rightarrow a_{0} = 0.
\end{split}
\]
Supongamos que $\displaystyle a_{0} = a_{1} = \cdots = a_{i-1} = 0 $ para $\displaystyle i < m $. Entonces, tenemos que 
\[ a_{i}\left(f^{*}\right)^{i}\left(\alpha \right) + \cdots + a_{m -1}\left(f^{*}\right)^{m -1}\left(\alpha \right)= 0 .\]
Componiendo, obtenemos que
\[
\begin{split}
	0 =& \left(a_{i}\left(f^{*}\right)^{i}\alpha \left(f^{m - i-1}\left(\vec{x}\right)\right) + \cdots + a_{m -1}\left(f^{*}\right)^{ m -1}\alpha \left(f^{m -i -1}\left(\vec{x}\right)\right)\right) \\
= & a_{i}\alpha\left(f^{m -1}\left(\vec{x}\right)\right) + \cdots + a_{m -1}\alpha \left(f^{2m - i -2}\left(\vec{x}\right)\right) \Rightarrow a_{i} = 0.
\end{split}
\]
Sea $\displaystyle W = L\left( \left\{ \alpha, f^{*}\left(\alpha\right), \cdots , \left(f^{*}\right)^{m -1}\left(\alpha \right)\right\} \right) $. Tenems que $\displaystyle \dim\left(W\right) = m $. Sea $\displaystyle L' = W^{\perp } $, entonces $\displaystyle \dim\left(L'\right) = \dim\left(W^{\perp }\right) = n - m $. Si $\displaystyle \vec{y} \in L \cap L' $, entonces
\[\vec{y} = a_{0}\vec{x} + a_{1}f\left(\vec{x}\right) + \cdots + a_{m -1}f^{m -1}\left(\vec{x}\right) ,\]
y también tenemos que
\[ \vec{0} = \left(f^{*}\right)^{m -1}\left(\alpha \right)\left(\vec{y}\right) = \alpha \left(a_{0}f^{m -1}\left(\vec{x}\right) + a_{1}f^{m}\left(\vec{x}\right) + \cdots + a_{m -1}f^{2m-2}\left(\vec{x}\right)\right) = a_{0}\alpha\left(f^{m -1}\right)\left(\vec{x}\right) \Rightarrow a_{0} = 0.\]
Supongamos que $\displaystyle a_{0}= a_{1} = \cdots = a_{i-1} = 0 $ con $\displaystyle i < m $. Así, tenemos que
\[
\begin{split}
 & a_{i}f^{i}\left(\vec{x}\right) + \cdots + a_{m -1}f^{m -1}\left(\vec{x}\right) = \vec{0} \\
& \vec{y} = a_{i}f^{i}\left(\vec{x}\right) + \cdots + a_{m -1}f^{m -1}\left(\vec{x}\right) 
\end{split}
\]
\[
\begin{split}
	\vec{0} = \left(f^{*}\right)^{m -i-1}\left(\alpha \right)\left(a_{i}f^{i}\left(\vec{x}\right) + \cdots + a_{m -1}f^{m -1}\left(\vec{x}\right)\right) = \alpha\left(a_{i}f^{m -1}\left(\vec{x}\right)+a_{m -1}f^{2m -i-1}\left(\vec{x}\right)\right) \Rightarrow a_{i} = 0.
\end{split}
\]
Así, tenemos que $\displaystyle a_{i} = 0 $ para $\displaystyle i = 0, 1, \ldots, m -1 $, por lo que $\displaystyle V = L \oplus L'$.
\end{proof}
\begin{observation}
\normalfont Supongamos que $\displaystyle P_{cf}\left(x\right) = \left(-1\right)^{n}\left(x -\lambda_{1}\right)^{r_{1}} \cdots \left(x - \lambda_{k}\right)^{r_{k}} $. Entonces, tenemos que $\displaystyle V = \Ker\left(f -\lambda_{1}id _{V}\right)^{r_{1}} \oplus \cdots \oplus \Ker\left(f - \lambda_{k}id _{V}\right)^{r_{k}} $. Tenemos que $\displaystyle \left(f -\lambda_{i}id _{V}\right)^{r_{i}} $ es nilpotente con índice de nilpotencia menor o igual que $\displaystyle r_{i} $.
\end{observation}
Consideremos $\displaystyle f \in \End\left(f\right) $ tal que $\displaystyle f^{m} = 0 $ pero $\displaystyle f^{ m -1}\neq 0 $. Sea $\displaystyle \Ker\left(f^{m -1}\right)\subsetneq \Ker\left(f^{m}\right) $, sea $\displaystyle L \in \mathcal{L}\left(V\right) $ tal que $\displaystyle L \oplus \Ker\left(f^{m -1}\right) = \Ker\left(f^{m}\right) $ y sea $\displaystyle \left\{ \vec{x}_{1}, \ldots, \vec{x}_{r}\right\}  $ base de $\displaystyle L $. Entonces, la familia de vectores
\[ \left\{ \vec{x}_{1}, \ldots, \vec{x}_{r}, f\left(\vec{x}_{1}\right), \ldots, f\left(\vec{x}_{r}\right), \ldots, f^{m-1}\left(\vec{x}_{1}\right), \ldots, f^{m -1}f\left(\vec{x}_{r}\right)\right\}  \]
es linealmente independiente. En efecto, si $\displaystyle c^{i}_{j} \in \K $, para $\displaystyle i = 1, \ldots, r $ y $\displaystyle j = 0, \ldots, m -1 $, tenemos la combinación lineal
\[\sum^{r, m -1}_{i = 1, j=0}c^{i}_{j}f^{j}\left(\vec{x}_{i}\right) = \vec{0} .\]
Tenemos que si aplicamos $\displaystyle f^{m -1} $,
\[
\begin{split}
	\vec{0} = f^{m -1}\left(\sum^{r, m -1}_{i = 1, j=0}c^{i}_{j}f^{j}\left(\vec{x}_{i}\right)\right) =& \sum^{r, m -1}_{i = 0, j = 0}c^{i}_{j}f^{j+m -1}\left(\vec{x}_{i}\right)=\sum^{r}_{i = 1}c^{i}_{1}f^{m -1}\left(\vec{x}_{i}\right) = c^{1}_{1}f^{m-1}\left(\vec{x}_{1}\right) + \cdots + c^{r}_{1}f^{m -1}\left(\vec{x}_{r}\right) \\
	= & f^{m -1}\left(c^{1}_{1}\vec{x}_{1}+ \cdots + c^{r}_{1}\vec{x}_{r}\right).
\end{split}
\]
Entonces, tenemos que $\displaystyle c^{1}_{1}\vec{x}_{1}+ \cdots + c^{r}_{1}\vec{x}_{r} \in L \cap \Ker\left(f^{m -1}\right) $, por lo que
\[ c^{1}_{1}\vec{x}_{1}+ \cdots + c^{r}_{1}\vec{x}_{r} = \vec{0} .\]
Al ser linealmente independientes, tenemos que $\displaystyle c^{i}_{1} =0 $ para $\displaystyle i = 1, \ldots, r $. Repitiendo este cálculo, llegamos a que $\displaystyle c^{i}_{j} = 0 $ para $\displaystyle i = 1, \ldots, r $ y $\displaystyle j = 0, \ldots, m -1 $. \\ \\
A continuación, tenemos que $\displaystyle \Ker\left(f^{m -2}\right) \subsetneq \Ker\left(f^{m -1}\right) $, ampliamos la base de $\displaystyle \Ker\left(f^{m -2}\right) $ hasta obtener una base de $\displaystyle \Ker\left(f^{m -1}\right) $. Hacemos esto con todos los núcleos. 
\begin{observation}
\normalfont El número de bloques de Jordan de orden mayor o igual que $\displaystyle i $ es la dimensión del núcleo de $\displaystyle f^{i} $ menos la dimensión del núcleo de $\displaystyle f^{i-1} $.
\end{observation}
Dado $\displaystyle f \in \End\left(V\right) $, si existe canónica de $\displaystyle f $, tenemos que el polinomio característico tendrá la forma: $\displaystyle P_{cf}\left(x\right) = \left(x-\lambda_{1}\right)^{r_{1}} \cdots \left(x - \lambda_{k}\right)^{r_{k}} $. Entonces, tenemos que $\displaystyle V = L_{1} \oplus \cdots \oplus L_{k} $, donde $\displaystyle \dim\left(L_{i}\right) = \Ker\left(f-\lambda_{i}id _{V}\right)^{r_{i}} $ con $\displaystyle i =1, \ldots, k $. Así, si $\displaystyle \dim\left(L_{i}\right) = r_{i}$,
\[P_{cf}\left(x\right) = \left(x - \lambda_{1}\right)^{s_{1}} \cdots \left(x - \lambda_{k}\right)^{s_{k}}, \; s_{i} \leq r_{i} .\]
tenemos que $\displaystyle f|_{L_{i}} \in \End\left(L_{i}\right) $, $\displaystyle f -\lambda_{i}id _{V} $ es nilpotente de límite de nilpotencia $\displaystyle s_{i} $. Además, tenemos que $\displaystyle \Ker\left(f - \lambda_{i} id _{V}\right) \subsetneq \cdots \subsetneq \Ker\left(f - \lambda_{i} id _{V}\right)^{s _{i}-1} = \Ker\left(f - \lambda_{i}id _{V}\right)^{s_{i}} $.
\begin{eg}
	\normalfont Consideremos la matriz $\displaystyle A = \begin{pmatrix} 4 & 5 & - 2 \\ - 2 & - 2 & 1 \\ - 1 & - 1 & 1 \end{pmatrix} $. Tenemos que $\displaystyle P_{cA}\left(x\right) = - \left(x-1\right)^{3} $. Calculamos la dimensión de $\displaystyle \Ker\left(f - id _{V}\right) $,
	\[A - I = \begin{pmatrix} 3 & 5 & -2 \\ -2 & - 3 & 1 \\ - 1 & - 1 & 0 \end{pmatrix}, \; \left(A - I\right)^{2} = \begin{pmatrix} 1 & 2 & -1 \\ -1 & -2 & 1 \\ - 1 & - 2 & 1 \end{pmatrix} .\]
Entonces, tenemos que $\displaystyle \dim\left(\Ker\left(f - id _{V}\right)\right) = 1 $ y $\displaystyle \dim\left(\Ker\left(f - id _{V}\right)^{2}\right) = 2 $. Así, debe ser que $\displaystyle \left(A - I\right)^{3} = 0 $. Tenemos que la ecuación de $\displaystyle \Ker\left(f - id _{V}\right)^{2} $ será $\displaystyle x +2y - z = 0 $. Tenemos que coger un vector en $\displaystyle L_{2} $ que no esté en $\displaystyle L_{1} $. Entonces, cogemos cogemos $\displaystyle \vec{u}_{1} = \left(1,0,0\right) $ y $\displaystyle \vec{u}_{2} = \left(3, -2, -1\right) = \left(f - id _{V}\right)\left(1,0,0\right) $. 
Para conseguir $\displaystyle \vec{u}_{3} $:
	\[ \left(A - I\right)^{2} \begin{pmatrix} 1 \\ 0 \\ 0 \end{pmatrix} = \begin{pmatrix} 1 \\ - 1 \\ 1 \end{pmatrix} .\]
Así, $\displaystyle \vec{u}_{3} = \left(1, - 1, - 1\right) $. Entonces, tenemos que la matriz cambio de base será
	\[ C = \begin{pmatrix} 1 & 3 & 1 \\ 0 & - 2& - 1 \\ 0 & - 1 & -1 \end{pmatrix} .\]
Y la forma de Jordan será
	\[ C^{-1}AC = \begin{pmatrix} 1 & 0 & 0 \\ 1 & 1 & 0 \\ 0 & 1 & 1 \end{pmatrix} .\]
\end{eg}
\begin{eg}
	\normalfont Consideremos $\displaystyle B = \begin{pmatrix} 1 & 1 & 1 \\ - 1 & 0 & - 1 \\ 1 & - 1 & 0 \end{pmatrix} $. Entonces, $\displaystyle P_{cB} = -\left(x-1\right)^{2}\left(x+1\right) $. Así, $\displaystyle V = \Ker\left(f - id _{V}\right)^{2} \oplus \Ker\left(f + id _{V}\right) $. Vamos a calcular la dimensión de $\displaystyle L_{1} $:
	\[\left(B - I\right)^{2} = \begin{pmatrix} 0 & - 2 & - 2 \\ 0 & 1 & 1 \\ 0 & 3 & 3 \end{pmatrix} .\]
Así, $\displaystyle \dim\left(L_{1}\right) = 2 $ y $\displaystyle \dim\left(L_{2}\right) = 1 $. Ahora buscamos un vector que esté en $\displaystyle L_{1} $ pero no en $\displaystyle L_{2} $. Obtenemos las ecuaciones $\displaystyle y + z = 0 $ y $\displaystyle x = 0 $. Entonces, cogemos $\displaystyle \vec{u}_{1} = \left(1, 0, 0\right) $ y $\displaystyle \vec{u}_{2} = \left(0, -1, 1\right) $.  
\end{eg}

