\chapter{Espacios Vectoriales}

Consideramos un cuerpo conmutativo con característica distinta de 2, es decir, $\displaystyle 1 + 1 \neq 0 $. A este cuerpo lo llamaremos $\displaystyle \K $. 

\begin{fdefinition}[ $\displaystyle \K $-Espacio vectorial]
\normalfont Un conjunto $\displaystyle V \neq \emptyset $ es un $\displaystyle \K $-\textbf{espacio vectorial} si se tienen definidas dos aplicaciones 
\[
\begin{split}
& + : V \times V \to V \\ 
& \left(\vec{x}, \vec{y}\right) \to \vec{x} + \vec{y} \\
& \cdot : \K \times V \to V \\
& \left(a, \vec{x}\right) \to a \cdot \vec{x},
\end{split}
\]
tales que verifican que
\begin{description}
\item[(1)] $\displaystyle \left(V, +\right) $ es un cuerpo abeliano.
	\subitem[Commutatividad.] $\displaystyle \forall \vec{x}, \vec{y}\in V, \; \vec{x} + \vec{y} = \vec{y} + \vec{x}$.
	\subitem[Asociatividad.] $\displaystyle \forall \vec{x}, \vec{y}, \vec{z} \in V, \; \left(\vec{x} + \vec{y}\right) + \vec{z} = \vec{x} + \left(\vec{y} + \vec{z}\right). $ 
	\subitem[Existencia del elemento neutro.] $\displaystyle \exists \vec{0} \in V, \forall \vec{x} \in V, \; \vec{0} + \vec{x} = \vec{x} $.
	\subitem[Existencia del opuesto.] $\displaystyle \forall \vec{x}\in V, \exists -\vec{x} \in V, \; \vec{x} + \left(-\vec{x}\right) = \vec{0}. $ 
	\footnote{En la propiedad del elemento neutro y del opuesto, como la conmutatividad es un requisito no hay que especificar que el elemento neutro funciona por ambos lados, al igual que el opuesto.} 
\item[(2)] $\displaystyle \forall \vec{x}, \vec{y} \in V, \forall a \in \K, \; a \cdot \left(\vec{x} + \vec{y}\right) = a \cdot \vec{x} + a \cdot \vec{y}. $ 
\item[(3)] $\displaystyle \forall \vec{x} \in V, \forall a,b \in \K, \; \left(a + b\right) \cdot \vec{x} = a \cdot \vec{x} + b \cdot \vec{x}. $ 
\item[(4)] $\displaystyle \forall \vec{x} \in V, \forall a,b \in \K, \; \left(a \cdot b\right) \cdot \vec{x} = a \cdot \left(b \cdot \vec{x}\right). $ 
\item[(5)] $\displaystyle \forall \vec{x}\in V, \; 1 \cdot \vec{x}=\vec{x}. $ 
\end{description}
\end{fdefinition}

Si considero a $\displaystyle \R $ como un cuerpo, tenemos que $\displaystyle \C $ es un $\displaystyle \R $-espacio vectorial (dimensión 2) y un $\displaystyle \C$-espacio vectorial (dimensión 1). 

\begin{ftheorem}[]
\normalfont Sea $\displaystyle V $  un $\displaystyle \K $-espacio vectorial, entonces se verifica que:
\begin{description}
\item[(a)] $\displaystyle \forall \vec{x} \in V, \; 0 \cdot \vec{x} = \vec{0}. $ 
\item[(b)] $\displaystyle \forall \vec{x} \in V, \forall a \in \K, \; \left(-a\right) \cdot \vec{x} = - a \cdot \vec{x}. $ 
\item[(c)] $\displaystyle \forall a \in \K, \; a \cdot \vec{0} = \vec{0}. $ 
\item[(d)] $\displaystyle a \cdot \vec{x} = \vec{0} \Rightarrow \vec{x}=\vec{0} \lor a = 0 $.
\end{description}
\end{ftheorem}

\begin{proof}
\begin{description}
\item[(a)]
	\[
	\begin{split}
	& \vec{x} = \left(1 + 0\right) \cdot \vec{x} = 1 \cdot \vec{x} + 0 \cdot \vec{x} = \vec{x} + 0 \cdot \vec{x} \\
	\iff & - \vec{x} + \vec{x} = - \vec{x} + \vec{x} + 0 \cdot \vec{x} \\
	\iff & 0 = 0 \cdot \vec{x}.
	\end{split}
	\]
Se puede hacer de otra manera:
\[0 \cdot \vec{x} = \left(0 + 0\right) \cdot \vec{x} = 0 \cdot \vec{x} + 0 \cdot \vec{x} \iff 0 = 0 \cdot \vec{x} .\]
\item[(b)] 
	\[
	\begin{split}
	& 0 \cdot \vec{x} = \left(a + \left( - a\right)\right) \cdot \vec{x} = a \cdot \vec{x} + \left(-a\right) \cdot \vec{x} \\
	\iff & - a \cdot \vec{x} = - a \cdot \vec{x} + a \cdot \vec{ x} + \left(- a \right) \cdot \vec{x} \\
	\iff & - a \cdot \vec{x} = \left(- a \right) \cdot x.
	\end{split}
	\]
\item[(c)] 
	\[ a \cdot \vec{x} = a \cdot \left(\vec{x} + \vec{0}\right) = a \cdot \vec{x} + a \cdot \vec{0} \iff - a \cdot \vec{x} + a \cdot \vec{x} = - a \cdot \vec{x} + a \cdot \vec{x} + a \cdot \vec{0} .\]
	\[\therefore \vec{0} = a \cdot \vec{0} .\]
También se puede hacer de la siguiente manera:
\[a \cdot \vec{0} = a \cdot \left(\vec{0} + \vec{0}\right) = a \cdot \vec{0} + a \cdot \vec{0} \iff0 = a \cdot \vec{0} .\]
\item[(d)] Si $\displaystyle a = 0 $, hemos ganado. Si $\displaystyle a \neq 0 $, $\displaystyle \exists \frac{1}{a} \in \K $. Por tanto, 
	\[\vec{0} = \frac{1}{a} \cdot \vec{0} = \frac{1}{a} \cdot \left(a \cdot \vec{x}\right) = \left(\frac{1}{a} \cdot a\right) \cdot \vec{x} = \vec{x} .\]
\end{description}
\end{proof}

\section{Subespacios vectoriales}

\begin{fdefinition}[Subespacio vectorial]
\normalfont Un conjunto $\displaystyle L \neq \emptyset $ y $\displaystyle L \subset V $ es \textbf{parte estable} si 
\begin{description}
\item[(i)] $\displaystyle \forall \vec{x}, \vec{y} \in L, \; \vec{x} + \vec{y} \in L $.
\item[(ii)] $\displaystyle \forall a \in \K, \forall \vec{x} \in V, \; a \cdot \vec{x} \in L. $ 
\end{description}
\end{fdefinition}

\begin{ftheorem}[]
\normalfont Sea $\displaystyle L \neq \emptyset $ y $\displaystyle L \subset V $, entonces $\displaystyle L $  es parte estable si y sólo si $\displaystyle L $ es subespacio vectorial.
\end{ftheorem}

\begin{proof}
\begin{description}
\item[(i)] Si $\displaystyle L $ es un subespacio vectorial es trivial.
\item[(ii)] Si $\displaystyle L $ es parte estable, tenemos que para $\displaystyle \vec{x} \in L $ se verifica la propiedad conmutativa, asociativa, etc, dado que $\displaystyle L \subset V $. Además, dado que $\displaystyle \cdot : \K \cdot L \to L $ y $\displaystyle \left(-1\right) \cdot \vec{x} = - \vec{x} $, tenemos que si $\displaystyle \vec{x} \in L $ entonces $\displaystyle - \vec{x} \in L $. Además, $\displaystyle \vec{x} + \left(- \vec{x}\right) = \vec{0} \in L $. El resto de propiedades se derivan de que $\displaystyle L \subset V $.
\end{description}
\end{proof}

\begin{fdefinition}[Combinación lineal]
\normalfont $\displaystyle \vec{x} \in V $ es la \textbf{combinación lineal} de $\displaystyle \vec{x}_{1}, \vec{x}_{2}, \ldots, \vec{x}_{p} $ con coeficientes $\displaystyle a^{1}, a^{2}, \ldots, a^{p} $ si existen $\displaystyle \vec{x}_{i} \in V $ y $\displaystyle a ^{i} \in \K $, con $\displaystyle p \in \N $ ($\displaystyle 1 \leq i \leq p $) tales que:
\[\vec{x} = a^{1} \cdot \vec{x}_{1} + a^{2} \cdot \vec{x}_{2} + \cdots + a ^{p} \cdot \vec{x}_{p} .\]
\footnote{Los $\displaystyle a^{i} $ no denotan exponente sino que se trata de una forma de numeración.} 
\end{fdefinition}

\textbf{Nota.} Podemos apreciar que, dadas las condiciones del subespacio vectorial, cualquier combinación lineal de vectores $\displaystyle \vec{x}_{1}, \vec{x}_{2}, \ldots, \vec{x}_{p}\in L $ es un vector de $\displaystyle L $.

\begin{ftheorem}[]
\normalfont Sea $\displaystyle H \subset V $ con $\displaystyle H \neq \emptyset $. Definimos $\displaystyle L\left(H\right) $  como el conjunto de todas las combinaciones lineales finitas de $\displaystyle H $, es decir:
\[L\left(H\right) = \left\{  a^{1}\vec{x}_{1} + a^{2}\vec{x}_{2} + \cdots + a^{p}\vec{x}_{p}\; : \; p \in \N, \vec{x}_{i}\in H, a^{i} \in \K\right\}  .\]
Se verifica que 
\begin{description}
\item[(1)] $\displaystyle H \subset L\left(H\right). $ 
\item[(2)] $\displaystyle L\left(H\right) $ es un espacio vectorial sobre $\displaystyle \K $. 
\item[(3)] $\displaystyle L\left(H\right) $ es el menor subespacio vectorial que contiene a $\displaystyle H $. Es decir, si $\displaystyle L $ es un subespacio vectorial y $\displaystyle H \subset L $, entonces $\displaystyle L\left(H\right) \subset L $.
\end{description}
\end{ftheorem}

\begin{proof}
\begin{description}
\item[(1)] Tenemos que si $\displaystyle \vec{x} \in H $ entonces
	\[ \vec{x} = \underbrace{1 \cdot \vec{x}}_{\text{combinación lineal}} \in L\left(H\right) .\]
\item[(2)] Sean $\displaystyle \vec{x}, \vec{y} \in L\left(H\right) $, queremos ver que $\displaystyle \vec{x} + \vec{y} \in L\left(H\right) $. Dado que $\displaystyle \vec{x}, \vec{y} \in L\left(H\right) $, se pueden expresar como combinación lineal de otros vectores en $\displaystyle H $. 
	\[\exists p \in \N,\; \exists \vec{x}_{1}, \vec{x}_{2}, \ldots, \vec{x}_{p} \in H, \;\exists a^{1}, a ^{2}, \ldots, a^{p} \in \K ,\]
	tales que 
	\[\vec{x} = a^{1}\vec{x}_{1} + a ^{2}\vec{x}_{2} + \cdots + a^{p}\vec{x}_{p} .\]
De manera similar, como $\displaystyle \vec{y}\in L\left(H\right) $, 
	\[\exists q \in \N,\; \exists \vec{y}_{1}, \vec{y}_{2}, \ldots, \vec{y}_{q} \in H,\; \exists b^{1}, b ^{2}, \ldots, b^{p} \in \K ,\]
	tales que
	\[\vec{y} = b^{1}\vec{y}_{1} + b^{2}\vec{y}_{2} + \cdots + b^{q}\vec{y}_{q} .\]
Entonces, 
\[\vec{x} + \vec{y} = \left(a^{1}\vec{x}_{1} + a ^{2}\vec{x}_{2} + \cdots + a^{p}\vec{x}_{p}\right)+ \left(b^{1}\vec{y}_{1} + b^{2}\vec{y}_{2} + \cdots + b^{q}\vec{y}_{q} \right) .\]
Como $\displaystyle \forall \vec{x}_{i}, \vec{y}_{i} \in H $, tenemos que $\displaystyle \vec{x} + \vec{y} \in L\left(H\right) $. \\ \\
A continuación, demostramos que si $\displaystyle \vec{x} \in L\left(H\right) $ entonces $\displaystyle a \cdot \vec{x} \in L\left(H\right) $. Como $\displaystyle \vec{x} \in L\left(H\right) $, 
\[\exists p \in \N,\; \exists \vec{x}_{1}, \vec{x}_{2}, \ldots, \vec{x}_{p} \in H, \;\exists a^{1}, a ^{2}, \ldots, a^{p} \in \K ,\]
tales que 
\[\vec{x} = a^{1}\vec{x}_{1} + a ^{2}\vec{x}_{2} + \cdots + a^{p}\vec{x}_{p} .\]
Por tanto, 
\[
\begin{split}
	a \cdot \vec{x} = & a \cdot \left(a^{1}\vec{x}_{1} + a ^{2}\vec{x}_{2} + \cdots + a^{p}\vec{x}_{p}\right) \\
	= & a \cdot \left(a^{1}\vec{x}_{1}\right) + a \cdot \left(a^{2}\vec{x}_{2}\right) + \cdots + a \cdot \left(a^{p}\vec{x}_{p}\right) \\
	= & \left(a \cdot a^{1}\right) \cdot \vec{x}_{1} + \left(a \cdot a^{2}\right) \cdot \vec{x}_{2} + \cdots + \left(a \cdot a ^{p}\right) \cdot \vec{x}_{p}  .
\end{split}
\]
Aprovechamos las propiedades de $\displaystyle V $ como espacio vectorial y el hecho de que $\displaystyle H \subset V $ (hemos utilizado la propiedad distributiva). Como $\displaystyle \forall a \cdot a^{i}\in\K $ y $\displaystyle \vec{x}_{i} \in H $, $\displaystyle a \cdot \vec{x} $ se trata de una combinación lineal y, por tanto, $\displaystyle a \cdot \vec{x} \in L\left(H\right) $.
\item[(3)] Si $\displaystyle \vec{x} \in L\left(H\right) $, tenemos que $\displaystyle \exists p \in \N, \; \exists \vec{x}_{i}\in H, \; \exists a ^{i} \in \K $ con $\displaystyle 1 \leq i \leq p $, tales que 
	\[ \vec{x} = a^{1}\vec{x}_{1}+a^{2}\vec{x}_{2} + \cdots + a^{p}\vec{x}_{p} .\]
Como $\displaystyle H \subset L $, $\displaystyle x_{i} \in L $ y, como $\displaystyle L $ es un subespacio vectorial, tenemos que $\displaystyle  a^{1}\vec{x}_{1}+a^{2}\vec{x}_{2} + \cdots + a^{p}\vec{x}_{p} \in L  $, por lo que $\displaystyle \vec{x}\in L $.
\end{description}
\end{proof}

\begin{fdefinition}[]
\normalfont $\displaystyle L\left(H\right) $ es el subespacio generado por $\displaystyle H $ o $\displaystyle H $ es un sistema de generadores de de $\displaystyle L\left(H\right) $. Si $\displaystyle L\left(H\right) = V $ diremos que $\displaystyle H $ es sistema de generadores.
\end{fdefinition}

\section{Bases de un espacio vectorial}

\begin{fdefinition}[]
\normalfont $\displaystyle V $ es \textbf{finito generado} si existe un sistema de generadores formado por un número finito de vectores. Es decir, si $\displaystyle \exists \left\{ \vec{x}_{1}, \vec{x}_{2}, \ldots, \vec{x}_{p}\right\} = H $ tal que $\displaystyle V = L\left(H\right) $, es decir, $\displaystyle \exists \{\vec{x}_{1}, \vec{x}_{2}, \ldots, \vec{x}_{p}\} \subset V $ tales que $\displaystyle \forall \vec{x}\in V, \exists a^{1}, a^{2}, \ldots, a^{p} \in \K $ tales que
\[\vec{x} = a^{1}\vec{x}_{1} + a^{2}\vec{x}_{2} + \cdots + a^{p}\vec{x}_{p} .\]
\end{fdefinition}

\begin{fdefinition}[]
\normalfont Una familia de vectores  $\displaystyle\left\{ \vec{x}_{1}, \vec{x}_{2}, \ldots, \vec{x}_{p}\right\} $ es \textbf{linealmente dependiente} si uno de ellos es combinación lineal de los otros. 
\[\exists i = 1, 2, \ldots, p, \; \vec{x}_{L} \in L\left(\left\{ \vec{x}_{1}, \vec{x}_{2}, \ldots, \vec{x}_{p}\right\}\right)  .\]
Es decir, 
\[
\begin{split}
& \exists i = 1, 2, \ldots, p,\; \exists a^{1}, a^{2}, \ldots, a^{i-1}, a^{i+1}, \ldots, a^{p} \in \K \\
& \vec{x}_{i} = a^{1}\vec{x}_{1} + \cdots + a^{i-1}\vec{x}_{i-1} + a^{i+1}\vec{x}_{i+1} + \cdots + a^{p}\vec{x}_{p}..
\end{split}
\]

Es decir, si $\displaystyle \vec{x}_{j} = \vec{0} $, $\displaystyle \left\{ \vec{x}_{1}, \vec{x}_{2}, \ldots, \vec{x}_{p}\right\} $ es dependiente, pues 
\[\vec{0} = 0 \cdot \vec{x}_{1} + \cdots + 0 \cdot \vec{x}_{j-1} + 0 \cdot \vec{x}_{j+1} + \cdots + 0 \cdot \vec{x}_{p}  .\]
\end{fdefinition}

\begin{ftheorem}[]
\normalfont Sea $\displaystyle \left\{ \vec{x}_{1}, \vec{x}_{2}, \ldots, \vec{x}_{p}\right\} \subset V $. Son linealmente dependientes si y solo si $\displaystyle \exists a^{1}, a^{2}, \ldots, a^{p} \in \K $ no todos nulos tales que 
\[\vec{0} = a^{1}\vec{x}_{1} + a^{2}\vec{x}_{2} + \cdots + a^{p}\vec{x}_{p} .\]
\end{ftheorem}

\begin{proof}
\begin{description}
\item[(i)] Supongamos que la familia es linealmente dependiente. Por tanto, $\displaystyle \exists i = 1, \ldots, p $ tal que $\displaystyle \vec{x}_{i} \in L\left(\left\{ \vec{x}_{1}, \vec{x}_{2}, \ldots, \vec{x}_{i-1}, \vec{x}_{i+1}, \ldots, \vec{x}_{p}\right\}\right) $. Por tanto, existen $\displaystyle a^{i}\in\K $ tales que 
	\[\vec{x}_{i} = a^{1}\vec{x}_{1} + \cdots + a^{i-1}\vec{x}_{i-1} + a^{i+1}\vec{x}_{i+1} + \cdots + a^{p}\vec{x}_{p} .\]
Si sumamos el opuesto a ambos lados tenemos que 
\[ \vec{0} = \vec{x}_{i} - \vec{x}_{i} = a^{1}\vec{x}_{1} + \cdots + a^{i-1}\vec{x}_{i-1} +\left(-1\right)\vec{x}_{1} + a^{i+1}\vec{x}_{i+1} + \cdots + a^{p}\vec{x}_{p} .\]
\item[(ii)] Suponemos que $\displaystyle \exists a^{1}, a^{2}, \ldots, a^{p} \in \K $ no todos nulos tales que 
	\[a^{1}\vec{x}_{1} + a^{2}\vec{x}_{2} + \cdots + a^{p}\vec{x}_{p} = \vec{0} .\]
Como no todos los escalares son nulos, podemos encontrar $\displaystyle a^{i} \neq 0 $, y $\displaystyle a^{i} $ tiene inversa. 
\[\therefore \left(-1\right)a^{i}\vec{x}_{i} = a^{1}\vec{x}_{1} + a^{2}\vec{x}_{2} + \cdots + a^{i-1}\vec{x}_{i-1}+a^{i+1}\vec{x}_{i+1} + \cdots + a^{p}\vec{x}_{p} .\]
Aprovechando las propiedades:
\[
\begin{split}
\vec{x}_{i} & = \frac{-a^{1}}{a^{i}}\vec{x}_{1} + \cdots + \frac{-a^{i-1}}{a^{i}}\vec{x}_{i-1}+  \frac{-a^{i+1}}{a^{i}}\vec{x}_{i+1} + \cdots + \frac{-a^{p}}{a^{i}}\vec{x}_{p}.
\end{split}
\]
\end{description}
\end{proof}

\begin{fcolorary}[]
\normalfont La familia de vectores $\displaystyle \left\{ \vec{x}_{1}, \vec{x}_{2}, \ldots, \vec{x}_{p}\right\} \subset V  $ es \textbf{linealmente independiente} si y solamente si 
\[a^{1}\vec{x}_{1} + a^{2}\vec{x}_{2} + \cdots + a^{p}\vec{x}_{p} = \vec{0} \Rightarrow a^{1} = a^{2}= \cdots = a^{p} = 0 .\]
\end{fcolorary}


\begin{ftheorem}[]
\normalfont Una familia de vectores $\displaystyle \left\{ \vec{x}_{1}, \vec{x}_{2}, \ldots, \vec{x}_{p}\right\}  $ es linealmente independiente si y solamente si $\displaystyle \forall \vec{x} \in L\left(\left\{ \vec{x}_{1}, \vec{x}_{2}, \ldots, \vec{x}_{p}\right\} \right), \exists! a^{1}, a^{2}, \ldots, a^{p} \in \K $ tales que 
\[\vec{x} = a^{1}\vec{x}_{1} + a^{2}\vec{x}_{2} + \cdots + a^{p}\vec{x}_{p} .\]
\end{ftheorem}

\begin{proof} 
\begin{description}
\item[(i)] Si $\displaystyle \left\{ \vec{x}_{1}, \vec{x}_{2}, \ldots, \vec{x}_{p}\right\}  $ es linealmente independiente y sea $\displaystyle \vec{x} \in L\left(\left\{ \vec{x}_{1}, \vec{x}_{2}, \ldots, \vec{x}_{p}\right\} \right) $. Supongamos que existen otros escalares $\displaystyle b^{1}, b^{2}, \ldots, b^{p} \in \K $ tales que 
\[\vec{x} = b^{1}\vec{x}_{1} + \cdots + b^{p}\vec{x}_{p} .\]
Tenemos que 
\[
\begin{split}
\vec{0} = & \vec{x} - \vec{x} = \left(a^{1} \vec{x}_{1} + \cdots a^{p} \vec{x}_{p}\right) - \left(b^{1} + \cdots + b^{p}\vec{x}_{p}\right) \\
= & \left(a^{1} - b^{1}\right)\vec{x}_{1} + \cdots + \left(a^{p} -b^{p}\right)\vec{x}_{p} .
\end{split}
\]
Como se trata de una familia linealmente independiente, tenemos que $\displaystyle \forall i, \; 1\leq i \leq p $, 
\[a^{i} - b^{i} = 0 \iff a^{i} = b^{i} .\]
\item[(ii)] Recíprocamente, tenemos que si $\displaystyle a^{1}, a^{2}, \ldots, a^{p} \in \K $ tales que 
	\[a^{1}\vec{x}_{1} + \cdots + a^{p}\vec{x}_{p}= \vec{0} .\]
Esto puede pasar si $\displaystyle a^{1} = a^{2} = \cdots = a^{p} = 0 $. Como $\displaystyle a^{i} $ son únicos, tenemos que si hay alguno no nulo, la combinación lineal no va a ser nula. Por tanto, será linealmente independiente.
\end{description}
\end{proof}

\begin{fdefinition}[Base]
\normalfont Una \textbf{base} de un espacio vectorial $\displaystyle V $ es un sistema de generadores linealmente independientes.
\end{fdefinition}

\begin{fcolorary}[]
\normalfont Una familia de vectores $\displaystyle B \subset V$ es una base de $\displaystyle E $ si y solo si $\displaystyle \forall \vec{x}\in V $ se expresa de manera única como combinación lineal de elementos de $\displaystyle B $.
\end{fcolorary}

\begin{ftheorem}[]
	\normalfont Si $\displaystyle V \neq \left\{ 0\right\}  $, es finitamente generado, entonces $\displaystyle \exists \left\{ \vec{u}_{1}, \vec{u}_{2}, \ldots, \vec{u}_{n}\right\}  $ es base de $\displaystyle V $. Es decir, todo espacio vectorial $\displaystyle V \neq \left\{ 0\right\}  $ generado por un número finito de vectores tiene una base finita. 
\end{ftheorem}

\begin{proof}
	Sea $\displaystyle \left\{ \vec{x}_{1}, \vec{x}_{2}, \ldots, \vec{x}_{p}\right\}  $ un sistema de generadores de $\displaystyle V $. Si $\displaystyle \left\{ \vec{x}_{1}, \vec{x}_{2}, \ldots, \vec{x}_{p}\right\} $ son linealmente independientes, forman una base (hemos ganado). Sino, uno se puede expresar como combinación lineal de los otros, por lo que $\displaystyle \exists i = 1, 2, \ldots, p $ tal que $\displaystyle \vec{x}_{i} \in L \left(\left\{ \vec{x}_{1}, \vec{x}_{2}, \ldots, \vec{x}_{i-1}, \vec{x}_{i+1}, \ldots, \vec{x}_{p}\right\}\right) $, por lo que $\displaystyle \exists b^{1}, b^{2}, \ldots, b^{p} \in \K $ 
	\[\vec{x}_{i} = b^{1}\vec{x}_{1} + b^{2}\vec{x}_{2} + \cdots + b^{i-1}\vec{x}_{i-1} +b^{i+1}\vec{x}_{i+1} + \cdots + b^{p}\vec{x}_{p}  .\]
Dado que $\displaystyle \left\{ \vec{x}_{1}, \vec{x}_{2}, \ldots, \vec{x}_{p}\right\} $ es un sistema de generadores de $\displaystyle V $, $\displaystyle \forall \vec{x}\in V, \exists a^{1}, a^{2}, \ldots, a^{p} \in \K $ tales que 
\[
\begin{split}
	\vec{x}  = & a^{1}\vec{x}_{1} + a^{2}\vec{x}_{2} + \cdots + a^{p}\vec{x}_{p}  \\
	= & a^{1}\vec{x}_{1} + \cdots + a^{i-1}\vec{x}_{i-1} + a^{i} \left(b^{1}\vec{x}_{1} + b^{2}\vec{x}_{2} + \cdots + b^{i-1}\vec{x}_{i-1} +b^{i+1}\vec{x}_{i+1} + \cdots + b^{p}\vec{x}_{p}\right)+a^{i+1}\vec{x}_{i+1} + \cdots + a^{p}\vec{x}_{p} \\
	= & \left(a^{1} + a^{i}b^{1}\right) \vec{x}_{1} + \cdots + \left(a^{i-1} + a^{i}b^{i-1}\right) \vec{x}_{i-1} + \left(a^{i}b^{i+1}+a^{i+1}\right)\vec{x}_{i+1}+\cdots+\left(a^{i}b^{p}+a^{p}\right)\vec{x}_{p}.
\end{split}
\]
Por tanto, el conjunto $\displaystyle \left\{ \vec{x}_{1}, \vec{x}_{2}, \ldots, \vec{x}_{p}\right\}- \left\{ x_{i}\right\}  $ también es un sistema de generadores de $\displaystyle V $. Si $\displaystyle \left\{ \vec{x}_{1}, \vec{x}_{2}, \ldots, \vec{x}_{i-1}, \vec{x}_{i+1}, \vec{x}_{p}\right\}  $ es linealmente independiente, es un sistema de generadores de $\displaystyle V $. Si es linealmente dependiente repetimos el proceso hasta tener $\displaystyle \left\{ \vec{x}_{i}\right\}  $, que no puede ser 0, porque $\displaystyle V \neq 0 $, y $\displaystyle \left\{ x_{i}\right\}  $ es linealmente independiente.
\end{proof}

\textbf{Observación.} De esto podemos concluir que todo sistema de generadores contiene una base.

\begin{ftheorem}[Teorema de Steinitz]
\normalfont Sea $\displaystyle \left\{ \vec{y}_{1}, \vec{y}_{2}, \ldots, \vec{y}_{p}\right\} \subset V $ una base de $\displaystyle V $ y sea $\displaystyle \left\{ \vec{x}_{1}, \vec{x}_{2}, \ldots, \vec{x}_{q}\right\} \subset V $ linealmente independiente, entonces $\displaystyle q \leq p $ y se puede obtener una nueva base sustituyendo $\displaystyle q $ de los vectores $\displaystyle \vec{y}_{i} $ por $\displaystyle \left\{ \vec{x}_{1}, \vec{x}_{2}, \ldots, \vec{x}_{p}\right\}  $.
\end{ftheorem}

\begin{proof}
	Se trata de introducir uno por uno los vectores $\displaystyle \left\{ \vec{y}_{1}, \ldots, \vec{y}_{p}\right\}  $ por los vectores de la base dada. 
Sea $\displaystyle \vec{x}_{1} \in V $, entonces $\displaystyle \exists a^{1}, a^{2}, \ldots, a^{p}\in \K $ tales que
\[\vec{x}_{1} = a^{1}\vec{y}_{1} + a^{2}\vec{y}_{2} + \cdots + a^{p}\vec{y}_{p} = \sum^{p}_{i=1}a^{i}\vec{y}_{i} .\]
Existe al menos un $\displaystyle a^{i} \neq 0 $ (porque $\displaystyle \vec{x}_{1} $ no es nulo). Sea $\displaystyle a^{1} \neq 0 $. 
\[
\begin{split}
	\vec{y}_{1} = \left(a^{1}\right)^{-1}\vec{x}_{1} - \sum^{p}_{i=2}\left(a^{1}\right)^{-1}a^{i}\vec{y}_{i}
\end{split}
\]
Entonces, $\displaystyle \forall x \in V, \exists b^{1}, b^{2}, \ldots, b^{p}\in\K $, 
\[
\begin{split}
	\vec{x} = & b^{1}\vec{y}_{1} + b^{2}\vec{y}_{2} + \cdots b^{p}\vec{y}_{p} \\
	= & b^{1} \left(\frac{1}{a^{1}}\vec{x}_{1} + \left(-\frac{a^{2}}{a^{1}}\right)\vec{y}_{2} + \cdots + \left(- \frac{a^{p}}{a^{1}}\right)\vec{y}_{p}\right) + b^{2}\vec{y}_{2} + \cdots + b^{p}\vec{y}_{p} \\
	= & \frac{b^{1}}{a^{1}} \vec{x}_{1} + \left(b^{1}\left(-\frac{a^{2}}{b^{1}}\right)+b^{2}\right)\vec{y}_{2} + \cdots + \left(b^{1}\left(-\frac{a^{p}}{a^{1}}\right)+b^{p}\right)\vec{y}_{p}.
\end{split}
\]
Hemos llegado a la conclusión de que $\displaystyle \left\{ \vec{x}_{1}, \vec{y}_{2}, \ldots, \vec{y}_{p}\right\}  $ forman un sistema de generadores de $\displaystyle V $. Además, son linealmente independientes, pues 
\[\vec{0} = b^{1}\vec{x}_{1} + b^{2}\vec{y}_{2} + \cdots + b^{p}\vec{y}_{p} \Rightarrow b^{1}\left(\sum^{p}_{i=1}a^{i}\vec{y}_{i}\right) + b^{2}\vec{y}_{2} + \cdots + b^{p}\vec{y}_{p} = \vec{0} \]
\[ b^{1}a^{1}\vec{y}_{1} + \sum^{p}_{i=2}\left(b^{1}a^{i} + b^{i}\right)\vec{y}_{i} = \vec{0}  \]
\[\Rightarrow b^{1}a^{1} = 0, \; b^{1}a^{i}+b^{i}=0, \; i \geq 2 .\]
pues $\displaystyle \left\{ \vec{y}_{1}, \vec{y}_{2}, \ldots, \vec{y}_{p}\right\} $ son una base. Como $\displaystyle a^{1} \neq 0 $, tenemos que $\displaystyle b^{1}=b^{i} =0 $. Por tanto, $\displaystyle \left\{ \vec{x}_{1}, \vec{y}_{2}, \ldots, \vec{y}_{p}\right\} $ es una base de $\displaystyle V $. \\ \\
Supongamos que $\displaystyle i < \min\left(p,q\right) $ y que $\displaystyle \left\{ \vec{x}_{1}, \vec{x}_{2}, \ldots, \vec{x}_{i-1}, \vec{x}_{i}, \vec{y}_{i+1}, \ldots, \vec{y}_{p}\right\}  $ es sistema de generadores. Entonces, $\displaystyle \exists c^{1}, c^{2}, \ldots, c^{i}, d^{i+1}, \ldots, d^{p} \in \K $ tales que 
\[
\begin{split}
	\vec{x}_{i+ 1} = & c^{1}\vec{x}_{1} + c^{2}\vec{x}_{2} + \cdots + c^{i}\vec{x}_{i} + d^{i+1}\vec{y}_{i+1} + \cdots + d^{p}\vec{y}_{p} = \sum^{i}_{j= 1}c^{j}\vec{x}_{j} + \sum^{p}_{j=i+1}d^{j}\vec{y}_{j}.
\end{split}
\]
El procedimiento anterior nos asegura que podemos sustituir $\displaystyle \vec{x}_{i+1} $ por cualquier vector con coeficiente no nulo. Por tanto, tenemos que demostrar que existe un coeficiente del segundo sumatorio no nulo. Si fueran todos nulos, tendríamos que $\displaystyle \vec{x}_{i+1} $ se puede expresar como combinación lineal de los vectores $\displaystyle \left\{ \vec{x}_{1}, \vec{x}_{2}, \ldots, \vec{x}_{p}\right\} $, esto contradice que sean linealmente independientes. 
\end{proof}

\begin{fcolorary}[]
\normalfont Si el espacio vectorial $\displaystyle V $ tiene una base finita, todas las bases de $\displaystyle V $ tienen el mismo número de vectores.
\end{fcolorary}

\begin{proof}
Sean $\displaystyle B_{1} =\left\{ \vec{x}_{1}, \vec{x}_{2}, \ldots, \vec{x}_{p}\right\} $ y $\displaystyle B_{2}=\left\{ \vec{y}_{1}, \vec{y}_{2}, \ldots, \vec{y}_{q}\right\} $ dos bases de $\displaystyle V $. Como $\displaystyle B_{1} $ es una base y $\displaystyle B_{2} $ es un conjunto de vectores linealmente independientes, tenemos que todas las bases de $\displaystyle V $ han de ser finitas. Entonces, como $\displaystyle B_{1} $ y $\displaystyle B_{2} $ son bases y, consecuentemente, linealmente independientes, tenemos que $\displaystyle p \leq q $ y $\displaystyle q \leq p $, por lo que $\displaystyle p = q $.
\end{proof}

\begin{fdefinition}[Dimensión]
\normalfont La \textbf{dimensión} de un espacio vectorial $\displaystyle V $ sobre un cuerpo $\displaystyle \K $ es el número de elementos de sus bases, si son finitas. Si no lo son, diremos que $\displaystyle V $ es de dimensión infinita.
\end{fdefinition}

\begin{fcolorary}[]
\normalfont La dimensión de un espacio vectorial coincide con el número máximo de elementos linealmente independientes, y también con el número mínimo de generadores.
\end{fcolorary}

\begin{fcolorary}[]
\normalfont Todo conjunto de vectores linealmente independientes puede completarse hasta obtener una base.
\end{fcolorary}

\begin{flema}[]
	\normalfont Si $S \subset V$ es linealmente independiente y $\displaystyle \vec{x} \in V $ y $\displaystyle \vec{x} \not\in L\left(S\right) $, tenemos que la familia $\displaystyle S \cup \left\{ \vec{x}\right\}   $ es linealmente independiente. 
\end{flema}

\begin{proof}
Sean $\displaystyle a,a^{i} \in \K $ y 
\[a\vec{x} + a^{1}\vec{x}_{1} + \cdots + a^{p}\vec{x}_{p} = \vec{0} .\]
Si $\displaystyle a \neq 0 $, entonces $\displaystyle \vec{x} $ se puede expresar como combinación lineal de $\displaystyle S $, pero por hipótesis esto no es posible. Por tanto, debe ser que $\displaystyle a = 0 $ y, consecuentemente, $\displaystyle \forall a^{i} = 0 $, pues $\displaystyle S $ es linealmente independiente. Por tanto, $\displaystyle S \cup \left\{ \vec{x}\right\}  $ también es linealmente independiente.
\end{proof}

\begin{fprop}[]
\normalfont Si $\displaystyle V $ es finitamente generado y $\displaystyle L $ es subespacio vectorial de $\displaystyle V $, entonces $\displaystyle L $ es infinitamente generado y 
\[\dim L \leq \dim V .\]
Además, 
\[\dim L = \dim V \iff L = V .\]
\end{fprop}

\begin{proof}
Si $\displaystyle L = \left\{ 0\right\}  $ no hay nada que probar (no tiene bases). En caso contrario, existe $\displaystyle \vec{x}_{1} \in L $. Si $\displaystyle L = L\left( \left\{ \vec{x}_{1}\right\} \right) $, tenemos que $\displaystyle \vec{x}_{1} $ es una base. En caso contrario, existe $\displaystyle \vec{x}_{2} \in L $ con $\displaystyle \vec{x}_{2} \not\in L \left( \left\{ \vec{x}_{1}\right\} \right) $ . Si $\displaystyle L = L\left( \left\{ \vec{x}_{1}, \vec{x}_{2}\right\} \right) $, $\displaystyle \left\{ \vec{x}_{1}, \vec{x}_{2}\right\}  $ forman una base. Sabemos que son linealmente independientes por el lema anterior. En algún momento llegaremos a que $\displaystyle L\left( \left\{ \vec{x}_{1}, \vec{x}_{2}, \ldots, \vec{x}_{n}\right\} \right) $ forman una base, pues un corolario anterior nos dice que hay un número máximo de vectores linealmente independientes. \\ \\
Además, si $\displaystyle \dim L = n $ y $\displaystyle \left\{ \vec{x}_{1}, \vec{x}_{2}, \ldots, \vec{x}_{n}\right\}  $ es una base de $\displaystyle L $, por el teorema de Steinitz, también es una base de $\displaystyle V $. Por tanto, 
\[ L = L\left( \left\{ \vec{x}_{1}, \vec{x}_{2}, \ldots, \vec{x}_{n}\right\} \right) = V .\]
\end{proof}

\begin{ftheorem}[Teorema de apliación de base]
\normalfont Sea $\displaystyle L $ un subespacio vectorial de $\displaystyle V $ y sea $\displaystyle \left\{ \vec{x}_{1}, \vec{x}_{2}, \ldots, \vec{x}_{p}\right\}  $ una base de $\displaystyle L $. Entonces existe $\displaystyle \left\{ \vec{u}_{p+1}, \vec{u}_{p+2}, \ldots, \vec{u}_{n}\right\} \subset V $ tales que $\displaystyle \left\{ \vec{x}_{1}, \vec{x}_{2}, \ldots, \vec{x}_{p}, \vec{u}_{p+1}, \vec{u}_{p+2}, \ldots, \vec{u}_{n}\right\}  $ son base de $\displaystyle V $.
\end{ftheorem}

\begin{proof}
Si $\displaystyle \dim V = n $ tenemos que existe un número finito de generadores que forman una base de $\displaystyle V $, y consideramos que los vectores $\displaystyle \left\{ \vec{x}_{1}, \vec{x}_{2}, \ldots, \vec{x}_{p}\right\}  $ forman una base de $\displaystyle L $. Entonces, $\displaystyle p \leq n $ y, por el teorema de Steinitz, se puede obtener una nueva base sustituyendo $\displaystyle \left\{ \vec{x}_{1}, \vec{x}_{2}, \ldots, \vec{x}_{p}\right\}  $ por $\displaystyle p $ vectores de $\displaystyle \left\{ \vec{u}_{1}, \vec{u}_{2}, \ldots, \vec{u}_{n}\right\}  $.
\end{proof}

\section{Suma directa de subespacios.}

\textbf{Notación. } 
\[\mathcal{P}\left(V\right) = \left\{ A \; : \; A \subset V\right\}  .\]
\[\mathcal{L}\left(V\right) = \left\{ L \in \mathcal{P}\left(V\right) \; : \; L \; \text{es subespacio vectorial de $\displaystyle V $ }\right\}  .\]
Por tanto, 
\[\mathcal{L}\left(V\right) \subset \mathcal{P}\left(V\right) .\]

\begin{ftheorem}[]
	\normalfont $\displaystyle \forall I $ conjunto, $\displaystyle \forall i \in I $, si $\displaystyle L_{i} \in \mathcal{L}\left(V\right) $ entonces
	\[\bigcap_{i \in I}L_{i} \in \mathcal{L}\left(V\right) .\]
Es decir, la intersección de espacios vectoriales es un espacio vectorial.
\end{ftheorem}

\begin{proof}
$\displaystyle \forall \vec{x}, \vec{y} \in \bigcap_{i \in I} L_{i} $ implica que $\displaystyle \vec{x}, \vec{y} \in L_{i}, \forall i \in I $.Como $\displaystyle L_{i} $ son subespacios vectoriales:
\[\vec{x}+\vec{y} \in L_{i}, \forall i \in I\; \Rightarrow \; \vec{x} + \vec{y} \in \bigcap_{i \in I}L_{i} .\]
Similarmente, si $\displaystyle \vec{x} \in \bigcap_{i \in I}L_{i} $ y $\displaystyle a \in \K $, tenemos que $\displaystyle \vec{x} \in L_{i}, \forall i \in I $. Como $\displaystyle L_{i} $ son subespacios vectoriales, son parte estable, por lo que 
\[a \cdot \vec{x} \in L_{i}, \forall i \in I \; \Rightarrow \; a \cdot \vec{x} \in \bigcap_{i \in I}L_{i} .\]
\end{proof}

\textbf{Observación. }Sin embargo, no tiene que cumplirse necesariamente que $\displaystyle L_{1} \cup L_{2} \in \mathcal{L}\left(V\right) $ si $\displaystyle L_{1}, L_{2} \in \mathcal{L}\left(V\right) $.

\begin{eg}
	\normalfont Sean $\displaystyle \left\{ \vec{u}, \vec{v}\right\} \subset V $ linealmente independientes y $\displaystyle L_{1} = L\left( \left\{ \vec{u}\right\} \right) $ y $\displaystyle L_{2} = L\left( \left\{ \vec{v}\right\} \right) $ las rectas que generan. Asumamos que $\displaystyle \vec{u} + \vec{v} \in L_{1} \cup L_{2} $. Sin pérdida de generalidad, $\displaystyle \vec{u} + \vec{v} \in L_{1} $. Por tanto, $\displaystyle \exists a \in \K $ tal que $\displaystyle \vec{u} + \vec{v} = a \vec{u} $. De esta manera, 
	\[\left(a-1\right) \vec{u} + \vec{v} = \vec{0} .\]
Esto es absurdo, pues hemos dicho que estos vectores son linealmente independientes.
\end{eg}

\begin{fdefinition}[]
	\normalfont Si $\displaystyle L_{1}, L_{2} \in \mathcal{L}\left(V\right) $, definimos $\displaystyle L_{1}+L_{2} $ al menor subespacio vectorial generado por la unión.
	\[L_{1}+L_{2} = L\left(L_{1} \cup L_{2}\right) .\]
\end{fdefinition}

\begin{ftheorem}[]
	\normalfont Sean $\displaystyle L_{1}, L_{2} \in \mathcal{L}\left(V\right) $ y sea $\displaystyle L' = \left\{ \vec{x}_{1} + \vec{x}_{2} \; : \; \vec{x}_{1} \in L_{1}, \vec{x}_{2} \in L_{2}\right\}  $. Tenemos que $\displaystyle L' = L_{1} + L_{2} $.
\end{ftheorem}

\begin{proof}
Si $\displaystyle \vec{x}_{1}\in L_{1} $, tenemos que $\displaystyle \vec{x}_{1} = \vec{x}_{1} + \vec{0} \in L' $, pues $\displaystyle \vec{x}_{1}\in L_{1} $ y $\displaystyle \vec{0} \in L_{2} $. Por tanto, $\displaystyle L_{1}\subset L' $. Similarmente, $\displaystyle L_{2} \subset L' $. Consecuentemente, $\displaystyle L_{1}\cup L_{2} \subset L' $. \\ \\
Además, tenemos que $\displaystyle L' \in \mathcal{L}\left(V\right) $, pues $\displaystyle \forall \vec{x}, \vec{y} \in L' $ tenemos que $\displaystyle \exists \vec{x}_{1},\vec{y}_{1}\in L_{1} $ y $\displaystyle \exists \vec{x}_{2}, \vec{y}_{2} \in L_{2} $. 
\[\vec{x} + \vec{y} = \left(\vec{x}_{1} + \vec{x}_{2}\right) + \left(\vec{y}_{1}+ \vec{y}_{2}\right) = \underbrace{\left(\vec{x}_{1} + \vec{y}_{1}\right)}_{\in L_{1}} + \underbrace{\left(\vec{x}_{2} + \vec{y}_{2}\right)}_{\in L_{2}} .\]
Por tanto, $\displaystyle \vec{x} + \vec{y} \in L' $. Similarmente, si $\displaystyle a \in \K $ y $\displaystyle \vec{x} \in L' $ tenemos que existen $\displaystyle \vec{x}_{1} \in L_{1} $ y $\displaystyle \vec{x}_{2} \in L_{2} $ tales que 
\[a \cdot \vec{x} = a \cdot \left(\vec{x}_{1} + \vec{x}_{2}\right) = \underbrace{a\vec{x}_{1}}_{\in L_{1}}+\underbrace{a\vec{x}_{2}}_{\in L_{2}} .\]
Por tanto, $\displaystyle a \cdot \vec{x} \in L' $. Por tanto, $\displaystyle L' \in \mathcal{L}\left(V\right) $. \\\\
A continuación demostramos que si $\displaystyle L \in \mathcal{L}\left(V\right) $ y $\displaystyle L_{1}\cup L_{2}\subset L $, entonces $\displaystyle L' \subset L $.  Tenemos que $\displaystyle \forall\vec{x}\in L' $ existen $\displaystyle \vec{x}_{1}\in L_{1} $ y $\displaystyle \vec{x}_{2}\in L_{2} $ tales que $\displaystyle \vec{x} = \vec{x}_{1}+\vec{x}_{2} $. Por tanto, $\displaystyle \vec{x} \in L $.
\[\therefore L ' \subset L .\]
Por todo ello, $\displaystyle L' = L_{1}+L_{2} $.
\end{proof}

\begin{ftheorem}[Fórmula de Grassmann]
\normalfont Supongamos que $\displaystyle V $ es de dimensión finita, por lo que todos los conjuntos que vamos a tratar a continuación son de dimensión finita.
\[\dim \left(L_{1} + L_{2}\right) = \dim L_{1} + \dim L_{2} - \dim \left(L_{1} \cap L_{2}\right) .\]
\end{ftheorem}

\begin{proof}
	Sea $\displaystyle \left\{ \vec{u}_{1}, \vec{u}_{2}, \ldots, \vec{u}_{m}\right\}  $ una base de $\displaystyle L_{1} \cap L_{2} $, y sea $\displaystyle v = \dim \left(L_{1} \cap L_{2}\right) $. Podemos ampliar esta base hasta obtener una base de $\displaystyle L_{1} $ y $\displaystyle L_{2} $. Sea $\displaystyle \left\{ \vec{u}_{1}, \vec{u}_{2}, \ldots, \vec{u}_{m}, \vec{u}_{m+1} \ldots, \vec{u}_{r}\right\} $ una base de $\displaystyle L_{1} $. Sea $\displaystyle \left\{ \vec{u}_{1}, \vec{u}_{2}, \ldots, \vec{u}_{m}, \vec{v}_{m+1}, \ldots, \vec{v}_{s}\right\} $ base de $\displaystyle L_{2} $ . Queremos ver que $\displaystyle \left\{ \vec{u}_{1}, \vec{u}_{2}, \ldots, \vec{u}_{m}, \ldots, \vec{u}_{m+1}, \ldots,\vec{u}_{r}, \vec{v}_{m+1}, \ldots, \vec{v}_{s}\right\} $  es base de $\displaystyle L_{1}+L_{2} $. Primero queremos ver que es sistema de generadores. Sea $\displaystyle \vec{x} \in L_{1}+L_{2} $. Entonces, existen $\displaystyle \vec{x}_{1}\in L_{1} $ y $\displaystyle \vec{x}_{2}\in L_{2} $ tales que $\displaystyle \vec{x} = \vec{x}_{1} + \vec{x}_{2} $. Como anteriormente hemos definido bases para $\displaystyle L_{1} $ y $\displaystyle L_{2} $, tenemos que existen $\displaystyle a^{i}\in\K $ tales que 
	\[\vec{x}_{1} = a^{1}\vec{u}_{1} + a^{2}\vec{u}_{2} + \cdots + a^{r}\vec{u}_{r} .\]
Similarmente, existen $\displaystyle b^{i}\in \K $ tales que
\[\vec{x}_{2} = b^{1}\vec{u}_{1} + \cdots + b^{m}\vec{u}_{m}+b^{r+1}\vec{v}_{r+1} + \cdots + b^{s}\vec{v}_{s} .\]
Por tanto, tenemos que $\displaystyle \vec{x} $ se puede expresar como combinación lineal de $\displaystyle \left\{ \vec{u}_{1}, \ldots, \vec{u}_{m}, \vec{u}_{m+1}, \ldots, \vec{u}_{r}, \vec{v}_{m+1}, \ldots, \vec{v}_{s}\right\}  $. 
\[\vec{x} = \vec{x}_{1} + \vec{x}_{2} = \sum^{r}_{i=1}a^{i}\vec{u}_{i} + \sum^{m}_{i=1}b^{1}\vec{u}_{i} + \sum^{s}_{i = m+1}b^{i}\vec{v}_{i} .\]
Por tanto, es sistema de generadores, ahora tenemos que ver que son linealmente independientes. Sean $\displaystyle a^{i}, b^{j} \in \K $ tales que 
\[
\begin{split}
  a^{1}\vec{u}_{1} + a^{2}\vec{u}_{2} + \cdots +a^{m}\vec{u}_{m} + \cdots + a^{r}\vec{u}_{r+1} + \cdots + \cdots + b^{m+1}\vec{v}_{m+1} + \cdots + b^{s}\vec{v}_{s} = \vec{0}  
\end{split}
\]
\[ \sum^{r}_{i=1}a^{i}\vec{u}_{i} + \sum^{s}_{i = m+1}b^{i}\vec{v}_{i} = \vec{0}. \]

Sea $\displaystyle \vec{y} = \sum^{r}_{i = 1} a^{i}\vec{u}_{i}$ por lo que $\displaystyle \vec{y} \in L_{1} $. Entonces, 
\[-\vec{y} = \sum^{s}_{i = m+1}b^{i}\vec{v}_{i} \in L_{2} .\]
Por tanto, $\displaystyle \vec{y} \in L_{1}\cap L_{2} $. Entonces, $\displaystyle \vec{y} $ se puede expresar como combinación lineal de los vectores de la base de la intersección. Es decir, existen $\displaystyle c^{i} \in \K $ tales que 
\[\vec{y} = c^{1}\vec{u}_{1} + c^{2}\vec{u}_{2} + \cdots + c^{m}\vec{u}_{m} .\]
Entonces, 
\[\vec{0} = \vec{y} - \vec{y} = \sum^{m}_{j=1}c^{j}\vec{u}_{j} + \sum^{s}_{i = m+1}b^{i}\vec{v}_{i} = \vec{0} .\]
Como la base de $\displaystyle L_{2} $ es linealmente independiente (porque es una base), tenemos que $\displaystyle c^{1} = c^{2} = \cdots = c^{r} = b^{1} = \cdots = b^{q} = 0 $. Consecuentemente, $\displaystyle \sum^{r}_{i=1}a^{i}\vec{u}_{i} = \vec{0} $ y, dado que $\displaystyle \left\{ \vec{u}_{1}, \ldots, \vec{u}_{r}\right\}  $ es base de $\displaystyle L_{1} $, tenemos que $\displaystyle a^{i}=0, \forall i = 1, \ldots, r $. Por tanto, hemos visto que el conjunto que estábamos estudiando es base. \\ \\
Esta base tiene dimensión $\displaystyle r + s = r + \left(s + m\right) - m$.
\end{proof}

Si $\displaystyle L_{1}\cap L_{2} = \left\{ \vec{0}\right\}  $, tenemos que $\displaystyle \dim L_{1} + \dim L_{2} = \dim \left(L_{1} + L_{2}\right) $.

\begin{fdefinition}[]
	\normalfont Sean $\displaystyle L_{1}, L_{2} \in \mathcal{L}\left(V\right) $ su suma es directa si $\displaystyle L_{1} \cap L_{2} = \left\{ \vec{0}\right\}  $ y se escribe de la siguiente manera:
	\[L_{1} \oplus L_{2}.\]
\end{fdefinition}

\begin{fprop}[]
	\normalfont Sean $\displaystyle L_{1}, L_{2} \in \mathcal{L}\left(V\right) $. Entonces
	\[L_{1} \oplus L_{2} \iff \forall \vec{x}\in L_{1} + L_{2}, \; \exists!\vec{x}_{1} \in L_{1}, \vec{x}_{2} \in L_{2}, \; \vec{x} = \vec{x}_{1} + \vec{x}_{2} .\]
\end{fprop}

\begin{proof}
\begin{description}
\item[(i)] Supongamos que $\displaystyle L_{1} \oplus L_{2} $. Asumimos que existen, $\displaystyle \vec{x}_{1}, \vec{y}_{1} \in L_{1} $ y $\displaystyle \vec{x}_{2}, \vec{y}_{2} \in L_{2} $ tales que si $\displaystyle \vec{x} \in L_{1} + L_{2} $ tenemos que 
	\[\vec{x} = \vec{x}_{1} + \vec{x}_{2} = \vec{y}_{1} + \vec{y}_{2} .\]
Entonces, 
\[
\begin{split}
& \vec{0} = \left(\vec{x}_{1} - \vec{y}_{1}\right) + \left(\vec{x}_{2} - \vec{y}_{2}\right) .\\ 
\Rightarrow & \vec{x}_{1} - \vec{y}_{1} = \vec{y}_{2} - \vec{x}_{2}
\end{split}
\]
Consecuentemente, $\displaystyle \vec{x}_{1} - \vec{y}_{1}, \vec{y}_{2} - \vec{x}_{2} \in L_{1} \cap L_{2} $, por tanto, $\displaystyle \vec{x}_{1}-\vec{y}_{1} = \vec{0} \iff \vec{x}_{1} = \vec{y}_{1} $. Similarmente, $\displaystyle \vec{y}_{2} = \vec{x}_{2} $.
\item[(ii)] Suponemos que $\displaystyle \forall \vec{x}\in L_{1} + L_{2}, \; \exists!\vec{x}_{1} \in L_{1}, \vec{x}_{2} \in L_{2}, \; \vec{x} = \vec{x}_{1} + \vec{x}_{2} $. Queremos ver que $\displaystyle L_{1}\cap L_{2} = \left\{ \vec{0}\right\}  $. Sea $\displaystyle \vec{x} \in L_{1} \cap L_{2} $. Tenemos que $\displaystyle \vec{x} = \vec{x} + \vec{0} = \vec{0} + \vec{x} $. Tenemos que $\displaystyle \vec{0} \in L_{1} \cap L_{2} $. Como la expresión de $\displaystyle \vec{x} $ ha de ser única, tenemos que $\displaystyle \vec{x} = \vec{0} $, por lo que $\displaystyle L_{1}\cap L_{2} = \left\{ \vec{0}\right\}  $.
\end{description}
\end{proof}

Tenemos que si $\displaystyle L_{1}, L_{2}, \ldots, L_{k} \in \mathcal{L}\left(V\right) $, 
\[\left(L_{1} + L_{2}\right) + L_{3} = \left\{ \left(\vec{x}_{1} + \vec{x}_{2}\right) + \vec{x}_{3}\; : \; \vec{x}_{1} \in L_{1}, \vec{x}_{2} \in L_{2}, \vec{x}_{3} \in L_{3}\right\} = L_{1}+\left(L_{2} + L_{3}\right) .\]
Generalmente, esta suma es asociativa, es decir se puede escribir
\[L_{1} + L_{2} + \cdots + L_{k} = \left\{ \vec{x}_{1} + \vec{x}_{2} + \cdots + \vec{x}_{k} \; : \; \vec{x}_{i} \in L_{i}\right\}  .\]

\begin{fdefinition}[]
\normalfont Diremos que la suma $\displaystyle L_{1} + L_{2} + \cdots + L_{k}$ es directa si $\displaystyle \forall \vec{x} \in L_{1} + L_{2} + \cdots + L_{k}, \exists! \vec{x}_{i} \in L_{i} $ tales que $\displaystyle \vec{x} = \sum^{k}_{i=1}\vec{x}_{i} $. Esto se denotará de la siguiente manera:
\[L_{1} \oplus L_{2} \oplus \cdots \oplus L_{k}.\]
\end{fdefinition}

\begin{fprop}[]
\normalfont Si $\displaystyle L_{1}, \ldots, L_{k} $, entonces $\displaystyle L_{1} \oplus \cdots \oplus L_{k} $ si y solo si $\displaystyle \forall i = 1, \ldots, k $, 
\[L_{i} \cap \left(L_{1} + L_{2} + \cdots + L_{i-1} + L_{i+1} + \cdots + L_{k}\right) = \left\{ \vec{0}\right\}  .\]
\end{fprop}

\begin{proof}
\begin{description}
\item[(i)] Demostramos la primera implicación. Supongamos que $\displaystyle \forall i = 1, \ldots, k $, y sea $\displaystyle \vec{x} \in L_{i} \cap \left(L_{1} + \cdots + L_{i-1} + L_{i+1} + \cdots + L_{k}\right) $. Queremos decir que $\displaystyle \vec{x} = \vec{0} $. Tenemos que, dado que $\displaystyle \vec{x}\in L_{i} $:
	\[\vec{x} = \vec{0} + \cdots + \vec{0} + \vec{x}_{i} + \vec{0} + \cdots + \vec{0} .\]
Como $\displaystyle \vec{x} \in L_{i} \cap \left(L_{1} + \cdots + L_{i-1} + L_{i+1} + \cdots + L_{k}\right) $, tenemos que $\displaystyle \vec{x}\in L_{1} + \cdots + L_{i-1}+L_{i+1}+\cdots + L_{k} $ :
\[\vec{x} = \vec{x}_{1} + \cdots + \vec{x}_{i-1} + \vec{x}_{i+1} + \cdots + \vec{x}_{k} .\]
Como $\displaystyle L_{1} \oplus \cdots \oplus L_{k} $, tenemos que la expresión de $\displaystyle \vec{x} $ es única, por lo que $\displaystyle \vec{x}=\vec{0} $.
\item[(ii)] Demostramos la siguiente implicación. Asumimos que 
	\[\displaystyle \forall i = 1, \ldots, k; \; L_{i} \cap \left(L_{1} + L_{2} + \cdots + L_{i-1} + L_{i+1} +\cdots + L_{k}\right) = \left\{ \vec{0}\right\}.\] 
Sea 
	\[\vec{x} = \sum^{k}_{i=1}\vec{x}_{i} = \sum^{k}_{i=1}\vec{y}_{i} .\]
Con $\displaystyle \vec{x}_{i}, \vec{y}_{i} \in L_{i} $, tenemos que 
\[\vec{0} = \left(\vec{x}_{1}-\vec{y}_{1}\right) + \cdots + \left(\vec{x}_{i} + \vec{y}_{i}\right) + \cdots + (\vec{x}_{k}+\vec{y}_{k}) .\]
\[\therefore \vec{y}_{i}-\vec{x}_{i} = \left(\vec{x}_{1} - \vec{y}_{1}\right) + \cdots + \left(\vec{x}_{i-1}-\vec{y}_{i-1}\right) + \left(\vec{x}_{i+1} - \vec{y}_{i+1}\right) + \cdots + \left(\vec{x}_{k} - \vec{y}_{k}\right) .\]
Por lo que $\displaystyle \vec{y}_{i} - \vec{x}_{i} = \vec{0}, \forall i = 1, \ldots, k $. Si esto no fuera cierto para algún $\displaystyle \vec{x}_{j}- \vec{y}_{j} $, podríamos despejarlo y tendríamos que está en la intersección pero a la vez no es $\displaystyle \vec{0} $, lo cual contradice nuestra hipótesis.
\end{description}
\end{proof}

\begin{fdefinition}[]
	\normalfont $\displaystyle L_{1}, L_{2} \in \mathcal{L}\left(V\right) $ son \textbf{complementarios} si $\displaystyle L_{1} \oplus L_{2} $ y $\displaystyle V = L_{1} \oplus L_{2} $. 
	\[L_{1}\oplus L_{2} = V \iff \forall\vec{x}\in V, \exists! \vec{x}_{1}\in L_{1}, \vec{x}_{2}\in L_{2}, \; \vec{x} = \vec{x}_{1} + \vec{x}_{2} .\]	
\end{fdefinition}

\begin{ftheorem}[]
\normalfont Sea $\displaystyle L \in \mathcal{L}\left(V\right) $, entonces existe $\displaystyle L' \in \mathcal{L}\left(V\right) $ tal que $\displaystyle L \oplus L' = V $.
\end{ftheorem}

\begin{proof}
	Sea $\displaystyle \left\{ \vec{u}_{1}, \vec{u}_{2}, \ldots, \vec{u}_{r}\right\}  $ una base de $\displaystyle L $ y sea $\displaystyle \left\{ \vec{u}_{1}, \ldots, \vec{u}_{r}, \vec{u}_{r+1}, \ldots, \vec{u}_{n}\right\}  $ base de $\displaystyle V $. La manera de ampliar una base no es única. Sea $\displaystyle L' $ el subespacio generado por los vectores que he añadido a la base de $\displaystyle L $ para formar la base de $\displaystyle V $, es decir, 
	\[L' = L(\left\{ \vec{u}_{r+1}, \ldots, \vec{u}_{n}\right\})  .\]
Tenemos que $\displaystyle \forall \vec{x}\in V $, existen $\displaystyle a^{i}\in\K $ tales que 
\[\vec{x} = a^{1}\vec{u}_{1}+ \cdots + a^{r}\vec{u}_{r} + a^{r+1}\vec{u}_{r+1} +\cdots + a^{n}\vec{u}_{n} .\]
Sea $\displaystyle \vec{x}_{1} = a^{1}\vec{u}_{1}+ \cdots + a^{r}\vec{u}_{r} \in L $ y $\displaystyle \vec{x}_{2} = a^{r+1}\vec{u}_{r+1} +\cdots + a^{n} \in L' $. Vamos a ver que $\displaystyle L \cap L' = \left\{ \vec{0}\right\}  $. Sabemos que $\displaystyle \left\{ \vec{0}\right\} \subset L \cap L' $. Si $\displaystyle \vec{x}\in L\cap L' $. Entonces existen $\displaystyle a^{i}\in\K $ tales que
\[\vec{x} = a^{1}\vec{u}_{1} + \cdots + a^{r}\vec{u}_{r} .\]
Como $\displaystyle \vec{x}\in L' $, existen $\displaystyle a^{j} \in \K $ tales que 
\[\vec{x} = a^{r+1} \vec{u}_{r+1} + \cdots + a^{n}\vec{u}_{n} .\]
Entonces, 
\[\vec{0} = \left(a^{1}\vec{u}_{1} + \cdots + a^{r}\vec{u}_{r}\right) - \left(a^{r+1} \vec{u}_{r+1} + \cdots + a^{n}\vec{u}_{n}\right) .\]
Esto es una combinación lineal de los elementos de una base que me da el vector nulo. Como las bases son linealmente independientes, tenemos que 
\[a^{1} = \cdots = a^{r} = a^{r+1} = \cdots = a^{n} = 0 .\]
Por tanto, $\displaystyle \vec{x} = \vec{0} $. Consecuentemente, la suma es directa. 
\end{proof}

\section{Espacio vectorial cociente}

\begin{fdefinition}[]
\normalfont Definimos la relación de equivalencia:
\[\vec{x} \mathcal{R} \vec{y} \iff \vec{y}-\vec{x} \in L .\]
Si $\displaystyle \vec{x}\in V $,
\[\displaystyle \left[\vec{x}\right] = \left\{ \vec{y} \in V \; : \; \vec{y}-\vec{x} \in L\right\} = \left\{ \vec{y} \in V \; : \; \vec{y}-\vec{x} = \vec{l}\right\}  = \left\{ \vec{x} + \vec{l} \; : \; \vec{l}\in L\right\}  = \vec{x} + L  \]
\end{fdefinition}

\begin{ftheorem}[]
\normalfont Sea $\displaystyle V $ un $\displaystyle \K $-espacio vectorial. Si $\displaystyle \vec{x}, \vec{y} \in V $ decimos que $\displaystyle \vec{x} R \vec{y} $ si $\displaystyle \vec{x} - \vec{y} \in L $, donde $\displaystyle L \in \mathcal{L}\left(V\right) $. Tenemos que $\displaystyle R $ es una relación de equivalencia en $\displaystyle V $.
\end{ftheorem}

\begin{proof}
\begin{description}
\item[(i)] Reflexiva. Si $\displaystyle \vec{x} \in V $, tenemos que 
	\[\vec{x}-\vec{x} \in \vec{0} \in L .\]
\item[(ii)] Simétrica. Si $\displaystyle \vec{x}, \vec{y} \in V $ tal que $\displaystyle \vec{x} R \vec{y} $, tenemos que 
	\[\vec{x} - \vec{y} \in L \iff \vec{y} - \vec{x} \in L .\]
Pues $\displaystyle \vec{y}-\vec{x} = \left(-1\right)\left(\vec{x}-\vec{y}\right) \in L $. 
\item[(iii)] Transitiva. Si $\displaystyle \vec{x} R \vec{y} $ y $\displaystyle \vec{y} R \vec{z} $, tenemos que $\displaystyle \vec{x}-\vec{y} \in L $ y $\displaystyle \vec{y}-\vec{z} \in L $. Como $\displaystyle L $ es espacio vectorial tenemos que:
	\[\left(\vec{x}-\vec{y}\right)+\left(\vec{y}-\vec{z}\right) = \vec{x}-\vec{z} \in L .\]
\end{description}
\end{proof}

Si $\displaystyle \vec{x} \in V $, tenemos que 
\[ \left[\vec{x}\right]  = \left\{ \vec{y} \in V \; : \; \vec{y}R\vec{x}\right\} = \left\{ \vec{y}\in V\; : \; \vec{y}-\vec{x} \in L\right\} = \left\{ \vec{y} \in V \; : \; \exists \vec{l}\in L, \; \vec{y}-\vec{x} = \vec{l}\right\} = \left\{ \vec{x}+\vec{l} \; : \; \vec{l}\in L\right\} = \vec{x} + L  .\]
\begin{fdefinition}[Espacio cociente]
\normalfont 
\[V / R = V / L = \left\{ \vec{x} + L \; : \; \vec{x} \in V\right\}  .\]
\end{fdefinition}

En $\displaystyle V / L $ defino una suma y un producto por escalares:
\[
\begin{split}
+ : & V/L \times V/L \to V/L \\
    & \left(\vec{x} + L, \vec{y} + L\right) \to \left(\vec{x} + L\right) + \left(\vec{y}+L\right) = \left(\vec{x}+\vec{y}\right) + L.
\end{split}
\]
Si $\displaystyle \vec{x} + L  = \vec{x'} + L$ y $\displaystyle \vec{y}+L = \vec{y'}+ L $. Tenemos que ver que si $\displaystyle \vec{x} + L = \vec{y} + L $, entonces $\displaystyle \vec{x'} + L = \vec{y'} + L $. \footnote{El objetivo, tanto en la suma como en el producto por escalares, es ver que realmente se trata de una aplicación, es decir, si cojo $\displaystyle [\vec{x}] = [\vec{x'}]$ y $\displaystyle [\vec{y}] = [\vec{y'}] $, me tiene que dar que $\displaystyle [\vec{x} + \vec{y}] = [\vec{x'} + \vec{y'}] $.}  Tenemos que 
\[\vec{x'} = \vec{x} + \vec{l} \quad \text{y} \quad \vec{y'} = \vec{y} + \vec{l'}, \quad \text{con}\; \vec{l}, \vec{l'}\in L .\]
\[\vec{x'} + \vec{y'} = \left(\vec{x}+ \vec{l}\right) + \left(\vec{y}+\vec{l'}\right) = \left(\vec{x} + \vec{y}\right) + \left(\vec{l} + \vec{l'}\right) .\]
Entonces, 
\[\left(\vec{x'}+ \vec{y'}\right)-\left(\vec{x}+\vec{y}\right) = \vec{l}+\vec{l'}\in L .\]
Entonces, $\displaystyle \left(\vec{x}+\vec{y}\right)R\left(\vec{x'}+\vec{y'}\right) $, por lo que sus clases de equivalencia son iguales. 
\begin{ftheorem}[]
\normalfont $\displaystyle \left(V/L, +\right) $ es un grupo abeliano. 
\end{ftheorem}

\begin{proof}
\begin{description}
\item[(i)] $\displaystyle \forall\vec{x}, \vec{y} \in V $, 
	\[\left(\vec{x} + L\right) + \left(\vec{y} + L\right) = \left(\vec{x} + \vec{y}\right) + L = \left(\vec{y}+\vec{x}\right)+L .\]
\item[(ii)] $\displaystyle \forall \vec{x}, \vec{y}, \vec{z} \in V $,
	\[\left(\left(\vec{x}+L\right)+\left(\vec{y}+L\right)\right)+\left(\vec{z}+L\right) = \left(\left(\vec{x}+\vec{y}\right)+L\right)+\left(\vec{z}+L\right) = \left(\left(\vec{x}+\vec{y}\right)+\vec{z}\right)+L = \left(\vec{x}+L\right)+\left(\left(\vec{y}+L\right)+\left(\vec{z}+L\right)\right) .\]
\item[(iii)] Si consideramos la clase $\displaystyle \vec{0}+L $, tenemos que $\displaystyle \forall \vec{x} \in V $, 
	\[\left(\vec{0}+L\right)+\left(\vec{x}+L\right) = \left(\vec{0}+\vec{x}\right) + L =\vec{x}+ L .\]
\item[(iv)] Si $\displaystyle \vec{x} \in V $, el opuesto será, $\displaystyle -\left(\vec{x}+L\right) = \left(-\vec{x}\right) + L $.
	\[\left(\vec{x} + L\right) + \left(-\left(\vec{x}+L\right)\right)=\left(\vec{x} + L\right)+\left(\left(-\vec{x}\right)+L\right) = \left(\vec{x}-\vec{x}\right)+L = \vec{0} + L .\]
\end{description}
\end{proof}

El producto por escalares está definido así:
\[
\begin{split}
	\cdot & \K \times V/L \to V/L \\
	    \forall a \in \K, \forall \vec{x} \in V  & a \cdot \left(\vec{x} + L\right) = \left(a\vec{x}\right)+ L.
\end{split}
\]
Queremos ver que se trata de una aplicación. Si $\displaystyle \vec{x'}+L = \vec{x}+L $, queremos ver que $\displaystyle \left(a\vec{x'} \right)+ L = \left(a\vec{x} \right)+ L $. Tenemos que $\displaystyle \vec{x'} - \vec{x} \in L $, por lo que $\displaystyle a\vec{x}'-a\vec{x} \in L $. Por tanto, 
\[\left(a\vec{x}\right)+ L = \left(a\vec{x'}\right)+L .\]

\begin{observation}
\normalfont Estas operaciones hacen $\displaystyle V/L $ un espacio vectorial sobre $\displaystyle \K $.
\end{observation}

\begin{ftheorem}[]
\normalfont La dimensión del espacio vectorial $\displaystyle V/L $ se puede expresar como:
\[ \dim\left(V/L\right) = \dim\left(V\right) - \dim\left(L\right) .\]
\end{ftheorem}

\begin{proof}
	Sea $\displaystyle \left\{ \vec{u}_{1}, \ldots, \vec{u}_{r}\right\}  $ base de $\displaystyle L $ y sea $\displaystyle \left\{ \vec{u}_{1}, \ldots, \vec{u}_{r}, \ldots, \vec{u}_{n}\right\}  $ una base de $\displaystyle V $. Si $\displaystyle i = 1, \ldots, r $, tenemos que 
	\[\vec{u}_{i} = \vec{u}_{i} - \vec{0} \in L \Rightarrow \vec{u}_{i}+L = \vec{0} + L .\]
	Vamos a demostrar que $\displaystyle \left\{ \vec{u}_{r+1}+L, \ldots, \vec{u}_{n}+L\right\}  $ base de $\displaystyle V/L $. Primero vemos que es un sistema de generadores. Si $\displaystyle \vec{x} \in V $, queremos ver que la clase de $\displaystyle \vec{x} $ se puede expresar como combinación lineal de estas clases. Existen $\displaystyle a^{j}\in\K $ tales que:
	\[ \vec{x} = \underbrace{ a^{1}\vec{u}_{1}+ \cdots + a^{r}\vec{u}_{r}}_{\in L}+ a^{r+1}\vec{u}_{r+1}+\cdots+a^{n}\vec{u}_{n}  .\]
	Entonces, $\displaystyle \vec{x} $ está relacionado con $\displaystyle a^{r+1}\vec{u}_{r+1}+\cdots+a^{n}\vec{u}_{n} $. Por tanto, 
	\[\vec{x} + L = \left(a^{r+1}\vec{u}_{r+1} + \cdots + a^{n}\vec{u}_{n} \right)+L = a^{r+1}\left(\vec{u}_{r+1}+L\right)+\cdots+a^{n}\left(\vec{u}_{n}+L\right).\]
	Por tanto, $\displaystyle \left\{ \vec{u}_{r+1} + L, \ldots, \vec{u}_{n}+ L\right\}  $ es un sistema de generadores de $\displaystyle V/L $. Ahora tenemos que ver que son linealmente independientes. 
	\[b^{r+1}\left(\vec{u}_{r+1}+L\right) + \cdots + b^{n}\left(\vec{u}_{n}+L\right) = \vec{0} + L.\]
Tenemos que
\[(b^{r+1}\vec{u}_{r+1}+ \cdots + b^{n}\vec{u}_{n}) + L = \vec{0} + L .\]
Entonces, 
\[b^{r+1}\vec{u}_{r+1} + \cdots + b^{n}\vec{u}_{n} - \vec{0} \in L .\]
Entonces, podemos escribir la expresión anterior como combinación lineal de la base de $\displaystyle L $: 
\[b^{r+1}\vec{u}_{r+1}+\cdots +b^{n}\vec{u}_{n} = b^{1}\vec{u}_{1} + \cdots + b^{r}\vec{u}_{r} .\]
\[\Rightarrow \left(-b^{1}\right)\vec{u}_{1} + \left(-b^{2}\right)\vec{u}_{2} + \cdots + \left(-b^{r}\right)\vec{u}_{r} + \cdots + b^{n}\vec{u}_{n} = \vec{0} .\]
Dado que $\displaystyle \left\{ \vec{u}_{1}, \ldots, \vec{u}_{r}, \ldots, \vec{u}_{n}\right\}  $ es una base de $\displaystyle V $ , $\displaystyle b^{1} = b^{2} = \cdots = b^{r} = \cdots = b^{n} = 0 $.
\end{proof}

\begin{observation}
\normalfont Podemos formar la base de un espacio cociente $\displaystyle V/L $ a partir de la expansión de la base de $\displaystyle L $ para obtener una base de $\displaystyle V $.
\end{observation}

