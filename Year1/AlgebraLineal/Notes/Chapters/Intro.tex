\section{Introducción}

El cuerpo de los números reales cumple los siguientes requisitos:\\ \\
$\displaystyle \left(\R, +\right) $ es un grupo abeliano:\\ 
Definimos suma y producto como
\[
\begin{split}
	+ : \R^{2} &\to \R \\
	\left(a,b\right)&\to a+b \\
	\cdot : \R^{2} &\to\R \\
	\left(a,b\right)&\to a \cdot b.
\end{split}
\]
\begin{enumerate}
\item La suma es asociativa
	\[\forall a,b,c \in \R, \; \left(a+b\right)+c = a+ \left(b+c\right) .\]
\item Existe un elemento neutro
	\[\exists!0 \in \R, \; \forall a \in \R, \; 0+a=a+0=a .\]
\item Existe el opuesto
	\[\forall a \in \R, \; \exists -a \in \R, \; a + \left(-a\right) = \left(-a\right)+a = 0 .\]
\item La suma es conmutativa
	\[\forall a,b\in \R, \; a+b = b+a .\]
\item El producto es asociativo, 
	\[\forall a,b,c\in \R, \; a\cdot\left(b\cdot c\right)=\left(a\cdot b\right)\cdot c .\]
\item El producto es distributivo con respecto a la suma (distributivo por la izquierda y por la derecha),
	\[\forall a,b,c \in \R, \; a\cdot\left(b+c\right)=a\cdot b+ a \cdot c .\]
\item Existe la unidad,
	\[\exists! 1 \in \R, \forall a \in \R, 1\cdot a = a \cdot 1 = a .\]
\item Existe la inversa, 
	\[\forall a \in \R-\{0\} \footnote{Utilizamos la notación $\displaystyle \R^{*} $  por sencillez para denotar $\displaystyle \R- \left\{ 0\right\}  $} , \exists a^{-1}\in \R, a\cdot a^{-1}=a^{-1}\cdot a = 1 .\]
\end{enumerate}

\begin{fdefinition}[Anillo]
\normalfont Se denomina \textbf{anillo} a un conjunto y dos operaciones $\displaystyle \left(R, +, \cdot\right) $ que verifica las propiedades (1)-(6). Si se verifica también (7), se llama \textbf{anillo con unidad}. 
\end{fdefinition}

\begin{fdefinition}[Cuerpo]
\normalfont Se denomina \textbf{cuerpo} a un conjunto con al menos dos elementos ($\displaystyle 1 \neq0 $) y dos operaciones $\displaystyle \left(R, +, \cdot\right) $ que cumple las propiedades (1)-(8). Si también se verifica que la multiplicación es commutativa, decimos que se trata de un \textbf{cuerpo abeliano}. 
\end{fdefinition}

\begin{fdefinition}[]
\normalfont Un conjunto $\displaystyle V \neq \emptyset $ es un $\displaystyle \R $-espacio vectorial si existen dos operaciones 
\[
\begin{split}
	& +: V\times V \to V, \; \left(\vec{x},\vec{y}\right)\to \vec{x} + \vec{y} \\ 
	& \cdot : \R \times V \to V, \; \left(a, \vec{x}\right) \to a \cdot \vec{x}
\end{split}
\]
que verifican que
\begin{description}
\item[(i)] $\displaystyle \left(V, +\right) $ es un grupo abeliano.
\item[(ii)] Se cumple la propiedad distributiva, 
	\[\forall a \in \R, \forall \vec{x}, \vec{y} \in V, \; a\left(\vec{x}+\vec{y}\right) = a\vec{x}+a\vec{y} .\]
\item[(iii)] Se cumple otra propiedad distributiva, 
	\[\forall a, b \in \R, \; \vec{x}\in V, \; \left(a+b\right)\vec{x} = a\vec{x} + b\vec{x}.\]\item[(iv)] Se cumple la propiedad asociativa,
	\[\forall a,b \in \R, \; \vec{x} \in V, \; a\left(b\vec{x}\right) = \left(a\cdot b\right)\vec{x} .\]
\item[(v)] $\displaystyle \exists 1 \in \R, \forall \vec{x} \in V, \; 1 \cdot \vec{x} = \vec{x} $. 
\end{description}
\end{fdefinition}

\begin{fdefinition}[]
\normalfont Se define $\displaystyle \R^{n} $, con $\displaystyle n \in \N $, como
\[\R^{n}= \left\{ \left(x_{1},x_{2}, \ldots, x_{n}\right) : \; x_{i}\in\R\right\}  .\]
\end{fdefinition}

\begin{fdefinition}[]
\normalfont Se define la suma $\displaystyle + $ en $\displaystyle \R^{n} $ de la siguiente manera:
\[\forall \vec{x}, \vec{y}, \; \vec{x} + \vec{y} = \left(x_{1}+y_{1}, x_{2}+y_{2}, \ldots, x_{n}+y_{n}\right) .\]
\end{fdefinition}

Utilizamos las propiedades de $\displaystyle \R $ como cuerpo abeliano para justificar que $\displaystyle \left(\R^{n}, +\right) $ es un grupo abeliano. 

\begin{fdefinition}[]
\normalfont Definimos el producto escalar en $\displaystyle \R^{n} $ de la siguiente manera,
\[\forall a \in \R, \forall \vec{x} \in \R^{n}, \; a \cdot \vec{x} = \left(a\cdot x_{1}, a\cdot x_{2}, \ldots, a\cdot x_{n}\right) .\]
\end{fdefinition}

Una consecuencia clara de esto es que para todo $\displaystyle \vec{x}\in \R^{n} $ se cumple que 
\[0 \cdot \vec{x} = \vec{0} .\]
Al igual que antes, podemos utilizar las propiedades de $\displaystyle \left(\R, +, \cdot\right) $ como cuerpo abeliano para justificar que $\displaystyle \R^{n} $ es un $\displaystyle \R $-espacio vectorial. \\ \\
Por las definiciones anteriores tenemos que para todo $\displaystyle \vec{x} \in \R^{n} $, 
\[
\begin{split}
 \vec{x} & = \left(x_{1}, x_{2}, \ldots, x_{n}\right) \\
& = \left(x_{1}, 0, \ldots, 0\right)+\left(0, x_{2}, \ldots, 0\right) + \cdots + \left(0, \ldots, x_{n}\right) \\
& = x_{1}\left(1, 0, \ldots, 0\right)+x_{2}\left(0,1, \ldots, 0\right) + \cdots + x_{n}\left(0, \ldots, 1\right)..
\end{split}
\]

Además, podemos concluir que si 
\[x_{1}\left(1, \ldots, 0\right)+x_{2}\left(0,1, \ldots,0\right)+\cdots+x_{n}\left(0, \ldots , 1\right) = y_{1}\left(1, \ldots, 0\right)+y_{2}\left(0,1, \ldots,0\right)+\cdots+y_{n}\left(0, \ldots , 1\right) ,\]
entonces $\displaystyle \forall i, 1\leq i \leq n $, $\displaystyle x_{i}=y_{i} $. 

\begin{fdefinition}[Sistema de ecuaciones homogéneo]
\normalfont Sea $\displaystyle H $ un sistema de ecuaciones homogéneo:
\[
\begin{split}
& a_{1}^{1}x_{1}+a_{2}^{1}x_{2}+\cdots+a_{n}^{1}x_{n}=0 \\
& \vdots \\
& a^{m}_{1}x_{1}+a^{m}_{2}+\cdots+a^{m}_{n}x^{n}=0. 
\end{split}
\]
donde $\displaystyle m,n \in \N $ y $\displaystyle a^{j}_{i}\in\R $. Definimos $\displaystyle L $ como el conjunto de soluciones de $\displaystyle H $:
\[L = \left\{  \left(x_{0}^{1}, x_{0}^{2}, \ldots, x^{n}_{0}\right)\; : \; x_{0}^{1}, x^{2}_{0}, \ldots, x^{n}_{0}\; \text{es solución de $\displaystyle H $ }\right\} \subset \R^{n} .\]
\footnote{ $\displaystyle a^{i}_{j}  $ no es exponente sino una forma de numeración.} 
\end{fdefinition}

\begin{ftheorem}[]
\normalfont Si $\displaystyle \vec{x_{0}}, \vec{y_{0}}\in L $, se cumple que 
\[\vec{x_{0}}+\vec{y_{0}}\in L .\]
\end{ftheorem}

\begin{proof}
Tenemos que $\displaystyle \forall i, \; 1 \leq i \leq m $, 
\[
\begin{split}
& a^{i}_{1}\left(x_{0}^{1}+y_{0}^{1}\right) + \cdots + a^{i}_{n}\left(x^{n}_{0}+y^{n}_{0}\right) \\
= & a^{i}_{1}x^{1}_{0}+a^{i}_{1}y^{1}_{0}+\cdots+a^{i}_{n}x^{n}_{0}+a^{i}_{n}y^{n}_{0} \\
= & \underbrace{\left(a^{i}_{1}x_{0}^{1}+a^{i}_{2}x^{2}_{0}+\cdots+a^{i}_{n}x^{n}_{0}\right)}_{0} + \underbrace{(a^{i}_{1}y^{1}_{0}+a^{i}_{2}y^{2}_{0}+\cdots+a^{i}_{n}y^{n}_{0})}_{0}\\
= & 0 .
\end{split}
\]
\end{proof}

\begin{ftheorem}[]
\normalfont Si $\displaystyle \vec{x_{0}}\in L $ y $\displaystyle a \in \R $, se cumple que 
\[a \vec{x_{0}}=\left(ax_{0}^{1}, ax^{2}_{0}, \ldots, a x^{n}_{0}\right) \in L .\]
\end{ftheorem}

\begin{proof}
Tenemos que para $\displaystyle \forall i, \; 1 \leq i \leq m $, 
\[
\begin{split}
& a^{i}_{1}\left(ax^{1}_{0}\right)+a^{i}_{2}\left(ax^{2}_{0}\right)+\cdots+a^{i}_{n}\left(ax^{n}_{0}\right) \\
= & a \underbrace{(a^{i}_{1}x^{1}_{0}+a^{i}_{2}x^{2}_{0}+\cdots+a^{i}_{n}x^{n}_{0})}_{0} \\
= & a \cdot 0 \\
= & 0 .
\end{split}
\]
\end{proof}

\begin{ftheorem}[]
\normalfont Por lo visto anteriormente, $\displaystyle L \subset \R^{n} $ es un \textbf{subespacio vectorial} sobre $\displaystyle \R $. 
\end{ftheorem}

\begin{proof}
Muchas de las propiedades de un espacio vectorial automáticamente se heredan a un subespacio vectorial. Las únicas excepciones son la definición de la suma, del producto y la existencia del elemento neutro $\displaystyle 0 $. En este caso, hemos comprobado que la suma está definida en $\displaystyle L $ y que existe la multiplicación $\displaystyle \cdot:\R\times L \to L  $ definida en $\displaystyle L $. Además, $\displaystyle \vec{0}\in L $ es una solución trivial.
\end{proof}

Consideramos un sistema de ecuaciones no homogéneo $\displaystyle S $:
\[
\begin{split}
& a_{1}^{1}x_{1}+a_{2}^{1}x_{2}+\cdots+a_{n}^{1}x_{n}=b^{1} \\
& \vdots \\
& a^{m}_{1}x_{1}+a^{m}_{2}+\cdots+a^{m}_{n}x^{n}=b^{m}. 
\end{split}
\]
Consideramos que $\displaystyle \mathcal{L}\subset \R^{n} $ es el conjunto de las soluciones. 
\[\mathcal{L} = \left\{ \left(x^{1}_{0}, x^{2}_{0}, \ldots, x^{n}_{0}\right)\; : \; x^{1}_{0}, x^{2}_{0}, \ldots, x^{n}_{0}\; \text{es solución de $\displaystyle S $ }\right\}  .\]

Entonces, ya no se cumple necesariamente que la suma de dos soluciones también es solución. Si $\displaystyle \vec{x_{0}}, \vec{y_{0}}\in\mathcal{L} $, $\displaystyle \forall j, \; 1\leq j \leq m$, 
\[
\begin{split}
&a^{j}_{1}\left(x^{1}_{0}+y^{1}_{0}\right) + a^{j}_{2}\left(x^{2}_{0}+y^{2}_{0}\right)+\cdots + a^{j}_{0}\left(x^{n}_{0}+y^{n}_{0}\right)\\
= & \left(a^{j}_{1}x^{1}_{0}+a^{j}_{2}x^{2}_{0}+\cdots+a^{j}_{n}x^{n}_{0}\right) + \left(a^{j}_{1}y^{1}_{0}+a^{j}_{2}y^{2}_{0}+\cdots+a^{j}_{n}y^{n}_{0}\right) \\
= & b^{j}+b^{j} = 2 b^{j} \neq b^{j}.
\end{split}
\]

Si $\displaystyle \vec{X_{0}}\in L $ y $\displaystyle \vec{x_{0}}\in\mathcal{L} $, tenemos que 
\[\vec{X_{0}}+\vec{x_{0}} = b^{j}\in\mathcal{L} .\]

