\chapter{Espacios afines}
\begin{fdefinition}[Espacio afín]
\normalfont Un conjunto $\displaystyle \mathcal{A} \neq \emptyset $ tiene una estructura de \textbf{espacio afín} asociado a $\displaystyle V $ si se tiene definida una aplicación
\[
\begin{split}
	+ : \mathcal{A} \times V \to & \mathcal{A} \\
	\left(A, \vec{u}\right) \to & A + \vec{u}
\end{split}
\]
que cumple las siguientes propiedades:
\begin{description}
\item[(1)] $\displaystyle \forall A \in \mathcal{A} $, $\displaystyle \forall \vec{u}, \vec{v} \in V $, $\displaystyle \left(A + \vec{u}\right) + \vec{v} = A + \left(\vec{u} + \vec{v}\right) $.
\item[(2)] $\displaystyle \forall A \in \mathcal{A} $, $\displaystyle A + \vec{0} = A $.
\item[(3)] $\displaystyle \forall A,B \in \mathcal{A} $, $\displaystyle \exists! \overrightarrow{AB} \in V $ tal que $\displaystyle A + \overrightarrow{AB} = B $.
\end{description}
\end{fdefinition}
\begin{observation}
\normalfont Si $\displaystyle \mathcal{A} $ es un espacio afín asociado a $\displaystyle V $, $\displaystyle \forall \vec{u} \in V $, la traslación de vector $\displaystyle \vec{u} $ es la aplicación
\[
\begin{split}
	\tau_{\vec{u}} : \mathcal{A} \to & \mathcal{A} \\
	A \to & A + \vec{u}.
\end{split}
\]
\end{observation}
\begin{fprop}[]
\normalfont 
\begin{description}
\item[(a)] $\displaystyle \tau_{\vec{0}} = id _{\mathcal{A}} $.
\item[(b)] $\displaystyle \forall \vec{u}, \vec{v} \in V $, $\displaystyle \tau_{\vec{u}}\circ\tau_{\vec{v}} = \tau_{\vec{u} +\vec{v}} $.
\item[(c)] Las traslaciones son biyectivas.
\item[(d)] \textbf{Identidad de Chasles:} $\displaystyle \forall A,B,C \in \mathcal{A} $, 
	\[\overrightarrow{AB} = \overrightarrow{AC} +\overrightarrow{CB} .\]
\item[(e)] $\displaystyle \forall A \in \mathcal{A} $, $\displaystyle \vec{0} = \overrightarrow{AA} $.
\item[(f)] $\displaystyle \forall A,B \in \mathcal{A} $, $\displaystyle \overrightarrow{AB} = -\overrightarrow{BA} $.
\item[(g)] $\displaystyle \forall A,B \in \mathcal{A} $, $\displaystyle \forall \vec{u} \in V $, $\displaystyle \overrightarrow{A\left(B+\vec{u}\right)} = \overrightarrow{AB} + \vec{u}$. Similarmente, $\displaystyle \overrightarrow{A\left(\overrightarrow{AB}+\vec{u}\right)} = B + \vec{u}$.
\item[(h)] $\displaystyle \overrightarrow{\left(A+\vec{u}\right)\left(B+\vec{v}\right)} = \overrightarrow{AB} + \left(\vec{v}-\vec{u}\right)$.
\item[(i)] $\displaystyle \forall A,B,C,D \in \mathcal{A} $, $\displaystyle \overrightarrow{AB} = \overrightarrow{CD} \iff \overrightarrow{AC} = \overrightarrow{BD}$.
\end{description}
\end{fprop}
\begin{proof}
\begin{description}
\item[(a)] Trivial.
\item[(b)] Se deduce directamente de la condición \textbf{(1)} de la definición anterior.
\item[(c)] Se deduce fácilmente de la condición \textbf{(3)} de la definición anterior.
\item[(d)]
	\[\left(A + \overrightarrow{AC}\right) + \overrightarrow{CB} = C + \overrightarrow{CB} = B .\]
\item[(e)] Basta ver que $\displaystyle A + \vec{0} = A $.
\item[(f)]
	\[\overrightarrow{AB} + \overrightarrow{BA} = \overrightarrow{AA} = \vec{0} .\]	
\end{description}
\end{proof}
\begin{observation}
\normalfont De \textbf{(b)} obtenemos que $\displaystyle \tau_{\vec{u}}\circ\tau_{-\vec{u}} = \tau_{\vec{0}}= id _{\mathcal{A}} $.
\end{observation}
Sea $\displaystyle O \in \mathcal{A} $, definimos la aplicación
\[
\begin{split}
	\varphi_{O} : V \to & \mathcal{A} \\
	\vec{u} \to & O + \vec{u}.
\end{split}
\]
\begin{fprop}[]
\normalfont $\displaystyle \varphi_{O} $ es biyectiva.
\end{fprop}
\begin{observation}
\normalfont Definimos la aplicación  
\[
\begin{split}
	\displaystyle \psi_{O} : \mathcal{A} \to  & V \\
	A \to & \overrightarrow{OA}.
\end{split}
\]
Tenemos que $\displaystyle \varphi_{O}\circ\psi_{O} = id _{\mathcal{A}} $ y $\displaystyle \psi_{O}\circ\varphi_{O} = id _{V} $.
\end{observation}
\begin{proof}
$\displaystyle \forall A \in \mathcal{A} $, se tiene que 
\[\varphi_{O} \circ\psi_{O}\left(A\right) = \varphi_{O}\left(\overrightarrow{OA}\right) = O + \overrightarrow{OA} = A = id _{\mathcal{A}}\left(A\right) .\]
Similarmente, $\displaystyle \forall \vec{u} \in V $, 
\[\psi_{O}\circ\varphi_{O}\left(\vec{u}\right) =\psi_{O}\left(O + \vec{u}\right) = \overrightarrow{O O} + \vec{u} = \vec{u} = id _{V}\left(\vec{u}\right).\]
\end{proof}
Consideremos el par $\displaystyle \left\{ O, \left\{ \vec{u}_{1}, \ldots, \vec{u}_{n}\right\} \right\}  $. Si $\displaystyle \vec{x} = x^{1}\vec{u}_{1} + \cdots + x^{n}\vec{u}_{n} $, tenemos que la aplicación
\[
\begin{split}
 f : V \to & \K^{n} \\
 \vec{x} \to & \left(x^{1}, \ldots, x^{n}\right)
\end{split}
\]
es un isomorfismo. 
\begin{observation}
\normalfont Tenemos que 
\[
\begin{split}
	f\circ\psi_{O} : \mathcal{A} \to & \K^{n} \\
	A \to & \left(x^{1}, \ldots, x^{n}\right),
\end{split}
\]
donde $\displaystyle \overrightarrow{OA} = x^{1}\vec{u}_{1} + \cdots +x^{n}\vec{u}_{n}  $. 
\end{observation}
\begin{fdefinition}[Sistema de coordenadas cartesianas]
	\normalfont El par $\displaystyle \left(O, \left\{ \vec{u}_{1}, \ldots, \vec{u}_{n}\right\} \right) $ siendo $\displaystyle O \in \mathcal{A} $ y $\displaystyle \left\{ \vec{u}_{1}, \ldots, \vec{u}_{n}\right\}  $ base de $\displaystyle V $, lo llamaremos \textbf{sistema de coordenadas cartesianas}. 
\end{fdefinition}
\begin{observation}
\normalfont Sean $\displaystyle \left(O, \left\{ \vec{u}_{1}, \ldots, \vec{u}_{n}\right\} \right) $ y $\displaystyle \left(O', \left\{ \vec{v}_{1}, \ldots, \vec{v}_{n}\right\} \right) $ dos sistemas de referencia cartesianos. Supongamos que 
\[
\begin{split}
	\overrightarrow{O O '} = & a^{1}\vec{u}_{1} + \cdots + a^{n}\vec{u}_{n} \\
	\vec{v}_{1} = & a^{1}_{1}\vec{u}_{1} + \cdots + a^{n}_{1}\vec{u}_{n} \\
	\vdots & \\
	\vec{v}_{n} = & a^{1}_{n}\vec{u}_{1} + \cdots + a^{n}_{n}\vec{u}_{n}	.
\end{split}
\]
Si $\displaystyle X \in \mathcal{A} $, supongamos que 
\[\overrightarrow{OX} = x^{1}\vec{u}_{1} + \cdots + x^{n}\vec{u}_{n}, \quad \overrightarrow{O'X}=x'^{1}\vec{v}_{1} + \cdots + x'^{n}\vec{v}_{n} .\]
Vamos a estudiar cómo cambiar de un sistema de referencia a otro. 
\[
\begin{split}
	\overrightarrow{O'X} = & x'^{1}\vec{v}_{1} + \cdots x'^{n}\vec{v}_{n} \\
	= & x'^{1}\left(a^{1}_{1}\vec{u}_{1} + \cdots + a^{n}_{1}\vec{u}_{n}\right) + \cdots + x'^{n}\left(a^{1}_{n} \vec{u}_{1} + a^{n}_{n}\vec{u}_{n}\right) \\
	= & \left(x'^{1}a^{1}_{1} + \cdots + x'^{n}a^{1}_{n}\right)\vec{u}_{1} + \cdots + \left(x'^{1}a^{n}_{1} + \cdots + x'^{n}a^{n}_{n}\right)\vec{u}_{n} \\
	= & \overrightarrow{OX}-\overrightarrow{O O'}.
\end{split}
\]
Similarmente,
\[
\begin{split}
	\overrightarrow{OX} = & x^{1}\vec{u}_{n} + \cdots + x^{n}\vec{u}_{n} \\
	= & \left(a^{1} + x'^{1}a^{1}_{1} + \cdots + x'^{n}a_{n}^{1}\right)\vec{u}_{1} + \cdots + \left(a^{n} + x'^{1}a_{1}^{n} + \cdots + x'^{n}a^{n}_{n}\right)\vec{u}_{n} .
\end{split}
\]
Así, matricialmente obtenemos que
\[\begin{pmatrix} x^{1} \\ x^{2} \\ \vdots \\ x^{n} \end{pmatrix} = \begin{pmatrix} a^{1} \\ a^{2} \\ \vdots \\ a^{n} \end{pmatrix} + \begin{pmatrix} a^{1}_{1} & \cdots & a^{1}_{n} \\
\vdots & & \vdots \\
a^{n}_{1} & \cdots & a^{n}_{n}\end{pmatrix}\begin{pmatrix} x'^{1} \\ x'^{2} \\ \vdots \\ x'^{n} \end{pmatrix} .\]
Otra forma de escribirlo es,
\[ \begin{pmatrix} 1 \\ x^{1} \\ x^{2} \\ \vdots \\ x^{n} \end{pmatrix} =  \begin{pmatrix} 1 & 0 & \cdots & 0 \\a^{1} & a^{1}_{1}  & \cdots & a^{1}_{n} \\
\vdots & \vdots & \vdots & \vdots \\
a^{n} & a^{n}_{1} & \cdots & a^{n}_{n}\end{pmatrix}\begin{pmatrix} 1 \\ x'^{1} \\ x'^{2} \\ \vdots \\ x'^{n} \end{pmatrix}.\]
\end{observation}
\section{Subespacios afines}
\begin{fdefinition}[Subespacios afines]
\normalfont Un subconjunto $\displaystyle \mathcal{L}  $ de $\displaystyle \mathcal{A} $ es un \textbf{subespacio afín} si $\displaystyle \mathcal{L} = \emptyset $ o $\displaystyle \exists L \in \mathcal{L}\left(V\right) $ tal que $\displaystyle \forall A \in \mathcal{L} $, $\displaystyle \forall \vec{u} \in L $, $\displaystyle A + \vec{u} \in \mathcal{L} $ es decir, si $\displaystyle \left(\mathcal{L}, +|_{\mathcal{L}\times L}\right) $ es un espacio afín asociado a $\displaystyle L $.
\end{fdefinition}
\begin{observation}
\normalfont Si $\displaystyle \mathcal{L} \neq \emptyset $ es un subespacio afín y $\displaystyle A \in \mathcal{L} $, es fácil ver que $\displaystyle B \in \mathcal{L} \iff\overrightarrow{AB} \in L$. 
\end{observation}
\begin{fdefinition}[Variedad lineal afín]
	\normalfont Se dice que $\displaystyle \mathcal{L} = A + L = \left\{ A + \vec{u} \; : \; \vec{u} \in L\right\}$ con $\displaystyle L \in \mathcal{L}\left(V\right) $ es una \textbf{variedad lineal afín}.
\end{fdefinition}
\begin{observation}
\normalfont La variedad lineal afín que pasa por $\displaystyle A $ y tiene por dirección $\displaystyle L \in \mathcal{L}\left(V\right) $ es $\displaystyle A + L $.
\[ \mathcal{L} = A + L = \left\{ A +\vec{u} \; : \; \vec{u} \in L\right\} .\]
\end{observation}
\begin{observation}
	\normalfont Las variedades lineales afines son subespacios afines. Por tanto, se tiene que $\displaystyle B \in A + L \iff \overrightarrow{AB} \in L$.
\end{observation}
\begin{fprop}[]
\normalfont $\displaystyle B \in A + L \Rightarrow A + L = B + L $.
\end{fprop}
\begin{proof}
	Si $\displaystyle B \in A + L $ se tiene que $\displaystyle \overrightarrow{AB} \in L $. Así,
	\[ C \in A + L \iff \overrightarrow{AC} \in L \iff \overrightarrow{BC} = \underbrace{\overrightarrow{BA}}_{\in L} + \underbrace{\overrightarrow{AC}}_{\in L} \iff C \in B + L .\]
\end{proof}
\begin{observation}
	\normalfont Si $\displaystyle \mathcal{L} = A + L $ es una variedad lineal afín y $\displaystyle P,Q \in A + L $, entonces $\displaystyle L = \left\{ \overrightarrow{PQ} \; : \; P,Q \in \mathcal{L}\right\}  $. En efecto, si $\displaystyle P,Q \in A + L $, podemos considerar $\displaystyle A + L = P + L $. Además, dado $\displaystyle \vec{u} \in L $, existe un único $\displaystyle Q \in \mathcal{L} $ tal que $\displaystyle P + \vec{u} = Q $, por lo que $\displaystyle Q \in \mathcal{L} $.  
\end{observation}
\begin{fdefinition}[Dimensión de una variedad afín]
\normalfont Si $\displaystyle \mathcal{L} \neq \emptyset $ es una variedad lineal afín de dirección $\displaystyle L \in \mathcal{L}\left(V\right) $ diremos que $\displaystyle \dim\left(\mathcal{L}\right) = \dim\left(L\right) $.
\end{fdefinition}
\begin{ftheorem}[]
	\normalfont Sean $\displaystyle \mathcal{L}_{1} = A + L_{1} $ y $\displaystyle \mathcal{L}_{2} = B + L_{2} $. Entonces, $\displaystyle \mathcal{L}_{1} \cap \mathcal{L}_{2} \neq \emptyset \iff \overrightarrow{AB} \in L_{1} + L_{2}$.
\end{ftheorem}
\begin{proof}
\begin{description}
	\item[(i)] Si $\displaystyle \mathcal{L}_{1} \cap \mathcal{L}_{2} \neq \emptyset $, tenemos que $\displaystyle \exists C \in \mathcal{L}_{1} \cap \mathcal{L}_{2} $. Así, $\displaystyle \overrightarrow{AC} \in L_{1} $ y $\displaystyle \overrightarrow{BC} \in L_{2} $. Así, tenemos que 
		\[\overrightarrow{AB} = \underbrace{\overrightarrow{AC}}_{\in L_{1}} + \underbrace{\overrightarrow{CB}}_{\in L_{2}} .\]
		Así, $\displaystyle \overrightarrow{AB} \in L_{1} + L_{2} $.
	\item[(ii)] Si $\displaystyle \overrightarrow{AB} \in L_{1} + L_{2} $, $\displaystyle \exists \vec{v}_{1} \in L_{1}, \vec{v}_{2} \in L_{2} $ tal que $\displaystyle \overrightarrow{AB} = \vec{v}_{1} +\vec{v}_{2} $. Tenemos que $\displaystyle A + \vec{v}_{1} \in \mathcal{L}_{1} $. Así,
		\[ \overrightarrow{\left(A + \vec{v}_{1}\right)B} = \overrightarrow{AB} - \vec{v}_{1} = \vec{v}_{1} +\vec{v}_{2} -\vec{v}_{1} = \vec{v}_{2} \in L_{2} .\]
		Por tanto, $\displaystyle A + \vec{v}_{1} \in B + L_{2} = \mathcal{L}_{2}$. Por tanto, $\displaystyle A + \vec{v}_{1} \in \mathcal{L}_{1} \cap \mathcal{L}_{2} $.
\end{description}
\end{proof}
\begin{fprop}[]
\normalfont Sean $\displaystyle \mathcal{L}_{i} = A + L_{i} $ para $\displaystyle i \in I $. Entonces, $\displaystyle \bigcap_{i\in I}\mathcal{L}_{i} $ es una variedad lineal afín que, si no es vacío, tiene por dirección $\displaystyle \bigcap_{i \in I}L_{i} \in \mathcal{L}\left(V\right) $.
\end{fprop}
\begin{proof}
Supongamos que $\displaystyle \bigcap_{i \in I}\mathcal{L}_{i} \neq \emptyset $ y sea $\displaystyle A \in \bigcap_{i \in I}\mathcal{L}_{i} $. Entonces $\displaystyle \forall i \in I $ se tiene que $\displaystyle \mathcal{L}_{i} = A + L_{i} $. Tenemos que 
\[ B \in \bigcap_{i \in I}\mathcal{L}_{i} \iff B \in \mathcal{L}_{i} = A + L_{i}, \; \forall i \in I \iff \overrightarrow{AB} \in L_{i}, \; \forall i \in I \iff \overrightarrow{AB} \in \bigcap_{i \in I}L_{i}.\]
Así, hemos visto que $\displaystyle \bigcap_{i \in I}\mathcal{L}_{i} = A+ \bigcap_{i \in I}L_{i} $.
\end{proof}
\begin{fdefinition}[]
\normalfont Dos variedades lineales afines $\displaystyle \mathcal{L}_{1} = A + L_{1} $ y $\displaystyle \mathcal{L}_{2} = B + L_{2} $ son \textbf{complementarias} si lo son sus direcciones, es decir, si $\displaystyle L_{1} \oplus L_{2} = V $.
\end{fdefinition}
\begin{eg}
	\normalfont Consideremos en $\displaystyle \R^{4} $ los planos $\displaystyle \rho_{1} $ y $\displaystyle \rho_{2} $. Tenemos que $\displaystyle \rho_{1} = A + P_{1} $ y $\displaystyle \rho_{2} = B + P_{2} $ donde $\displaystyle P_{1} = L\left( \left\{ \vec{u}_{1}, \vec{u}_{2}\right\} \right) $ y $\displaystyle P_{2} = L\left( \left\{ \vec{u}_{3}, \vec{u}_{4}\right\} \right) $, donde $\displaystyle \left\{ \vec{u}_{1}, \vec{u}_{2}\right\}  $ y $\displaystyle \left\{ \vec{u}_{3}, \vec{u}_{4}\right\}  $ son linealmente independientes. 
	\begin{itemize}
		\item Si $\displaystyle \left\{ \vec{u}_{1}, \vec{u}_{2}, \vec{u}_{3}, \vec{u}_{4}\right\}  $ son linealmente independientes, tenemos que $\displaystyle P_{1} \oplus P_{2} = \R^{4} $, por lo que $\displaystyle \rho_{1}$ y $\displaystyle \rho_{1}$ son complementarios. Tenemos que $\displaystyle \rho_{1} \cap \rho_{2} \neq \emptyset $, puesto que $\displaystyle P_{1} + P_{2} = \R^{4}$ y $\displaystyle \overrightarrow{AB} \in \R^{4} $. Así, $\displaystyle \rho_{1} $ y $\displaystyle \rho_{2} $ se cortan en un solo punto.
		\item Supongamos que $\displaystyle \dim\left(L\left( \left\{ \vec{u}_{1}, \vec{u}_{2}, \vec{u}_{3}, \vec{u}_{4}\right\} \right)\right) = 3$ (con $\displaystyle \left\{ \vec{u}_{1}, \vec{u}_{2}, \vec{u}_{3}\right\}  $ linealmente independientes) y que $\displaystyle \dim\left(L\left( \left\{ \vec{u}_{1}, \vec{u}_{2}, \vec{u}_{3},\vec{u}_{4}, \overrightarrow{AB}\right\} \right)\right)=4 $. Entonces, tenemos que $\displaystyle \rho_{1} \cap \rho_{2} = \emptyset $.
	\end{itemize}
\end{eg}
\begin{observation}
	\normalfont Si $\displaystyle A \in \mathcal{A} $ y $\displaystyle L \in \mathcal{L}\left(V\right) $, la variedad lineal afín que pasa por $\displaystyle A $ y tiene por dirección $\displaystyle L $ es $\displaystyle \mathcal{L} = A + L = \left\{ A +\vec{u} \; : \; \vec{u} \in L\right\}  $. Diremos que $\displaystyle \dim\left(\mathcal{L}\right) = \dim\left(L\right) $.
\end{observation}
\begin{fdefinition}[]
\normalfont Dos variedades lineales afines $\displaystyle \mathcal{L}_{1} = A + L_{1} $ y $\displaystyle \mathcal{L}_{2} = B + L _{2} $ son \textbf{paralelas} si $\displaystyle L_{1} \subset L_{2} $ o $\displaystyle L_{2} \subset L_{1} $.
\end{fdefinition}
\begin{fprop}[]
\normalfont Si $\displaystyle \mathcal{L}_{1} = A + L_{1} $ y $\displaystyle \mathcal{L}_{2} = A + L_{2} $ son paralelas, entonces $\displaystyle \mathcal{L}_{1} \cap \mathcal{L}_{2} = \emptyset $ o una de ellas está contenida en la otra. 
\end{fprop}
\begin{proof}
Si $\displaystyle \exists C \in \mathcal{L}_{1} \cap \mathcal{L}_{2} $ supongamos sin pérdida de generalidad que $\displaystyle L_{1} \subset L_{2} $, entonces tenemos que
\[ \mathcal{L}_{1} = C + L_{1}, \; \mathcal{L}_{2}= C + L_{2} .\]
Así, tenemos que $\displaystyle \forall A \in \mathcal{L}_{1} $, $\displaystyle \exists \vec{u} \in L_{1} $ tal que $\displaystyle A = C + \vec{u} \in \mathcal{L}_{1} $. Dado que $\displaystyle L_{1} \subset L_{2} $, tenemos que $\displaystyle C + \vec{u} \in C + L_{2} = \mathcal{L}_{2} $. Por tanto, $\displaystyle A \in \mathcal{L}_{2} $. 
\end{proof}
\begin{fdefinition}[Variedad lineal afín suma]
	\normalfont Sean $\displaystyle \mathcal{L}_{1} = A + L_{1} $ y $\displaystyle \mathcal{L}_{2} = B + L_{2} $. Llamaremos \textbf{variedad lineal afín suma} de $\displaystyle \mathcal{L}_{1} $ mas $\displaystyle \mathcal{L}_{2} $ a la variedad lineal $\displaystyle \mathcal{L}_{1} + \mathcal{L}_{2} $ más pequeña que contiene a $\displaystyle \mathcal{L}_{1} \cup \mathcal{L}_{2} $.
\end{fdefinition}
\begin{observation}
	\normalfont Supongamos que $\displaystyle \mathcal{L}_{1} + \mathcal{L}_{2} = A + L' = B + L' $, entonces $\displaystyle \overrightarrow{AB} \in L'$. Si $\displaystyle \vec{u} \in L_{1} $ tenemos que $\displaystyle A + \vec{u} \in \mathcal{L}_{1} $ por lo que $\displaystyle \vec{u} \in L' $. Similarmente, si $\displaystyle \vec{v} \in L_{2} $, entonces $\displaystyle B + \vec{v} \in \mathcal{L}_{2} $ y $\displaystyle \vec{v} \in L' $. Así, $\displaystyle L\left( \left\{ \overrightarrow{AB}\right\} \right) + L_{1} + L_{2} \subset L' $. Así, $\displaystyle \mathcal{L}_{1} + \mathcal{L}_{2} = A + \left(L_{1} + L_{2} + L\left( \left\{ \overrightarrow{AB}\right\} \right)\right) $. 
\end{observation}
\begin{ftheorem}[Fórmula de Grassman]
\normalfont Sean $\displaystyle \mathcal{L}_{1} $ y $\displaystyle \mathcal{L}_{2} $ variedades lineales afines. Entonces
\begin{description}
\item[(a)] Si $\displaystyle \mathcal{L}_{1} \cap \mathcal{L}_{2} \neq \emptyset $, $\displaystyle \dim\left(\mathcal{L}_{1} + \mathcal{L}_{2}\right) = \dim\left(\mathcal{L}_{1}\right) + \dim\left(\mathcal{L}_{2}\right) - \dim\left(\mathcal{L}_{1} \cap \mathcal{L}_{2}\right) $.
\item[(b)] Si $\displaystyle \mathcal{L}_{1} \cap \mathcal{L}_{2} = \emptyset $, $\displaystyle \dim\left(\mathcal{L}_{1} + \mathcal{L}_{2}\right) = \dim\left(\mathcal{L}_{1}\right) + \dim\left(\mathcal{L}_{2}\right) - \dim\left(L_{1} \cap L_{2}\right) + 1 $.
\end{description}
\end{ftheorem}
\begin{proof} Tenemos que $\displaystyle \dim\left(\mathcal{L}_{1} + \mathcal{L}_{2}\right) = \dim\left(L_{1} + L_{2} + L\left( \left\{ \overrightarrow{AB}\right\} \right)\right) $.
\begin{description}
	\item[(a)] Si $\displaystyle \mathcal{L}_{1} \cap \mathcal{L}_{2} \neq \emptyset $, tenemos que $\displaystyle L\left( \left\{ \overrightarrow{AB}\right\} \right) \subset L_{1} + L_{2}$. Así, tenemos que 
		\[
		\begin{split}
			\dim\left(L_{1} + L_{2} + L\left( \left\{ \overrightarrow{AB}\right\} \right)\right) = & \dim\left(L_{1}\right) + \dim\left(L_{2}\right) - \dim\left(L_{1} \cap L_{2}\right) \\
			= &  \dim\left(\mathcal{L}_{1}\right)+ \dim\left(\mathcal{L}_{2}\right) - \dim\left(\mathcal{L}_{1} \cap \mathcal{L}_{2}\right) . .
		\end{split}
		\]
	\item[(b)] Si $\displaystyle \mathcal{L}_{1} \cap \mathcal{L}_{2} = \emptyset $, tenemos que $\displaystyle \overrightarrow{AB} \not\in L_{1} + L_{2} $, por lo que 
		\[\dim\left(\left(L_{1}+L_{2}\right) + L\left( \left\{ \overrightarrow{AB}\right\} \right)\right) = \dim\left(L_{1}\right) + \dim\left(L_{2}\right) -\dim\left(L_{1}\cap L_{2}\right) + 1.\]
Si $\displaystyle \mathcal{L}_{1} \cap \mathcal{L}_{2} = \emptyset $, tenemos que 
\[\dim\left(\mathcal{L}_{1}\right) + \dim\left(\mathcal{L}_{2}\right) - \dim\left(L_{1}\cap L_{2}\right)+1 .\]
\end{description}
\end{proof}
\begin{eg}
\normalfont En $\displaystyle \R^{3} $ consideramos $\displaystyle \rho_{1} = A + R $ con $\displaystyle \dim\left(R\right)=1 $ y $\displaystyle \rho_{2} = B + P $ con $\displaystyle P \in \mathcal{L}\left(\R^{3}\right) $ y $\displaystyle \dim\left(P\right) = 2 $. 
\begin{itemize}
\item Si $\displaystyle \rho_{1} \cap \rho_{2} = \emptyset $, tenemos que $\displaystyle \dim\left(\rho_{1} + \rho_{2}\right) = 4 - \dim\left(P \cap R\right) $, por tanto, $\displaystyle \dim\left(P \cap R\right) = 1 $ y $\displaystyle R \subset P $.
\item Si $\displaystyle \rho_{1} \cap \rho_{2} \neq \emptyset $, tenemos que $\displaystyle \dim\left(\rho_{1} + \rho_{2}\right) = 2 + 1 - \dim\left(P \cap R\right) $. 
\end{itemize}
\end{eg}
\begin{observation}
\normalfont 
Sean $\displaystyle \left\{ A_{0}, A_{1}, \ldots, A_{p}\right\} \subset \mathcal{A} $. Tenemos que 
\[ \mathcal{L} = A_{0} + L\left( \left\{ \overrightarrow{A_{0}A_{1}}, \ldots, \overrightarrow{A_{0}A_{p}}\right\} \right) .\]
\end{observation}
\begin{fprop}[]
	\normalfont Sean $\displaystyle \left\{ A_{0}, A_{1}, \ldots, A_{p}\right\}  \subset \mathcal{A} $. Los siguientes enunciados son equivalentes:
	\begin{description}
		\item[(a)] $\displaystyle \left\{\overrightarrow{A_{0}A_{1}}, \ldots, \overrightarrow{A_{0}A_{p}}\right\} $ son linealmente indepependientes.
		\item[(b)] $\displaystyle \forall i = 0, \ldots, p $, $\displaystyle \left\{ \overrightarrow{A_{i}A_{0}}, \ldots, \overrightarrow{A_{i}A_{i - 1}}, \overrightarrow{A_{i}A_{i+1}}, \ldots, \overrightarrow{A_{i}A_{p}}\right\}  $ son linealmente independientes.
		\item[(c)] $\displaystyle \forall O \in \mathcal{A} $, si 
			\[ \lambda^{0}\overrightarrow{OA_{0}} + \lambda^{1}\overrightarrow{OA_{1}} + \cdots +\lambda_{p} \overrightarrow{OA_{p}} = \vec{0}.\]
			\[\lambda^{0} + \lambda^{1} + \cdots + \lambda^{p} = 0 .\]
			Entonces, $\displaystyle \lambda^{0}= \lambda^{1} = \cdots = \lambda^{ p}=0 $.
	\end{description}
\end{fprop}
\begin{proof}
\begin{description}
\item[(a) $\displaystyle \Rightarrow $ (b)] Sea $\displaystyle i = 0, \ldots, p $ y
	\[ \lambda^{0}\overrightarrow{A_{i}A_{0}} +  \cdots  + \lambda^{i-1} \overrightarrow{A_{i}A_{i - 1}} + \lambda^{i+1} \overrightarrow{A_{i}A_{i+1}} +  \cdots  + \lambda^{p} \overrightarrow{A_{i}A_{p}} = \vec{0} .\]
	Así, tenemos que
	\[ - \lambda^{0}\overrightarrow{A_{0}A_{i}} + \lambda^{1}\left(\overrightarrow{A_{0}A_{1}}-\overrightarrow{A_{0}A_{i}}\right) + \cdots + \lambda^{p}\left(\overrightarrow{A_{0}A_{p}} - \overrightarrow{A_{0}A_{i}}\right) = \vec{0} .\]
Así, tenemos que
\[ = - \sum^{p}_{i \neq j = 0}\lambda^{j}\overrightarrow{A_{0}A_{i}} + \lambda^{1}\overrightarrow{A_{0}A_{1}} + \cdots + \lambda^{i-1}\overrightarrow{A_{0}A_{i-1}} + \lambda^{i+1}\overrightarrow{A_{0}A_{i+1}}+ \cdots + \lambda^{p}\overrightarrow{A_{0}A_{p}} = \vec{0}.\]
Así, tenemos que $\displaystyle \lambda^{j} = 0 $ si $\displaystyle i \neq j $, por lo que $\displaystyle \sum^{p}_{i \neq j=0}\lambda^{j}= 0 $, por lo que $\displaystyle \lambda^{i} = 0 $. 
\item[(b) $\displaystyle \Rightarrow $ (a)] Trivial.
\item[(a) $\displaystyle \Rightarrow $ (c)] Sean $\displaystyle \lambda^{0}, \lambda^{1}, \ldots, \lambda^{p} \in \K $ tales que  
\[ \lambda^{0}\overrightarrow{OA_{0}} + \lambda^{1}\overrightarrow{OA_{1}} + \cdots + \lambda^{p}\overrightarrow{OA_{p}} = \vec{0},\quad \lambda^{0} + \lambda^{1} + \cdots + \lambda^{p} = 0.\]
	Por tanto, tenemos que 
\[\lambda^{0} = - \lambda^{1}-\lambda^{2}-\cdots - \lambda^{p} .\]
Así, tenemos que 
\[\lambda^{1}\left(\overrightarrow{OA_{1}}-\overrightarrow{OA_{0}}\right) + \lambda^{2}\left(\overrightarrow{OA_{2}}-\overrightarrow{OA_{0}}\right) + \lambda^{p}\left(\overrightarrow{OA_{p}}-\overrightarrow{OA_{0}}\right) = \lambda^{1}\overrightarrow{A_{0}A_{1}} + \cdots + \lambda^{p}\overrightarrow{A_{0}A_{p}} = \vec{0} .\]
Así, tenemos que $\displaystyle \lambda^{1} = \cdots = \lambda^{p} = 0 $, por lo que $\displaystyle \lambda^{0} = 0 $.
\item[(c) $\displaystyle \Rightarrow $ (a)] Consideremos 
	\[ \lambda^{1}\overrightarrow{A_{0}A_{1}} + \cdots + \lambda^{p}\overrightarrow{A_{0}A_{p}} = \vec{0}.\]
	Sea $\displaystyle O \in \mathcal{A} $, tenemos que 
	\[ \lambda^{1}\left(\overrightarrow{OA_{1}}-\overrightarrow{OA_{0}}\right) + \lambda^{2}\left(\overrightarrow{OA_{2}}-\overrightarrow{OA_{0}}\right) + \lambda^{p}\left(\overrightarrow{OA_{p}}-\overrightarrow{OA_{0}}\right) = \vec{0}.\]
Tenemos entonces que
\[ \overrightarrow{OA_{0}} \left(-\lambda^{1}-\cdots -\lambda^{p}\right) + \lambda^{1}\overrightarrow{OA_{1}} + \cdots + \lambda^{p}\overrightarrow{OA_{p}} = \vec{0}.\]
Así, tenemos que 
\[ \left(-\lambda^{1} - \cdots - \lambda^{p}\right) + \lambda^{1}+\lambda^{2} +\cdots + \lambda^{p}=0 \Rightarrow \lambda^{i} = 0, \; i = 0, \ldots, p .\]
\end{description}
\end{proof}
\begin{fdefinition}[]
	\normalfont Diremos que que $\displaystyle \left\{ A_{0}, A_{1}, \ldots, A_{p}\right\} \subset \mathcal{A} $ son \textbf{afinmente independientes} si $\displaystyle \left\{ \overrightarrow{A_{0}A_{1}}, \ldots, \overrightarrow{A_{0}A_{p}}\right\}  $ son linealmente independientes.
\end{fdefinition}
\begin{fprop}[]
\normalfont Si $\displaystyle \left\{ A_{0}, \ldots, A_{p}\right\} \subset \mathcal{A} $ son afinmente independientes, existen $\displaystyle A_{p + 1}, \ldots, A_{n} \in\mathcal{A} $ tales que $\displaystyle \left\{ A_{0}, A_{1}, \ldots, A_{p}, A_{p + 1}, \ldots, A_{n}\right\}  $ son afimente independientes.
\end{fprop}
\begin{proof}
	Sean $\displaystyle \left\{ \overrightarrow{A_{0}A_{1}}, \ldots, \overrightarrow{A_{0}A_{p}}\right\}  $ linealmente independientes. Sea $\displaystyle \left\{ \overrightarrow{A_{0}A_{1}}, \ldots, \overrightarrow{A_{0}A_{p}}, \vec{u}_{p + 1}, \ldots, \vec{u}_{n}\right\} $ base de $\displaystyle V $. Entonces tomamos $\displaystyle A_{p + 1}= A_{0} + \vec{u}_{p +1} $  y, en general, $\displaystyle A_{i} = A_{0} + \vec{u}_{i} $.
\end{proof}
\section{Ecuaciones cartesianas y paramétricas}
\subsection*{Cálculo de las ecuaciones cartesianas y paramétricas de una variedad lineal afín}
Sea $\displaystyle \left(O, \left\{ \vec{u}_{1}, \ldots, \vec{u}_{n}\right\} \right) $ un sistema de referencia cartesiano de $\displaystyle \mathcal{A} $ y sea $\displaystyle \mathcal{L} = A + L\left( \left\{ \vec{v}_{1}, \ldots, \vec{v}_{p}\right\} \right) $ con $\displaystyle \dim\left(\mathcal{L}\right) = p $. Tenemos que existen $\displaystyle a^{1}, \ldots, a^{p} \in \K $ tales que
\[ \overrightarrow{OA} = a^{1}\vec{u}_{1} + \cdots + a^{n}\vec{u}_{n}.\]
Como $\displaystyle \left\{ \vec{v}_{1}, \ldots, \vec{v}_{p}\right\} \subset V $, se tiene que
\[
\begin{cases}
\vec{v}_{1} = a^{1}_{1}\vec{u}_{1} + \cdots + a^{n}_{1}\vec{u}_{n} \\
\vdots \\
\vec{v}_{p} = a^{1}_{p}\vec{u}_{1} + \cdots + a^{n}_{p}\vec{u}_{n}
\end{cases}
.\]
Si $\displaystyle X \in \mathcal{A} $, tenemos que existen $\displaystyle x^{1}, \ldots, x^{n} \in \K $ tales que $\displaystyle \overrightarrow{OX} = x^{1}\vec{u}_{1} + \cdots + x^{n}\vec{u}_{n} $. Ahora, tenemos que $\displaystyle X \in A + L\left( \left\{ \vec{v}_{1} + \cdots + \vec{v}_{p}\right\} \right) \iff \overrightarrow{AX} \in L\left( \left\{ \vec{v}_{1}, \ldots, \vec{v}_{p}\right\} \right) \iff \exists \lambda^{1}, \ldots, \lambda^{p} \in \K	$ tales que
\[
\begin{split}
	\overrightarrow{OX} - \overrightarrow{OA} = \overrightarrow{AX} = \lambda^{1}\vec{v}_{1} + \cdots + \lambda^{p}\vec{v}_{p} \iff & \left(x^{1}-a^{1}\right)\vec{u}_{1} + \cdots + \left(x^{n} - a^{n}\right)\vec{u}_{n} \\
	= & \lambda^{1}\left(a^{1}_{1}\vec{u}_{1} + \cdots + a^{n}_{1}\vec{u}_{n}\right) + \cdots + \lambda^{p}\left(a^{1}_{p}\vec{u}_{1} + \cdots + a^{n}_{p}\vec{u}_{n}\right) \\
	= & \left(\lambda^{1}a^{1}_{1}+\cdots +\lambda^{2}a^{1}_{p}\right)\vec{u}_{1} + \cdots + \left(\lambda^{1}a^{n}_{1} + \cdots + \lambda^{p}a^{n}_{p}\right)\vec{u}_{n}.
\end{split}
\]
Así, nos quedan las \textbf{ecuaciones paramétricas} de la variedad lineal afín $\displaystyle A + L $:
\[
\begin{cases}
x^{1}-a^{1} = \lambda^{1}a^{1}_{1} + \cdots + \lambda^{p}a^{1}_{p} \\
\vdots \\
x^{n}-a^{n} = \lambda^{1}a^{n}_{1} + \cdots + \lambda^{p}a^{n}_{p}
\end{cases}
.\]
Por el teorema de Rouché-Fröbenius, esto es cierto si y solamente si
\[ \ran\begin{pmatrix} a^{1}_{1} & \cdots & a^{1}_{p} & x^{1}-a^{1} \\
\vdots & \vdots & \vdots & \vdots \\
a^{n}_{1} & \cdots & a^{n}_{p} & x^{n}-a^{n}\end{pmatrix} = \ran\begin{pmatrix} a^{1}_{1} & \cdots & a^{1}_{p} \\
\vdots & & \vdots \\
a^{n}_{1} & \cdots & a^{n}_{p}\end{pmatrix} = p.\]
	Suponemos que $\displaystyle \begin{vmatrix} a^{1}_{1} & \cdots & a^{1}_{p} \\
	\vdots & & \vdots \\
a^{p}_{1} & \cdots & a^{p}_{p}\end{vmatrix} = \alpha \neq 0 $. Lo visto anteriormente es cierto si y solo si 
		\[ \begin{vmatrix} a^{1}_{1} & \cdots & a^{1}_{p} & x^{1}-a^{1} \\
		\vdots & \vdots & \vdots & \vdots \\
	a^{p}_{1} & \cdots & a^{p}_{p} & x^{p}-a^{p} \\
a^{p + i}_{1} & \cdots & a^{ p + i}_{ p + i} & x^{ p + i}- a^{p + i}\end{vmatrix}  = 0 , \; i = 1, \ldots, n - p.\]
Así, tenemos que 
\[
\begin{cases}
 \left(x^{1}-a^{1}\right)b^{1}_{1} + \cdots + \left(x^{n}-a^{n}\right)b^{n} +\left(x^{p + 1}-a^{p + 1}\right)\alpha = 0 \\
 \vdots \\
 \left(x^{1}-a^{1}\right)b^{n-p}_{1} + \cdots + \left(x^{p}-a^{p}\right)b^{n-p} + \left(x^{n}-a^{n}\right)\alpha = 0.
\end{cases}
.\]
Tenemos, pues, las \textbf{ecuaciones cartesianas} de la variedad lineal afín $\displaystyle A + L $: 
\[
\begin{cases}
c^{1}_{1}x^{1} + \cdots + c^{1}_{p}x^{p} + \alpha x^{p + 1} = d^{1} \\
\vdots \\
c^{n - p}_{1}x^{1} + \cdots + c^{n-p}_{p}x^{p} + \alpha x^{p + 1} = d^{n - p} 
\end{cases}
.\]
Así, tenemos que $\displaystyle \det\left(\alpha\left(I_{\left(n-p\right)\times\left(n-p\right)}\right)\right) = \alpha^{n-p} \neq 0 $. 
\subsection*{Determinación de una variedad lineal afín a partir de sus ecuaciones}
Ahora vamos a ver que los sistemas lineales definen variedades lineales afines. Consideremos el mismo sistema de referencia cartesiano, $\displaystyle \left(O, \left\{ \vec{u}_{1}, \ldots, \vec{u}_{n}\right\} \right) $ y el sistema de ecuaciones
\[\left(S\right)
\begin{cases}
a^{1}_{1}x^{1} + \cdots + a^{1}_{n}x^{n} = b^{1} \\
\vdots \\
a^{n}_{1} x^{1} + \cdots + a^{n}_{n}x^{n} = b^{n}
\end{cases}
.\]
Ahora, consideremos $\displaystyle \mathcal{L} = \left\{ C \in \mathcal{A} \; : \; \overrightarrow{OC} = c^{1}\vec{u}_{1} + \cdots + c^{n}\vec{u}_{n}, \; \left(c^{1}, \ldots, c^{n}\right) \; \text{solución de }\left(S\right)\right\}  $. Consideremos ahora el sistema 
\[\left(H\right) 
\begin{cases}
a^{1}_{1}x^{1} + \cdots + a^{1}_{p}x^{p}= 0 \\
\vdots \\
a^{n-p}_{1}x^{1} + \cdots + a^{n-p}_{p}x^{p}= 0
\end{cases}
.\]
Tomamos $\displaystyle L = \left\{ \vec{x} = x^{1}\vec{u}_{1} + \cdots + x^{n}\vec{u}_{n} \; : \; \left(x^{1}, \ldots, x^{n}\right) \; \text{solución de }\left(H\right)\right\}  \in \mathcal{L}\left(V\right) $. Si $\displaystyle \mathcal{L} = \emptyset $, es una variedad lineal afín. Ahora, si $\displaystyle A \in \mathcal{L} $, tenemos que $\displaystyle \forall B \in \mathcal{L} $, $\displaystyle B = A + \overrightarrow{AB} $ y $\displaystyle \overrightarrow{AB} \in L $. Tenemos que $\displaystyle \dim\left(L\right) = n - \left(n-p\right) = p $. Supongamos que existe $\displaystyle A \in \mathcal{L} $ tal que $\displaystyle \mathcal{L}= A + L $.
\section{Aplicaciones afines}
Sean los $\displaystyle \K- $espacios vectoriales $\displaystyle V_{1} $ y $\displaystyle V_{2} $, y $\displaystyle \mathcal{A}_{1} $ y $\displaystyle \mathcal{A}_{2} $ sus espacios afines asociados, respectivamente. 
\begin{fdefinition}[Aplicación afín]
	\normalfont Una aplicación $\displaystyle f : \mathcal{A} \to \mathcal{A}' $ es \textbf{afín} si existe $\displaystyle \vec{f} : V \to V'$ lineal tal que $\displaystyle f\left(A + \vec{u}\right)= f\left(A\right) + \vec{f}\left(\vec{u}\right) $, $\displaystyle \forall A \in \mathcal{A},\forall\vec{u}\in V $. A $\displaystyle \vec{f} $ se la llama \textbf{aplicación lineal asociada} a $\displaystyle f $.
\end{fdefinition}
\begin{fprop}[]
	\normalfont La aplicación $\displaystyle f : \mathcal{A} \to \mathcal{A}' $ es afín si y solo si existe $\displaystyle \vec{f} : V \to V' $ lineal tal que $\displaystyle \forall A,B \in \mathcal{A} $ se tiene que $\displaystyle \overrightarrow{f\left(A\right)f\left(B\right)} = \vec{f}\left(\overrightarrow{AB}\right) $.
\end{fprop}
\begin{proof}
\begin{description}
	\item[(i)] Supongamos que $\displaystyle f $ es afín. Sean $\displaystyle A, B \in \mathcal{A} $. Entonces, tenemos que $\displaystyle B = A + \overrightarrow{AB} $. Por tanto, tenemos que $\displaystyle f\left(B\right) = f\left(A\right) + \vec{f}\left(\overrightarrow{AB}\right) $, por lo que $\displaystyle \overrightarrow{f\left(A\right)f\left(B\right)} = \vec{f}\left(\overrightarrow{AB}\right)$. 
	\item[(ii)] Sean $\displaystyle A \in \mathcal{A} $, $\displaystyle \vec{u} \in V $, y $\displaystyle B = A + \vec{u} $. Tenemos que $\displaystyle \overrightarrow{f\left(A\right)f\left(B\right)} = \vec{f}\left(\overrightarrow{AB}\right)$, por tanto, $\displaystyle f\left(B\right) = f\left(A + \overrightarrow{AB}\right) = f\left(A\right) + \vec{f}\left(\overrightarrow{AB}\right) = f\left(A\right) + \vec{f}\left(\vec{u}\right)$. 
\end{description}
\end{proof}
\begin{ftheorem}[]
\normalfont Sea $\displaystyle l : V\to V'$ lineal y $\displaystyle A \in \mathcal{A} $ y $\displaystyle A' \in \mathcal{A}' $. Entonces, $\displaystyle \exists! f : \mathcal{A}\to \mathcal{A}' $ afín tal que $\displaystyle f\left(A\right) = A' $ y $\displaystyle \vec{f} = l $.
\end{ftheorem}
\begin{proof}
\begin{description}
	\item[Unicidad.] Supognamos que existe $\displaystyle f $ afín tal que $\displaystyle f\left(A\right) = A' $ y $\displaystyle \vec{f} = l $. Tenemos que $\displaystyle \forall B \in \mathcal{A} $,
		\[f\left(B\right) = f\left(A + \overrightarrow{AB}\right) = f\left(A\right) + l\left(\overrightarrow{AB}\right) = A' + l\left(\overrightarrow{AB}\right) .\]
	\item[Existencia.] $\displaystyle \forall B \in \mathcal{A} $ defino $\displaystyle f\left(B\right) = A' + l\left(\overrightarrow{AB}\right) $. Así, si $\displaystyle C \in \mathcal{A} $ se tiene que $\displaystyle f\left(C\right) = A' + l\left(\overrightarrow{AC}\right) $. Por tanto,
	\[ \overrightarrow{f\left(B\right)f\left(C\right)} = \overrightarrow{\left(A'+l\left(\overrightarrow{AB}\right) \right)\left(A'+l\left(\overrightarrow{AC}\right)\right)} = \overrightarrow{A'A'} + l\left(\overrightarrow{AC}\right)-l\left(\overrightarrow{AB}\right) = l\left(\overrightarrow{AC}-\overrightarrow{AB}\right) = l\left(\overrightarrow{BC}\right).\]
	Además, $\displaystyle f\left(A\right)= A' + l\left(\overrightarrow{A'A'}\right) = A' $.
\end{description}
\end{proof}
\begin{fcolorary}[]
\normalfont Si $\displaystyle f,g : \mathcal{A} \to \mathcal{A}' $ son tales que $\displaystyle f\left(A\right)=g\left(A\right) $ para algún $\displaystyle A \in \mathcal{A} $ y $\displaystyle \vec{f} = \vec{g} $, entonces $\displaystyle f = g $.
\end{fcolorary}
\begin{ftheorem}[]
\normalfont Sea $\displaystyle f : \mathcal{A} \to \mathcal{A'} $ una aplicación afín y $\displaystyle \vec{f} : V \to V' $ su aplicación lineal asociada. Entonces,
\begin{description}
\item[(a)] $\displaystyle f $ es inyectiva $\displaystyle \iff  $ $\displaystyle \vec{f} $ es inyectiva.
\item[(b)] $\displaystyle f $ es sobreyectiva $\displaystyle \iff  $ $\displaystyle \vec{f} $ es sobreyectiva.
\item[(c)] $\displaystyle f $ es biyectiva $\displaystyle \iff  $ $\displaystyle \vec{f} $ es biyectiva.
\end{description}
\end{ftheorem}
\begin{proof}
\begin{description}
	\item[(a)] Supongamos que $\displaystyle f $ es inyectiva, $\displaystyle \vec{u} \in \Ker\left(\vec{f}\right)  $ y sea $\displaystyle B = A + \vec{u} $. Tenemos que 
	\[f\left(B\right) = f\left(A + \vec{u}\right) = f\left(A\right) + \vec{f}\left(\vec{u}\right) = f\left(A\right) .\]
	Si $\displaystyle \vec{u} \neq 0 $, tenemos que $\displaystyle B \neq A $, por lo que $\displaystyle f $ no es inyectiva, lo cual es una contradicción. Recíprocamente, supongamos que $\displaystyle \vec{f} $ es inyectiva y $\displaystyle f\left(A\right) = f\left(B\right) $ con $\displaystyle A,B \in \mathcal{A} $. Entonces tenemos que $\displaystyle \overrightarrow{f\left(A\right)f\left(B\right)} = \vec{f}\left(\overrightarrow{AB}\right) = \vec{0} $, por lo que $\displaystyle \overrightarrow{AB} = \vec{0} $ y $\displaystyle A = B $.
\item[(b)] Sea $\displaystyle A \in \mathcal{A} $ y $\displaystyle f\left(A\right) \in \mathcal{A}' $. Supongamos que $\displaystyle f $ es sobreyectiva. Sea $\displaystyle \vec{u'} \in V' $. Entonces, $\displaystyle f\left(A\right) + \vec{u}' \in \mathcal{A}' $, por lo que existe $\displaystyle B \in \mathcal{A} $ tal que $\displaystyle f\left(B\right) = f\left(A\right) + \vec{u}' $, por lo que $\displaystyle \vec{u}' = \overrightarrow{f\left(A\right)f\left(B\right)} = \vec{f}\left(\overrightarrow{AB}\right) \in V'$. \\
	Recíprocamente, supongamos que $\displaystyle \vec{f} $ es sobreyectiva y sea $\displaystyle B' \in \mathcal{A}' $. Tenemos que $\displaystyle \overrightarrow{f\left(A\right)B'} \in V'$, por tanto $\displaystyle \exists \vec{u} \in V $ tal que $\displaystyle \vec{f}\left(\vec{u}\right) = \overrightarrow{f\left(A\right)B} $ y, por tanto, $\displaystyle B' = f\left(A\right) + \vec{f}\left(\vec{u}\right) = A' + \vec{f}\left(\vec{u}\right) $.
\item[(c)] Trivial a partir de \textbf{(a)} y \textbf{(b)}.
\end{description}
\end{proof}
\begin{fprop}[]
	\normalfont Sean $\displaystyle f : \mathcal{A} \to \mathcal{A}' $ y $\displaystyle g : \mathcal{A'} \to \mathcal{A} $ aplicaciones afines, entonces $\displaystyle g\circ f $ es afín y $\displaystyle \overrightarrow{g \circ f} = \vec{g}\circ \vec{f} $.
\end{fprop}
\begin{proof}
Tenemos que $\displaystyle \forall A,B \in \mathcal{A} $, 
\[\overrightarrow{\left(g\circ f\left(A\right)\right)\left(g\circ f\left(B\right)\right)} = \overrightarrow{\left(g\left(f\left(A\right)\right)\right)\left(g\left(f\left(B\right)\right)\right)} = \vec{g}\left(\overrightarrow{f\left(A\right)f\left(B\right)}\right) = \vec{g}\left(\vec{f}\left(\overrightarrow{AB}\right)\right) = \vec{g}\circ\vec{f}\left(\overrightarrow{AB}\right) .\]
\end{proof}
\begin{fdefinition}[Isomorfismo de espacios afines]
\normalfont Sea $\displaystyle f : \mathcal{A} \to \mathcal{A}' $ una aplicación afín. Diremos que $\displaystyle f $ es un \textbf{isomorfismo} de espacios afines si existe $\displaystyle g : \mathcal{A}' \to \mathcal{A} $ afín tal que $\displaystyle g\circ f = id _{\mathcal{A}} $ y $\displaystyle f\circ g = id _{\mathcal{A}'} $.
\end{fdefinition}
\begin{fprop}[]
\normalfont Sea $\displaystyle f : \mathcal{A} \to \mathcal{A}' $ una aplicación afín. Son equivalentes:
\begin{description}
\item[(a)] $\displaystyle f $ es isomorfismo de espacios afines.
\item[(b)] $\displaystyle f $ es biyectiva.
\item[(c)] $\displaystyle \vec{f} $ es isomorfismo de espacios vectoriales.
\end{description}
\end{fprop}
\begin{proof}
\begin{description}
\item[(a) $\displaystyle \Rightarrow $ (b) $\displaystyle \Rightarrow $ (c)] Trivial (la última implicación es trivial por el último teorema).
\item[(c) $\displaystyle \Rightarrow $ (a)] Por el teorema anterior, tenemos que si $\displaystyle \vec{f} $ es biyectiva, entonces $\displaystyle f $ también lo es. Por tanto, sabemos que existe $\displaystyle g = f^{-1} $, ahora tenemos que ver que $\displaystyle g $ es afín. Sea $\displaystyle A \in \mathcal{A} $ y $\displaystyle f\left(A\right) = A' \in \mathcal{A}' $. Como hemos visto, $\displaystyle \exists \vec{f}^{-1} : V' \to V $ lineal. Sea $\displaystyle g : \mathcal{A}' \to \mathcal{A} $ la única aplicación afín tal que $\displaystyle g\left(A'\right) = A $ y $\displaystyle \vec{g} = \vec{f}^{-1} $.
Entonces, 
\[g\circ f\left(A\right) = g\left(f\left(A\right)\right) = g\left(A'\right) = A = id _{\mathcal{A}}\left(A\right), \quad \overrightarrow{g \circ f} = \vec{g}\circ\vec{f} = id _{V} .\]
\[ f\circ g\left(A'\right)= f\left(A\right) = A' = id _{\mathcal{A}'}\left(A'\right), \quad \overrightarrow{f\circ g} = \vec{f}\circ \vec{g} = id _{V'}.\]
\end{description}
\end{proof}
\begin{observation}
\normalfont Sea $\displaystyle f : \mathcal{A} \to \mathcal{A}' $ afín con $\displaystyle \dim\left(\mathcal{A}\right) = \dim\left(\mathcal{A}'\right) $. Entonces, para ver que $\displaystyle f $ es biyectiva basta con ver que $\displaystyle f $ es inyectiva, es decir, basta con comprobar que $\displaystyle \vec{f} $ es inyectiva.
\end{observation}
\begin{fprop}[]
\normalfont Sea $\displaystyle f : \mathcal{A} \to \mathcal{A}' $ una aplicación afín  y $\displaystyle \vec{f} : V \to V' $ su aplicación lineal asociada. 
\begin{description}
\item[(a)] Si $\displaystyle \mathcal{L} = A + L $ es una variedad lineal afín de $\displaystyle \mathcal{A} $, entonces $\displaystyle f\left(\mathcal{L}\right) $ es una variedad lineal afín.
\item[(b)] Si $\displaystyle \mathcal{L}' = A' + L' $ es una variedad lineal afín de $\displaystyle \mathcal{A}' $, entonces $\displaystyle f^{-1}\left(\mathcal{L}'\right) $ es una variedad lineal afín que, si no es vacía \footnote{La inversa no tiene por qué existir, porque $\displaystyle f $ no tiene por qué ser biyectiva.} , tiene por dirección $\displaystyle \vec{f}^{-1}\left(L'\right) $.
\end{description}
\end{fprop}
\begin{proof}
\begin{description}
	\item[(a)] Si $\displaystyle \mathcal{L} = A + L = \left\{ A +\vec{u} \; : \; \vec{u} \in L\right\}  $. Tenemos que 
		\[f\left(\mathcal{L}\right) = \left\{ f\left(A+\vec{u}\right) = f\left(A\right) + \vec{f}\left(\vec{u}\right) \; : \; \vec{u} \in L\right\} = f\left(A\right) + \left\{ \vec{f}\left(\vec{u}\right) \; : \; \vec{u} \in L\right\} = f\left(A\right) + \vec{f}\left(L\right) \in \mathcal{L}\left(\mathcal{L}'\right) .\]
	\item[(b)] Supongamos que $\displaystyle f^{-1}\left(\mathcal{L}'\right) \neq \emptyset $ y sea $\displaystyle A \in f^{-1}\left(\mathcal{L}'\right) $. Esto es cierto si y solo si $\displaystyle f\left(A\right) \in \mathcal{L}' $. Así, tenemos que 
		\[B \in f^{-1}\left(\mathcal{L}'\right) \iff  f\left(B\right) \in \mathcal{L}' = A + L' \iff \overrightarrow{f\left(A\right)f\left(B\right)}=\vec{f}\left(\overrightarrow{AB}\right) \in L' \iff \overrightarrow{AB} \in \vec{f}^{-1}\left(L'\right) .\]
		Así, tenemos que $\displaystyle B \in A + \vec{f}^{-1}\left(L'\right) $. Así, $\displaystyle f^{-1}\left(\mathcal{L}'\right) = A + \vec{f}^{-1}\left(L'\right) $ y $\displaystyle \vec{f}^{-1}\left(L'\right) \in \mathcal{L}\left(V\right) $.
\end{description}
\end{proof}
\subsection{Traslaciones}
Si $\displaystyle \vec{u} \in V $, se define la aplicación
\[
\begin{split}
	\tau_{\vec{u}} : \mathcal{A} \to & \mathcal{A} \\
A \to & \tau_{\vec{u}}\left(A\right) = A +\vec{u}.
\end{split}
\]
Así, $\displaystyle \forall A,B \in \mathcal{A} $,
\[\overrightarrow{\tau_{\vec{u}}\left(A\right)\tau_{\vec{u}}\left(B\right)} = \overrightarrow{\left(A +\vec{u}\right)\left(B + \vec{u}\right)} = \overrightarrow{AB} + \vec{u}-\vec{u} = id _{V}\left(\overrightarrow{AB}\right).\]
\begin{observation}
\normalfont Así, las traslaciones son aplicaciones afines que tienen como aplicación lineal asociada la identidad. Con el siguiente resultado estudiamos el recíproco.
\end{observation}
\begin{ftheorem}[]
\normalfont Sea $\displaystyle f : \mathcal{A} \to \mathcal{A} $ una aplicación lineal afín tal que $\displaystyle \vec{f} = id _{V} $, entonces existe $\displaystyle \vec{u} \in V $ tal que $\displaystyle f = \tau_{\vec{u}} $.
\end{ftheorem}
\begin{proof}
	Sea $\displaystyle A \in \mathcal{A} $ y $\displaystyle \vec{u} = \overrightarrow{Af\left(A\right)} $. Tenemos que $\displaystyle \forall B \in \mathcal{A} $,
	\[\overrightarrow{f\left(A\right)f\left(B\right)} = \vec{f}\left(\overrightarrow{AB}\right)= id _{V}\left(\overrightarrow{AB}\right) = \overrightarrow{AB} .\]
Así, tenemos que 
\[f\left(B\right) = f\left(A\right) + \overrightarrow{AB} \Rightarrow \overrightarrow{Bf\left(B\right)} = \overrightarrow{B\left(f\left(A\right)+\overrightarrow{AB} \right)} = \overrightarrow{Bf\left(A\right)} +\overrightarrow{AB} = \overrightarrow{Af\left(A\right)} =\vec{u} .\]
\end{proof}
\subsection{Funciones afines}
\begin{fdefinition}[Función afín]
\normalfont Una \textbf{función afín} definida en $\displaystyle \mathcal{A} $ es una aplicación afín $\displaystyle f : \mathcal{A} \to \K $.
\end{fdefinition}
Sea $\displaystyle f : \mathcal{A} \to \K $ una aplicación afín, tenemos que $\displaystyle \vec{f} \in V^{*}$, por lo que si $\displaystyle \vec{f} \neq 0 $, se tiene que $\displaystyle \Ker\left(\vec{f}\right) $ es un hiperplano vectorial. Además, se tiene que $\displaystyle \forall \alpha \in \K $,
\[f^{-1}\left( \left\{ \alpha \right\} \right) = \left\{ A \in \mathcal{A} \; :\; f\left(A\right) = \alpha \right\} \neq \emptyset .\]
Si $\displaystyle \alpha \in \K $, $\displaystyle f^{-1}\left( \left\{ \alpha \right\} \right) $ es una variedad lineal afín no vacía de dirección $\displaystyle \vec{f}^{-1}\left( \left\{ 0\right\} \right) = \Ker\left(\vec{f}\right) $. Si $\displaystyle f: \mathcal{A} \to \K $, $\displaystyle \forall \alpha \in \K $, $\displaystyle f^{-1}\left( \left\{ \alpha \right\} \right) $ es un hiperplano de $\displaystyle \mathcal{A} $. En efecto, el conjunto $\displaystyle \left\{ \alpha \right\} $ es una variedad lineal de $\displaystyle \K $ de dirección $\displaystyle \left\{ \vec{0}\right\}  $, por lo que $\displaystyle f^{-1}\left( \left\{ \alpha \right\} \right) $ es una variedad afín de $\displaystyle \mathcal{A} $ de dirección $\displaystyle \vec{f}^{-1}\left( \left\{ \vec{0}\right\} \right) $.
\begin{fprop}[]
\normalfont 
Recíprocamente, si $\displaystyle \mathcal{H} $ es un hiperplano de dirección $\displaystyle H $, $\displaystyle \forall \alpha \in \K $, $\displaystyle \exists f : \mathcal{A} \to \K $ afín tal que $\displaystyle \mathcal{H} = f^{-1}\left( \left\{ \alpha \right\} \right) $.
\end{fprop}
\begin{proof}
Sea $\displaystyle A \in \mathcal{H} $ y $\displaystyle f : \mathcal{A} \to \K $ la única aplicación afín tal que $\displaystyle f\left(A\right) = \alpha  $ y tiene por aplicación lineal asociada $\displaystyle l \in V^{*} $ con $\displaystyle \Ker\left(l\right) = H $.
\end{proof}
\begin{fprop}[]
	\normalfont Sea $\displaystyle f : \mathcal{A} \to \mathcal{A} $ una aplicación afín. Entonces $\displaystyle \mathcal{L} = \left\{ A \in \mathcal{A} \; : \; f\left(A\right) = A\right\}  $ es una variedad lineal afín que, si no es vacía, tiene por dirección $\displaystyle L = \left\{ \vec{u} \in V \; : \; \vec{f}\left(\vec{u}\right) = \vec{u}\right\} $.
\end{fprop}
\begin{proof}
Si $\displaystyle \mathcal{L} = \emptyset $, no hay nada que demostrar. Supongamos, entonces, que $\displaystyle \mathcal{L} \neq \emptyset $ y sea $\displaystyle A \in \mathcal{L} $. Tenemos que si $\displaystyle B \in \mathcal{A} $, entonces 
\[ B \in \mathcal{L} \iff f\left(B\right) = B \iff \overrightarrow{Af\left(B\right)} =\overrightarrow{AB} =\vec{f}\left(\overrightarrow{AB} \right)\iff \overrightarrow{AB} \in L.\]
Así, hemos visto que $\displaystyle \mathcal{L} = A + L $.
\end{proof}
\subsection{Homotecias}
\begin{fdefinition}[Homotecia]
	\normalfont Sea $\displaystyle \alpha \in \K / \left\{ 0,1\right\}  $. Una aplicación afín $\displaystyle f : \mathcal{A} \to \mathcal{A} $ es una \textbf{homotecia} si su aplicación lineal asociada es la homotecia vectorial de razón $\displaystyle \alpha  $. 
\end{fdefinition}
\begin{observation}
\normalfont 
\[
\begin{split}
	\vec{f} : V & \to V\\
	\vec{x} & \to \alpha \vec{x}.
\end{split}
\]
\end{observation}
\begin{ftheorem}[]
	\normalfont Sea $\displaystyle f $ una homotecia de razón $\displaystyle \alpha  $ ($\displaystyle \alpha \not\in \left\{ 0,1\right\}  $), entonces $\displaystyle \exists ! C \in \mathcal{A} $ tal que $\displaystyle f\left(C\right)= C $.
\end{ftheorem}
\begin{proof}
\begin{description}
\item[Unicidad.] Supongamos que existe $\displaystyle C \in \mathcal{A} $ tal que $\displaystyle f\left(C\right) = C $. Tenemos que $\displaystyle \forall A \in \mathcal{A} $, 
	\[f\left(A\right) = f\left(C + \overrightarrow{CA} \right) = f\left(C\right) + \vec{f}\left(\overrightarrow{CA} \right) = C + \alpha \overrightarrow{CA}  .\]
Entonces, tenemos que 
\[\overrightarrow{Af\left(A\right)}  = \overrightarrow{A\left(C + \alpha \overrightarrow{CA} \right)} = \overrightarrow{AC} + \alpha \overrightarrow{CA} = \left(1-\alpha \right)\overrightarrow{AC}  .\]
Por tanto, $\displaystyle \overrightarrow{AC} = \frac{1}{1-\alpha }\overrightarrow{Af\left(A\right)}  $.
\item[Existencia.] Sea $\displaystyle A \in \mathcal{A} $ y sea $\displaystyle C = A + \frac{\overrightarrow{Af\left(A\right)} }{1-\alpha } $. Tenemos que 
	\[
	\begin{split}
		f\left(C\right) = & f\left(A + \frac{\overrightarrow{Af\left(A\right)} }{1-\alpha }\right) = f\left(A\right) +\frac{1}{1-\alpha }\vec{f}\left(\overrightarrow{Af\left(A\right)} \right) = f\left(A\right)+\frac{\alpha }{1-\alpha }\overrightarrow{Af\left(A\right)} \\
		= &  A + \overrightarrow{Af\left(A\right)} +\frac{\alpha }{1-\alpha }\overrightarrow{Af\left(A\right)} = A + \left(1+\frac{\alpha }{1-\alpha }\right)\overrightarrow{Af\left(A\right)} = A + \frac{1}{1-\alpha }\overrightarrow{Af\left(A\right)} = C  .
	\end{split}
	\]
	Así, la homotecia de centro $\displaystyle C $ y razón $\displaystyle \alpha \in \K / \left\{ 0,1\right\}  $ está definida por $\displaystyle f\left(A\right) = C + \alpha \overrightarrow{CA}  $, $\displaystyle \forall A \in \mathcal{A} $.	
\end{description}
\end{proof}
\begin{observation}
	\normalfont El conjunto $\displaystyle \mathcal{H} = \left\{ f : \mathcal{A} \to \mathcal{A} \; : \; f \; \text{homotecia o traslación}\right\}  $ es un grupo que tiene como operación la comosición de funciones.
\end{observation}
\begin{observation}
\normalfont Una aplicación afín es una homotecia o traslación si y solo si su aplicación lineal asociada es una homotecia vectorial, esto es cierto si y solo si $\displaystyle \vec{f} $ deja invariante todas las rectas vectoriales, es decir, si para cualquier recta $\displaystyle \mathcal{R} $, $\displaystyle f\left(\mathcal{R}\right) $ es una recta paralela a $\displaystyle \mathcal{R} $.
\end{observation}
\begin{fdefinition}[Simetría]
\normalfont Sea $\displaystyle C \in \mathcal{A} $. La \textbf{simetría} respecto de $\displaystyle C $ es la homotecia de centro $\displaystyle C $ y razón $\displaystyle -1 $. 
\end{fdefinition}
\begin{fprop}[]
\normalfont 
Si $\displaystyle f $ es la simetría respecto de $\displaystyle C $, $\displaystyle f $ es involutiva ($\displaystyle f^{2} = id _{\mathcal{A}} $).
\end{fprop}
\begin{proof}
Si $\displaystyle f $ es la simetría respecto de $\displaystyle C $, $\displaystyle f^{2}\left(C\right) = C = id _{\mathcal{A}}\left(C\right) $.
Además, $\displaystyle \vec{f} = - id _{V} $, por lo que $\displaystyle \vec{f}^{2} = id _{V} $. Por tanto tenemos que $\displaystyle f^{2} = id _{\mathcal{A}} $.
\end{proof}
\begin{observation}
\normalfont Si $\displaystyle f : \mathcal{A} \to \mathcal{A} $ es una homotecia involutiva, entonces $\displaystyle f^{2} = id _{\mathcal{A}} $ y $\displaystyle \vec{f}^{2} = id _{V} $, por lo que $\displaystyle \alpha^{2} = 1 $ y $\displaystyle \alpha = -1 $.
\end{observation}
\subsection{Proyecciones}
\begin{fdefinition}[Proyección]
\normalfont Una \textbf{proyección} es una aplicación afín $\displaystyle f : \mathcal{A} \to \mathcal{A} $ tal que $\displaystyle f = f^{2}$.
\end{fdefinition}
\begin{observation}
\normalfont Si $\displaystyle f $ es una proyección, $\displaystyle \vec{f}  $ será una proyección vectorial. En efecto, tenemos que
\[
\begin{split}
	f\left(A + \vec{u}\right) = & f\left(A\right) + \vec{f}\left(\vec{u}\right) \\
	f^{2}\left(A + \vec{u}\right) = & f\left(f\left(A\right) + \vec{f}\left(\vec{u}\right)\right) = f^{2}\left(A\right) + \vec{f}^{2}\left(\vec{u}\right).
\end{split}
\]
Como $\displaystyle f\left(A\right) = f^{2}\left(A\right) $, es trivial que $\displaystyle \vec{f}\left(\vec{u}\right) = \vec{f}^{2}\left(\vec{u}\right) $. Así, tenemos que $\displaystyle \vec{f^{2}} = \left(\vec{f}\right)^{2} = \vec{f} $. El recíproco no es cierto.
\end{observation}
\begin{ftheorem}[]
	\normalfont Una aplicación $\displaystyle f : \mathcal{A} \to \mathcal{A} $ es una proyección si y solo si existen dos subespacios vectoriales suplementarios $\displaystyle L_{1} $ y $\displaystyle L_{2} $ con $\displaystyle L_{1} \oplus L_{2} = V $, y una variedad lineal afín $\displaystyle \mathcal{L} $ de dirección $\displaystyle L_{1} $ tal que 
	\[ \left\{ f\left(A\right)\right\} = \mathcal{L} \cap \left(A+L_{2}\right), \; \forall A \in \mathcal{A} .\]
\end{ftheorem}
\begin{proof}
\begin{description}
\item[(i)] Sabemos que si $\displaystyle f $ es una proyección, entonces $\displaystyle \vec{f} $ es una proyección vectorial, por lo que existen $\displaystyle L_{1}, L_{2} \in \mathcal{L}\left(V\right) $ tales que $\displaystyle L_{1}\oplus L_{2} = V $, con $\displaystyle L_{1} = \Imagen\left(\vec{f}\right) $ y $\displaystyle L_{2} = \Ker\left(\vec{f}\right) $. \\
	Tenemos que $\displaystyle \forall A \in \mathcal{A} $, $\displaystyle f^{2}\left(A\right) = f\left(A\right) $, por lo que todo punto de la imagen de $\displaystyle f $ es invariante por $\displaystyle f $. Así, tenemos que la variedad lineal afín $\displaystyle \mathcal{L} = \left\{ A \in \mathcal{A} \; : \; f\left(A\right) = A\right\}  \neq \emptyset $, por lo que tiene como dirección a $\displaystyle L_{1} $. \\
	Ahora, tenemos que $\displaystyle \forall B \in \mathcal{A} $, 
	\[ \vec{f}\left(\overrightarrow{Bf\left(B\right)}\right) = \overrightarrow{f\left(B\right)f^{2}\left(B\right)} = \vec{0} .\]
	Por tanto, $\displaystyle \overrightarrow{Bf\left(B\right)} \in L_{2} $ y $\displaystyle f\left(B\right) \in B + L_{2} $. Como vimos antes, también tenemos que $\displaystyle f\left(B\right) \in \mathcal{L} $. Así, hemos obtenido que $\displaystyle f\left(B\right) $ es el único punto que pertenece a la intersección de las variedades lineales afines suplementarias $\displaystyle \mathcal{L} $\footnote{Es fácil comprobar que $\displaystyle \mathcal{L} = \Imagen\left(f\right) $.}  y $\displaystyle B + L_{2} $, es decir, 
	\[ \left\{ f\left(B\right)\right\} = \mathcal{L} \cap \left(B + L_{2}\right) .\]
\item[(ii)] Consideremos una variedad lineal afín $\displaystyle \mathcal{L} $ de dirección $\displaystyle L_{1} $ y $\displaystyle L_{1} \oplus L_{2} = V $. Vamos a ver que la aplicación $\displaystyle f : \mathcal{A} \to \mathcal{A}$ definida por $\displaystyle \left\{ f\left(B\right)\right\} = \mathcal{L} \cap \left\{ B + L_{2}\right\}  $, $\displaystyle \forall B \in \mathcal{A} $, es una proyección. \\
	Primero vamos a ver que $\displaystyle f $ es afín. Sea $\displaystyle p \in \End\left(V\right) $ la proyección vectorial de base $\displaystyle L_{1} $ y dirección $\displaystyle L_{2} $. Si $\displaystyle A,B \in \mathcal{A} $, 
\[
\begin{split}
f\left(A\right) \in \mathcal{L}, \; \overrightarrow{Af\left(A\right)} \in L_{2} \\
f\left(B\right) \in \mathcal{L}, \; \overrightarrow{Bf\left(B\right)} \in L_{2}.
\end{split}
\]
Así, tenemos que 
\[B = A + \overrightarrow{AB} = A + \overrightarrow{Af\left(A\right)} + \overrightarrow{f\left(A\right)f\left(B\right)} + \overrightarrow{f\left(B\right)B} .\]
Como $\displaystyle \overrightarrow{Af\left(A\right)}, \overrightarrow{f\left(B\right)B} \in L_{2} $ y $\displaystyle \overrightarrow{f\left(A\right)f\left(B\right)} \in L_{1} $, se tiene que $\displaystyle p\left(\overrightarrow{AB}\right) = \overrightarrow{f\left(A\right)f\left(B\right)} $. Así, $\displaystyle f $ es una aplicación afín que tiene como aplicación lineal asociada a $\displaystyle p $. \\
Ahora vamos a ver que $\displaystyle f^{2} = f $. En efecto, tenemos que $\displaystyle \forall A \in \mathcal{A} $, $\displaystyle f\left(A\right) \in \mathcal{L} $, y como todo punto de $\displaystyle \mathcal{L} $ es invariante, se tiene que $\displaystyle f\left(f\left(A\right)\right) = f\left(A\right) $.
\end{description}
\end{proof}
\begin{fdefinition}[]
\normalfont Diremos que $\displaystyle f $ es la proyección de la \textbf{base} $\displaystyle \mathcal{L} $ y \textbf{dirección} $\displaystyle L_{2} $.
\end{fdefinition}
\subsection{Simetrías}
\begin{fdefinition}[Simetría]
\normalfont Una aplicación afín $\displaystyle f : \mathcal{A} \to \mathcal{A} $ es una simetría si $\displaystyle f^{2} = id _{\mathcal{A}} $.
\end{fdefinition}
\begin{observation}
\normalfont Si $\displaystyle f $ es una simetría, $\displaystyle \vec{f} $ es una simetría vectorial (el recíproco no tiene por qué ser cierto). En efecto, $\displaystyle \left(\vec{f}\right)^{2} = \vec{f^{2}} = id _{V} $. 
\end{observation}
\begin{ftheorem}[]
\normalfont Una aplicación $\displaystyle f : \mathcal{A} \to \mathcal{A} $ es una simetría si y solo si existen $\displaystyle L_{1}, L_{2} \in \mathcal{L}\left(V\right) $ tales que $\displaystyle L_{1} \oplus L_{2} = V $, y una variedad lineal afín $\displaystyle \mathcal{L} $ de dirección $\displaystyle L_{1} $, tales que 
\[ \forall A \in \mathcal{A}, \; f\left(A\right) = A + 2\overrightarrow{AA_{1}} ,\]
donde $\displaystyle A_{1} $ es la proyección de $\displaystyle A $ sobre $\displaystyle \mathcal{L} $ en la dirección $\displaystyle L_{2} $.
\end{ftheorem}
\begin{proof}
\begin{description}
\item[(i)] En primer lugar, vamos a ver que la variedad lineal afín de los puntos invariantes, $\displaystyle \mathcal{L} = \left\{ A \in \mathcal{A} \; : \; f\left(A\right) = A\right\}  $, no es vacía. Sea $\displaystyle A \in \mathcal{A} $ y $\displaystyle M = P_{m}\left(A,f\left(A\right)\right) $ el punto medio de $\displaystyle A $ y $\displaystyle f\left(A\right) $, es decir, 
	\[ M = P_{m}\left(A,f\left(A\right)\right) = A + \frac{1}{2}\overrightarrow{Af\left(A\right)} .\]
Tenemos que $\displaystyle M \in \mathcal{L}$, en efecto:
\[
\begin{split}
	f\left(M\right) = & f\left(A + \frac{1}{2}\overrightarrow{Af\left(A\right)} \right) = f\left(A\right) + \frac{1}{2}\vec{f}\left(\overrightarrow{Af\left(A\right)} \right) = f\left(A\right) +\frac{1}{2}\overrightarrow{f\left(A\right)A}  \\
	= &  A + \overrightarrow{Af\left(A\right)} +\frac{1}{2}\overrightarrow{f\left(A\right)A}  = A + \frac{1}{2}\overrightarrow{Af\left(A\right)}  = M.
\end{split}
\]
Así, hemos visto que $\displaystyle \mathcal{L} \neq \emptyset $. Por otro lado, por ser $\displaystyle \vec{f} $ una simetría, $\displaystyle \exists L_{1}, L_{2} \in \mathcal{L} $ tales que $\displaystyle L_{1}\oplus L_{2} = V $, con $\displaystyle L_{1} = \Ker\left(f-id _{V}\right) $ y $\displaystyle L_{2}=\Ker\left(f+id _{V}\right) $. Además, $\displaystyle \vec{f} = p_{1}-p_{2} $. Así, tenemos que $\displaystyle \mathcal{L} $ tiene por dirección $\displaystyle L_{1} $. \\
Ahora, consideremos $\displaystyle A \in \mathcal{A} $ y sea $\displaystyle A_{1} $ la proyección de $\displaystyle A $ sobre $\displaystyle \mathcal{L} $ en la dirección $\displaystyle L_{2} $. Tenemos que, como $\displaystyle \overrightarrow{AA_{1}}\in L_{2} $,
\[\overrightarrow{f\left(A\right)f\left(A_{1}\right)} = \vec{f}\left(\overrightarrow{AA_{1}} \right) = \left(p_{1}-p_{2}\right)\left(\overrightarrow{AA_{1}}\right) = \overrightarrow{A_{1}A}  .\]
Como $\displaystyle f\left(A_{1}\right) = A_{1} $, tenemos que $\displaystyle \overrightarrow{f\left(A\right)A_{1}}=\overrightarrow{A_{1}A} $, por lo que $\displaystyle A_{1} $ es el punto medio de $\displaystyle A $ y $\displaystyle f\left(A\right) $, y podemos escribir que 
\[ f\left(A\right) = A + 2 \overrightarrow{AA_{1}} .\]
\item[(ii)] Recíprocamente, consideremos la variedad lineal afín $\displaystyle \mathcal{L} $ de dirección $\displaystyle L_{1} $ y $\displaystyle L_{2} \in \mathcal{L}\left(V\right) $ tal que $\displaystyle L_{1} \oplus L_{2} = V $. Vamos a ver que la aplicación definida por 
	\[ f\left(A\right)=A+\frac{1}{2}\overrightarrow{AA_{1}}, \; \forall A \in \mathcal{A} .\]
	donde $\displaystyle A_{1} $ es la proyección de $\displaystyle A $ sobre $\displaystyle \mathcal{L} $ con dirección $\displaystyle L_{2} $, es una simetría. \\
	En primer lugar, vamos a ver que $\displaystyle f $ es afín. Sean $\displaystyle A,B \in \mathcal{A} $. Tenemos que ver que $\displaystyle \overrightarrow{f\left(A\right)f\left(B\right)} = \vec{f}\left(\overrightarrow{AB} \right) $. Tenemos que 
\[f\left(A\right) = A + 2\overrightarrow{AA_{1}}, \quad f\left(B\right) = B + 2\overrightarrow{BB_{1}}  .\]
Podemos ver que si $\displaystyle A \in \mathcal{L} \cap \left(A + L_{2}\right) $, entonces $\displaystyle A = A_{1} $. Ahora, tenemos que \footnote{Recordar que $\displaystyle id _{V} = p_{1} + p_{2} $.} 
\[
\begin{split}
	\overrightarrow{f\left(A\right)f\left(B\right)} = & \overrightarrow{\left(A + 2\overrightarrow{AA_{1}} \right)\left(B + 2\overrightarrow{BB_{1}} \right)} =\overrightarrow{AB} +2\left(\overrightarrow{BB_{1}} -\overrightarrow{AA_{1}} \right) \\
	= &  \overrightarrow{AB} +2\left(\overrightarrow{BA_{1}} +\overrightarrow{A_{1}B_{1}} -\overrightarrow{AB} -\overrightarrow{BA_{1}} \right) 
	= -\overrightarrow{AB} +2\overrightarrow{A_{1}B_{1}} \\
	= &  -\overrightarrow{AB} +2p_{1}\left(\overrightarrow{AB} \right) = \left(-id _{V} + 2 p_{1}\right)\left(\overrightarrow{AB} \right) =\left(-p_{1}-p_{2}  + 2p_{1}\right)\left(\overrightarrow{AB} \right) = \left(p_{1}-p_{2}\right)\left(\overrightarrow{AB} \right) .
\end{split}
\]
Así, hemos visto que $\displaystyle f $ es afín y que su aplicación lineal asociada es $\displaystyle p_{1} - p_{2} $. Ahora vamos a ver que es involutiva. En efecto, tenemos que $\displaystyle A $ y $\displaystyle f\left(A\right) $ tienen la misma proyección sobre $\displaystyle \mathcal{L} $ con dirección $\displaystyle L_{2} $, por lo que $\displaystyle A_{1} $ es el punto medio de $\displaystyle A $ y $\displaystyle f\left(A\right) $ y de $\displaystyle f\left(A\right) $ y $\displaystyle f^{2}\left(A\right) $. Así, tenemos que $\displaystyle f^{2}\left(A\right) = A $, por lo que $\displaystyle f $ es involutiva. 
\end{description}
\end{proof}
\begin{fdefinition}[]
\normalfont $\displaystyle \mathcal{L} $ es la \textbf{base} de $\displaystyle f $ y $\displaystyle L_{2} $ su \textbf{dirección}.
\end{fdefinition}
\section{Matriz de una aplicación afín}
Sean $\displaystyle \left(O, \left\{ \vec{u}_{1}, \ldots, \vec{u}_{n}\right\} \right) $ y $\displaystyle \left(O', \left\{ \vec{v}_{1}, \ldots, v_{m}\right\} \right) $ un sistema de referencia cartesiana de $\displaystyle \mathcal{A} $ y $\displaystyle \mathcal{A}' $, respectivamente. Sea $\displaystyle f : \mathcal{A} \to \mathcal{A}' $ una aplicación afín y $\displaystyle \vec{f} $ su aplicación lineal asociada. Sabiendo que $\displaystyle f\left(O\right) \in \mathcal{A}' $,
\[
\begin{split}
	\overrightarrow{O'f\left(O\right)} =&  a^{1}\vec{v}_{1} + \cdots + a^{m}\vec{v}_{m} \\
	\vec{f}\left(\vec{u}_{1}\right) =&  a^{1}_{1}\vec{v}_{1} + \cdots + a^{m}_{1}\vec{v}_{m} \\
& \vdots \\
	\vec{f}\left(\vec{u}_{n}\right) =& a^{1}_{n}\vec{v}_{1} + \cdots + a^{m}_{n}\vec{v}_{m}.
\end{split}
\]
Si $\displaystyle X \in \mathcal{A} $ tenemos que $\displaystyle f\left(X\right) \in \mathcal{A}' $, además si $\displaystyle \overrightarrow{OX} = x^{1}\vec{u}_{1} + \cdots + x^{n}\vec{u}_{n} $ tendremos que
\[\overrightarrow{O'f\left(X\right)} = x'^{1}\vec{v}_{1} + \cdots + x'^{m}\vec{v}_{m}.\]
Así, $\displaystyle X = O + \overrightarrow{OX} $ y $\displaystyle f\left(X\right) = f\left(O\right) + \vec{f}\left(\overrightarrow{OX}\right) $, por lo que $\displaystyle \overrightarrow{O'f\left(X\right)} = \overrightarrow{O'f\left(O\right)}+\vec{f}\left(\overrightarrow{OX}\right) $. Así, tenemos que 
\[
\begin{split}
& x'^{1}\vec{v}_{1} + \cdots + x'^{m}\vec{v}_{m} \\
= & a^{1}\vec{v}_{1}+\cdots + a^{m}\vec{v}_{m} + \vec{f}\left(x^{1}\vec{u}_{1} + \cdots + x^{n}\vec{u}_{n}\right) \\
= & a^{1}\vec{v}_{1} + \cdots + a^{m}\vec{v}_{m} + x^{1}\left(a^{1}_{1}\vec{v}_{1} + \cdots + a^{m}_{1}\vec{v}_{m}\right)+ \cdots + x^{n}\left(a^{1}_{n}\vec{v}_{1} + \cdots + a^{m}_{n}\vec{v}_{m}\right) \\
= & \left(a^{1}+x^{1}a^{1}_{1} + \cdots + x^{n}a^{1}_{n}\right)\vec{v}_{1} + \cdots + \left(a^{m} + x^{1}a^{m}_{1} + \cdots + x^{n}a^{m}_{n}\right)\vec{v}_{m}.
\end{split}
\]
Así, obtenemos el siguiente sistema de ecuaciones:
\[
\begin{cases}
x'^{1} = a^{1}+x^{1}a^{1}_{1} + \cdots + x^{n}a^{1}_{n} \\
\vdots \\
x'^{m} = a^{m}+x^{1}a^{m}_{1} + \cdots + x^{n}a^{m}_{n}
\end{cases}
.\]
Matricialmente, obtenemos la expresión
\[\begin{pmatrix} x'^{1} \\ \vdots \\ x'^{m} \end{pmatrix} = \begin{pmatrix} a^{1} \\ \vdots \\ a^{m} \end{pmatrix} + \mathcal{M}_{ \left\{ \vec{u}_{i}\right\} \left\{ \vec{v}_{j}\right\} }\left(f\right) \begin{pmatrix} x^{1} \\ \vdots \\ x^{n} \end{pmatrix} .\]
También lo podemos expresar de la forma
\[ \begin{pmatrix} 1 \\ x'^{1} \\ \vdots \\ x'^{m} \end{pmatrix} = \begin{pmatrix} 1 & 0 & \cdots & 0 \\
a^{1} & a^{1}_{1} & \cdots & a^{1}_{n} \\
\vdots & \vdots & \vdots & \vdots \\
a^{m} & a^{m}_{1} & \cdots & a^{m}_{n}\end{pmatrix} .\]

