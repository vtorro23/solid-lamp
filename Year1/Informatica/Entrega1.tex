\documentclass{article}

% packages

\usepackage{graphicx} % Required for images
\usepackage[spanish]{babel}
\usepackage{mdframed}
\usepackage{amsthm}
\usepackage{amssymb}
\usepackage{fancyhdr}
\usepackage{amsmath}
\usepackage{geometry}[margin=1in]
\usepackage{pgfplots}
\usepackage{listings}
\usepackage{xcolor}

% for math environments

\theoremstyle{definition}
\newtheorem{theorem}{Teorema}
\newtheorem{definition}{Definición}
\newtheorem{ej}{Ejercicio}[section]
\newtheorem{sol}{Solución}[section]

% for headers and footers

\pagestyle{fancy}

\fancyhead[R]{Victoria Eugenia Torroja}
% Store the title in a custom command
\newcommand{\mytitle}{}

% Redefine \title to store the title in \mytitle
\let\oldtitle\title
\renewcommand{\title}[1]{\oldtitle{#1}\renewcommand{\mytitle}{#1}}

% Set the center header to the title
\lhead{\mytitle}

% Custom commands

\newcommand{\R}{\mathbb{R}}
\newcommand{\C}{\mathbb{C}}
\newcommand{\F}{\mathbb{F}}
\newcommand{\N}{\mathbb{N}}
\newcommand{\Q}{\mathbb{Q}}
\newcommand{\Z}{\mathbb{Z}}
\newcommand{\K}{\mathbb{K}}

\definecolor{codegreen}{rgb}{0,0.6,0}
\definecolor{codegray}{rgb}{0.5,0.5,0.5}
\definecolor{codepurple}{rgb}{0.58,0,0.82}
\definecolor{backcolour}{rgb}{0.95,0.95,0.92}

\lstdefinestyle{mystyle}{
    backgroundcolor=\color{backcolour},   
    commentstyle=\color{codegreen},
    keywordstyle=\color{magenta},
    numberstyle=\tiny\color{codegray},
    stringstyle=\color{codepurple},
    basicstyle=\ttfamily\footnotesize,
    breakatwhitespace=false,         
    breaklines=true,                 
    captionpos=b,                    
    keepspaces=true,                 
    numbers=left,                    
    numbersep=5pt,                  
    showspaces=false,                
    showstringspaces=false,
    showtabs=false,                  
    tabsize=2
}

\lstset{style=mystyle}


\begin{document}

\title{Informática - Entrega 1}
\author{Victoria Eugenia Torroja Rubio}
\date{19/10/2024}

\maketitle

\begin{ej}
Debido a la escasez de agua se pretende implantar un sistema de tarifas que penalice el consumo excesivo de este recurso, de acuerdo con la siguiente tabla:
\begin{center}
\begin{tabular}{c c}
	Consumo (m3) & euros/m3 \\
	Primeros 100 & 0.15 \\
	De 100 a 500 & 0.20 \\
	De 500 a 1000 & 0.35 \\
	A partir de 1000 & 0.80
\end{tabular}
\end{center}
 Implementar una función que tenga como parámetro el consumo de agua en m3 y calcule la factura de acuerdo con la tabla anterior.
\end{ej}

\begin{sol} La solución del ejercicio es:
\begin{lstlisting}[language = Python]
consumo = float(input("Consumo de agua en m3: "))
    
#incorporamos el caso "consumo < 0" para que el "else" final se corresponda con el caso "consumo >= 1000"

if consumo < 0:
  factura = "Error, no se pueden registrar valores negativos del consumo"
  
elif consumo <= 100:
    factura = 0.15 * consumo
    
elif consumo > 100 and consumo <= 500:
    factura = 0.20 * consumo
    
elif consumo > 500 and consumo < 1000:
    factura = 0.35 * consumo
    
else:
    factura = 0.8 * consumo
    
print("Su factura en euros es de:", factura)
\end{lstlisting}
\end{sol}

Esta es mejor solución (malinterpreté mal el enunciado).
\begin{lstlisting}[language = Python]
consumo = float(input("Consumo de agua: "))

if consumo < 0:
	print("Error. El consumo no puede ser negativo")

elif consumo <= 100:
	factura = consumo * 0.15
	print(factura)

elif consumo <= 500:
	factura = 15 + (consumo - 100) * 0.20
	print(factura)

elif consumo < 1000:
	factura = 15 + 80 + (consumo - 500) * 0.35	
	print(factura)

else:
	factura = 15 + 80 + 175 + (consumo - 1000) * 0.8
\end{lstlisting}
\end{document}
