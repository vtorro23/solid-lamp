\documentclass{article}

% packages

\usepackage{graphicx} % Required for images
\usepackage[spanish]{babel}
\usepackage{mdframed}
\usepackage{amsthm}
\usepackage{amssymb}
\usepackage{fancyhdr}
\usepackage{amsmath}
\usepackage{geometry}[margin=1in]
\usepackage{pgfplots}
\usepackage{listings}
\usepackage{xcolor}

% for math environments

\theoremstyle{definition}
\newtheorem{theorem}{Teorema}
\newtheorem{definition}{Definición}
\newtheorem{ej}{Ejercicio}[section]
\newtheorem{sol}{Solución}[section]

% for headers and footers

\pagestyle{fancy}

\fancyhead[R]{Victoria Eugenia Torroja}
% Store the title in a custom command
\newcommand{\mytitle}{}

% Redefine \title to store the title in \mytitle
\let\oldtitle\title
\renewcommand{\title}[1]{\oldtitle{#1}\renewcommand{\mytitle}{#1}}

% Set the center header to the title
\lhead{\mytitle}

% Custom commands

\newcommand{\R}{\mathbb{R}}
\newcommand{\C}{\mathbb{C}}
\newcommand{\F}{\mathbb{F}}
\newcommand{\N}{\mathbb{N}}
\newcommand{\Q}{\mathbb{Q}}
\newcommand{\Z}{\mathbb{Z}}
\newcommand{\K}{\mathbb{K}}

\definecolor{codegreen}{rgb}{0,0.6,0}
\definecolor{codegray}{rgb}{0.5,0.5,0.5}
\definecolor{codepurple}{rgb}{0.58,0,0.82}
\definecolor{backcolour}{rgb}{0.95,0.95,0.92}

\lstdefinestyle{mystyle}{
    backgroundcolor=\color{backcolour},   
    commentstyle=\color{codegreen},
    keywordstyle=\color{magenta},
    numberstyle=\tiny\color{codegray},
    stringstyle=\color{codepurple},
    basicstyle=\ttfamily\footnotesize,
    breakatwhitespace=false,         
    breaklines=true,                 
    captionpos=b,                    
    keepspaces=true,                 
    numbers=left,                    
    numbersep=5pt,                  
    showspaces=false,                
    showstringspaces=false,
    showtabs=false,                  
    tabsize=2
}

\lstset{style=mystyle}


\begin{document}

\title{Informática}
\author{Victoria Eugenia Torroja Rubio}
\date{9/10/2024 - }

\maketitle

\section{Tema 1}

\subsection{Hoja 1}

\begin{ej}
Escribe un programa en Python que nos diga cuál es el volumen de un cono con un radio de la base de 14,5 y una altura de 26,79. La fórmula que debes usar es:
\[\frac{\pi \times radio^2 \times altura}{3} .\]
Recuerda que el valor (aproximado) de $\displaystyle \pi $  es 3,141592.
\end{ej}

\begin{sol}
La solución es:
\begin{lstlisting}[language = Python]
pi = 3.14159

volume = (pi * 14.5 * 26.79) / 3

print("El volumen (u3) es: ", volume)
\end{lstlisting}
\end{sol}

\begin{ej}
Modifica el programa anterior para que use tres variables: radio, altura y volumen. Las dos primeras se inicializarán a 14,5 y 26,79 respectivamente. La tercera obtendrá el resultado de la fórmula.
\end{ej}

\begin{sol}
La solución es:
\begin{lstlisting}[language = Python]
pi = 3.14159

radius = 14.5

height = 26.79

volume = (pi * radius * height) / 3

print("El volumen(u3) es: ", volume)
\end{lstlisting}
\end{sol}

\begin{ej}
Escribe un programa en Python que lea del teclado un número (float) de grados Fahrenheit y lo convierta a Celsius mostrando el resultado en la pantalla.
\[C = \frac{5}{9} \cdot \left(F - 32\right) .\]
\end{ej}

\begin{sol}
La solución es:
\begin{lstlisting}[language = Python]
fahrenheit = float(input("Introduzca una temperatura en fahrenheit: "))

celsius = 5 / 9 * (fahrenheit - 32)

print("La temperatura en celsius es: ", celsius)
\end{lstlisting}
\end{sol}

\begin{ej}
Escribe un programa que lea del teclado un tiempo transcurrido en segundos y muestre en la pantalla las horas, los minutos y los segundos equivalentes.
\end{ej}

\begin{sol}
La solución es:
\begin{lstlisting}[language = Python]
tiempo = int(input("Tiempo en segundos: "))

hora = int(tiempo / 3600)
minuto = int((float(tiempo/3600) - hora) * 60)
segundo = int(((float(tiempo/3600) - hora) * 60 - minuto) * 60)

print("Hora: ", hora, "Minutos: ", minuto, "Segundos: ", segundo)
\end{lstlisting}
\end{sol}

\begin{ej}
El área de un triángulo se puede calcular mediante la ley del seno: si se conocen dos lados del triángulo, lado1 y lado2, y el ángulo a existente entre ellos. Dicha ley establece que
\[A = \frac{1}{2} \times \text{lado1} \times \text{lado2} \times \sin\alpha .\]
Implementa un programa que calcule el área de un triángulo de esta manera. El programa deberá solicitar al usuario los dos lados y el ángulo que estos forman (en grados). Ten en cuenta que la función sin() espera que el ángulo se proporcione en radianes. Ángulo en radianes = Ángulo en grados x $\displaystyle \pi $  / 180.
\end{ej}

\begin{sol}
La solución es la siguiente.
\begin{lstlisting}[language = Python]
import math

pi = 3.14159

lado_1 = float(input("Inserte la longitud del primer lado: "))
lado_2 = float(input("Inserte la longitud del segundo lado: "))
alpha = float(input("Inserte el valor del angulo que forman el primer y el segundo lado en grados: "))

#cambiamos los grados a radianes
alpha = alpha * pi / 180

area = 1 / 2 * lado_1 * lado_2 * math.sin(alpha)

print("El area es:", area)
\end{lstlisting}
\end{sol}

\begin{ej}
Escribe un programa en Python que pida al usuario el valor de dos variables reales $\displaystyle x $ e $\displaystyle y $, y a continuación muestre el resultado de aplicarles la siguiente fórmula:
\[f\left(x,y\right) = \sqrt{1,531^{\left(x+y\right)} + \frac{ \left|e^{x}-e^{y}\right|\times\left(\sin\left(x\right)-\tan\left(y\right)\right)}{\log_{10}\left(y\right)\times3,141592^{x}}} .\]
Declara constantes para los valores fijos.
\end{ej}

\begin{sol}
La solución es la siguiente.
\begin{lstlisting}[language = Python]
import math

x = float(input("Valor de x: "))
y = float(input("Valor de y: "))

z = 1.531
pi = 3.141592

w = math.sqrt(z ** (x + y) + (abs(math.exp(x) - math.exp(y)) * (math.sin(x) - math.tan(y))) / (math.log10(y) * (pi ** x)))

print("f(x,y) = ", w)
\end{lstlisting}
\end{sol}

\begin{ej}
Escribe un programa en Python que pida al usuario los datos de un préstamo hipotecario (capital prestado, interés anual y años que dura el préstamo) y le muestre la cuota mensual que habrá de pagar y el total de lo pagado una vez terminado el plazo, distinguiendo la cantidad de amortización y la de intereses. \\ \\
La fórmula que nos da la cuota mensual es:
\[\text{cuota} = \frac{\text{capital} \times \text{ratio}}{100 \times \left(1 - \left(1 + \frac{\text{ratio}}{100}\right)^{-\text{plazo}}\right)} .\]
Donde el ratio es el interés mensual y el plazo está indicado en meses. La cantidad de amortización es el capital prestado; el resto son intereses.
\end{ej}

\begin{sol}
La solución es la siguiente.
\begin{lstlisting}[language = Python]
cap_prestado = float(input("Capital prestado en euros: "))
ratio = float(input("Interes anual: "))
plazo = float(input("Tiempo que dura el prestamo en meses: "))

cuota = (cap_prestado * ratio) / (100 * (1-(1 + ratio / 100)**(- plazo)))

print("La cuota es:", cuota)
\end{lstlisting}
\end{sol}

\begin{ej}
Trabajando con triángulos. Dadas tres cantidades reales positivas, escribe funciones para dilucidar las siguientes situaciones:
\begin{description}
\item[(a)] ¿Es un triángulo? Si los valores de dichas cantidades pueden corresponder a las longitudes de los lados de un triángulo. Para ello, tenga en cuenta, el teorema de desigualdad triangular de la geometría euclidiana.
\item[(b)] ¿Es escaleno? En el caso de que las medidas puedan corresponder a las longitudes de los lados de un triángulo, si dicho triángulo es escaleno.
\item[(c)] ¿Es equilátero? En el caso de que las medidas puedan corresponder a las longitudes de los lados de un triángulo, si dicho triángulo es equilátero.
\item[(d)] ¿Es isósceles? En el caso de que las medidas puedan corresponder a las longitudes de los lados de un triángulo, si dicho triángulo es isósceles.
\item[(e)] ¿Es rectángulo? En el caso de que las medidas puedan corresponder a las longitudes de los lados de un triángulo, si dicho triángulo es rectángulo.
\end{description}
\end{ej}

\begin{sol}
La solución es la siguiente.
\begin{lstlisting}[language = Python]
lado_1 = float(input("Lado 1 (numero real positivo): "))
lado_2 = float(input("Lado 2 (numero real positivo): "))
lado_3 = float(input("Lado 3 (numero real positivo): "))

# apartado (a)

if lado_1 + lado_2 >= lado_3 and lado_1 + lado_3 >= lado_2 and lado_2 + lado_3 >= lado_1:
  triangle = True
  print("Es un triangulo")
else:
  triangle = False
  print("No es un triangulo")

# apartado (b)

if triangle == True:
  if lado_1 == lado_2 or lado_2 == lado_3 or lado_1 == lado_3:
    print("No es escaleno")
  else:
    print("Es escaleno")
else:
  print("No se trata de un triangulo")

# apartado (c)

if triangle == False:
  print("No se trata de un triangulo")
elif lado_1 == lado_2 and lado_2 == lado_3:
  print("Es un triangulo equilatero")
else: 
  print("No es un triangulo equilatero")

# apartado (d)

if triangle == False:
  print("No se trata de un triangulo")
elif lado_1 == lado_2 or lado_2 == lado_3 or lado_1 == lado_3:
  print("Es un triangulo isosceles")
else: 
  print("No es un triangulo isosceles")

# apartado (e)
a = lado_2 ** 2 + lado_3 ** 2
b = lado_1 ** 2 + lado_3 ** 2
c = lado_1 ** 2 + lado_2 ** 2

if triangle == False:
  print("No se trata de un triangulo")
elif lado_1 ** 2 == a or lado_2 ** 2 == b or lado_3 ** 2 == c:
  print("Es un triangulo rectangulo")
else: 
  print("No es un triangulo rectangulo")
\end{lstlisting}
\end{sol}

\subsection{Hoja 2}

\begin{ej}
Escribe un programa en Python que pida al usuario tres valores enteros y los muestrede menor a mayor separados por comas. Por ejemplo, si el usuario introduce 10, 4 y 6, el resultado será: 4,6,10.
\end{ej}

\begin{sol} La solución es:
\begin{lstlisting}[language = Python]
val_1 = int(input("Inserte un numero entero: "))
val_2 = int(input("Inserte un numero entero: "))
val_3 = int(input("Inserte un numero entero: "))

if val_1 <= val_2 and val_2 <= val_3:
	print(val_1, ",", val_2, ",", val_3)

elif val_2 <= val_1 and val_1 <= val_3:
	print(val_2, ",", val_1, ",", val_3)

elif val_3 <= val_1 and val_1 <= val_2:
	print(val_3, ",", val_1, ",", val_2)

elif val_3 <= val_2 and val_2 <= val_1:
	print(val_3, ",", val_2, ",", val_1)

elif val_1 <= val_3 and val_3 <= val_2:
	print(val_1, ",", val_3, ",", val_2)

else:
	print(val_2, ",", val_3, ",", val_1)
\end{lstlisting}
También se puede hacer de esta manera:
\begin{lstlisting}[language = Python]
# Pedir tres valores enteros al usuario
a = int(input("Introduce el primer valor: "))
b = int(input("Introduce el segundo valor: "))
c = int(input("Introduce el tercer valor: "))

# Comparaciones para encontrar el orden de los tres numeros
if a <= b and a <= c:
    if b <= c:
        print(f"{a},{b},{c}")
    else:
        print(f"{a},{c},{b}")

elif b <= a and b <= c:
    if a <= c:
        print(f"{b},{a},{c}")
    else:
        print(f"{b},{c},{a}")

else:  # c es el menor
    if a <= b:
        print(f"{c},{a},{b}")
    else:
        print(f"{c},{b},{a}")
\end{lstlisting}
\footnote{Esta segunda manera está hecha por el ChatGPT, cuando pone print(), hay que ponerlo como en la primera manera.} 
\end{sol}

\begin{ej}
Escribe una función que permita calcular las soluciones a una ecuación de segundo $\displaystyle Ax^{2} + Bx + C = 0 $.
\end{ej}

\begin{sol}
La solución es la siguiente.
\begin{lstlisting}[language = Python]
import math

A = float(input("Inserte el primer valor: "))
B = float(input("Inserte el segundo valor: "))
C = float(input("Inserte el tercer valor: "))

# definimos el discriminante para averiguar si la ecuacion va a tener solucion o no
D = B ** 2 - 4 * A * C

if A == 0:
	print("El primer valor no puede ser nulo.")

elif D == 0:
	sol = - B / (2 * A)
	print("Tenemos que x es igual a: ", sol)

elif D > 0:
	sol_1 = (- B + math.sqrt(B ** 2 - 4 * A * C)) / (2 * A)
	sol_2 = (- B - math.sqrt(B ** 2 - 4 * A * C)) / (2 * A)
	print("Tenemos que las soluciones de la ecuacion son ", sol_1, sol_2)

else: # D < 0
	print("Esta ecuacion no tiene soluciones reales.")
\end{lstlisting}
\end{sol}

\begin{ej}
Debido a la escasez de agua se pretende implantar un sistema de tarifas que penalice el consumo excesivo de este recurso, de acuerdo con la siguiente tabla:
\begin{center}
\begin{tabular}{c c}
	Consumo (m3) & euros/m3 \\
	Primeros 100 & 0.15 \\
	De 100 a 500 & 0.20 \\
	De 500 a 1000 & 0.35 \\
	A partir de 1000 & 0.80
\end{tabular}
\end{center}
 Implementar una función que tenga como parámetro el consumo de agua en m3 y calcule la factura de acuerdo con la tabla anterior.
\end{ej}

\begin{sol}
Entrega 1.
\end{sol}

\begin{ej}
Dados dos números enteros n y m, escribe una función en Python que calcule el signo de su producto (+ si el producto es positivo, - si el producto es negativo y 0 si el producto es cero) sin llegar a calcular dicho producto.
\end{ej}

\begin{sol}
La solución es la siguiente.
\begin{lstlisting}[language = Python]
a = int(input("Insert whole number here: "))
b = int(input("Insert another whole number here: "))

if a == 0 or b == 0: 
	valor = "0"

elif (a > 0 and b > 0) or (a < 0 and b < 0):
	valor = "positivo"

else:
	valor = "negativo"

print("El producto es", valor)
\end{lstlisting}
\end{sol}

\begin{ej}
Escribe una función que, dada una temperatura, indique la actividad más apropiada para dicha temperatura teniendo en cuenta los siguientes criterios.
\begin{center}
\begin{tabular}{c c}
	Actividad & Temperatura idónea \\
	Natación & temp $\displaystyle > $  30 \\
	Tenis & 20 $\displaystyle < $  temp $\displaystyle \leq 30 $  \\
	Golf & 10 $\displaystyle < $  temp $\displaystyle \leq 20 $  \\
	Esquí & 5 $\displaystyle < $  temp $\displaystyle \leq 10 $ \\
	Parchís & temp $\displaystyle \leq 5 $ 
\end{tabular}
\end{center}
\end{ej}

\begin{sol}
La solución es la siguiente.
\begin{lstlisting}[language = Python]
temp = float(input("Temperature in degrees celcius: "))

if temp < 0: 
	activity = "Error. Can't have negative temperature."
elif temp <= 5: 
	activity = "Parchis"
elif temp <= 10:
	activity = "Esqui"
elif temp <= 20: 
	activity = "Golf"
elif temp <= 30:
	activity = "Tenis"
else:
	activity = "Natacion"
print("Your recommended activity is:", activity)
\end{lstlisting}
\end{sol}

\end{document}
