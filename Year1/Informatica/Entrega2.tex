\documentclass{article}

% packages

\usepackage{graphicx} % Required for images
\usepackage[spanish]{babel}
\usepackage{mdframed}
\usepackage{amsthm}
\usepackage{amssymb}
\usepackage{fancyhdr}
\usepackage{amsmath}
\usepackage{geometry}[margin=1in]
\usepackage{pgfplots}
\usepackage{listings}
\usepackage{xcolor}

% for math environments

\theoremstyle{definition}
\newtheorem{theorem}{Teorema}
\newtheorem{definition}{Definición}
\newtheorem{ej}{Ejercicio}[section]
\newtheorem{sol}{Solución}[section]

% for headers and footers

\pagestyle{fancy}

\fancyhead[R]{Victoria Eugenia Torroja}
% Store the title in a custom command
\newcommand{\mytitle}{}

% Redefine \title to store the title in \mytitle
\let\oldtitle\title
\renewcommand{\title}[1]{\oldtitle{#1}\renewcommand{\mytitle}{#1}}

% Set the center header to the title
\lhead{\mytitle}

% Custom commands

\newcommand{\R}{\mathbb{R}}
\newcommand{\C}{\mathbb{C}}
\newcommand{\F}{\mathbb{F}}
\newcommand{\N}{\mathbb{N}}
\newcommand{\Q}{\mathbb{Q}}
\newcommand{\Z}{\mathbb{Z}}
\newcommand{\K}{\mathbb{K}}

\definecolor{codegreen}{rgb}{0,0.6,0}
\definecolor{codegray}{rgb}{0.5,0.5,0.5}
\definecolor{codepurple}{rgb}{0.58,0,0.82}
\definecolor{backcolour}{rgb}{0.95,0.95,0.92}

\lstdefinestyle{mystyle}{
    backgroundcolor=\color{backcolour},   
    commentstyle=\color{codegreen},
    keywordstyle=\color{magenta},
    numberstyle=\tiny\color{codegray},
    stringstyle=\color{codepurple},
    basicstyle=\ttfamily\footnotesize,
    breakatwhitespace=false,         
    breaklines=true,                 
    captionpos=b,                    
    keepspaces=true,                 
    numbers=left,                    
    numbersep=5pt,                  
    showspaces=false,                
    showstringspaces=false,
    showtabs=false,                  
    tabsize=2
}

\lstset{style=mystyle}


\begin{document}

\title{Informática - Deberes 2}
\author{Victoria Eugenia Torroja Rubio}
\date{28/10/2024}

\maketitle

\begin{ej}
Si depositamos una cantidad C en un banco a un interés del I \% anual, al cabo de un mes el banco nos debe una cantidad igual a C · i/100. Ten cuidado porque, por defecto, el banco habla del interés anual y a la hora de hacer el cálculo debemos usar el interés mensual i = I/12.
Escribe una función que dados un capital C, un interés I y un número de meses n, calcule el dinero que tendremos al cabo de los n meses. Ten en cuenta que, cada mes, la cantidad abonada en concepto de intereses queda depositada en la cuenta.
\end{ej}

\begin{sol}
La solución será la siguiente. 
\begin{lstlisting}[language = Python]
def ej1(C : float, I : float, n : int) -> float:
    i = I / 12
    r = 1 + ( i / 100 )
    resultado = C * (r ** n)
    return resultado

control1 = ej1(1, 10, 10)
control2 = ej1(1, 10, 20)
control3 = ej1(1, 10, 30)

print(control1, control2, control3)
\end{lstlisting}
\end{sol}

\begin{ej}
Supón que eres muy ahorrador y aportas cada mes una cantidad constante $\displaystyle c $ a tu cuenta. Al final del primer mes tendrás $\displaystyle c + c \cdot i / 100 $, la cantidad $\displaystyle c $ que aportas al principio más los intereses de ese mes. No deberías tener dificultad en saber que al final del segundo mes tendrías $\displaystyle \left(c \cdot \left(1 + i / 100\right)+c\right) \cdot \left(1 + i / 100\right) $. Escribe una función que calcule el dinero que tendrás al cabo de $\displaystyle n $ meses.
\end{ej}


\end{document}
